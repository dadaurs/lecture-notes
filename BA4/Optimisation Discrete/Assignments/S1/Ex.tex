\documentclass[11pt, a4paper]{article}
\usepackage[utf8]{inputenc}
\usepackage[T1]{fontenc}
\usepackage[francais]{babel}
\usepackage{lmodern}
\usepackage{amsmath}
\usepackage{amssymb}
\usepackage{amsthm}
\renewcommand{\vec}[1]{\overrightarrow{#1}}
\newcommand{\del}{\partial}
\DeclareMathOperator*{\sgn}{sgn}
\DeclareMathOperator*{\id}{Id}
\DeclareMathOperator*{\im}{Im}
\DeclareMathOperator*{\re}{Re}
\DeclareMathOperator*{\vol}{Vol}
\newcommand\norm[1]{\left\vert#1\right\vert}
\newcommand\ns[1]{\left\vert\left\vert\left\vert#1\right\vert\right\vert\right\vert}
\newcommand\Norm[1]{\left\lVert#1\right\rVert}
\newcommand\N[1]{\left\lVert#1\right\rVert}
\newcommand\abs[1]{\left\vert#1\right\vert}
\newcommand\inj{\hookrightarrow}
\newcommand\surj{\twoheadrightarrow}
\newcommand\ded[1]{\overset{\circ}{#1}}
\newcommand\sidenote[1]{\footnote{#1}}
\newcommand\eng[1]{\left\langle#1\right\rangle}
\newcommand\hr{
    \noindent\rule[0.5ex]{\linewidth}{0.5pt}
}

\newcommand{\incfig}[1]{%
    \def\svgwidth{\columnwidth}
    \import{./figures}{#1.pdf_tex}
}
\newcommand{\filler}[1][10]%
{   \foreach \x in {1,...,#1}
    {   test 
    }
}

\newcommand\contra{\scalebox{1.5}{$\lightning$}}
\makeatother
\def\@lecture{}%
\newcommand{\lecture}[3]{
    \ifthenelse{\isempty{#3}}{%
        \def\@lecture{Lecture #1}%
    }{%
        \def\@lecture{Lecture #1: #3}%
    }%
    \subsection*{\@lecture}
    \marginpar{\small\textsf{\mbox{#2}}}
}

\begin{document}
\title{Série 1}
\author{David Wiedemann}
\maketitle
\section{Problem Set 2 Exercise 3}
We first note that we may rewrite
\begin{align*}
	P &= \quad &\min \text{  }  &2x+3|y-10|\\
	  &  \text{ s.t. } & & |x-2|+y \leq 5
	  \intertext{as}
	P' &= \quad &\min \text{  }  &2x+3w\\
	  &  \text{ s.t. } && |x-2|+y \leq 5\\
	  & & & w \geq y-10\\
	  & & & w \geq  10-y\\
	  \intertext{Applying the same trick for $|x-2|$ yields}
	P'' &= &\min \text{  } & 2x+3w\\
	  &  \text{ s.t. }& & z+y \leq 5\\
	  & & & w \geq y-10\\
	  & & & w \geq  10-y\\
	  & & & z \geq x-2\\
	  & & & z \geq 2-x\\
	  \intertext{Note that positivity conditions on $w$ and $z$ are not needed as at least one $y-10$ or $10-y$ ( respectively $x-2$ or $2-x$ ) will be positive. }
\end{align*}
We now justify why $P$ and $P'$ have the same objective value.\\
First of, suppose $P$ has an optimal solution $v=(  x_0,y_0) $, we then pretend $v_\ast =( x_0,y_0, |y_0-10|) $ ( where the last coordinate is the value of $w$ ) also is an optimal solution for $P'$ which takes the same objective value as $P$.\\
Indeed, the fact that it takes the same objective value as $P$ is clear.\\
Now suppose $P'$ has a smaller optimal solution than $P$, then, as the constraints on $x$ are left unchanged, this implies the value of $w$ is smaller than the value of $|y-10|$, but this is impossible since we force $w \geq y-10$ and $w \geq 10-y$.\\
The very same argument applied to go from $P'$ to $P''$ works to show $P''$ has the same optimal objective value as $P'$.\\
Hence $P''$ is a linear program with the same optimal objective value as $P$ 
\section{Problem Set 3 Exercise 2}
Let $v$ be a vertex of a polyhedron $P$, by definition this implies there exists a vector $c$ such that $c\cdot v > c\cdot y \forall y \in P \setminus \left\{ v \right\} $.\\
Now suppose $v$ is not an extreme point, this means there exists $x,y\in P\setminus \left\{ v \right\} $ such that $ v = \lambda x + ( 1-\lambda) y$.\\
Now first note that $c\cdot x \neq c\cdot y$, indeed, if $c\cdot x = c\cdot y$, then
\[ 
c\cdot v = c\cdot ( \lambda x+ ( 1-\lambda) y) = \lambda c\cdot x + ( 1-\lambda) c\cdot x = c\cdot x
\]
Which is impossible since $v$ is the unique optimal solution.\\
Hence, we may suppose without loss of generality that $c\cdot x >c\cdot y$, but then
\[ 
c\cdot x= \lambda c \cdot x + ( 1-\lambda) c\cdot x > \lambda c\cdot x + ( 1-\lambda) c\cdot y = c\cdot v
\]
Which contradicts $v$ being a vertex of $P$.\\
\section{Problem Set 3 Exercise 3}
We set $v_i$ to be the vectors corresponding to the different planes $P_1,\ldots, P_4$  and $k_i$ the corresponding constant terms.
To determine in which order the ray passes through the $P_i$, we have to solve the equations
\[ 
v_i\cdot ( x^{*}+ \lambda d) = k_i
\]
for $\lambda$.\\
We first solve the equation in general and then plug in the values, ie. we solve
\begin{align*}
\begin{pmatrix}
a\\b\\c\\
\end{pmatrix} \cdot \left( 
	\begin{pmatrix}
		0\\1\\1
	\end{pmatrix} 
	+ \lambda 
	\begin{pmatrix}
	1 \\1\\-1
	\end{pmatrix} 	
\right) &=k\\
a\lambda + b + b\lambda + c -c\lambda &= k\\
( a+b-c) \lambda &= k -b-c\\
\lambda &= \frac{k-b-c}{a+b-c}
\end{align*}
Now pluging in the different values for $a,b,c$ and $k$ , we get
\begin{align*}
\lambda_1 = `` -\frac{5}{0}'' \text{ is undefined , since} \\
( 1 2 3) x^{*}\neq 0\\
\intertext{We conclude the ray never passes through $P_1$ }
\lambda_2= \frac{1}{4}\\
\lambda_3 = 0\\
\lambda_4= \frac{5}{2}	
\end{align*}
From this we conclude that the ray
\begin{itemize}
\item Never passes through $P_1$ 
\item Then passes through 
	\begin{enumerate}
	\item $P_3$ 
	\item $P_2$ 
	\item $P_4$ 
	\end{enumerate}
\end{itemize}










\end{document}
