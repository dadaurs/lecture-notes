\documentclass[11pt, a4paper]{article}
\usepackage[utf8]{inputenc}
\usepackage[T1]{fontenc}
\usepackage[francais]{babel}
\usepackage{lmodern}
\usepackage{amsmath}
\usepackage{amssymb}
\usepackage{faktor}
\usepackage{tikz}
\usepackage{tikz-cd}
\usepackage{amsthm}
\renewcommand{\vec}[1]{\overrightarrow{#1}}
\newcommand{\del}{\partial}
\DeclareMathOperator*{\sgn}{sgn}
\DeclareMathOperator*{\id}{Id}
\DeclareMathOperator*{\im}{Im}
\DeclareMathOperator*{\re}{Re}
\DeclareMathOperator*{\vol}{Vol}
\newcommand\norm[1]{\left\vert#1\right\vert}
\newcommand\ns[1]{\left\vert\left\vert\left\vert#1\right\vert\right\vert\right\vert}
\newcommand\Norm[1]{\left\lVert#1\right\rVert}
\newcommand\N[1]{\left\lVert#1\right\rVert}
\newcommand\abs[1]{\left\vert#1\right\vert}
\newcommand\inj{\hookrightarrow}
\newcommand\surj{\twoheadrightarrow}
\newcommand\ded[1]{\overset{\circ}{#1}}
\newcommand\sidenote[1]{\footnote{#1}}
\newcommand\eng[1]{\left\langle#1\right\rangle}
\newcommand\hr{
    \noindent\rule[0.5ex]{\linewidth}{0.5pt}
}

\newcommand{\incfig}[1]{%
    \def\svgwidth{\columnwidth}
    \import{./figures}{#1.pdf_tex}
}
\newcommand{\filler}[1][10]%
{   \foreach \x in {1,...,#1}
    {   test 
    }
}

\newcommand\contra{\scalebox{1.5}{$\lightning$}}
\makeatother
\def\@lecture{}%
\newcommand{\lecture}[3]{
    \ifthenelse{\isempty{#3}}{%
        \def\@lecture{Lecture #1}%
    }{%
        \def\@lecture{Lecture #1: #3}%
    }%
    \subsection*{\@lecture}
    \marginpar{\small\textsf{\mbox{#2}}}
}

\begin{document}
\title{Série 9}
\author{David Wiedemann, Matteo Mohammedi, Nino Courtecuisse}
\maketitle
\section*{a)}
Etant donné un groupe $G$, on construit une présentation de la manière suivante.
\begin{itemize}
\item On choisit un ensemble de générateurs $A$  pour $G$, ceci est toujours possible, quitte à prendre l'ensemble $G$ tout entier.
\item On construit le groupe libre à $A$ générateurs $F( A) $.
\item On définit l'application $\phi: F( A) \to G$ en envoyant chaque générateur sur l'élément associé.
\item En tant que relateurs, on choisit $I= \ker \phi$.
\end{itemize}
On prétend alors que $ \langle A| I\rangle$ est une présentation de $G$.\\
En effet, $I$ étant déjà un groupe normal, la présentation $ \langle A, I\rangle$ est isomorphe (par définition) à $ \faktor { F( A) } { I} $ et le premier théorème d'isomorphisme implique que $ \faktor { F( A) } { I} \simeq G$.\\

On montre que $ \faktor { \mathbb{Z}} { 5 \mathbb{Z}} = \langle a | a^{5}\rangle $.\\
En effet, le groupe $ G =\langle a | a^{5}\rangle$ a précisément $5$ éléments, puisque $5$ est un nombre premier, on en déduit l'isomorphisme ci-dessus.\\
De plus $ D_{10} = \langle a,b | a^{2},b^{5}, abab\rangle$.\\
En effet, utilisons la définition de groupe dihédral (démontrée en structures algébriques je crois) $ D_{2n} = C_n\rtimes C_2$ ou l'action de $C_2$ sur $C_n$ est donnée par l'inversion d'éléments.\\
Alors posons $G= \langle a,b | a^{2}, b^{5}, abab\rangle$ et définissons l'application $\phi: F( a,b)  \to D_{10} $ qui envoie $a$ sur $( 0,1) $ et $b$ sur $( 1,0) $.\\
Alors $\phi$ passe au quotient de $G$ et induit une application $ \overline{\phi}: G\to D_{10}$, cette application est surjective puisque $\phi$ l'est.\\
De plus, on vérifie facilement que $G$ possède 10 éléments (chaque mot a un représentant de la forme $a^{k}b^{j}$ avec $ 0 \leq k \leq 2$ et $ 0 \leq j \leq 5$), ainsi $ \overline{\phi}$ est un morphisme surjective entre deux ensembles de même cardinalité, ie. un isomorphisme.
\section*{b)}
On montre le résultat pour un ensemble de générateurs arbitraires et de relateurs finis.\\
Soit $G$ un groupe, $A$ un ensemble de générateurs et $I$ un ensemble de relateurs finis tel que $G = \langle A | I \rangle$.\\
On procède par induction, sur le nombre de générateurs.\\
Si $|I|= 0$, le résultat est immédiat en prenant $X= \vee_A S^{1}$ qui aura comme groupe fondamental le groupe libre à $A$ générateurs, ie., $G$.\\
Supposons donc maintenant le résultat démontré pour $|I|=n$ et montrons le résultat pour $n+1$ relateurs.\\
Soient $r_1, \ldots, r_{n+1} $ nos $n+1$ relateurs, par hypothese de récurrence, on peut construire un espace dont le groupe fondamental est donné par la présentation $ \langle A| r_1, \ldots, r_n\rangle$, appelons cet espace $X$.\\
De par notre construction inductive, on sait que $X$ contiendra toujours un sous-espace homéomorphe à $\vee_A S^{1}$.\\
On construit maintenant le pushout suivant:
\[\begin{tikzcd}
	{ S^1} & {X} \\
	{ e^{2}} & X'
	\arrow["\tilde{ r }_{n+1} ", from=1-1, to=1-2]
	\arrow["\iota"', from=1-1, to=2-1]
	\arrow[from=1-2, to=2-2]
	\arrow[from=2-1, to=2-2]
\end{tikzcd}\]
Ou $\iota$ est l'inclusion triviale et $\tilde{ r }_{n+1} $ est une application ayant le même type d'homotopie que le mot donné par $r_{n+1} $ dans $\vee_A S^{1}\subset X$ .\\
Etant donné que $X'$ est obtenu comme attachement cellulaire, le corrolaire de Seifert-Van Kampen vu en cours s'applique et on obtient un pushout de groupes donné par
\[\begin{tikzcd}
	{ \mathbb{Z}} & { \langle A| r_1, \ldots, r_n\rangle} \\
	{ 1} & \pi_1( X' )
	\arrow["\tilde{ r }_{n+1} ", from=1-1, to=1-2]
	\arrow[from=1-1, to=2-1]
	\arrow[from=1-2, to=2-2]
	\arrow[from=2-1, to=2-2]
\end{tikzcd}\]
Et ainsi, $\pi_1( X') $ admet la présentation $\langle A | r_1, \ldots, r_{n+1} \rangle$ 
\section*{c)}
\subsubsection*{ Espace avec $ \faktor { \mathbb{Z}} { 5\mathbb{Z}} $ comme Groupe Fondamental}
Etant donné que $ \faktor { \mathbb{Z}} { 5\mathbb{Z}} $ admet la présentation $ \langle a| a^{5}\rangle$, l'espace obtenu par le pushout suivant
\[\begin{tikzcd}
	{ S^{1}	} & { S^{1}} \\
	{ e^{2}} & { S_1 \cup_5 e^{2}} \\
	\arrow[" r_5 ", from=1-1, to=1-2]
	\arrow[ " \iota", from=1-1, to=2-1]
	\arrow[from=1-2, to=2-2]
	\arrow[from=2-1, to=2-2]
\end{tikzcd}\]
Où $\iota$ est l'inclusion et $r_5$ est une quelconque application de degre 5.\\
De par la partie b), on sait que cet espace aura $ \faktor { \mathbb{Z}} { 5\mathbb{Z}} $ comme groupe fondamental
\subsubsection*{ Espace avec $D_{10}$ comme Groupe Fondamental}
On utilise la présentation $ \langle \tau, \sigma | \tau^{5}, \sigma^{2} , \sigma\tau\sigma\tau\rangle$.\\
On notera $S^{1}_\sigma, S^{1}_\tau$ pour deux copies distinctes de $S^{1}$.\\
Fixons $T: S^{1}\to S^{1}_\sigma \vee S^{1}_\tau$ l'inclusion de $S^1$ dans la copie $S^{1}_{\tau} $ du wedge et $\Sigma: S^{1}\to S^{1}_\sigma\vee S^{1}_\tau$ l'inclusion de $S^{1}$ dans la copie $S^{1}_{\sigma} $ du wedge.\\
On construit alors les pushouts consécutifs suivants:
\[\begin{tikzcd}
	{ S^{1}	} & { S^{1}_\sigma \vee S^{1}_\tau} \\
	{ e^{2}} & { \left( S^{1}_\sigma \vee S^{1}_\tau \right) \cup_{\tau^{5}}  e^{2}} \\
	\arrow[" T^{5} ", from=1-1, to=1-2]
	\arrow[ " \iota", from=1-1, to=2-1]
	\arrow[from=1-2, to=2-2]
	\arrow[from=2-1, to=2-2]
\end{tikzcd}\]
Ou $T^{5}$ est le lacet $T$ concaténé $5$ fois avec lui même.
\[\begin{tikzcd}
	{ S^{1}	} & {  \left( S^{1}_\sigma \vee S^{1}_\tau \right) \cup_{\tau^{5}}  e^{2}} \\
	{ e^{2}} & {  \left(  \left( S^{1}_\sigma \vee S^{1}_\tau \right) \cup_{\tau^{5}}  e^{2} \right)\cup_{\sigma^{2}} e^{2}} \\
	\arrow[" \Sigma^{2} ", from=1-1, to=1-2]
	\arrow[ " \iota", from=1-1, to=2-1]
	\arrow[from=1-2, to=2-2]
	\arrow[from=2-1, to=2-2]
\end{tikzcd}\]
Où $\Sigma^{2}$ est la concaténation de $\Sigma$ avec soi-même.\\
Et finalement, on construit le pushout
\[\begin{tikzcd}
	{ S^{1}	} & {  \left(  \left( S^{1}_\sigma \vee S^{1}_\tau \right) \cup_{\tau^{5}}  e^{2} \right)\cup_{\sigma^{2}} e^{2}} \\
	{ e^{2}} & { \left(  \left(  \left( S^{1}_\sigma \vee S^{1}_\tau \right) \cup_{\tau^{5}}  e^{2} \right)\cup_{\sigma^{2}} e^{2} \right)\cup_{\sigma\tau\sigma\tau} e^{2} = X} \\
	\arrow[" S \ast T \ast S\ast T ", from=1-1, to=1-2]
	\arrow[ " \iota", from=1-1, to=2-1]
	\arrow[from=1-2, to=2-2]
	\arrow[from=2-1, to=2-2]
\end{tikzcd}\]
De par la partie $b$, l'espace $X$ ainsi obtenu aura bien $\langle \sigma, \tau | \sigma^{2}, \tau^{5}, \sigma\tau\sigma \tau \rangle = D_{10}  $ comme groupe fondamental.
%On prétend d'abord que $ C( \vee_A S^{1}) $ a le meme type d'homotopie que $ \vee_A e^{2}$, en effet, ce sont les deux des espaces contractibles parce que
%\begin{itemize}
%\item Le cone sur un espace est toujours contractible, ie. $C( \vee_A S^{1}) $ est homotope au singloton.
%\item Une $2$-cellule est toujours contractible et donc un wedge de 2-cellules l'est aussi.
%\end{itemize}
%On définit maintenant une application $f:\coprod_{a\in A}  S^{1}_a\to \vee_I S^{1}$, ici on note $S^{1}_a$ pour garder en tête quelle copie de $S^{1}$ on utilise. \\
%Par la propriété universelle du coproduit, il suffit de déterminer $f$ sur chaque $S^{1}_a$ et on définit
%$f|_{ S^{1}_a} = a$ ou $a$ est simplement le mot définit par le relateur $a$ dans $\vee_I S^{1}$ .\\
%Cette application $f$ passe au quotient de $\bigvee_A S^{1}$(tous les points de base des $S^{1}$  sont envoyes sur le meme point dans $\vee_I S^{1}$)   et induit donc une application $\overline{f}: \bigvee_A S^{1}\to \bigvee_I S^{1}$.\\
%On construit maintenant le pushout
%\[\begin{tikzcd}
	%{\bigvee_A S^1} & {\bigvee_I S^1} \\
	%{\bigvee_A e^2} & X
	%\arrow["{\overline{f}}", from=1-1, to=1-2]
	%\arrow["\iota"', from=1-1, to=2-1]
	%\arrow[from=1-2, to=2-2]
	%\arrow[from=2-1, to=2-2]
%\end{tikzcd}\]
%où $\iota$ est simplement l'inclusion.\\
%On prétend maintenant que l'espace $X'$ obtenu par le pushout suivant
%\[\begin{tikzcd}
	%{\bigvee_A S^1} & {\bigvee_I S^1} \\
	%{C\left( \bigvee_A S^{1}\right) } & X'
	%\arrow["{\overline{f}}", from=1-1, to=1-2]
	%\arrow["\iota"', from=1-1, to=2-1]
	%\arrow[from=1-2, to=2-2]
	%\arrow[from=2-1, to=2-2]
%\end{tikzcd}\]
%aura le même type d'homotopie que $X$.\\
%En effet, soit $\phi: C\left(\vee_A S^{1} \right) \to \vee_A e^{2}$ et $\psi:\vee_A e^{2} \to C\left( \vee_A S^{1} \right)$ deux applications telles que $\phi\circ\psi \simeq \id_{\vee_A e^{2}} $ et $\psi\circ\phi \simeq \id_{C( \vee_A S^{1} )} $.\\
%Notons qu'on peut choisir ces deux homotopies de telle manière qu'elle fixe le sous-espace $\vee_A S^{1}$. En effet, l'homotopie collapse simplement les hauts des cones sur le wedge des cercles.\\
%Alors, on construit les applications $\phi \coprod \id : \vee_A e^{2}\coprod \vee_I S^{1}\to C\left( \vee_A S^{1}\right) \coprod \vee_I S^{1}$ et $\psi \coprod \id$ de la même manière.\\
%Alors, l'homotopie $H: \phi\circ \psi \simeq \id$ induit une homotopie $\phi\coprod \id \circ \psi \coprod \id\simeq \id$ qui passe au quotient du recollement, finalement, on obtient une homotopie induite $\tilde H$ 
%\[ 
%\tilde H : \left( \phi \cup_{ \vee_A S^{1}}  \id  \right)\circ \left( \psi \cup_{\vee_A S^{1}} \id  \right)\simeq \id
%\]
%Par symmétrie de l'argument, on en déduit que $X$ et $X'$ ont le même type d'homotopie.\\

%Or, par le corrolaire de Seifert-Van Kampen pour les attachements de cellule, le foncteur $\pi_1$ va préserver le pushout ci-dessus et on obtient le pushout suivant dans $Gr$ 
%\[\begin{tikzcd}
	%{F(A)} & {F(I)} \\
	%1 & {\pi_1(X') = \pi_1(X)}
	%\arrow["{\overline{f}_\ast}", from=1-1, to=1-2]
	%\arrow["{\iota_*}"', from=1-1, to=2-1]
	%\arrow[from=1-2, to=2-2]
	%\arrow[from=2-1, to=2-2]
%\end{tikzcd}\]
%Etant donné que l'application $\overline{f}_\ast$ envoie chaque générateur de $F( A) $ sur le mot associé dans $I$, il est clair que $\pi_1(X) = \pi_1( X') = \langle A | I\rangle= G$.\\
%\section*{c)}
%Notons que $ \faktor { \mathbb{Z}} { 5 \mathbb{Z}} = \langle a | a^5\rangle$ et on construit donc le pushout suivant:
%\[\begin{tikzcd}
	%{ S^1} & { S^1} \\
	%{e^{2}} & S^{1}\cup_5 e^{2}	
	%\arrow["{f}", from=1-1, to=1-2]
	%\arrow["\iota"', from=1-1, to=2-1]
	%\arrow[from=1-2, to=2-2]
	%\arrow[from=2-1, to=2-2]
%\end{tikzcd}\]
%Ou ici, $f$ est n importe quelle application de degre $5$.\\
%Par la partie $b$, on conclut que l'espace ci-dessus a $ \faktor { \mathbb{Z}} { 5 \mathbb{Z}} $ comme groupe fondamental.\\

%Pour construire un espace dont le groupe fondamental est $ D_{10}= \langle a,b | a^{2}, b^{5}, abab\rangle $.\\
%On choisit trois lacets $\gamma_{a^{2}} , \gamma_{b^{5}} , \gamma_{abab} : S^{1}\to S^{1}\vee S^{1}$ qui sont des représentants des mots $ a^{2}, b^{5}, abab$ respectivement dans le groupe fondamental de $ S^{1}\vee S^{1}$.\\
%Par la propriete universelle du coproduit, on a alors une application $\gamma: S^{1}\coprod S^{1}\coprod S^{1}\to S^1\vee S^{1}$, cette application passe au quotient et induit une application $ \overline{\gamma}: S^{1}\vee S^{1}\vee S^{1}\to S^{1}\vee S^{1}$.\\
%On forme alors le pushout suivant
%\[\begin{tikzcd}
	%{ S^1\vee S^{1}\vee S^{1}} & { S^1\vee S^{1}} \\
	%{e^{2}\vee e^{2}} & X
	%\arrow["{\overline{\gamma}}", from=1-1, to=1-2]
	%\arrow["\iota"', from=1-1, to=2-1]
	%\arrow[from=1-2, to=2-2]
	%\arrow[from=2-1, to=2-2]
%\end{tikzcd}\]
%Grace a la partie $b$, on sait que le groupe fondamental de cet espace sera $ D_{10} $.
\end{document}
