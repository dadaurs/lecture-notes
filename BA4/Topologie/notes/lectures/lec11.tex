\documentclass[../main.tex]{subfiles}
\begin{document}
\lecture{11}{Mon 11 Apr}{Fin Seifert Van-Kampen}
On veut montrer que l'application $\phi$ definie la derniere fois est une application de $\ker N= \ker ( \pi_1 A \ast \pi_1 B \to \pi_1A \ast_{\pi_1 C	} \pi_1 B )  $.\\
On decoupe donc $I\times I$ en $nm$ carres tels que $H|_{ C_k} \subset A$ ou $B$.\\
On construit $\omega_k$ comme indique sur le dessin.\\
On a $\omega_0 = c_{x_0} $ et $ \omega_{nm} = \gamma$.\\
Ainsi, $H|_{C_k} $ fournit une homotopie entre $\omega_k$ et $\omega_{k-1} $.\\
On va montrer que $\omega_k \ast \overline{\omega_{k-1} } \in N$ .\\
En effet, alors $\omega_{nm} = \omega_{nm} \ast \overline{\omega_{nm-1} }\ldots \ast \omega \ast \overline{\omega_0}$.\\
Pour chaque $k$, on fixe $ H|_{C_k} $ est vue dans ou $B$ si elle est dans $C= A\cap B$.\\
De meme, pour chaque chemin correspondant aux 4 cotes du rectangle.\\
Mettons que pour $k$, l'homotopie est dans $A$, alors les cotes "a gauche et en haut" sont choisis dans $A$.\\
Ainsi, les deux autres sont choisis dans $A$ ou $B$ selon $k=1$ et $k-n$.\\
S'ils sont tous dans $A$, alors $H|_{C_k} $ est une homotopie dans $A$ entre chemin dans $A$, $\omega_k\ast \overline{ \omega_{k-1} } \in N $.\\
Supposons qu'un cote au moins est un chemin dans $B$ 
\[ 
\omega_k \ast \overline{\omega_{k-1} }= \lambda_1 \ast \lambda_2 \ast \overline{\lambda_2} \ast \lambda_3 = \lambda_1 \ast \lambda_3
\]
Comme le point $y$ ( le cote en haut a droite du carre) appartient a $C$, on choisit un chemin $\gamma^{1}$ de $H( y) $ a $x_0$, de meme $\gamma^{2}$ pour $z$ ( le point en bas a droite du carre).
\begin{align*}
	\lambda_1 \ast \lambda_3 &= \lambda_1 \ast \gamma^{1}\ast \overline{\gamma^{1}} \ast \lambda_4 \ast \gamma^{2}\ast \overline{\gamma^{2}} \ast \lambda_5
	\intertext{Le chemin $\lambda_4$ est dans $B$, appelons $\lambda_4'$ le meme chemin vu dans $A$ , $H|_{C_k} $ est une homotopie dans $A$ tel que le chemin $\lambda_1\simeq \overline{\lambda_5}\ast \overline{\lambda_4'}$ }
	&= \overline{\lambda_5} \ast \gamma^{2} \ast \overline{\gamma^{2}} \ast \overline{\lambda_4'} \ast \gamma^{1}\ast \overline{\gamma^{1}} \ast \lambda^{4} \ast \gamma^{2} \ast \overline{\gamma^{2}}\ast \lambda_5\\
	&= \overline{\epsilon} \ast \overline{\alpha}\ast \beta \ast \epsilon	
\end{align*}
Represente un conjugue par $\epsilon$ dans $\pi_1 A\ast \pi_1 B$ du meme lacet $\lambda_4$ vu dans $B$ ou $\lambda_4'$ vu dans $A$.\\
Par definition de $\pi_1 A \ast_{\pi_1 C} \pi_1B$ , c'est un element de $N$ 

\begin{exemple}
$ \mathbb{R}P^{2} = \faktor { D^{2}} { \sim} \simeq S^{1}\cup_2 e^{2}$.\\
Pour recouvrir $ \mathbb{R}P^{2}$ par des ouverts, on epaissit $ \mathbb{R}P^{1}$ et on amincit $ e^{2}$.\\
On pose $ A =  \dot{ D }_{\frac{3}{4}} , B = D^{2}\setminus D_{\frac{1}{4}} $.\\
Comme $A,B,C$ sont satures, $q( A) ,q( B) , q( C) $ sont ouverts, on a donc
\[ 
q( A) = \ast, q( B) = q( S^{1}) = S^{1} \text{ et } q( C) = q( S^{1}) = S^{1}
\]
L'inclusion $C \subset B$ induit une application $q( C) \to q( B) $ 
\end{exemple}
\subsection{Groupe fondamental d'un Wedge}
On suppose que tous nos espaces sont pointes et \underline {bien pointes} dans le sens ou le point de base $x_0$ admet un voisinage ouvert et contractile au sens pointe, $U \simeq \left\{ 0 \right\} $ et l'homotopie $\id_U \simeq c_{x_0} $ fixe $x_0$ 
\begin{exemple}
$S^{1}$ est bien pointe, toutes les surfaces aussi, le peigne du topologue ne l'est pas en $ ( 0,1) $.
\end{exemple}
\begin{lemma}
Si $X,Y$ sont bien pointes, $X\vee Y$ aussi ( $X\times Y$ aussi).
\end{lemma}
\begin{proof}
Soient $U\ni x_0, V \ni y_0$ des voisinages contractiles de $x_0$ resp. $y_0$.\\
Dans $X \vee Y= X\coprod Y / x_0\sim y_0$, on choisit l'image de $U \coprod V \subset X\coprod Y$.\\
Les homotopies $H: \id_U \simeq c_{x_0} , F: \id_V\simeq c_{y_0} $ donne une homotopie $H\coprod F: \id_{U\coprod V} \to c_{x_0} \coprod c_{y_0} $ passe au quotient comme elle preserve le point de base.
\end{proof}
\begin{propo}
Soient $ ( X,x_0) , ( Y,y_0) $ deux espaces bien pointes, alors $\pi_1( X\vee Y) \simeq \pi_1X \ast \pi_1 Y$ 
\end{propo}
\begin{proof}
On prend $A= X \vee V$ et $B = Y\vee U$, alors on conclut par le lemme ci-dessus et Seifert.
\end{proof}







\end{document}	
