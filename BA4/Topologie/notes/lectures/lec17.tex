\documentclass[../main.tex]{subfiles}
\begin{document}
\lecture{17}{Mon 23 May}{Monodromie (mot cool)}
\begin{propo}
L'application $p: \tilde X \to X$ d'evaluation en $1$ est un revetement.
\end{propo}
\begin{proof}
	$\tilde X$ est connexe par arcs, $\Gamma$ est un chemin du point de base $ [ c_{x_0} ] \in \tilde X$ vers $[\gamma]$.\\
	Comme $X$ est connexe par arcs, $p$ est surjective.\\
	Pour montrer la continuite, notons que $p^{-1}( U) = \bigcup U_{[\gamma]} $ qui est ouvert.\\
	Les $U \in B$ sont des ouverts trivialisant dont les $U_{ [ \gamma] } $ sont des feuillets, pour des classes d'homotopie relatives de chemins de $x_0$ vers $x$ distinctes.\\
	On peut choisir et fixer un point d'arrivee $x$ de nos chemins $\gamma$.
Alors, si $U_{ [ \gamma] } \cap U_{ [ \gamma'] }\neq \emptyset $, alors ils sont egaux a $ U_{ [ \alpha] } $ pour  $ [ \alpha] $ dans l'intersection.\\
Comme $p(  U_{ [ \gamma] } ) = U\in B$, on a que $p$ est ouverte, il suffit donc de montrer que $p$ est une bijection lorsqu'on la restreint a $U_{ [\gamma] } $.\\
La surjectivite est clare, pour l'injectivite, supposons que $ p[ \gamma\ast\beta] $ et $ p [ \gamma \ast \beta'] $ et donc $\beta( 1) = y = \beta'( 1) $.\\
Comme $\beta' \ast \overline{\beta}$ est entierement contenu dans $U$, il est trivial et donc $ [ \gamma\ast \beta'] = [ \gamma \ast \beta' \ast \overline{\beta} \ast \beta] = [ \gamma\ast\beta] $.
\end{proof}
\begin{thm}
	Le revetement $p$ est un revetement universel (donc \underline{le} revetement universel).
\end{thm}
\begin{proof}
Comme $p_\ast : \pi_1 ( \tilde X , [ c_{x_0} ] ) \to \pi_1( X,x_0)$ est injective, il suffit de montrer que pour un lacet $\tilde\omega: I\to \tilde X$ base en $ [ c_{x_0} ], p_0\tilde\omega = \omega$ est homotopiquemeent constant dans $X$.\\
Soit $\Omega$ le chemin dans $\tilde X$ qui releve $\omega$, ie.
\[ 
\Omega( t) = [ \omega|_{ [ 0,t] } ] 
\]
Au debut, $\Omega( 0) = [ c_{x_0} ] $ et $\Omega( 1) = [ \omega] $.\\
Les deux chemins $\tilde \omega$ et $\Omega$ relevent tous deux $\omega$, par unicite $\Omega= \tilde \omega$, si bien que $ [ \omega] = [ c_{x_0} ] $.
\end{proof}
\subsection{Monodromie}
Soit $p: E \to X $ un revetement, $e_0\in E, x_0 = p( e_0) $, soit $F_{x_0} = p^{-1}( x_0) \ni e_0$.\\
L'action de monodromie $F_{x_0} \times \pi_1( X,x_0) $ qui envoie $ ( e_0, [ \omega] ) \to \tilde \omega ( 1) $ ou $\tilde \omega$ est le seul relevement de $\omega$ qui commence en $x_0$.
\begin{propo}
La monodromie est transitive, le stabilisateur de $e_0$ est $p_\ast( \pi_1( E,e_0) ) $ et $ F_x = \faktor{ \pi_1 ( X,x_0) }{ p_\ast ( \pi_1 ( E,e_0) ) }$ 
\end{propo}
\begin{proof}
Comme $E$ est connexe par arcs, on choisit un chemin $\alpha$ de $e_0$ vers $e_1$ dans $E$ pour $e_0, e_1\in F_x$, alors $p\circ\alpha$ est un lacet qui agit sur $e_0$ de la maniere voulue.\\
Le stabilisateur de $e_0$ est le sous-groupe des classes d'homotopie des lacets $ [ \omega] \in \pi_1( X,x_0) $ qui se relevent en un lacet  $\tilde \omega$ dans $ ( E,e_0) $.\\
Donc $ [ \omega] = p_\ast [ \tilde \omega] $.\\
En conclusion, par transitivite, $F_{ x_0} $ est en bijection avec $ \faktor{\pi_1 X}{ \text{ Stab } }$ et on a bien que $ \text{ Stab } = p_\ast \pi_1( E,e_0) $ 
\end{proof}
\begin{rmq}
Pour $e_0,e_1\in F_{x_0} $, $ p_\ast \pi_1( E,e_0)  $ et $ p_\ast \pi_1 ( E,e_1) $ ne sont pas egales, elles sont seulement conjuguees dans $ \pi_1( X,x_0) $.\\
\end{rmq}
\begin{propo}
Deux revetements $p: E\to X$ et $p': E'\to X$ sont isomorphes si et seulement si $ p_\ast ( \pi_1 ( E,e_0) ) $ et $ p_\ast'\pi_1( E',e_0') $ sont conjugues.
\end{propo}
\begin{proof}
Soit $f: E\to E'$ un isomorphisme de revetements.\\
Comme $f( e_0) = e_1'\neq e_0'$.\\
Les images des groupes fondamentaux sont conjugues par la remarque ci-dessus.\\
Supposons donc que $\im p_\ast$ et $\im p_\ast'$ sont conjugues, il existe donc $ [ \omega] \in \pi_1 ( X,x_0) $ tel que $ [ \omega]^{-1} p_\ast \pi_1( E,e_0) [ \omega] = p_\ast' \pi_1( E',e_0') $.\\
On releve alors $\omega$ en $\tilde \omega$ dans $E$ avec $\tilde \omega( 0) = e_0$, appelons $\epsilon = \tilde \omega( 1) \in F_{x_0} $.\\
Alors par la remarque $p_\ast \pi_1( E, \epsilon)$ est egal a $ p_\ast' \pi_1( E',e_0') $.\\
Ainsi, $( E, \epsilon) \simeq ( E', e_0') $ et donc $E\simeq E'$.
\end{proof}
\begin{thm}[Theoreme de classification]
Soit $Cov( X) $ l'ensemble des classes d'isomorphisme de revetement de $X$ et $Conj( G) $ l'ensemble des classes de conugaison des sous-groupes de $\pi_1( X,x_0)$.\\
Alors 
\begin{align*}
	\Phi: Conj( G) & \to Cov( X) \\
	[ H] &\mapsto p_H :X_H\to X
\end{align*}
ou $X_H$ est defini tel que dans la serie.\\
Alors $\Phi$ est une bijection d'inverse $\Psi$ qui associe a $( p:E\to X ) \mapsto p_\ast \pi_1( E,e_0)  $ 
\end{thm}








		
\end{document}	
