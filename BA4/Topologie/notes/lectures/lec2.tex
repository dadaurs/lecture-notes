\documentclass[../main.tex]{subfiles}
\begin{document}
\lecture{2}{Sat 26 Feb}{Conditions de Separation}
\subsection{Conditions de separation du quotient}
On donne une condition necessaire et une condition suffisante pour que le quotient soit separe
\begin{propo}
Soit $\sim$ une relation d'equivalence sur un espace $X$. Si $ X /\sim$ est separe, le graphe $\Gamma$  de la relation est ferme dans $X\times X$ 
\end{propo}
\begin{proof}
Si $X /\sim$ est separe, par un lemme, la diagonale $\Delta \subset X /\sim\times X/\sim$ est ferme.\\
Considerons $q \times q : X\times X \to X /\sim \times X /\sim$. Cette application est continue et donc $( q\times q)^{-1}( \Delta) $ est un ferme de $X \times X$. Or cette preimage est l'ensemble des paires de points $( x,y)\in X\times X $ tq $q( x) = q( y) \iff x\sim y$.
\end{proof}
On donne maintenant une condition suffisante permettant de conclure qu'un quotient est separe.
\begin{propo}
Soit $\sim$ une relation d'equivalence sur un espace $X$ separe. Si $q^{-1}( q( x) ) $ est compact pour tout point $x\in X$ et de plus que pour $F \subset X$ ferme $q^{-1}( q( F) ) $ est ferme, alors le quotient est separe.
\end{propo}
\begin{proof}
Soit $ \overline{x}= q( x) $ et $\overline{y}=q( y) $ deux points distincts de $X /\sim$.\\
Les saturations $q^{-1}( \overline{x}) , q^{-1}( \overline{y}) $ sont des compacts par hypothese.\\
Comme $X$ est separe, on peut separer des compacts avec des ouverts disjoints $U$ et $V$.\\
On a donc
\[ 
q^{-1}( \overline{x}) \subset U, q^{-1}( \overline{y}) \subset V \text{ et } U\cap V= \emptyset
\]
Posons $E= X\setminus U, F= X\setminus V$ deux fermes de X.\\
Par hypothese, les saturations $q^{-1}( q( E) ) $ et $ q^{-1}( q( F) ) $ sont fermes. Ainsi $U' = X \setminus q^{-1}( q( E) ) $ et $V' = X \setminus q^{-1}( q( F) ) $ sont des ouverts. On observe que $ E \subset q^{-1}( q( E) ) , F \subset q^{-1}( q( F) ) $, alors $U' \subset U, V' \subset V$.\\
De plus $ q^{-1}( q( x) ) \subset U' $ et $ q^{-1}( q( y) ) \subset V'$.\\
Il reste a montrer que $ q( U') $ et $q( V') $ sont ouverts dans $X /\sim$ et disjoints. Pour le premier point, il suffit de verifier que $ q^{-1}( q( U') ) $ est ouvert dans $X$. On pretend que $q^{-1}( q( U') ) = U'$.\\
En effet, $U' \subset q^{-1}( q( U') ) $ est toujours vrai, il faut donc montrer l'inclusion inverse.\\
Soit $u \in q^{-1}( q( U') ) $, donc $q( u) \in q( U') $. Donc $q( u) \notin q( E) $ et donc $u \in U'$ 
Le meme resultat est vrai pour $V'$.\\

Il faut donc finalement encore montrer que $q( U') $ et $q( V') $ sont des voisinages ouverts, de $ \overline{x}$ et $ \overline{y}$ disjoints.\\
Supposons qu'il existe $u' \in U', v' \in V'$ tel que $q( u') = q( v') $. Alors $u' \in q^{-1}( q( v') ) \subset q^{-1}( q( V') ) = V'  $.\\
Donc $U' \cap V' \neq \emptyset$, contradiction.
\end{proof}



\end{document}	
