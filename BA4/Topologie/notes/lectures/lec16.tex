\documentclass[../main.tex]{subfiles}
\begin{document}
\lecture{16}{Mon 16 May}{Revetement universel}
\begin{exemple}
Comme contreexemple, notons que par exemple pour $p: \mathbb{R}\to S^{1}$ l'application $d: S^{1}\to S^{1}$ n'admet pas de relevement.
\end{exemple}
\subsection{Revetements universels}
\begin{defn}[Revetement Universel]
	Un revetement universel de $X$ est un revetement $p: E\to X$ avec $\pi_1E =1$ 
\end{defn}
\begin{rmq}
Si un revetement universel existe, alors pour tout revetement $q: E'\to X$, le revetement de $p: E\to X$ se factorise a travers $q'$.\\
Ceci suit du fait que $\im p_\ast < \im q_\ast$.\\
De plus, si $p$ existe et $U$ est un ouvert trivialisant contenant un point $x\in X$, alors on a que l'inclusion $\iota: U\to X$ se factorise a travers $E$.\\
En effet, si $U_e\xrightarrow{j} E$ est l'inclusion d'un feuillet et $q: U_e \to U$ est l'homeo.,alors $j\circ q^{-1}$ releve $\iota$.\\
Par la proposition, $\iota_\ast \pi_1( U;x) \subset p_\ast \pi_1 ( E,e) =1$.\\
Ainsi, tous les lacets de $U$ bases en $x$ sont homotopiquement triviaux dans $X$.	
\end{rmq}
\begin{defn}[Semi-localement simplement connexe]
	Un espace $X$ est semi-localement simplement connexe ( $\frac{1}{2}$-loc 1-connexe) si tout point $x\in X$ admet un voisinage ouvert tel que l'inclusion $\iota : U \hookrightarrow X$ induit l'homomorphisme trivial sur les groupes fondamentaux.
\end{defn}
\begin{lemma}
L'ensemble de tous les ouverts $U \subset X$ connexes par arcs tel que $\pi_1 ( U,x) \to \pi_1( X,x) $ forment une base pour la topologie de $X$ 	
\end{lemma}
On definit $\tilde B$ une base d'une topologie sur $\tilde X$ en posant pour tout $[\gamma]\in \tilde X$ et tout ouvert $U\in B$ avec $\gamma( 1) \in U$ comme etant $U_\gamma= \left\{ [ \alpha] \in \tilde X| \exists\beta \text{ chemin de $U$ tel que } [ \alpha] = [ \gamma\circ\beta] \right\} $ .\\
Pour $\gamma:I\to X, \gamma( 0) = x_0,\gamma( 1) = x, [ \gamma] \in \tilde X$, pour $x\in U \in B$, on pose $U_{[\gamma]} = \left\{ [ \gamma\ast\beta] |\beta:I\to U, \beta( 0) = x \right\} $.\\
On pose alors $ \tilde B = \left\{ U_{ [ \gamma] } |\gamma, U \right\} $.
\begin{lemma}
$\tilde B$ est la base d'une topologie sur $\tilde X$ 
\end{lemma}



\end{document}	
