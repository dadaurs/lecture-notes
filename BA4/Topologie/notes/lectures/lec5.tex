\documentclass[../main.tex]{subfiles}
\begin{document}
\lecture{5}{Sat 12 Mar}{Homotopies et groupe fondamental}
\section{Homotopies et Groupe Fondamental}
\subsection{Homotopie}
\begin{defn}[Homotopie entre applications]
	Soient $f,g: X\to Y$ des applications. On dit que $f$ et $g$ sont homotopes et on note $f\simeq g$ s'il existe une application $H:X\times I\to Y$ tel que $H( -, 0) =f$ et $H( -,1) =g$.\\
	On appelle $H$ une \underline { homotopie}.
\end{defn}
\begin{propo}
La relation $\simeq $ est une relation d'equivalence.
\end{propo}
\begin{proof}
\subsection*{Reflexivite}
Suit du fait qu'on peut definir une homotopie constante.
\[ 
H:X\times I \to Y: ( x,t) \mapsto f( x) 
\]

\subsection*{Symetrie}
La symetrie suit du fait qu'on peut parcourir une homotopie dans l'autre sens.\\
Ainsi, soit $H:X\times I\to Y$ une homotopie entre $f$ et $g$. On pose
\[ 
G:X\times I \to Y : ( x,t) \mapsto H( x,1-t) 
\]

\subsection*{Transitivite}
Supposons que $H:X\times I\to Y, G:X\times I \to Y$ sont des homotopies, $f\simeq g\simeq h$. On construit une homotopie $K :X\times I\to Y$ entre $f$ et $h$ 
\[ 
	( x,t) \mapsto
	\begin{cases}
H( x,2t) \text{ si } 0 \leq t	\leq \frac{1}{2}\\
G( x,2t-1) \text{ si  } \frac{1}{2}< t \leq 1	
	\end{cases}
\]
On voit que $K$ est continue et montre que $f\simeq h$.
\end{proof}
\begin{defn}[Classes d'homotopie]
On note $ [ X,Y] $ l'ensemble des classes d'homotopies d'applications $f:X\to Y$.\\
C'est donc $\faktor{  C( X,Y) }{\simeq} $.
\end{defn}
		


	
\end{document}	
