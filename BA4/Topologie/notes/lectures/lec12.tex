\documentclass[../main.tex]{subfiles}
\begin{document}
\lecture{12}{Thu 21 Apr}{Retractes}
\begin{defn}[Retracte]
	Un sous-espace $\iota: A\hookrightarrow X$ est un retracte de $X$  s'il existe une retraction $r:X\to A$ tel que $r\circ \iota= \id_A$.
\end{defn}
\begin{exemple}
\begin{enumerate}
\item $S^1$ est un retracte de $S^1 \vee S^{1}$.\\

\item Tout point $x_0\in X$ est un retracte de $X$ $r: X\to \left\{ x_0 \right\} $ .
\end{enumerate}
\end{exemple}
\begin{rmq}
$S^{1}$ n'est pas un retracte du disque $D^{2}$, il n'existe aucune application continue $r: D^{2} \to S^{1}$ tel que $r( x) = x $ si $x\in S^{1}$.\\
Sinon $ \pi_1S^{1}\xrightarrow{\iota_\ast} \pi_1 D^{2} \xrightarrow{r_\ast} \pi_1 S^{1}$ .\\
Si la composition $r\circ \iota$ etait l'identite, la composition $ \mathbb{Z}\to 0 \to \mathbb{Z} $ serait l'identite, ce qui est impossible.
\end{rmq}
\begin{defn}[Retracte de deformation ]
	Un retracte $\iota: A \hookrightarrow X$ est un retracte de deformation de $X$ s'il existe une homotopie $\iota \circ r \simeq \id_X$
\end{defn}
\begin{exemple}
Le peigne du topologue $P$ $ \left\{ ( 0,1)  \right\} \subset P$ est un retracte de deformation.\\
On definit l'homotopie $H$ en trois temps.
\begin{enumerate}
\item Contracter les dents du peigne
\item Contracter la base du peigne.
\item Remonter en $ ( 0,1) $ 
\end{enumerate}
Cette homotopie n'a pas fixe le point $ ( 0,1) $ 
\end{exemple}
\begin{defn}[Retracte de deformation fort]
	Un retracte de deformation $\iota: A\hookrightarrow X$ est un retracte de deformation fort si l'homotopie $H: \iota\circ r \simeq \id_X$ peut etre choisie relative a $A$, ie, $H( a,t) =a$ pour tout $t\in I$ et pour tout $a\in A$ 
\end{defn}
\begin{exemple}[Le collier]
	Soit $A$ un espace et $ Col( A) = A \times [ 0, \frac{3}{4}[ $, l'inclusion de $A\times 0\hookrightarrow Col( A) $ est un retracte de deformation fort.\\
	On pose $r( a,t) = ( a,0) \forall a\in A, \forall t$.\\
	On definit donc $H: Col( A) \times I \to Col( A) $ en envoyant $( a,t,s) \mapsto ( a,ts) $ qui verifie clairement toutes les hypotheses ci-dessus.
\end{exemple}

	

		
\end{document}	
