\documentclass[../main.tex]{subfiles}
\begin{document}
\lecture{6}{Mon 14 Mar}{Homotopies}
\begin{defn}[Espaces Homotopes]
	Deux espaces $X$ et $Y$ sont homotopes ou homotopiquement equivalent, note $X\simeq Y$, s'il existe $f:X\to Y$ et $g:Y\to X$ tel que
	\[ 
	g\circ f \simeq \id_X \text{ et } f\circ g \simeq \id_Y
	\]
	On dit que $f$ et $g$ sont des equivalences homotopiques et qu'elles sont inverses homotopiques l'une de l'autre.
\end{defn}
\begin{propo}
\[ 
CX\simeq \ast
\]

\end{propo}
\begin{proof}
Posons $CX = \faktor { X\times I} { X\times 0} $.\\
On pose $f: \ast \to CX$ par $f( \ast) = [ x,1] $ et on prend $g:CX\to \ast$.\\
On a $g\circ f=\id_\ast$, il reste a voir que $f\circ g\simeq \id_{CX} $. On construit une homotopie $H:CX\times I\to CX$, defini par
\[ 
H( [ x,t] ,s) \mapsto [ x,ts] 
\]
C'est une application ( trivialement bien definie) et c'est une homotopie entre $f\circ g\simeq \id_{CX} $ 
\end{proof}

\begin{rmq}
Si $f$ et sont des applications pointees $( X,x_0) \to ( Y,y_0) $ qui sont homotopes au sens non pointe, il est faux en general que $f\simeq_{\ast} g$ au sens pointe.\\
Par exemple $f,g: S^1\to S^1\bigvee S^1$, $f$ est donnee par $a$ et $g$ est donnee par $b\star a\star b^{-1}$ ( concatenation).\\
On a que $f\simeq g$ pour $f_t: S^1\to S^{1}\bigvee S^{1}$ donne par $b|_{[1-t,1]} \star a \star \overline{b}|_{[0,t]} $ 
\end{rmq}
\subsection{Attachement de cellules}
\underline{But}: $f\simeq g: A\to X$, alors
\[ 
X\cup_f CA\simeq X\cup_G CA
\]
\begin{propo}
Si $f,g: A\to X$ sont homotopes, alors $X\cup_f CA \simeq X\cup_g CA$ 
\end{propo}
\begin{proof}
Pour comparer les deux espaces $Y= X\cup_f CA$ et $Y' = X\cup_g CA$, on construit des applications $h:Y\to Y'$ et $k:Y'\to Y$.\\
On definit $h:Y\to Y'$ par la propriete universelle du pushout.\\
On choisit $\iota':X\to Y'$ l'application donnee par la construction de $Y'$.\\
On pose 
\begin{align*}
	\alpha: CA&\to Y'
	[ a,t] &\mapsto 
	\begin{cases}
	H( a,2t) \text{ si } t \leq \frac{1}{2}\\
	[ a,2t-1] \text{ si } t >\frac{1}{2}
	\end{cases}
\end{align*}
Si $t=0$, alors $H( a,0) = f( a) $ donc le diagramme commute.\\
Si $t=\frac{1}{2}, H( a,1) = g( a) $. On construit $k$ comme $h$, mais avec $H( -, 1-t) $.\\
On doit montrer que $k\circ h \simeq \id_Y$ ( et de meme $h\circ k\simeq \id_{Y'} $) 
\end{proof}
\begin{crly}
Si $f,g: S^{n-1}\to X$ et $f\simeq g$, alors $X\cup_f e^{n}\simeq X\cup_g e^{n}$.
\end{crly}
\begin{crly}
Si $f:A\to X$ est homotope a $c_x$ constante, alors $X\cup_f CA\simeq X\bigvee \sum A$ 
\end{crly}
\subsection{Homotopie et $\pi_0$ }
Soit $S_0= \left\{ \pm 1 \right\} $ sphere unite de $ \mathbb{R}$.\\
On etudie les applications pointees de $( S_0,1 )\to ( X,x_0) $. Ainsi $f( 1) = x_0$ et $f( -1) = x$ abritraire.\\
Deux telles applications $f$ donnee par $x$ et $f'$ donee par $x'$ sont homotopes ( au sens pointe) s'il existe une homotopie pointee
\[ 
H: S^{0}\times I\to X
\]
$H$ est donc simplement donne par $H( -1,t) $, un chemin dans $X$ de $x$ vers $x'$.\\
Donc $x$ et $x'$ sont dans la meme composante connexe par arcs.

\begin{propo}
	L'ensemble $\pi_0X$ des composantes connexes par arcs est en bijection avec $[S_0,X]_\ast$ 
\end{propo}
\subsection{Invariance Homotopique}
Soit $f:X\to Y$, elle induit une application
\begin{align*}
	f_\ast: [ A,X] &\to [ A,Y] \\
	[ g] &\mapsto [ f\circ g] 
\end{align*}
\begin{proof}
On veut montrer que l'application ci-dessus est bien definie.\\
Si $g\sim g'$ via l'homotopie $G$, alors $f\circ g \simeq f\circ g'$ via $f\circ G$ 
\end{proof}
\begin{propo}
Si $f\simeq f': X\to Y$, alors $f_\ast = f'_\ast$.
\end{propo}
\begin{proof}
On choisit $H:X\times I\to Y$ une homotopie entre $H( -,0) =f$ et $H( -,1) = f'$.\\
On veut montrer que $f\circ g\simeq f'\circ g$.\\
On construit $G: A\times I\to X\times I\to Y$ en envoyant
\[ 
	( a,t) \mapsto ( g( a) ,t ) \mapsto H( g( a) ,t) 
\]
	
\end{proof}
\begin{crly}
Si $X\simeq Y$, alors $ [ A,X] \simeq [ A,Y] $ comme ensembles.
\end{crly}
\begin{proof}
	On a $f:X\to Y$ et $f':Y\to X$ inverses homotopes l'une de l'autre. Alors $ [ A,X] \to [ A,Y]\to [ A,X]  $ 
\end{proof}




\end{document}	
