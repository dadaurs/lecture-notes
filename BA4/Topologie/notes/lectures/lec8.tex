\documentclass[../main.tex]{subfiles}
\begin{document}
\lecture{8}{Mon 28 Mar}{Tore a deux trous}
\begin{exemple}
Posons $T=S=T^{2}$, on construit $T^{2}\# T^{2}$, un tore a deux trous.\\
$ T^{2}= \faktor { I\times I} { \sim}  = \faktor { \mathbb{R}^{2}} { \mathbb{Z}^{2}} $.\\
On va choisir des points $s,t$ dans $S$ et $T$ respectivement de coordonnees $ ( \frac{4}{5},\frac{4}{5}) $ et $ ( \frac{1}{5}, \frac{4}{5}) $.\\
On choisit $ U$ et $V$ comme deux goutes autour de $s$ respectivement $V$.\\
Le quotient du pentagone par $ A\sim A'$ donne un espace homeomorphe a $ I\times I \setminus U$.
\end{exemple}

\end{document}	
