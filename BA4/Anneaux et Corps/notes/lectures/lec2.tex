\documentclass[../main.tex]{subfiles}
\begin{document}
\lecture{2}{Wed 23 Feb}{Ideaux}
\begin{propo}
$a\in A$ inversible $\implies$ non diviseur de zero
\end{propo}
\begin{proof}
$c=1\cdot c = bac=0$ 
\end{proof}
\begin{crly}
$a\in A^{\times}$ $\implies$ inverse est unique.
\end{crly}
\begin{defn}
$A$ est integre s'il est commutatif et sans diviseurs de 0
\end{defn}
\subsection{Corps des Fractions}
\begin{defn}[Corps de Fractions]
	$A$ integre, si $A \mapsto K$ est un sous-anneau d'un corps $K$ tel que $\forall x \in K$, $\exists a \in A,b\in A\setminus \left\{ 0 \right\} $ $x=\frac{a}{b}$.\\
	Alors $K$ est un corps de fractions.

\end{defn}
\begin{thm}
	Pour chaque anneau integre, il existe un corps de fractions $A\to K$.
\end{thm}
\begin{proof}
$K= A\times A\setminus \left\{ 0 \right\} /\sim$ avec $( a,b) \sim ( a',b') \iff ab'=ba'$ 
\end{proof}
\begin{propo}[Propriete universelle du corps des fractions]
Pour chaque morphisme $\iota:A\to L$ avec $L$ un corps, $\iota$ factorise de maniere unique a travers $Frac( A) $ 
\end{propo}
Car c'est un objet universel dans $\ast \downarrow U$, le corps des fractions est unique.
\subsection{Ideaux et anneaux quotients}
Soit $I \subset A$ un sous-ensemble.
\begin{defn}[Ideal]
	Un sous-ensemble $I$ est un ideal si c'est un sous-groupe additif qui est stable par multiplication par les elements de $A$.
\end{defn}





\end{document}	
