\documentclass[11pt, a4paper]{article}
\usepackage[utf8]{inputenc}
\usepackage[T1]{fontenc}
\usepackage[francais]{babel}
\usepackage{lmodern}
\usepackage{mathtools}
\usepackage{tikz}
\usepackage{tikz-cd}


\usepackage{amsmath}
\usepackage{amssymb}
\usepackage{amsthm}
\renewcommand{\vec}[1]{\overrightarrow{#1}}
\newcommand{\del}{\partial}
\DeclareMathOperator*{\sgn}{sgn}
\DeclareMathOperator*{\id}{Id}
\DeclareMathOperator*{\im}{Im}
\DeclareMathOperator*{\re}{Re}
\DeclareMathOperator*{\vol}{Vol}
\newcommand\norm[1]{\left\vert#1\right\vert}
\newcommand\ns[1]{\left\vert\left\vert\left\vert#1\right\vert\right\vert\right\vert}
\newcommand\Norm[1]{\left\lVert#1\right\rVert}
\newcommand\N[1]{\left\lVert#1\right\rVert}
\newcommand\abs[1]{\left\vert#1\right\vert}
\newcommand\inj{\hookrightarrow}
\newcommand\surj{\twoheadrightarrow}
\newcommand\ded[1]{\overset{\circ}{#1}}
\newcommand\sidenote[1]{\footnote{#1}}
\newcommand\eng[1]{\left\langle#1\right\rangle}
\newcommand\hr{
    \noindent\rule[0.5ex]{\linewidth}{0.5pt}
}

\newcommand{\incfig}[1]{%
    \def\svgwidth{\columnwidth}
    \import{./figures}{#1.pdf_tex}
}
\newcommand{\filler}[1][10]%
{   \foreach \x in {1,...,#1}
    {   test 
    }
}

\newcommand\contra{\scalebox{1.5}{$\lightning$}}
\makeatother
\def\@lecture{}%
\newcommand{\lecture}[3]{
    \ifthenelse{\isempty{#3}}{%
        \def\@lecture{Lecture #1}%
    }{%
        \def\@lecture{Lecture #1: #3}%
    }%
    \subsection*{\@lecture}
    \marginpar{\small\textsf{\mbox{#2}}}
}

\DeclareMathOperator{\fr}{Frac}
\begin{document}
\title{Series 2 Exercise 7}
\author{David Wiedemann}
\maketitle
\section{$\nu( 1) = 0$ and $\nu( -1) =0$  }
Indeed, note that 
\[ 
\nu( 1\cdot 1) = \nu( 1) +\nu( 1) = \nu(1) \iff 2\nu( 1) =\nu( 1) \iff \nu( 1) =0
\]
Now for the second part, notice that since $-1\cdot -1 = 1$ we get
\[ 
\nu( -1\cdot -1) = \nu( -1) +\nu( -1) = \nu( 1) = 0\iff 2\nu( -1) =0 \iff \nu( -1) =0
\]
\section{$R_\nu$ is a subring of $K$ }
To show $R_\nu$ is a subring, we have to show that $  1,0\in R_{\nu} $ and that $R_\nu$  is closed under addition and multiplication.\\

Using the first part of the exercise, we immediatly get that $1\in R_\nu$ since $\nu( 1) \geq 0$  and by definition $0\in R_\nu$.
\subsubsection*{ $R_\nu$ is closed under multiplication}
Let $x,y \in R_\nu\setminus \left\{ 0 \right\} 	$, we get that $\nu( x\cdot y) = \nu( x) +\nu( y) \geq 0$ since $\nu( x) ,\nu( y) \geq 0$ by hypothesis.\\
If either $x$ or $y$ is equal to 0, then clearly $x\cdot y = 0\in R_\nu.$\\
Hence $R_\nu$ is closed under multiplication.
\subsubsection*{ $R_\nu$ is closed under addition}
Indeed, let $x,y\in R_\nu$, now $ \nu( x+y) \geq \min ( \nu( x) ,\nu( y) ) \geq 0$ hence $x+y\in R_\nu$.\\

This shows that $R_\nu$ is a subring of $K$.
\section{$K$ is the fraction field of $R_\nu$ }
We explicit an isomorphism between $K$ and $\fr R_\nu$.\\
Let $j: R_\nu \to K$ be the inclusion of $R_\nu$ in $K$, this is obviously a ring homomorphism.\\
Now applying the universal property of the fraction field to $j$, we get a unique ring homomorphism $\phi: \fr R_\nu\to K$ making the following diagramm commute:

\[\begin{tikzcd}
	{R_\nu} & K \\
	{\fr R_\nu}
	\arrow["j", from=1-1, to=1-2]
	\arrow["\iota"', from=1-1, to=2-1]
	\arrow["{\exists ! \phi}"', from=2-1, to=1-2]
\end{tikzcd}\]
Recall from the proof of the universal property of the fraction field that $\phi$ is defined by
$\phi( \frac{a}{b}) = \iota( a) \cdot \iota( b) ^{-1}$.\\
We now show $\phi$ is injective.\\
Indeed, suppose $\phi( \frac{a}{b}) = \phi( \frac{c}{d}) $, then $j ( a) j( b) ^{-1}= j( c) j( d)^{-1}\implies j(a ) j( d) = j( c) j(b)\iff j( ad) = j( cb) $, ie. that $ad= cb$, which in turn implies $\frac{a}{b}= \frac{c}{d}$ in $ \fr R_\nu$. Here we used the fact that $j$ is injective. \\
Thus, we only need to show that $\phi$ is surjective.\\
Let $a\in K$, if $\nu( a) \geq 0$, then clearly $ \phi( \frac{a}{1}) = j( a) \cdot 1= a$.\\
If $\nu( a) <0$, then notice that $\frac{1}{\frac{1}{a}}\in \fr R_\nu$ since $ \nu( 1) = \nu( \frac{a}{a}) = \nu( a) + \nu( \frac{1}{a}) \iff \nu( a) = -\nu( \frac{1}{a}) $, hence if $a$ has a negative valuation, $\frac{1}{a}\in R_\nu$   \\
Finally the following calculation shows that $\phi$ is surjective.
\[
\phi\left( \frac{1}{\frac{1}{a}}\right) = j( 1) \cdot j\left( \frac{1}{a}\right)^{-1} = a 
\]
Hence $ \fr R_\nu \simeq K$ which concludes the proof.

\section{ For every $x\in \mathbb{Z}, \nu( x) \geq 0$ }
Obviously $ \nu( 0) $ is undefined so I guess this is a typo and I show the result for $ \mathbb{Z}\setminus \left\{ 0 \right\} $.\\

Indeed, since $\nu( 1) =\nu( -1) =0$, we get $\forall x \in \mathbb{Z}, x >0$ 
\[ 
\nu(x) = \nu( \underbrace{1+\ldots + 1}_{ x \text{ times } }) \underbrace{\geq}_{ \text{ since $\nu$ is a valuation } } 0 
\]
Similarly, if $x<0$, we may write $ \nu( x) = \nu( \underbrace{-1\ldots-1}_{x \text{ times } }) \geq 0 $ by the same argument as above.
\section{ If $\nu( p) =0$ for all primes $p$, then $\nu$ is trivial.}
First, note that, since we may write any integer as product of primes, we get that for all $x\in \mathbb{Z}\setminus \left\{ 0 \right\} $, $\nu( x) = \nu\left( \prod_{i=1}^{n} p_i\right) = \sum_{i=1}^{ n} \nu( p_i) = 0 $, where $\prod_{i=1}^{n}p_i$ is the decomposition of $x$ into prime factors.\\
For the general case, first notice that $\forall x\in \mathbb{Q}$, we have that $\nu( 1) = \nu( \frac{x}{x}) = \nu( x ) +\nu( \frac{1}{x}) =0$, hence $\nu( x^{-1}) = -\nu( x) $.\\
Hence, for $\frac{a}{b}\in \mathbb{Q}, a,b \in \mathbb{Z}$, we get $\nu( \frac{a}{b})= \nu( a ) -\nu( b) =0 $, thus implying $\nu$  is trivial.
\section{$\nu( p) \neq 0$ happens for at most one prime}
Let $p,q$ be primes in $\mathbb{Z}$ and suppose $\nu( p) ,\nu( q) \neq 0$, then by part 4, we know that $\nu( p),\nu( q) >0 $.\\
By Bezout's equality, there exist $a,b \in \mathbb{Z}$ such that $ a p + bq =1$.\\
Applying $\nu$ to the above equality, we get
\[ 
\nu( ap + bq)  = \nu( 1) = 0 \geq \min ( \nu( ap) , \nu( bq) ) 
\]
Hence, either $ \nu( ap) \leq 0$ or $\nu( bq) \leq 0$. Without loss of generality, suppose $ \nu( ap) \leq 0$, then $\nu( a) +\nu( p) \leq 0$ which means that $\nu( a) <0$ ( since by hypothesis, $\nu( p) >0$ ), however, this contradicts part 3.
\section{$p$-adic valuation}
Suppose $\nu( p) = c$, then clearly, $\nu( p^{i}) = i\cdot c$.\\
Furthermore, if $a,b\in \mathbb{Z}$ are coprime to $p$, then
\[
\nu( \frac{a}{b}) = \nu( p_{a,1} \ldots p_{a,n} 	) - \nu( p_{b,1} \ldots p_{b,m} ) \underbrace{=}_{ \text{ all } p_{k,j} \text{ are prime to  } p.}0 
\]
, where $p_{a,1}\ldots p_{a,n}  $ ( resp. $ p_{b,1} \ldots p_{b,m} $  ) is the prime decomposition of $a$ ( resp. $b$ ) and the last equality follows from part 6.\\
Combining both observations above, we get that $ \forall \frac{c}{d}\in \mathbb{Q}$, we may write
\[ 
\nu( \frac{c}{d}) = \nu( p^{i} \frac{c'}{d'}) =\nu(p^{i}) + \nu( \frac{c'}{d'}) = i\cdot c
\]
where in the first equality, we have simply isolated all factors from $c$ and $d$ which are powers of $p$, hence implying that $c'$ and $d'$ are both coprime to $p$\footnote{One can always do this, simply consider the prime decomposition of $c$ and $d$ and isolate all powers of $p$.}.\\
We now show that $\nu_p$ is indeed a discrete valuation on $ \mathbb{Q}$ when $c=1$.\\
Let $ p^{i} \frac{a}{b}, p^{j} \frac{c}{d}\in \mathbb{Q}$ be fractions of the form stated in the instruction.\\
We then have $\nu\left( p^{i} \frac{a}{b} p^{j} \frac{c}{d}	\right) = \nu( p^{i+j} \frac{ac}{bd})  $, since both $ac$ and $bd$ are prime to $p$, we get $\nu\left( p^{i+j} \frac{ac}{bd}\right) = ( i+j )\cdot c = i+j$, showing the first property of a discrete valuation.\\
Now suppose without loss of generality, that $ i<j$, then we may write 
\[ 
\nu( p^{i}\frac{a}{b}+ p^{j}\frac{c}{d}) = \nu\left( p^{i}\left( \frac{a}{b}+ p^{j-i} \frac{c}{d}\right) \right) 
\]
Furthermore, note that $ \frac{a}{b}+ p^{j-i}\frac{c}{d}= \frac{ad+ p^{j-i}cb}{bd}$, notice that $bd$ is clearly prime to $p$, furthermore, we may write
\[ 
ad+ p^{j-i}cb = p^{k} l \text{ for some integers  } k \text{ and } l.
\]
From this, we deduce that 
\[ 
\nu( p^{i} \frac{a}{b}+ p^{j} \frac{c}{d}) = \nu( p^{i}) + \nu( p^{k}l) \geq \nu( p^{i}) \geq  \min ( i,j) 	
\]
Where the last inequality followed from our assumption that $i < j$.
\section{Valuation ring of $\nu_p$ is not $ \mathbb{Z}$ }
Indeed, to show this we simply have to find an element of $ R_{ \nu_p }$ which is not in $ \mathbb{Z}$, to see this take any integer $a\in \mathbb{Z}$ prime to $p$ and note that
\[ 
\nu( \frac{p}{a}) = 1\implies \frac{p}{a}\in R_{\nu_p} 	
\]
But obviously $\frac{p}{a}\notin \mathbb{Z} $.












\end{document}
