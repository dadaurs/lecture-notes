\documentclass[../main.tex]{subfiles}
\begin{document}
\lecture{20}{Wed 18 May}{Application to PDE's}
\subsection{The heat equation (on $\mathbb{R}$ ) }
The heat equation has the general form
\[ 
\begin{cases}
	\del_t u - \del_{xx} u =0\\
	u( 0,x) = f( x) 
\end{cases}
\]
We want to solve this for $u: \mathbb{R}\times [ 0, \infty ) \to \mathbb{R} $.\\
We formally derive a solution using the following strategy:
\begin{enumerate}
\item Take the fourier transform in $x$ 
	\[ 
	\del_t \hat{u}( \xi,t) + 4\pi^{2}\xi^{2} \hat{u}( \xi,t) = 0
	\]
	Now notice that for a fixed $\xi$, this is an ode in $\hat{u}$, we can compute it's solution:
	\begin{align*}
		\del_t [ \ln \hat{u}] &= - 4\pi^{2}\xi^{2}\\
		\ln \hat{u}( \xi,t)  - \ln \hat{u}( \xi,0) = - 4\pi^{2}\xi^{2}t
		\intertext{Taking exponentials}
		\hat{u}( \xi,t) = \hat{u}( \xi,0) e^{- 4\pi^{2}\xi^{2}t} = \hat{f}( \xi) e^{-4\pi^{2}\xi^{2}t} 
	\end{align*}
	Inverting the Fourier Transform yields
	\begin{align*}
		u( x,t) &= f \ast \widehat{e^{- 4\pi^{2}t\xi^{2}} }
	\end{align*}
	
	
\end{enumerate}
\begin{thm}[Solution to heat equation]
	Let $f\in \mathcal{S}( \mathbb{R}) $, define the \underline{heat kernel} 
	\[ 
	H_t( x) \coloneqq \frac{1}{( 4\pi t )^{\frac{1}{2}}} e^{ - \frac{x^{2}}{4t}} 
	\]
	Let $u= f \ast H_t$, then $u$ is
	\begin{enumerate}
	\item $C^{2}$ for $x\in \mathbb{R},t >0$ and solves the heat equation $\del_t u - \del_{xx} u =0$ 
	\item $u(t,x) \to f( x) $ as $t\to 0$
	\item $u( t,\cdot) \to f$ in the $L^{2}$ norm as $t\to \infty $ 
	\end{enumerate}
	
\end{thm}
\begin{proof}
	\begin{enumerate}

	\item Take the Fourier transform of $u$, this gives

\[ 
	\hat{u} = \hat{f} \widehat{H_t}
\]
The Fourier inversion formula gives
\[ 
u( t,x) = \int_{-\infty}^{+\infty}\hat{f}( \xi) e^{- 4 \pi^{2} \xi^{2}t+ 2\pi i \xi x} d\xi
\]
As a remark, notice that if one looks at the heat equation on $( - \infty ,0) $, the formal computations are the same \underline{but} the formula for $u$ makes no sense ( as then the heat kernel increases a lot when $t\to \infty $) .\\
Is $u$ differentiable in $x$ ?\\
If it is, differentiating under the integral gives
\begin{align*}
\del_x u( t,x) = \int_{-\infty}^{+\infty} \hat{f}( \xi) e^{-4 \pi^{2}\xi^{2} t + 2\pi i\xi x} ( 2\pi i \xi) d\xi
\end{align*}
To show this rigorously yields
\begin{align*}
\frac{u( t,x+h) - u( t,x) }{h}= \int_{-\infty}^{+\infty} \hat{f}( \xi) e^{- 4 \pi^{2}\xi^{2}t} e^{2\pi i \xi x}  \frac{ e^{2\pi i\xi h} -1 }{h}d\xi	
\end{align*}
Which converges to the same formula as $h\to 0$.\\
Doing the exact same reasoning again gives
\[ 
\del_{xx} u( t,x) = \int_{-\infty}^{+\infty} \hat{f}( \xi) e^{-4\pi^{2}\xi^{2} t} e^{2\pi i x\xi} ( 2\pi i \xi) ^{2} d\xi
\]
\item Finally, we want to prove that $u( t,x) - f( x) \to 0$ as $t\to 0$, indeed
	\begin{align*}
u( t,x) - f( x) = \int_{-\infty}^{+\infty}	H_t( y) ( f( x-y) - f( x)  ) dx	
\intertext{We claim that for $t$ sufficiently smal, the integrand is smaller than $\epsilon$  }
	\end{align*}
	Fix $R>0$ such that $|f| < \frac{\epsilon}{4}$ outside $ [ -R,R] $.\\
	$f$ is uniformly continuous in $ [ -R-1, R+1] $ thus there exists $\delta$ such that $| f( x) - f( x-y) | \frac{\epsilon}{2}$ if $ |y| < \delta$.\\
	Then
	\begin{align*}
		\int_{-\infty}^{+\infty} H_t ( f( x-y) -f( x) ) dy & \leq \int_{ |y|< \delta }^{  } H_t( f( x-y) - f( x) ) + \int_{ |y|> \delta }^{  } H_t (f( x-y) - f( x)  ) dy\\
								   & \leq \frac{\epsilon}{2} + 2 \sup f \int_{[ -\delta,\delta]} H_t dy \leq \epsilon	
	\end{align*}
	for $t$ big enough.
	

	
\item To prove the third point compute
\begin{align*}
	\int_{-\infty}^{+\infty} | u( t,x) - f( x) |^{2}dx &\underbrace{=}_{ \text{ Plancherel } } | \hat{u}- \hat{f}| ^{2}d\xi\\
&= \int_{-\infty}^{+\infty} |\hat{f}| | e^{- 4 \pi^{2}\xi^{2}t} -1 | d\xi \to 0	
\end{align*}
as $t\to 0$ by dominated convergence ( the integrand is dominanted by $4 |\hat{f}|^{2}$ ) 

	\end{enumerate}


\end{proof}
\subsubsection{Discussion: What about uniqueness}
Notice that it is sufficient to prove that for $f \equiv 0$, the solution is uniquely equal to 0.
We will only sketch the proof of the following theorem:
\begin{thm}
	If $u: \mathbb{R}\times [ 0, \infty ) \to \mathbb{R} $ is
	\begin{enumerate}
	\item a solution of the heat equation and $u( x,0) = 0$ 
	\item Continuous in $ [ 0, \infty ) \times \mathbb{R}, C^{2}(  \mathbb{R}\times (   0, \infty ) )  $ 
	\item $u( \cdot, t) $ is Schwartz uniformly in $t$ 
	\end{enumerate}
	Then $u \equiv 0$ 
\end{thm}
\subsection{Heat equation on an interval}
Consider the PDE
\[ 
\begin{cases}
\del_t u = c^{2} \del_{xx} u\\
u( 0,x) = f( x) \\
u( t,0) = u( t,L) =0
\end{cases}
\]
where $L,c>0, f \in C^{0}( [ 0,L] ) $ such that $f( 0) = f( L) =0$.\\
Let's start with a formal computation again.
\begin{enumerate}
\item First notice that we may reduce to the case $c=L =1$, indeed, if $u$ solves the heat equation in this special case, then, $v= u( \frac{L^{2}t}{c}, Lx) $ solves the general problem.
\item We can now look for solutions of the form $V( x) W( t) $, then the heat equation rewrites as
	\[ 
	V( x) W' ( t) = V'' ( x) W( t) 
	\]
Dividing both sides gives
\[ 
\frac{W'( t) }{W( t) }= \frac{V''( x) }{V( x) }= \lambda
\]
Hence the heat equation may be rewritten as
\[ 
\begin{cases}
W' = \lambda W\\
V'' = \lambda V\\
V( 0) =V( 1) =0
\end{cases}
\]
Which we can easily solve

\end{enumerate}






\end{document}	
