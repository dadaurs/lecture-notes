\documentclass[../main.tex]{subfiles}
\begin{document}
\lecture{8}{Wed 23 Mar}{Lp spaces}
\section{ $L_p$ spaces}
\begin{defn}[Lp space]
	Let $f:\Omega\to \mathbb{R}\cup \left\{ \pm \infty  \right\} $ and $p \in [ 1, \infty ) $, we define
	\[ 
	\N { f} _{L_p( \Omega) } = \left( \int_{ \Omega }^{  } |f|^{p}\right) ^{\frac{1}{p}}
	\]
	and
	\[ 
	\left\{ f: \Omega\to \mathbb{R}\cup \left\{ \pm \infty  \right\} | \N { f} _{L_p( \Omega) } < \infty  \right\} 
	\]
	
	
\end{defn}
\begin{rmq}
If $p=1$, then $L^{1}( \Omega) $ are absolutely integrable functions.\\
We hope the definition above is a norm, but we need
\[ 
\N { f }  = 0 \iff f=0
\]
so we need to ask that $f=0$ almost everywhere.\\
We wish to identify in $L^{p}$ functions that coincide almost everywhere, so we need to identify as follows
\[ 
	( L^{p}( \Omega) , \N { \cdot} _{L_p} ) = \left\{ f: \Omega\to \mathbb{R}\cup \left\{ \pm \infty  \right\} : \N { f} < \infty  \right\} / \sim
\]
where $f\sim g \iff f=g $ ae.


\end{rmq}
\begin{defn}[L infinity]
	Define
	\[ 
	\N { f}_{L^{ \infty }( \Omega) } = \ess \sup_{x\in \Omega} |f|= \inf \left\{ \alpha: f< \alpha \text{ almost everywhere }  \right\} 
	\]
	
\end{defn}
If $f$ is continuous, the $\sup$ and $\ess\sup$ coincide.\\
Then $ L^{ \infty }( \Omega) $  is then defined as above.
\begin{propo}
Let $\Omega \subset \mathbb{R}^{n}$ be measurable and $1 \leq p \leq q \leq \infty $, then
\begin{itemize}
\item $L^{p}( \Omega) $ is a vector space
\item If $m( \Omega) < \infty $, then $\N { f}_{L_q} \leq K \N { f} _{L_q} \forall f$ where $K$ depends on $m( \Omega) , p$ and $q$.
\item if $m( \Omega) < \infty $, then $\lim_{p \to \infty } \N { f} _{L^{p}} = \N { f} _{L^{ \infty }} $ 
\item Minkowski inequality
	\[ 
	\N { f+g} _{L^p} \leq \N { f} _{L^p} + \N { g} _{L^p} 
	\]
	
\end{itemize}
In particular $\N{\cdot}_{L^p} $ is a norm.

\end{propo}
\begin{thm}[Hoelder inequality]
Let $\Omega$ be measurable, $ p \in [ 1, \infty ] $, then
\[ 
	\N { fg} _{L^{1}} \leq  \N{f}_{L^{p}} \N{g}_{L^{p'}} 
\]
where $p'$ satisfies $\frac{1}{p}+ \frac{1}{p'}= 1$ 
\end{thm}
\begin{proof}
	The inequality holds iff $ \N { \lambda_1 f \lambda_2 g}_{L^{p}} \leq  \N{\lambda_1 f}_{L^{p}} \N { \lambda_2 g}_{L^{p'}}  $ for all $\lambda_1, \lambda_2\in \mathbb{R}$.\\
	So we may reduce ourselves to the case
	\[ 
	\N { f} _{L^{p}} = \N { g} _{L^{p'}} =1
	\]
	Now
	\[ 
	\int_{  }^{  }|fg| \leq  \int \frac{|f|^{p}}{p}+ \frac{|f|^{p'}}{p'} = 1 
	\]
	
	
\end{proof}
\begin{proof}[Of second point above]
\begin{align*}
	\N{F}_{L^{P}} = \N{F^{p}}_{L^{1}} ^{\frac{1}{P}} \leq  ( \int_{  }^{  }F^{P \frac{Q}{P}}) ^{\frac{1}{Q}} ( \int_{  }^{  }1^{p'})^{\frac{1}{P}-\frac{1}{Q} }
\end{align*}

\end{proof}
\begin{proof}[Of fourth point]
\begin{align*}
	\N { f+g} _{L^{p}} ^{p}&=  \int_{  }^{  }|f+g| ^{p} \\
	& \leq  \int_{  }^{  } ( |f|+|g|) |f+g|^{p-1}\\
	&= \int_{  }^{  }|f| |f+g|^{p-1}+ \int_{  }^{  } |g| |f+g|^{p-1}\\
	&\leq ( \int_{  }^{  }|f|^{p}) ^{\frac{1}{p}} ( \int_{  }^{  }|f+g|^{p}) ^{ \frac{p-1}{p}}\\
	&= ( \N { f} + \N { g} ) \N { f+g} ^{p-1}	
\end{align*}

\end{proof}
\subsection{Completeness of $L^p$ }
\begin{thm}[Lp spaces are complete]
	Let $\Omega$ be measurable, $p \in [ 1, \infty ] $, then $L^{p}( \Omega) $ is complete, namely if
	\[ 
	\lim_{m,n \to  + \infty} \N { f_n - f_m} _{L^{p}} = 0
	\]
	then $\exists f \in L^p$ s.t. $ \lim_{n \to  + \infty} \N { f_n-f} _{L^{p}} =0$.\\
	Moreover, if the above holds then $\exists $ a subsequence $ \left\{ m_k \right\} $ s.t.
	\[ 
	f_{m_k} ( x) \to f( x) 
	\]
	Almost everywhere.
	
\end{thm}
\begin{rmq}
Taking the subsequence above is important, see exercises.
\end{rmq}
\begin{proof}
We prove the result for $p < \infty $, the case $p= \infty $ is an exercise.\\
We want to prove that $ \left\{ f_m \right\} $ is cauchy in $L^p$ implies there is a subsequence $f_m \to f$ in $L^p$ pointwise.\\
We look for a speedy converging subsequence.\\
Indeed, we know from hypothesis that there exists a subsequence $ \left\{ m_k \right\} $ st.
\[ 
\N { f_{m_k} - f_{m_{k+1} } } \leq  2^{-k}
\]
Now consider
\[ 
f( x) = f_{m_1} ( x) + \sum_{k}^{ } f_{m_{k+1} } - f_{m_k} ( x) 
\]
This is a reasonable definition, but is it well defined.\\
Namely is the series absolutely converging for almost every $x$?\\
Consider
\[ 
g_h( x) = |f_{m_1} ( x) | + \sum_{k=1}^{ h} |f_{m_{k+1} } ( x) -f_{m_k} ( x) |
\]
Is $\lim_{h \to  + \infty} g_h< \infty $ ae.? If yes, $f$ is well defined.\\
Indeed, 
\[ 
\N { g_j} _{L^{p}} \leq \N { f_{m_1} } _{L^{p}} + \sum \N { f_{m_{k+1} } - f_{m_k} } \leq  \N { f_{m_1} } + 1
\]
But now
\[ 
\int_{  }^{  }|g|^{p}= \lim_{h \to  + \infty} \int |g_h|^{p} < \infty 
\]
Hence $g$ is finite a.e. and
\[ 
f( x) = \lim f_{m_1}( x)  + \sum_{k}^{ } f_{m_{k+1} } -f_m = \lim f_{m_k} ( x) 
\]
And the convergence is dominated by $g $.\\
To prove $L^{p}$ convergence
\[ 
\N { f_{m_k} - f} _{L^{p}} ^{p} = \int_{  }^{  }|f_{m_k} - f|^{p}\to 0
\]


	
\end{proof}




	

	



\end{document}	
