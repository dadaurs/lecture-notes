\documentclass[../main.tex]{subfiles}
\begin{document}
\lecture{10}{Wed 30 Mar}{density of continuous functions}
\subsection{How to approximate a $C^{0}_c$ with $C^{ \infty }_c$ in $L^{p}$? }
We will use convolutions.\\
Let $\phi \in C^{ \infty }_c ( \mathbb{R}^n) $ such that $\phi \geq 0, \phi=0$ outside $B_1$ such that $ \int_{  }^{  } \phi =1$.\\
For instance, we can take $ \phi( x) = c e^{ \frac{1}{|x| -1}} $ if $|x| <1$.\\
The standard convolution kernel is
\[ 
\phi_\epsilon  = \epsilon^{-n} \phi(  \frac{1}{\epsilon} x) \quad \epsilon >0
\]
so that
\[ 
\int_{  }^{  } \phi_\epsilon ( x)  = \int_{  }^{  } \epsilon^{-n}\phi( \epsilon^{-1} x ) = 1
\]
Now, let $f\in C^{0}_c$ and define the convolution of $f$ and $\phi_\epsilon$ as
\[ 
f_\epsilon( x) = f \ast \phi_\epsilon ( x) = \int_{  }^{  } f( x-y) \phi_\epsilon( y) dy
\]
\begin{lemma}
$\forall \epsilon $ smal, we have that
\begin{enumerate}
\item $\supp f_\epsilon \subset \supp F + B_\epsilon$ 
\item $f_\epsilon$ is smooth
\item $\N { f_\epsilon}  \leq  \N { f} $ in $ L^{1}$ 
\item $f_\epsilon\to f$ uniformly.
\end{enumerate}

\end{lemma}
\begin{proof}
\begin{enumerate}
\item $f_\epsilon ( x) = \int_{ B_\epsilon }^{  } \underbrace{f( x-y)}_{ = 0} \phi_\epsilon( y) dy= 0$ 
\item Observe that 
	\begin{align*}
	f_\epsilon( x) = \int_{  }^{  } f( y) \phi_\epsilon( x-y) dy
	\intertext{Now we compute}
	\frac{1}{h} ( f_\epsilon( x+hv) - f_\epsilon( x) = \int_{  }^{  } f( y) \frac{\phi_\epsilon ( x+hv -y) - \phi_\epsilon ( x-y) }{h}) \\
	\end{align*}
	But now note that
	\[ 
	\del_v \phi_\epsilon( x-y) = \frac{\phi_\epsilon ( x+hv - y) - \phi_\epsilon( x-y) }{h}
	\]
	And this is dominated by $ \N { \nabla \phi_\epsilon} $.\\
	Hence the whole integral above is dominated and we get
	\begin{align*}
	= \int_{  }^{  } f( y)  \del_v \phi_\epsilon( x-y) 
	\end{align*}
	Hence $\nabla f_\epsilon = f \ast \nabla \phi_\epsilon$ and we conclude by induction on the degree of the derivative.
\item By definition
	\begin{align*}
		\int_{  }^{  }|f_\epsilon| &\leq  \iint |f( x )| \phi_\epsilon ( x-y) dydx\\
&= \int |f( y) | \underbrace{\int \phi_\epsilon ( x-y) dx}_{ =1 } dy = \N { f} 	
	\end{align*}

\item Since $f$ is uniformly continuous implies that $\forall \epsilon >0 \exists \delta >0$ such that $|x-y| < \delta \implies |f( x) - f( y) | <\epsilon$ 
	\begin{align*}
		|f( x) - f_\epsilon( x) | &= | f( x) - \int_{  }^{  } f( x-y) \phi_\epsilon ( y) dxdy |\\
		&= | \int ( f( x) - f( x-y) ) \phi_\epsilon( y) dxdy|\\
		&= \int_{ B_\epsilon }^{  } |f( x) - f( x-y) | \phi_\epsilon( y) dy \leq  \epsilon
	\end{align*}
\end{enumerate}

\end{proof}
\begin{rmq}
$L^{2}$ has a Hilbert Structure.\\
Define for $f,g\in L^{2}( \mathbb{R}^{n}) $ a scalar product
\begin{align*}
\langle f,g\rangle = \int_{ \Omega }^{  } f( x) \bar { g( x) } dx
\end{align*}
It has a few properties
\begin{itemize}
\item $ \langle f,f\rangle = \int |f|^{2}$ 
\item Hermitian property : $ \langle f,g \rangle = \bar  { \langle g,f \rangle } $ 
\item It is linear in its first component and anti linear in the second one.
\item Pythagoras theorem: if $ \langle f,g\rangle = 0$, then 
	\[ 
	\N { f+g}_{L^{2}} ^{2} = \N { f}_{L^{2}} ^{2} + \N { g} _{L^{2}} ^{2}
	\]
	
\end{itemize}


\end{rmq}
\begin{thm}[Egorov theorem]
	Let $\Omega \subset \mathbb{R}^n$ measurable, $m( \Omega) < \infty $, then, if
	\[ 
	f_k : \Omega \to \mathbb{R} \longrightarrow f \text{ ae. } 
	\]
	Given $\epsilon >0,\exists C_\epsilon$ closed contained in $\Omega$ such that
	\[ 
	m( \Omega\setminus C_\epsilon) < \epsilon
	\]
	and $f_k\to f$ uniformly in $C_\epsilon$ 
\end{thm}
\begin{proof}
Without loss of generality $f_k( x) \to f( x) \forall x\in \Omega$ ( up to throwing away a set of measure 0).\\
$\forall m,k$ define
\[ 
E^{m}_k = \left\{ x\in \Omega : |f_j( x) - f( x) | \leq \frac{1}{m}\forall j \geq K \right\} 
\]
For fixed $m$ , we have that
\[ 
E^{m}_k \subset E^{m}_{k+1} 
\]
and $E^{n}_k \to \Omega$ as $k\to \infty $.\\
Then
\[ 
m( \Omega\setminus E^{m}_k ) \to 0 \text{ as } k\to \infty 
\]
This means that $\forall n$ we can fix $k_n$ 
\[ 
m( \Omega \setminus E_{k_n}^{n}) \leq 2^{-n}
\]
Fix $\epsilon$ as in the statement, there exists $N$ such that 
\[ 
\sum_N^{ \infty } 2^{-n} \leq \frac{\epsilon}{2}
\]
Define $C_\epsilon = \bigcap_{n \geq N} E^{n}_{k_n} $.\\
In $C_\epsilon, f_j \to f$ uniformly, indeed
\[ 
\forall \delta >0 \text{ let  } n \text{ such that } \frac{1}{n}< \delta
\]
\[ 
|f_j( x) - f( x) | \leq \frac{1}{n} < \delta	
\]


\end{proof}
\begin{rmq}
$\forall E$ measurable $\exists C \subset E$ such that $C$ is closed and $m( E\setminus C) \leq \frac{\epsilon}{2}$ 
\end{rmq}
\subsection*{"Littlehood principles"}
\begin{itemize}
\item Every measurable setis nearly a finite union of balls
\item Every pointwise converging sequence of functions is nearly uniformly convergent.
\item Every measurable function is nearly continuous.
\end{itemize}






\end{document}	
