\documentclass[../main.tex]{subfiles}
\begin{document}
\lecture{4}{Thu 24 Mar}{Interpolation de fonctions}
\section{Interpolation de fonctions}
\subsection{Polynomes de Lagrange}
On considere le probleme d'interpolation a l'aide de polynomes.
\begin{thm}[Theoreme de Weierstrass]
	Soit $f\in C^{0}( [ a,b] ) $ alors il existe un polynome $p_n$ de degre  $n$ yrl wur
	\[ 
\lim_{n \to  + \infty} \N { f-p_n} =0		
	\]
	Pour la norme $L^{ \infty }$.
	
\end{thm}
Etant donne $f( x_0) ,\ldots, f( x_n) $, on cherche un polynome de degre $n$ qui approche $f( x) $.
\begin{defn}
	Etant donne une partition de $ [ a,b] $ $ x_0,\ldots, x_n$.\\
	On appelle $ \left\{ l_i( x)  \right\} $ les polynomes de lagrange, les polynomes $l_i( x)$ tels que 
	\[ 
		l_i( x_j) = \delta_{ij} , l_i \in \mathbb{P}_n
	\]
	
\end{defn}
En general, on a
\[ 
l_i( x) = \frac{\prod_{j=0, j\neq 1}^{n}( x-x_j) }{\prod_{j=0, j\neq i} ( x_i-x_j) 	}
\]
Ainsi, on peut considerer
\[ 
p_n( x) = \sum_{i=0}^{ n} f( x_i) l_i( x) 
\]
comme polynome interpolant et on remarque que $p_n( x_j)= f( x_j)  $.\\
On se demande donc maintenant pour $f\in C^{k}( [a,b ] ), k >0$ si on peut borner $ \N { f-p_n} $ par une quantite dependant de $n$.
\begin{propo}
	Etant donne une partition $x_i$.\\
Soit $d_n( x) $ une fonction de classe $C^{n}( [ a,b] ) $ tel que $d_n( x_i) =0 \forall x_i$ de la partition. Alors $\exists \xi \in ( x_0,x_n) $ tel que $d_n^{( n) }( \xi) =0$ 
\end{propo}
\begin{rmq}
Si  $f$ est reguliere, alors $f( x) -p_n( x) $ est reguliere et $f( x_i) -p_n( x_i) =0$ 
\end{rmq}
\begin{proof}
On doit appliquer le theoreme de rolle $n$ fois.\\
En effet, on a $d_n( x_0) = d_n( x_1) =0$ et donc $\exists y_0$ tel que $d'( y_0) =0$ et de maniere generale, on a 
\[ 
d_n( x_i) = d_n( x_{i+1} ) =0 \implies \exists y_i \text{ tel que } d'( y_i) =0
\]
On reapplique le theoreme de rolle a $y_1,\ldots, y_n$ 
\end{proof}
\begin{thm}[Representation de l'erreur]
	Soit $f\in C^{n+1}( [ a,b] ) $ et soit $p_n$ le polynome d'interpolation de $f$ sur la partition $( x_0,\ldots,x_n) $ alors $\forall x \in [ a,b] \exists \xi \in ( a,b) $ :
	\[ 
	f( x) - p_n( x) = f^{( n+1) }( \xi) \cdot \pi_n( x) 
	\]
	ou $\pi_n( x) = \frac{1}{( n+1) !}( x-x_0) \ldots ( x-x_n) $ 
\end{thm}
\begin{proof}
	On va demontrer le resultat pour tut point $x\in [ a,b] $.\\
	Si $x=x_i$, alors $f( x_i) -p_n( x_i) = f^{( n+1) }( \xi) \cdot 0$ ce qui est toujours vrai.\\
	Donc, si $x\neq x_i$ , alors $\pi_n( \overline{x}) \neq 0$.\\
	Donc $\exists \eta\in \mathbb{R}: f( x)-p_n( \overline{x}) = \eta\pi_n( \overline{x})  $.\\
	On peut donc prendre $d_{n+1} ( x) = f( x) -p_n( x) -\eta \pi_n( x) $, alors $d_{n+1} $ s'annule sur les $x_i$ et sur $ \overline{x}$.\\
	On peut donc appliquer la proposition d'avant a $d_{n+1} $,
	\[ 
	\exists \xi: d_{n+1} ^{( n+1) }( \xi) =0
	\]
	Ainsi
	\[ 
d_{n+1} ^{( n+1) }= f^{( n+1) }( x) - 0 - \eta \underbrace{\frac{d^{( n+1) }}{dx^{n+1}} \pi_n}_{=1}
	\]
	Et donc il existe $\xi$ tel que $f^{( n+1) }( \xi) - \eta= 0$ 	
	
\end{proof}
On va  essayer d'utiliser la representation de l'erreur pour trouver une estimation de l'erreur\\
En effet
\[ 
	\N { f( x) - p_n( x) } = \max_{x\in [ a,b] } \N { f^{( n+1) }( \xi) \pi_n( x) } \leq  \N { f^{( n+1) }( x) } \N{\pi_n}
\]
On a 
\[ 
\N { \pi_n} = \N { \frac{1}{( n+1) !}( x-x_0) \ldots ( x-x_{n-1} ) } \leq  \frac{1}{( n+1) !}( b-a)^{n-1}
\]
Ainsi
\[ 
\N { f-p_n} \leq \frac{1}{4} \frac{1}{( n+1) !}( b-a)^{n+1}\N { f^{( n+1) }} 
\]
Pour quelle classe de fonctions puis-je donc deduire que $ \lim_{n \to  + \infty} \N { f-p_n} =0$ ?\\
Clairement $f( x) = \frac{1}{1+x^{2}}$ n'appartient pas a cette classe.
\begin{thm}
	Soit $x_0,\ldots,x_n$ une partition equidistante de l'intervalle $ [ a,b] $ et soit $f: [ -\alpha,\alpha] \to \mathbb{R}$ une fonction analytique.\\
	Si $f$ admet un developpement en serie entiere en $x_0$ de rayong $R$ avec $R > 3\alpha$, alors $ \lim_{n \to  + \infty} \N { f-p_n} =0$ 
\end{thm}
En effet si $R> 3\alpha$ $\exists a\in \mathbb{R}^{+}, a<1$ $ \N { f-p_n} \leq C( R) a^{n+2}$ 	

		


\end{document}	
