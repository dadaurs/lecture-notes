\documentclass[../main.tex]{subfiles}
\begin{document}
\lecture{2}{Thu 10 Mar}{Integration Numerique}
\section{Integration Numerique}
On veut construire des algorithme pour calculer de maniere approchee $ \int_{ a }^{ b }f( x) dx$ 
\subsection{Formules d'integration de Newton-Cotes}
On ecrit
\[ 
\int_{ a }^{ b } f( x) dx = \sum_{i=0}^{ N-1}\int_{ x_i }^{ x_{i+1}  } f( x) dx
\]
Chacun des termes de la somme se reecrit comme
\[ 
\int_{ x_i }^{ x_{i+1}  } f( x) dx = \int_{ 0 }^{ 1 } f( x_i + th_i) h_i dt
\]
Et on trouve
\[ 
\int_{ a }^{ b }f( x) dx = \sum_{i=1}^{ N-1}h_i \int_{ 0 }^{ 1 } f( x_i + th_i) dt
\]
Ainsi, il suffit de trouver un algorithme pour calculer des integrales de la forme $ \int_{ 0 }^{ 1 }g( t) dt$.
La maniere la plus naive pour approximer cette integrale serait de prendre $ \int_{ 0 }^{ 1 } g( t) dt \approx g( \frac{1}{2}) $, et on note $Q_1^{nc}( g) = g( \frac{1}{2}) $.\\
Une maniere moins naive de faire est d'approcher $g$ par une fonction lineaire et de prendre l'approximation
\[ 
\int_{ 0 }^{ 1 }g( t) dt \approx \frac{1}{2} \left( g( 0) +g( 1) \right)= Q_2^{nc}( g) 	 ( \text{ formule de Newton-Cote a deux noeuds } ) 
\]
ou encore
\[ 
\int_{ 0 }^{ 1 } \approx \frac{1}{6}( g( 0) + 4 g( \frac{1}{2}) + g( 1) ) = Q_3^{nc}( g) ( \text{ formule de cote a trois noeuds ou formule de Simpson } ) 
\]
De maniere generale, on appelle formule de Newton-Cotes a $S$ noeuds
\[ 
\int_{ 0 }^{ 1 }g( t) dt \approx \int_{ 0 }^{ 1 }p( t) dt
\]
ou $p( t) $ est le polynome de degre $s-1$ passant par les points $ ( c_i, g( c_i) ) $, ou $ 0 \leq c_1 \leq \ldots \leq c_{s-1} < c_{s} \leq 1$.\\
Ainsi, de maniere generale
\[ 
Q_S^{nc}( g) = \sum_{i=1}^{ s} b_i g( c_i) 
\]
ou $b_i$ sont les poids des formules de N.C.\\
On veut donc essayer de trouver des formules qui donnennt les poids de l'integration de Newton-Cotes.
\begin{defn}[Formule de Quadrature]
	Une formule de quadrature $Q_s( f) $ est donnee par n'importe quelle ensemble de couples $ ( \left\{ b_i \right\}_{i=1}^{s}, \left\{ c_i \right\}_{i=1}^{s}) $ :
	\[ 
	Q_s( f) = \sum_{i=1}^{ N}b_i f( c_i) 
	\]
	
\end{defn}
\begin{defn}
	$Q_s( \cdot) $ 	est d'ordre $s$ quand elle est exacte sur tout polynomme de degre $ \leq s-1$
\end{defn}
\begin{rmq}
Par definition les formules $Q_s^{nc}$ sont d'ordre $s$.
\end{rmq}
\begin{thm}
	Etant donne $s$ noeuds distincts $ \left\{ c_i \right\}_{i=1}^{N}$, la formule donnee par $ \left( \left\{ b_i \right\}, \left\{ c_i \right\} \right) $ est d'ordre $s$ si et seulement si les poids verifient
	\[ 
	\sum_{i=1}^{ s} c_i^{q-1} b_i = \frac{1}{q} \quad \forall q = 1, \ldots, s
	\]
\end{thm}
\begin{proof}
Supposons que $Q$ est d'ordre $s$, alors prenons
\[ 
p( t) = t^{q} \quad q= 1\ldots s 
\]
On ecrit
\[ 
\int_{ 0 }^{ 1 }p( t) dt = \int_{ 0 }^{ 1 }t^{q-1}dt = \frac{1}{q}
\]
d'autre part
\[ 
\sum_{i=1}^{ s}b_i p( c_i)  = \sum_{i=1}^{ s}b_i p( c_i) = \sum_{i=1}^{ s}b_i c_i^{q-1}
\]
Dans l'autre sens, si $ \sum_{i=1}^{ s}c_i^{q-1}b_i = \frac{1}{q}$, alors la formule est exacte sur tout monome ( par le raisonnement ci-dessus), par linearite, elle sera donc exacte sur n'importe quel polynome.
\end{proof}
On montre maintenant qu'enfait les poids $ b_i$ sont uniques etant donne les $ c_i$, en effet, etant donne le theoreme ci-dessus, on a
\[ 
\begin{pmatrix}
	1 & 1 &1 &\ldots & 1\\
	c_1 & c_2 & c_3 & \ldots & c_s\\
	c_1^{2} & c_2^{2} & c_3^{2} & \ldots & c_s^{2}\\
	\vdots & \vdots & \vdots & \ddots & \vdots\\
	c_1^{s-1 }& c_2^{s-1}& c_3^{s-1}&\ldots & c_s^{s-1}
\end{pmatrix} 
\cdot
\begin{pmatrix}
b_1\\
\vdots \\
\vdots\\
\vdots\\
b_s\\
\end{pmatrix} 
=
\begin{pmatrix}
1\\
\frac{1}{2}\\
\frac{1}{3}\\
\vdots\\
\frac{1}{s}
\end{pmatrix} 
\]
Ainsi, soit la matrice $A$ ci-dessus est inversible, alors il y a un seul choix de poids pour la formule de N.C.\\
Par un theoreme d'algebre lineaire, la matrice est inversible
En appliquant donc ceci a une fonction $f$ generale, on trouve
\[ 
\int_{ a }^{ b }f( x) dx = \sum_{j=0}^{N-1} \int_{ x_j }^{ x_{j+1}  }f( x) dx = \sum_{j=0}^{ N-1}h_j \int_{ 0 }^{ 1 } f( x_j + th_j) dt
\]
\[ 
= \sum_{j=0}^{ N-1}h_j  Q_s^{nc} ( f( x_j + th_j) ) = \sum_{j=0}^{ N-1}h_j \sum_{i=1}^{ s}b_i f( x_j + c_i h_j) 
\]
\begin{rmq}
Pour les noeuds $c_i$ fixes, il existe un seul choix de poids qui garantit que $Q_s$ est d'ordre $s$.
\end{rmq}
\subsection*{Quel est le choix optimal des noeuds?}
\begin{itemize}
\item \textbf { Choix 1} Choisir des noeuds equidistants.\\
	Ce choix rend le calcul instable en arithmetique finie.\\
	En effet, supposons qu'on veut integrer $f( x) >0$, on aura $ \sum_{i=1}^{ s}f( ih) b_i$.\\
	Alors les poids oscillent fortement.
\item \textbf { Choix 2} On cherche a comprendre ou placer les noeuds pour maximiser l'ordre de la formule.
	\begin{exemple}
	On considere a nouveau la formule de Simpson
	\[ 
	Q_3^{nc}( g) = \frac{1}{6} \left[ g( 0) + 4 g( \frac{1}{2}) + g( 1) \right] 
	\]
	Ainsi, pour $c_i = 0,\frac{1}{2}, 1$ on a les poids $ b_i = \frac{1}{6}, \frac{2}{3}, \frac{1}{6}$ 
	Est-ce que cette formule est d'ordre $4$?
	\[ 
\int_{ 0 }^{ 1 }t^{3}dt = \frac{1}{4}= \sum_{i}^{ }b_i c_i^{3} = \frac{1}{4} ( \text{ en substituant les valeurs } ) 	
	\]
	Est-elle aussi d'ordre 5?
	\[ 
	\int_{ 0 }^{ 1 } t^{4}dt = \frac{1}{5}= \sum_{i}^{ } b_i c_i^{4}= \frac{2}{3}\frac{1}{16}+ \frac{1}{6}\neq \frac{1}{5}
	\]
		
	\end{exemple}
\end{itemize}
\subsection{Formules de quadrature d'ordre optimal}
On veut donc choisir des noeuds $c_1,\ldots, c_s$ pour maximiser l'ordre de la formule de quadrature

\begin{thm}[Thm. fondamental de la theorie de l'integration]
	Soit $ ( \left\{ b_i \right\} , \left\{ c_i \right\} ) $ une formule de quadrature d'ordre $s, Q_s( \cdot)  $.\\
	Soit $M( t) = ( t-c_1) ( t-c_2) \ldots ( t-c_s) $, alors la formule $Q_s( \cdot) $ est d'ordre $p \geq s+m$ si et seulement si
	\[ 
	\int_{ 0 }^{ 1 }M( t) g( t) =0
	\]
	
\end{thm}
\begin{proof}
Soit $f( t) $ un polynome de degre $s+m-1$, prenons $r( t) $ un polynome de degre $s-1$ passant par les points $ ( c_i, f( c_i) ) $.\\
Alors $f( t) -r( t) $ est un polynome de degre $s+m-1$ est un polynome s'annullant sur tous les noeuds.\\
Ainsi
\[ 
f( t) -r( t) = M( t) g_f( t) \text{ avec } \deg g_f \leq m-1
\]
$\Leftarrow$\\
Supposons que $ \int_{ 0 }^{ 1 }M( t) g( t) dt =0$ $\forall $ polynome $g( t) : \deg g \leq m-1$.\\
On demontre que la formule est d'ordre $s+m-1$.\\
Soit $f$ un polynome $\deg f \leq s+m-1$, on peut donc ecrire
\[ 
f( t) = r( t) + \underbrace{ \int_{ 0 }^{ 1 }M( t) g_f( t) dt}_{=0}
\]
De meme, on a que 
\[ 
Q_s( f) = \sum_{i=1}^{ s}b_i f( c_i) = \sum_{i=1}^{ s} b_i \left[ r( c_i) +  \underbrace{M( c_i) g_f( c_i)}_{=0	} \right] = \int_{ 0 }^{ 1 }r( t) dt	
\]
Et donc la formule est exacte\\
$\Rightarrow$ \\
Supposons que la formule est d'ordre $s+m$, demontrons que $ \int_{ 0 }^{ 1 }M( t) g( t) dt = 0 \forall g, \deg g \leq m-1$, ainsi
\[ 
\int_{ 0 }^{ 1 }M( t) g( t) dt = \sum_{i=1}^{ s}b_i M( c_i) g( c_i)  =0
\]



\end{proof}

	







\end{document}	
