\documentclass[../main.tex]{subfiles}
\begin{document}
\lecture{3}{Thu 17 Mar}{Integration Numerique}
\begin{lemma}
Si une formule a $s$ noeuds est d'ordre $p$, alors $p \leq 2s$ 
\end{lemma}
\begin{proof}
Supposons que $p = 2s+1$, si $Q_s$ est d'ordre $2s+1$, par le theoreme fondamental, ceci implique que
\[ 
\int_{ 0 }^{ 1 } M( t) g( t) =0 \forall g( t) : \deg g \leq s
\]
Ainsi, en particulier pour $g( t) = M( t) $ on a 
\[ 
\int_{ 0 }^{ 1 }M( t) ^{2} dt = 0 
\]
et donc $M( t)=0 $ 

\end{proof}
On se demande maintenant si on peut trouver la valeur des noeuds de maniere facile?
\subsection{Noeuds d'integration optimaux: Formule de Gauss}
\begin{defn}[Polynomes de Legendre]
	On considere la suite de polynomes $ \left\{ p_k \right\}_{k =0, \ldots, n} $, avec $\deg p_k = k$ et $ \int_{ -1 }^{ 1 } p_k( x) \cdot g( x) = 0 \forall g( x) \deg g \leq  k-1$ 
\end{defn}
\begin{thm}[Forme des polynomes de Legendre]
	Les polynomes de Legendre ont la forme
	\[ 
		p_k( x) = \frac{1}{2^{k}k!} \frac{d^{k}}{dx^{k}}\left[( x^{2}-1)^{k}\right]
	\]
	
\end{thm}
\begin{proof}
	On veut montrer que
\begin{align*}
\int_{ -1 }^{ 1 }p_k ( x) g( x) dx = 0 \forall g \quad \deg g \leq k-1
\end{align*}
\begin{align*}
\int_{ -1 }^{ 1 } \frac{d^{k}}{dx^{k}} [ ( x^{2}-1) ^{k}] g( x) dx = - \int_{ -1 }^{ 1 } \frac{d^{k}}{dx^{k-1}} \left[ ( x^{2}-1)^{k} \right] \frac{d}{dx} g( x) dx \left[ \frac{d^{k-1}}{dx^{k-1}} [ ( x^{2}-1)^{k}]\cdot g \right]_{-1}^{1}\\
= ( -1) ^{k}\int_{ -1 }^{ 1 } ( x^{2}-1) ^{k} \underbrace{\frac{d^{k}}{dx^{k}}g( x)}_{=0} 
\end{align*}


\end{proof}
\begin{thm}
	Toutes les racines de $P_k$ sont reelles, distinctes et dans l'intervalle $( -1,1) $.
\end{thm}
\begin{proof}
Par l'absurde supposons qu'il y a $\tau_1,\ldots, \tau_r$ racines distinctes de $p_k( x) $ dans l'intervalle $( -1,1) , r <k$.
Ainsi $g( x) = ( x-\tau_1) \ldots ( x- \tau_r) \quad \deg g \leq k-1$ 
Par hypothese, on a donc
\[ 
\int_{ -1 }^{ 1 } p_k ( x) g( x) = \int_{ -1 }^{ 1 } q g^{2} 
\]
Or $q$ ne change pas de signe, donc l'integrale ne peut pas etre nulle.
\end{proof}
\begin{lemma}
Les polynomes de Legendre se calculent par 
\[ 
	( k+1) P_{k+1} = ( 2k+1) x P_k - k P_{k-1} 
\]

\end{lemma}
On cherchait $c_1,\ldots, c_s$ tel que $\deg M = s$ et tel que
\[ 
\int_{ 0 }^{ 1 }M( t) g( t) =0 \forall g: \deg g \leq s-1
\]
Choisissons donc
\[ 
M( t) = P_s( 2t-1) 
\]
En effet
\[ 
\int_{ 0 }^{ 1 } M( t) g( t) dt = \int_{ -1 }^{ 1 }P_s( x) g( \frac{x}{2}+1) \frac{1}{2}dx
\]
On a que $P_s( 2t-1) $ a aussi $s$ racines distinctes dans l'intervalle $( 0,1) $.\\
Ces racines sont les deux d'integration optimaux.
\begin{defn}
	La formule de quadrature $ ( \left\{ b_i \right\} , \left\{ c_i \right\} ) $ avec $c_i$ choisis comme racines de $P_s( 2t-1) $ et $b_i$ les poids corresponndants s'appelle formule de quadrature de Gauss.
\end{defn}
\subsection{Etude d'erreur des formules de quadrature}
\begin{thm}[Erreurs dans les formules de quadrature]
	Soit $f \in C^{r}( [ a,b] ) , r \geq p$ .\\
	Soit $Q_s( \cdot) $ une formule de quadrature d'ordre $p$.\\
	\[ 
	I_n( f) = \sum_{j=0}^{ n-1}h_j \sum_{i=1}^{ s}b_i f( x_j + c_i h_j)
	\]
On a alors que
\[ 
| \int_{ a }^{ b } f( x) dx - I_n( f) | \leq  C \frac{h^{p}}{p!} \max_{x\in [ a,b] } |f^{( p) }( x) |
\]
ou $h= \max_j h_j$ et $C$ ne depend ni de $f$, ni de $p$ ni de $h$, mais depend de 
\[ 
\frac{\max h_i}{\min h_i}
\]
\end{thm}
\begin{proof}
Dans cette demonstration, $C$ indiquera une constante generique qui ne depend pas de $ h,f, p$.\\
On definit
\begin{align*}
	E_n( f) &= | \sum_{j=0}^{ h-1} \int_{ 0 }^{ 1 } f( x_j + h_j t ) dt - \sum_{i=1}^{ s}b_i f( x_j + h_j C_i) |
	\intertext{Posons $g( t) = f( x_j + h_j t) $ et}
E_h^{j} ( f) = |\int_{ 0 }^{ 1 } g( t) dt - \sum_{i=1}^{ s}b_i g( c_i)|  ( = E( g) ) \\
\intertext{Supposons d'abord que $g( x) $ est une fonction entiere, alors}
\sum_{r \geq 0}^{ } \frac{g^{( r) }( 0) }{r!}t^{r}\\
g( t) = \sum_{r=0}^{ p-1} \frac{g^{( r) }( 0) }{r!} t^{r} + \sum_{r \geq p}^{ } \frac{g^{( r) }( 0) }{r!}t^{r}
\intertext{La formule de quadrature est exacte sur la premiere partie, ainsi}
E( g) = | \int_{ 0 }^{ 1 } \sum_{r \geq p}^{ } \frac{g^{( r) }( 0) }{r!} t^{r} - \sum_{r \geq p}^{ } \sum_{i=1}^{ s} b_i \frac{g^{( r) }( 0) }{r!} c_i^{r}|\\
= | \sum_{r \geq p}^{ } \frac{g^{( r) }( 0) }{r!} \underbrace{ [ \frac{1}{r+1} - \sum_{i=1}^{ s}b_i c_i^{r}]}_{=C_r} |\\
= c_p  \frac{g^{( p) }( 0) }{p!} + \text{ "reste" } \\
\intertext{On a }
g^{p}( 0)  = ( f( x_j + th_j)^{( p) }) |_{t=0} = h_j ^{p}\cdot f^{( p) } ( x_j) \\
\intertext{On peut aussi montrer que}
c_p = | \frac{1}{p+1}- \sum_{}^{ } b_i c_i^{p}| \leq 2
\end{align*}
Ainsi $E_n^{j}( f) \leq  2 \frac{1}{p!} h_j^{p} |f^{( p) }( x_j) |$ 
\end{proof}

		
\end{document}	
