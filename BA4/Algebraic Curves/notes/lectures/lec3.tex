\documentclass[../main.tex]{subfiles}
\begin{document}
\lecture{3}{Fri 11 Mar}{Irreducible sets}
\subsection{Irreducible sets}
\begin{defn}[Irreducible set]
	An algebraic set $V \subset \mathbb{A}^{n}$ is irreducible if $\forall W_1,W_2\subset \mathbb{A}^{n}$ algebraic s.t. $ V = W_1\cup W_2$, then either $W_1=V$ or $W_2=V$ 	 
\end{defn}
\begin{exemple}
	\begin{itemize}
	\item Let $V = \left\{ x_1,\ldots,x_n \right\} \subset \mathbb{A}^{n}$ is irreducible iff $n=1$  
	\item Let $f( X,Y) = Y( X^{2}-Y) , V=V( f) \subset \mathbb{A}^{2}$ is not irreducible by taking $W_1= V( Y) ,W_2= V( X^{2}-Y) $ 	
	\end{itemize}
\end{exemple}
\begin{propo}
An algebraic set $V$ is irreducible iff $I( V) $ is prime.
\end{propo}
\begin{proof}
If $I( V) $ is not prime, let $F_1,F_2\notin I( V) $ s.t. $F_1,F_2\in I( V) $, then we can write $V= ( V\cap V( F_1) ) \cup ( V\cap V( F_2) ) $.\\
Conversely, if $V= W_1\cup W_2$ and $W_i\neq V$, then $I( W_i) \supsetneq I( V)$, pick $F_i\in I( W_i) \setminus I( V) $, then $F_1F_2= I( W_1) \cap I( W_2) = I( V) $.
\end{proof}
If $V \subset \mathbb{A}^{n}$ is irreducible, we can decompose it into a union of irreducible sets.\\
The union is always finite as the polynomial ring is noetherian.
\begin{thm}[Theorem name]
	Every $V \subset \mathbb{A}^{n}$ algebraic can be written uniquely ( up to ordering) as a union of irreducible sets.
	\[ 
	V= V_1\cup\ldots\cup V_k
	\]
	where the $V_i$'s are irreducible and $V_i \not \subset V_j\forall i\neq j$ 
			
\end{thm}
\begin{defn}[Irreducible Components]
	The $V_1\ldots V_k$ are irreducible components of $V$.
\end{defn}
\begin{rmq}
Applying $I$ in theorem 1.9, we get
\[ 
I( V) = I( V_1) \cap\ldots \cap I( V_k) 
\]
and $I( V_i) $ is the primary decomposition of $I( V) $ 
\end{rmq}
In general, it is quite difficult to find this decomposition.\\
For hypersurfaces, it's easy, for $I( F)$, write $F= F_1^{\alpha_1}\cdot\ldots F_k^{\alpha_k}$, then $V( F) = V( F_1) \cup\ldots\cup V( F_k) $.\\
\subsection{Algebraic subsets of $ \mathbb{A}^{2}$ }
\begin{lemma}
	Let $F,G\in k[X,Y]$ with no common factors, then $V( F) \cap V( G) $ is a finite set of points.
\end{lemma}
\begin{proof}
By Gauss's lemma, $F,G$ have no common factors in $k( X) [ Y] $. Since $k( x) [ Y] $ is a PID $\exists A,B\in k( X) $ such that 
\[ 
AF +BG=1
\]
Now there exists $C\in k[X]$ such that $AC,BC\in k[X]$ .\\
Let $( x,y) \in V( F,G) $, then $C( x) =0$ and hence there are only finitely many $x$'s possible.\\
By symmetry, the same is true for the $Y$ coordinate, hence $ |V( F,G) | < \infty $ 
\end{proof}
Using this,  we can now classify all algebraic subsets of $ \mathbb{A}^{2}$.
\begin{crly}
The irreducible algebraic subsets of $ \mathbb{A}^{2}$ are $ \mathbb{A}^{2}, V( F) $ with $F$ irreducible or singletons.
\end{crly}
\section{Affine algebraic varieties}
\begin{defn}[Affine algebraic variety]
	An affine algebraic variety is an irreducible affine algebraic set.
\end{defn}
\subsection{Zariski topology}
\begin{defn}[Zariski topology]
The Zariksi-topology on $ \mathbb{A}^{n}$ is the topology whose open sets are complements of algebraic sets.	
\end{defn}
\begin{lemma}
This indeed defines a topology on $ \mathbb{A}^{n}$ 
\end{lemma}
\begin{proof}
Certainly $\emptyset, \mathbb{A}^{n}$ are algebraic, hence their complements are open.\\
Let $ \left\{ U_i \right\} $ be a family of open sets, ie. such that
\[ 
U_i = \mathbb{A}^{2} \setminus V( I) 
\]
Then
\[ 
\bigcup U_i = \bigcup \mathbb{A}^{n}\setminus V( I_i) = \mathbb{A}^{n}\setminus \bigcap_i V( I_i) = \mathbb{A}^{n}\setminus V( \bigcup I) 
\]
Similarly, if $U_1,U_2$ are open, then
\[ 
U_1\cap U_2 = \mathbb{A}^{n} \setminus I( V_1V_2) 	
\]
is again open.

\end{proof}
\begin{exemple}
If $n=1$, then algebraically closed sets are either $ \mathbb{A}^{n}, \emptyset$ are finite union of points so the Zariski topology is the cofinite topology. Hence the open sets are huge.	
\end{exemple}
\begin{defn}
	For $V \subset \mathbb{A}^{n}$ an algebraic variety or set, the Zariski topology on $V$ is just the subspace topology.
\end{defn}
\begin{defn}[New definition of irreducibility]
	A non-empty subset $V$ of a topological space $X$ is irreducible if it cannot be expressed as $V= W_1\cup W_2$ where $W_1,W_2 \subsetneq V$ are closed subsets.	
\end{defn}
\begin{lemma}
A non-empty open subset of an irreducible topological space is again irreducible and dense.\\
Furthermore, if $ V \subset X$ is irreducible, then so is $ \overline{V}$ 
\end{lemma}
The proof is an exercise.
\begin{defn}[Quasi-affine algebraic variety]
	A quasi-affine variety is an open subset of an affine variety.	
\end{defn}
\begin{rmq}
By the lemma above, quasi-affine variety are also irreducible.
\end{rmq}
\subsection{Regular functions and coordinate rings}
Regular functions are the natural "continuous" functions on algebraic varieties.
\subsubsection{Affine case}
\begin{defn}
	Let $V \subset \mathbb{A}^{n}$ be an affine algebraic variety.\\
	A map
	\[ 
	f:V\to K= \mathbb{A}^{1}
	\]
	is regular if $\exists F\in k[X_1,\ldots,X_n]$ such that 
	\[ 
	f( X) =F( X) \forall X\in V
	\]
	The set $\Gamma( V) $ of regular functions on V	is a ring with the usual pointwise multiplication and addition. and is called the coordinate ring of $V$.
	
\end{defn}
\begin{lemma}
If $I=I( V) $ for some prime, then
\[ 
	\Gamma( V) \simeq \faktor { k[X_1,\ldots,X_n]} { I( V) } 
\]
In particular, $\Gamma( V) $ is a domain.
\end{lemma}
\begin{proof}
By definition, we have a surjective morphism
\[ 
	k[X_1,\ldots,X_n] \to \Gamma( V) 
\]
Now note that $F\in \ker\phi \iff F( X) =0\forall x \in V \iff F\in I( V) $ 
\end{proof}
\begin{defn}[Subobjects]
	An affine subvariety of $V$ is an affine variety contained in $V$.
\end{defn}
\begin{lemma}
There is a one-to-one correspondence between $V$ and $\Gamma( V) $ where 
\[ 
\left\{ \text{ algebraic subsets of } V \right\} \leftrightarrow \left\{  \text{ radical ideals of } \Gamma( V)  \right\} 
\]
\[ 
\left\{ \text{ algebraic subvarieties of } V \right\} \leftrightarrow \left\{  \text{ prime ideals of } \Gamma( V)  \right\} 
\]
\[ 
\left\{ \text{ points of } V \right\} \leftrightarrow \left\{  \text{ maximal ideals of } \Gamma( V)  \right\} 
\]

\end{lemma}
The proof is again an exercise.
\begin{defn}[Morphism]
	A morphism $\phi:V\to W$ between affine algebraic varieties $V \subset \mathbb{A}^{n}, W \subset \mathbb{A}^{m}$ is a map such that $\exists$  polynomials $T_1,\ldots,T_m\in k[X_1,\ldots,X_n]$ such that
	\[ 
	\phi( X) = ( T_1( X) ,\ldots,T_m( X) ) 
	\]
	Then $\phi$ is an isomorphism if there exists a morphism $\psi$ such that $\phi\circ\psi= \id$ and $\psi\circ\phi=\id$.
		
\end{defn}
\begin{exemple}
Take $V( X^{2}-Y) \subset \mathbb{A}^{2}$ the the projection $p: V( X^{2}-Y) \to \mathbb{A}^{1}$ on the first coordinate is an isomorphism with inverse $\psi( X) = ( X,X^{2}) $.\\
A non-example of a bijective map which is not an isomorphism:\\
$\phi: \mathbb{A}^{1}\to V( Y^{2}-X^{3}) $, $\phi( t ) = ( t^{2},t^{3}) $.\\
One can check that $\phi$ is bijective but not an isomorphism.
\end{exemple}


		






\end{document}	
