\message{ !name(../main.tex)}
\message{ !name(lectures/lec1.tex) !offset(-1) }

\lecture{1}{Fri 25 Feb}{Introduction}
Let $K$ be a field, given a set of polynomials $S= \left\{ f_1,\ldots \right\} $, we can consider $V( S) = \left\{ ( x_1,\ldots) \in K^{n}| f_i( x_1,\ldots) =0\forall i \right\} $.\\
Notice that if $a_1,\ldots\in K[x_1,\ldots]$ then also $\sum_i a_i ( x) f_i( x)=0 $ only depends on the ideal generated by $S$.\\
If $I( S) $ happens to be prime, we call $V$ an algebraic variety.
\section{Affine algebraic sets}
\subsection{Recollection on commutative algebra}
All rings are commutative and with unit.\\
Let $R$ be a ring.\\
\begin{itemize}
\item $R$ is an integral domain, or just domain if there are no zero divisors, ie, $\forall a,b\in R$ s.t.
	\[ 
	a.b=0\implies a=0 \text{ or } b=0
	\]

\item Any domain can be embedded into it's quotient ring.\\

\item A proper ideal $I$ is maximal if it's not contained in any other proper ideal
\item A proper ideal $I$ is prime if
	\[ 
	\forall a,b\in R, ab\in I\implies a\in I \text{ or } b \in I
	\]
	
\item A proper ideal $I$ is radiccal if
	\[ 
	a^{n}\in I\implies a \in I
	\]

\item For any ideal $I \subset R$, the radical $ \sqrt{I} $ is the smallest radical ideal containing $I$ 
\item 
	\begin{lemma}
	$I \subset R$ is maximal $\iff$ $R /I$ is a field
	\end{lemma}

\item 
	\begin{lemma}
	$I \subset R$ is prime $\iff$  $R /I$ is a domain
	\end{lemma}

\item 
	\begin{lemma}
	radical $\iff R /I$ has no nilpotent elements.
	\end{lemma}
	

\end{itemize}
Given a subset $S \subset R$ we can consider the ideal generated by $S$ 
\[ 
I( S) = \left\{ \sum_i a_i s_i \right\} 
\]
$I$ is finitely generated if $I=I( S) $ with $S$ finite.
\begin{itemize}
\item We say that $R$ is Noetherian $\not\exists$ a chain of strictly increasing ideals.
	Equivalently, every ideal is finitely generated.
\item 
	\begin{thm}
		In fact, hilbert's  basis theorem says that, if $R$ is Noetherian, then $R[x]$ is noetherian.
	\end{thm}
	
	In particular $K[x_1,\ldots,x_n]$ is Noetherian
\item $I$ is in principal if it is generated by one element.
\item A domain is called a principal ideal domain ( PID) if every ideal is principal.

\item $a\in R$ is irreducible if $a$ is not a unit, nor zero and if
	\[ 
	a=b.c
	\]
then either $b$ or $c$ are units.
\item A pid $ ( a) \subset R$ is prime $\iff$ $a$ is irreducible.
\item $R$ is a UFD if $R$ is a domain and elements in $R$ can be factored uniquely up to units and reordering into irreducible elements.
	\begin{thm}
		$R$ is a UFD $\implies R[x]$ is a UFD	
	\end{thm}
	And, if $R$ is a PID, then $R$ is a UFD
\item 
	\begin{thm}[Gauss Lemmma]
		$R$ is a UFD and $a\in R[X]$ irreducible, then also $a\in Q( R) [ X] $ is irreducible.
	\end{thm}

\item Localization\\
	Let $R$ be a domain, if $S \subset R$ is a multiplicative subset, then the localization of $R$ at $S$ is defined as
	\[ 
	S^{-1}R= \left\{ x\in Q( R) | x= \frac{a}{b},b\in S \right\} 
	\]
	If $M$ is an $R-$module, we have similarly
	\[ 
	S^{-1}M= \left\{ \frac{m}{s} | m\in M,s\in M \right\} / \left\{ \frac{m}{s}= \frac{m'}{s'}\iff ms'=sm' \right\} 
	\]
	If $ p\subset R$ is a prime ideal, then it's complement is a multiplicative subset and we define
	\[ 
	R_{p} = ( R\setminus p)^{-1}R
	\]

\item There is a 1-1 correspondence between $p \subset R$ prime and ideals of $R_p$, furthermore $R_p$ is a local ring

\item Localization is exact, in particular, given $I \subset p$ the short exact sequence
	\[ 
	o\to I \to R \to R /I \to 0
	\]
	gets sent to 
        
\begin{align}
\label{eq:1}
	0\to I_p\to R_p \to ( R /I)_p\to 0
\end{align}
	ie. localization commutes with taking quotients.
	
\end{itemize}
\subsection{Polynomial rings}
For $a\in \mathbb{N}^{n}$, we set
\[ 
	X^{a}= X_1^{a_1}\ldots\in k[X_1,\ldots]
\]
Thus for any $F\in k[X_1,\ldots,X_n]$, we can write it as 
\[ 
F= \sum_{a\in \mathbb{N}^{n}}^{ } \lambda_a X^{a}
\]
$F$ is homogeneous or a form of degree $d$ if the coefficients $\lambda_a=0$ unless $a_1+\ldots+a_n=d$.\\
Any $F$ can be written uniquely as $F=F_0+\ldots + F_d$ where $F_i$ is a form of degree $i$.\\
The derivative of $F= \sum_{a\in \mathbb{N}^{n}} \lambda_a X^{a}$ with repsect to $X_i$ is $ F_{X_i} = \frac{\del F}{\del X_i}$.\\
If $F$ is a form of degree $d$ we have
\begin{thm}[Euler's theorem]
	\[ 
	\sum_{i=1}^{ n} \frac{\del F}{\del X_i} X_i = d F
	\]
	
\end{thm}


\end{document}	

\message{ !name(../main.tex) !offset(-134) }
