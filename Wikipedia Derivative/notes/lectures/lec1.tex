\documentclass[../main.tex]{subfiles}
\begin{document}
\lecture{1}{Wed 09 Sep}{The derivative as a function}
Sometimes $f$ has a derivative at most, but not all, points of its domain.
The function whose value at a equals f'(a) whenever f'(a) is defined and elsewhere is undefined is also called the derivative of $f$. It is still a function, but it's domain is strictly smaller that the domain of $f$.\\

Using this idea, differentiation becomes a function of functions: The derivative is an operator whose domain is the set of all functions that have derivatives at every point of their domain and whose range is a set of functions. If we denote this operator by $\mathcal{D}$

\end{document}	
