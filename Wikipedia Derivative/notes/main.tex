\documentclass[a4paper,hidelinks]{article}
\usepackage[utf8]{inputenc}
\usepackage[T1]{fontenc}
\usepackage{textcomp}
\usepackage{hyperref}
\usepackage{amsmath}
\usepackage{bookmark}
\usepackage{float}
\usepackage{graphicx}
\usepackage{mdframed}
\usepackage[french]{babel}
\usepackage{bbm} %RCQ
\usepackage{verbatim}
\usepackage{varioref}
\usepackage{theoremref}
\usepackage{tikz}
\usepackage{listings}
\usepackage{faktor}
\usepackage{python}
\usepackage{xurl}
\usepackage{xcolor}
\usepackage[amsmath,hyperref,thmmarks]{ntheorem}
\usepackage[fixed]{fontawesome5}
\usepackage{imakeidx}
\makeindex
% figure support
\usepackage{import}
\usepackage{xifthen}
\usepackage{pdfpages}
\usepackage{transparent}
\usepackage{thmtools}
\usepackage{amssymb}
\usepackage{aligned-overset}
%\usepackage{fancyhdr}
\usepackage{stmaryrd} % for \lightning
%\pagestyle{fancy}
\definecolor{background}{HTML}{FFFFFF}
\definecolor{myred}{HTML}{A54242}
\definecolor{mygreen}{HTML}{8C9440}
\definecolor{myyellow}{HTML}{F0C674}
\definecolor{myblue}{HTML}{5f819d}
\definecolor{mymagenta}{HTML}{85678f}
\definecolor{mycyan}{HTML}{5e8d87}
\definecolor{mygray}{HTML}{373B41}
\renewcommand{\vec}[1]{\overrightarrow{#1}}
\newcommand{\del}{\partial}
\DeclareMathOperator*{\sgn}{sgn}
%\DeclareMathOperator*{\om}{Hom}
%\DeclareMathOperator*{\end}{End}
\DeclareMathOperator*{\id}{Id}
\DeclareMathOperator*{\im}{Im}
\DeclareMathOperator*{\re}{Re}

\setlength\theorempreskipamount{2ex}
\setlength\theorempostskipamount{3ex}
\allowdisplaybreaks

	
\newcommand\norm[1]{\left\vert#1\right\vert}
\newcommand\Norm[1]{\left\lVert#1\right\rVert}
\newcommand\abs[1]{\left\vert#1\right\vert}
\newcommand\inj{\hookrightarrow}
\newcommand\surj{\twoheadrightarrow}

\lstdefinestyle{mystyle}{
    backgroundcolor=\color{background},   
    commentstyle=\color{mygreen},
    keywordstyle=\color{mymagenta},
    numberstyle=\tiny\color{mygray},
    stringstyle=\color{mycyan},
    basicstyle=\ttfamily\footnotesize,
    breakatwhitespace=false,         
    breaklines=true,                 
    captionpos=b,                    
    keepspaces=true,                 
    numbers=left,                    
    numbersep=5pt,                  
    showspaces=false,                
    showstringspaces=false,
    showtabs=false,                  
    tabsize=2
}

\lstset{style=mystyle}

\newcommand\sidenote[1]{\footnote{#1}}

%\renewtheoremstyle{break}%
  %{\item{\theorem@headerfont
          %##1\ ##2\theorem@separator}\hskip\labelsep\relax\nobreakitem}%
  %{\item{\theorem@headerfont
          %##1\ ##2\ (##3)\theorem@separator}\hskip\labelsep\relax\nobreakitem}
%% make \th@nonumberbreak the same as \th@break, but remove "\ ##2"
%\let\th@nonumberbreak\th@break
%\xpatchcmd*{\th@nonumberbreak}{\ ##2}{}{}{}
%\makeatother

\theorempreskip{10pt}
\theorempostskip{5pt}
\theoremstyle{break}



    % theorem envs

\newmdtheoremenv{thm}{Theorème}
\newtheorem{defn}{Definition}

\newmdtheoremenv{propo}[thm]{Proposition}

\newmdtheoremenv{crly}[thm]{Corollaire}

\newmdtheoremenv{lemma}[thm]{Lemme}

\newmdtheoremenv{propr}[thm]{Propriete}

\newtheorem*{rmq}[thm]{Remarque}

\newtheorem{axiom}[thm]{Axiom}

\newtheorem*{exemple}[thm]{Exemple}

\theoremsymbol{\ensuremath{ \square }}
\newtheorem*{proof}{Preuve}
\theoremsymbol{}

\newtheorem{exo}[thm]{Exercice}

\newcommand\eng[1]{\left\langle#1\right\rangle}

%\hypersetup{
    %unicode=true,          % non-Latin characters in Acrobat’s bookmarks
    %pdftoolbar=false,        % show Acrobat’s toolbar?
    %pdfmenubar=false,        % show Acrobat’s menu?
    %pdffitwindow=true,     % window fit to page when opened
    %colorlinks=true,
    %allcolors=magenta,
%}

% tikz


% horizontal rule
\newcommand\hr{
    \noindent\rule[0.5ex]{\linewidth}{0.5pt}
}

\newcommand{\incfig}[1]{%
    \def\svgwidth{\columnwidth}
    \import{./figures}{#1.pdf_tex}
}
\newcommand{\filler}[1][10]%
{   \foreach \x in {1,...,#1}
    {   test 
    }
}

\def\mathnote#1{%
  \tag*{\rlap{\hspace\marginparsep\smash{\parbox[t]{\marginparwidth}{%
  \footnotesize#1}}}}
}
\pdfsuppresswarningpagegroup=1
\newcommand\contra{\scalebox{1.5}{$\lightning$}}
\makeatother
\def\@lecture{}%
\newcommand{\lecture}[3]{
    \ifthenelse{\isempty{#3}}{%
        \def\@lecture{Lecture #1}%
    }{%
        \def\@lecture{Lecture #1: #3}%
    }%
    \subsection*{\@lecture}
    \marginpar{\small\textsf{\mbox{#2}}}
}
\makeatletter
\renewcommand{\baselinestretch}{1.2}
\setcounter{secnumdepth}{3}
\setcounter{tocdepth}{3}
\makeatletter
\patchcmd{\chapter}{\if@openright\cleardoublepage\else\clearpage\fi}{}{}{}
\makeatother
%\titleformat{\chapter}%
  %{\huge\rmfamily\itshape\color{mymagenta}}% format applied to label+text
  %{\llap{\colorbox{mymagenta}{\parbox[c][1cm]{3cm}{\hfill\itshape\Huge\textcolor{background}{\thechapter}}}}}% label
  %{5pt}% horizontal separation between label and title body
  %{\faLeaf}% before the title body
  %[]% after the title body

%\titleformat{\section}%
  %{\normalfont\Large\rmfamily\itshape\color{myblue}}% format applied to label+text
  %{\llap{\colorbox{myblue}{\parbox{3cm}{\hfill\itshape\textcolor{background}{\thesection}}}}}% label
  %{5pt}% horizontal separation between label and title body
  %{}% before the title body
  %[]% after the title body

%% subsection format
%\titleformat{\subsection}%
  %{\normalfont\large\itshape\color{mygreen}}% format applied to label+text
  %{\llap{\colorbox{mygreen}{\parbox{3cm}{\hfill\textcolor{background}{\thesubsection}}}}}% label
  %{1em}% horizontal separation between label and title body
  %{}% before the title body
  %[]% after the title body

%% subsubsection format
%\titleclass{\subsubsection}{straight}
%\titleformat{\subsubsection}%
  %{\normalfont\large\itshape\color{myyellow}}% format applied to label+text
  %{\llap{\colorbox{myyellow}{\parbox{3cm}{\hfill\textcolor{background}{\thesubsubsection}}}}}% label
  %{1em}% horizontal separation between label and title body
  %{}% before the title body
  %[]% after the title body

\usepackage{subfiles}

\title{Wikipedia Derivative}
\author{David Wiedemann}
\date{}
\begin{document}
\maketitle
\tableofcontents
\listoftheorems[ignoreall, show={thm,propo,crly,axiom,defn,lemma}]
\clearpage


\chapter{Derivative}
The derivative of a function of a real variable measures the sensitivity to change of the function value ( output
value) with respect to its argument ( input value). Derivatives are a fundamental tool of calculus. For example, the
derivative of the position of a moving object with respect to time is the object's velocity: this measures how
quickly the position of the object changes when time advances.\\
The derivative of a function of a single variable at a chose input value, when it exists, is the slope of the tangent
line to the graph of the function at that point. The tangent line is the best linear approximation of the function
near that input value. For this reason, the derivative is often described as the "instantaneous rate of change", the ratio of the instantaneous change in the dependent variable to that of the independent variable.\\

Derivatives may be generalized to functions of several real variables. In this generalization, the derivative is reinterpreted as a linear transformation whose graph is ( after an appropriate translation) the best linear approximation to the graph of the original function. The Jacobian matrix is the matrix that represents this linear transformation with respect to the basis given by the choice of independent and dependent variables. It can be calculated in terms of the partial derivatives with respect to the independent variables. For a real-valued function of several variables. For a real-valued function of several variables, the Jacobian matrix reduces to the gradient vector.\\

The process of finding a derivative is called differentiation. The reverse process is called antidifferentiation. The fundamental theorem of calculus relates antidifferentiation with integration. Differentiation and integration constitute the two fundamental operations in single-variable calculus.\\

\section{Differentiation}
Differentiation is the action of computing a derivative. The derivative of a function $y = f ( x)$ of a variable $x $ is a measure of the rate at which the value $y$ of the function changes with respect to the change of the variable $x$. It is called the derivative of $f$ with respect to $x$. If $x$ and $y$ are real numbers, and if the graph of $f$ is plotted against $x$, the derivative is the slope of this graph at each point.\\

The simplest case, apart from the trivial case of a constant function, is when $y $ is a line. In this case, $y = f ( x) = mx +b$, for real numbers $m$ and $b$, and the slope $m $ is given by
\[ 
	m = \frac{\text{change in}y}{\text{change in}x} = \frac{\Delta y}{\Delta x},
\]
where the symbol $\Delta$ ( Delta) is an abbreviation for "change in", and the combinations $\Delta x$ and $\Delta y$ refer to corresponding changes, i.e.: $\Delta y = f ( x + \Delta x) - f ( x)$. The above formula holds because
\begin{align*}
	\Delta y &= f ( x + \Delta x)\\
&= m ( x + \Delta x) + b = mx + m\Delta x + b\\
&= y + m \Delta x.
\end{align*}
Thus
\[ 
\Delta y = m \Delta x.
\]
\begin{figure}[ht]
    \centering
    \incfig{slope}
    \caption{slope}
    \label{fig:slope}
\end{figure}

\begin{figure}[ht]
    \centering
    \incfig{tangentline}
    \caption{tangentline}
    \label{fig:tangentline}
\end{figure}
\begin{figure}[ht]
    \centering
    \incfig{not-a-tangent-line}
    \caption{Not a tangent line}
    \label{fig:not-a-tangent-line}
\end{figure}
This gives the value for the slope of a line.\\

If the function $f$ is not linear ( i.e. its graph is no straight line), then the change in $y$ divided by the change in $x$ varies over the considered range: differentiation is a method to find a unique value for this rate of change, not across a certain range ( $\Delta x$,) but at any given value of $x$.

The idea, illustrated by Figures 1 to 3, is to compute the rate of change as the limit value of the ratio of the differences $\Delta y / \Delta x$ as $\Delta x$ tends towards 0.

\section{Notation} Two disting notations are commonly used for the derivative, one deriving from Gottfried Wilhelm Leibniz and the other from Joseph Louis Lagrange. A third notation, first used by Isaac Newton, is sometimes seen in physics.\\

In Leibiz's notation, an infitesimal change in $x$ is denoted by $dx$, and the derivative of $y$ with respect to $x$ is written 
\[ 
\frac{dy}{dx}
\]
suggesting the ratio of two infitesimal quantities. ( The above expression is read as "the derivative of $y$ with respect to $ x$, "$dy$ by $dx$ ", or "$dy$ over $dx$ " The oral form "$dy dx$ " is often used conversationally, although it may lead to confusion.)\\

In lagrange's notation, the derivative with respect to $x$ of a function $f(x)$  is denoted $f'(x)$ ( read as "$f$ prime of $x$ ") or $f'_x(x)$ ( read as "$f$ prime $x$ of $x$), in case of ambiguity of the variable implied by the differentiation. Lagranges notation is sometimes incorrectly attributed to Newton.\\

Newtons's notation for differentiation ( also called the \textbf{ dot notation}for differentiation) places a dot over the dependent variable. That is, if $y$ is a function of $t$, then the derivative of $y$ with respect to $t$ is
\[ 
	\dot{y}
\]
Higher derivatives are represented using multiple dots, as in
\[ 
	\ddot{y}, \dddot{y}
\]
Newton's notation is generally used when the independent variable denotes time. If location $y$ is a function of $t$, then $\dot{y}$ denotes velocity and $\ddot{y}$ denotes acceleration.\\


\section{Rigorous definition}

The most common approach to turn this intuitive idea into a precise definition is to define the derivative a s a limit of difference quotients of real numbers. This is the approach described below.\\
Let $f$ be a real valued function defined in an open neighborhood of a real number $a$. In classical geometry, the tangent line to the graph of the function $f$ at $a$ was the unique line through the point $(a,f(a))$ that did not meet the graph transversally, meaning that the line did not pass straight through the graph. The derivative of $y$ with respect to $x$ at $a$ is, geometrically, the slope of the tangent line to the graph of $f$ at $(a,f(a))$.
The slope of the tangent line is very close to the slope of the line through $(a,f(a))$ and a nearby point ont the graph, for example $(a+h,f(a+h))$. These lines are called secant lines. A value of $h$ will, in general, give better approximations. The slope $m$ of the secant line is the difference between the $y$ values of these points divided by the difference between the $x$ values, that is,
\[ 
	m = \frac{\Delta f(a)}{\Delta a} = \frac{f(a+h) - f(a)}{h}
\]

This expression is Newton's difference quotient. Passing from an approximation to an exact answer is done using a limit. Geometrically, the limit of the secant lines is the tangent line. Therefore, the limit of the difference quotient as $h$ approaches zero, if it exists, should represent the slope of the tangent line to $(a,f(a))$. This limit is defined to be the derivative of the function $f$ at $a$ :

\[ 
	f'(a) = \lim_{h \to 0} \frac{f(a+h) -f(a)}{h}.
\]
When the limit exists, $f$ is said to be differentiable at $a$. Here $f'(a)$ is one of several common notations for the derivative ( see below). From this definition it is obvious that a differentiable function $f$ is increasing if and only if its derivative is positive, and is decreasing iff its derivative is negative.
This fact is used extensively when analyzing function behavior, e.g. when finding local extrema.

Equivalently, the derivative satisfies the property that
\[ 
	\lim_{h \to 0} \frac{f(a+h) - ( f(a) + f'(a)\cdot h)}{h} = 0,
\]
which has the intuitive interpretation ( see Figure 1) that the tangent line to $f$ at $a$ gives the best linear approximation
\[ 
	f(a+h) \approx f(a) + f'(a)h
\]
to $f$ near $a$ ( i.e. for small $h$). This interpretation is the easiest to generalize to other settings ( see below).\\
Substituting $0$ for $h$ in the difference quotient causes division by zero, so the slope of the tangent line cannot be found directly using this method. Instead, define $Q(h)$ to be the difference quotient as a function of $h$ :
\[ 
	Q(h) = \frac{f(a+h) - f(a)}{h}.
\]
$Q(h)$ is the slope of the secant line between $(a,f(a))$ and $(a+h,f(a+h))$. If $f$ is a continuous function, meaning that its graph is an unbroken curve with no gaps, then $Q$ is a continuous function away from $h=0$. If the limit $\lim_{h \to 0} Q(h)$ exists, meaning that there is a way of choosing a value for $Q(0)$ that makes $Q$ a continuous function, then the function $f$ is differentiable at $a$, and its derivative at $a$ equals $Q(0)$.\\

In practice, the existence of a continuous extension of the difference quotient $Q(h)$ to $h=0$ is shown by modifying  the numerator ti cancel $h$ in the denominator. Such manipulations can make the limit value of $Q$ for small $h$ clear even though $Q$ is still not defiend at $h=0$. This process can be long and tedious for complicated functions, and many shortcuts are commonly used to simplify the process.

\section{Definition over the hyperreals}
Relative to a hyperreal extension $\mathbb{R} \subset *\mathbb{R}$ of the real numbers, the derivative of a real function $y=f(x)$ at a real point $x$ can be defined as the shadow of the quotient $\frac{\Delta y}{\Delta x}$ for infinitesimal $\Delta x$, where $\Delta y = f(x+\Delta x) - f(x)$.\\
Here the natural extension of $f$ to the hyperreals is still denoted $f$. Here the derivative is said to exist if the shadow is independent of the infinitesimal chosen.
\section{Continuity and differentiability}
If $f$ is differentiablee at $a$, then $f$ must also be continuous at $a$. As an example, choose a point $a$ and let $f$ be the step function that returrns the value 1 for all $x$ less than $a$, and returns a value 10 for all $x$ greater than or equal to $a$. $f$ cannot have a derivative at $a$. If $h$ is negative, then $a+h$ is on the low part of the step, so the secant line from $a$ to $a+h$ is very steep, and as $h$ tends to zero the slope tends to infinity. If $h$ is positive, then $a+h$ is on the high part of the step, so the secant line from $a+h$ is on the high part of the step, so the secant line from $a$ to $a+h$ is very steep, and as $h$ tends to zero the slope tends to infinity. If $h$ is positive, then $a+h$ is on the high part of the step, so the secant line from $a$ to $a+h$ has slope zero. Consequently, the secant lines do not approach any single slope, so the  limit of the difference quotient does not exist.\\

However, even if a function is continuous at a point, it may not be differentiable there. For example, the absolute value function given by $f(x) = \abs{x}$ is continuous at $x=0$, but it is not differentiable there. If $h$ is positive, then the slope of the secant line from $0$ to $h$ is negative one. This can be seen graphically as a "kink" or a "cusp" in the graph at $x=0$.\\

\begin{marginfigure}
    \incfig{absolute-value-function}
    \caption{Absolute value function}
    \label{fig:absolute-value-function}
\end{marginfigure}
In summary, a function that has a derivative is continuous, but there are continuous functions that do not have a derivative.

Most functions that occur in practice have derivatives at all points or at almost every point. Early in the history of calculus, many mathematicians assumed that a continuous function was dThe derivative of a function of a real variable measures the sensitivity to change of the function value ( output
value) with respect to its argument ( input value). Derivatives are a fundamental tool of calculus. For example, the
derivative of the position of a moving object with respect to time is the object's velocity: this measures how
quickly the position of the object changes when time advances.\\
The derivative of a function of a single variable at a chose input value, when it exists, is the slope of the tangent
line to the graph of the function at that point. The tangent line is the best linear approximation of the function
near that input value. Differentiable at most points. Under mild conditions, for example if the function is a monotone function or a Lipschitz function, this is true. However, in 1872 Weierstrass found the first example of a function that is continuous everywhere but differentiable nowhere. This example is now know as the Weierstrass function. In 1931, Stefan Banach proved that the set of functions that have a derivative at some point is a meager set in the space of all continuous functions. Informally, this means that hardly do any random continuous functions have a derivative at even one point.
\section{The derivative as a function}
Let $f$ be a function that has a derivative at every point int its domain. We can then define a function that maps every point $x$ to the value of the derivative of $f$ at $x$. This function is written $f'$ and is called the derivative function or the derivative of $f$.\\

Sometimes $f$ has a derivative at most, but not all, points of its domain.
The function whose value at $a$ equals $f'(a)$ whenever $f'(a)$ is defined and elsewhere is undefi







\subfile{lectures/lec1.tex}
%EOL
\end{document}
