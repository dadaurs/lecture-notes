\documentclass[../main.tex]{subfiles}
\begin{document}
\lecture{4}{Thu 20 Apr}{things}
\begin{lemma}
	If $X$ is $G_+$-qc and $ \mathcal{S} \subset \mathcal{O}_X$ a sieve s.t. $X^{\ast}= \bigcup_{V\in \mathcal{S}} V^{\ast}$ then $\mathcal{S} /= X$
\end{lemma}
\begin{proof}
Assume that $ \mathcal{S}$ does not cover $X$.\\
Then
\[ 
	M = \left\{ \text{ sieves } T | \mathcal{S} \subset T \subset \mathcal{O}_X, T \text{ does not cover } X \right\} 
\]
is non empty and is partially ordered by inclusion.\\
Let $ \mathcal{L} \subset M$ be linearly order and let $R= \bigcup_{T \in L} T$.\\
If $R /= X$, then by quasi-compactness of $X$ there would be finitely many $V_i, V_i \in R$ such that $ [ V_1, \ldots, V_N] /= X$.\\
As $L$ is linearly ordered by inclusion, there is $T\in L$ such that $ \left\{ V_1,\ldots,V_N \right\} \subset T$. Then $T /= X$, contradicting the definition of $M$.\\
Thus $M$ has a maximal element $T$.\\
By a remark, $\xi = \mathcal{O}_X \setminus \tau \in X^{\ast} $ and $\xi \notin ( \bigcup_{V\in T} V^{\ast}) \supset \bigcup_{V \in T} V^{\ast}$.\\
Contradicting our assumption on $ \mathcal{S}$.
\end{proof}
\begin{propo}
If $X$ is a $G_+$-space with a $G_+$-base $B$ closed under taking intersections and such that the elements of $B$ are $G_+$-qc.\\
THen $X^{\ast}$ has sufficiently many Van-der Put points.\\
$X^{\ast}$ is also locally spectral.
\end{propo}
\begin{proof}
Let $ \mathcal{S} \subset \mathcal{O}_V, V \in \mathcal{O}_X$ be a sieve such that $U^{\ast}=\bigcup_{V\in \mathcal{S}} V^{\ast}$ but not $\mathcal{S} /= V$.\\
As $ [ B_V] /= V$ and by GTloc, there is $V \in [ B_U] $ which is not covered j
\end{proof}

\begin{defn}
A property $E$ of open subsets of a $G_+$-space is local iff
\begin{enumerate}
\item $E( U) $ and $V \subset U$ $\implies E( V) $ 
\item $ \left\{ V \in \mathcal{O}_U| E( V)  \right\} /= U \implies E( U) $ 
\end{enumerate}
\end{defn}
\begin{thm}
Having sufficiently many vdp points is a local property
\end{thm}
\begin{thm}
If $E$ is a local property and $E( \Omega) $ for all $\Omega \in B$ where $b$ is a $G_+ $ basefor $X$, then $E( X) $ 
\end{thm}

\begin{thm}
	If $ \mathcal{S}= [ V_1,\ldots,V_n] $ where $V_i$ are qc and $ \mathcal{S} /= U$ then $U$ is qc.
\end{thm}
\begin{proof}
Let $T /= U$, then $T|_{V_i} /= V_i$, hence finitely many $( W_{ij} ) , W_{ij } \in T|_{V_i} $ and 
\end{proof}




\end{document}	
