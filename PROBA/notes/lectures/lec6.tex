\documentclass[../main.tex]{subfiles}
\begin{document}
\lecture{6}{Wed 27 Oct}{Random Variables}
\section{Random Variables}
\begin{defn}[Random Variables]
	Let $( \Omega, \mathcal{F}, \mathbb{P}) $ be a probability space.\\
	Then $X: \Omega\mapstot \mathbb{R}$ measurable as a map $ ( \Omega, \mathcal{F}) \to ( \mathbb{R}, \mathcal{F}_E) $ is called a ( real) random variable.
\end{defn}
The pushforward measure $ \mathbb{P}_X ( F) = \mathbb{P}( X^{-1}( F) ) \forall F \in \mathcal{F}_E$ is called the law of $X$ 
\begin{rmq}
	There is a more general notion of $( \Omega_2, \mathcal{F}_2) $ valued random variable.
\end{rmq}
\begin{defn}[Equality of RV]
	$X,Y$ two random variables are called equal in law if
	\[ 
		\mathbb{P}_X( F) = \mathbb{P}_Y( F) \forall F \in \mathcal{F}_E
	\]
	
\end{defn}
\begin{defn}
$X$ is a R.V. we call the c.d.f. of $\mathbb{P}_X$ $F_X$ 
\[ 
	F_X( s) = \mathbb{P}_X( X \leq s ) 
\]
			
\end{defn}
\begin{propo}
	Each R.V. $X$ gives rise to a unique c.d.f. $F_X( s) = \mathbb{P}_X( X \leq s) $ and conversely, each c.d.f. gives rise to a unique law of a probability measure
\end{propo}
\begin{proof}
Follows directly from the proposition relating probability measures and c.d.f.
\end{proof}
\begin{lemma}
\begin{enumerate}
	\item $\mathbb{P}_X<s = F( S^{-}) $ 
	\item $\mathbb{P}_X( X=s) = F( s) -F( s^{-}) $ 
	\item $ \mathbb{P}_X( X\in ( a,b) ) = F( b^{-}) - F( a) $ 
\end{enumerate}
\end{lemma}
\begin{defn}
$X$ a R.V., $s\in \mathbb{R}$.\\
If $F( s) - F( s^{-}) >0 \iff \mathbb{P}_X( X=s) >0$, then $s$ is a atom of $X$ 
\end{defn}
\begin{lemma}
A R.V. can have at most countably many atoms or in other words,  a c.d.f. can have at most countably many jumps.
\end{lemma}
\begin{defn}
	$X$ a R.V.\\
	If $F_X$ increases by jumps, we call $X$  a discrete R.V.\\
	If $F_X$ is cts, we call $X$ a cts R.V.
\end{defn}
\begin{propo}
	$X$ a R.V. Then we can write $F( X) = aF_Y + b F_Z$ s.t. $a+b=1$ and $Y$ discrete, $Z$ cts R.V.
\end{propo}
\begin{proof}
If $F_X$ is discrete or cts, we are done.\\
$\exists S= \left\{ s_1,s_2, \ldots \right\}  $ s.t. $F_X( s_i) - F_X( s_i^{-}) >0$ iff $s_i \in S$ .\\
Consider
\[ 
	\hat{F}_Y( s) = \sum 1_{ \left\{ S \geq s_i \right\}  } ( F( s_i) -F( s_i^{-}) ) 
\]
and
\[ 
	\hat{F}_Z( s) = F_X( s) - \hat{F}_Y( s) 
\]
We now show that $ \hat{F}_Z$ continuous.\\
Finally, define
\[ 
	F_Y( s) = \frac{ \hat{F}_Y( s) }{ \hat{F}_Y( \infty ) }
\]
and similarly
\[ 
	F_Z( s) = \frac{\hat{F}_Z( s) }{\hat{F}_Z( \infty ) }
\]

\end{proof}
	






\end{document}	
