\documentclass[../main.tex]{subfiles}
\begin{document}
\lecture{1}{Wed 05 Apr}{Covering sieves}
\begin{thm}
	Every sieve $\tau$ containing a covering sieve $\tau'$ of $X$ is itself covering.\\
	The intersection of two covering sieves is covering.
\end{thm}
\begin{proof}
If $( v:V\to X) $ is a morphism in $\tau'$ then $v^{\ast}\tau= v \tau'^{\ast}$.\\
Let $\tau,\tau'$ be covering sieves of $X$ and $v:V\to X\in \tau$, then $v^{\ast }( \tau\cap \tau')= v^{\ast}\tau'$.\\\
This covers $V$ by GTTrans, by GTLoc, $\tau\cap \tau' $ covers $X$.
\end{proof}
\begin{rmq}
We are mostly interested in the case where the category $C$ is the poset of open subsets of a topological space.\\
Then a sieve in $V$ is just a set of open subsets of $V$ such that $V\in \tau$, $W \subset V, W$ open implies $W\in \tau$.\\
The pullback along the (unique if it exists) morphism $V\to U$ are just the open subsets of $V$.\\
We write $\tau /= V$ if $\tau$ is a sieve oveover$V$ which covers $V$.\\
If several grothendieck topologies must be distinguished, I will write $\tau /=_\pi V$ 
\end{rmq}
\begin{defn}
We will write $ [ V_i | i \in I ] $ for the sieve generated by the family $V_i$ of open subsets of $V$.
We have $ [ V_i] = \left\{ V\in O_X| \exists i \in I \text{ st } V \subset V_i \right\} =\bigcap_{\tau \text{ sieve in  } X \text{ containing  } V_i} \tau  $.\\
A sieve is finitely generated if it can be written as $ [ V_i] $ for finitely many $V_i$ 
\end{defn}
\begin{rmq}
More generally, we consider Grothendieck topologies on $B$, a topology base for $X$, considered as posets.
\end{rmq}
\begin{defn}
	Let $ [ \Omega_i]_B$ be the $B$-sieve generated by the $\Omega_i$, ie.
	\[ 
	\left\{ \theta\in B |\theta \subset \Omega_i \text{ for at least one  } i \in I\right\} 
	\]
The subscript $B$ will always be used when $B \subsetneq O_X$.	
\end{defn}
\begin{propo}
	Let $X$ be a topological space and $B$  a topology base for $X$.\\
	Then we have a bijection between
	\begin{itemize}
	\item Grothendieck topologices $T_B$ on $B$ 
	\item Grothendieck topologies $T$ on $O_X$ st. $[B_V]$ covers $V$.
	\end{itemize}
	If $T_B$ is given, $T$ is defined by $\tau /=_T V, \tau\cap B_\Omega /=_{T_B} \Omega$ for all $\Omega \in B_V$.\\
	When $T$ is given, $T_B$ is defined by
	\[ 
	\tau /=_{T_B} \Omega \text{ iff }  [ \tau] /=_T \Omega
	\]
\end{propo}




\end{document}	
