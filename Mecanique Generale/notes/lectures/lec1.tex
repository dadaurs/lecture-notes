\documentclass[../main.tex]{subfiles}
\begin{document}
\lecture{1}{Wed 16 Sep}{Cours de Physique Generale}
\section{Physique}
\begin{itemize}
	\item Science dont le but est d'etudier et de comprendre les composants de la matiere et leurs interactions mutuelles.\\
	\item Sur la base des proprietes observees de la matiere et des interactions, le physicien tente d'expliquer les phenomenes naturels observables.\\
	\item Les ``explications'' sont donnees sous forme de lois aussi fondamentales que possible: elles resument notre comprehension des phenomenes physiques.\\
	\item Les maths sont le language qu'on utilise pour decrire ces phenomenes.
\end{itemize}
\underline{Exemple}\\
Une particule se deplace sur un axe droit.\\
Au temps $t_1$ position $x_1 = x(t_1)$.
Au temps $t_2$ position $x_2 = x(t_2)$.
$\Delta x = x_2 -x_1$ et $\Delta t = t_2-t_1$\\
Donc la vitesse moyenne 
\[ 
	v_{moyenne} = \frac{\Delta x}{\Delta t}
\]
Mais on peut faire diminuer $\Delta t$, pour connaitre la vitesse moyenne sur un temps infinitesimal:
\[ 
	v = \lim_{\Delta t \to 0} \frac{\Delta x}{\Delta t} = \frac{dx(t)}{dt} = \dot{x}(t)
\]
Donc la vitesse instantanee est la derivee de la fonction $x(t)$ par rapport a $t$.

On peut faire la meme chose avec l'acceleration\\
Au temps $t_1$, vitesse $ v_1 = v(t_1)$.\\
Au temps $t_2$, vitesse $ v_2 = v(t_2)$.\\
Donc l'acceleration moyenne est
\[ 
	a_{moyenne}= \frac{v_2-v_1}{t_2 -t_1} = \frac{\Delta v}{\Delta t}
\]

Et donc par le meme raisonnement, l'acceleration instantanee est
\[ 
	a = \lim_{\Delta t \to 0} \frac{\Delta v}{\Delta t} = \frac{dv(t)}{dt} := \dot{v}(t) = \ddot{x}(t)
\]

\subsection{Exemple de loi physique: l'addition des vitesse}
Si je marche a la vitesse $v_{marche}$ sur un tapis , alors la vitesse par rapport au sol est
\[ 
	V = v_{marche} + v_{tapis}
\]
C'est la loi d'addition des vitesses de galilee.\\
Ici, c'est une addition \underline{vectorielle} qu'il faut faire.\\

Cette loi est 
\begin{itemize}
	\item independante des vitesses\\
	\item independante des objets en presence\\
	\item independante du temps ( hier, aujourd'hui, demain)\\
	\item etc...
\end{itemize}

\subsection{Lois de conservation}
Ce sont les lois les plus fondamentales.
\begin{itemize}
	\item Conservation de l'energie\\
	\item Conservation de la quantite de mouvement\\
	\item Conservation du moment cinetique
\end{itemize}
Ces lois sont valables dans toutes les situations ( classiques, relativistes ou quantiques) .\\
Ne peuvent pas etre formulees mathematiquement de facon unique.\\
Resultent des principes ``d'invariance''  (ou de symmetrie)  tres generaux.\\
\subsection{Invariance par changement de referentiel}
\begin{itemize}
	\item Changement de referentiel ( ou d'observateur):
	Referentiel $O'x'y'z'$ en mouvement par rapport au referentiel $Oxyz$ \\
\item Les lois de la physqiue sont-elles invariantes par rapport a n'importe quel changement de referentiel?\\
	Autrement dit, si les observateurs $O$ et $O'$ font la meme experience, obtiendront-ils le meme resultat?\\
\item Principe de Galilee:\\
	Les lois de la physique sont les memes (i.e. invariantes)  pour deux observateurs en mouvement rectiligne uniforme l'un par rapport a l'autre.
\end{itemize}
\section{La mecanique classique}
\begin{enumerate}
	\item Mecanique:\\
		science du mouvement ( ou du repos) de systemes materiels caracterises par des variables d'espace et de temps.\\
	\item Cinematique:\\
		Description du mouvement.\\
	\item Dynamique:\\
		Etude de la relation entre le mouvement et les causes de sa variation(forces, lois de Newton, th. du moment cinetique).\\
\item Statique:\\
	Etude et description de l'equilibre.
\end{enumerate}
\section{Objectifs du cours de mecanique generale}
$\bullet$ Apprendre a mettre sous forme mathematique un probleme, une situation physique:
\begin{itemize}
	\item Definir le probleme, le modeliser\\
	\item Choisir une description mathematique\\
	\item Poser les equations regissant la physique du probleme\\
	\item Resoudre et/ou discuter la solution
\end{itemize}
$\bullet$ Developper un ``savoir-faire''  pratique, mais egalement un esprit scientifique:
\begin{itemize}
	\item Reperer le sens physique derriere les equations\\
	\item Savoir formaliser mathematiquement la donnee d'un probleme physique.
\end{itemize}

\section{Le modele du ``point materiel'' }
\begin{defn}[Point materiel]\index{Point materiel}\label{defn:point_materiel}
	un systeme est assimile a un point geometrique auquel on attribue toute la masse de ce systeme, et dont l'etat est decrit en tout temps par une ( seule) position et une ( seule) vitesse.
\end{defn}
$\bullet$ Notion introduite par Newton.\\

On approxime un systeme a quelquechose de plus simple, le point peut etre ``gros'' ( exemple:la terre, le soleil).\\
Pas applicable dans toutes les situations; le modele a des limites..
\section{Mouvement Rectiligne Uniforme}
Mouvement d'un point materiel se deplacant en ligne droite a vitesse constante.\\
On definit un axe $x$ associe a la trajectoire rectiligne, avec une origine $O$.
\[ 
	v(t) := \frac{dx(t)}{dt} = \dot{x}(t) = v_0= \text{constante}
\]
La solution s'obtient en integrant le dessus: $x(t) = v_0 t + x_0$, ou $ x_0=$ constante.\\
On appelle le resultat de cette integration l'equation horaire.\\
\section{Mouvement rectiligne uniformement accelere}
Ici
\[ 
	a(t) := \frac{d^{2}x(t)}{dt^{2}	} = \ddot{x} ( t) = a_0 = constante	
\]
C'est une equadiff d'ordre 2 faisant intervenir la derivee seconde de $x(t)$.\\
Solution
\begin{align*}
	x(t) &= a_0 \frac{t^{2}}{2} + v_0 t + x_0\\
	v(t) &= \frac{dx}{dt} = a_0 t + v_0
\end{align*}
ou $ x_0$ et $ v_0$ sont des constantes.\\

\section{Lois de newton}
\begin{itemize}
	\item mouvement rectiligne uniforme $\Rightarrow \vec{F} = \vec{0}$ \\
	\item $\vec{F} = m \vec{a}$ \\
	\item Action reaction $\vec{F} = -\vec{F}$
\end{itemize}
\section{Force de pesanteur et chute des corps}
$\bullet$ L'attraction terrestre donne lieu a une force verticale ( le poids)  proportionelle a la masse $m$ :
\[ 
F= mg
\]
$g \approx 9.8 \frac{m}{s^{2}}$\\
$\bullet$ Application de la 2eme loi de Newton:\\
Si le poids est la seule force appliquee a un point materiel
\[ 
F = ma \Rightarrow a = g = constante
\]
Dans le vide, les corps ont un mouvement uniformement accelere



	

	










\end{document}	
