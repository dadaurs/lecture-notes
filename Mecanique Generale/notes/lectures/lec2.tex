\documentclass[../main.tex]{subfiles}
\begin{document}
\lecture{2}{Wed 16 Sep}{Notions Mathematiques}
\section{Quelques notions mathematiques}
\subsection{Fonctions}
\[ 
	F(x) \underbrace{ \longrightarrow }_{\text{derivee}} F'(x) = \frac{dF}{dx}(x) = \lim_{\Delta x \to 0} \frac{F(x+\Delta x) -F(x)}{\Delta x}
\]
Quand on parle d'un point dans l'espace, on aura tjrs 3 coordonnees
\begin{align*}
	x(t) \longrightarrow &x'(t) := \dot{x} ( t) := \dot{x}\\
	&x''(t) := \ddot{x}(t) := \ddot{x}
\end{align*}
\hr\\
\[ 
	\frac{d}{dt}(x^{2}(t)) = 2 x(t) \dot{x}(t) = 2x \dot{x}
\]
Si je fais
\[ 
	\frac{d}{dx}(x^{2}) = 2x
\]
\subsection{Equations Differentielles}
\[ 
	F''(x) = C
\]
Pour resoudre
\begin{align*}
	F'(x) &= Cx + D\\
	F(x) &= \frac{1}{2}Cx^{2} + Dx + E
\end{align*}
\[ 
	mg = F = ma = m\ddot{x}
\]
\hr\\
\[ 
	\ddot{x}= -c^{2}x
\]
On devine la solution:
\begin{align*}
	x(t) &= A \sin(Ct) + B \cos(Ct)\\
	\dot{x} = AC \cos ( Ct) - BC sin(Ct)\\
	\ddot{x}(t) &= -AC^{2}\sin(Ct) - BC^{2} \cos(Ct)\\
		    &= - C^{2} [ A sin(Ct) + Bcos(Ct)] = -C^{2} x(t)
\end{align*}
\section{Vecteurs}
\[ 
\vec{v} = \begin{pmatrix}
v_x \\
v_y\\
v_z
\end{pmatrix}
= 
\begin{pmatrix}
v_1\\
v_2\\
v_3
\end{pmatrix}
\]
\[ 
\vec{x} = 
\begin{pmatrix}
x_1\\
x_2\\
x_3\\
\end{pmatrix} = x_1 \cdot \vec{e_1} + x_2 \cdot \vec{e_2} + x_3 \cdot \vec{e_3}
\]
Le point $(x_1,x_2,x_3)$ on l'atteint en faisant une combinaison lineaire de $(e_1,e_2,e_3)$.
\section{Trigonometrie}
\begin{figure}[ht]
    \centering
    \incfig{cercletrigo}
    \caption{cercletrigo}
    \label{fig:cercletrigo}
\end{figure}


\end{document}	
