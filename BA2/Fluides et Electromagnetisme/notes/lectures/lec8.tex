\documentclass[../main.tex]{subfiles}
\begin{document}
\lecture{8}{Fri 19 Mar}{Ecoulement d'un fluide visqueux}
\subsection{Ecoulement d'un fluide visqueux}
On se restreins au cas des fluides incompressibles.\\
\subsubsection{Definition de la viscosite}
\begin{exemple}[Viscosimetre de Couette]
\begin{figure}[ht]
    \centering
    \incfig{viscosimetre}
    \caption{viscosimetre}
    \label{fig:viscosimetre}
\end{figure}
Observation:\\
Pour la meme frequence angulaire, le moment de force necessaire pour faire tourne le cylindre depend du type de fluide.

\end{exemple}
Imaginons l'experience suivante ( "Ecoulement de Couette"), on a

\begin{figure}[H]
    \centering
    \incfig{ecoulement-de-couette}
    \caption{ecoulement de couette}
    \label{fig:ecoulement-de-couette}
\end{figure}
Il nous faut une force externe ( force de cisaillement ), on remarque que
\[ 
	F_{ext} \propto  S \frac{u_0}{h}
\]
On definit $\eta$ par
\[ 
	F_{ext} = \eta S \frac{u_0}{h}
\]
$\eta$ s'appelle le coefficient de viscosite dynamique (  $[\eta] = \frac{N}{m^{2}}s= Pa \cdot s$).\\
On a
\[ 
	F_{ext} = \eta S \frac{u_0}{h} = \eta S \frac{\del u}{\del y}	
\]
Les fluides qui obeissent cette loi s'appellent les fluides Newtoniens.
\subsubsection{Force de viscosite par unite de volume et equation de Navier-Stokes incompressible}
Supposons $\vec{u}( \vec{r},t) = u_x( y) \vec{e}_x$
\begin{figure}[H]
    \centering
    \incfig{viscosite-par-volume}
    \caption{viscosite par volume}
    \label{fig:viscosite-par-volume}
\end{figure}
La force de viscosite sur un elelement
\begin{align*}
	F_{visc} &= - \eta S \frac{\del u_x}{\del y}( y) e_x + \eta S \frac{\del u_x}{\del y}( y+ dy) e_x\\
		 &= \eta S dy \frac{ \frac{\del u_x}{\del y}( y+dy) - \frac{\del u_x}{\del y}}{dy} e_x\\
		 &= dV \eta \frac{\del ^{2} u_x}{\del y^{2}}e_x
\end{align*}
Et donc le terme
\[ 
\eta \frac{\del ^{2} u_x}{\del y^{2}}
\]
est la force par unite de volume.\\
Dans le cas general, pour un fluide incompressible ( sans preuve) est donne par
\[ 
	\eta ( \frac{del^{2}}{\del x^{2}}+\frac{del^{2}}{\del y^{2}}+\frac{del^{2}}{\del z^{2}}) \vec{u} = \eta \Delta \vec{u}
\]
Les equations pour un fluide incompressible et visqueux sont donc:
\begin{align*}
\rho = \text{ const } \\
\nabla \cdot \vec{u} = 0\\
\rho \frac{D \vec{u}}{Dt} = \rho \vec{g} - \nabla p + \eta \Delta \vec{u}\rightarrow \text{ Navier-Stokes incompressible } 
\end{align*}
Les conditions de bords sont donnees par
\[ 
\vec{u}=0
\]
a l'interface avec des parois immobile.
\subsubsection{Resolution Navier-Stokes avec l'ecoulement de Poiseuille}
Ecoulement laminaire dans un tube
\begin{figure}[H]
    \centering
    \incfig{glicerine-dans-un-tube}
    \caption{glicerine dans un tube}
    \label{fig:glicerine-dans-un-tube}
\end{figure}

Dans un cas plus simple
\begin{figure}[H]
    \centering
    \incfig{ecoulement-plaques}
    \caption{ecoulement plaques}
    \label{fig:ecoulement-plaques}
\end{figure}
On suppose $\rho$ constant, on neglige la pesanteur et on pose
\[ 
	\vec{u}( \vec{r},t) = \vec{u}_x( y) \vec{e}_x
\]




\end{document}	
