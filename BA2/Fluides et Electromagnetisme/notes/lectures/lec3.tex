\documentclass[../main.tex]{subfiles}
\begin{document}
\lecture{3}{Tue 02 Mar}{Hydrostatique}
\subsection{Densite de force associee a la pression}
Calculons la force exercee sur un volume de fluide infinitesimal du a la pression.\\
\begin{figure}[H]
    \centering
    \incfig{cube_densite}
\end{figure}
On suppose qu'on connait $p( \vec{r}) $.
\[ 
	\vec{F}_1=p( \vec{r}( - \frac{dx}{2},0,0) )  dy dz \vec{e}_x
\]
Donc
\begin{align*}
	\sum_{i=1}^{ 6}\vec{F}_i &= ( -p( \frac{dx}{2},0,0) - p(  \frac{dx}{2},0,0) )  dy dz + \ldots\\
				 &= - \frac{p( \frac{dx}{2},0,0) - p( - \frac{dx}{2},0,0) }{dx}dx dy dz \vec{e}_x + \ldots \\
				 &= - \frac{\del p}{\del x} dV \vec{e}_x + \ldots = - \nabla p dV
\end{align*}

donc la densite de force associee a la pression est $-\nabla p$.
\subsection{Poussee d'Archimede}
Tout corps plonge dans un fluide recoit de la part de celui-ci une poussee verticale egale au poids du fluide deplace
\subsection{Tension superficielle}
Experience:\\
On a des tubes de largeurs differentes, ouverts en haut et plonge dans l'eau.\\
On note que le niveau d'eau monte a un niveau de $h \alpha \frac{1}{r}$ \\
Semble etre une contradiction de la pression hydrostatique.\\
On verra que ce phenomene est du a la tension superficielle.
La loi $p( h) = p_0 + \rho g h$ reste valable dans le fluide, mais pas necessairement a la surface.
\subsubsection{Origine et definition de la tension superficielle}
On considere a nouveau un fluide, il est constitue de particules ayant des interactions entre elles ( inter moleculaires, etc) \\
Il y a moins de telles liaisons pour une molecule a la surface du fluide. Pour amener cette molecule la-bas et pour augmenter la surface, il faut faire un travail.\\
Experience:\\
Soit un film de liquide ( eau savonneuse) tendu dans un cadre ABCD.\\
Si on tend le cadre, il ya une force qui s'y oppose.\\
Le travail est donc proportionel au changement de surface
\[ 
\Delta W = \gamma \Delta S = \gamma BC \Delta k \cdot 2
\]
Le 2 apparait parce que il y a 2 surfaces ( liquide/gaz)\\
Donc on a
\[ 
F= 2 F_\gamma = 2 BC \gamma
\]
L'interface liquide/gaz est un peu comme une membrane elastique, mais la force est independante de la deformation.\\
Experience\\
Mesure de $\gamma$ 
On plonge un cylindre attache a un newton metre dans le liquide.\\
On mesure la force necessaire pour faire apparaitre un film lie au cylindre et on prend la difference entre cette force et la force $F_G$.\\

\subsubsection{Quelques consequences immediates de la tension superficielle}
Les bulles de savon minimisent leur surface et c'est pour cela qu'elles sont spheriques.\\
Meme chose pour les bulles d'eau en apesanteur.\\
Meme chose pour les cheveux mouilles qui collent.\\
Certains objets ( trombone, punaises) ou des insectes qui flottent ( qui marchent sur la surface) 



\end{document}	
