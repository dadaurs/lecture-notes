\documentclass[../main.tex]{subfiles}
\begin{document}
\lecture{14}{Tue 20 Apr}{Loi de Gauss}
\begin{figure}[H]
    \centering
    \incfig{conducteur-avec-deux-charges-positives}
    \caption{conducteur avec deux charges positives}
    \label{fig:conducteur-avec-deux-charges-positives}
\end{figure}
Le debut des lignes de champ $E$: charges positives ou a l'infini, et la fin des lignes de champ sont a l'infini ou sur la charge negative.\\
Les lignes de champ peuvent etre parametrees $\vec{l}( s) $ ( courbe parametree par $s$) determinee par la condition $ \frac{ d \vec{l}}{ ds}\times  \vec{E}( \vec{l}( s) ) = 0 \forall s$ \footnote { il faut que $\frac{d \vec{l}}{ds}$ soit parallele a $ \vec{E}( \vec{l}( s) ) $} 
\subsection{Loi de Gauss}
\subsubsection{Enonce de la loi de Gauss}
\begin{center}
``\textit{ Le flux du champ electrique a travers une surface $S$ fermee est egale a la somme des charges electriques contenues dans le volume $V$ delimite par $S$, divise par la permitivite du vide.}'' 
\end{center}
\[ 
\iint_S \vec{E} \cdot d \vec{S} = \frac{1}{\epsilon_0} \iiint_V \rho_{el} dV
\]
On peut aussi reecrire ceci comme
\[ 
	\frac{1}{\epsilon_0} \iiint_V \rho_{el} dV = \iint_S \vec{E}\cdot d\vec{S} = \iiint_V \nabla \cdot \vec{E} dV
\]
Ce qui est valable pour tout volume $V$. La forme differentielle de la loi de gauss s'ecrit donc
\[ 
	\nabla \cdot \vec{E} = \frac{\rho_{el} }{\epsilon_0}
\]
\begin{exemple}
On suppose que le plan $x-y$  est charge uniformement avec $\sigma_{el}  = \text{ const } >0$
\begin{figure}[H]
    \centering
    \incfig{plan-charge-xy}
   \caption{plan charge xy}
    \label{fig:plan-charge-xy}
\end{figure}
\end{exemple}
On a bien sur que $S= \del V$, et on note $A$ pour l'aire de la section du cylindre.
\begin{align*}
	\iint_S \vec{E}\cdot d \vec{S}  &= \iint_{S_{top} } E_z( z)  \vec{e}_z \cdot \vec{e}_z d S\\
					&+ \iint_{S_{bot} } E_z( -z)  \vec{e}_z \cdot - \vec{e}_z d S\\
					&= E_z( z)  \iint_{S_{top} }  dS + E_z( -z) ( -1) \iint_{S_{bot} } dS\\
					&= A E_z( z) - E_z( -z) A = 2 E_z( z) A
&= \frac{1}{\epsilon_0} \text{ charge totale dans $V$ } \\
&= E_z( z) = \frac{\sigma_{el} }{2 \epsilon_{0}}
\end{align*}
\subsubsection{Preuve de la loi de Gauss}
\subsubsection*{Charge ponctuelle $q$, $S$ une sphere de rayon $r$ autour de $q$}

\begin{figure}[H]
    \centering
    \incfig{surface-avec-charge-ponctuelle}
    \caption{surface avec charge ponctuelle}
    \label{fig:surface-avec-charge-ponctuelle}
\end{figure}
\begin{align*}
\iint_S \vec{E} \cdot d \vec{S} &= \iint_S |\vec{E}| d S\\
&= |\vec{E}| \iint_S d S\\
&= 4 \pi r^{2} |\vec{E}|
\end{align*}
En calculant l'autre cote de la loi de Gauss, on trouve
\begin{align*}
\frac{1}{\epsilon_0} \iiint_V\rho_{el}  d V = \frac{q}{\epsilon_0}
\end{align*}
Et donc 
\[ 
|\vec{E}| = \frac{1}{4 \pi \epsilon_0} \frac{q}{r^{2}}
\]
Ce qui est vrai par la loi de Coulomb.

\subsubsection*{Charge ponctuelle $q$ entouree d'une surface fermee $S$ arbitraire}

\begin{figure}[H]
    \centering
    \incfig{surface-arbitraire,-charge-ponctuelle}
    \caption{surface arbitraire, charge ponctuelle}
    \label{fig:surface-arbitraire,-charge-ponctuelle}
\end{figure}
Le flux a travers $d\vec{S}'$ est donnne par
\[ 
E' dS'
\]
Le flux a travers $d \vec{S}$ est donne par
\[ 
E dS \cos \theta 
\]
En comparant les normes, on trouve que
\[ 
	E = E' \frac{r'^{2}}{r^{2}} ( \text{ loi de Coulomb } ) 
\]
Par geometrie, on a aussi que
\[ 
d S = d S' \frac{r^{2}}{r'^{2}} \frac{1}{\cos \theta}
\]

Le flux a travers $dS$ peut s'ecrire comme
\[ 
E dS \cos \theta = E' \frac{r'^{2}}{r^{2} } dS' \frac{r^{2}}{r'^{2}} \frac{1}{\cos \theta }\cos \theta = E' dS'
\]
Et donc les deux flux sont les memes.\\
Ceci est valable pour chaque element $dS$ et donc le flux de  $E$ a travers $S$ est le meme que le flux de $E$ a travers $S'$ et donc la loi de Gauss est satisfaite dans ce cas.
\subsubsection*{Charge ponctuelle $q$ a l'exterieur d'une surface $S$ fermee arbitraire}
\begin{figure}[H]
    \centering
    \incfig{charge-q-a-l-exterieur}
    \caption{charge q a l exterieur}
    \label{fig:charge-q-a-l-exterieur}
\end{figure}
Pour la preuve on va utiliser
\begin{figure}[H]
    \centering
    \incfig{preuve-charge-a-l-exterieur}
    \caption{preuve charge a l exterieur}
    \label{fig:preuve-charge-a-l-exterieur}
\end{figure}
On a donc que la surface $S_1 + S_2$ est fermee contenant $q$, donc on peut ecrire que
\begin{align*}
	\iint_{S_1+S_2}  \vec{E} \cdot d \vec{S} &=  \frac{q}{\epsilon_0}\\
	&=  \iint_{S_1}  \vec{E} \cdot d \vec{S} + \iint_{S_2}  \vec{E} \cdot d \vec{S}
\end{align*}
Si on laisse tendre $d \to 0$, 

\[ 
\iint_{S_1}  \vec{E} \cdot d \vec{S} = \frac{q}{\epsilon_0}
\]

Et donc
\[ 
\iint_{S_2}  \vec{E} \cdot d \vec{S} =0
\]



Le cas general pour une densite de charge abritraire ( qui est juste une somme de charges ponctuelles) suit du principe de superposition.
\subsection{Le potentiel electrostatique}
En electrostatique, on peut definir une fonction scalaire $\phi( \vec{r}) $ tel que $\vec{E} = - \nabla \phi$. 
On appelle $\phi$ le potentiel elctrostatique.\\
Montrons l'existence de $\phi$.\\
Soit une charge ponctuelle $q_1$ a la position $\vec{r}_1$ :
\[ 
	\phi( \vec{r}) = \frac{1}{4 \pi \epsilon_0} \frac{q_1}{|\vec{r}-\vec{r}_1|} + \text{ constante } 
\]
On calcule
\begin{align*}
- \nabla \phi( \vec{r}) = 
\begin{pmatrix}
	- \frac{\del}{\del x} ( \frac{1}{4 \pi \epsilon_0} \frac{q_1}{ \sqrt{( x-x_1) ^{2} + ( y-y_1) ^{2} + ( z-z_1) ^{2}} }) \\
	- \frac{\del}{\del y} ( \frac{1}{4 \pi \epsilon_0} \frac{q_1}{ \sqrt{( x-x_1) ^{2} + ( y-y_1) ^{2} + ( z-z_1) ^{2}} }) \\
	- \frac{\del}{\del z} ( \frac{1}{4 \pi \epsilon_0} \frac{q_1}{ \sqrt{( x-x_1) ^{2} + ( y-y_1) ^{2} + ( z-z_1) ^{2}} }) 
\end{pmatrix}\\
= 
- \frac{1}{4 \pi \epsilon_0}q_1 ( -\frac{1}{2}) 
\begin{pmatrix}
	\frac{q_1}{ ( \sqrt{( x-x_1) ^{2} + ( y-y_1) ^{2} + ( z-z_1) ^{2}})^{\frac{3}{2}} }  2 ( x-x_1) )\\
	\frac{q_1}{ ( \sqrt{( x-x_1) ^{2} + ( y-y_1) ^{2} + ( z-z_1) ^{2}} })^{\frac{3}{2}} ) 2 ( y-y_1) )\\
	\frac{q_1}{ ( \sqrt{( x-x_1) ^{2} + ( y-y_1) ^{2} + ( z-z_1) ^{2}} } )^{\frac{3}{2}}) 2 ( z-z_1) )
\end{pmatrix}\\
=
\frac{q_2}{4\pi \epsilon_0} \frac{1}{|\vec{r}-\vec{r_z}|^{3}}
\begin{pmatrix}
x-x_1\\
y-y_1\\
z-z_1
\end{pmatrix}
=  \frac{q_1}{4 \pi \epsilon_0} \frac{ \vec{r}-\vec{r_1}}{| \vec{r}-\vec{r}_1|^{3}}
\end{align*}
Dans le cas general, on a
\subsubsection*{Distribution de charges arbitraires}
\[ 
	\phi( \vec{r}) = \frac{1}{4 \pi \epsilon_0} \iiint_{V}  \frac{\rho_{el} ( \vec{r}) }{|\vec{r}-\vec{r}'} dV' + \text{ const. } 
\]
\begin{align*}
	- \nabla \phi( \vec{r})  &= \frac{1}{4 \pi \epsilon_0} \iiint_{V}  \rho_el( \vec{r}) ( -\nabla)  \frac{1}{|\vec{r}-\vec{r}'|} dV'\\
				 &= \frac{1}{4 \pi \epsilon_0} \iiint_{V}  \frac{ \rho_{el} ( \vec{r}) ( \vec{r}-\vec{r}') }{|\vec{r}-\vec{r}'|^{3}}= \vec{E}( \vec{r}) 
\end{align*}
Pour une densite de charges de surface
\[ 
	\phi ( \vec{r})  = \frac{1}{ 4 \pi \epsilon_0} \iint_{S}  \frac{\sigma_{el} ( \vec{r}') }{|\vec{r}-\vec{r}'|} dS' + \text{ const } 
\]
\subsubsection{Quelques proprietes de $\phi$ ( et de $E$) en electrostatique}
\begin{enumerate}
\item Les unites de $\phi$ sont le Volt $V = \frac{kg \cdot m^{2}}{ s ^{2}} \frac{1}{C}= \frac{ \text{ ``energie''  } }{ \text{ ``charge''  } }$.\\
\item $\phi$ est seulement definit a une constante pres ( la constante d'integration.)
\item Le potentiel electrostatique du a differentes charges esst additif.
\item 
	\[ 
		\nabla \times ( \nabla \phi) = 
		\begin{pmatrix}
		\del_x \\ \del_y \\ \del_z
		\end{pmatrix}
		\times
		\begin{pmatrix}
		\del_x \phi \\ \del_y\phi \\ \del_z \phi
		\end{pmatrix}
		=0
	\]
	Et donc
	\[ 
		\nabla \times \vec{E} = - \nabla \times ( \nabla \phi)  = 0
	\]
	
	Par le theoreme de stokes, on a donc
	\[ 
	\oint_{\gamma} \vec{E} \cdot \vec{dl} = \iint_S \nabla\times \vec{E} \cdot \vec{dS} =0
	\]
	
\end{enumerate}



		




\end{document}	
