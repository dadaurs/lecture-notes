\documentclass[../main.tex]{subfiles}
\begin{document}
\lecture{27}{Fri 04 Jun}{Suites ondes electromagnetiques}
Une onde e.m. transporte de l'energie.\\
\[ 
S= \text{ Energie transportee le long de la direction $\vec{k}$ par unite de surface } 
\]
Dans le temps $dt$, la partie de l'onde dans le vollume $Ac dt$ traverse la surface $A$.\\
L'energie electrique et magnetique dans ce volume est donnee par
\begin{align*}
	E_{EM} = \iiint_{V=Ac dt} e_{EM} dV &= \iiint_{V= Ac dt} \frac{1}{2}( \epsilon E^{2}+ \frac{1}{\mu_0}B^{2}) dV\\
	\intertext{Pour $c dt << \lambda$ , l'integrande est quasiconstant}
					    &= \frac{1}{2}( \epsilon_0 E^{2} + \frac{1}{\mu_0}B^{2}) Ac dt\\
	S &= \frac{1}{\mu_0 c}E_{0X} ^{2}\cos^{2}( \omega t - kz + \phi) \text{ pour une onde plane sinusoidale } 
\end{align*}
En general,
$S = \frac{\vec{E}\times \vec{B}}{\mu_0}=$ vecteur de Poynting.	

\end{document}	
