\documentclass[../main.tex]{subfiles}
\begin{document}
\lecture{22}{Tue 18 May}{electromagnetisme}
\subsection{Loi d'Ampere}
Soit un contour $C$ ferme et des conducteurs portant des courants $I_{m} m=1,2,\ldots$\\
Dans ce cas
\begin{align*}
	\oint_{C} \vec{B}\cdot \vec{dl} &= \mu_0 ( \text{ somme des courants inclus dans  } C) 
\end{align*}
\subsubsection{Champ $B$ d'un courant rectiligne}
On considere un fil rectiligne infini portant un courant $I$.\\
L'experience montre que le champ $B$ genere par ce courant est
\begin{itemize}
\item Dans la direction $e_{\theta} $ 
\item Oriente par la regle de la main droite.
\item Norme de $B \propto \frac{I}{r}$
\end{itemize}
En SI:  $\vec{B}( \vec{r}) = \frac{\mu_0}{2 \pi}\frac{I}{R}\vec{e}_{\theta} $.\\
On a
\[ 
\mu_0= 4 \pi 10^{-7}\frac{N}{A^{2}}
\]
\subsubsection{Vers la loi d'Ampere}
Calculons la circulation 
\[ 
\oint_{\gamma} \vec{B} \cdot \vec{dl}	
\]
Pour un fil rectiligne infini.\\
Cas 1:\\
$\gamma$ un cercle de rayon $r$ autour de $I$.\\
On a donc
\begin{align*}
	\oint_{\gamma} \vec{B}\cdot \vec{dl} &= \int_{ 0 }^{ 2\pi }\vec{B}( \vec{r}) e_\theta e_{\theta} r d \theta\\
					     &= B( r) r \int_{ 0 }^{ 2\pi } d \theta\\
					     &= \mu_0 I	
\end{align*}
cas 2:\\
Chemin pas autour de $I$ 
\begin{align*}
	\oint_\gamma \vec{B}\cdot \vec{dl} &= \int_{ \gamma_1 }^{  } \vec{B}\cdot \vec{dl} +\int_{ \gamma_2 }^{  } \vec{B}\cdot \vec{dl} +\int_{ \gamma_3 }^{  } \vec{B}\cdot \vec{dl} +\int_{ \gamma_4 }^{  } \vec{B}\cdot \vec{dl} \\
	&= - \frac{\mu_0I}{2\pi r_1} r_1 \theta_\gamma + 0 + 0 + \frac{\mu_0I}{2\pi r_2	r_2 \theta_\gamma}=0
\end{align*}
On peut montrer que pour des contours fermes arbitraires ne contenant pas de courant, on a bien que l'integrale de contour est nulle.
De plus, comme pour le champ electrique, le principe de superposition s'applique pour le champ $\vec{B}$ ( fait experimental)
\subsubsection{Generalisation de la loi d'ampere integrale et differentielle}
Les regles explicitees ci-dessus restent valables pour des courants non-rectilignes ( sans preuve).\\
Donc, pour des courants ( non-necessairement rectilignes) $I_1, \ldots, I_k$ passsant a travers un contour $\gamma$, on a
\[ 
	\oint_\gamma \vec{B}\cdot \vec{dl} = \mu_0( \sum_i I_i) 
\]
Dans le cas d'une densite de courant $\vec{j}( \vec{r}) $, on a
\[ 
	\oint \vec{B} \cdot \vec{dl} = \mu_0  \iint_S \vec{j}\cdot \vec{dS}
\]
Ou $S$ est la surface delimitee par le chemin ferme $\gamma$.\\
Ou l'orientation de $\vec{dl}$ et $\vec{dS}$ est selon la regle de la main droite.\\
On peut utiliser la loi d'Ampere integrale pour determiner $\vec{B}$ dans des cas simples.
\begin{exemple}[Champ a l'interieur d'une bobine]
	On considere un chemin passant a l'interieur de la bobine
\begin{align*}
\oint_\gamma \vec{B} \cdot \vec{dl} &= Bl\\
&=\mu_0 nlI\\
\Rightarrow B \approx \mu_0nI
\end{align*}

Dans le cas ou $l >>R$,  le fil est mince et que les enroulement sont serres, alors on a egalite dans l'expression ci-dessus.\\
	

\end{exemple}
\subsubsection*{Loi d'Ampere differentielle}
\begin{align*}
\oint_\gamma \vec{B} \cdot \vec{dl} &= \mu_0 \iint_S \vec{j} \cdot \vec{dS}
\intertext{Theoreme de Stokes implique}
\iint_S \nabla \times \vec{B} \cdot \vec{dS} &=\mu_0 \iint_S \vec{j} \cdot \vec{dS}\\
\iint_S ( \nabla \times \vec{B} - \mu_0 \vec{j}  )\cdot \vec{dS}
\intertext{Ceci etant vrai pour toute surface $S,$ on conclut que}
\nabla \times \vec{B} = \mu_0 \vec{j}
\end{align*}
\subsection{Loi de Gauss pour $B$}
Par opposition a l'electrostatique, les lignes du champ $B$ ne sont jamais interrompues.\\
Elles se ferment sur elles meme ou vont de l'infini a l'infini.
Donc il n'y a pas de generation des lignes de champ magnetique a l'interieur d'une surface fermee.\\
Donc,
\[ 
\iint_S \vec{B} \cdot \vec{dS} = 0 \forall S \text{ fermee } 
\]
En appliquant le theoreme de la divergence, on en deduit
\begin{align*}
\iiint_V \nabla \cdot B dV = 0 \forall v\\
\Rightarrow \nabla \vec{B} =0
\end{align*}
\subsection{Loi de Biot-Savart}
Quel est le champ $\vec{B}$ cree par un fil mince de forme arbitraire, parcouru par un courant $I$.\\
\begin{figure}[H]
    \centering
    \incfig{biot-savart}
    \caption{Biot-Savart}
    \label{fig:biot-savart}
\end{figure}
On montre que ( preuve dans Villard III) 
\[ 
	d \vec{B}( \vec{p}) = \frac{\mu_0 I}{4\pi} \frac{\vec{dl} \times \vec{r}}{|r|^{3}}
\]
En integrant, on trouve donc 
\[ 
	\vec{B}( \vec{p})= \frac{\mu_0I}{4 \pi} \int_{ \text{ fil } } \frac{\vec{dl} \times \vec{r}}{|r|^{3}}
\]




\end{document}	
