\documentclass[../main.tex]{subfiles}
\begin{document}
\lecture{7}{Tue 16 Mar}{Theoreme de Bernoulli}
\subsection{Theoreme de Bernoulli}
On considere un ecoulement laminaire au travers d'un tube
\begin{figure}[H]
    \centering
    \incfig{tube-2}
    \caption{tube 2}
\end{figure}
On s'interesse a la pression au travers du tube.\\
On remarque que la pression a diminue dans la partie etroite du tube si $\vec{u} \neq 0$.\\
On explique ce phenome par la theoreme de Bernoulli.\\
On considere un fluide parfait( pas de viscosite), on considere aussi qu'il est incompressible, et qu'il est en ecoulement stationnaire ( toutes les derivees partielles sont egales a 0), dans un champ de pesanteur $\vec{g}$ constant.

\begin{figure}[H]
    \centering
    \incfig{tube-de-courant}
    \caption{tube de courant}
\end{figure}
On considere un tube de courant dans ce fluide et on le suit.\\
A $t= t_0$ : on a une section $ABCD$, en $t= t_0+ \Delta t$ : $A'B'C'D'$. 
On choisint $S_A$ et $\Delta t$ tres petit, donc la distance entre $AB$ et $A'B' \approxeq \vec{u}_A \Delta t$.\\
On utilise maintenant que
\[ 
\text{ Masse de } ABCD = \text{ Masse de  } A'B'C'D', \]

\begin{align*}
	m_{A'B'C'D'} &=m_{ABCD} - S_A u_A \Delta t  \rho + S_C u_c \Delta t \rho\\
S_A u_A \Delta t \rho &= S_C u_c \Delta t \rho
\end{align*}
Donc si la section diminue, la vitesse augmente.
On definit
\[ 
V_{ABA'B'} = S_A u_A \Delta t = V_{CDC'D'} =: \Delta V
\]
Vu qu'on considere un fluide parfait, pour notre tube de courant, on a 
\[ 
\Delta E_{cin}  + \Delta E_{pot}  = \Delta W
\]

Le changement d'energie cinetique est donne par
\[ 
	\Delta E_{cin}  = \frac{1}{2}\rho u_c ^{2} \Delta V - \frac{1}{2}\rho u_A^{2}\Delta V
\]
Et pour l'energie potentielle
\[ 
	\Delta E_{pot}  = \rho \Delta V g z_z - \rho \Delta V z_A g
\]
Le travail $\Delta W$ est du au travail des forces de pression ( $\Delta W = \vec{F} \cdot \Delta \vec{x}$) 

Donc
\[ 
	\Delta W = p_A S_A u_A \Delta t - p_C S_C u_C \Delta t
\]
Donc
\[ 
	\Delta W = ( p_A - p_C ) \Delta V
\]
Donc
\[ 
	\frac{1}{2}\rho u_C^{2} - \frac{1}{2}\rho u_A ^{2} + \rho g z_c - \rho g z_A = p_A - p_C
\]
Et donc
\begin{thm}[Theoreme de Bernoulli]
\[ 
\frac{1}{2}\rho u_A ^{2} + \rho g z_A + p_A = \frac{1}{2}\rho u_C ^{2} + \rho g z_c + p_c
\]
\end{thm}
Ce qui implique que $\frac{1}{2}\rho u^{2} + \rho g z + p$ est constant le long d'une ligne de courant.
\subsection{Applications de Bernoulli}
\begin{figure}[ht]
    \centering
    \incfig{tube-de-pitot}
    \caption{Tube de Pitot}
    \label{fig:tube-de-pitot}
\end{figure}
La difference de pression entre le point 1 et 2 permet de mesure la vitesse.





\end{document}	
