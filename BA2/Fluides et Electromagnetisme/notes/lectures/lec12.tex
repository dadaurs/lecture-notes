\documentclass[../main.tex]{subfiles}
\begin{document}
\lecture{12}{Tue 13 Apr}{Ondes lineaires}
\subsection{Ondes lineaires a la surface d'un fluide parfait}
On considere a nouveau une petite vague a la surface.\\
On defint par $h$ la profondeur de l'eau non perturbee.\\
Si on neglige la viscosite et la tension superficielle et on reste en regime lineaire, on trouve la relation de dispersion
\[ 
	\omega ^{2} = gk \tanh ( kh) 
\]
\subsubsection*{\textbf{ Cas 1 }: Eau profonde}
Si $kh = \frac{2 \pi h}{\lambda} >> 1$, alors chaque point se deplace en "cercle"
\subsubsection*{\textbf{ Cas 1 }: Eau peu profonde}
Si $kh << 1$, les trajectoires sont applaties mais ne deviennent pas negligeable. Alors
\[ 
\frac{\omega }{k} = \sqrt{g h} 
\]
\section{L'electromagnetisme}
On connait quatre types de forces fondamentales
\begin{itemize}
\item L'interaction electromagnetique
\item L'interaction nucleaire forte 
\item L'interaction nucleaire faible
\item L'interaction gravitationelle
\end{itemize}
\subsection*{L'interaction electromagnetique}
\begin{itemize}
\item Attractive ou repulsive
\item Responsable de la plupart des phenomenes quotidiens
\begin{itemize}
\item electricite
\item lumiere
\item structure des atomes et molecules
\item gouverne les reactions chimiques
\end{itemize}
\end{itemize}
\subsection{Electrostatique}
\subsubsection{Charge electrique}
On definit deux types de charges, appelees positives et negatives.\\
Un corps est dit:
\begin{itemize}
\item neutre si les charges positives et negatives s'annullent
\item charge s'il y a un exces d'un type de charge
\end{itemize}
\subsubsection{L'unite de la charge}
On utilisera $q$ pour indiquer la charge, l'unite est le coulomb $C$.\\
Dans le systeme international (SI), le coulomb est une unite "derivee".\\
1 coulomb est donne par un ampere pendant 1 seconde.\\
Il existe une charge elementaire $e$.\\
Un electron a une charge $-e$ et un proton a une chargte $+e$.
Toutes les charges observables sont des multiples de $e$.
\subsubsection*{Conservation de la charge}
La quantite de charge totale est toujours conservee.
\subsection{Loi de Coulomb}
On considere deux charges ponctuelles $q_1$ et $q_2$ et on considere le vecteur $r_{12}$ qui relie les deux points.
La loi de Coulomb donne
\[ 
\vec{F}_{1\to 2}  = \frac{1 }{4 \pi \epsilon_0}\frac{q_1 q_2}{|r_{12}|^{2}} \frac{r_{12}}{| r_{12}|}
\]
Ou encore
\[ 
	\vec{F}_{1\to 2}  = \frac{1 }{4 \pi \epsilon_0}\frac{q_1 q_2}{|r_{12}|^{3}} r_{12}
\]
Par Newton on a bien sur que
\[ 
\vec{F}_{2\to 1} = - \vec{F}_{1\to 2} 
\]
\subsubsection{Principe de superposition}
Pour trois charges $q_1,q_2$ et $q_3$ on a que la force sur $q_3$ en presence de $q_1$ et $q_2$ est donne par
\[ 
F_{res } = F_{1\to 3}  + F_{2 \to 3}  
\]













\end{document}	
