\documentclass[../main.tex]{subfiles}
\begin{document}
\lecture{18}{Tue 04 May}{Energie stockee dans un condensateur}
\subsection{Energie stockee dans un condensateur}
On va essayer de trouver l'expression de la densite d'energie electrique.
\begin{figure}[H]
    \centering
    \incfig{condensateur-arbitraire}
    \caption{deux condensateurs arbitraire}
    \label{fig:condensateur-arbitraire}
\end{figure}
On definit $U = \phi_A - \phi_B$.\\
Le travail pour deplacer une charge $-dq$ du conducteur $A$ au conducteur $B$ est donnee par
\[ 
	dW = ( \phi _B - \phi_A) dq = U dq = \frac{q}{c}dq
\]
Donc le travail pour charger le condensateur a la charge $q$ est:
\[ 
W = \int_{ 0 }^{ q }dW = \int_{ 0 }^{ q } \frac{1}{c} q dq = \frac{q^{2}}{2 C}
\]
Pour un condensateur plan, on a
\[ 
W= \frac{1}{2} c u^{2} = \frac{1}{2} \frac{\epsilon_0 S_c}{d}E^{2} d^{2}= \frac{1}{2} \epsilon_0 E^{2} \underbrace{S_c d}_{ \text{ Volume entre les deux plaques } }
\]
Ce travail est stocke dans le champ $e$, alors
\[ 
e_E = \frac{1}{2}\epsilon_0 E^{2}
\]
est la densite d'energie electrique dans le vide.
\subsection{Capacite avec un dielectrique }
Observation\\
Si on insere un dielectrique entre deux condensateurs plans, $u$ diminue.\\
Donc, car $C= \frac{q}{u}$, la capacite a augmente.
\begin{defn}[Moment dipolaire]
	Le moment dipolaire $\vec{p}$ pour un systeme de charges globalement neutre.\\
Dans le cas des charges ponctuelles
\[ 
\vec{p}= \sum_{i}^{ } \vec{r}_i q_i; \text{ avec } \sum_{i}^{ }q_i =0
\]


\end{defn}
\begin{itemize}
\item $\vec{p}$ ne depend pas de l'origine du systeme de coordonnees, en effet pour deux systemes de coordonnees $O$ et $O'$, on a
	\[ 
		p' = \sum_{i}^{ }r_i' q_i = \sum_{i}^{ }( s+ r_i) q_i  = s \sum_{i}^{ }q_i + \sum_{i}^{ }r_i q_i
	\]
	
\item Dans un champ $E$ uniforme, la force sur le systeme est nulle
	\[ 
	F_{ res } = \sum_{i}^{ }q_i E = E \sum_{i}^{ }q_i =0
	\]

\item Par contre, le moment de force en general est $\neq 0$; $E$ uniforme $ \Rightarrow $
	\[ 
	M_{res} = \sum_{i}^{ }r_i \times F_i = \sum_{j}^{ }r_i \times E q_i = ( \sum_{i}^{ }r_i q_i  )\times E = p \times E
	\]
	
\end{itemize}
			




\end{document}	
