\documentclass[../main.tex]{subfiles}
\begin{document}
\lecture{21}{Fri 14 May}{Magnetostatique}
\section{Magnetostatique}
Situations avec des courants qui dependent du temps.
\subsection*{Observations experimentales}
\begin{enumerate}
	\item Boussole ( aiguille aimantee) est devie par un aimant.
	\item Une boussole est aussi deviee par un fil portant un courant
	\item Un aimant peut exercer une force sur un fil portant un courant
	\item Un fil portant un courant exerce une force sur un autre fil portant un courant.
\end{enumerate}
Ces effets ne sont pas dus aux interactions electrostatiques, car
\begin{enumerate}
\item Les aimants ne sont pas charges; pas d'effet sur un dielectrique.
\item Un aimant influence une boussole a travers une cage faraday.
\item Les fils portant un courant ne sont pas charges.
\end{enumerate}
On a donc affaire a un nouveau type d'interaction $\to $ definition d'un nouveau champ vectoriel $\to $ Champ Magnetique $\vec{B}$.\\
\begin{itemize}
\item $\vec{B}$ est genere par un aimant ou un courant et exerce une force sur un aimant ou un courant.
\item La direction de $\vec{B}$ est definit par l'orientation d'une boussole.
\end{itemize}
\subsection{Definition du champ magnetique et force de Lorentz}
$\vec{B}$ exerce une force sur un fil portant un courant.\\
Est-ce une force sur le conducteur ou sur les charges en mouvement ?\\
$\to$ sur les charges en mouvement.\\
DEs experiences montrent que la force sur une charge $q$ et de vitesse $\vec{v}$ est
\begin{itemize}
\item perpendiculaire a la vitesse $\vec{v}$ et $\vec{B}$.
\item proportionelle a $|\vec{v}|$ et $q$
\end{itemize}
$\to $ $\vec{F}\propto q ( \vec{v}\times \vec{B}) $.\\
En SI,  on definit l anorme de $\vec{B}$ tel qu'on aie egalite.\\
\[ 
[ B] = \frac{N}{C} \frac{S}{m}= T = \text{ Tesla. } 
\]
Donc, la force en presence de $\vec{E}$ et de $\vec{B}$ est
\[ 
	\vec{F}= q ( \vec{E} + \vec{v}\times \vec{B}) 
\]
Donc la foce sur un element de fil $\vec{dl}$ portant un courant $I:$ 
\[ 
	\vec{dF}= nS dl q ( \vec{u}\times \vec{B}) 
\]
Comme $\vec{u} || \vec{dl}$, on peut ecrire $dl \vec{u}=u \vec{dl}$, donc
\[ 
	\vec{dF}= n Suq ( \vec{dl}\times \vec{B}) =I ( \vec{dl}\times \vec{B}) 
\]


\end{document}	
