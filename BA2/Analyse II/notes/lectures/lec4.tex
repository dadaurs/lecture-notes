\documentclass[../main.tex]{subfiles}
\begin{document}
\lecture{4}{Wed 03 Mar}{Boules sur $R^n$}
\subsection{Suites sur $R^{n}$}
\begin{rmq}
	Supposons que $ \left\{ x^{( k) } \right\}  \to \vec{x}$ par rapport a la norme euclidienne. Et oit $||| \cdot |||$ une autre norme sur $ \mathbb{R}^n$. Puisque toutes les normes sont equivalentes sur $\mathbb{R}^{n}$ $|||\vec{x}||| \leq c ||\vec{x}||_2$ Donc toutes les suites converge peu importe la norme.
\end{rmq}
En particulier, on peut choisir la norme infinie.
\begin{lemma}
	Une suite $ \left\{ x^{( k) } \right\} $ converge si et seulement si toutes les composantes convergent.
\end{lemma}
\begin{defn}[Suites de Cauchy]
	On dit qu'une suite $ \left\{ x^{( k) } \right\} $ est de Cauchy si
	\[ 
		\forall \epsilon>0 \exists N>0: \forall k,l \geq N \N{x^{( k) } - x^{( l) }} \leq \epsilon
	\]
	
	
\end{defn}
\begin{thm}
	Une suite converge si et seulement si elle est de Cauchy.
\end{thm}
\begin{proof}
	Si la suite $x^{( k )}$ converge $\iff \left\{ x^{( k) }_i \right\} $ converge pour tout $i=1,\ldots,n$ donc toutes ces suites sont de Cauchy et donc $x^{( k) }$ converge.
\end{proof}
\begin{thm}[Bolzano-Weierstrass]
	Soit $ \left\{ x^{( k) } \right\} $ une suite bornee.\\
	Alors il existe une sous-suite $ \left\{ x^{( k_j) } \right\} $ qui converge
\end{thm}
\begin{proof}
	Si $ \left\{ x^{( k) } \right\} $ est bornee, en particulier chaque suite $x^{( k)_i}$ sera bornee.\\
	En $i=1$, la suite $x^{( k) }$ est bornee, donc il existe une sous-suite convergente vers une valeur $x_1$.\\
	On considere les index de cette sous-suite et on reapplique l'argument ci-dessus en $i=2$, etc.
\end{proof}

\subsection{Topologie de $ \mathbb{R}^n$}
\begin{defn}[Boule]\label{defn:Bouleboule}
	Pour tout $x \in \mathbb{R}^n$ et $\delta>0$, la boule ouverte centree en $x$ et de rayon $\delta$ 
	\[ 
		B( x,\delta) = \left\{  y \in \mathbb{R}^n : \N{y-x} < \delta \right\} 
	\]
	La boule fermee 
	\[ 
		\overline{B}( x,\delta) = \left\{  y \in \mathbb{R}^n : \N{y-x} \leq \delta \right\} 
	\]
	La sphere centree en $x$ et de rayon  $\delta$
	\[ 
		S( x,\delta) = \left\{  y \in \mathbb{R}^n : \N{y-x} = \delta \right\} 
	\]
\end{defn}
\subsection{Classification des points d'un ensemble $E \subset \mathbb{R}^n$}
Le complementaire de $E$ est
\[ 
E^{c}= \left\{ y \in \mathbb{R}^n , y \notin E \right\} 
\]
On dit que $x$ est un point interieur de $E$ si $\exists \delta: B( x,\delta) \subset E$,
on dit que $x$ est un point frontiere de $E$ si $\forall \delta B( x,\delta) \cap E\neq \emptyset  $ et $B( x,\delta) \cap E^{c} \neq \emptyset $
On dit que $E^{o}$ est l'ensemble des points interieurs de $E$, $E^{o}$ est appele l'interieur de $E$.

On note $\del E $ l'ensemble des points frontieres, appele la frontiere ou le bord de $E$.\\
On dit que $x$ est un point adherent de $E$ si $\forall \delta >0 ,B( x,\delta) \cap E \neq \emptyset $
On note $ \bar E$ l'ensemble des points adherents de $E$, appele l'adherence de $E$.\\
On a $\bar E = E \cup \del E$ \\
On dit que $x$ est un point isole si 
\[ 
	\exists \delta>0 B( x,\delta) \cap E = \left\{ x \right\} 
\]
On dit que $x$ est un point d'accumulation de $E$, si  $\forall \delta>0$ 
\[ 
	B( x,\delta) \cap ( E\setminus \left\{ x \right\} ) \neq \emptyset 
\]
Donc, en particulier, si on prend $ \delta = \frac{1}{k},k \in \mathbb{N}$
\[ 
	\exists x^{( k) } \in E, \text{ tel que } \N{x^{( k) }-x} \leq \frac{1}{k}
\]
La suite $x^{( k) }$ converge vers $x$.
\begin{defn}
	Soit $E$ un ensemble de $ \mathbb{R}^n$, on dit que $E$ est ouvert si tous ses points sont interieurs
	
\end{defn}
\begin{defn}
	$E$ est ferme si $E^{c}$ est ouvert.
	
\end{defn}



\end{document}	
