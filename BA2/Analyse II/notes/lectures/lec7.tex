\documentclass[../main.tex]{subfiles}
\begin{document}
\lecture{7}{Mon 15 Mar}{Prolongement par continuite}
\subsection{Prolongement par continuite}
\begin{defn}[Prolongement par continuite]
	Soit $f: E\subset \mathbb{R}^n \to \mathbb{R}^m$ continue, avec $E \neq \overline{E}$, soit $x_0 \in \overline{E}\setminus E$. Une fonction $\tilde f: E\cup \left\{ x_0 \right\} \to \mathbb{R}^m$ est appellee un prolongement si
	$\tilde f$ est continue en $x_0$ et coincide avec $f$ sur $E$.
	
\end{defn}
Le prolongement par continuite est uniquement defini par 
$\tilde f (x) =f( x) $ si $x \in E$ et $\tilde f ( x_{0})= \lim_{x \to x_0} f( x)  $ si la limite existe.

\begin{thm}[Prolongement par continuite sur l'adherence]
	Soit $E\subset \mathbb{R}^n$ non vide et $f: E \to \mathbb{R}^n$ continue sur $E$. Supposons que $\forall x \in \overline{E}\setminus E$ la limite $\lim_{y \to  x} f( y)  $ existe. Alors on peut definir un prolongement $\tilde f: \overline{E} \to \mathbb{R}^m$, $\tilde f( x) =f( x)\forall x \in E$ et $\tilde f ( x) = \lim_{y \to x} f( y) $ sinon, de plus $\tilde f $ est continue sur $\overline{E}$.
 
	
\end{thm}
\begin{proof}
	Si $x \in E$, $f( x) $ est continue en $x$ donc $\tilde f( x) =f( x) $ est continue en $x$.
	On a
	\[ 
		\tilde f ( x) = \lim_{y \to x, y \in E} f( y) = \lim_{y \to x, y \in E} \tilde f( y) 
	\]
	Pour montrer que $\tilde f$ est continue en $x$, il faut montrer que $\tilde f( x) = \lim_{y \to x, y \in \overline{E}} \tilde f ( y) $\\
	Il faut montrer que pour toute suite $x^{(k ) } \subset \overline{E}$ convergeant en $x \in \overline{E}\setminus E$ on a
	\[ 
		\lim_{k \to  + \infty} \tilde f( x^{( k) }) = \tilde f ( x) 	
	\]
	On construit une deuxieme suite $y^{( k) }$ convergent vers $x$.\\
	Si $x^{( k) } \in E$, alors $y^{( k) }= x^{( k) }$.\\
	Si $x^{( k) } \in \overline{E}\setminus E$ on peut toujours trouver une valeur $y^{( k) }\in E$ tel que $\N{y^{( k) -x^{( k) }}}\leq 2^{-k}$, $\N{f( y^{( k) }- \tilde f ( x^{( k) }) )} \leq 2^{-k}$.\\
	On aura donc
	\[ 
		\N{y^{( k) }-x} \leq \N{y^{( k) }-x^{( k) }} + \N{x^{( k) }-x}
	\]
	Ainsi $y^{( k) }\subset E$ converge vers $x$, et ainsi 
	\[ 
		\lim_{k \to  + \infty} \tilde f ( x^{( k) }) = \lim_{k \to  + \infty}  ( \tilde f ( x^{k}) -  \tilde f ( y^{k}) ) + \lim_{k \to  + \infty} \tilde f ( y^{( k )}) = \lim_{k \to  + \infty} \tilde f( y^{( k) }) 
	\]
	
	
	
\end{proof}
\begin{thm}
	Soit $E\subset \mathbb{R}^n$ non vide $f : E \to \mathbb{R}^n$ uniformement continue.
	Alors $f$ peut etre prolongee par continuite sur $\overline{E}$ et le prolongement $\tilde f :\overline{E}\to \mathbb{R}^m$ est uniformement continu.
	
\end{thm}
\begin{defn}
Soit $E \subset \mathbb{R}^n$ non vide, $f: E \to \mathbb{R} $
Si $\sup f= \infty$  on dit que $f $ n'est pas bornee superieurement.\\
Si $M< \infty $ on appelle $M$ la borne superieure de $f$.\\
S'il existe $x_M \in E, f( x_M) = M$ alors on dit que $M$ est le maximum de $f$ sur $E$ et $x_M $ est un point maximum de $f$.
Meme definition pour borne inferieure.

\end{defn}
\begin{thm}
Soit $E$ non vide et compact, $f: E \to \mathbb{R}$ continue. Alors $f$ atteint son maximum et son minimum sur $E$. 
\end{thm}
\begin{proof}
	Par l'absurde $f$ n'est pas bornee, il existe $x^{( k) }$ tel que $|f( x^{( k) })| >k$ \\
	Mais $E$ est compact, donc il existe une sous-suite $x^{( k_i) }$ qui converge, or $f$ est continue, donc
	\[ 
		\lim_{i \to  + \infty} f( x^{( k_i) }) = f( x) < \infty \contra
	\]
	
Supposons que $f$ n'atteint pas ses bornes\\
Il existe $x^{( k) }$ qui converge vers le sup, or $E$ est ferme.
\end{proof}
\begin{thm}
Soit $E \subset \mathbb{R}^n$ non vide, compact, connexe par arcs, et $f: E \to \mathbb{R}$ continue. Alors $f$ atteint toutes les valeurs entre son minimum et maximum.
\end{thm}
\begin{proof}
$f$ est continue sur un compact donc $f$ atteint son min et son max.\\
Puisque $E$ est connexe, il existe  $\gamma$ un chemin du minimum au maximum.  On conclut par TVI sur la fonctionm $f \circ \gamma$
\end{proof}
\begin{thm}
Soit $E  \subset \mathbb{R}^n$ non vide et compact avec $f: E \to \mathbb{R}^m$ continue. Alors $f$ est uniformement continue sur $E$.
\end{thm}







\end{document}	
