\documentclass[../main.tex]{subfiles}
\begin{document}
\lecture{20}{Mon 10 May}{Proprietes de l'integrale de Riemann}
\subsection{Integrabilite sur un domaine quelconque}

\begin{defn}
	Soit $E \subset \mathbb{R}^n$ borne et $f: E \to \mathbb{R}$ bornee.\\
	Soit $ R \subset \mathbb{R}^n$ un pave contenant $E$ et $\tilde f: R \to \mathbb{R}$ le prolongement de $f$ par au dehors de $E$,
	\[ 
		\tilde f( x) = f( x) , x \in E, \tilde f( x) =0 si x \in R \setminus E
	\]
	
	On dit que $f$ est integrable au sens de Riemann sur $E$, si $\tilde f \in \mathcal{R}( R) $ ( est Riemann-integrable sur $R$) .\\
	Dans ce cas, on note
	\[ 
		\int_E f( x) dx = \int _R \tilde f ( x) dx
	\]
	
\end{defn}
\begin{rmq}
Cette definition de depend pas du choix de $R$.
\end{rmq}
\subsection{Proprietes de l'integrale de Riemann}
\begin{itemize}
	\item Linearite: $\forall f,g \in \mathcal{R}( E) , \forall \alpha, \beta \in \mathbb{R} $
	\[ 
		\int_E ( \alpha f+ \beta g) = \alpha\int_E f + \beta\int_E g
	\]
	Il s'ensuit que $ \mathcal{R}( E) $ est un espace vectoriel
\item Monotonie: $\forall f, g \in \mathcal{R}( E) $, si $f( x) \leq g( x) \forall x \in E,$ alors
	\[ 
		\int_E f( x) dx \leq \int_E g( x) dx
	\]
	
\item Si $f \in \mathcal{R}( E) $, alors $ | f| \in \mathcal{R}( E) , f_+ = \max \left\{ f, 0 \right\} \in \mathcal{R}( E) , f_- = \max \left\{ -f, 0 \right\} \in \mathcal{R}( E) $ 
		On montre d'abord que $f \in \mathcal{R}( E) \Rightarrow  f_+ \in \mathcal{R}( E) $.\\
		$f \in \mathcal{R}( E) \Rightarrow \exists R \subset \mathbb{R}^n$ contenant $E$ et $\tilde f \in \mathcal{R}( R) $.\\
		Donc $\forall \epsilon ,$ il existe une partition $P_\epsilon$ de $R$ tel que
		\[ 
			\overline { S} ( \tilde f, P_\epsilon) - \underline { S} ( \tilde f, P_\epsilon) <\epsilon
		\]
		\begin{align*}
			\forall Q \in P_e \text{ on a  } \sup_Q \tilde f_+ - \inf_Q \tilde f_+ \leq  \sup_Q \tilde f - \inf_Q \tilde f	\\
		\end{align*}
		Si $\sup \tilde f \geq \inf_Q \tilde f \geq 0$, alors $\tilde f_+ = f $ sur $Q$, et on a egalite.\\
		Si $\inf_Q \tilde f \leq \sup_Q \tilde f \leq 0$, alors $\tilde f_+ =0$, et on a l'inegalite.\\
		Si $\sup_Q \tilde f \geq 0 \geq \inf_Q \tilde f$, alors $\sup_Q \tilde f_+ - \underbrace{\inf_Q \tilde f_+}_{=0} = \sup_Q \tilde f \leq  \sup_Q \tilde f - \inf_Q \tilde f$\\
		Ce qui montre l'inegalite, et ce qui implique que $\tilde f _+$ est integrable.\\
		Mais alors $f_-$ est integrable et $|f| = f_+ - f_ \in \mathcal{R}( E) $
	
	
	
\item Si $f \in \mathcal{R}( E)$, alors $| \int_E f( x) dx | \leq  \int_E |f( x) | dx$ 
	En effet, on a 
	\begin{align*}
	f \in \mathcal{R}( E) \Rightarrow |f| \in \mathcal{R}( E) , \text{ de plus } \\
	f( x) \leq  |f( x) | \Rightarrow  \int_E f( x) dx \leq  \int_E |f( x) | dx	\\
	- f( x) \leq  |f| \forall x \in E\Rightarrow  -\int_E f( x) dx \leq  \int_E |f( x) | dx
	\end{align*}
\item Si $f,g \in \mathcal{R}( E) $, alors $fg \in \mathcal{R}( E) $
	Si $f,g \in \mathcal{R}( E) , f,g$ sont bornes.\\
	Soit $M \geq 0: f( x) \leq  M, g( x) \leq  M\forall x \in E$, alors $\forall \epsilon>0$, il existe un pave $R \subset \mathbb{R}^n$ contenant $E$ et une partition $P_\epsilon$ de $R$ tel que
	\begin{align*}
		\overline{S}( f,P_\epsilon) - \underline { S} ( f,P_\epsilon) &\leq  \frac{\epsilon}{2M}\\
	\overline{S}( g,P_\epsilon) - \underline { S} ( g,P_\epsilon) &\leq  \frac{\epsilon}{2M}, 	
	\end{align*}
	Maintenant, $\forall Q \in P_\epsilon$, alors
	\begin{align*}
		\sup Q fg - \inf_Q fg &\leq  \sup_Q f \sup_Q g - \inf_Q \inf_g\\
				      & \leq \underbrace{\sup f}_{ \leq M} ( \sup g - \inf g) + \underbrace{\inf g}_{ \leq M} ( \sup f - \inf f) 
	\end{align*}
\end{itemize}
\subsection{Ensembles mesurables au sens de Jordan}
\begin{defn}[Ensemble mesurable au sens de Jordan]
	On dit que $E \subset \mathbb{R}^n$ borne est mesurable au sens de Jordan ( ou Jordan-mesurable) si la fonction $ \mathbb{I}_E: E \to \mathbb{R}$, $ \mathbb{I}_E ( x) =1 \forall x \in E$ est integrable sur $E$ au sens de Riemann.\\
	Dans ce cas, on pose $\vol( E) = \int_E \mathbb{I}_E( x) dx$.\\
	On dit que $E$ est negligeable si $E$ est Jordan mesurable et $\vol ( E) =0 $.
\end{defn}
\subsubsection{Caracterisation des ensembles}
\subsubsection*{Ensembles mesurables}

$E$ mesurable $\iff $ $\int_E \mathbb{I}_E ( x) dx $ existe $ \mathbb{I}_E \in \mathcal{R}( E) $, donc, $\forall \epsilon>0,$ il existe un pave $R \subset \mathbb{R}^n$ contenant $E$ et une partition $P_\epsilon$ de $R$ tel que
\begin{align*}
	\overline{S}( \mathbb{I}_E, P_\epsilon) - \underline { S}  ( \mathbb{I}_E, P_\epsilon) &<\epsilon\\
\overline{S}( \mathbb{I}_E, P_\epsilon) - \underline { S}  ( \mathbb{I}_E, P_\epsilon) &= \sum_{Q \in P_\epsilon} ( \sup_Q \mathbb{I}_E- \inf_Q \mathbb{I}_E) Vol( Q) \\
										       &= \sum_{ \substack { Q \in P_\epsilon\\ Q \cap E \neq \emptyset \\ Q \cap R\setminus E \neq \emptyset} } Vol( Q)
\end{align*}
\begin{lemma}
Soit $e \subset \mathbb{R}^n$ borne et $R \subset \mathbb{R}^n$ un pave contenant.\\
Alors $E$ est mesurable au sens de Jordan si et seulement si $\forall \epsilon>0$, il existe une partition $P_\epsilon$ de  $R$ tel que
\[ 
	\sum_{ \substack { Q \in P_\epsilon\\ Q \cap E \neq \emptyset \\ Q \cap R\setminus E \neq \emptyset} } Vol( Q) < \epsilon
\]

\end{lemma}
\subsubsection*{Ensembles negligeables}
\begin{lemma}
	Soit $E \subset \mathbb{R}^n$ borne, $E$ est negligeable si et seulement si $\forall \epsilon>0$, il existe $K \in \mathbb{N}^{*}$ et une collection de paves tel que $E \subset \bigcup_{i=1} ^{L}Q_i$ et $\sum_{i=1} ^{L}\vol( Q_i) \leq  \epsilon$
\end{lemma}
\begin{proof}
	$E$ mesurable $ \Rightarrow $ $\exists \left\{ Q_1, \ldots, Q_L \right\} : E \subset \bigcup Q_i, \sum \vol ( Q_i) < \epsilon $\\
	Donc $E$ mesurable $\iff$ $ \mathbb{I}_E \in \mathcal{R}( E) $ et  $\int_E \mathbb{I}_E	dx =0$.\\
	Soit $R$ un pave contenant $E$, $\forall \epsilon>0, \exists P_\epsilon$ de $R$.\\
	\begin{align*}
		\overline{S} (  \mathbb{I}_E, P_\epsilon) - \underline { S} ( \mathbb{I}_E,P_\epsilon)  \leq \epsilon
	\end{align*}
	Donc, il faut que $\mathbb{S} ( \mathbb{I}_E, P_\epsilon)< \epsilon $
	\[ 
		\mathbb{S}(  \mathbb{I}_E, P_\epsilon) 	= \sum_{ Q \cap E\neq \emptyset}  \vol Q < \epsilon
	\]
	Supposons que $\forall \epsilon >0$, il existe une collection $ \left\{ Q_1, \ldots, Q_L \right\} $ telle que $E \subset \bigcup Q_i, \sum_i \vol ( Q_i) < \epsilon$\\
	Il existe toujours une partition tensorielle $P$ tel que
	\[ 
	Q_i = \bigcup _{\substack {  Q \in P\\ Q \in Q_i} } Q
	\]
	\begin{align*}
		\underline { S} ( \mathbb{I}_E, P) \geq 0\\
		\mathbb{S}(  \mathbb{I}_E, P) = \sum_{\substack{ Q \in P\\Q \cap E \neq \emptyset }} \vol Q = \sum_{i=1}^{ L} \sum _{ \substack { Q \in P_\epsilon\\ Q \subset Q_i} } \vol ( Q) < \epsilon
	\end{align*}
Donc
\begin{align*}
	\overline{S}( \mathbb{I}_E, P) - \underline { S} ( \mathbb{I}_E, P) < \epsilon\\
	\overline{S}( \mathbb{I}_E, P)  < \epsilon
	\Rightarrow 
\end{align*}

	
\end{proof}
\begin{thm}
Un ensemble borne $E \subset \mathbb{R}^n$ est mesurable si et seulement si $\del E$ est negligeable.
\end{thm}
\begin{crly}
Soient $E, F \subset \mathbb{R}^n$ borne et mesurables, alors
\begin{itemize}
\item $E \cap F, E \cup F, E \setminus F, \ded E, \overline{E}$ sont mesurables.
\end{itemize}

\end{crly}
\begin{proof}
\begin{itemize}
\item $\mathbb{I}_{E \cap F} = \mathbb{I}_E \mathbb{I}_F$, idem pour le reste.
\end{itemize}

\end{proof}








\end{document}	
