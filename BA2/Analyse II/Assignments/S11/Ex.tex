\documentclass[11pt, a4paper]{article}
\usepackage[utf8]{inputenc}
\usepackage[T1]{fontenc}
\usepackage[francais]{babel}
\usepackage{lmodern}
\usepackage{amsmath}
\usepackage{amssymb}
\usepackage{amsthm}
\renewcommand{\vec}[1]{\overrightarrow{#1}}
\newcommand{\del}{\partial}
\DeclareMathOperator*{\sgn}{sgn}
\DeclareMathOperator*{\id}{Id}
\DeclareMathOperator*{\im}{Im}
\DeclareMathOperator*{\re}{Re}
\DeclareMathOperator*{\vol}{Vol}
\newcommand\norm[1]{\left\vert#1\right\vert}
\newcommand\ns[1]{\left\vert\left\vert\left\vert#1\right\vert\right\vert\right\vert}
\newcommand\Norm[1]{\left\lVert#1\right\rVert}
\newcommand\N[1]{\left\lVert#1\right\rVert}
\newcommand\abs[1]{\left\vert#1\right\vert}
\newcommand\inj{\hookrightarrow}
\newcommand\surj{\twoheadrightarrow}
\newcommand\ded[1]{\overset{\circ}{#1}}
\newcommand\sidenote[1]{\footnote{#1}}
\newcommand\eng[1]{\left\langle#1\right\rangle}
\newcommand\hr{
    \noindent\rule[0.5ex]{\linewidth}{0.5pt}
}

\newcommand{\incfig}[1]{%
    \def\svgwidth{\columnwidth}
    \import{./figures}{#1.pdf_tex}
}
\newcommand{\filler}[1][10]%
{   \foreach \x in {1,...,#1}
    {   test 
    }
}

\newcommand\contra{\scalebox{1.5}{$\lightning$}}
\makeatother
\def\@lecture{}%
\newcommand{\lecture}[3]{
    \ifthenelse{\isempty{#3}}{%
        \def\@lecture{Lecture #1}%
    }{%
        \def\@lecture{Lecture #1: #3}%
    }%
    \subsection*{\@lecture}
    \marginpar{\small\textsf{\mbox{#2}}}
}

\begin{document}
\title{Série 11}
\author{David Wiedemann}
\maketitle
Pour simplifier la notation, pour $f: \mathbb{R}\to \mathbb{R}$ et $a,b \in \mathbb{R}$, on définit
\[ 
	\mathcal{G}( f, [ a,b] ) = \left\{ ( x, f( x) ) : a \leq x \leq b  \right\} 
\]
\hr
Soit $\epsilon>0$ .\\
On va montrer que le bord de l'ensemble $A$ est négligeable.\\
Pour montrer celà, on va montrer qu'il existe une collection finie de pavés de $R_1, \ldots, R_k$, tel que
\[ 
\del A \subset \bigcup_{i=1}^{k} R_k \text{ et que } \vol (  \bigcup_{i=1} ^{k}R_k ) < \epsilon.
\]
Posons d'abord $R_1 = [ 0, \frac{\epsilon}{12}] \times [ 0,3] $.\\
Car $2 + \sin ( \frac{1}{x} )$ est continue sur $[\frac{\epsilon}{12}, 1]$, par un exercice ( Série 21, exercice 1) , on sait qu'il existe un ensemble de pavés $A_1, \ldots , A_j$, tel que $\vol \left(   \bigcup_{i=1} ^{j}A_i \right) < \frac{\epsilon}{4}$ et $ \mathcal{G}\left( 2 + \sin ( \frac{1}{x}) , [ \frac{\epsilon}{2}, 1] \right) \subset \bigcup_{i=1} ^{j}A_i $.\\
On définit alors pour  $1< i \leq j+1$, $R_i = A_{i-1} $.\\
De plus, on définit encore
\begin{align*}
	R_{j+2} &= \left[ 1- \frac{\epsilon}{8( 2 + \sin ( 1) )}, 1+ \frac{\epsilon}{8 ( 2 + \sin ( 1) ) }\right] \times [ 0, 2 + \sin ( 1) ] \\
	R_{j+3} &= [ 0, 1] \times [ - \frac{\epsilon}{8}, \frac{\epsilon}{8}]\\
\end{align*}
Faisons d'abord l'observation que $\del A \subset \bigcup_{i=1} ^{j+3}R_i$.\\
De plus, 
\[ 
	\vol ( \bigcup_{i=1} ^{j+3} R_i) \leq \vol ( R_1) + \vol ( \bigcup_{i=1} ^{j}A_i) + \vol ( R_{j+2} ) + \vol ( R_{j+3} ) < \frac{\epsilon}{4} +  \frac{\epsilon}{4} +  \frac{\epsilon}{4} +  \frac{\epsilon}{4} =\epsilon.
\]
Et ainsi, le bord de $A$ est négligeable, ce qui implique que $A$ est mesurable au sens de Jordan.
	




\end{document}
