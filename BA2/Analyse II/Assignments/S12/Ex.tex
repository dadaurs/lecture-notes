\documentclass[11pt, a4paper]{article}
\usepackage[utf8]{inputenc}
\usepackage[T1]{fontenc}
\usepackage[francais]{babel}
\usepackage{lmodern}
\usepackage{amsmath}
\usepackage{amssymb}
\usepackage{amsthm}
\renewcommand{\vec}[1]{\overrightarrow{#1}}
\newcommand{\del}{\partial}
\DeclareMathOperator*{\sgn}{sgn}
\DeclareMathOperator*{\id}{Id}
\DeclareMathOperator*{\im}{Im}
\DeclareMathOperator*{\re}{Re}
\DeclareMathOperator*{\vol}{Vol}
\newcommand\norm[1]{\left\vert#1\right\vert}
\newcommand\ns[1]{\left\vert\left\vert\left\vert#1\right\vert\right\vert\right\vert}
\newcommand\Norm[1]{\left\lVert#1\right\rVert}
\newcommand\N[1]{\left\lVert#1\right\rVert}
\newcommand\abs[1]{\left\vert#1\right\vert}
\newcommand\inj{\hookrightarrow}
\newcommand\surj{\twoheadrightarrow}
\newcommand\ded[1]{\overset{\circ}{#1}}
\newcommand\sidenote[1]{\footnote{#1}}
\newcommand\eng[1]{\left\langle#1\right\rangle}
\newcommand\hr{
    \noindent\rule[0.5ex]{\linewidth}{0.5pt}
}

\newcommand{\incfig}[1]{%
    \def\svgwidth{\columnwidth}
    \import{./figures}{#1.pdf_tex}
}
\newcommand{\filler}[1][10]%
{   \foreach \x in {1,...,#1}
    {   test 
    }
}

\newcommand\contra{\scalebox{1.5}{$\lightning$}}
\makeatother
\def\@lecture{}%
\newcommand{\lecture}[3]{
    \ifthenelse{\isempty{#3}}{%
        \def\@lecture{Lecture #1}%
    }{%
        \def\@lecture{Lecture #1: #3}%
    }%
    \subsection*{\@lecture}
    \marginpar{\small\textsf{\mbox{#2}}}
}

\begin{document}
\title{Série 12}
\author{David Wiedemann}
\maketitle
Il est clair que $u \in C^{1}( ] t_0, + \infty[  ) $ car $u'( t) = f( t, u( t) ) $ et donc, car $f$ est continue sur $I \times \mathbb{R}$, et donc $u$ est continument différentiable sur $]t_0, + \infty [  $.\\
Il nous faut donc uniquement montrer que $ u'( t_0) = \lim_{t \to t_0 +} u'( t) $, ie. que $u'$ est continue à droite en $t_0$.\\
Définissons 
\[ 
	F( t) = \int_{ t_0 }^{ t }u'( s) ds
\]
Le théorème fondamental de l'analyse donne que
\[ 
	F( t) = u( t) - u( t_0) .
\]
Ainsi, on sait que $ \int_{ t_0 }^{ t }u'( s) ds$ existe bien pour tout $t$.\\
Or, car pour $c \in  I$ de $u'$ sur $[t_0, c] $ est égal à l'intégrale de $u'$ sur $]t_0,c]$, on a
\[ 
	F( t) = \int_{ t_0 }^{ t }f( s,u( s) ) ds.
\]
Ainsi, par le théorème fondamental de l'analyse, on déduit également que
\[ 
	F'( t) = f( t,u( t) ) \forall t \in I
\]
Ecrivons maintenant le développement de Taylor de $F( t) $ au point $t_0$.\\
Soit $x \in I \setminus \left\{ t_0 \right\} $.
\begin{align*}
	F(x)= u( x) - u( t_0) &= F( t_0) + f( t, u( t) )  ( x- t_0 ) + r( x) \\
\intertext{où  $ \lim_{x \to t_0} \frac{r( x) }{x-t_0}= 0$.
	 Notons qu'on a  $F( t_0) = 0$.}
	\Rightarrow \frac{u( x) - u( t_0)}{x-t_0} &= f( t_0,u( t_0) ) + \frac{r( x) }{x-t_0}
\end{align*}
On peut maintenant prendre la limite à droite $x \to t_0$ des deux côtés de l'égalité.\\
L'égalité de droite devient.
\[ 
	\lim_{x \to t_0+ }  f( t_0, u( t_0) )  + \frac{r( x) }{x-t_0} = f( t_0, u( t_0) ) 
\]
Tandis qu'à gauche, on a 
\[ 
	\lim_{x \to t_0 +}  \frac{u( x) - u( t_0) }{ x-t_0} = u'_+( t_0).
\]
Et on en déduit l'égalité
\[ 
	u'_+ ( t_0) = u'( t_0) = f( t_0,u( t_0) ) 
\]
Ainsi, $u'$ est continue sur $I$, et donc $u \in C^{1}( I) $.






\end{document}
