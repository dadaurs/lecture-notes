\documentclass[11pt, a4paper]{article}
\usepackage[utf8]{inputenc}
\usepackage[T1]{fontenc}
\usepackage[francais]{babel}
\usepackage{lmodern}
\usepackage{amsmath}
\usepackage{amssymb}
\usepackage{amsthm}

\usepackage{mathtools}


\renewcommand{\vec}[1]{\overrightarrow{#1}}
\newcommand{\del}{\partial}
\DeclareMathOperator*{\sgn}{sgn}
\DeclareMathOperator*{\id}{Id}
\DeclareMathOperator*{\im}{Im}
\DeclareMathOperator*{\re}{Re}
\DeclareMathOperator*{\vol}{Vol}
\newcommand\norm[1]{\left\vert#1\right\vert}
\newcommand\ns[1]{\left\vert\left\vert\left\vert#1\right\vert\right\vert\right\vert}
\newcommand\Norm[1]{\left\lVert#1\right\rVert}
\newcommand\N[1]{\left\lVert#1\right\rVert}
\newcommand\abs[1]{\left\vert#1\right\vert}
\newcommand\inj{\hookrightarrow}
\newcommand\surj{\twoheadrightarrow}
\newcommand\ded[1]{\overset{\circ}{#1}}
\newcommand\sidenote[1]{\footnote{#1}}
\newcommand\eng[1]{\left\langle#1\right\rangle}
\newcommand\hr{
    \noindent\rule[0.5ex]{\linewidth}{0.5pt}
}

\newcommand{\incfig}[1]{%
    \def\svgwidth{\columnwidth}
    \import{./figures}{#1.pdf_tex}
}
\newcommand{\filler}[1][10]%
{   \foreach \x in {1,...,#1}
    {   test 
    }
}

\newcommand\contra{\scalebox{1.5}{$\lightning$}}
\makeatother
\def\@lecture{}%
\newcommand{\lecture}[3]{
    \ifthenelse{\isempty{#3}}{%
        \def\@lecture{Lecture #1}%
    }{%
        \def\@lecture{Lecture #1: #3}%
    }%
    \subsection*{\@lecture}
    \marginpar{\small\textsf{\mbox{#2}}}
}

\begin{document}
\title{Série 8}
\author{David Wiedemann}
\maketitle
\section*{1}
Il suffit de montrer que 
\[ 
	\nabla f( a)  \cdot ( \gamma'( 0) ) = 0
\]
En effet, car l'image de $\gamma$ est un sous-ensemble de  $\Sigma$, on a
\[ 
	f(\gamma( x)  )  =0
\]
En dérivant par rapport à $x$, on trouve
\begin{align*}
	f'( \gamma( x) ) \cdot \gamma'( x) &=0\\
	\intertext{En substituant $x=0$, on trouve }
	\nabla f( a)  \cdot \gamma'( 0) &= 0
\end{align*}
Ce qui montre l'égalité.
\section*{2}
On montre la double inclusion des ensembles.\\
\hr
\textbf{ $ T_a( \Sigma) \subset \Pi_a( \Sigma) - a$ }\\
Cette inclusion est immédiate par la partie 1, en effet, soit $z \in \mathbb{R}^n$ tel qu'il existe un chemin $\gamma:]-1,1[ \to \Sigma$ satisfaisant $\gamma( 0) = a$ et $\gamma'( 0) = z$.\\
Alors la partie 1 implique immédiatement que $\gamma'( 0) \in \Pi_a( \Sigma) -a$ et donc que $z \in \Pi_a( \Sigma) -a$.\\
Ainsi, l'inclusion est démontrée.\\
\hr
\textbf{ $ T_a( \Sigma) \supset \Pi_a( \Sigma) - a$ }\\
Pour alléger la notation, dans ce qui suit on utilisera la notation $ \del_i f( c) \coloneqq \frac{\del f}{\del x_i}( c) $.\\
Soit $z= ( z_1,\ldots,z_n) \in \Pi_a( \Sigma) -a$.\\
Sans perte de généralité, on va supposer que $\del_n f ( a) \neq 0$, dans le cas contraire, il suffirait d'échanger les indexes.\\
Ainsi, par le théorème de la fonction implicite, il existe une fonction implicite $\phi : U \subseteq \mathbb{R}^{n-1}\to \mathbb{R}$ qui satisfait les hypothèses du théorème de la fonction implicite, notamment qu'il existe un voisinage de $a : U\subset \mathbb{R}^{n-1}$ tel que, $\forall x \in U, ( x,\phi( x) ) \in \Sigma$.\\
Soit $b \in U$ tel que $( b, \phi( b) ) = a$.
Définissons maintenant
\[ 
	z^{*} \coloneqq ( z_1, \ldots, z_{n-1} )  \in \mathbb{R}^{n-1}.
\]
Calculons la dérivée directionelle selon $z^{*}$ de $\phi $:
\begin{align*}
	D_{z^{*}} \phi( b) &= \nabla \phi( b) \cdot z^{*}\\
			   &= \sum_{i=1} ^{n-1} - \frac{1}{\del_n f( a) } \del_i f( a) z_i \\
			   \intertext{En utilisant que }
			   \nabla f ( a) \cdot z &= 0\\
			   \sum_{i=1}^{ n-1}  \del_i f( a) z_i &= - \del_n f( a) z_n\\
			   \intertext{On trouve que }
			   D_{z^{*}} \phi( b)  &= z_n
\end{align*}
Considérons donc maintenant un chemin $\beta :]c,d[\to \mathbb{R}^{n}$ défini par
\[ 
	\beta( \lambda) = \left( \lambda z^{*}+ b, \phi( \lambda z^{*}+ b) \right) 
\]
ou $c<0< d, \quad d,c \in [-1,1]$ sont tels que $\lambda z^{*}+b \in U \forall \lambda \in ]c,d[$, l'existence de $c$ et $d$ est donnée par le fait que $U$ est ouvert et contient $b$.
Notons que
\[ 
	\beta'( 0) = \left( z^{*}, D_{z^{*}}\phi( b)  \right) = z \text{ et } \beta( 0) = \left( b,\phi( b) 	 \right) = a
\]
Ainsi, n'importe quel chemin $\gamma \in C^{1}( ]-1,1[ , \Sigma) $  qui satisfait que $\gamma( \lambda) = \beta( \lambda) \forall \lambda \in ]c,d[$ satisfait les conditions de l'ensemble $T_a( \Sigma) $.\\
On en déduit que $z \in T_a( \Sigma) $ et ainsi
\[ 
	T_a( \Sigma) = \Pi_a( \Sigma) -a.
\]













\end{document}
