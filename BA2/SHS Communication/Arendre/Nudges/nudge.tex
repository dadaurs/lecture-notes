\documentclass[11pt, a4paper]{article}
\usepackage[utf8]{inputenc}
\usepackage[T1]{fontenc}
\usepackage[francais]{babel}
\usepackage{lmodern}
\PassOptionsToPackage{hyphens}{url}\usepackage{hyperref}


\usepackage{amsmath}
\usepackage{amssymb}
\usepackage{amsthm}
\begin{document}
\title{Nudges pour inciter à la vérification d'information}
\author{David Wiedemann}
\maketitle
Bonjour,
Je partage également l'avis que le partage d'informations peu fiables fait partie des enjeux les plus importants auxquels nous faisons face.
Des nudges poussant les gens à la pensée critique sont, selon moi, des mesures indispensables si nous voulons empêcher la prolifération de théories du complot et d'opinions extrémistes.
Ce phénomène d'extrémisation a été accentué par la pandémie, des études montrent   que 25\% des citoyens américains pensent que la pandémie a été planifiée.\\
Ces statistiques montrent à quel point il est vital de mettre sur place des mesures qui incitent à la pensée critique.\\
Cependant, il est mon avis que des options telles que celles proposées ici sont, d'un côté difficile à implémenter et pas nécessairement très efficace.\\
Tout d'abord, remarquons que les entreprises qui sont derrière cette hausse de mauvaise information profitent du phénomène. Les algorithmes utilisés par les réseaux sociaux sont créés d'une manière qui incite à l'extrémisation Ces problèmes ont été soulevés, notamment pendant l'audition de Mark Zuckerberg au sénat Américain.
Bien que Facebook nie ses intentions explicites quant à ceci, les conséquences du phénomène sont visibles.

En deuxième lieu, il faut malheureusement remarquer qu'il y aura toujours des personnes qui croient en une information, même si elle est qualifiée de fausse. Le phénomène psychologique qui mène à ceci a été décrit dans la vidéo sur les nudges. Le problème est simplement qu'on aura tendance à croire une information qui s'aligne avec notre opinion, même si elle a été identifiée comme fausse.
Des études récentes ont même montrés qu'une personne a tendance à partager des informations, même si elle sait qu'elles sont fausses ( j'ai mis le lien vers cette étude plus bas) 

Selon moi, un nudge doit se faire d'une manière plus subtile et contenir deux aspects.\\
Voici mon idée pour un tel nudge:\\
Tout d'abord, il faut que les réseaux sociaux arrètent de recommander des sites web qui sont liés à des informations extrémistes ou des théories du complot.\\
En deuxième lieu, il faut tenter de promouvoir le partage de "bonnes" information, on pourrait par exemple plus recommender des posts contenant des liens vers de "bonnes" informations.
Ceci inciterait indirectement le partage de nouvelles fiables.

Bien entendu, déterminer ce qui est de la "bonne" information n'est pas facile, mais ces nudges, si implémentés, auraient certainement le bénéfice de réduire la quantité et l'influence des fake news.\\

 \subsection*{Sources}
 \url{https://www.pewresearch.org/fact-tank/2020/07/24/a-look-at-the-americans-who-believe-there-is-some-truth-to-the-conspiracy-theory-that-covid-19-was-planned/} \\
 \url{https://www.npr.org/sections/money/2020/08/04/898596655/are-conspiracy-theories-good-for-facebook}
 \url{https://pubmed.ncbi.nlm.nih.gov/32603243/}

%[La résolution des exercices ici...] %commentaire



\end{document}
