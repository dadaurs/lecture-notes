\documentclass[11pt, a4paper, twoside]{article}
\usepackage[utf8]{inputenc}
\usepackage[T1]{fontenc}
\usepackage[francais]{babel}
\usepackage{lmodern}

\usepackage{amsmath}
\usepackage{amssymb}
\usepackage{amsthm}
\begin{document}
\title{Series 3}
\author{David Wiedemann}
\maketitle
\section*{1}
We simply apply the expansion for the square of the geometric series, this result has been proven during the lectures:
\[ 
	\frac{1}{( 1-2x)^{2} } = \sum_{i=0}^{ \infty  } ( i+1) ( 2x) ^{i} =\sum_{i=0}^{ \infty  } ( i+1) 2^{i} x ^{i} 
\]
Hence the coefficient for the 5th power is given by
\[ 
6 \cdot 2^5 = 192
\]
\section*{2}
First notice that we can factorize a $x^8$ in the expression:
\[ 
	\left( \sum_{i=2}^{ \infty } x^{i} \right)^{4} = x^8 \left( \sum_{i=}^{ \infty } x^{i}\right)^{4}
\]
Thus, we only have to find the coefficient of $x^7$ in the formal series expansion of
\[ 
 \left( \sum_{i=0}^{ \infty } x^{i} \right)^{4}	
\]
This counting problem is equivalent to distributing 7 identical objects between 4 persons, thus, by a theorem proven in the first lecture, the result is
\[ 
\binom { 4+7-1} { 4-1} = \binom { 10 } 3 = 120.
\]










%[La résolution des exercices ici...] %commentaire



\end{document}
