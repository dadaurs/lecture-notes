\documentclass[11pt, a4paper, twoside]{article}
\usepackage[utf8]{inputenc}
\usepackage[T1]{fontenc}
\usepackage[francais]{babel}
\usepackage{lmodern}

\usepackage{amsmath}
\usepackage{amssymb}
\usepackage{amsthm}

%%%%%%%%%%%%%%%%%%%%%%%%%%%%%%%%%%%%%%%%%%%

\date{}

\begin{document}
\title{Exercises Week 1}
\author{David Wiedemann}
\maketitle
\section*{Part 1}

We describe the process which leads to the construction of such a word.\\
Since the order of the letters matter, we proceed letter by letter.\\
The first letter gives $m$ choices.\\
For the $n-1$ remaining letters, there are $m-1$ choices per letter.
Thus, the answer is 
\[ 
	m \cdot ( m-1)^{n-1}
\]
Where we have used the formula for a $n-1$ letter word in an alphabet of $m-1$ letters.\\
In the specific case where $m=n=1$, the above expression is undefined, however the result is clearly 1.
\section*{Part 2}
In order to solve the problem, we differentiate between the different amount of \textit{a}'s the word could contain.\\
Indeed, by differentiating the number of \textit{a}'s we will be able to consider a pair of \textit{a}'s as one letter, hence simplifying the computations.\\
\begin{itemize}
	\item If the word contains 0 \textit{a}'s, we simply form a 10 letter word in an alphabet of  size 2:
		\[ 
		2^{10}
		\]
		
	\item If the word contains 2 a's, it is as if the word contains 9 letters, one of which is replaced by a ``double a''. Once the spot for the double a is chosen, we build a 8 letter word in a 2 letter alphabet, hence we have
\[ 
\binom 9 1 \cdot 2^{8}
\]
\item The same reasoning for 4 a's yields
	\[ 
	\binom 8 2 \cdot 2^{6}
	\]

\item For 6 a's we have
	\[ 
	\binom 7 3 \cdot 2^{4}
	\]

\item For 8 a's, the result is
	\[ 
	\binom 6 4 \cdot 2^{2}
	\]

\item And finally, for 10 a's we simply have
	\[ 
	1
	\]
\end{itemize}
We can now sum up these possibilities to get the desired result
\[ 
2^{10} + 9 \cdot 2^{8} + 28 \cdot 2^{6} + 35 \cdot 2^{4} + 15 \cdot 2^{2} + 1 = 5741
\]





\end{document}
