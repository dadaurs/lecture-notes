\documentclass[11pt, a4paper, twoside]{article}
\usepackage[utf8]{inputenc}
\usepackage[T1]{fontenc}
\usepackage[francais]{babel}
\usepackage{lmodern}

\usepackage{amsmath}
\usepackage{amssymb}
\usepackage{amsthm}
\usepackage{mathtools}

\DeclarePairedDelimiter\floor{\lfloor}{\rfloor}
\begin{document}
\title{Exercise for Submission 5}
\author{David Wiedemann}
\maketitle
We will show this result by using the binomial theorem.\\
We will denote by $\floor x$  the biggest integer satisfying $\floor x \leq x$, ie. the floor of the real number $x.$
\begin{align*}
	\frac{1}{2} \left( ( 1+ \sqrt{2} )^{n} + ( 1- \sqrt{2} ) ^{n}	  \right) &= \frac{1}{2} \left( \sum_{k=0}^{n } \binom n k \sqrt{2} + \sum_{k=0}^{ n} \binom n k ( - \sqrt{2} )  \right)\\
		 &= \frac{1}{2} \Bigg( \sum_{k=0}^{ \floor { \frac{n}{2} }} \binom { n} { 2k} \sqrt{2}^{2k} + \sum_{k=1}^{ \floor { \frac{n}{2} }} \binom { n} { 2k-1} \sqrt{2}^{2k-1}\\
		 &+\sum_{k=0}^{ \floor { \frac{n}{2} }} \binom { n} { 2k}( - \sqrt{2} )^{2k} + \sum_{k=1}^{ \floor { \frac{n}{2} }} \binom { n} { 2k-1} (- \sqrt{2} )^{2k-1}   \Bigg) 
\end{align*}
Notice that all the uneven powers simplify and that $( - \sqrt{2} ^{2k}) = \sqrt{2} ^{2k}$, thus we are left with
\begin{align*}
	\frac{1}{2} \left( 2 \cdot\sum_{k=0}^{ \floor { \frac{n}{2} }} \binom { n} { 2k} \sqrt{2}^{2k} \right) &= \sum_{k=0}^{ \floor { \frac{n }{2}} } \binom n { 2k} 2 ^{k}	 \\
\end{align*}
Which proves that  $ \frac{1}{2} (  ( 1+ \sqrt{2} )^{n} + ( 1- \sqrt{2} )^{n}) $ always is a whole number.



\end{document}
