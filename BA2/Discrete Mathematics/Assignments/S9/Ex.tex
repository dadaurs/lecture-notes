\documentclass[11pt, a4paper]{article}
\usepackage[utf8]{inputenc}
\usepackage[T1]{fontenc}
\usepackage[francais]{babel}
\usepackage{lmodern}
\usepackage{amsmath}
\usepackage{amssymb}
\usepackage{amsthm}
\renewcommand{\vec}[1]{\overrightarrow{#1}}
\newcommand{\del}{\partial}
\DeclareMathOperator*{\sgn}{sgn}
\DeclareMathOperator*{\id}{Id}
\DeclareMathOperator*{\im}{Im}
\DeclareMathOperator*{\re}{Re}
\DeclareMathOperator*{\vol}{Vol}
\newcommand\norm[1]{\left\vert#1\right\vert}
\newcommand\ns[1]{\left\vert\left\vert\left\vert#1\right\vert\right\vert\right\vert}
\newcommand\Norm[1]{\left\lVert#1\right\rVert}
\newcommand\N[1]{\left\lVert#1\right\rVert}
\newcommand\abs[1]{\left\vert#1\right\vert}
\newcommand\inj{\hookrightarrow}
\newcommand\surj{\twoheadrightarrow}
\newcommand\ded[1]{\overset{\circ}{#1}}
\newcommand\sidenote[1]{\footnote{#1}}
\newcommand\eng[1]{\left\langle#1\right\rangle}
\newcommand\hr{
    \noindent\rule[0.5ex]{\linewidth}{0.5pt}
}

\newcommand{\incfig}[1]{%
    \def\svgwidth{\columnwidth}
    \import{./figures}{#1.pdf_tex}
}
\newcommand{\filler}[1][10]%
{   \foreach \x in {1,...,#1}
    {   test 
    }
}

\newcommand\contra{\scalebox{1.5}{$\lightning$}}
\makeatother
\def\@lecture{}%
\newcommand{\lecture}[3]{
    \ifthenelse{\isempty{#3}}{%
        \def\@lecture{Lecture #1}%
    }{%
        \def\@lecture{Lecture #1: #3}%
    }%
    \subsection*{\@lecture}
    \marginpar{\small\textsf{\mbox{#2}}}
}

\begin{document}
\title{Exercise 9}
\author{David Wiedemann}
\maketitle
As shown in the course, to each unlabeled tree $T$ with $n$ vertices, we can associate a ( not necessarily unique)  sequence $S \in \left\{ 1,-1 \right\} ^{2n-2}$.\footnote { We simply substituted $+$ with $1$ and $-$ with $-1$.} \\
$T$ can then be uniquely reconstructed from this sequence.\\
Let us show that any sequence in $ S$ corresponding to such a tree satisfies 
\[ 
	\sum_{i=1}^{ k} S_i \geq 0 \quad \forall 1 \leq  k \leq 2n-2
\]
and 
\[ 
	\sum_{i=1}^{ 2n-2} S_i = 0 
.\]
\hr
First, notice that the second equality follows immediatly from the way we define the contouring path of an unlabelled tree.\\
\hr
To show that $\sum_{i=1}^{ k} S_i \geq 0 $, we use induction.\\
For $k=1$, it does not make sense to reduce the distance to the root vertex, so $S_1>0$.\\
Suppose shown for  $1 \leq k < 2n-2$, we will now show the result for $k+1$.\\
If $ \sum_{i=1}^{ k}S_i >0$, then $S_{k+1} $ can take any value and the result will still hold.\\
If $ \sum_{i=1}^{ k}S_i =0$, the distance to the root vertex is 0 and hence it is impossible to further reduce the distance to the root node, this implies that $S_{k+1} =1$, which concludes the proof.\\
\hr
As shown in the fourth exercise sheet, the number of sequences satisfying these properties is given by $b_{n-1} $.\\
Since there is a surjection from the set of such sequences to the set of unlabelled graphs, we deduce that
 \[ 
b_{n-1} \geq T^{n}.
\]







\end{document}
