\documentclass[../main.tex]{subfiles}
\begin{document}
\lecture{4}{Sun 14 Mar}{Combinatorial applications of polynomials and generating series}
We note that arithmetic operations with finite sets have similarities.
\[ 
	( a+b) \cdot c = a\cdot c + b \cdot c
\]
\[ 
	( A \cup B ) \cap C = A \cap C \cup B \cap C
\]
\begin{exemple}
Prove the identity
\[ 
	\sum \binom n i ^{2} = \binom 2n n
\]
Consider
\[ 
	( 1+x) ^{n} \cdot ( 1+x) ^{n}= ( 1+x) ^{2n}
\]
By computing the coefficients of $x^{n}$, we find the desired equality.

\end{exemple}
\begin{thm}[Multinomial theorem]
	\[ 
		( x_1 + \ldots + x_n) ^{k} = \sum_{i_1, \ldots, \geq 0, i_1 + i_2 + \ldots =k} \frac{k!}{i_1! \ldots i_n!}x_1^{i_1}x_2 ^{i_2} \ldots x_n^{i_n}
	\]
	
	
\end{thm}
\begin{proof}
Note that
\[ 
\frac{k!}{i_1! \ldots i_n!}
\]
is the number of sequences of length $k$ from the letters "$x_1, x_2,\ldots$ " such that $x_j$ is used $i_j$ times.
\end{proof}
\begin{defn}[Generating series]
	Let $a_n$ be a sequence of complex numbers, then the generating series of this sequence is
	\[ 
		a( x) = \sum_{n=0}^{ \infty }a_n x^{n}
	\]
\end{defn}
\begin{defn}[Formal power series]
	A formal power series is an infinite sum
	\[ 
		a( x) = \sum a_n x^{n}
	\]
	where $a_n$ is a sequence of complex numbers and $x$ is a formal variable.
	
\end{defn}
\begin{propo}
	Let $a( x) = \sum a_n x^{n} $ be a formal power series. Suppose that there exists a positive real number $K$ such that $|a_n| < K^{n}$ for all $n$. Then the series converges absolutely for all $x \in ]-\frac{1}{k}, \frac{1}{k}[$.\\
	Moreover, the function $a( x) $ as derivatives of all orders at $0$.
\end{propo}
We can add and multiply formal power series.\\
However, in general, substitution is not well defined
\[ 
	a( b( x) ) = \sum_{n=0}^{ \infty }a_n b( x)^{n} = \sum_{n=0}^{ \infty }a_n ( \sum_{m=0}^{ \infty }b_m x^{m}) ^{n}
\]
It is only well defined if $b_0=0$.\\
We can also differentiate, resp. integrate formal power series.
\begin{thm}[Generalized binomial theorem]
	For every $r \in \mathbb{R}$, we have
	\[ 
		( 1+x)^{r}= \binom r 0  + \binom r 1 x \ldots
	\]
	where 
	\[ 
		\binom r k = \frac{r ( r-1) \ldots ( r-k+1)  }{k!}
	\]
	
	
\end{thm}




\end{document}	
