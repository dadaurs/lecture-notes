\documentclass[../main.tex]{subfiles}
\begin{document}
\lecture{2}{Sat 27 Feb}{factorials and birthday paradox}
\begin{thm}[Stirling's formula]
	\[ 
	n! ~ \sqrt{2 \pi n} n^{n}e^{-n}	
	\]
	meaning the ration goes to 1.
	
	
\end{thm}
\begin{proof}
Euler's integral for $n!$ gives
\[ 
n! = \int_{ 0 }^{ \infty  }x^{n}e^{-x}dx
\]
This is proven by induction on $n$.\\
The base case $n=0$ simply gives 1.\\
For the integration step, we integrate by parts, giving
\[ 
\int_{ 0 }^{ \infty  }x^{n}e^{-x} = \int_{ 0 }^{ \infty  }e^{-x} \frac{d}{dx}x^{n} dx
\]

To prove Stirlings formula, we take
\[ 
	xt^{n}e^{-x}= \exp( n \log x -x ) 
\]
We now taylor expand around the maximum, this yields
\[ 
	n \log x - x = n \log n - n - \frac{1}{2n}( x-n)^{2} + \ldots
\]
integrating this gives the desired formula.


\end{proof}

\end{document}		
