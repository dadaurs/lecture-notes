\documentclass[../main.tex]{subfiles}
\begin{document}
\lecture{5}{Sat 20 Mar}{Binary trees}
\begin{defn}[Binary Tree]
	A binary tree is either empty, or consists of one distinguished vertex called the root, plus an ordered pair of binary trees calle de left subtree and the right subtree.
\end{defn}
We denote by $b_n$ the number of binary trees with $n$ vertices. We want to fin a closed formula for $b_n$
The inductive definition yields
\[ 
b_n = b_0 \cdot b_{n-1}  + b_1 \cdot b_{n-2}  + \ldots + b_{n-1}  \cdot b_0
\]
Consider
\[ 
	b( x)  = \sum b_n x^{n}
\]
And we use
\[ 
 b_n = \sum b_k \cdot b_{n-k-1} 
\]
Now we use
\begin{align*}
b( x) \cdot b( x)  = \sum_{k=0}^{ \infty } \left( \sum_{m=0}^{ \infty }b_m b_{k-m}  \right) x^{k}\\
= \frac{1}{x} \left( \sum_{k=1}^{ \infty } b_k x^{k}\right)  = \frac{1}{x}(  b( x) -b_0) 
\end{align*}
Hence, $b( x) $ satisfies
\[ 
	x b^{2}( x) -b( x) +1=0
\]
Hence
\[ 
	b( x)  = \frac{1+ \sqrt{1-4x} }{2x} \text{ and } b( x) = \frac{1- \sqrt{1-4x} }{2x}
\]
are solutions.\\
Note that the first solution is not bounded around 0.\\
However, the second solution is smooth around 0 because
\[ 
\tilde b ( x) := \frac{1- \sqrt{1-4x} }{2x} = \frac{2}{1+ \sqrt{1-4x} }
\]
Hence, $\tilde b ( x) $has derivatives of all orders.\\
We want to establish the connection between $\tilde b $ and $b$.\\
Consider the taylor expansion of $\tilde b $
\[ 
	\tilde b ( x) = \sum_{n=0}^{ \infty }\tilde b_n \cdot x^{n}
\]
Still, $\tilde b$ satisfies the quadratic equation, we want to show 
\[ 
 \tilde b _n = \sum \tilde b_k \cdot \tilde b_{n-k-1} 
\]
By taylors theorem
\[ 
	\tilde b ( x) = \tilde b_0 + \tilde b_1 x + \ldots + O( x^{n+1}) 
\]
We substitute this into the quadratic equation, which yields
\[ 
	x (  \tilde b_0 + \ldots \tilde b_n x^{n}+ O( x^{n+1}) )^{2} - (  \tilde b_0 + \ldots + \tilde b_n x^{n} + O( x^{n+1}) ) +1 = 0
\]
Solving for $\tilde b_n$ yields the desired equation.\\
Applying the generalized binomial theorem gives a closed form for $b_n$
\[ 
	b_n = - \frac{1}{2} ( -4)^{n+1}  \binom {  \frac{1}{2}} {  n+1} 
\]
We define the $b_n$ 's as Catalans number.

\end{document}	
