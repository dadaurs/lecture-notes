\documentclass[../main.tex]{subfiles}
\begin{document}
\lecture{9}{Sat 24 Apr}{Basics of graph theory}
\section{Graph Theory}
\subsection{Basic Definitions}
\begin{defn}[Graph]
	A graph $G$  is an ordered pair $( V,E) $ where $V$ is a set of elements called vertices and $E$ is a set of 2-element subsets of $V$.	
\end{defn}
\begin{exemple}
$V = \left\{ 1,2,3,4 \right\} $ and $E = \left\{ \left\{ 1,2 \right\}, \left\{ 2,3 \right\} , \left\{ 3,4 \right\} , \left\{ 4,1 \right\}   \right\} $
\begin{figure}[H]
    \centering
    \incfig{four-edged-graph}
    \caption{four edged graph}
    \label{fig:four-edged-graph}
\end{figure}
This is called a undirected simple graph.
\end{exemple}
\subsection*{Non-examples}
\begin{figure}[H]
    \centering
    \incfig{multiple-edges}
    \caption{multiple edges}
    \label{fig:multiple-edges}
\end{figure}
\begin{figure}[H]
    \centering
    \incfig{loops}
    \caption{loops}
    \label{fig:loops}
\end{figure}
\begin{figure}[H]
    \centering
    \incfig{direction-of-edges}
    \caption{direction of edges}
    \label{fig:direction-of-edges}
\end{figure}
\subsection{Important graphs}
\begin{defn}[Complete graph]
	Let $V$ be a finite set.\\
	A complete graph on vertices $V$ (  or a clique) is the graphe $G = ( V, \binom V 2) $
\end{defn}
A complete graph with $n $ vertices is denoted $K_n$ 
\begin{figure}[H]
    \centering
    \incfig{k4}
    \caption{$K_4$}
    \label{fig:k4}
\end{figure}
\begin{defn}[Cycle graph]
	The cycle $C_n$ is the graph
	\[ 
	V = \left\{ 1,2,\ldots,n \right\} \quad E= \left\{ \left\{ 1,2 \right\} , \ldots, \left\{ n-1,n \right\} , \left\{ n,1 \right\}  \right\} 
	\]
	
\end{defn}
\begin{figure}[H]
    \centering
    \incfig{c5}
    \caption{ $C_5$}
    \label{fig:c5}
\end{figure}
Let $G= c( V,E) $ be a graph
\begin{defn}[Adjacent vertices]
	Let $v_1,v_2\in V$ be vertices of $G$. If $ \left\{ v_1,v_2 \right\} \in E$we say that $v_1$ and $v_2$ are connected by and edge or adjacent.
\end{defn}
\begin{defn}[Degree of a vertex]
	A degree of a vertex $v \in V$ is the number of edges adjacent to it.
\end{defn}
\begin{lemma}[The hand shake lemma]
	The sum of degrees of all vertices in a finite graph $G$ is always an even lemma.
\end{lemma}
\begin{proof}
The sum of degrees of all vertices is equal to twice the number of edges.
\end{proof}
\subsection{Wals and paths}
\begin{defn}[Walk]
A walk on a graph $G$ is a sequence of nodes $v_0, v_1, \ldots, v_k$ such that $v_i$ is adjacent to $v_{i+1} $ for all $i<k$.
\end{defn}
\begin{defn}[Path]
	A path is a walk such that all its vertices are distinct.
\end{defn}
\begin{defn}[Closed Walk]
	A closed walk is a walk such that the first vertex coincides with the last one.
\end{defn}
\begin{defn}[Connected graph]
	A graph $G= ( V,E) $ is connected if for every two vertices $u,v \in V$ there exists a path in $G$ between them.
\end{defn}
\begin{defn}[Cycle]
	A cycle in a graph $G= ( V,E) $ is a sequence of distinct vertices $v_1, \ldots, v_r \in V$ such that $v_{ i \mod r} $ is adjacent to $v_{i+1 \mod r} $.
\end{defn}
\begin{defn}[Tree]
	A tree is a connected graph without cycles.
\end{defn}
\begin{defn}[Leaf]
	A vertex of degree $1$ in a tree is called a leaf.
\end{defn}
\begin{lemma}
	Every finite tree with $n \geq 2$ vertices has at least two leaves
\end{lemma}
\begin{proof}
	Consider a path of maximum length, say $v_1, v_2, \ldots, v_n$.\\
	A tree is connected, therefore such a path exists and has lenght at least 2.\\
	The initial point $v_1$ and the final point $v_n$ mus be leaves otherwise the path can be extended.
\end{proof}
\begin{lemma}
Every tree with $n$ vertices has exactly $n-1$ edges.
\end{lemma}
\begin{proof}
We prove the lemma by induction on $n$.\\
For $n=1$, it is clear that the graph contains 0 edges.\\
Suppose the lemma is true for all threes with  $ \leq n$ vertices.\\
Let $T= ( V,E) $ be a tree with $n+1$ vertices.\\
By the previous lemma $T$ has a leave, say $v \in V$.\\
Let $e$ be the unique edge adjacent to $v$.\\
We define a new graph $T' \coloneqq  ( V \setminus \left\{ v \right\} , E \setminus\left\{ e \right\} ) $.\\
The new graph $T'$ contains no cycles and is connected, hence $T$ is a tree.\\
By induction hypothesis 
\[ 
| E \setminus \left\{ e   \right\} | = | V \setminus \left\{ v \right\} | - 1 = n-1
\]
Which concludes the proof.

\end{proof}




\end{document}	
