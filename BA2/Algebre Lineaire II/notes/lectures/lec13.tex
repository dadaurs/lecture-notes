\documentclass[../main.tex]{subfiles}
\begin{document}
\lecture{13}{Tue 13 Apr}{Matrices Symmetriques}
\begin{thm}[Theoreme Spectral]
Soit $A \in \mathbb{R}^{n\times n}$ symmetrique, alors il existe $P \in \mathbb{R}^{n\times n}$ orthogonale tel que
\[ 
P^{T}\cdot A \cdot P
\]
est diagonale.\\
Donc $A $ est congruent a une matrice diagonale et est semblable $D$.
\end{thm}
\begin{lemma}
Soit $A \in \mathbb{C}^{n\times n}$ une matrice hermitienne, alors toutes ses valeurs propres sont reelles.
\end{lemma}
\begin{proof}
	Soit $\lambda \in \mathbb{C}$ une valeur propre et $v \in \mathbb{C}^{n}\setminus \left\{ 0 \right\} $ un vecteur propre associe a $\lambda$. On va montrer que $\lambda v^{T} \overline{v}= \overline{\lambda} v$.\\
	On a 
	\begin{align*}
	\lambda v^{T} \overline{v} = v^{T} A^{T} \overline{v}= v^{T} \overline{A} \overline{v} = v^{T} \overline{\lambda} \overline{v} = \overline{\lambda }v \overline{v}
	\end{align*}
	
\end{proof}
\begin{crly}
Soit $A \in \mathbb{R}^{n\times n}$ resp. $\mathbb{C}^{n\times n}$ une matrice symmetrique resp., hermitienne. Alors $A$ possede une valeur propre reelle.
\end{crly}
\begin{proof}
Les valeurs propres de $A$ sont les racines relles resp. complexes du polynome characteristique de $A$.\\
Soit $\lambda \in \mathbb{C}$ une racine, donc $\lambda$ est une valeur propre de $A$ sur $\mathbb{C}^{n}$, par le lemme ci-dessus, $\lambda$ est reel.\\ 
Et donc $\lambda$ est une valeur propre d'une matrice reelle de $A$.
\end{proof}
Prouvons maintenant le theoreme spectral.
\begin{proof}
		On demontre le cas reel.\\
		Soit $A \in \mathbb{R}^{n\times n}$ symmetrique. Il existe $U \in \mathbb{R}^{n\times n}$ orthogonale tel que $U^{T}A U$ est orthogonale.\\
		On procede par recurrence.\\
		Le cas $n=1$, $A= ( a_{11}) $ est clair.\\
		Pour $n>1$, soit $\lambda \in \mathbb{R}$ une valeur propre de $A$ et $v \in \mathbb{R}^{n}\setminus \left\{ 0 \right\} $ un vecteur propre associe  tel que $v^{T}v =1$.\\
		Soit $  \left\{ v_1, u_2, \ldots \right\} $ une base de $ \mathbb{R}^n$.\\
		Avec Gram-Schmidt, on peut supposer que cette base est orthonormale.\\
		Soit $U$ la matrice donnee par les colonnes  $(  u_2,\ldots, u_n) \in \mathbb{R}^{n \times ( n-1) }$, on considere $U\\^{T}A U\in \mathbb{R}^{( n-1) \times ( n-1) }$, c'est une matrice symmetrique ( parce que $A$ est symmetrique).\\
		Par recurrence, il existe une matrice orthogonale tel que $K^{T}U^{T}A U K $ est diagonale et reelle.\\
		Posons $P =( v, U\cdot K) \in \mathbb{R}^{n\times n}$.\\
		$P$ est orthogonale, en effet
		\[ 
		P^{T}P = 
		\begin{pmatrix}
		v^{T}\\ K^{T}U^{T}
		\end{pmatrix}
		\begin{pmatrix}
		v\\ U K
		\end{pmatrix}
		= 
		\begin{pmatrix}
			v^{T}v & v^{T}U K\\
			K^{T}U^{T}v& K^{T}U^{T}U K
		\end{pmatrix}
		=
		\begin{pmatrix}
			1 & 0\\
			0 & \id\\
		\end{pmatrix}
		\]
		Et donc 
		 \begin{align*}
	P^{T}A P =
	\begin{pmatrix}
	v^{T}\\K^{T}U^{T}
	\end{pmatrix}
	A
	( V, UK) 
	\end{align*}
	Or $v$ est orthogonal a tous les $u_i$  et donc cette matrice est orthogonale.
		
		
\end{proof}








\end{document}	
