\documentclass[../main.tex]{subfiles}
\begin{document}
\lecture{10}{Wed 24 Mar}{Produits Scalaires}
\subsection{Theoreme de Sylvester}
Soit $V$ un espace vectoriel de dimension finie sur $\mathbb{R}$ et $\eng , $ une forme bilineaire symmetrique.\\
Alors, il existe une base orthogonale $B$ qui diagonalise $A^{\eng ,}$.\\
Pour tout $i$, si $\eng { v_i,v_i} >0$, alors $v_i := \frac{v_i}{\sqrt{\eng{v_i,v_i}} }$.\\
	Si $\eng{v_i,v_i} <0$, si $v_i:= \frac{v_i}{|\sqrt{\eng { v_i,v_i} }| }$
	Une telle base est appelee une base de Sylvester.
\begin{propo}
	Soit $V$ un espace vectoriel sur $\mathbb{R}$ de dimension finie et soit $\eng ,$ une forme bilineaire symmetrique, alorts il existe une base de Sylvester de $V$.
\end{propo}
\begin{thm}[Theoreme de Sylvestre]
	Soient $ B=\left\{ v_1, \ldots, v_n \right\} $ et $B'= \left\{ w_1, \ldots, w_n \right\} $ deux bases de Sylvester tel que $\eng { v_i,v_i}=1 $ $\forall i\leq r$ et $\eng { v_i,v_i} \leq 0 \forall i>r$ et  $\eng { w_i,w_i}=1 $ $\forall i\leq r'$ et $\eng { w_i,w_i} \leq 0 \forall i>r'$, alors $r=r'$
\end{thm}
\begin{proof}
	On montre que $v_1, \ldots, v_r, w_{r'+1} , w_n$ est un ensemble libre.\\
Supposons que la famille n'est pas libre, alors il existe $\alpha_1, \ldots, \alpha_r, \beta_{r'+1} , \ldots \beta_n$ pas tous nuls tel que
\[ 
\alpha_1 v_1 + \ldots = \beta_{r'+1} w_{r'+1} + \ldots
\]
Alors on a
\[ 
	\eng { \sum_{i}\alpha_i v_i, \sum_{i}\alpha_i v_i} = \eng { \sum_{i=r'+1} \beta_i w_i } 
\]
Ainsi
\[ 
\alpha_1^{2} \eng { v_1, v_1}  + \ldots = \beta_{r'+1}^{2} \eng { w_{r'+1},w_{r'+1}  }  + \ldots
\]




\end{proof}
\begin{defn}[Espace de Nullite]
	Soit $V$ un espace vectoriel sur $K$ et $\eng .$ une forme bilineaire symmetrique, alors
	\[ 
	V^{0}= \left\{ v: \eng { v,x} =0 \forall x \in V \right\} 
	\]
	est appele l'espace de nullite
\end{defn}
\begin{lemma}
	Soit $B = \left\{ v_1, \ldots, v_n \right\} $ une base orthogonale de $V$. Alors $V^{0}$ est engendre par $v_i$ tel que $\eng{ v_i, v_i } = 0$
\end{lemma}
\begin{proof}
On ecrit $B= \left\{ v_1, \ldots, v_n \right\} $ ou $\eng { v_i,v_i} \neq0 \forall i<l$ et  egal a $0$ sinon.\\
Soit $v \in V$ tel que $v= \sum \alpha_i v_i$ tel que $\alpha_j \neq 0 $ pour $j<l$.\\
On choisit $x= v_j$ et on voit que $\eng { x,v_j }\neq 0$
\end{proof}
On voit donc que le nombre de $v_i$ dans une base de Sylvester tel que $\eng { v_i,v_i} =0$ est toujours egal a $\dim V^{0}$ et est donc invariant.



\end{document}	
