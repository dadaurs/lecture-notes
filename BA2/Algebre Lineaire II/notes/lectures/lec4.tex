\documentclass[../main.tex]{subfiles}
\begin{document}
\lecture{4}{Wed 03 Mar}{Polynomes 2}
\subsection{Factorisation en elements irreductibles}
Un polynome $p( x) $ est irreductible si le degre de $p$ est $\geq 1$, $p( x) \neq 0$.\\
Si $h | p$, alors $h=a$ ou $h=a \cdot p$.\\
Tout $f( x) \in K[x]$ se laisse factoriser
\[ 
	f( x) = a \prod_i p_i( x) , p_i( x) \text{ irreductibles, unitaires } 
\]
Est-ce que cette factorisation est unique?\\
\begin{thm}
	Soit $p( x) \in K[x] \setminus \left\{ 0 \right\}  $ irreductible et supposons que $p | f_1( x) \ldots f_k( x) $, alors il existe $i$ tel que $p( x) | f_i( x) $
\end{thm}
\begin{proof}
Par recurrence, il suffit de demontrer l'assertion pour $k=2$.\\
Supposons que $p| f \cdot g$, $f,g \in K[x] \setminus \left\{ 0 \right\} $.\\
Si $p \not | f$, alors $\gcd ( p,f) =1$. Donc, il existe $u,v \in K[x]$ tel que $up + vf=1$, donc on a
\[ 
upg + vfg =g \Rightarrow p | upg + vfg \Rightarrow p | g
\]

\end{proof}
\begin{thm}[La factorisation est unique]
	La factorisation est unique a l'ordre pres des $p_i$.
\end{thm}
\begin{proof}
	Soit $f( x) = a \prod p_i( x) $ et $f( x) = a \prod q_j( x) $ une autre factorisation en elements irreductible.\\
	Par recurrence sur $k$.\\
	Si $k=1$, alors
	\[ 
		ap_1( x)  = a q_1( x) \ldots q_l( x) 
	\]
	Et donc $q_1( x) = p_1( x) $, car $p_1$ est irreductible.
	Si $k>1$, 
	\[ 
		a p_1( x) \ldots p_k( x) = a q_1( x) \ldots q_l( x) 
	\]
	Grace au theoreme ci-dessus, $p_1| q_j$ pour un certain $j$ $\iff p_1 = q_j$.
	Et donc on obtient
	\[ 
		p_2( x) \ldots = q_1( x) \ldots q_l( x) 
	\]
	Par recurrence, cette factorisation existe et est la meme a ordre pres.	

\end{proof}
\begin{crly}
	Soit $f( x) \in K[x]\setminus \left\{ 0 \right\}$ et $\alpha_1 \ldots$ des racines de $f$ de multiplicite $k_1, \ldots, k_l$ respectivement.\\
	Alors il existe $g( x) \in K[x]$ tel que
	\[ 
		f( x) = g( x) \prod ( x-\alpha_i)^{k_i}
	\]
	
\end{crly}
\begin{proof}
Exercice
\end{proof}

\section{Valeurs et Vecteurs Propres}
\begin{defn}[Vecteur propre]\label{defn:Vecteur proprevecteur_propre}
	Soit $V$ un espace vectoriel sur $K$ et $f$ un endomorphisme sur $V$.\\
	Un vecteur propre de $f$ associe a la valeur propre $\lambda \in K$ est un vecteur $v \neq 0$ satisfaisant
	\[ 
		f( v) = \lambda v
	\]
	
\end{defn}
\begin{lemma}
Soit  $B = \left\{ v_1, \ldots, v_n \right\} $ une base de $V$ et $A \in K^{n\times n}$ la matrice de l'endomorphisme $f$ relatif a $B$.\\
La matrice $A$ est une matrice diagonale 
\[ 
A = 
\begin{pmatrix}
	\lambda_1 & 0 & 0 \\
	0 & \ddots & 0 \\
	0 & 0 & \lambda_n \\
\end{pmatrix}
\]
$\iff$ $v_i$ est un vecteur propre associe a la valeur propre $\lambda_i$.

\end{lemma}
\begin{proof}
On a
\[ 
	[ f( v_i) ]_B = A e_i = \lambda_i e_i 
\]
Donc $v_i$ est un vecteur propre associe a $\lambda_i$.\\
Dans l'autre sens, les arguments sont similaires.
\end{proof}
\begin{defn}
	Un endomorphisme $f$ sur un espace vectoriel de dimension finie est appele diagonalisable s'il existe une base tel que $ \left\{ v_1, \ldots  \right\} $ de $V$ composee de vecteurs propres.
\end{defn}







\end{document}	
