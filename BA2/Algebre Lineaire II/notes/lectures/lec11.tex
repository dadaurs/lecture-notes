\documentclass[../main.tex]{subfiles}
\begin{document}
\lecture{11}{Tue 30 Mar}{Formes Bilineaires definies positives et Espaces Euclidiens}
\subsection{Formes Bilineaires symmetriques definies positives}
Ici, $V$ sera toujours un espace vectoriel reel.\\
\begin{defn}[Formes Bilineaires definies positives]
	Une forme bilineaire $\eng .$ est definie positive, si
	\[ 
	\forall v \in V\setminus \left\{ 0 \right\} : \eng { v,v} >0
	\]
	Une f.b.s. definie positive est appellee un produit scalaire.
\end{defn}
\begin{defn}[Norme d'un vecteur]
	La longueur( ou norme) d'un vecteur de $v \in V$ :
	\[ 
	\N { v} = \sqrt{\eng { v,v} } 
	\]
	
\end{defn}
\begin{defn}
	Un espace vectoriel reel muni d'un produit scalaire est appele espace euclidien.
\end{defn}
\begin{propo}
Pour $u \in V, \alpha \in \mathbb{R}$,
\[ 
\N { \alpha \cdot u} = | \alpha| \N u
\]

\end{propo}
\begin{proof}
\begin{align*}
\N { \alpha \cdot u}  = \sqrt{\eng { \alpha u, \alpha u} } = | \alpha| \N u
\end{align*}
\end{proof}
\begin{thm}[Theoreme de Pythagore]
	Pour $v,w \in V:$, si $\eng { v,w} =0$, alors
	\[ 
	\N { v+ w} ^{2} = \N v ^{2} + \N w ^{2}
	\]
	
\end{thm}
\begin{proof}
\begin{align*}
	\N { v+w} ^{2} &= \eng { v+w, v+w} \\
		       &= \eng { v,v}  + \eng { v,w} + \eng { w,v}  + \eng { w,w} \\
		       &= \eng { v,v}  + \eng { w,w} 
\end{align*}

\end{proof}
\begin{propo}[Regle du parallelogramme]
	Pour $u,w \in V:$ 
	\begin{align*}
\N { u+w} ^{2} + \N { u-w} ^{2} = 2 \N { u} ^{2} + 2 \N { w} ^{2}	
	\end{align*}
	Sans preuve( facile) 
\end{propo}
Soit $w,v \in V$, on cherche $\alpha$ tel que
\[ 
\eng { v-\alpha w,w}  = 0
\]
Donc
\[ 
\alpha = \frac{\eng { v,w} }{\eng { w,w} }
\]
On appelle $\alpha$ la composante de $v$ sur $w$ et $\alpha w$ la projection de $v$ sur $w$.

\begin{thm}[Inegalite Cauchy-Schwarz]
Pour tout $v,w \in V$,
\[ 
| \eng { v,w} | \leq \N v \N w
\]

\end{thm}
\begin{proof}
On considere d'abord le cas special $\N { w} =1$.\\
Donc, $\alpha= \eng{v,w}$, le theoreme de pythagore donne
\begin{align*}
\N { v} ^{2} = \N {v- \alpha w} ^{2} + \N { \alpha \cdot  w  }^{2} \geq \alpha^{2} \cdot \N { w} ^{2} = \alpha^{2} = | \eng { v,w} |^{2}
\end{align*}
Le cas general donne donc
\[ 
\eng { v, \N w \frac{w}{\N w}} \leq \N { w} ^{2} \N { v} ^{2}
\]

\end{proof}
\begin{thm}[Inegalite triangulaire]
	\[ 
	\N { v+w} \leq \N v + \N w
	\]
\end{thm}
\begin{proof}
	\begin{align*}
		\N { v+w} ^{2} &= \eng { v+w, v+w} ^{2}\\
			       &= \N v ^{2}+ w \eng { v,w}  + \N w ^{2}\\
			       &\leq (\N v + \N w ) ^{2}
	\end{align*}
\end{proof}
\subsection{La methode de Gram Schmidt}
Pour $\eng .$ un produit scalaire, on a 
\[ 
\forall v \in V \setminus \left\{ 0 \right\} , \eng { v,v} \neq 0
\]
\begin{lemma}
soit $V$ un espace euclidient et soient $v_1, \ldots, v_n$ deux-a-deux orthogonaux. Soit $v \in V$, il existe $a_1, \ldots, a_n \in \mathbb{R}$ uniques tel que
\[ 
v - a_1v_1 - \ldots - a_n v_n
\]
est orthogonal a chaque $v_i$
\end{lemma}
\begin{proof}
\[ 
\eng { v - \sum_{i=1}^{ n}a_i v_i, v_j} = \eng { v,v_j} - \eng {  \sum_{i=1}^{ n}a_i v_i,v_j} = \eng { v,v_j} - a_j \eng { v_j,v_j} 
\]
On peut donc poser $a_j = \frac{\eng { v,v_j} }{\eng { v_j,v_j} } $

\end{proof}
\subsection*{Le procede de Gram-Schmidt}
Soit $V$ un espace vectoriel euclidien  et  $\left\{ v_1, \ldots, v_n \right\}$.\\
Il existe un ensemble libre  $ \left\{ u_1, \ldots, u_n \right\}  $ tel que
\begin{enumerate}
\item $\eng { u_i, u_j} =0 \forall i \neq j$ 
\item $\forall k \in \left\{ 1, \ldots, n \right\} $ :
	\[ 
		span \left\{ v_1, \ldots, v_k \right\} = span \left\{ u_1, \ldots, u_k \right\} 
	\]
	
\end{enumerate}
Pour ceci, on itere sur tous les elements de $ \left\{ v_1, \ldots, v_n \right\} $, on pose
\begin{align*}
	u_1&= v_1\\
	u_2 &= v_2 - \frac{\eng { v_2,u_1} }{\eng { u_1,u_1} } \cdot u_1\\
	    & \vdots\\
	u_3 &= v_2 - \frac{\eng { v_2, u_1} }{\eng { u_1,u_1} }\cdot u_1 - \frac{\eng {  v_3,u_2}}{\eng { u_2,u_2} } u_2	
\end{align*}
etc.\\
Pour $i \in \left\{ 1, \ldots, k \right\} $ :
\[ 
u_1 = v_i - \sum_{j=1}^{ i-1} \frac{\eng { v_i,u_j} }{\eng { u_j,u_j} } u_j
\]
Par induction, on demontre que
	\[ 
		span \left\{ v_1, \ldots, v_i \right\} = span \left\{ u_1, \ldots, u_{i-1} , v_i \right\} 
	\]
Or $u_i$ est combinaison lineaire des autres elements de la famille.
\begin{crly}
Soit $A \in \mathbb{R}^{m \times n}$ une matrice de rang-colonne plein.\\
On peut factoriser $A$ comme
\[ 
A = A' \cdot
\begin{pmatrix}
	1 & \ldots & \mu_{ij} \\
	\vdots & \ddots &\\
	0 & & 1
\end{pmatrix}
\]
Tel que $A'$ est compose de colonnes 2-a-2 orthogonales pour le produit scalaire standard.

\end{crly}
\begin{proof}
	Pour $a_i$ les colonnes de $A$, Gram-Schmidt donne 
\[ 
a_{i} ' = a_i - \sum_{j=1}^{ i-1} \frac{\eng { a_i, a_j;} }{\eng { a_j ', a_j'} }a_j '
\]
Donc
\begin{align*}
a_i = \sum_{j=1}^{ i-1} \frac{\eng { a_i, a_j'} }{\eng { a_j', a_j'} }\cdot a_j' + a_i' \Rightarrow A = A' \cdot 
\begin{pmatrix}
	1 & \ldots & \mu_{ij} \\
	\vdots & \ddots &\\
	0 & & 1
\end{pmatrix}
\end{align*}


\end{proof}




					





\end{document}	
