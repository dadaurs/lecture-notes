\documentclass[../main.tex]{subfiles}
\begin{document}
\lecture{1}{Tue 23 Feb}{Introduction}
\section{Polynomes}
\begin{defn}[Centre d'un anneau]\label{defn:Centre d'un anneaucentre_d_un_anneau}
	Le centre $Z( R) $ est l'ensemble des elements $x$ satisfaisant
	\[ 
	\left\{ x \in R | ra = ar \forall a \in R \right\} 
	\]
	
\end{defn}
\begin{defn}[Diviseurs de 0]
	$a$ est un element non nul d'un anneau $R$ satisfaisant qu'il existe $b\in R$ tel que $ab=0$ ou  $ba=0$.

\end{defn}
\begin{defn}[Anneau integre]\label{defn:Anneau integreanneau_integre}
	Si un anneau est commutatif et n'a pas de diviseurs de 0, alors l'anneau est integre.
\end{defn}
\begin{thm}
	Soit $R$ un anneau, alors il existe un anneau $S \supseteq R$ ( $R$ est un sous-anneau) et $\exists x \in S \setminus R$ tel que 
	\begin{itemize}
		\item $ax=xa$, $\forall a \in R$ 
		\item Si $a_0 + \ldots + a_n x^{n} =0$ et $a_i \in R \forall i$ alors $a_i=0 \forall i$
	\end{itemize}
	Cet $x$ est appele indeterminee ou variable.
	
\end{thm}
\begin{defn}[Polynome]\label{defn:Polynomepolynome}
	Un polynomer sur $R$ est une expression de la forme
	\[ 
		p( x) = a_0 + \ldots + a_n x^{n}
	\]
	ou $a_i$ est le i-eme coefficient de $p( x)$.\\
	$ R[x]$ est l'ensemble des polynomes sur $R$.
\end{defn}
%\[ 
	%S \supseteq R[x] \supseteq R
%\]
\begin{thm}
	$R[X]$ est un sous-anneau.  $R$ est sans diviseurs de $0$ $\Rightarrow$ $R[X]$ est sans diviseurs de 0.\\
	De meme, si $R$ est commutatif, $R[x]$ aussi.
\end{thm}
\begin{proof}
	Soit $f( x) = \sum a_i x_i, g( x ) = \sum b_i x^{i} $ de degre $n$ resp. $m$.
	\[ 
		f( x) + g( x) = \sum_{i=1}^{ \max( m,n) }( a_i +b_i) x^{i}
	\]
De meme, on a
\[ 
	f( x) \cdot g( x) = a_0b_0 + \ldots = \sum_{k=0}^{m+n} \left( \sum_{i+j=k} a_i b_j \right) x^{k}
\]
Donc $R[X]$ est stable pour $+,\cdot$ et donc immediatement pour $-$, donc $R[X]$ est un sous-anneau de $S$.\\
Soient $f( x) ,g( x) \neq 0$ et $n=\max \left\{ i:a_i=0 \right\} $, le $m+n$-ieme coefficient de $f( x) g( x) $ est $a_n b_m$ et donc si  $R$ est integre,$R[x]$ l'est aussi.
\end{proof}
\begin{defn}[Degre d'un polynome]\label{defn:Degre d'un polynomedegre_d_un_polynome}
	Soit $f( x) = a_0 + \ldots \in R[X]$, $f( x) \neq 0$. On definit
\[ 
	\deg( f) = \max \left\{ i: a_i=0 \right\} 
\]
Ce dernier terme s'appelle le coefficient dominant de $f$, de plus on definit
\[ 
	f( x) =0: \deg( f) = - \infty 
\]
Si $\deg(f) =0$, alors $f$ est une constante.
\end{defn}
\begin{thm}
	Soit $R$ un anneau, $f,g \in R[X] \neq 0$ tel que au moins un de leur coefficients dominants de $f$ ou de $g$ ne sont pas des diviseurs  de 0.Alors $\deg( f\cdot g) = \deg( f) + \deg(  g )$
\end{thm}
\begin{proof}
	Soit $f( x) = a_0 + \ldots, g( x) = b_0 + \ldots$,$\deg f = n, \deg g = m$. Le $n+m$ ieme coefficient de $f\cdot g = a_n \cdot b_m \neq 0$
\end{proof}
Soit $p( x) \in R[x]$, ce polynome induit une application $f_p:R \to R$, on ecrit aussi $p( r) $
\begin{thm}
Soit $K$ un corps et $r_0,r_1,\ldots,r_n \in K$ des elements distincts et soient $g_0,\ldots,g_n \in K$.\\
Il existe un seul polynome $f\in K[x]$ tel que
\begin{enumerate}
\item $\deg f \leq n$ 
\item $f( r_i) = g_i$
\end{enumerate}

\end{thm}
\begin{proof}
On cherche $a_0,\ldots a_n$ tel que
\[ 
a_0 + a_1r_i + \ldots a_n r_i^{n}=g_i
\]
Donc, on cherche
\begin{align*}
\begin{pmatrix}
	1 & r_0 &\ldots & r_0^{n}\\
	\vdots & \ldots & \ldots & \ldots
\end{pmatrix}
\begin{pmatrix}
a_0\\
a_1\\
\ldots
\end{pmatrix}
= \begin{pmatrix}
g_1\\
\ldots\\
\ldots
\end{pmatrix}
\end{align*}
Il faut donc montrer que la matrice ci-dessus a un determinant non nul.\\
On le montre par induction sur $n$.\\
Dans le cas $n=0$, le determinant vaut trivialement 1.\
Dans le cas $n>0$, on a
\begin{align*}
	\det
\begin{pmatrix}
	1 & 0 \ldots\\
	1(r_1-r_0) & \ldots\\
	\ldots & \ddots\\
	1( r_n-r_0) & \ldots
\end{pmatrix}
= ( r_1-r_0) ( r_2-r_0)   \ldots \det( V( r_1, \ldots, r_n) ) \neq 0
\end{align*}
\end{proof}

\end{document}
