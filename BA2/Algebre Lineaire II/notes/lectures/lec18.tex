\documentclass[../main.tex]{subfiles}
\begin{document}
\lecture{18}{Wed 28 Apr}{Minimisation de la norme de Frobenius}
Etant donne $A \in \mathbb{R}^{m\times n}, k \in \mathbb{N}$, on veuut trouver $B\in \mathbb{R}^{m\times n}$ tel que $rang( B) \leq k$ et
\[ 
min_{C\in \mathbb{R}^{m\times n}, rang C \leq k} \N { A-C} _F
\]
est atteint a $B$.\\
\begin{defn}[Norme de Frobenius]
	Soit $A \in \mathbb{R}^{m\times n}$,
	\[ 
		\N { A} _F =\sqrt{  \sum_{i,j} a_{i,j} ^{2} }
	\]
		
\end{defn}
\begin{defn}[Trace]
	$A \in K^{n\times n}$, la trace de $A$ est definie par
	\[ 
		Tr( A) = \sum_{i=1}^{ n}a_{ii} 
	\]
	
\end{defn}

\begin{lemma}
On a 
\[ 
	Tr( A\cdot B) = Tr( B\cdot A) 
\]
pour toute matrices dans $K^{n\times n}$
\end{lemma}
\begin{proof}
	\begin{align*}
		( AB) _{ii} &= \sum_{k=1}^{ n}a_{ik} b_{ki} \\
		Tr( AB) &= \sum_{i=1}^{ n} \sum_{k=1}^{ n}a_{ik} b_{ki} \\
		&= \sum_{k=1}^{ n} \sum_{i=1}^{ n} b_{ki} a_{ik} =Tr ( BA) 
	\end{align*}
	

\end{proof}
\begin{lemma}
Soit $A \in \mathbb{R}^{m\times n}$, alors
\[ 
\N { A} _F ^{2} = \sum_{i=1}^{ r} \sigma_i ^{2}
\]
ou $\sigma_i$ sont les valeurs singulieres.
\end{lemma}
\begin{proof}
\begin{align*}
	\N { A} _F^{2} = Tr( A^{T}A) \\
	= Tr ( Q^{T} D^{T}P^{T}P D Q) \\
	= Tr( Q^{T}D^{2}Q) \\
	= Tr ( D^{2}) = \sum \sigma_i^{2}
\end{align*}

\end{proof}
On veut donc trouver $B\in \mathbb{R}^{m\times n}$ tel que 
\begin{itemize}
\item $rang B \leq k$ 
\item $\sum \N { a_i -b_i}^{2} $ est minimale
\end{itemize}
Pour $A \in \mathbb{R}^{m\times n}$, $A= PDQ$, avec $P= ( v_1,\ldots,v_m) $ et $Q = \begin{pmatrix}
u_1^{T}\\ \vdots\\ u_m^{T}
\end{pmatrix} $.\\
Rappel: le span $ \left\{ u_1,\ldots,u_k \right\} $ minimise
\[ 
	\sum_{i=1}^{ m}d( \alpha_i,H) ^{2}
\]
\begin{defn}
	On definit
	\[ 
	A_k = \sum_{i=1}^{ k}v_i \sigma_i u_{i} ^{T}
	\]
\end{defn}
Clairement $rang ( A_k ) \leq k$.
\begin{lemma}
	Les lignes de $A_k$ sont les projections des lignes correspondantes de $A$ dans le $ span \left\{ u_1,\ldots,u_k \right\} $.\\
\end{lemma}
\begin{proof}
Soit $a^{T}$ une ligne de $A$.\\
La projection 
\[ 
\tilde A^{T} = \sum_{i=1}^{ k}( a^{T}u_i ) u_i^{T}
\]
Alors les projections de toutes les lignes de  $A$ sont
\[ 
\sum_{i=1}^{ k}Au_i u_i^{T} = \sum_{i=1}^{ k}\sigma_i v_i u_i ^{T}= A_k		
\]
\begin{thm}
Soit $ B\in \mathbb{R}^{m\times n}, rang B  \leq k$ alors
\[ 
\N { A- A_k} _F^{2} \leq  \N { A-B} _F^{2}
\]

\end{thm}
\begin{proof}
Soit $A = \begin{pmatrix}
a_1^{T}\\\vdots\\a_m^{T}
\end{pmatrix} $, $B\begin{pmatrix}
b_1^{T}\\\vdots\\b_m^{T}
\end{pmatrix}$ et $A_k = \begin{pmatrix}
\tilde a_1^{T}\\\vdots\ \tilde a_m^{T}
\end{pmatrix} $.\\
Soit $H = span \left\{ b_1,\ldots,b_k' \right\} $ sont une base de l'espace engendre par les lignes de $B$, alors
\[ 
	\N { A-B} _F^{2} = \sum_{i=1}^{ m}\N { a_i-b_i} ^{2} \geq \sum_{i=1}^{ m}d( a_i,H) ^{2}
\]
Soit $\tilde H = span \left\{ u_1,\ldots,u_k \right\} $, alors
\[ 
	\sum_{i=1}^{ m}d( a_i,H) ^{2} \geq \sum_{i=1}^{ m}d( a_i, \tilde H) ^{2}= \sum_{i=1}^{ m}\N { a_i - \tilde a_i}_F ^{2 }
\]


\end{proof}



\end{proof}

	

	


\end{document}	
