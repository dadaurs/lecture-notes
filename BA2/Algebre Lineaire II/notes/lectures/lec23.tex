\documentclass[../main.tex]{subfiles}
\begin{document}
\lecture{23}{Tue 18 May}{forme normale de Jordan}
On cherche une base tel que les $N_i$ de la decomposition en forme de Jordan aient des 1 dans la diagonale superieure.
Ainsi, on aura bel et bien que $DN=ND$.
 \begin{thm}
	 Soit $N \in K^{n\times n}$ une matrice nilpotente.\\
	 Alors, il existe $P \in K^{n\times n}$ tel que
	 \[ 
	 P^{-1}NP
	 \]
	 est en forme normale de Jordan.
	
\end{thm}
Alors, en utilisant ce resultat, on pose
\[ 
\begin{pmatrix}
	P_1^{-1} & & \\
		 & \ddots &\\
		 & & P_k^{-1}
\end{pmatrix} P^{-1} AP \begin{pmatrix}
	p_1 & & \\
		 & \ddots &\\
		 & & P_k
\end{pmatrix}
\]
Dans les blocs, on aura donc des elements de la forme
\[ 
	P_i^{-1}( \lambda_i \id +N_i) P_i
\]
\begin{thm}
\end{thm}
Soit $N \in K^{n\times n}$ nilpotente, il existe une base $B$ de $K^{n}$ de la forme
\[ 
x_1,Nx_1 , \ldots, N^{m_1-1}x_1, \ldots N^{m_n-1}x_n
\]
tel que $N^{m_i}x_i =0$.
En inversant l'ordre de la base, on obtient la matrice de passage desiree.
\begin{proof}
Pour $x \in K^{n\times n}\setminus \left\{ 0 \right\} $, on appelle la duree de vie de $x:$ 
\[ 
\min_{ N^{j}x\neq 0} = m_x
\]
On appelle l'orbite de $x$ :
\[ 
x, Nx, \ldots, N^{m_k-1}x
\]
Posons $E$ pour la concatenation des orbites.\\
\underline{Invariante}:\\
On va maintenir $x_1, \ldots, x_k \in K^{n}$ tel que les orbites concatenees des $x_i$ engendrent $K^{n}$.\\
Posons $x_i =e_i, k=n$, on a clairement que la concatenation des orbites engendre $K^{n}$.\\
\begin{itemize}
\item Si $E$ est lineairement dependante on va remplacer un $x_i$ par $y$ d'une telle maniere que l'invariante est satisfaite.
	Ou bien effacer un $x_i$ tel que l'invariante est satisfaite.
\end{itemize}
On suppose $E$ lineairement dependant, alors
\[ 
	\beta_0^{1}x_1 + \ldots + \beta_{m_{x_1} -1} ^{1}N^{m_{x_1} -1} + \ldots + \beta_{m_{x_k} -1} N^{m_{x_k} }x_k =0 \quad ( *) 
\]
est une cl non triviale.
\begin{itemize}
\item Cas 1: $\exists i $ tel que $m_{x_i} =1$, $\beta_0^{1}\neq 0$.\\
	Alors l'orbite de $x_i$ est dans le span des orbites de $x_1, \ldots x_{i-1} ,x_{i+1} , \ldots, x_k$\\
	Alors on efface $x_i$

\item Cas 2: Maintenant,  on applique  $N$ a  $( *) $ tel que pas tous les $\beta_i^{j}N^{k}N^{i}x_j=0$
\end{itemize}
Cette demarche nous donne un sous-ensemble $J \subset [ k] $ et $\gamma_j \neq 0, j \in J$ tel que
\[ 
\sum_{j \in J}^{ } \gamma_j N^{m_{x_j} -1}x_j=0
\]
Soit $m= \min_{j \in J} m_{x_j} $ et $i \in J$ un index ou le min est atteint.
Alors on a
\[ 
	0 = N^{m_{x_i} -1} \left( \sum_{j\in J, j\neq i}^{ } \gamma_j N^{m_{x_j} -1 - m_{x_i} + 1} x_j + \gamma_i x_i\right) 
\]

				

	
\end{proof}

	


\end{document}	
