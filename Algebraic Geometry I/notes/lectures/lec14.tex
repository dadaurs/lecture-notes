\documentclass[../main.tex]{subfiles}
\begin{document}
\lecture{14}{Mon 28 Nov}{Twisting sheaves}
\begin{defn}
	Let $A$ be a graded ring and $M$ a graded $A$-module.\\
	Let $X= Proj A$
	\begin{enumerate}
		\item The structure sheaf  $\O_X$ on $X$ is the unique sheaf of rings on $Proj A$ such that $\O_X|_{D_+( a) } =\widetilde{A_{( a) } }$ 
		\item The sheaf of $\O_X$-modules associated to $M$ is the unique sheaf $\widetilde{M}$ such that $\widetilde{M}|_{D_+( a) } = \widetilde{M_{( a) }} $ 
		\item The $n$-th (Serre) twisting sheaf on $Proj A$ is $\widetilde{A( n) }$ where $A( n) = \bigoplus_{d\in \mathbb{Z}} A( n) _d$, where $A( n)_d= A_{n+d} $ 
		\item The $n$-th twist $\widetilde{M( n) }$ 	of $\widetilde{M}$ is defined as $\widetilde{M( n) }$ where we shift the grading as above.
	\end{enumerate}
\end{defn}
$Proj A$ is is a scheme and $\widetilde{M} \in QCoh( Proj A) $ 
\begin{crly}
Let $A$ be a graded ring, $M$  a graded $A$-module, $X= Proj A$.\\
For every $p\in D_+( b) \subset D_+( a) \subset Proj A$.\\
We find that $\widetilde{M}_p\simeq M_{( p) } $ 
\end{crly}
\begin{proof}
	By definition of $\tilde M$ we have $\widetilde { M}|_{D_+( a) } = \widetilde{M_{( a) } }$.\\
	We need to check that $M( p) = \lim_{p \in D_+( a) } M_{( a) } $ 
\end{proof}
\begin{rmq}
	Thinking of $\O_{Proj A} $ and $\widetilde{M}$ as their own sheafification, we obtain descriptions $\O( U) $ and $\widetilde{M}( U) $ similar to the description of the structure sheaf.
\end{rmq}
\begin{defn}
	Let $A_0$ be a ring, $n \geq 1$, $A= A_0[x_0,\ldots,x_n]$.\\
	Equip $A$ with the standard grading.\\
	The scheme $Proj A$ is called Projective $n$-space over $A$ and is denoted $\mathbb{P}^{n}_{A_0} $ \\
\end{defn}
\begin{rmq}
For any ring $A_0$, $\mathbb{P}^{n}_{A_0} \simeq \mathbb{P}^{n}_{ \mathbb{Z}} \times_{ \spec \mathbb{Z}} \spec A_0$.\\
If $X$ is any scheme, we define $ \mathbb{P}^{n}_X = \mathbb{P}^{n}_{\mathbb{Z}} \times_{\spec \mathbb{Z}} X$ 
\end{rmq}
\begin{rmq}
	Consider $\mathbb{P}^{n}_{A_0} = Proj A_0[x_0,\ldots,x_n]$.
	\begin{enumerate}
	\item $D_+( x_i) \simeq \spec A_{( x_i) } = \spec A_0 [ \frac{x_0}{x_i},\ldots] \simeq \mathbb{A}^{n}_{A_0} $ 
	\item If $A_0=K$ is a field, then $V_+( x_i) \subset \mathbb{P}^{n}_{A_0} $ with its reduced induced scheme structure is isomorphic to $ \mathbb{P}^{n-1}_{K} $ 
	\end{enumerate}
	
\end{rmq}
\begin{defn}
Let $A$ be a graded ring, then the affine cone over $Proj A$ is $\spec A$.
\end{defn}
\begin{defn}
	A graed ring $A$ is finitely generated in degree $1$ if it is generated by finitely many elements of $A_1$  as an $A_0$ algebra.
\end{defn}
\begin{lemma}
Assume $A$ is f.g. in degree in degree 1, then there exists $n>0$ and a closed immersion $Proj A \to \mathbb{P}^{n}_{A_0} $ over $\spec A_0$.
\end{lemma}
\begin{proof}
	There exists a graded surjection $A_0[x_1,\ldots,x_n]\to A$ and then apply an exercise.
\end{proof}
\begin{rmq}
If $A$ is f.g. over $A_0$, then $A^{( n) }= \bigoplus A^{( n) }_d$ with $A^{( n) }_d = A_{nd} $ is f.g. in degree $1$ for a suitable $n$.\\
By an exercise on the next sheet $Proj A\simeq Proj A^{( n) }$.
\end{rmq}

\begin{thm}
	Let $A$ be a graded ring, f.g. in degree $1$ then $\pi_A: Proj A \to \spec A_0$ is proper.
\end{thm}
\begin{propo}
We know that $\mathbb{P}^{n}_{A_0} = \simeq \mathbb{P}^{n}_{\mathbb{Z}} \times_{\spec \mathbb{Z}} \spec A_0$ and that there is a closed immersion $ Proj A \to \mathbb{P}^{n}_{A_0} $.\\
Since proper is stable stable under base change and ( COMP) and closed immersions are proper, it suffices to show $\pi: \mathbb{P}^{n}_{\mathbb{Z}} \to \spec \mathbb{Z}$ is proper.\\
We want to show that $\pi$ is separated, of finite type and satisfies the existence in the valuative criterion.\\
\subsection*{$\pi$ is separated}
Suffices to show that $D_+( x_i) \cap D_+( x_j) \to D_+( x_i) \times_{\spec \mathbb{Z}} D_+( x_j) $ is a closed immersion.\\
We have to show that $ \mathbb{Z}[ \frac{x_0}{x_i},\ldots, \frac{x_n}{x_i}]\otimes \mathbb{Z}[ \frac{x_0}{x_j},\ldots, \frac{x_n}{x_j}]\to \mathbb{Z}[ \frac{x_0^{2}}{x_ix_j}, \frac{x_0x_1}{x_ix_j}\ldots ]$ is surjective, which is obvious.
\subsection*{$\pi$ is of finte type}
$ \mathbb{P}^{n}_{ \mathbb{Z}} = \bigcup D_+( x_i) = \bigcup \spec \mathbb{Z}[ \frac{x_0}{x_i},\ldots, \frac{x_n}{x_i}]$ and these specs are of finite type.
\end{propo}
	



\end{document}	
