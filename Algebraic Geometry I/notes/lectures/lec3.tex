\documentclass[../main.tex]{subfiles}
\begin{document}
\lecture{3}{Mon 17 Oct}{Kernels/cokernels of sheaves}
\subsection{Kernels, cokernels, exactness}
In this chapter, every (pre)-sheaf is a (pre)sheaf of Abelian groups.\\
\begin{defn}[Subsheaf]
	Let $\F$ be a (pre)sheaf on $X$.\\
	Then a sub(pre)sheaf of $\F$ is a (pre)sheaf $\G$ such that $\G( U) \subset \F( U) $ for every open and the restriction maps are induced by $\F$.
\end{defn}
\begin{defn}[Kernel, cokernel of presheaves]
Let $\phi:\F\to \G$ be a morphism of sheaves
\begin{enumerate}
\item The presheaf kernel of $\phi$ is the presheaf $\ker^{pre}( \phi) $ defined by $\ker^{pre}( \phi) ( U)= \ker( \phi( U) )  $ 
\item The presheaf image is defined as $\im^{pre}( \phi) ( U) = \im ( \phi( U) ) $ 
\item The presheaf cokernel is $\coker^{pre}( \phi) ( U) = \coker( \phi( U) ) $.
\end{enumerate}
In each case, the restriction maps are induced by those in of $\F$ or $\G$.

\end{defn}
\begin{lemma}
If $\F$ and $\G$ are sheaves, then the presheaf kernel is a sheaf.
\end{lemma}
\begin{proof}
Let $U \subset X$ open and $U = \bigcup U_i$ an open cover, $s_i \in \ker^{pre}( \phi)( U_i)  $ such that $s_i|_{U_i\cap U_j }= s_j |_{U_i\cap U_j}  $.\\
Since $\F$ is a sheaf, $\exists s \in \F( U) $ such that $s|_{U_i} = s_i$.\\
Since $\ker^{pre}( \phi) ( U_i) = \ker( \phi( U_i) ) $, we have $\phi( U_i) ( s_i) =0$.\\
Thus
\[ 
\phi( U) ( s) |_{U_i} = \phi( U_i) ( s|_{U_i} ) = 0
\]
Since $\G$ is a sheaf, $\phi( U) ( s) =0 \implies s \in \ker^{pre}( \phi) ( U) $.
\end{proof}
\begin{exemple}
By an exercise, the image presheaf and cokernel presheaf are, in general, no sheaves, even if $\F$ and $\G$ are.
\end{exemple}
\begin{defn}[Cokernel/image of morphisms of sheaves]
	Let $\phi:\F\to \G$ be a morphism of sheaves
	\begin{enumerate}
	\item sheaf kernel: $\ker^{pre}( \phi) $ 
	\item sheaf image $( \im ^{pre}( \phi) )^{+}$
	\item sheaf cokernel $( \coker ^{pre}( \phi) )^{+}$
	\end{enumerate}
	
\end{defn}
\begin{lemma}[cokernels are cokernels]
Let $\phi:\F\to \G$ be a morphism of sheaves
\begin{enumerate}
\item $\ker\phi\to \F$ is a categorical kernel in $Sh( X) $ 
\item $\G\to \coker\phi$ is a categorical cokernel in $Sh( X) $.
\end{enumerate}
\end{lemma}
\begin{proof}
	\begin{enumerate}
	\item 
This means that for each commutative diagram with solid arrows, the dotted arrow is unique\\
"Insert cokernel/kernel diagram here"\\
This holds for every open $U$ and so the kernel is a sheaf.
\item The appropriate diagram commutes and we use the universal property of sheafification.
	\end{enumerate}
\end{proof}
\begin{propo}
Let $\phi:\F\to \G$ be a morphism of sheaves of abelian groups, then the following are equivalent
\begin{enumerate}
\item $\phi$ is a monomorphism in $Sh( X) $ 
\item $\ker( \phi) = 0 $ 
\item $\ker ( \phi( U) ) =0$ 
\item $\ker( \phi_x) = 0$ 
\end{enumerate}
\end{propo}
\begin{proof}
Recall $\phi$ is a monomorphism if for every $\psi: \F' \to \F , \phi\circ\psi = 0 \implies \psi= 0$ .\\
The implication $1 \implies 2$ follows by applying the monomorphism property to $\ker \phi \to \F$ 
$2 \implies 1$ If $\phi\circ\psi= 0$, then $\psi$ factors through the kernel $\ker\phi\to \F$ and so $\psi = 0$ \\
$2 \iff 3$ Since $\ker ( \phi) ( U) = \ker ( \phi( U) ) $ \\
$3 \implies 4$  Follows because taking direct limits is exact.\\
$4 \implies 3$ Let $s\in \F( U) $ with $\phi( U) ( s) =0$, then $\phi_x( s_x) = ( \phi( U) ( s) ) _x = 0$.\\
So $s_x =0 \forall x\in U$ and so $s =0$ 
\end{proof}
\begin{propo}
Let $\phi: \F\to \G$ be a morphism of sheaves of abelian groups, then the following are equivalent
\begin{enumerate}
\item $\phi$ is an epimorphism in $Sh( X) $ 
\item $\coker ( \phi) =0$ 
\item $\coker( \phi_x) =0$ 
\end{enumerate}
\end{propo}
\begin{proof}
Recall that $\phi$ is an epimorphism if for every $\psi:\G\to \G', \psi\circ\phi = 0 \implies \psi =0$ \\
$1\implies 2 $ Apply epimorphism property to $\G \to \coker( \phi) $ \\
$2\implies 3$ We have
\begin{align*}
0 &= ( \coker \phi)_x\\
&= ( \coker^{pre}\phi) _x = \coker ( \phi_x) 
\end{align*}
$3 \implies 1$ \\
Let $\psi: \G \to \G'$ such that $\psi\circ\phi=0$, this implies that $0 = ( \psi\circ\phi)_x = \psi_x \circ\phi_x$.\\
Since $\phi_x$ is an epimorphism of abelian groups, we get $\psi_x=0$.\\
As the hom sheaf is a sheaf, we get that $\psi =0$ 

\end{proof}
\begin{rmq}
If $\coker( \phi( U) ) =0 \forall U \subset X\implies \coker( \phi) =0$ but the converse is not true.
\end{rmq}
\begin{crly}
If $\phi:\F\to \G$ is a morphism of sheaves, then the following are equivalent
\begin{enumerate}
\item $\phi$ is an isomorphism
\item $\phi( U) $ is an isomorphism $\forall U \subset X$ open
\item $\phi_x$ is an isomorphism $\forall x \in X$ 
\end{enumerate}
\end{crly}
\begin{proof}

$1\implies 2$ since taking sections is a functor\\
$2\implies 3$ since taking limits is functorial\\
$2\implies 1$ because $( \phi( U) )^{-1}$ defines a morphism of sheaves\\
$3\implies 2$ Need to show surjectivity of $\phi( U) $.\\
Let $t\in \G( U) $, since $\phi_x$ is an isomorphism $\forall x \in U$, we find $s_x \in \F_x$ such that $\phi_x(s_x)= t_x $.\\
There exists an open neighbourhood and $s_{V_x} \subset \F( V_x) $ such that $( s_{V_x})_x = s_x$\\
Since 
\[ 
	( \phi( V_x) ( s_{V_x} ) ) _x = t_x
\]
we can choose $V+x$ sufficiently small such that $\phi( V_x) ( s_{V_x} ) = t|_{V_x} $.\\
Since $\phi( V_x\cap V_y) $ is injective and $\phi( V_x\cap V_y) ( s_{V_x} |_{V_x\cap V_y} ) = t |_{V_x\cap V_y} = \phi( V_x\cap V_y) ( s_{V_y} |_{V_x\cap V_y} ) $, we have $s_{V_x} |_{V_x\cap V_y} = s_{V_y} |_{V_x\cap V_y} $.\\
Thus there exists $s\in \F( U) $ such that $s|_{V_x} = s_{V_x} $ and $\phi( U) ( s) |_{V_x} = t|_{V_x} $ and thus $\phi( U) ( s) =t$.
\end{proof}
\begin{defn}[Exact Sequence of sheaves]
	A sequence of sheaves $\F_1\xto { \phi_1} \F_2 \xto{\phi_2} \F_3$  is called exact if $\ker \phi_2= \im \phi_1$ 
\end{defn}
\begin{crly}
A sequence of sheaves is exact iff the associated sequence on stalks is exact for all points.
\end{crly}











			


	

	
\end{document}	
