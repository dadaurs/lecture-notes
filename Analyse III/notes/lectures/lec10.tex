\documentclass[../main.tex]{subfiles}
\begin{document}
\lecture{10}{Mon 25 Oct}{...}
\section{Formule de Cauchy}
Donne les valeurs d'une fonction holomorphe a l'interieur d'un lacet en terme des valeurs sur le lacet.
\begin{thm}[Formule de Cauchy]
	Soit $f: D( z,\rho) \to \mathbb{C}$ holomorphe.\\
	Soit $\gamma: [ 0,1] \to D( z,\rho) $ un lacet homotope dans $D( z,\rho) \setminus \left\{ z \right\} $ a $\del D( z,\epsilon) $ pour $\epsilon>0$, oriente dans le sens trigonometrique.\\
	Alors
	\[ 
		f( z) = \frac{1}{2\pi i}\oint_\gamma \frac{f( \zeta) }{\zeta-z}d\zeta
	\]
	
\end{thm}
\begin{crly}
	Soit $f:U\to \mathbb{C}$ holomorphe et $\gamma$ un lacet homotope dans $U\setminus z$ a $\del D( z,\epsilon) $ dans $z\in U$, alors
	\[ 
		f( z) = \frac{1}{2\pi i}\oint_\gamma \frac{f( \zeta) }{\zeta-z}d\zeta
	\]
\end{crly}
\begin{proof}
Par hypothese sur $\gamma$ et par holomorphie de $f$ 
\begin{align*}
	\frac{1}{2\pi i} \oint \frac{f( \zeta) }{\zeta-z} &= \frac{1}{2\pi i}\oint_{\del D( z,\epsilon) } \frac{f( \zeta) }{\zeta-z}d\zeta
\end{align*}
pour tout $\epsilon \in ( 0,\rho) $.\\
Faisons tendre $\epsilon\to 0$, on utilise
\[ 
	f( \zeta) = f( z) + f'( z) ( \zeta-z) + o( \zeta-z) 
\]
On a donc
\begin{align*}
	&\frac{1}{2\pi i}\oint_{\del D( z,\epsilon) } \frac{f( z) + f'( z) ( \zeta-z) + o( \zeta-z) }{\zeta-z} d\zeta\\
	=& f( z) + f'( z) \oint 1 d\zeta + \oint o( 1) d\zeta\\
	=& f( z) + 0 + o( \epsilon) 
\end{align*}
\subsection*{Consequences}
\begin{enumerate}
\item Analyticite\\
	\begin{thm}
		Soit $f: D( z_*,\rho) \to \mathbb{C}$ holomorphe, alors $f$ est analytique, donnee par 
		\[ 
			f( z) = \sum a_n ( z-z_*) ^{n}
		\]
		de rayon de convergence $ \geq \rho$ avec
		\[ 
			a_n = \frac{1}{2\pi i} \oint \frac{f( \zeta) }{( \zeta-z_* )^{n+1}}d\zeta
		\]
		
		
	\end{thm}
	\begin{proof}
	Supposons $z_*=0$, on a 
\begin{align*}
	f( z) = \frac{1}{2\pi i }\oint \frac{f( \zeta) }{\zeta-z}d\zeta
\end{align*}
Comme 
\[ 
\frac{1}{\zeta-z}= \frac{1}{\zeta} \sum_{n=0}^{ \infty }\frac{1}{\zeta^{n}}z^{n}
\]
On a 
\begin{align*}
	f( z) &= \frac{1}{2\pi}\oint_{\del D( 0,\rho) } \sum \frac{f( \zeta) }{\zeta^{n+1}}z^{n}d\zeta\\
	      &= \sum_{n=0}^{ \infty } ( \frac{1}{2\pi i}\oint \frac{f( \zeta) }{\zeta^{n+1}}d\zeta ) z^{n}
\end{align*}
	\end{proof}
\item 
Ainsi, $f$ holomorphe implique $f'$ holomorphe.

	
\end{enumerate}
\begin{thm}[Morera]
	Si $f:U \to \mathbb{C}$ continue satisfait $\forall \gamma$ contractible
	\[ 
		\oint_\gamma f( z) dz =0	
	\]
Alors $f$ est analytique.	
\end{thm}
\begin{proof}
	Soit $z_*\in U$ et $\epsilon>0$ tq $D( z_*, \epsilon) \subset U$.\\
	Comme tout lacet est contractible dans $D( z_*,\epsilon) $, et ainsi $f$ analytique. La condition de Morera dans $D( z,\epsilon)$ est satisfaite.
\end{proof}



\end{proof}
	

\end{document}	
