\documentclass[../main.tex]{subfiles}
\begin{document}
\lecture{23}{Mon 13 Dec}{Analyse vectorielle}
On se demande si un champ vectoriel derive d'un potentiel

\begin{propo}
Soit $x\in U$ et $T \subset \mathbb{R}^{2}$ un triangle, soit $T_{x,\epsilon} = x + \epsilon T$.\\
Alors si on orient $\del T_{x,\epsilon} $ dans le sens trigonometrique, alors
\[ 
\int_{ \del T_{x,\epsilon}  }^{  }X\cdot ds = Aire( T_{x,\epsilon} ) rot( X) 
\]

\end{propo}
\begin{crly}
On peut remplacer $T$ par tout polygone en le decoupant en triangles.
\end{crly}
\begin{rmq}
On voit que si $X$  a un potentiel $rot X =0$ 
\end{rmq}
En 3D, plus complique car il faut choisir comment representer des petits lacets
\begin{propo}
Soit $T \subset \mathbb{R}^{3}$ le  triangle avec sommets $0,v,w$ apparaissant dans cet ordre suivant l'orientation de $\del T$.\\
Alors 
\[ 
\int_{ \del T_{x,\epsilon}  }^{  } X ds = \frac{1}{2}\epsilon rot X \cdot ( v\times w) 
\]

\end{propo}
\subsection{Formule de Green-Riemann et Kelvin-Stokes}
Que dire de 
\[ 
\int_{ \gamma }^{  }X \cdot ds
\]
si $\gamma$ est le bord de $U \subset \mathbb{R}^{2}$ 
\begin{thm}[Green-Riemann]
	\[ 
	\int_{\gamma} X\cdot ds = \int_{ U }^{  }rot X du dv
	\]
	
\end{thm}






\end{document}	
