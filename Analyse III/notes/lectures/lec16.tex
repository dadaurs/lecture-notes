\documentclass[../main.tex]{subfiles}
\begin{document}
\lecture{16}{Mon 15 Nov}{Applications du theoreme des residus}
\subsection{2 classes d'integrales souvent calculables par des residus}
\begin{itemize}
\item Transformee de Fourier:
	\[ 
	\hat{f}( \xi) = \frac{1}{2\pi}\int_{ - \infty  }^{ \infty  } f( x) e^{ -i \xi x} dx
	\]

\item Integrales de la forme
	\[ 
	\frac{1}{2\pi}\int_{ 0 }^{ 2\pi }f( \sin( x) , \cos( x) ) dx
	\]
	
	
\end{itemize}
\subsection*{Motivation de la transformee de Fourier}
Il y a un theoreme qui dit que si $f$ est raisonnable
\[ 
f( x) = \int_{ - \infty  }^{ \infty  } e^{i\xi x} \hat{f}( \xi) d\xi
\]
\subsection*{Comment calculer une transformee de Fourier}
Idee: Calculer pour un lacet demi-cercle de rayon $R$.\\
Avec un peu de chance, la partie $ [ -R, R] $ quand $R\to \infty $ devrait nous donner $ \int_{ - \infty  }^{ \infty  }\ldots$ et avec un peu de chance l'integrale sur le demi-cercle $\to 0$.
\begin{propo}
Si $f$ meromorphe sur $\mathbb{C}$ avec un nombre fini de poles et $f( z) = O( \frac{1}{z}) $ quand $|z|\to \infty $, alors pour $\xi>0	$, alors
\[ 
\hat{f}( \xi) =i \sum_{ z_* \in \text{ poles inferieur } }^{ }	res_{z_*} ( e^{- i \xi z} f( z) )
\]
et on sommme sur les singularites du plan inferieur si $\xi <0$ 

\end{propo}
\begin{rmq}
Si $f( z) = o( \frac{1}{z}) $ marche aussi pour $\xi = 0$ 
\end{rmq}
\begin{propo}
Si on souhaite integrer $ \int_{ 0 }^{ 2\pi } R( \sin\theta,\cos\theta) d\theta $, on a
\[ 
\int_{ 0 }^{ 2\pi } R( \sin\theta,\cos\theta) d\theta  = 2\pi i \sum_{z\in D( 0,1), \text{ poles }  } res_{z_*} ( f)
\]
ou $f( z) = R( \frac{1}{2i}( z-\frac{1}{z}) , \frac{1}{2}( z+\frac{1}{z}) ) $ 

\end{propo}






\end{document}	
