\documentclass[../main.tex]{subfiles}
\begin{document}
\lecture{20}{Thu 02 Dec}{Application conforme de Riemann}
\begin{lemma}
Soit $\psi\in \Sigma_{\Omega,z_*} $, si $\psi$ n'est pas surjective, il existe $\xi\in \Sigma_{\Omega,z_*} $ tel que $\xi'( z_*)> \psi'( z_*)  $.
\end{lemma}
\begin{proof}
Soit $\alpha\in  \mathbb{D}\setminus \psi( \Omega) $, posons $\psi_0= \phi_\alpha\circ\psi$, comme $\psi_0$ ne s'annulle pas sur $\Omega$, il existe $\psi_1= \sqrt{\psi_0} $.\\
Posons $\beta= \psi_1( z_*) $ et choisissons $\theta$ tel que
\[ 
\psi_2= \phi_{\beta,\theta} \circ\psi_1\in \Sigma_{\Omega,z_*} 
\]
Posons $\xi= \psi_2$.\\
Pourquoi a-t'on $\xi'( z_*) >\psi'( z_*) $?\\
\[ 
\xi= \phi_{\beta,\theta} \circ \sqrt{\phi_{\alpha} \circ\psi} 
\]
Alors
\[ 
\psi = \phi_{-\alpha} \left( ( \phi_{- e^{i\beta} \theta,-\theta}\circ\xi ) ^{2} \right)
\]

Donc
\[ 
\psi'( z_*) = F'( 0) \xi'( z_*) 
\]
Pourquoi $F'( 0) <1$?\\
$F:\mathbb{D}\to \mathbb{D}$ et $F( 0) =0$.\\
Or $F$ n'est pas bijective, et donc, par le lemme de Schwarz, $F'( 0) < 1$

\end{proof}
\begin{defn}[Famille Normale]
	Soit $\Omega \subset \mathbb{C}$, une famille de fonctions $ \mathcal{F}$ est dite normale si $\forall ( f_n)_{n\in \mathbb{N}} $, il existe une sous-suite $f_{n_k} $ qui converge uniformement sur les compacts de $\Omega$.\\
	Soit $ \mathcal{F}$ une famille de fonctions holomorphes $\Omega\to \mathbb{C}$ uniformement bornee sur les compacts de $\Omega$. Alors $ \mathcal{F}$ est normale.
\end{defn}
\begin{proof}
Considerons $\Sigma_{\Omega,z_*} $.\\
Soit $f_n$ une suite de fonctions dans $\Sigma_{\Omega,z_*} $ tel que $f_n'( z_*) \to \sup_{f\in \Sigma_{\Omega,z_*} } f'( z_*) $.\\
Comme $\Sigma_{\Omega,z_*} $ est une famille normale il existe $f_{n_k} \to f$ uniformement sur les compacts.\\
$f$ est holomorphe et $f'( z_*) < \infty $.\\
Il reste seulement a montrer que $f\in \Sigma_{\Omega,z_*} $ et donc que $f$ est injective.\\
Soit $z\in \Omega$ et $\alpha= f( z) $, montrons qu'il n'existe pas de $w\in\Omega$ tel que $f( w) =\alpha$, donc que $f( z) -\alpha$ ne s'annule qu'en $z$.\\
$f$ n'est pas constante car $f'( z_*) >0$.\\
Donc les zeros de $f-\alpha$ sont isoles.\\
Soit donc $w\in \Omega\setminus \left\{ z \right\} $, montrons que $f( w) -\alpha$ est non nul.\\
On sait comme $f-\alpha$ qu'il existe un voisinage $D( w,\delta) $ tel que $D( z_2,\delta) \subset \Omega\setminus \left\{ z_1 \right\} $  et sur $\del D( w,\delta) $, $f-\alpha$ ne s'annulle pas.\\
On a que le nombre de zeros sur $D( w,\delta) $  est donne par
\[ 
\frac{1}{2\pi i}\int_{\del D( w,\delta) } \frac{f'( \zeta) }{f( \zeta)-\alpha }d\zeta= \lim_{k \to  + \infty} \frac{1}{2\pi i} \oint \frac{f'_{n_k} ( z) }{f_{n_k}( z) -f_{n_k} ( z_*)  } dz =0
\]
On a donc prouve l'existence, il reste l'unicite.
En effet, $( f_1\circ f_2^{-1} )'( 0) >0$ et $( f_1\circ f_2^{-1}) ( \mathbb{D}) = \mathbb{D}$ et donc $f_1\circ f_2= \id\implies f_1= f_2$ 
\end{proof}
\section{Fonctions Elliptiques}
Une fonction $f: \mathbb{C}\to \mathbb{C}$ est dite bi-periodique s'il existe $T_1,T_2\in \mathbb{C}$ $ \mathbb{R}$-lineairement independantes tel que $f( z + T_1) = f( z+ T_2) = f( z) \forall z$.\\
\begin{propo}
Une fonction holomorphe biperiodique est toujours constante
\end{propo}
\begin{proof}
\[ 
\sup_{z\in \mathbb{C}} f( z) = \sup_{z\in K} |f( z) | < \infty 
\]
ou $K$ est compact tel que $\forall z \in \mathbb{C}$ $\exists ( k_1,k_2) \in \mathbb{Z}^{2}$ tel que $z-k_1T_1- k_2T_2\in K	$.\\
Par Liouville $f$ constante.\\

\end{proof}
\begin{defn}
	Une fonction meromorphe biperiodique est dite elliptique.	
\end{defn}
		
	


\end{document}	
