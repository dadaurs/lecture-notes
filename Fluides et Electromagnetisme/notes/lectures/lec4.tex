\documentclass[../main.tex]{subfiles}
\begin{document}
\lecture{4}{Fri 05 Mar}{Interfaces solide/liquide}
\subsection{Interface solide/liquide/gaz}
On considere une goutte sur une surface.\\
En equilibre la somme des forces sur la ligne tripe est nulle.\\
Selon l'axe horizontal, on trouve
\[ 
\gamma_{sg} = \gamma_{sl} + \cos\theta \gamma_{lg} 
\]
Cette propriete s'appelle la loi de Young.
Si $0<\theta < 90$, on a un bon mouillage.\\
Si $\gamma_{sg} -\gamma_{sl} >\gamma_{lg}$, on a $\cos \theta >1$, cette situation est non-stationnaire.
On parle alors de mouillage total.\\
Si $-\gamma_{lg} < \gamma_{sg} -\gamma_{sl} <0$, alors $-1< \cos\theta <0$ et donc $90<\theta <180$, $\Rightarrow$ mauvais mouillage.\\
Si $\gamma_{sq} - \gamma_{sl} < -\gamma_{lg} $, alors $\cos \theta <-1$, on parle alors de super-hydrophobie ou effet lotus.
\subsection{Loi de laplace}
Notons qu'a l'interieur d'un ballon, il y a une surpression.\\
Et a l'interieur d'une goutte d'eau, d'une bulle de savon, \ldots ?\\
On suppose une goutte de liquide spherique en apesanteur.
Les forces s'appliquant sur la goutte donnent
\[ 
\sum F_i^{ext}= 0 = \vec{F}_{p_2} + \underbrace{\vec{F}_{p_1}}_{= \pi R^{2}p_1 \vec{e}_z} + \underbrace{\vec{F}_{\gamma}}_{=-2\pi R \gamma \vec{e}_z} 
\]
Pour $\vec{F}_{p_2} $, on a 
\[
	\vec{F}_{p_2} = \int \int - p_2 \vec{dS} = \int_{ 0 }^{ 2\pi } \int_{ 0 }^{ \frac{\pi}{2} }- p_2 \vec{e}_r R^{2} \sin\theta d \theta d\phi = \vec{e}_z \int_{ 0 }^{ 2\pi } \int_{ 0 }^{ \frac{\pi}{2} }- p_2 R^{2} \sin \theta \cos \theta d\theta d \phi = - \pi R^{2}p_{2} \vec{e}_z
\]
Donc
\[ 
	\sum F^{ext}= - 2 \pi R \gamma \vec{e}_z + \pi R^{2}p_1 \vec{e}_z
\]
Et donc
\[ 
p_1 - p_2 = \frac{2\gamma}{R}
\]




\end{document}	
