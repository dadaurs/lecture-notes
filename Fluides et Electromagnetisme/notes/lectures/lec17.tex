\documentclass[../main.tex]{subfiles}
\begin{document}
\lecture{17}{Fri 30 Apr}{Capacite et condensateurs}
\subsection{Capacite et condensateur}
\subsubsection{Definition de la capacite}
\begin{itemize}
\item Deux conducteurs initialement non-charge.
	\[ 
		\phi_A = \phi_B =0
	\]
	
\item Charge $-q$ enlevee de $A$ et placee sur $B$
\end{itemize}
On a 
\[ 
\phi_A- \phi_B = - \int_B^{A} \vec{E}\cdot \vec{dl}	>0	
\]
Or
\begin{align*}
	\vec{E}( \vec{r}) \text{ est proportionnel a } q
\end{align*}
car
\[ 
\nabla \cdot \vec{E} = \frac{\rho_{el} }{\epsilon_0}; \nabla\times \vec{E}=0
\]
et donc pour
\[ 
\phi_A-\phi_B
\]
aussi.\\
Et donc
\[ 
\frac{q}{\phi_A-\phi_B}= C = \text{ const. } 
\]
Souvent, on ecrit
\[ 
\phi_A-\phi_B = U
\]
On appelle cette constante $C$ la capacite entre les deux conducteurs.
\[ 
[ C] = [ \frac{Q}{U}] = \frac{C}{V}= \frac{C^{2}s^{2}}{kg m^{2}}= \frac{A^{2}s^{4}}{kg m^{2}}= \text{ Farad } =F
\]
Dans le vide, $C$ ne depend que de la geoetrie des deux conducteurs.
\subsubsection{Condensateur}
Un condensateur est un systeme de deux conducteurs avec charge $\pm q$.\\
Normalement, la geometrie est choisie pour maximiser la capacite.
\begin{figure}[H]
    \centering
    \incfig{condensateur-cylindrique}
    \caption{condensateur cylindrique}
    \label{fig:condensateur-cylindrique}
\end{figure}
Soit $l$ la longueur du condensateur, on suppose que $l>>R_2-R_1 \Rightarrow $ les effets de bord sont negligeables.\\
Pour appliquer, la loi de gauss, on considere un cylindre compris entre $R_1$ et $R_2$.
\begin{align*}
	\iint_S \vec{E}\cdot \vec{dS}= \frac{q}{\epsilon_0} \approx 2\pi r l E( r) 
\end{align*}
Donc
\[ 
	E( r) = \approx = \frac{q}{2\pi r\epsilon_0 l}
\]
Donc
\begin{align*}
	\phi_A- \phi_B &= - \int_{ R_2 }^{ R_1 }\vec{E}\cdot \vec{dl} = \int_{ R_1 }^{ R_2 } E( r) e_r ( - e_r) dl\\
		       &= \int_{ R_1 }^{ R_2 } E( r) dr = \frac{q}{2\pi \epsilon_0 l} \int_{ R_1 }^{ R_2 } \frac{dr}{r}	\\
		       &= \frac{q}{2 \pi \epsilon_0 l}\ln (  \frac{R_2}{R_1}) 	\\
	\Rightarrow  C &= \frac{q}{\phi_A-\phi_B}= \frac{2\pi \epsilon_0 l}{ \ln ( \frac{R_2}{R_1}) }
\end{align*}
\begin{figure}[H]
    \centering
    \incfig{condensateur-plan}
    \caption{condensateur plan}
    \label{fig:condensateur-plan}
\end{figure}
On suppose a nouveau $d<<a$ et $d<<b \Rightarrow $ effet de bord negligeable.
\[ 
\iint_S \vec{E}\cdot \vec{dS} = \frac{q}{\epsilon_0}\approx E S_c \Rightarrow  E = \frac{q}{\epsilon_0 S_c}
\]
\begin{align*}
	\phi_B - \phi_A = - \int_{ A }^{ B } \vec{E}\cdot \vec{dl} = - \int_{ 0 }^{ d }E( x) e_x e_x dx\\
	= - \int_{ 0 }^{ d }E dx = - Ed = \frac{- dq}{\epsilon_0 S_c}
\end{align*}
Donc $c= \frac{q}{\phi_A- \phi_B}= \frac{\epsilon_0S_c}{d}$




\end{document}	
