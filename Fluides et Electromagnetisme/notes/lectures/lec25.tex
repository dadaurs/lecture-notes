\documentclass[../main.tex]{subfiles}
\begin{document}
\lecture{25}{Fri 28 May}{...}
\section{Equations de Maxwell}
\subsection{Critique de l'equation d'Ampere}
La loi d'Ampere $\nabla \times \vec{B}= \mu_0 \vec{j}$ de la magnetostatique n'est pas valable en general $ \frac{\del}{\del t}\neq 0$ .\\
\begin{align*}
	\nabla\cdot ( \nabla\times \vec{B}) = 
	\begin{pmatrix}
	\del x\\\del y\\ \del z
	\end{pmatrix} \cdot 
	\begin{pmatrix}
	\del y B_z -\del_zB_y\\\del_zB_x-\del_xB_z\\\del_xB_u-\del_yB_x
	\end{pmatrix} =0
\end{align*}
Donc la divergence de la loi d'Ampere, on trouve que
\[ 
\nabla \cdot \vec{j}=0
\]
Mais si on combine cela avec l'equation de continuite de la charge electrique, on trouve que
\[ 
\frac{\del \rho_{el} }{\del t}=0
\]
Ce qui n'est pas toujours le cas.\\
On considere l'enclenchement d'un circuit RC.\\
On a
\begin{align*}
	\oint_{\gamma} \vec{B}\cdot \vec{dl} = \mu_0 \iint_{S} \vec{j}\cdot \vec{dS} = \mu_0 I( t) \text{ pour $S=S_1$  } \\
	\intertext{Cependant, si $S_2$ passe au milieu d'un condensateur}
	=0
\end{align*}
Ce qui est une contradiction.\\
Ainsi, la loi d'ampere n'est pas valable dans ce cas non statique.
\subsection{Courant de deplacement de Maxwell}
Maxwell a postule un nouveau terme dans la loi d'Ampere si $\frac{\del }{\del t}\neq 0$ :
\begin{align*}
\nabla\times \vec{B} = \mu_0 \vec{j} + \mu_0 \epsilon_0 \frac{\del \vec{E}}{\del t} \to \text{ Loi d'ampere maxwell } 
\end{align*}
On calcule donc
\begin{align*}
\nabla \cdot ( \nabla \times \vec{B}) = \mu_0 \nabla \cdot \vec{j} + \epsilon_0 \mu_0 \nabla \cdot \frac{\del \vec{E}}{\del t}\\
= \mu_0 \nabla \vec{j} + \epsilon_0 \mu_0 \frac{\del}{\del t}\nabla \cdot \vec{E}\\
= \mu_0 \nabla \vec{j} + \epsilon \mu_0 \frac{\del }{\del t}( \frac{\rho_{el} }{\epsilon_0}) \\
\Rightarrow \nabla \cdot \vec{j} + \frac{\del \rho_{el} }{\del t}=0	
\end{align*}
\subsection{Les equations de Maxwell}
Valable en general
\begin{align*}
\nabla \cdot \vec{E}= \frac{\rho_{el} }{\epsilon_0}\\
\nabla\times \vec{E}= - \frac{\del \vec{B}}{\del t}\\
\nabla\cdot \vec{B}=0\\
\nabla\times \vec{B}= \mu_0 \vec{j} + \epsilon_0 \mu_0 \frac{\del \vec{E}}{\del t}\\
F= q( \vec{E}+ \vec{v}\times \vec{B}) 	
\end{align*}



\end{document}	
