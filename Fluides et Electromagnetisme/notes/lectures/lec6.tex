\documentclass[../main.tex]{subfiles}
\begin{document}
\lecture{6}{Fri 12 Mar}{Equations de continuite}
On a trouve que
\[ 
	\frac{d}{dt} \int \int \int _V \rho( \vec{r},t) dV = \int \int \int_V \frac{\del}{\del t}\rho( \vec{r},t) 
\]
Pour la partie de droite, on a
\[ 
	- \int \int_S \rho( \vec{r},t) \vec{u}( \vec{r},t) \vec{dS} = - \int \int \int_V \nabla ( \rho\vec{u})  dV
\]

Donc
\[ 
	\int \int \int_V( \frac{\del}{\del t + \nabla( \rho\vec{u}) dV =0 }) 
\]
Pour tout volume $V$

Et donc
\[ 
	\frac{\del \rho}{\del t}+ \nabla\cdot( \rho \vec{u}) =0
\]

\subsection{Equation d'Euler}
On fait un bilan de la quantite de mouvement.
On considere un fluide parfait ( pas de frottement interne).\\
La quantite de mouvement dans $V( t) $, on a 
\[ 
	\vec{p} = \int \int \int_{V( t) }  \rho \vec{u} dV
\]
Seconde loi de newton:
\[ 
	\frac{d \vec{p}}{dt}= \sum \text{ forces externes sur la partie du fluide contenue dans  } V( t) 
\]
On va montrer que 
\[ 
	\frac{d \vec{P}}{dt} = \int \int \int_{V( t) }  \frac{D}{Dt}( \rho \vec{u}) + \rho \vec{u}( \nabla \cdot \vec{u}) dV
\]
\begin{align*}
	&= \int \int\int_{V( t) } \rho \vec{g} dV - \int \int_{S( t) }  p \vec{dS}	\\
	&= \int \int\int_{V( t) } \rho \vec{g} dV  - \int \int\int_{V( t) }  \nabla p dV	
\end{align*}
Donc
\[ 
	\frac{D}{Dt}( \rho \vec{u}) + \rho \vec{u} ( \nabla \cdot \vec{u}) = \rho \vec{g} - \nabla p
\]
\begin{align*}
	\rho \frac{D}{Dt}\vec{u} + \vec{u} \frac{D}{Dt}\rho + \rho \vec{u}( \nabla \cdot \vec{u})\\
	= \rho \frac{D}{Dt}\vec{u} + \vec{u} ( \frac{D\rho}{Dt} + \rho ( \nabla \cdot \vec{u}) ) 
\end{align*}
Le dernier terme est nul par l'equation de continuite, et on trouve
\[ 
\rho \frac{\vec{Du}}{Dt} = \rho \vec{g} - \nabla p
\]

\begin{rmq}
En general, les fluides ont de la viscosite.\\
De plus, tout comme l'equation de continuite, l'equation d'Euler est non-lineaire, la solution a l'equation differentielle est generalement extremement complique.
\end{rmq}
\subsection{Equation d'etat}
equations:
\[ 
	\frac{\del \rho}{\del t} + \nabla \cdot ( \rho \vec{u}) =0
\]
\[ 
\rho \frac{D \vec{u}}{Dt}= \rho \vec{g} - \nabla p
\]

Il nous manque encore une equation pour decrire un fluide en mouvement: l'equation d'etat, qui depend du type de fluide.
\[ 
	\frac{D}{Dt}( p\rho^{-\gamma}) =0
\]
ou $\gamma$ est l'indice d'adiabicite.





\end{document}	
