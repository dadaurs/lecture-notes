\documentclass[../main.tex]{subfiles}
\begin{document}
\lecture{10}{Fri 26 Mar}{Ondes Stationnaires}
Une onde stationnaire est de la forme
\[ 
	y_0( x,t) = d \sin( \frac{\pi n}{l}x) \cos( \frac{\pi c}{n} t + \phi) 
\]
la frequence est donnee par 
\[ 
\nu = \frac{c}{\lambda}= \frac{cn}{ 2\lambda}
\]
Ici, $c$ est la vitesse de l'onde ( $c= \sqrt{ \frac{T}{\mu} }$) 
Le mouvement arbitraire de la corde est une somme infine des modes normaux
\subsection{Ondes en $3D$}
En 3 dimensions, l'equation d'onde devient
\[ 
\frac{\del ^{2} y_0}{\del t^{2}} = c^{2} \Delta y_0
\]
Les ondes sinusoidales deviennent des ondes sinusoidales planes
\[ 
	y_0*( \vec{r},t)  = A e^{i( \omega t - \vec{k} \cdot \vec{r}) } 
\]
ou $\vec{k}$ est le vecteur d'ondes.\\
Une surface equiphase est definie par
\[ 
\omega t - \vec{k} \cdot \vec{r} = \text{ const } 
\]

On peut choisir $e_x || \vec{k}$ et alors
on a 
\[ 
\omega t - 
\begin{pmatrix}
k \\ 0 \\0
\end{pmatrix}
\cdot
\begin{pmatrix}
x\\y\\z
\end{pmatrix}
= \omega t - kx 
\]
Et donc
\[ 
x = \frac{\omega }{k} t - \frac{\phi_0}{k}
\]
Donc les surfaces equiphases sont des plans et donc l'onde ( et les surfaces equiphases)  se deplacent selon $\vec{k}$, avec la vitesse $\frac{\omega}{k}=c$
\subsection{Quelques consequences du principe de superposition}
Superposition de deux ondes de frequence $\omega + \Delta \omega$ et $\omega - \Delta \omega$, de vecteurs d'onde $k+ \Delta k$ et $k - \Delta k$ et de dephasage $\phi_1$ et $\phi_2$, de meme amplitude.\\

\[ 
	y_0 = A e^{i( \omega + \Delta \omega) t - ( k + \Delta k)x + \phi_1}  + A e^{i( \omega - \Delta \omega) t - ( k - \Delta k)x + \phi_2} 
\]
Alors on a
\[ 
	A e^{i( \omega t - kx) } (  e^{i ( \phi_1 - \frac{\phi_1+\phi_2}{2 } + \frac{\phi_1+\phi_2}{2 } + \Delta \omega t - \Delta k x) } + e^{i ( \phi_2 - \frac{\phi_1+\phi_2}{2 } + \frac{\phi_1+\phi_2}{2 } - \Delta \omega t + \Delta k x) }  ) 
\]
\begin{align*}
	= 2A e^{i( \omega t - kx + \frac{\phi_1 + \phi_2}{2}) } ( e^{i ( \frac{\phi_1 - \phi_2}{2}+ \Delta \omega t - \Delta k x) } + e^{i ( -\frac{\phi_1 - \phi_2}{2}- \Delta \omega t + \Delta k x) }) \\
	=2A e^{i( \omega t - kx + \frac{\phi_1 + \phi_2}{2}) } cos( ( \frac{\phi_1 - \phi_2}{2}+ \Delta \omega t - \Delta k x) ) \\
\end{align*}
La partie reelle est donc donne par 
\[ 
	y_0 = 2 A \cos (  \omega t - kx + \frac{\phi_1 + \phi_2}{2})\cos (  \frac{\phi_1 - \phi_2}{2}+ \Delta \omega t - \Delta k x)  
\]
Dans le cas $\Delta \omega =0 \Rightarrow \Delta k = 0$ et donc
\[ 
	y_0 = 2A \cos (\frac{\phi_1-\phi_2}{2}\cos( \omega t -kx + \frac{\phi_1 + \phi_2}{2})  ) 
\]
\subsection{Vitesse de phase et de groupe}
$\Delta \omega \neq 0$ mais $\Delta \omega << \omega$.\\
La vitesse de l'enveloppe $V_G$ est appelee la vitesse de groupe
\[ 
v_G = \frac{\Delta \omega }{\Delta k} \simeq \frac{d\omega}{d k}
\]
pour $\omega= ck$ on a une onde sans dispersion.\\
Pour une onde avec dispersion  ( $\frac{\omega}{k} \neq \text{ const. } $) on a la relation de dispersion $\omega = \omega( k) $ on a en general
\[ 
v_G = .\frac{d\omega}{dk}
\]







\end{document}	
