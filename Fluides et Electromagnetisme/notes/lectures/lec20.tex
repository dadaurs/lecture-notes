\documentclass[../main.tex]{subfiles}
\begin{document}
\lecture{20}{Tue 11 May}{circuits electriques}
\subsubsection*{Exemples de conducteur}

\begin{itemize}
\item Metaux: $e^{-}$ ``libres'': $n^{-}\neq 0, n^{+}=0$ 
\item electrolytes: presence des ions positifs et negatifs qui permet un courant
\item plasmas: $e^{-}$ et ions mobiles : $n^{+}\neq 0, n^{-}=0$
\end{itemize}
Par simplicite, on supposera qu'un seul type de porteur mobile de charges, qui est positif.\\
\subsection{ La densite de courant }
La densite de courant $\vec{j}$ est donnee par 
\[ 
	\vec{j}( \vec{r},t ) = \rho_{el} ( \vec{r},t)  \cdot \vec{u}( \vec{r},t) = q n ( \vec{r},t) \cdot \vec{u}( \vec{r},t) 
\]
Le courant a travers une surface $S$, on a 
\[ 
	I( t) = \iint_S \vec{j} \cdot \vec{dS}
\]
\subsection{La resistance $R$, la loi d'Ohm et l'effet Joule}
\subsubsection{La resistance}

\begin{figure}[ht]
    \centering
    \incfig{circuit-avec-resistance}
    \caption{circuit avec resistance}
    \label{fig:circuit-avec-resistance}
\end{figure}
Une resistence $R( U) = \frac{ \text{ voltage applique } }{ \text{ courant } }= \frac{U}{I} $, $ [ R] = \frac{V}{A}= \text{ Ohm } = \Omega$	
\begin{exemple}[Diode a vide]
\begin{figure}[ht]
    \centering
    \incfig{diode-a-vide}
    \caption{diode a vide}
    \label{fig:diode-a-vide}
\end{figure}
La cathode est un filament chauffe $\to$ emission d'electrons.\\
\begin{itemize}
\item $U_A > U_C$ \\
	Alors les electrons sont accelere vers l'anode, et il y a donc un courant qui circule de l'anode vers la cathode.
\item Si $U_A < U_C$ \\
	On trouve que les electrons restent autour de la cathode, et il n'y a donc pas de courant.
\end{itemize}
\end{exemple}
Dans certains cas ( par exemple les metaux et les electrolytes a $T=$ const), $I \propto U$. \\
Donc, $R$ est independant de $U$, et on parle alors de loi d'Ohm.
\[ 
R= \frac{U}{I}= \text{ const. } 
\]
De maniere plus generale, on ecrity la loi d'Ohm de la maniere
\[ 
\vec{j} = \sigma_c \vec{E} = \frac{1}{\rho_c} \vec{E}
\]
A partir de $\vec{j}= \sigma_c E$, on peut deriver
\[ 
U=RI
\]
On considere un conducteur cylindrique et deux plaques chargees avec un potentiel $\phi_A$ resp. $\phi_B$, on a
\[ 
U = \phi_A - \phi_B = El = \frac{j }{\sigma_c} l = \frac{j S}{\sigma_c S}l = \frac{I}{\sigma_c S} l = RI
\]
Donc $U = RI$ avec  $ R = \frac{l}{\sigma_c S} = \frac{\rho_c l}{S}= \text{ const. } $.\\
La resistance depend donc de la geometrie $( S,l) $ et du type de conducteur $ (  \sigma_c, \rho_c) $.\\
\subsubsection{L'effet joule}
Si une charge $q$ se deplace de $A$ a $B$, le travail exerce par le champ $E$ sur la charge est $( \phi_A- \phi_B)q $ ( propriete 6 ) du potentiel.\\
Dans une resistance, cette energie est transformee en chaleur $\to $ effet Joule.\\
On aimerait calculer la puissance de production de chaleur, elle est donnee par
\[ 
	P = \frac{ dq  ( \phi_A - \phi_B) }{dt} = \frac{dq}{dt} U = IU = \frac{U ^{2}}{R}
\]
Cette puissance est fourne piar l'appareil a fem.
\subsection{Conservation de charge, equations de continuite}
La charge est toujours conservee, donc on peut considerer un volume fixe de surface $S$, donc on a
\begin{align*}
	\frac{d}{dt} \iiint_V \rho_{el}  ( \vec{r},t) dV &= - \iint_S \vec{j}( \vec{r},t) \\
	\iiint_V \frac{\del \rho_{el} }{\del t} ( \vec{r},t) dV = - \iiint_V \nabla \cdot \vec{j}( \vec{r},t) dV
\end{align*}
Valable $\forall V$, et on en deduit que
\[ 
\frac{\del \rho_{el}}{\del t} + \nabla \cdot \vec{j} =0
\]
\subsection{Circuits electriques et lois de Kirchhoff}
On considere ici les elements suivants:
\begin{itemize}
	\item Une source de tension continue ( appareil a fem.) 
	\item resistence
	\item capacite 
		\begin{itemize}
		\item cas 1: Le condensateur se charge, donc $U= \frac{q}{c}; I = \frac{dq }{dt}$
		\item cas 2: Le condensateur se decharge, donc $U= \frac{q}{c}; I =- \frac{dq }{dt}$
		\end{itemize}

	\item fil: element de resistance negligeable
		
\end{itemize}
\subsection{Lois de Kirchhoff}
\begin{itemize}
\item K1, loi des noeuds\\
	La somme de tous les courants qui arrivent a un noeud d'un circuit est egale a la somme de tous les courants qui le quittent \\
	$\to $ dans les noeuds, il n'y a pas d'accumulation de charge
\item K2, Loi des mailles \\
	La somme algebrique des variations  de potentiel le long de toutes les mailles fermees d'un circuit est nul.\\
	Ceci vient du fait que
	\[ 
\oint _\gamma \vec{E} \vec{dl} =0	
	\]
Cette loi est valable en electrostatique, magnetostatique ou si les variations temporelles ne sont pas trop importantes.	
\end{itemize}








\end{document}	
