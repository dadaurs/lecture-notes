\documentclass[../main.tex]{subfiles}
\begin{document}
\lecture{26}{Tue 01 Jun}{consequences des equations de maxwell}
\subsection{Ondes electromagnetiques dans le vide}
Dans le vide, $\rho_{el} = 0, \vec{j} = 0$, alors on peut reecrire les equations de maxwell comme
\begin{enumerate}
\item $\nabla \cdot \vec{E} = 0$ 
\item $\nabla \times \vec{E} = - \frac{\del \vec{B}}{\del t}$ 
\item $\nabla\cdot \vec{B} = 0$ 
\item $\nabla \times \vec{B} = \epsilon \mu_0 \frac{\del \vec{E}}{\del t}$ 
\end{enumerate}
\subsubsection*{Identite utile}

Soit $\vec{A}$ un champ vectoriel, alors
\[ 
	\nabla \times ( \nabla \times \vec{A}) = \nabla ( \nabla \cdot A) - \Delta \vec{A}
\]
Prenons le rotationnel de $ ( 2) $, on obtient
\begin{align*}
	\nabla \times ( \nabla \times \vec{E}) &= - \nabla \times ( \frac{\del \vec{B}}{\del t}) \\
					       &=  - \frac{\del}{\del t}(  \epsilon_0 \mu_0 \frac{\del \vec{E}}{\del t}) \\
					       &= \nabla ( \underbrace{\nabla \cdot \vec{E}}_{ = 0}) - \Delta \vec{E}\\
					       \intertext{Finalement, }
					       \frac{\del ^{2} E}{\del t ^{2}}= \frac{1}{\epsilon_0 \mu_0 }\Delta \vec{E}
\end{align*}
$\vec{E}$ satisfait une equation d'onde!\\
$\vec{B}$ aussi: rot de $( 4) + ( 2) $ 
Ainsi, on trouve 
\[ 
\Rightarrow \frac{\del ^{2}B}{\del t^{2}}= \frac{1}{\epsilon_0 \mu_0} \Delta \vec{B}
\]
En observant que $c= \frac{1}{ \sqrt{\epsilon_0 \mu_0} }$ , Maxwell a conclut correctement que la lumiere est une onde electromagnetique.\\
C'est vrai pour d'autres types de radiation.
\subsection{Ondes planes dans le vide }
On cherche des solutions du type ondes sinusoidales planes, avec la notation complexe.
\begin{align*}
\tilde { \vec{E}} ( \vec{r},t) =\tilde {  \vec{E_0}  } e^{i ( \omega t - \vec{k} \cdot \vec{r}) } \\
\tilde { \vec{B}} ( \vec{r},t) =\tilde {  \vec{B_0}   }e^{i ( \omega' t - \vec{k'} \cdot \vec{r}) } 
\end{align*}
$\tilde { \vec{E_0} }$ et $\tilde { \vec{B_0} }$ sont des vecteurs .\\
Pour satisfaire l'equation d'onde, il faut que
\[ 
	- \omega^{2} = \frac{1}{\epsilon_0 \mu_0} (  - \vec{k}^{2}) \Rightarrow 
\]
En appliquant le fait que $\nabla \cdot \vec{E} =0$ 
\begin{align*}
\begin{pmatrix}
\del x\\ \del y \\ \del z
\end{pmatrix} \cdot 
\begin{pmatrix}
	E_{0x} e^{i ( \omega t - kz) } \\
	E_{0y} e^{i ( \omega t - kz) } \\
	E_{0z} e^{i ( \omega t - kz) } \\
\end{pmatrix} = 0
\end{align*}
Et on conclut que $E_{0z} = 0 \Rightarrow  \vec{E_0} \perp \vec{k}$, donc l'onde est transversale.
On prend le cas ( particulier ) $E_{0} || e_x $, alors on a
\[ 
	\vec{E}( \vec{r},t) = E_{ox} e^{i ( \omega t - kz) } e_x
\]
On a, par la loi de Faraday,
\[ 
\nabla \times \vec{E} = - \frac{\del \vec{B}}{\del t}
\]
\begin{align*}
\begin{pmatrix}
\del x\\ \del y \\ \del z
\end{pmatrix} \times
\begin{pmatrix}
	E_{0x}  e^{i ( \omega t - k z) } \\ 0 \\0
\end{pmatrix} =
\begin{pmatrix}
	0 \\ E_{0x} e^{i ( \omega t - kz) } ( - ik) \\ 0
\end{pmatrix} \\
= - \frac{\del \vec{B}}{\del t} = 
- \begin{pmatrix}
	B_{0x} i \omega' e^{i ( \omega' t - k' \vec{r}) } \\
	B_{0y} i \omega' e^{i ( \omega' t - k' \vec{r}) } \\
	B_{0z} i \omega' e^{i ( \omega' t - k' \vec{r}) } 
\end{pmatrix} 
\intertext{Ainsi}
B_{0x} = B_{0z} = 0\\
\intertext{Il nous reste}
- ik E_{0x} e^{i ( \omega t - kz) }  = - i \omega' B_{0y}  e^{i ( \omega' t - k_x' x - k_y' y - k_z' z) } 
\end{align*}
Valable $\forall t $ et $\vec{r}$, donc on trouve que $\omega' = \omega $ et $k = k_z' $, $k_x' = k_y' = 0$ .
On trouve donc que
\[ 
B_{0z} = \frac{k}{\omega}E_{0x}  = \frac{1}{c} E_{0x} 
\]
Ecrivons 
\[ 
E_{ox}  = E e^{i \phi} 
\]
Ainsi, $B_{0y}  = \frac{1}{c }E_{0x} = \frac{1}{c }E e^{i \phi} = B_{0y} e^{i \phi} $\\
Finalement
\begin{align*}
\vec{E}( \vec{r},t) = E e^{i ( \omega t - kz + \phi) } e_x\\
\vec{B}( \vec{r},t) = B_{0y}  e^{i ( \omega t - k z + \phi ) } e_y 
\end{align*}
On peut verifier facilement que $\nabla \cdot \vec{B} = 0$ et $\nabla \times \vec{B} = \frac{1}{c^{2}} \frac{\del \vec{E}}{\del t}$ sont satisfaites.\\
La partie physique ``est '' la partie reelle.
\begin{rmq}[additionelles]
\begin{itemize}
\item On a discute des ondes electromagnetiques dans le vide, cependant, dans un medium, la vitesse de phase des ondes electromagnetiqes peut etre different: $c \to \frac{c}{n} $, alors on a
	\[ 
	\omega = \frac{c}{n} |\vec{k}|
	\]
	$n$ est l'index de refraction.
\end{itemize}
\end{rmq}





	



\end{document}	
