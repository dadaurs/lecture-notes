\documentclass[../main.tex]{subfiles}
\begin{document}
\lecture{15}{Fri 23 Apr}{Proprietes additionelles}
\begin{exemple}
On considere une sphere avec $\sigma_{el} =$ const.
\begin{figure}[H]
    \centering
    \incfig{sphere-chargee}
    \caption{sphere chargee}
    \label{fig:sphere-chargee}
\end{figure}
\end{exemple}
On a que l'inverse de la propriete 4 ci-dessus est egalement vraie:
\[ 
	\text{ Pour } \vec{A}( \vec{r}) \text{ avec } \nabla\times \vec{A} = 0 \forall \vec{r}
\]
Alors il existe $\phi( \vec{r}) $ tel que $\vec{A}= - \nabla \phi$.\\
Le travail pour deplacer une charge $q$ dans un champ $\vec{E}$ de $A $ a $B$ ne depend pas du chemin suivi et est egal a $q ( \phi( B) - \phi( A) ) $
\begin{proof}
\begin{figure}[H]
    \centering
    \incfig{dessin-point-a-point-b}
    \caption{dessin point A point B}
    \label{fig:dessin-point-a-point-b}
\end{figure}
\end{proof}
On considere $\gamma$ parametre par $\vec{l}( s) , s\in [ 0,1] $ \\
\begin{align*}
\Rightarrow \omega_\gamma &= \underbrace{-}_{ \text{ On considere le travail contre la force } } \int_{\gamma} \vec{F} \cdot d \vec{l}\\
&= - \int_{\gamma} q \vec{E} \cdot \vec{dl}\\
&= \int_{\gamma}  q \nabla \phi \vec{dl}\\
&= \int_{ 0 }^{ 1 }q \nabla \phi( \vec{l}( s) ) \cdot \frac{\vec{dl}}{ds}ds\\
&= \int_{ 0 }^{ 1 } q \frac{d \phi( \vec{l}( s) ) }{ds} ds\\
&= q \phi( \vec{l}( s) ) \vert_{0} ^{1}\\
&= q ( \phi( B) - \phi( A)  ) = \int_{A\to B}  \vec{E} \cdot \vec{dl} 
\end{align*}
est le changement de l'energie potentielle electrostatique par unite de charge si $q$ est deplace de $A$ a $B$.\\
On remarque que la difference de potentiel est independante du chemin choisi.\\
On appele $\phi( B) - \phi( A) $ la tension entre $B$ et $A$.\\

On a aussi que le travail pour deplacer une charge $q$ le long d'une courbe fermee est toujours nul, ce qui suit immediatement de la propriete precedente.\\
\begin{figure}[H]
    \centering
    \incfig{travail-sur-une-courbe-fermee}
    \caption{travail sur une courbe fermee}
    \label{fig:travail-sur-une-courbe-fermee}
\end{figure}
donc $q \vec{E}$ est une force conservative.\\

On peut donc definir des surfaces equipotentielles $ \left\{ \vec{r} : \phi( \vec{r}) = \text{ const }  \right\} $.\\
Comme $\vec{E} = - \nabla \phi$, les lignes du champ electrique $E$ sont perpendiculaires aux surfaces equipotentielles
\begin{figure}[H]
    \centering
    \incfig{champ-electrique-perpendiculaire}
    \caption{champ electrique perpendiculaire}
    \label{fig:champ-electrique-perpendiculaire}
\end{figure}
On prouve ceci avec l'expansion de Taylor
\begin{align*}
	\phi( \vec{r} + \vec{h})  &= \phi( \vec{r}) + \nabla \phi( \vec{r}) \cdot \vec{h} + )( |h|^{2}) \\
				  &= \phi( \vec{r}) - \vec{E}( \vec{r}) \cdot h + O( |h|^{2}) \\
\intertext{Pour $|h|$ tres petit:}
				  &=\phi( \vec{r}+\vec{h}) = \phi( \vec{r})  - \vec{E}( \vec{r}) \cdot h
\end{align*}
Donc, si $\vec{h}$ est tangentiel a la surface equipotentielle au point $r$, on peut ecrire que
\[ 
	\phi( \vec{r}+ \vec{h})\approx \phi( \vec{r}) 
\]
Et donc 
\[ 
	\vec{E}( \vec{r}) \cdot \vec{h}=0
\]
Et donc $\vec{E}$ pointe dans la direction perpendiculaire aux surfaces equipotentielles.
\subsection{Le role des conducteurs en electrostatique}
\subsubsection{Proprietes de base}
\begin{figure}[H]
    \centering
    \incfig{champ-electrique-avec-charges}
    \caption{champ electrique avec charges et conducteur}
    \label{fig:champ-electrique-avec-charges}
\end{figure}
En electrostatique, il n'y a pas de courant, donc
\begin{itemize}
	\item $\vec{E}=0$ a l'interieur d'un conducteur ( sinon, les charges libres produiraient un courant) 
	\item $\nabla \cdot \vec{E}=0 = \frac{\rho_{el} }{\epsilon_0}$, donc il n'y a pas de densite de charge a l'interieur du conducteur.
	\item Le conducteur delimite une region ou $\phi$ ( le potentiel) est constant.
	\item $\vec{E}= - \nabla \phi \Rightarrow \vec{E}$ est perpendiculaire a la surface du conducteur.
	\item Des charges peuvent se trouver seulement a la surface du conducteur $ \Rightarrow  \sigma_{el} =0$ possible.
\end{itemize}




\end{document}	
