\documentclass[../main.tex]{subfiles}
\begin{document}
\lecture{1}{Tue 23 Feb}{Introduction}
\section{Notations du cours et maths necessaires}
\subsection{Scalaires et Vecteurs}
On distingue les quantites scalaires  ( pression, masse, la charge electrique ) et les quantites vectorielles (vitesse, force) .\\
Dans un repere 3D, les vecteurs de base unitaires $e_x, e_y, e_z$\\
On definit un champ scalaire( resp. vectoriel)  par une fonction $p( \vec{r},t) $ qui depend de la position et du temps.
\subsection{L'operateur $\nabla$ ( nabla) et la definition du gradient, de la divergence et du rotationnel}
En coordonnes cartesiennes, on a
\[ 
	\nabla = \left( \frac{\del}{\del x},\frac{\del}{\del y}, \frac{\del}{\del z}\right) 
\]
On note
\[ 
	\frac{\del p}{\del x}( \vec{r},t) = \lim_{h \to 0} \frac{p( x+h,y,z,t) - p( x,y,z,t) }{h}
\]
\begin{itemize}
	\item Le gradient, note $\nabla f$ d'un champ scalaire $f( \vec{r},t) $ est un champ vectoriel donne par
		\[ 
			\nabla f( \vec{r},t) = e_x \frac{\del f}{\del x} + e_y \frac{\del f}{\del y} + e_z \frac{\del f}{\del z}
		\]
	
	\item La divergence, notee $\nabla \cdot \vec{u}$ d'un champ vectoriel $\vec{u}( \vec{r},t) $ est un champ scalaire donne par
		\[ 
		\nabla \cdot  \vec{u} = \frac{\del u_x}{\del x} + \frac{\del u_y}{\del y} + \frac{\del u_z}{\del z}
		\]

	\item Le rotationnel $\nabla \times \vec{u}$ d'un champ vectoriel est un champ vectoriel donne par
		\[ 
			\nabla \times \vec{u}( \vec{r},t) =( \frac{\del}{\del x}, \frac{\del}{\del y}, \frac{\del}{\del z}) \times ( u_x,u_y,u_z) 
		\]
		
\end{itemize}
\begin{rmq}
On peut utiliser $\nabla$ comme un vecteur, mais il faut faire attention a ce que les operations sont pas commutatives.
\end{rmq}
\begin{rmq}
Souvent, on ecrit $\del_x$ pour $\frac{\del}{\del x}$
\end{rmq}
\begin{rmq}
Les expressions du gradient, divergence, rotationel sont independantes du systeme de coordonnees
\end{rmq}
\subsection{Formules d'integration}
\begin{thm}[Theoreme du gradient]\label{thm:Theoreme du gradienttheoreme_du_gradient}
	Soit un volume $V$ quelconque dans l'espace et soit $S$ la surface fermee limitant le volume $V$( on note $S= \del V$).\\
	A chaque element de la surface, on assimile un vecteur orthogonal a la surface en ce point. On le note $d \vec{S}$ et il represente le "petit element" de surface.\\
	Alors on a
	\[ 
	\int \int_S f d \vec{S} = \int \int \int_V \nabla f dV
	\]
	
\end{thm}
	
\begin{thm}[Theoreme de La divergence( de Gauss) ]\label{thm:Theoreme de La divergence( de Gauss) theoreme_de_la_divergence_de_gauss_}
	Le flux d'un champ vectoriel $\vec{A}( \vec{r},t) $ au travers d'une surface $S$ :
	\[ 
	\phi = \int \int_S \vec{A} \cdot \vec{dS}
	\]
	Soit une surface fermee $S= \del V$ et $d \vec{S}$ qui point vers l'exterieur de $V$, alors on a
	\[ 
		\int \int _S \vec{A} \cdot d \vec{S} = \int \int \int _V ( \nabla \cdot \vec{A}) dV
	\]
	
	
\end{thm}
\begin{thm}[Theoreme de Stokes]\label{thm:Theoreme de Stokestheoreme_de_stokes}
	On definit la circulation d'un champ vectoriel $\vec{A}( \vec{r},t) $ le long d'une courbe fermee $\Gamma$ :
\[ 
\Sigma = \oint_\Gamma \vec{A} \cdot d \vec{l}
\]
Dans ce cas la, on a
\[ 
	\oint_\Gamma \vec{A} \times \vec{dl} = \int \int_S ( \nabla \times \vec{A}) \cdot \vec{dS}
\]
L'orientation relative de $\vec{dl}$ et $\vec{dS}$ est donnee par la regle de la main droite.


\end{thm}
\section{Fluides au repos}
\subsection{Introduction}
On appelle un fluide un corps qui est a l'etat liquide, gazeux, ou plasma, systeme d'un grand nombre de particules qui est susceptible de s'ecouler facilement.\\
Autrement dit, un corps deformable/qui n'a pas de forme propre.\\
Pour beaucoup d'applications: un fluide est decrit par sa densite de masse $\rho( \vec{r},t) $, la pression ( $p( \vec{r},t)$ ) et la vitesse $\vec{u}( \vec{r},t) $ \\
Dans ce chapitre, on suppose $\vec{u}( \vec{r},t) =0$,$\rho( \vec{r},t ) = \rho( \vec{r}) $ et $p( \vec{r},t) = p( \vec{r}) $ \\
\subsection{Densite de fluide}
Supposons un recipient avec un fluide dedans et un systeme de coordonnees.\\
On note
\[ 
	\bar{\rho} = \frac{\Delta m}{\Delta V}
\]
pour la densite moyenne.\\
On prend ensuite la limite $\Delta V \to dV$ et on obtient ainsi 
\[ 
	\rho( \vec{r},t) = \lim_{\Delta V \to dV} \frac{\Delta m}{\Delta V}
\]






\end{document}	
