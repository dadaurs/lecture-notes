\documentclass[../main.tex]{subfiles}
\begin{document}
\lecture{7}{Wed 02 Nov}{convolutions and Hausdorff measures}
\begin{thm}
	Let $p \in [ 1, \infty ) $, $\Omega \subset \mathbb{R}^n$ open, then $C^{ \infty }_c ( \Omega, \mathbb{K}) $ is dense in $L^{p}( \Omega, \mathbb{K}) $.\\
\end{thm}
\subsection{Convolutions}
Let $f,g \in L^{1}(  \mathbb{R}^{n}) $, $( f\ast g) ( x) = \int_{ \mathbb{R}^n }^{  } f( x-y) g( y) dy$ 
\begin{lemma}
Let $\eta\in L^{1}( \mathbb{R}^n) $, for $r>0$, let $\eta_r( x) = \frac{1}{r^{n}}\eta( \frac{x}{r}) $.\\
Then
\begin{itemize}
\item $\eta_r\in L^{1}, \N { \eta_r}_1 = \N { \eta}_1$.
\item If $f\in L^{p}( \mathbb{R}^n) $, then $ \N { \eta_r \ast f}_p \leq  \N { \eta_r}_{1} \N{f}_p$ 
\item If $p< \infty , \eta \geq 0$, $\N { \eta}_1 =1$, then $f\ast \eta_r \to f$ in $L^{p}(  \mathbb{R}^n) $ 
\item If $\eta \in C^{ \infty }$, then $\eta_r \ast f \in C^{ \infty }$, $\del^{\alpha}( \eta_r \ast f) = ( \del^{\alpha} \eta_r) \ast f$ 
\item If $\eta\in C_c( \mathbb{R}^n) $, $ f \in L^{1}_{loc} ( \mathbb{R}^m) $, then $\eta_r \ast f\in C^{0}$.\\
\end{itemize}
\end{lemma}
\begin{lemma}
If $ \int_\omega \phi f d \mathcal{L}^{n}=0 \forall \phi \in C^{ \infty }_c$, then $f=0$.
\end{lemma}
\begin{lemma}
Let $p \in [ 1, \infty ] $ $\Omega$ open and $f_j \to f$ in $L^{p}( \Omega) $, then $\forall \phi \in C^\infty c( \Omega) $
\[ 
\int_{ \Omega }^{  } f_j \phi dx \to \int_\omega f \phi dx
\]
\end{lemma}
\subsection{Hausdorff Measures}
\begin{defn}
	Let $( X,d) $ be a metric space, $\delta>0$ and $s \in [ 0, \infty ) $.\\
	Define
	\[ 
		\mathcal{H}_\delta^{s}: P( X) \to [ 0, \infty  ], \mathcal{H}_\delta^{s}( E) = \frac{\omega_s}{2^{2}}\inf \left\{ \sum_{  h \in \mathbb{N} } ( diam F_h)^s: E \subset \bigcup_{h \in \mathbb{N}} F_h, diam F_h \leq \delta \right\} 
	\]
Where $\omega_s = \frac{\pi^{\frac{s}{2}}}{\Gamma( 1+ \frac{s}{2}) }$.\\
Then
\[ 
\mathcal{H}^{s}( E) = \lim_{\delta\to 0} H_\delta^{s}( E) 
\]

	
\end{defn}
\begin{lemma}
All Borel sets are $ \mathcal{H}^{s}$-measurable and $( X,B, \mathcal{H}^{s}|_B) $ is a measure space.
\end{lemma}
\begin{thm}
	\[ 
	\mathcal{H}^{n}|_{m^{n}} = \mathcal{L}^{n}
	\]
	
\end{thm}
\begin{thm}
	For any $E \subset X$, there is a number $s_E \in [ 0, \infty ] $ such that 
	\[ 
	\mathcal{H}^{s}( E) = \infty \forall s \in [ 0, s_E)
	\]
	and 
	\[ 
		\mathcal{H}^{s}( E) =0 \forall s \in ( s_E, \infty ]
	\]
	
\end{thm}
\begin{defn}[Hausdorff Dimension]
	\[ 
	dim_{H} ( E) = \coloneqq s_E
	\]
	
\end{defn}
\begin{exemple}
\[ 
\dim_H ( B( 0,1) ) = n-1
\]

\end{exemple}
\begin{thm}
	If $U \subset \mathbb{R}^{k}$ and $ \psi \in C^{1}( U, \mathbb{R}^n) $ an injective immersion, then $\psi( U) $ is $ \mathcal{H}^{k}$ measurable, $ \mathcal{H}^{k}|_{\psi( U) } $ is $\sigma$-finite and
	\[ 
	\mathcal{H}^{k}( \psi( U) ) = \int_U ( \det D\psi^{T}D\psi)^{\frac{1}{2}}d \mathcal{L}^{k}	
	\]
	
\end{thm}
\section{Sobolev Spaces}
Let $\Omega \subset \mathbb{R}^n$ open, $f\in L^{p}( \Omega) $.\\
DOes it have a "derivative" in $L^{p}$ ?\\
Notice that
\[ 
\int_{ \Omega }^{  } f \del_i \phi d \mathcal{L}^{n}= - \int_{ \Omega }^{  } ( \del_i f) \phi d \mathcal{L}^{n} + \int_{ \del \Omega }^{  } f \phi \nu_i d \mathcal{H}^{n-1}
\]
if $f,\phi \in C^{1}(  \mathbb{R}^n) $, $\Omega$ bounded, $\mathcal{H}^{n-1}( \del \Omega \setminus \del_r \Omega) =0$ 

Morally, the last term is 0, because $\del \Omega$ is a null-set.
\begin{defn}
	Let $\Omega \subset \mathbb{R}^n$ open, $f \in L^{1}_{loc} ( \Omega) $.\\
	$f$ is weakly differentiable if $\exists g \in L^{1}_{loc} ( \Omega, \mathbb{R}^n) $ 
	\[ 
	\int_\omega f \del_i \phi d \mathcal{L}^{n}= - \int_{ \Omega }^{  }g_i \phi d \mathcal{L}^{n}\forall \phi \in C^{1}_c( \Omega) 
	\]
$f$ is $k$-times weakly differentiable if $\forall \alpha\in \mathbb{N}^{n}, |\alpha| \leq k$, $\exists g_\alpha \in L^{1}_{loc} $ such that
\[ 
\int_{ \Omega }^{  } f D^{\alpha}\phi d \mathcal{L}^{n}= ( -1)^{|\alpha|} \int_{ \Omega }^{  }g_\alpha \phi d \mathcal{L}^{n}
\]
We write $g_\alpha = D^{\alpha} f$ 

\end{defn}
\begin{rmq}
If $f\in C^{k}( \Omega) $, then it is $k$-times weakly differentiable and the derivatives are the classical ones.
\end{rmq}
\begin{exemple}
If $\Omega= \mathbb{R}$, then $f( x) = |x|$ has a weak derivative.\\
Let $g( x) $ be the step function, we want to show that for $\phi \in C^{1}_c ( \mathbb{R}) $, we want to show that
\[ 
\int_{ \mathbb{R} }^{  } |x| \phi'( x) dx = - \int_{ 0 }^{  \infty  } \phi( x) dx + \int_{ - \infty  }^{0  } \phi( x) dx
\]
Notice that 
\[ 
\int_{ 0 }^{ \infty  }\phi( x) dx = x\phi( x) |_0^{ \infty }- \int_{ 0 }^{  \infty  } x\phi'( x) dx
\]
The first term is 0 and so the equality holds.

\end{exemple}



\end{document}	
