\documentclass[../main.tex]{subfiles}
\begin{document}
\section{Introduction}
\lecture{1}{Wed 12 Oct}{Introduction}

Main reference is "Functional Analysis" by H.W. Alt.
\subsection{Topological Spaces}
\begin{defn}[Topological space]
	Let $X$ be a set, a topology is a subset $\tau \subset P( X) $ is a topology if
	\begin{itemize}
		\item $\emptyset, X \in \tau$ 
		\item any union of opens is open
		\item Finite intersections of opens are open.
	\end{itemize}
\end{defn}
\begin{defn}[Properties]
	For $A \subset X$ , $ \overline A$ is the smallest closed set containing $A$ and the interior $A^{o}$  is the biggest open set contained in $A$.\\
	Finally, the boundary is $\del A = \overline A \setminus A^{o}$.\\
	$X$ is separable if $\exists $ a dense countable subset
\end{defn}
\begin{defn}[Sequences]
	Let $x: \mathbb{N}\to X, \overline{x}\in X$, $\lim x_k = \overline x \iff $ any neighbourhood $U \in T$ of $x$ eventually contains $x_k$ 
\end{defn}
\begin{defn}[Continuity]
A function $f:X\to Y$ is continuous if $\forall U \in \tau_Y, f^{-1}( U) $.\\
This is different from sequential continuity $x_n \to \overline{x} \implies f( x_n) \to f( \overline{x})  $ .\\
$f$ is continuous at $x\in X$ if $\forall V \in S$ st $f( x) \in V \implies f^{-1}( V) \in \tau_X$ 
\end{defn}

\end{document}	
