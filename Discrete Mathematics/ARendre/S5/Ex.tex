\documentclass[11pt, a4paper, twoside]{article}
\usepackage[utf8]{inputenc}
\usepackage[T1]{fontenc}
\usepackage[francais]{babel}
\usepackage{lmodern}

\usepackage{amsmath}
\usepackage{amssymb}
\usepackage{amsthm}
\usepackage{import}
\usepackage{xifthen}
\usepackage{pdfpages}
\usepackage{transparent}
\usepackage{thmtools}
\usepackage{amssymb}
\usepackage{aligned-overset}

\newcommand{\incfig}[1]{%
    \def\svgwidth{\columnwidth}
    \import{./figures}{#1.pdf_tex}
}

\begin{document}
\title{Exercise for Week 4}
\author{David Wiedemann}
\maketitle
\section*{1}
Let us first count the case $n=3$.\\
We denote each point by it's index, here are all the possibilities:
\begin{align*}
	\left\{ ( 1,2), ( 3,4), ( 5,6)  \right\} \\
	\left\{ ( 1,2), ( 3,6), ( 4,5)  \right\} \\
	\left\{ ( 1,4) , ( 2,3) , ( 5,6)  \right\} \\
	\left\{ ( 1,6) , ( 2,3) , ( 4,5)  \right\} \\
	\left\{ ( 1,6) , ( 2,5) , ( 3,4)  \right\} \\
\end{align*}
Hence, there are 5 possibilities for the case $n=3$.


\section*{2}
Let us search a recursive formula.\\
We will denote by $a( n) $ the number of ways to do this for a circle with $2n$ labelled points.\\
We define $a( 0) =1 $.\\

We first label each point on the circle in the following way:

\begin{figure}[ht]
    \incfig{labelling}
\end{figure}


First we choose what point to connect with point 1, there are exactly $n$ points to choose, since choosing a point with an uneven index will clearly force 2 lines to intersect.\\
Hence, the coordinates of the chosen point have to be even, suppose we choose the point with coordinates $2k$.\\
Now we are left with two subset of points, one contains $2n -2k$ points, the other one contains $2k-2$ points.
Hence, the formula of arrangements for a fixed $k $ is given by
\[ 
	a( n-k) \cdot a( k-1) 
\]
Summing over all possibilities for $k$ yields
\[ 
a(n) =\sum_{k=1}^{n} a( n-k) \cdot a(k-1) 
\]

\end{document}
