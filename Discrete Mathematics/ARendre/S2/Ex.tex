\documentclass[11pt, a4paper, twoside]{article}
\usepackage[utf8]{inputenc}
\usepackage[T1]{fontenc}
\usepackage[francais]{babel}
\usepackage{lmodern}

\usepackage{amsmath}
\usepackage{amssymb}
\usepackage{amsthm}

%%%%%%%%%%%%%%%%%%%%%%%%%%%%%%%%%%%%%%%%%%%


\begin{document}
\title{Exercise to Submit Week 2}
\author{David Wiedemann}
\maketitle
\section*{1}

We search for all the ways to build a word containing two a's, one b and one c.\\
It is clearly given by $ \frac{4!}{2!}=12 $.\\
This clearly determines the coefficient of $a^{2}bc$.
\section*{2}
We first calculate the probability that the all the boys have a different number, then we calculate the probability that at least two girls have the same number and finally we can multiply these probabilities.
There are $80\cdot \ldots \cdot 70$ ways for the boys to choose 10 different numbers for $[80]$ and there are $80^{ 10 }$ for them to choose different numbers. Hence the probability for this event is 
\[ 
	P( \text{ All boys have different numbers }   )= \frac{80\cdot \ldots \cdot 70}{80^{10}}
\]
To calculate the probability that at least two girls have the same number we first calculate the probability that all girls have different numbers.\\
This probability is given by
\[ 
	P( \text{ All girls have different numbers } ) =\frac{80 \cdot \ldots \cdot 65}{80^{15}}
\]
Now, the probability that at least two girls have the same number is given by
\[ 
	P(  \text{ At least two girls have the same number } ) = 1- P( \text{ All girls have different numbers }) 
\]
We can now calculate the probability demanded in the exercise which gives
\begin{align*}
	P( \text{ Both events } ) &=P(  \text{ At least two girls have the same number } ) \cdot P( \text{ All boys have different numbers }   ) \\
				  &= \frac{80\cdot \ldots \cdot 70}{80^{10}} \cdot \left(1-\frac{80 \cdot \ldots \cdot 65}{80^{15}} \right) 
\end{align*}







%[La résolution des exercices ici...] %commentaire



\end{document}
