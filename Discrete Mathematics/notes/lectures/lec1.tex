\documentclass[../main.tex]{subfiles}
\begin{document}
\lecture{1}{Mon 22 Feb}{Introduction}
\section{Counting}
\subsection{Finite sets}
Let $A$ be a finite set. We denote by $|A|$ the cardinality of $A$.
\begin{defn}[First Numbers]\label{defn:First Numbersfirst_numbers}
	We denote by $[n]$ the set of $n$ first natural numbers.
\end{defn}
\subsection{Bijections}
\begin{thm}
	If there exists a bijection between finite sets $A$ and $ B$ then $|A|=|B|$.
\end{thm}
\subsection{Operations with finite sets}
\begin{itemize}
\item union
\item intersection
\item product
\item exponentiation
\item quotient
\end{itemize}
\begin{defn}[Cartesion product]\label{defn:Cartesion productcartesion_product}
	 \[ 
		 A \times B = \left\{ ( a,b) | a \in A, b \in B \right\} 
	\]
	
\end{defn}
\begin{thm}
\[ 
|A\times B| = |A| |B|
\]

\end{thm}
\begin{defn}[Disjoint union]\label{defn:Disjoint uniondisjoint_union}
	Define 
	\[ 
	A \sqcup B = A \times \left\{ 0 \right\} \cup B \times \left\{ 1 \right\} 
	\]
			
\end{defn}
\begin{thm}
	\begin{align*}
	|A \sqcup B| = |A| + |B|
	\end{align*}
\end{thm}
\begin{defn}[Exponential object	]
	 \[ 
	A ^{B} = \left\{ f| f \text{ is a function from $A$ to $B$ }  \right\} 
	\]
\end{defn}
\begin{thm}
\[ 
|A ^{B}| = |A| ^{|B|}
\]

\end{thm}
\begin{defn}[Binomial coefficient]\label{defn:Binomial coefficientbinomial_coefficient}
	A binomial coefficient $\binom n k$ is the number of ways in which one can choose $k$ objects out of $n$ distinct objects assuming order doesn't matter.	
\end{defn}
\begin{propo}
	 \[ 
		 \binom n k = \frac{n!}{( n-k) ! k!}
	\]
	
\end{propo}
\begin{propo}
The following identities hold:
\begin{enumerate}
\item 
	\[ 
	\binom n k +  \binom n { k+1} = \binom { n+1} { k+1} 
	\]

\item $\binom n k$ is the $k$-th element in the $n$-th line of Pascal's triangle.
	
\end{enumerate}
\end{propo}
\begin{proof}
	Each subset of $[n+1]$ either contains $n+1$ or not.\\
	Number of $( k+1) $-element subsets containing $n+1$ is  $\binom n k$ \\
	Number of $( k+1) $-element subsets not containing $n+1$ is $\binom n { k+1} $
\end{proof}
\begin{propo}
The number of subsets of an $n$-element  set is $2^{n}$, since we have
\[ 
2^{n} = \sum \binom n i
\]

\end{propo}
\begin{propo}
The number of subsets of even cardinality is the same as even cardinality: $2^{n-1}$
\end{propo}
\begin{proof}
Consider
\[ 
	\phi: 2^{[n]}\to 2^{[n]}
\]
defined by
\[ 
	\phi( A) = A \Delta \left\{ 1 \right\} = 
	\begin{cases}
	A \setminus \left\{ 1 \right\} , \text{ if } 1 \in A\\
	A \cup \left\{ 1 \right\} , \text{ otherwise } 
	\end{cases}
\]
The cardinality of subsets $A$ and $\phi( A)$ always have different parity.\\
Since $\phi \circ \phi = \id $ we deduce that $\phi$ is a bijection between the set of odd and even subsets is the same.

\end{proof}
\begin{thm}
\[ 
	( 1+x) ^{n} = \sum \binom n i x^{i}
\]

\end{thm}
\begin{proof}
In lecture notes.
\end{proof}
\begin{propo}
Assume we have $k$ identical objects and $n$ different persons. Then ne number of ways in which one can distribute this $k$ objects among the $n$ persons equals
\[ 
\binom { n + k-1} { n-1} = \binom { n+k-1} { k } 
\]
Equivalently, it is the number of solutions of the equation $ x_1 + ... + x_n = k$

\end{propo}
\begin{proof}
Let $ \mathcal{A}$ be the set of all solutions of the equation.
Let  $\mathcal{B}$ be the set of all subsets of cardinality $n-1$  in $k+n-1$.\\
we construct a bijection $\psi: \mathcal{A} \to \mathcal{B}$ in the following way
\[ 
	A = ( x_1, \ldots, x_n ) \mapsto B = \left\{ x_1 +1, x_1+x_2 +2 \ldots \right\} 
\]
It suffices to show that $\psi$ is invertible.
Let $B \in \mathcal{B}$.
Suppose that $b_1 \ldots, b_{n-1}$ are the elements of $B$, ordered.
Then the preimage is an $n$-tuple of integers $( x_1, \ldots) $ defined by
\begin{align*}
x_1 = b_1 -1\\
x_i = b_i - b_{i-1}\\
x_n = k+n-1 - b_{n-1} 
\end{align*}
It is easy to see from these equations that the x_i are non-negative and their sums yield $k$.



\end{proof}




	









\end{document}	
