\documentclass[../main.tex]{subfiles}
\begin{document}
\lecture{10}{Sun 02 May}{graph isomorphism}
\subsection{Graph isomorphisms}
Two graphs $G= (V,E)$ and $G'= ( V',E') $ are considered identical if they have the same set of vertices and edges
	
\begin{defn}[Graph isomorphisms]
	Two graphs $G=( V,E) $ and $G'=( V',E') $ are called isomorphic if there is a bijection $f : V \to V'$ such that for all $x, y \in V$ :
	\[ 
		\left\{ x,y \right\} \in E \text{ if and only if }   \left\{ f( x) ,f( y)  \right\} \in E'
	\]
	Such an $f$ is called an isomorphism of the graphs $G$ and G', we write $G \simeq G'$
\end{defn}
\begin{rmq}
In general, deciding whether two graphs are isomorphic is a difficult computational problem.\\
Finding efficient algorithms is an active research area.
\end{rmq}
\subsubsection{Number of isomorphism classes of graphs}
What is the number of isomorphism classes of graphs with vertices $ \left\{ 1,2,\ldots, n \right\} $.\\
The number of different graphs is $2^{ \binom n 2}$.\\
Each isomorphism class contains at most $n!$ elements, therefore
\[ 
\frac{2^{ \binom n 2}}{n!} \leq  | \text{ isomorphism classes } \leq  2^{ \binom n 2}
\]
\begin{rmq}
An isomorphism class is also called an unlabeled graph.
\end{rmq}
\subsection{Trees}
\begin{thm}[Cayley]
	The number of trees on $n$ labelled vertices is $n^{n-2}$.
\end{thm}
\begin{proof}
We will prove this theorem by using Pruefer codes.\\
We will define a one-to-one correspondence between the set of all trees on $n$ labelled vertices and the set of sequences of length $n-2$ consisting of numbers in $ \left\{ 1, \ldots, n \right\} $.\\
The result will then follow by comparing the cardinality of both sets.\\
Consider the following algorithm.
\begin{itemize}
\item Input: a tree of vertices $ \left\{ 1, \ldots, n \right\} $ 
\item Step 1: Find the leaf with smallest label and writie down the number of its neighbours
\item Step 2: Delete this leaf and the only edge adjacent to it
\item Repeat until we are left with only two vertices.
\item Output: string of labels we have written down.
\end{itemize}
The reverse construction is done through the following algorithm.\\
Input: a sequence $ ( a_1, \ldots, a_{n-2} ) \in [ n] ^{n-2}$
\begin{itemize}
\item Step 1: Draw $n$ nodes $ 1 2 \ldots n$ 
\item  Step 2: Make the list $= ( 1, 2 \ldots, n) $ 
\item  Step 3: If there are only two numbers left on the list, connect them with and edge and stop. Otherwise, continue to step 4
\item Find the smallest number in the list which is not in the sequence. Take the first number in the sequence. \\
	Add and edge connecting the nodes whose labels correspond to those numbers.
\item Delete the smallest number which is not in the sequence from the list and the first number in the sequence. This gives a smaller list and shorter sequence. The return to step 3.
\end{itemize}
\begin{thm}
	These algorithms provide a map, name them $\psi_1,$ from the set of trees of vertices $ \left\{ 1,2 \ldots, n \right\} $ to the set $ [ n] ^{n-2}$ 
	\[ 
		\psi_1 ( \text{ tree } ) = \text{ Pruefer code of a tree }
	\]
	Algorithm 2 provides a map, name it $\psi_2$, from the set of sequences $ [ n] ^{n-2}$ to the set of trees of vertices $[n]$
	\[ 
		\psi_2 ( \text{ sequence } ) = \text{ tree } 
	\]
	The maps are inverse to each other, namely
	\[ 
		\psi_1 \circ \psi_2 = \psi_2 \circ \psi_1 = \id
	\]
	
	
\end{thm}
\begin{proof}
We prove the theorem by induction on n.\\
For $n=2$, the unique tree is $ 1 - 2$ and its Pruefer code is  $ \emptyset$.\\
Let $T$ be a tree of vertices $ [ n] $.\\
We apply algorithm 1 to $T$.\\
Let $T_1$ be the tree obtained from $T$ after applying step 2 once.\\
Observe that $ a_2, \ldots, a_{n-2} $ is the Pruefer code of the tree $T_1$. Note that the vertices of $T_1$ are labelled by $ [ n]  \setminus \left\{ n_0 \right\}  $, where $n_0$ is the smallest leaf in $T$.\\
Also, the set $ \left\{ 1,2 \ldots, n \right\} \setminus \left\{ a_1, a_{2} , \ldots, a_{n-2} \right\} $ is the set of all leaves of $T$.\\
Now we apply algorithm 2 to the sequence $ ( a_1, \ldots, a_{n-2} ) $.\\
The smallest element of the list $ [ n] $ which is not in the sequence $ ( a_1, \ldots, a_{n-2} ) $ is the smallest leaf $n_0$ of $T$.\\
The first iteration adds an edge between nodes $n_0$ and $a_1$.\\
By the assumption, algorithm 2 applied to the sequence $ ( a_2,\ldots, a_{n_{n-2} } ) $ will give us the tree $T_{1}$ with vertices labeled by the set $ \left\{ 1, \ldots, n \right\} \setminus \left\{ n_0 \right\} $.\\
This finishes the proof

\end{proof}



\end{proof}
\subsubsection{Estimating the number of unlabelled trees}
There is no explicit formula for the number of unlabelled trees with $n$ vertices.
\begin{thm}
	Let $T_n$ be the number of unlabelled trees with $n$ vertices. Then
\[ 
\underbrace{2^{n} }_{ \text{ for } n>30}\leq  \frac{n^{n-2}}{n!} \leq  T_n \leq  4^{n-1}
\]
	
\end{thm}
\begin{proof}
First, we prove that $T_n \geq \frac{n^{n-2}}{n!}$.\\
We know that
\[ 
	T_n \geq \frac{| \text{ labelled trees } |}{ \text{ max number of trees in one equivalence class } }
\]
This max number of trees is $n!$.\\
Now we show that $T_{n} \leq 4^{n-1}$.\\
Take an unlabelled tree $T$ with $n$ vertices.\\
Choose one vertex of $T$ and call it a root.\\
Embed $T$ into a plane
\begin{figure}[H]
    \centering
    \incfig{embedded-tree}
    \caption{embedded tree}
    \label{fig:embedded-tree}
\end{figure}
We start from the root and go counterclockwise around the tree and write
\begin{itemize}
\item $+$ if the distance to the root increases
\item $-$ if the distance to the root decreases
\end{itemize}
At the end of the path, we obtain a sequence $ \left\{ +,- \right\} ^{2n-2}$.\\
An unlabelled tree can be uniquely reconstructed from the sequence.\\
\begin{itemize}
\item Each tree correspond to at least one and possibly more sequences
\item Each sequence corresponds to one unlabelled tree
\item Some sequences do not correspond to any tree	
\end{itemize}


\end{proof}
																	
\end{document}	
