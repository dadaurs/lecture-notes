\documentclass[../main.tex]{subfiles}
\begin{document}
\lecture{14}{Sat 29 May}{Finite probability spaces}
\section{Finite probability spaces and probabilistic methods}

\begin{defn}[Finite probability space]
	A finite probabilite space is a pair $( \Omega, P) $ , where $\Omega$ is a finite set and $P : 2^{\Omega}\to [ 0,1] $ such that
	\begin{enumerate}
		\item $P( \emptyset) =0$ 
		\item $P( \Omega) =1$ 
		\item $P( A\cup B) = P( A) + P( B) $ for any two disjoint sets $A,B \subset \Omega$ 
	\end{enumerate}
\end{defn}
The set $\Omega$ can be thought as the set of all possible outcomes of some random experiment.
The elements of $\Omega$ are called elementary events. Subsets of $\Omega$ are called events.\\
Let $w\in \Omega	, A,B \subset \Omega$ 
\begin{itemize}
\item $w \in A \leftrightarrow$ event $A$ occured
\item $w \in A\cap B \leftrightarrow$ both events occured
\item $A\cap B = \emptyset \leftrightarrow$ events $A$  and $B$ are incompatible
\item $P( A) \leftrightarrow$ the probability of event $A$ 
\end{itemize}
\subsection{A random graph}
We define 
\[ 
\Omega= \mathcal{G}_n \coloneqq \text{ set of all possible labelled graphs on vertex set  } V = \left\{ 1, \ldots, n \right\} 
\]
for $A \subset \mathcal{G}_n$ .\\
We define the probability function to be
\[ 
	P( A)  \coloneqq \frac{|A|}{|\Omega|}= |A| 2^{- \binom n 2}
\]
\begin{propo}
A random graph is almost never a tree, i.e.
\[ 
	\lim_{n \to  + \infty} P( `` \text{ A graph in $\mathcal{G}_n$ is a tree } '' ) =0
\]

\end{propo}
\begin{proof}
\[ 
	P( T_n) = \frac{|T_n|}{| \mathcal{G}_n|}
\]
By caley's theorem $|T_n| = n^{n-2}$ , hence
\[ 
	\lim_{n \to  + \infty} \frac{n^{n-2}}{2^{ \frac{n ( n-1) }{2}}}= \lim_{n \to  + \infty} e^{- \ln( 2) \frac{n( n-1) }{2}+ ( n-2) \ln ( n) } =0
\]

\end{proof}
\begin{defn}[independent events]
	Two events $A,B$ in a  probability space $(  \Omega, P) $ are called independent if
	\[ 
		P( A \cap B) = P( A) \cdot P( B) 
	\]
	
\end{defn}
One can easily generate this definition
\begin{defn}
	Events $A_1, \ldots, A_n \subset \Omega	$ are independent if for each set of indices $I \subset [ n] $ 
	\[ 
		P( \bigcap_{i \in I} A_i) = \prod_{i\in I} P( A_i) 		
	\]
	
	
\end{defn}
\begin{defn}
	Let $( \Omega, P) $ be a finite probability space. A random variable on $\Omega$ is any map $f: \Omega \to \mathbb{R}$.
\end{defn}
\begin{defn}
	Let $( \Omega, P) $ be a finite probability space, and let $f$ be a random variable on it.\\
	The expectation of $f$ is a real number $ \mathbb{E}( f) $ defined by the formula
	\[ 
		\mathbb{E}( f)  \coloneqq \sum_{w \in \Omega}^{ }P(  \left\{ w \right\} ) \cdot f( w) 
	\]
	
\end{defn}
Is there an easy way to compute $ \mathbb{E}( f) $ ?
\begin{defn}[Indicator]
	Let $A \subset \Omega$ be an event in a probability space $( \Omega, P) $ . The indicator of the event $A$ is the random variable $I_A : \Omega \to \left\{ 0,1 \right\} $ defined as
	\[ 
		I_A( w) \coloneqq  
		\begin{cases}
		1 \text{ fo } w \in A\\
		0 \text{ for } w \notin A
		\end{cases}
	\]
	
\end{defn}
\begin{lemma}
	For any event $A$ , we have $\mathbb{E}( I_A) = P( A) $ 
\end{lemma}
\begin{proof}
\[ 
	\mathbb{E}( I_A) = \sum_{w \in \Omega}^{ }I_A( w) P( \left\{ w \right\} ) = \sum_{w \in A}^{ }P( \left\{ w \right\} ) = P( A) 
\]

\end{proof}
\begin{thm}[linearity of expectation]
	Let $f,g$ be arbitrary random variables on a finite probability space $ ( \Omega, P) $ and let $\alpha \in \mathbb{R}$ . Then
	\[ 
		\mathbb{E}( \alpha f) = \alpha \mathbb{E}( f) \text{ and } \mathbb{E}( f+g)= \mathbb{E}( f) + \mathbb{E}( g) 
	\]
\end{thm}
\subsection{Probabilistic method}
\begin{thm}
	Let $(  \Omega, P) $ be a finite probability space and let $f: \Omega \to \mathbb{R}$ be a random variable.\\
	If $\mathbb{E}( f) = m$, then there exists a t least one elementary event $w_1$ such that $f( w_1) \geq m$ .\\
	Analogously, there exists at least one elementary event $w_2$ such that $f( w_2) \leq m$ 
\end{thm}
We consider two applications of this theorem.
\begin{thm}
Let $G$ be a graph with an even number $2n$ of vertices and with $m>0$ edges.\\
Then the set $V= V( G) $ can be divided into two disjoint $n-$element subsets $A$ and $B$ in such a way that more than $\frac{m}{2}$ edges go between $A$ and $B$ .
\end{thm}
\begin{proof}
	Consider the probability space $( \Omega, P) $ where $\Omega = \binom V n$  with the probability measyre
	\[ 
		P( S) \coloneqq  \frac{|S|}{| \Omega|} \text{ for } S \subset \Omega
	\]
	Let $A \in \Omega$ be a random $n$ -element subset of $V( G)$ .\\
	Define $B \coloneqq V( G) \setminus A$ its complement.\\
	Consider the following random variable:
	\[ 
		X( A) \coloneqq | \text{ edges between $A$ and $B$  } | = | \left\{ \left\{ a,b \right\} | a \in A, b \in B, \left\{ a,b \right\} \in E( G)  \right\} .	
	\]
	Let's compute $ \mathbb{E}(X)$.\\
	For $e = \left\{ u,v \right\}  \in E( G) $ we define the event
	\[ 
	C_e \coloneqq \left\{ A \in \Omega | |A\cap e| =1 \right\} 		
	\]
	We have 
	\[ 
		X = \sum_{e \in E( G) }^{ }I_{C_e} 
	\]
	and therefore
	\[ 
		\mathbb{E}( X) = \sum_{e \in E( G) }^{ } \mathbb{E}( I_{C_e} ) = \sum_{e \in E( G) }^{ }P( C_e) 
	\]
	We compute $P( C_{e} ) = \frac{2 \binom { 2n-2} { n-1} }{ \binom { 2n} n}> \frac{1}{2}$.\\
	Thus 
	\[ 
		\mathbb{E}( X) = \sum_{e \in E( G) }^{ }P( C_e) > \frac{m}{2}
	\]
	By theorem, $X( A) > \frac{m}{2}$ for some $A \in \Omega$ .
	
\end{proof}
\begin{defn}	
	Let $G$ be a graph. A set $S \subset V( G) $ 	is an independent set if no two vertices of $S$ are connected by an edge.\\
	We define $\alpha( G) $ to be the size of the largest independent set of vertices in the graph $G$.
\end{defn}
\begin{thm}[Turan's theorem]
For every graph $G$ we have
\[ 
	\alpha( G)  \geq \frac{|V( G) |^{2}}{2 |E( G) | + |V( G)| }
\]

\end{thm}
\begin{lemma}
For any graph $G$ we have
\[ 
	\alpha( G)  \geq \sum_{v\in V( G) }^{ }\frac{1}{\deg v + 1}
\]

\end{lemma}
\begin{proof}
Suppose that the vertices of $G$ are numberd $1 , \ldots, n$ .\\
Pick a random permutation $\pi$ of the vertices.\\
Define the set $M( \pi) \subset V( G) $ by
\[ 
	M( \pi) \coloneqq  \left\{ v \in V | \text{ all neighbours $u$ of $v$ satisfy $\pi( u) >\pi ( v) $  }  \right\} 
\]
$M( \pi) $ is an independent set of $G$ .\\
Therefore $|M( \pi) | \leq  \alpha ( G) $ for any permutation $\pi$ , hence
\[ 
	\mathbb{E}( |M( \pi) |) \leq  \alpha( G) 
\]
Now we calculate the expected size of $M$ .\\
Let $v \in V$ , $A_V \coloneqq $ the event `` $v \in M( \pi) $ '' 
Then
\[ 
	P( A_v) = \frac{1}{\deg v + 1}
\]
The lemma follows.




\end{proof}
\begin{proof}[of Turan's theorem]
	Let $|V( G) | = n$ and $d_1, \ldots, d_n$ be the degrees of the vertices of $G$ .\\
	Then
	\[ 
		\sum_{i=1}^{ n}d_i = 2 |E( G) |	
	\]
	\begin{align*}
	\sum_{i=1}^{ n}\frac{1}{d_i+1} \geq \frac{n^{2}}{d_1 + \ldots + d_n + n}= \frac{n^{2}}{2 |E| + n}
	\end{align*}
	Which proves Turan's theorem.
	

\end{proof}

	


 

\end{document}	
