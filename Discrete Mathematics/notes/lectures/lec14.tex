\documentclass[../main.tex]{subfiles}
\begin{document}
\lecture{14}{Sat 29 May}{Finite probability spaces}
\section{Finite probability spaces and probabilistic methods}

\begin{defn}[Finite probability space]
	A finite probabilite space is a pair $( \Omega, P) $ , where $\Omega$ is a finite set and $P : 2^{\Omega}\to [ 0,1] $ such that
	\begin{enumerate}
		\item $P( \emptyset) =0$ 
		\item $P( \Omega) =1$ 
		\item $P( A\cup B) = P( A) + P( B) $ for any two disjoint sets $A,B \subset \Omega$ 
	\end{enumerate}
\end{defn}
The set $\Omega$ can be thought as the set of all possible outcomes of some random experiment.
The elements of $\Omega$ are called elementary events. Subsets of $\Omega$ are called events.\\
Let $w\in \Omega	, A,B \subset \Omega$ 
\begin{itemize}
\item $w \in A \leftrightarrow$ event $A$ occured
\item $w \in A\cap B \leftrightarrow$ both events occured
\item $A\cap B = \emptyset \leftrightarrow$ events $A$  and $B$ are incompatible
\item $P( A) \leftrightarrow$ the probability of event $A$ 
\end{itemize}
\subsection{A random graph}
We define 
\[ 
\Omega= \mathcal{G}_n \coloneqq \text{ set of all possible labelled graphs on vertex set  } V = \left\{ 1, \ldots, n \right\} 
\]
for $A \subset \mathcal{G}_n$ .\\
We define the probability function to be
\[ 
	P( A)  \coloneqq \frac{|A|}{|\Omega|}= |A| 2^{- \binom n 2}
\]
\begin{propo}
A random graph is almost never a tree, i.e.
\[ 
	\lim_{n \to  + \infty} P( `` \text{ A graph in $\mathcal{G}_n$ is a tree } '' ) =0
\]

\end{propo}
\begin{proof}
\[ 
	P( T_n) = \frac{|T_n|}{| \mathcal{G}_n|}
\]
By caley's theorem $|T_n| = n^{n-2}$ , hence
\[ 
	\lim_{n \to  + \infty} \frac{n^{n-2}}{2^{ \frac{n ( n-1) }{2}}}= \lim_{n \to  + \infty} e^{- \ln( 2) \frac{n( n-1) }{2}+ ( n-2) \ln ( n) } =0
\]

\end{proof}
\begin{defn}[independent events]
	Two events $A,B$ in a  probability space $(  \Omega, P) $ are called independent if
	\[ 
		P( A \cap B) = P( A) \cdot P( B) 
	\]
	
\end{defn}
One can easily generate this definition
\begin{defn}
	Events $A_1, \ldots, A_n \subset \Omega	$ are independent if for each set of indices $I \subset [ n] $ 
	\[ 
		P( \bigcap_{i \in I} A_i) = \prod_{i\in I} P( A_i) 		
	\]
	
	
\end{defn}



\end{document}	
