\documentclass[../main.tex]{subfiles}
\begin{document}
\lecture{8}{Sat 17 Apr}{Moebios inversion formula}
\section{Moebius inversion formulas}
\subsection{Moebf}


Let $f: \mathbb{Z}_{\geq 1}  \to \mathbb{C}$ be a function, we define a new function $F: \mathbb{Z}_{\geq 1} \to \mathbb{C}$ by
\[ 
	F( n) \coloneqq \sum_{d|n} f( d) , n  \in   \mathbb{Z}_{\geq 1} 
\]
\begin{exemple}
	Let $f( n) =1$ for all $n \in \mathbb{Z}_{\geq 1} $, then $F( n) = \sum_{d|n} 1$ which is the number of divisors of $n$.
\end{exemple}
\subsection*{Question}
Suppose that we know $F$. How do we recover $f$?
\begin{defn}[Moebius function]
	\[ 
	\mu: \mathbb{Z}_{\geq 1} \to \mathbb{Z}
	\]
	is defined as follows.\\
	Suppose that $n \in \mathbb{Z}_{\geq 1} $ has the prime factorization
	\[ 
	n = p_1 ^{e_1}\cdot \ldots \cdot p_r ^{e_r}
	\]
	then 
	\begin{align*}
		\mu( n) \coloneqq 
		\begin{cases}
		1 \text{ for } n=1\\
		0 \text{ if some } e_i >1\\
		( -1) ^{r} \text{ if }  e_1 = e_2= \ldots = 1
		\end{cases}
	\end{align*}
	
\end{defn}
\begin{lemma}
For $n \in \mathbb{Z}_{\geq 1} $ we have
\[ 
	\sum_{d|n} \mu ( d)  = 
	\begin{cases}
	1, \text{ if }  n=1\\
	0 , \text{ if }  n >1
	\end{cases}
\]

\end{lemma}
\begin{proof}
By induction, we check that 
\[ 
	\sum_{d|1} \mu( d )  = \mu ( 1) =1
\]
Now suppose that $n \in \mathbb{Z}_{\geq 1} $ has the prime factorization
\[ 
n= p_1^{e_1 }\cdot \ldots \cdot p_r^{e_r}
\]
Set $n^{*} \coloneqq  \prod p_i$, the square free part of $n$.\\
If $d| n$ and $d\not| n*$, then $d$ has a prime divisor of multiplicity $>1$, then $\mu( d) = 0$. Hence
\[ 
	\sum_{d|n}  \mu( d) = \sum_{d| n^*} \mu( d)
\]
Now can easily compute
\begin{align*}
\sum_{d| n^{*}} \mu( d)  &= \sum_{d| p_1\ldots p_r} \mu ( d)
&= \sum_{I \subset \left\{ 1, \ldots, r \right\} } \mu ( \prod_{i \in I} p_i) \\
&= \sum_{I \subset \left\{ 1,\ldots,r \right\} } ( -1) ^{|I|}\\
&= \sum_{k=0}^{ r} ( -1) ^{k} \binom r k = ( 1-1) ^{r}= 0	
\end{align*}

\end{proof}
\begin{thm}[Moebius inversion formula]
	Let $f,F: \mathbb{Z}_{\geq 1} \to \mathbb{C}$ be such that
	\begin{align}
		F( n) = \sum_{d|n} f( d) , \quad n \in \mathbb{Z}_{\geq 1} 
	\end{align}
	Then
	\begin{align}
		f( n) = \sum_{d|n} \mu( d) F( \frac{n}{d}) 
	\end{align}
	Moreover, $( 2) $ implies $( 1) $
\end{thm}
\begin{proof}
Let $d $ and $n$ be positive integers such that $d|n $.\\
Then $F( \frac{n}{d}) = \sum_{d'| \frac{n}{d}} f( d') $, therefore
\[ 
	\sum_{d|n} \mu( d) F( \frac{n}{d})  = \sum_{d|n} \mu( d)  \sum_{d'| \frac{n}{d}} f( d') 
\]
Consider the set $S_n$ of all pairs $( d,d') \in \mathbb{Z}_{\geq 1} $ such that 
\[ 
d|n \text{ and } d' | \frac{n}{d}
\]
If $n$ and its divisor $d'$ are fixe, then $d$ runs over all divisors of $\frac{n}{d'}$.\\
Hence, we can change the order of summation
\begin{align*}
	\sum_{d|n} \sum_{d' | \frac{n}{d}} \mu( d)  f( d') &= \sum_{( d,d') \in S_n} \mu( d) f( d')\\
	&= \sum_{d'|n} \sum_{d| \frac{n}{d'}} f( d' )  \mu( d)
\end{align*}
Using the lemma above, we get
\begin{align*}
	\sum_{d'|n} \sum_{d| \frac{n}{d'}} f( d' )  \mu( d) &= \sum_{d'|n} f( d') \sum_{d| \frac{n}{d'}} \mu( d) \\
	&=f( n) 
\end{align*}
We now show the other implication, namely\\
Let $F: \mathbb{Z}_{\geq 0} \to \mathbb{C} $ be any function.\\
Set $f( n) \coloneqq \sum_{d|n} \mu(d) F( \frac{n}{d}) $, then for $n \in \mathbb{Z}_{\geq 1} $ 
\begin{align*}
	\sum_{d|n} f( d) = \sum_{d|n} \sum_{d'|d} \mu( d') F( \frac{d}{d'}) 
\end{align*}
we make the change of variables
\[ 
	\left\{ ( d,d') | d|n, d'|d \right\} \to \left\{ ( d'',d') | d'' |n, d' |\frac{n}{d''} \right\} 
\]
Then
\[ 
	= \sum_{d''|n} \sum_{d'| \frac{n}{d''}} \mu( d') F( d'') = \sum_{d''|n}^{ } F( d'') \sum_{d'| \frac{n}{d''}}^{ }\mu( d') = F( n) 
\]
\end{proof}
\subsection{Computing the number of cyclic sequences}
\begin{defn}[Linear sequence]
	Let $A$ be a set. A linear sequence of length $n$ in the alphabet $A$ is an element of $A^{n}$ :
	\[ 
		a= ( a_1, \ldots, a_n) , a_k \in A  \text{ for }  k=1, \ldots,n
	\]
The number of linear sequences of length $n $ in an alphabet of size $r$ is $r^{n}$.\\
Consider the following equivalence relation on the set of linear sequences
\[ 
	( a_1, \ldots, a_n) \sim ( a_2, \ldots, a_n, a_1) 
\]
Two linear sequences are equivalent if one of them can be obtained from another by a cyclic shift.
\end{defn}
\begin{exemple}
\begin{figure}[H]
    \centering
    \incfig{linear-sequence}
    \caption{Linear sequence}
    \label{fig:linear-sequence}
\end{figure}
\end{exemple}
\begin{defn}[Cyclic sequence]
	A cyclic sequence of length $n$ in an alphabet $A$ is an equivalence class of linear sequences with respect to the relation $\sim$.		
\end{defn}
\begin{propo}
	The number of  $T( n, r) $ of cyclic sequences of length $n$ on an alphabet of size $r$ is
	\[ 
		T( n,r)  = \frac{1}{n} \sum_{d|n}^{ } \phi( \frac{n}{d}) r^{d}
	\]
	Here, $\phi( \cdot) $ is the Eulers totient function.
\end{propo}
\begin{defn}[Period of cyclic sequence]
	A period of a cyclic sequence $( a_1, \ldots, a_n) $ is a minimal number $k \in \left\{ 1, 2, \ldots, n \right\}  $ such that 
	\[ 
		( a_1, a_2, \ldots, a_n)  = ( a_{1+k} , a_{2+k} , \ldots, a_1, \ldots, a_k) 
	\]
	are equal as linear sequences.

\end{defn}
\subsection*{Exercise}

Show that a $k$ is always a divisor of $n$.

Let $M( d,r) $ be the number of cyclic sequences of length $d$ and of period $r$.\\
We notice that
\[ 
	r^{n} = \sum_{d|n}^{ } d \cdot M( d,r) 
\]
Indeed, notice that there exists $\pi$ a projection from the linear sequences of length $n$ into the cyclic sequences.\\
The number of preimages of a cyclic sequence under the map $\pi$ is $d$, the period of the sequence, therefore, if we denote by $ \mathcal{L} ( n,r)   $ the set of linear sequences, we get
\[ 
	r^{n}=| \mathcal{L} ( n,r) | = \sum_{d|n}^{ } d \cdot | M( d,r) |
\]
Applying the Moebius inversion formula, we obtain
\[ 
	n \cdot M( n,r)  = \sum_{d|n}^{ } \mu ( \frac{d}{n}) r^{d}
\]
Each cyclic sequence has a well defined period $d$ and it corresponds to the unique cyclic sequence of length $d$ and period $d$.
Thus
\[ 
	T( n,r)  = \sum_{d|n}^{ } M( d,r) 
\]
Combining both formulas above, we get
\begin{align*}
	T( n,r) &= \sum_{d|n}^{ } M( d,r) \\
		&= \sum_{d|n}^{ } \frac{1}{d} \sum_{d'|d}^{ } \mu( \frac{d}{d'}) r^{d'}\\
		&= \sum_{d'|n}^{ } \sum_{d''| \frac{n}{d'}}^{ } \frac{1}{d'\cdot d''} \mu( d'') r^{d'}
\end{align*}
Now we have to compute the sum 
\[ 
	\sum_{d''| \frac{n}{d'}}^{ } \frac{1}{d''} \mu( d'') 
\]
As an exercise, show that for $n \in \mathbb{Z}_{ \geq 1}, \sum_{d|n}^{ } \frac{1}{d} \mu( d)  = \frac{\phi( n) }{n} $.\\
Using this, we finally obtain that
\[ 
	T( n,r) = \sum_{d'|n}^{ } \frac{\phi( \frac{n}{d'}) }{n} r^{d'}
\]
\subsection{Moebius inversion for posets}
\begin{defn}[Binary relation ]
	A binary relation on a set $A$ is a subset $R \subseteq A \times A$.\\
	A relation is antisymmetric provided $(a,b) \in R$ and $( b,a) \in R$ imply $a=b$.\\
	A relation is transitive if $( a,b) \in R$ and $( b,c) \in R$ imply $( a,c) \in R$ \\
	A relation is reflexive if $( a,a) \in R$ for all $a\in R$.
\end{defn}
\begin{exemple}
\begin{enumerate}
\item The relation $ \leq $ on $\mathbb{Z}$ is antisymmetric, transitive, and reflexive
\item The relation $<$ on  $\mathbb{Z}$ is antisymmetric, transitive and not reflexive
\item The relation \textit { coprime}  is not antysimmetric, not transitive and not reflexive
\end{enumerate}
		
\end{exemple}
\begin{defn}[Partial Order]
	A partial order on a set $A$ is an antisymmetricm reflexive and transitive relation $R \subseteq A \times A$.\\
	A partially ordered set ( or poset)  is a set together with a partial order	
\end{defn}
\begin{exemple}
Let $A$ be  a set. The set $2^{A}$ of subsets of $A$ is a partially ordered set by inclusion
\begin{itemize}
\item Reflexivity: $X \subseteq X$ 
\item Transitivity : if $X  \subseteq y$ and $Y \subseteq Z$ then $X \subseteq Z$ 
\item Antisymnmetric: if $X  \subseteq Y $ and $Y \subseteq X$ then $X=Y$
\end{itemize}

\end{exemple}
\begin{exemple}
The set $\mathbb{Z}_{ \geq 1} $ is partially ordere by the relation $d$ divides $n$.\\
\begin{itemize}
\item Reflexivity: $n$ divides $n$ 
\item Transitivity: if $d|n$ and $ d'| d$ then $d'|n$ 
\item Antisymnmetric: if $n|m$ and $m|n$ then $m=n$.
\end{itemize}

\end{exemple}
We can represent such relations with Hasse diagrams in the following way
\begin{figure}[H]
    \centering
    \incfig{hasse-diagram-12}
    \caption{Hasse diagram 12}
    \label{fig:hasse-diagram-12}
\end{figure}
\begin{defn}[Locally finite poset]
	A partially ordered set $( X, \geq ) $ is locally  finite if for all $x,y\in X$ the interval
	\[ 
	[ x,y] \coloneqq \left\{ z \in X| y \leq z \leq x \right\} 
	\]
	is finite.\\
We say that $0\in X$ is a zero element if $0\leq x$ for all $x \in X$
\end{defn}
Let $ ( X, \leq ) $ be a partiallly ordered locally finite set with $0$.\\
Suppose that $f: X \to \mathbb{C}$ is a function.\\
We define a new function $F: X \to \mathbb{C}$ by
\[ 
	F( x) \coloneqq  \sum_{y \leq x}^{ } f( y) 
\]
How do we recover $f$ from $F$?
\begin{thm}[Moebius inversion for posets]
	Given a partially ordered set $X$, there is a two varaiable function $M:X \times X\to \mathbb{R}$ such that 
	\[ 
		F( x) = \sum_{y \leq  x}^{ } f( y)  \iff f( x) = \sum_{y \leq x}^{ } F( y)  M( x,y) 
	\]
	$M$ is called the Moebius function of the poset.
\end{thm}
\begin{defn}[Incidence algebra $A( X) $]
	Given a partially ordered set $X$, the incidence algebra $A( X) $ is the set of complex valued functions $f: X^{2} \to \mathbb{C}$ satisfying $f( x,y) =0$ unless $x \leq y$.\\
	$A( X) $ is a vector space over $\mathbb{C}$ with respect to pointwise addition and multiplication by scalars.\\
To make it into an algebra we need one more operation
	
\end{defn}
\begin{defn}[Convolution]
	Given $f, g \in A( X) $, their convolution $f \ast g$ is defined by
	\[ 
		f \ast g ( x,y)  = \sum_{x \leq z \leq y}^{ } f( x,z) g( z,y) 
	\]
	
\end{defn}
\begin{defn}
	The delta function $\delta $ is the element of $A( X) $ 
	\[ 
		\delta ( x,y)  = 
		\begin{cases}
		1 \text{ if }  x=y\\
		0 \text{ otherwise } 
		\end{cases}
	\]
	

\end{defn}
\begin{rmq}
Convolution is not always commutative.
\end{rmq}
\begin{lemma}
	A function $f\in A( X) $ has a left and right inverse with respect to the convolution if and only if $f( x,x) \neq 0$ for all $x\in X$
\end{lemma}
\begin{proof}
	Given $f\in A( X) $ we find $g \in A( X) $ such that
	\[ 
		f \ast g ( x,y) = \sum_{x \leq z \leq y }^{ } f( x,z) g( z,y)  = \delta ( x,y) 
	\]
	For all $x \in X$ 
	\[ 
		f \ast g( x,x)  = f( x,x)  g( x,x)  = \delta ( x,x)  = 1
	\]
	Therefore, the condition $f( x,x) \neq 0 $ is necessary for the existence of the inverse.\\
	We define $g( x,x) \coloneqq f( x,x) ^{-1}$.\\
	To define $g( x,y) $ for $x <y$, we assume by induction, that we have already found all $g( z,y) $ for all $z$, satisfying $x<z \leq y$.\\
	Then
	\begin{align*}
	\delta( x,y) = 0 = \sum_{x \leq z \leq y}^{ } f( x,z)  g( z,y) \\
	- f( x,x) g( x,y) = \sum_{x \leq z \leq y}^{ } f( x,z)  g( z, y) 
	\end{align*}
	We can now solve for $g( x,y) $.\\
	Finally, if $f\ast g_1 =\delta$ and $g_2 \ast f = \delta$, then
	\[ 
	g_2 = g_2 \ast \delta = g_2 \ast f \ast g_1 = \delta \ast g_1 = g_1
	\]
	Therefore, the left and the right inverses of $f$ coincide.
\end{proof}
\begin{defn}[Zeta function]
	The Zeta function $Z( x,y) $ of the poset $( X, \leq ) $ is the function
	\[ 
		Z( x,y)  =
		\begin{cases}
		1 \text{ if }  x \leq y\\
		0 \text{ otherwise } 
		\end{cases}
	\]
	
\end{defn}
\begin{defn}[Moebius function]
	The Moebius function $M( x,y) $ is the inverse of the zeta function $Z$ with respect to convolution.
\end{defn}
We can now prove the Moebius inversion for posets

\begin{proof}
	Let $f:X \to \mathbb{C}$ be a function and $F:X \to \mathbb{C}$ be defined by
	\[ 
		F( x) = \sum_{y \leq x }^{ } f( y) 
	\]
Then, for a fixed $x \in X$ :
\begin{align*}
	\sum_{y \leq x}^{ } F(y) M( y,x)  &= \sum_{y \leq x}^{ } M( y,x) \sum_{z \leq y}^{ } f( z) \\
					  &= \sum_{z \leq  y \leq x}^{ } f( z) M( y,x) \\
					  &= \sum_{z \leq x}^{ } f( z)  \sum_{z \leq  y \leq x}^{ } M( y,x) \\
					  &= \sum_{z \leq  x}^{ } f( z)  \left( \sum_{z \leq  y \leq x}^{ } Z( z,y) M( y,x) \right) \\
					  &= \sum_{z \leq  y}^{ } f( z)  ( Z \ast M) ( z,x) \\
					  &= \sum_{z \leq  y}^{ } f( z)  \delta( z,x) 
					  &= f( x) 	
\end{align*}
Now we show that the inverse is also true.\\
Let $F: X \to \mathbb{C}$ be a function and we define $f: X \to \mathbb{C}$ by
\[ 
	f( x) = \sum_{y \leq x}^{ } F( y) M( y,x) , x \in X
\]
Then,
\begin{align*}
	\sum_{y \leq x}^{ } f( y) &= \sum_{y \leq x}^{ } \sum_{z \leq  y}^{ } F( z) M( z,y) \\
				  &= \sum_{z \leq  y \leq x}^{ } F( z)  M( z,y) \\
				  &= \sum_{z \leq  y \leq x}^{ } F( z) M( z,y) Z( z,y) \\
				  &= \sum_{z \leq  x}^{ } F( z)  \left( \sum_{z \leq y \leq x}^{ } M( z,y) Z( y,x)  \right) \\
				  &= \sum_{z \leq  x}^{ } F( z)  \left( \sum_{z \leq  y \leq x}^{ } M( z,y) Z( y,x)  \right) \\
				  &= \sum_{z \leq  x}^{ } F( z)  \delta( z,x) = F( x) 
\end{align*}
\end{proof}
\begin{lemma}
	Let $2^{A}$ be the set of subsets of a finite set $A$.\\
	$2^{A}$ is partially ordered by inclusion.\\
	The Moebius function of $2^{A}$ is given by
	\[ 
		M( x,y) = ( -1) ^{|x| - |y|}
	\]
	Where $x \subseteq y \subseteq A$
\end{lemma}
\begin{proof}
We have to show that $M \ast Z = \delta$. Let $x,y$ be two subsets of $A$ such that $x \subseteq y$.\\
We compute
\begin{align*}
	M \ast Z ( x,y)  &= \sum_{x \subseteq z \subseteq y}^{ } M( x,z) Z( z,y) \\
			 &= \sum_{x \subseteq z \subseteq  y}^{ } ( -1) ^{|x| - |z| }\\
			 &= \sum_{w \subseteq y \setminus x }^{ } ( -1)^{|w|}
			 &=
			 \begin{cases}
			 0 \text{ if }  |y \setminus x| \geq 1\\
			 1, |y\setminus x| = \emptyset
			 \end{cases}
			 = \delta( x,y) 
\end{align*}

\end{proof}












		



\end{document}	
