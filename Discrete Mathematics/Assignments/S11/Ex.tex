\documentclass[11pt, a4paper]{article}
\usepackage[utf8]{inputenc}
\usepackage[T1]{fontenc}
\usepackage[francais]{babel}
\usepackage{lmodern}
\usepackage{amsmath}
\usepackage{amssymb}
\usepackage{amsthm}
\renewcommand{\vec}[1]{\overrightarrow{#1}}
\newcommand{\del}{\partial}
\DeclareMathOperator*{\sgn}{sgn}
\DeclareMathOperator*{\id}{Id}
\DeclareMathOperator*{\im}{Im}
\DeclareMathOperator*{\re}{Re}
\DeclareMathOperator*{\vol}{Vol}
\newcommand\norm[1]{\left\vert#1\right\vert}
\newcommand\ns[1]{\left\vert\left\vert\left\vert#1\right\vert\right\vert\right\vert}
\newcommand\Norm[1]{\left\lVert#1\right\rVert}
\newcommand\N[1]{\left\lVert#1\right\rVert}
\newcommand\abs[1]{\left\vert#1\right\vert}
\newcommand\inj{\hookrightarrow}
\newcommand\surj{\twoheadrightarrow}
\newcommand\ded[1]{\overset{\circ}{#1}}
\newcommand\sidenote[1]{\footnote{#1}}
\newcommand\eng[1]{\left\langle#1\right\rangle}
\newcommand\hr{
    \noindent\rule[0.5ex]{\linewidth}{0.5pt}
}

\newcommand{\incfig}[1]{%
    \def\svgwidth{\columnwidth}
    \import{./figures}{#1.pdf_tex}
}
\newcommand{\filler}[1][10]%
{   \foreach \x in {1,...,#1}
    {   test 
    }
}

\newcommand\contra{\scalebox{1.5}{$\lightning$}}
\makeatother
\def\@lecture{}%
\newcommand{\lecture}[3]{
    \ifthenelse{\isempty{#3}}{%
        \def\@lecture{Lecture #1}%
    }{%
        \def\@lecture{Lecture #1: #3}%
    }%
    \subsection*{\@lecture}
    \marginpar{\small\textsf{\mbox{#2}}}
}

\begin{document}
\title{Exercise 11}
\author{David Wiedemann}
\maketitle
Without loss of generality, we can suppose that $e= \left\{ 1,n \right\} $.\\
This means that we can write $L( G) $ in the following way
\begin{align*}
L &= \begin{pmatrix}
	n-2 & -1 & -1 & \ldots  & 0\\
	-1 &n-1 & -1 & \ldots & -1\\
	-1 & -1 &n-1 &  \ldots & -1\\
	\vdots & \vdots & \vdots  & \ddots & \vdots\\
	0 & -1 &-1 &  \ldots  & n-2\\
\end{pmatrix} 
\intertext{Now we remove the last line and last column to obtain $L_0$, we also take the determinant of both sides}
\det L_0 &= \begin{vmatrix}
	n-2 & -1 & -1 & \ldots  & -1\\
	-1 &n-1 & -1 & \ldots & -1\\
	-1 & -1 &n-1 &  \ldots & -1\\
	\vdots & \vdots & \vdots  & \ddots & \vdots\\
	-1 & -1 &-1 &  \ldots  & n-1\\
\end{vmatrix}\quad \text{ Note that we now have a $n-1 \times n-1$ matrix. } 
\intertext{We now add each row to the top row, this does not change the value of the determinant.}
\det L_0 &= \begin{vmatrix}
	0 & 1 & 1 & \ldots  & 1\\
	-1 &n-1 & -1 & \ldots & -1\\
	-1 & -1 &n-1 &  \ldots & -1\\
	\vdots & \vdots & \vdots  & \ddots & \vdots\\
	-1 & -1 &-1 &  \ldots  & n-1\\
\end{vmatrix} 
\intertext{We can now add the first line to every other line, yielding the following matrix}
\det L_0 &= \begin{vmatrix}
	0 & 1 & 1 & \ldots  & 1\\
	-1 &n & 0 & \ldots & 0\\
	-1 & 0 &n &  \ldots &0\\
	\vdots & \vdots & \vdots  & \ddots & \vdots\\
	-1 & 0 &0 &  \ldots  & n\\
\end{vmatrix} 
\intertext{Then we exchange the first and last row, as well as the first and last collumn, this, again, does not change the value of the determinant.}
\det L_0 &= \begin{vmatrix}
	n &0 & 0 & \ldots & -1\\
	0 & n &0 &  \ldots & -1\\
	\vdots & \vdots & \vdots  & \ddots & \vdots\\
	1 & 1 &1&  \ldots  &0\\
\end{vmatrix} 
\intertext{ Now we add each row $\frac{-1}{n}$ times to the $n-1$-th row.}
\det L_0 &= \begin{vmatrix}
	n &0 & 0 & \ldots & -1\\
	0 & n &0 &  \ldots & -1\\
	\vdots & \vdots & \vdots  & \ddots & \vdots\\
	0 & 0 &0&  \ldots  & \frac{n-2}{n} \\
\end{vmatrix} 
\intertext{Since this now is a upper triangular matrix, we can easily compute it's determinant.}
	\det L_0 &= n^{n-2} \frac{n-2}{n} = n^{n-3} ( n-2) 
\end{align*}
We can now use Kirchhof's theorem to see that
\[ 
	\# \textbf{ number of spanning trees of $G$ } = n^{n-3} ( n-2) .
\]
Note that if $n=1$ , removing an edge yields an empty tree, which clearly doesn't contain a spanning tree.





\end{document}
