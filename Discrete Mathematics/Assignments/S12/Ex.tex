\documentclass[11pt, a4paper]{article}
\usepackage[utf8]{inputenc}
\usepackage[T1]{fontenc}
\usepackage[francais]{babel}
\usepackage{lmodern}
\usepackage{amsmath}
\usepackage{amssymb}
\usepackage{amsthm}
\renewcommand{\vec}[1]{\overrightarrow{#1}}
\newcommand{\del}{\partial}
\DeclareMathOperator*{\sgn}{sgn}
\DeclareMathOperator*{\id}{Id}
\DeclareMathOperator*{\im}{Im}
\DeclareMathOperator*{\re}{Re}
\DeclareMathOperator*{\vol}{Vol}
\newcommand\norm[1]{\left\vert#1\right\vert}
\newcommand\ns[1]{\left\vert\left\vert\left\vert#1\right\vert\right\vert\right\vert}
\newcommand\Norm[1]{\left\lVert#1\right\rVert}
\newcommand\N[1]{\left\lVert#1\right\rVert}
\newcommand\abs[1]{\left\vert#1\right\vert}
\newcommand\inj{\hookrightarrow}
\newcommand\surj{\twoheadrightarrow}
\newcommand\ded[1]{\overset{\circ}{#1}}
\newcommand\sidenote[1]{\footnote{#1}}
\newcommand\eng[1]{\left\langle#1\right\rangle}
\newcommand\hr{
    \noindent\rule[0.5ex]{\linewidth}{0.5pt}
}

\newcommand{\incfig}[1]{%
    \def\svgwidth{\columnwidth}
    \import{./figures}{#1.pdf_tex}
}
\newcommand{\filler}[1][10]%
{   \foreach \x in {1,...,#1}
    {   test 
    }
}

\newcommand\contra{\scalebox{1.5}{$\lightning$}}
\makeatother
\def\@lecture{}%
\newcommand{\lecture}[3]{
    \ifthenelse{\isempty{#3}}{%
        \def\@lecture{Lecture #1}%
    }{%
        \def\@lecture{Lecture #1: #3}%
    }%
    \subsection*{\@lecture}
    \marginpar{\small\textsf{\mbox{#2}}}
}

\begin{document}
\title{Exercise 12}
\author{David Wiedemann}
\maketitle
\section*{1}
We prove the double implication.\\
$ \Rightarrow $ \\
Let us denote by $A$ the adjacency matrix.\\
First suppose that $G$ contains a cycle of length three, without loss of generality, we can suppose that the three vertices of the cycle are numbered by $1,2$ and $3.$ \\
Hence, $ ( 1,2) \in E( G) $ , and we deduce that the $( 1,2) $ entry of the adjacency matrix is different from 0.\\
Now consider the $( 1,2) $ entry of $A^{2}$, applying the formula for matrix multiplication yields
\[ 
\left( A^{2} \right)_{1,2} = \sum_{i=1}^{ n} A_{1,i} A_{i,2} 
\]
Note that if $i=3$ , by defintion $A_{1,3} A_{3,1} =1$, and, since all other terms of the sum are nonnegative, $ ( A^{2}) _{1,2} \geq 1$.\\
$\Leftarrow$ \\
Now suppose that $G$ is a graph such that $A_{i,j} \neq 0$ and $( A^{2}) _{i,j} \neq 0 $.\\
This implies that the vertices $i $ and $j$ are adjacent.\\
Furthermore, this implies that
\[ 
\sum_{k=1}^{ n}A_{i,k} A_{k,j} \neq 0
\]
This implies that there exists $l \in [ n] $ such that $A_{i,l} = A_{l,j} = 1$, and hence $ ( i,j) , ( i,l) , ( l,j) \in E( G) $, which means $G$ contains a triangle.
\section*{2}
We will proceed by induction on the number $n$  of vertices of $T$ which are not leafs.\\
If $n=1$, let $v$ be the vertice of degree different to 1 and $l_1, \ldots, l_k$ the set of all leafs.\\
Since $ f( l_i) = g( l_i) \forall i \in [ k]  $ and since $f$ and $G$ are bijections on the set of vertices, we immediatly deduce that $f( v) = g( v) $ .\\
Suppose the result shown for all natural numbers smaller than $n$, we will now show it for $n+1$.\\
Let $l_1, \ldots, l_k$ again be the set of all the leafs.\\
We first prove that all vertices $v$  which are adjacent to a leaf satisfy $f( v) = g( v) $ .\\
Indeed, let $l$ be a leaf adjacent to $v$ , then $( v,l) \in E( T) $ implies that $f(v) , f( l) $ are adjacent and that $g( v) , g( l) $ are adjacent.\\
However, since by hypothesis $g( l) = f( l) $, and since a graph isomorphism preserves the degrees of all vertices, we deduce that $f( v) = g( v) $ .\\
We now consider a new tree $T' = T - \left\{ l_1, \ldots, l_k \right\} $.\\
Note that all leafs of $l \in T'$ satisfy that $ f( l) = g( l) $.\\
Also note that, since a tree always contains a leaf, the number of vertices which are not leafs of $T'$ is smaller than the number of vertices which are not leafs of $T$.\\
Thus, we can apply our induction hypothesis to see that $\forall v \in V(  T')$ , $g( v) = f( v) $ and this concludes the proof.

\end{document}
