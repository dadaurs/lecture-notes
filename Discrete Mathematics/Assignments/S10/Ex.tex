\documentclass[11pt, a4paper]{article}
\usepackage[utf8]{inputenc}
\usepackage[T1]{fontenc}
\usepackage[francais]{babel}
\usepackage{lmodern}
\usepackage{amsmath}
\usepackage{amssymb}
\usepackage{amsthm}
\renewcommand{\vec}[1]{\overrightarrow{#1}}
\newcommand{\del}{\partial}
\DeclareMathOperator*{\sgn}{sgn}
\DeclareMathOperator*{\id}{Id}
\DeclareMathOperator*{\im}{Im}
\DeclareMathOperator*{\re}{Re}
\DeclareMathOperator*{\vol}{Vol}
\newcommand\norm[1]{\left\vert#1\right\vert}
\newcommand\ns[1]{\left\vert\left\vert\left\vert#1\right\vert\right\vert\right\vert}
\newcommand\Norm[1]{\left\lVert#1\right\rVert}
\newcommand\N[1]{\left\lVert#1\right\rVert}
\newcommand\abs[1]{\left\vert#1\right\vert}
\newcommand\inj{\hookrightarrow}
\newcommand\surj{\twoheadrightarrow}
\newcommand\ded[1]{\overset{\circ}{#1}}
\newcommand\sidenote[1]{\footnote{#1}}
\newcommand\eng[1]{\left\langle#1\right\rangle}
\newcommand\hr{
    \noindent\rule[0.5ex]{\linewidth}{0.5pt}
}

\newcommand{\incfig}[1]{%
    \def\svgwidth{\columnwidth}
    \import{./figures}{#1.pdf_tex}
}
\newcommand{\filler}[1][10]%
{   \foreach \x in {1,...,#1}
    {   test 
    }
}

\newcommand\contra{\scalebox{1.5}{$\lightning$}}
\makeatother
\def\@lecture{}%
\newcommand{\lecture}[3]{
    \ifthenelse{\isempty{#3}}{%
        \def\@lecture{Lecture #1}%
    }{%
        \def\@lecture{Lecture #1: #3}%
    }%
    \subsection*{\@lecture}
    \marginpar{\small\textsf{\mbox{#2}}}
}

\begin{document}
\title{Exercise 10}
\author{David Wiedemann}
\maketitle
We will write $V $ for the set of vertices of  $G$.\\
Let $e \in E( G) \setminus E( T) $, we can write $e$ as $e= \left\{ a,b \right\} , a,b \in V$.\\
Since $E( T)$ is connected, there exists a path in $T$ of the form $(a, v_1, \ldots, v_n,b), v_i \in G $.\\
We define $ K = T + e - \left\{ v_n,b \right\} $.\\
Clearly, $K$ is still spanning since it still contains $v_n, a $ and $b$ \footnote { $v_n$ is still contained in $K$ because $ \left\{ v_{n-1} , v_n \right\} $ is contained in $K$} .\\
We now show that $K$ is a tree.\\
First, we show that $K$ still is connected.\\
Indeed, consider two vertices $x,y \in V$.\\
Consider the path of $T$ which would connect $x$ to $y$: $x, u_0,\ldots, u_n, y$.\\
If there exists $0 \leq i \leq n$ such that $ \left\{ u_i, u_{i+1}  \right\} =e$, replace $ u_i, u_{i+1} $ in the path with the path connecting $a$ and $b$, we are left with a path contained in $K$.\\
Hence, $K$ is connected.\\
We now show that $K$ contains no cycle.\\
For the sake of contradiction, suppose $K$ contains a cycle of the form $c_0, \ldots, c_k, c_0$.\\
If $\forall 0 \leq j \leq k, \left\{ c_j, c_{j+1}  \right\} \neq e$, then the cycle is contained in $T$ which is a contradiction since $T$ is a tree.\\
Hence, suppose there exists a $j$ such that $ \left\{ c_j, c_{j+1}  \right\} = e$, without loss of generality, suppose that $c_j =a$ and $c_{j+1} =b$	.\\
If that were the case, we could again create a new cycle of the form $ c_0, \ldots, c_{j-1} , a,v_1 , \ldots, v_n, b , c_{j+2} ,\ldots, c_n, c_0 $.\\
This cycle is clearly contained in $T$.\\
Hence, the existence of a cycle in $K$ implies the existence of a cycle in $T$, which is impossible since $T$





%Since $e$ is not in a spanning tree, we know that $a$ and $b$ are not leafs.\\
%Indeed, if they were leafs, they wouldn't be contained in the edges of $T$.\\
%Note that there exists at least one edge, connected to $e$ which is contained in $E( T) $, without loss of generality we can suppose it to be of the form $e'= \left\{ b,c \right\} \in E( T)  $.\\

%We now consider the graph given by $K=T - e' + e $.\\
%We first prove that it is a tree, we will then show that it is spanning.\\
%First, notice that $K$ does not contain any cycles.\\
%For the sake of contradiction, suppose it contains a cycle of the form $ l_0, \ldots, l_n, l_0$, $l_i \in V( T) \forall 0 \leq i \leq  n $.\\
%If $l_i \neq a,b$, then the cycle is contained in $T$, which contradicts the fact that $T$ is a tree.\\
%Hence, suppose that $\exists k$ such that $l_k =a, $




%First, note that we can suppose that $G$ is not a tree, and thus contains a cycle.\\
%Indeed, if $G$ is a tree, it's spanning tree is $G$, and hence $E( G) \setminus E( T) = \emptyset$.\\
%Thus, suppose that $G$ is a graph with at least one cycle.\\
%We will now show that, if $e \in E( G) \setminus E( T) $, then $e$ is contained in a cycle of $G$.\\
%For the sake of contradiction, suppose that $e= \left\{ a,b \right\} ,\quad a,b \in V$  is not contained in a cycle.\\
%Then there exists $c \in V$ such that there only is one path $ v_0, \ldots, v_k$ such that $v_0=a, v_k= c$.\\
%Since the spanning tree contains $c$ as a vertice, we deduce that the spanning tree must contain $a$ as a vertex.\\
%\hr
%Hence, choose an edge $e' \in E( T) $ of the form $ \left\{ a,d \right\} $ and which is contained in the same cycle as $e$, hence $e'$ is not a leaf.\\
%Remove $e'$ and add $e$ from $T$, note that the resulting graph is still a tree since we have not added any cycles and that it still contains all the edges of $G$ since $b$ is not a leaf ( it is contained in a cycle).



	


\end{document}
