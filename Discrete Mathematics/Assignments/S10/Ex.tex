\documentclass[11pt, a4paper]{article}
\usepackage[utf8]{inputenc}
\usepackage[T1]{fontenc}
\usepackage[francais]{babel}
\usepackage{lmodern}
\usepackage{amsmath}
\usepackage{amssymb}
\usepackage{amsthm}
\renewcommand{\vec}[1]{\overrightarrow{#1}}
\newcommand{\del}{\partial}
\DeclareMathOperator*{\sgn}{sgn}
\DeclareMathOperator*{\id}{Id}
\DeclareMathOperator*{\im}{Im}
\DeclareMathOperator*{\re}{Re}
\DeclareMathOperator*{\vol}{Vol}
\newcommand\norm[1]{\left\vert#1\right\vert}
\newcommand\ns[1]{\left\vert\left\vert\left\vert#1\right\vert\right\vert\right\vert}
\newcommand\Norm[1]{\left\lVert#1\right\rVert}
\newcommand\N[1]{\left\lVert#1\right\rVert}
\newcommand\abs[1]{\left\vert#1\right\vert}
\newcommand\inj{\hookrightarrow}
\newcommand\surj{\twoheadrightarrow}
\newcommand\ded[1]{\overset{\circ}{#1}}
\newcommand\sidenote[1]{\footnote{#1}}
\newcommand\eng[1]{\left\langle#1\right\rangle}
\newcommand\hr{
    \noindent\rule[0.5ex]{\linewidth}{0.5pt}
}

\newcommand{\incfig}[1]{%
    \def\svgwidth{\columnwidth}
    \import{./figures}{#1.pdf_tex}
}
\newcommand{\filler}[1][10]%
{   \foreach \x in {1,...,#1}
    {   test 
    }
}

\newcommand\contra{\scalebox{1.5}{$\lightning$}}
\makeatother
\def\@lecture{}%
\newcommand{\lecture}[3]{
    \ifthenelse{\isempty{#3}}{%
        \def\@lecture{Lecture #1}%
    }{%
        \def\@lecture{Lecture #1: #3}%
    }%
    \subsection*{\@lecture}
    \marginpar{\small\textsf{\mbox{#2}}}
}

\begin{document}
\title{Exercise 10}
\author{David Wiedemann}
\maketitle
First, note that we can suppose that $G$ is not a tree, and thus contains a cycle.\\
Indeed, if $G$ is a tree, it's spanning tree is $G$, and hence $E( G) \setminus E( T) = \emptyset$.\\
Thus, suppose that $G$ is a graph with at least one cycle.\\
We will now show that, if $e \in E( G) \setminus E( T) $, then $e$ is contained in a cycle of $G$.\\
For the sake of contradiction, suppose that $e= \left\{ a,b \right\} $ \footnote { $a,b \in V$}  is not contained in a cycle.\\
Then there exists $c \in V$ such that there only is one path $ v_0, \ldots, v_k$ such that $v_0=a, v_k= c$.\\
Since the spanning tree contains $c$ as a vertice, we deduce that the spanning tree must contain $a$ as a vertex.\\
\hr
Hence, choose an edge $e' \in E( T) $ of the form $ \left\{ a,d \right\} $ and which is contained in the same cycle as $e$, hence $e'$ is not a leaf.\\
Remove $e'$ and add $e$ from $T$, note that the resulting graph is still a tree since we have not added any cycles and that it still contains all the edges of $G$ since $b$ is not a leaf ( it is contained in a cycle).



	


\end{document}
