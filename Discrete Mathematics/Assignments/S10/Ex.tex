\documentclass[11pt, a4paper]{article}
\usepackage[utf8]{inputenc}
\usepackage[T1]{fontenc}
\usepackage[francais]{babel}
\usepackage{lmodern}
\usepackage{amsmath}
\usepackage{amssymb}
\usepackage{amsthm}
\renewcommand{\vec}[1]{\overrightarrow{#1}}
\newcommand{\del}{\partial}
\DeclareMathOperator*{\sgn}{sgn}
\DeclareMathOperator*{\id}{Id}
\DeclareMathOperator*{\im}{Im}
\DeclareMathOperator*{\re}{Re}
\DeclareMathOperator*{\vol}{Vol}
\newcommand\norm[1]{\left\vert#1\right\vert}
\newcommand\ns[1]{\left\vert\left\vert\left\vert#1\right\vert\right\vert\right\vert}
\newcommand\Norm[1]{\left\lVert#1\right\rVert}
\newcommand\N[1]{\left\lVert#1\right\rVert}
\newcommand\abs[1]{\left\vert#1\right\vert}
\newcommand\inj{\hookrightarrow}
\newcommand\surj{\twoheadrightarrow}
\newcommand\ded[1]{\overset{\circ}{#1}}
\newcommand\sidenote[1]{\footnote{#1}}
\newcommand\eng[1]{\left\langle#1\right\rangle}
\newcommand\hr{
    \noindent\rule[0.5ex]{\linewidth}{0.5pt}
}

\newcommand{\incfig}[1]{%
    \def\svgwidth{\columnwidth}
    \import{./figures}{#1.pdf_tex}
}
\newcommand{\filler}[1][10]%
{   \foreach \x in {1,...,#1}
    {   test 
    }
}

\newcommand\contra{\scalebox{1.5}{$\lightning$}}
\makeatother
\def\@lecture{}%
\newcommand{\lecture}[3]{
    \ifthenelse{\isempty{#3}}{%
        \def\@lecture{Lecture #1}%
    }{%
        \def\@lecture{Lecture #1: #3}%
    }%
    \subsection*{\@lecture}
    \marginpar{\small\textsf{\mbox{#2}}}
}

\begin{document}
\title{Exercise 10}
\author{David Wiedemann}
\maketitle
As a preliminary result, we will show that, given a graph $G$ and a walk $( u_1, \ldots, u_k) $ of $G$, there exists $I \subset [ n] $, such that $ ( v_1, \ldots, v_j)  $, $ \left\{ v_i, 0< i \leq j \right\} \subset \left\{ u_i, 0 <i \leq k \right\} $ is a path, and such that $v_1= u_1, v_j = u_k$ .\\
Indeed, consider the walk $( u_1, \ldots, u_k) $ of $G$, if $\forall i,j \in [ n] , i\neq j: u_i \neq u_j$, then $u_1, \ldots, u_k$ already is a walk and we are finished.\\
Set $I= [ n] $
Hence, suppose that there exists $i,j \in [ n] , i\neq j$ such that $u_i = u_j$, without loss of generality we can suppose that $i<j$.\\
Then, we can redefine  $I = I\setminus \left\{ i+1, \ldots, j \right\} $.\\
We now obtain a new walk given by $ ( v_1, \ldots, v_j) = \left\{ u_i \right\}_{i=1, i \in I} ^{n}$.\\
We now repeat this algorithm until we are left with a path.\\
Since a walk is always of finite length and the size of $I$ decreases by at least 1 at each step, we are guaranteed that the algorithm finishes.\\
Also notice that $v_1$ and $v_j$ always stay the same at each step of the algorithm, which proves that $v_1= u_1, v_j = u_k$.\\
\hr
We will write $V $ for the set of vertices of  $G$.\\
Let $e \in E( G) \setminus E( T) $, we can write $e$ as $e= \left\{ a,b \right\} , a,b \in V$.\\
Since $E( T)$ is connected, there exists a path in $T$ of the form $(a, v_1, \ldots, v_n,b), v_i \in G $.\\
We define $ K = T + e - \left\{ v_n,b \right\} $.\\
Clearly, $K$ is still spanning since it still contains $v_n, a $ and $b$ \footnote { $v_n$ is still contained in $K$ because $ \left\{ v_{n-1} , v_n \right\} $ is contained in $K$} .\\
We now show that $K$ is a tree.\\

First, we show that $K$ still is connected.\\
Indeed, consider two vertices $x,y \in V$.\\
Consider the path of $T$ which would connect $x$ to $y$: $x, u_0,\ldots, u_n, y$.\\
If $\forall i \in [ n] $, $ \left\{ u_1, u_{i+1}  \right\} \neq \left\{ v_n,b \right\} $, 
If there exists $0 \leq i \leq n$ such that $ \left\{ u_i, u_{i+1}  \right\} = \left\{ v_n,b \right\}  $, replace $ u_i, u_{i+1} $ in the path with the path $v_n, v_{n-1} ,\ldots, a, b$, we are left with a walk contained in $K$.\\
Using the lemma proven, we extract a path from this walk.\\
This walk being contained in $K$, we deduce that $K$ is connected.\\

We now show that $K$ contains no cycle.\\
For the sake of contradiction, suppose $K$ contains a cycle of the form $c_0, \ldots, c_k, c_0$.\\
If $\forall 0 \leq j \leq k, \left\{ c_j, c_{j+1}  \right\} \neq e$, then the cycle is contained in $T$ which is a contradiction since $T$ is a tree.\\
Hence, suppose there exists a $j$ such that $ \left\{ c_j, c_{j+1}  \right\} = e$, without loss of generality, suppose that $c_j =a$ and $c_{j+1} =b$	.\\
If that were the case, we could again create a new walk of the form $ c_0, \ldots, c_{j-1} ,$ $ a,v_1 ,  \ldots, v_n, b , c_{j+2} ,\ldots, c_n, c_0 $, which by definition is contained in $T$ and in $K$.\\
We now extract a path from this walk, clearly this path is a cycle contained in $T$.\\
Hence, the existence of a cycle in $K$ implies the existence of a cycle in $T$, which is impossible since $T$ is a tree.\\
\hr
We deduce that $K$ also is a spanning tree of $T$, and the result follows.





\end{document}
