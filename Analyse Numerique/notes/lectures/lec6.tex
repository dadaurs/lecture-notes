\documentclass[../main.tex]{subfiles}
\begin{document}
\lecture{6}{Thu 07 Apr}{Effets des erreurs d'arrondissement}
\subsection{Erreurs d'arrondissement}
En realite, lorsqu'on interpole en pratique, a chaque etape de calcul, on commet une erreur d'arrondissement.\\
On remarque donc par exemple que, lorsqu'on interpolle sur des noeuds equidistants une fonction telle que $\sin x$ des grandes erreurs aux bords, meme si l'on s'attendrait a obtenir une convergence uniforme.\\
Donc on pose $ \hat{f} ( x_i) = f( x_i) ( 1+ \epsilon) $ avec $\epsilon$ une certaine erreur machine et on veut etudier l'erreur due au "round off".\\
En substituant cette valeur dans les valeurs de $ p_n$ on obtient
\begin{align*}
\hat{p}_n \coloneqq \sum_{i}^{ } \hat{f}( x_i)  l_i( x) 
\end{align*}
Ou les $l_i$ sont les polynomes interpolant.\\
On peut donc calculer la difference entre $p_n$ et $ \hat{p}_n$ 
\begin{align*}
| p_n - \hat{p}_n | \leq   \sum_{i}^{ } |\epsilon f( x_i) l_i( x) | \leq  \epsilon \N { f}_{\infty}   \sum_{i}^{ } |l_i( x) |	
\end{align*}
Ceci motive la definition suivante
\begin{defn}[Lebesgue constant]
	\[ 
	\Lambda_n = \max_{x} \sum_i |l_i( x) |
	\]
	et clairement $\Lambda_n$ va dependre du choix des $x_i$.
\end{defn}
Le calcul ci-dessus montre que
\begin{thm}
	Soit $ n \in \mathbb{N}, f\in C^{0}( [ a,b] ) $ et $p_n$ les polynomes d'interpolations de Lagrange, alors
	\[ 
	\N { p_n - \hat{p}-n} _{ \infty } \leq \epsilon \Lambda_n \N { f} _{ \infty } 	
	\]
	
\end{thm}
Donc pour controler l'erreur, il nous faut controler $ \Lambda_n$, enfait
\begin{thm}[Behaviour of lebesgue constant]
	\begin{itemize}
	\item Si les noeuds sont equidistants, alors
		\[ 
		\Lambda_n \approx \frac{2^{n+1}}{\epsilon n \log n} \text{ quand }  n \to \infty 
		\]
		
	\item Pour les points de chebychev, on a
		\[ 
		\lambda_n \approx \frac{2}{\pi} \log n \text{ quand } n \to \infty 		
		\]
	\end{itemize}
\end{thm}
Mais meme dans le cas des noeuds optimaux, on voit que l'erreur va tout de meme tendre vers l'infini.\\
On essaie donc d'approximer les fonctions par des fonctions lineaires.
\subsection{Interpolation par polynomes par parties}
\begin{defn}
	Pour un $N \in \mathbb{N}$ fixe et $s \in \mathbb{N}$. On considere $ f\in C^{0}$ et une partition $a_i$ d'un intervalle $ [ a,b] $. Pour chaque $ i$, on construit $p^{( i )}$  le polynome d'interpolation de lagrange locale pour $ s$ points choisis dans $ [ a_i, a_{i+1} ) $ .\\
	On recolle  alors les $ p^{( i) }$ en une fonction $ \tilde{p}_s$ 
\end{defn}
Et on a un theoreme qui nous borne l'erreur:
\begin{thm}
	Soit $ N \in \mathbb{N}, N \geq 1$ et $s\in \mathbb{N}, f \in C^{s+1}( [ a,b] ) $ et $ \tilde{p}_s$ le polynome d'interpolation par parties sur une partition generale, alors
	\[ 
	\N { f - \tilde { p} _s} _{ \infty } \leq  \frac{H^{s+1}}{4 ( s+1) !}\N { f^{( s+1) }} _{ \infty } 
	\]
	ou $ H \coloneqq  \max |a_{i+1} -a_i|$ 
\end{thm}
\begin{proof}
On a
\[ 
\N { f- \tilde { p} _s} _{ \infty } = \max_i \N { f- p_s^{ ( i) }} _{ \infty , [ a_i, a_{i+1} ) } \leq  \frac{1}{4 ( s+1) !} H^{s+1}\N { f^{( s+1) }} _{ \infty } 
\]

\end{proof}
\subsection{Approximation dans la norme $ L^{2}$ }
Etant donne $f$ , on veut trouver le meilleur polynome $ p^{\ast}$ qui minimisera la distance dans la norme $ L^{2}$, ie.
\[ 
	p^{\ast} = \mathrm{argmin}_{q_n \in \mathbb{P}_n} \N { f- q_n} _2^{2}
\]
\begin{thm}
	La solution optimale est donnee par
	\[ 
	p^{\ast} = \sum_k \alpha_k p_k \text{ avec } \alpha_k = \frac{ \int_{ a }^{ b }f p_k dt}{\int_{ a }^{ b }|p_k|^{2}dt}
	\]
		
\end{thm}






\end{document}	
