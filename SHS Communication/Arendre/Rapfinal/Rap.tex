\documentclass[11pt, a4paper]{article}
\usepackage[utf8]{inputenc}
\usepackage[T1]{fontenc}
\usepackage[francais]{babel}
\usepackage{lmodern}
\usepackage{amsmath}
\usepackage{amssymb}
\usepackage{amsthm}
\renewcommand{\vec}[1]{\overrightarrow{#1}}
\newcommand{\del}{\partial}
\DeclareMathOperator*{\sgn}{sgn}
\DeclareMathOperator*{\id}{Id}
\DeclareMathOperator*{\im}{Im}
\DeclareMathOperator*{\re}{Re}
\DeclareMathOperator*{\vol}{Vol}
\newcommand\norm[1]{\left\vert#1\right\vert}
\newcommand\ns[1]{\left\vert\left\vert\left\vert#1\right\vert\right\vert\right\vert}
\newcommand\Norm[1]{\left\lVert#1\right\rVert}
\newcommand\N[1]{\left\lVert#1\right\rVert}
\newcommand\abs[1]{\left\vert#1\right\vert}
\newcommand\inj{\hookrightarrow}
\newcommand\surj{\twoheadrightarrow}
\newcommand\ded[1]{\overset{\circ}{#1}}
\newcommand\sidenote[1]{\footnote{#1}}
\newcommand\eng[1]{\left\langle#1\right\rangle}
\newcommand\hr{
    \noindent\rule[0.5ex]{\linewidth}{0.5pt}
}

\newcommand{\incfig}[1]{%
    \def\svgwidth{\columnwidth}
    \import{./figures}{#1.pdf_tex}
}
\newcommand{\filler}[1][10]%
{   \foreach \x in {1,...,#1}
    {   test 
    }
}

\newcommand\contra{\scalebox{1.5}{$\lightning$}}
\makeatother
\def\@lecture{}%
\newcommand{\lecture}[3]{
    \ifthenelse{\isempty{#3}}{%
        \def\@lecture{Lecture #1}%
    }{%
        \def\@lecture{Lecture #1: #3}%
    }%
    \subsection*{\@lecture}
    \marginpar{\small\textsf{\mbox{#2}}}
}

\begin{document}
\title{Rapport Final}
\author{David Wiedemann\\
Section de Mathématiques\\
SCIPER 324259}
\maketitle
Dans le cadre de notre projet, nous avons décidé d'aborder la problématique de la 5G dans les pays en voie de développement.\\
Nous comptons décrire les problèmes qui sont inévitablements liés à l'arrivée de cette nouvelle technologie.
En effet, les pays en voie de développements sont les pays qui peuvent le plus profiter d'une connectivité plus élevée, mais il est particulièrement difficile d'implémenter ces nouvelles technologies dans ces endroits.
En effet, la 5G a le potentiel de permettre aux enfants d'accéder à une éducation qualitative même s'ils viennent de milieux non-privilégiés, ou encore permettre des interventions médicales à distance.\\

Venons-en donc aux liens entre ce projet et le contenu du cours.
Tout d'abord,  le projet met en avant un lien important entre la communication et l'humanitaire, les réseaux 5G permettraient d'accélérer drastiquement les aides humanitaires en cas de catastrophes naturelles, de conflits armés ou autres désastres.\\

En deuxième lieu, notons que, en facilitant l'accès aux technologies de la communication dans ces pays, on garantit une liberté d'expression à la population qui n'est pas directement accessible au gouvernement.
L'importance de cette liberté d'expression est d'une grande importance, en guise d'exemple récent, on peut prendre le coup d'état récent au Myanmar.
L'accès  de la population à internet a permis une réponse internationale rapide, et a également permis de documenter précisément les évènements.
De plus l'accès facilité à internet d'une plus grande partie de la population garantira, à long terme, une plus grande inclusion des cultures sur internet.\\

Finalement, comme décrit dans le premier paragraphe, la 5G rentre bien sur dans le cadre de l'enseignement digital.
La 5G pourrait permettre de détruire le fossé digital qui existe aujourd'hui entre différentes communauté. 
En effet, beaucoup de régions rurales n'ont pas un accès stable à internet, une infrastructure 5G moderne permettrait de réduire ces inégalités digitales.





\end{document}
