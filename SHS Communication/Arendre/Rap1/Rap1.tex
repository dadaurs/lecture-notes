\documentclass[11pt, a4paper]{article}
\usepackage[utf8]{inputenc}
\usepackage[T1]{fontenc}
\usepackage[francais]{babel}
\usepackage{lmodern}

\usepackage{amsmath}
\usepackage{amssymb}
\usepackage{amsthm}

%%%%%%%%%%%%%%%%%%%%%%%%%%%%%%%%%%%%%%%%%%%

\title{L'influence de l'exploration spatiale sur nos moyens de communication}
\author{David Wiedemann,\\
	Section Mathématiques,\\ SCIPER 324259\\
269 Mots
}
\date{}

\begin{document}

\maketitle
L'exploration de l'espace n'a pas pour unique but d'assouvir la curiosité de l'espèce humaine, elle entraine également un grand nombre de bienfaits indirectement lié à l'exploration spatiale.\\
En effet, les développements technologiques liés à l'exploration spatiale trouvent souvent des utilités dans des domaines complètements différents.\\
Peut-être l'exemple le plus direct de ceci est l'utilisation des satellites de communication.
Ces satellites permettent, entre autre, la diffusion de signaux télévision et radio.
De nos jours, l'existence de ces satellites est devenue indispensable, notamment pour assurer l'accès à internet dans des régions isolées du monde.\\

Mais l'exploration spatiale possède des bénéfices moins direct sur nos moyens de communication.
Les progrès qui ont été faits pour permettre l'atterrissage sur la lune ou, dans un cadre plus récent, l'exploration de mars, ont permis la miniaturisation de circuits électroniques.
Sans cette miniaturisation, les technologies de communication seraient impensables. 
En effet, l'existence de services tels que ceux fourni par Google ( notamment le cloud, les engins de recherche, etc) nécessitent ces technologies.\\

On a donc vu que l'exploration spatiale possède une influence sur le développement technologique, mais qu'en est-il du côté humain?\\
Comme souligné par Claude Nicollier, l'exploration spatiale est une quête internationale, regroupant plusieurs grandes puissances politiques telle que les Etats-Unis, la Russie et la Chine.\\
En effet, la conquête spatiale est une activité et un but regroupant les humains plutôt qu'une nation, et dans ce sens, elle facilite la communication et le travail international.\\ 
Selon moi, ce phénomène possède des bénéfices dépassant de loin un cadre technique, il permet une approche aux enjeux mondiaux radicalement différente.\\



\end{document}
