\documentclass[../main.tex]{subfiles}
\begin{document}
\lecture{5}{Sat 02 Oct}{Adjonctions}
\subsection{Equivalence de categories}
\begin{defn}[Equivalence de categories]
	Un foncteur $F:C \to D$ est une Equivalence de categories s'il existe un foncteur $F': C\to D$ tel que
	\[ 
	\sigma: \id_C \overset{ \simeq }{\to} F'\circ F \text{ et } \tau: \id_D \overset{ \simeq}{\to} F\circ F'
	\] 
	
\end{defn}
\begin{rmq}
Si $F$ est un isomorphisme de categories, c'est aussi une equivalence de categories.
\end{rmq}
\begin{exemple}
Soit $\un$ la categorie avec un seul objet $\ast$ et un seul morphisme $\id$. Soit $C$ la categorie $\ob C= \left\{ a,b \right\} $ et deux morphismes non-identite $f:a\to b$ et $g:b\to a$ qui sont des isomorphismes. Alors les categories $\un$ et $C$ sont equivalentes.\\
On definit $F :\un \to C$  par $F( \ast) = a, F( \id) = \id_a$.\\
On definit $F': C \to \un$ par $F'( a) = F'( b) = \ast$.\\
On a que $F'\circ F = \id_{\un} $ donc la transformation naturelle $\sigma= \id_{\id_{ \un }} $ est triviale.\\
Dans l'autre sens, $F\circ F'\neq \id_{C} $ , cependant $\exists \tau:\id_C\to F\circ F'$ defini par
\[ 
\tau: \ob C\to \mor C
\]
donne par
\begin{align*}
	\tau( a) = \id_a, \tau( b) =g
\end{align*}
Verifions la naturalite:\\
Commencons par $f:a\to b$, on a
\[ 
\id_a \circ \id_a = g \circ f 
\]
ce qui est vrai par definition de $C$.\\
De meme
\[ 
	\id_a\circ g = \id_a \circ g
\]

\end{exemple}
\subsection{Adjonctions}
On veut generaliser la notion d'equivalence de categories, dont il y a beaucoup d'exemples interessants ( surtout en theorie des groupes) 
\begin{defn}[Adjonctions]
	Un couple de foncteurs $L:C\to D$ et $R:D\to C$ forme une adjonction s'il existe des transformations naturelles
\[ 
\eta: \id_C\to R\circ L \text{ et } \epsilon:L\circ R \to \id_D
\]
tel que les diagrammes suivants commutent
\[ 
	\begin{tikzcd}
L(c) \arrow[rr, "L(\eta_c)"] \arrow[rrd, "Id_{L(c)}"] &  & L\circ R\circ L(c) \arrow[d, "\epsilon_{L(c)}"] \\
                                                      &  & L(c)                                           
\end{tikzcd}
\]
\[
\begin{tikzcd}
R(d) \arrow[rr, "\eta_{R(d)}"] \arrow[rrd, "\id_{R(d)}"] &  & R\circ L\circ R(d) \arrow[d, "R(\epsilon_d"] \\
                                                        &  & R(d)                                         
\end{tikzcd}
\]
pour tout $c\in \ob C, d \in \ob D$ .\\
\end{defn}
Analysons ces identites triangulaires.\\
La premiere identite veut dire $\forall c \in \ob C, \eta_c : c \to RL( c) $, on peut lui appliquer $L$ et on trouve
\[ 
	L( c) \overset { L( \eta_c) } { \longmapsto} LRL( c) 
\]
On peut maintenant considerer $\epsilon_{L( c) } : LRL( c) \to L( c) $ pour revenir a $L( c) $ 
\[ 
	L( c) \overset { L( \eta_c) } { \longmapsto} LRL( c) \overset { \epsilon_{L( c) }  } { \longmapsto} L( c) 
\]
et on veut que cette suite de composition soit egale a $\id_{L( c) } $ .\\
Pour la deuxieme identite, soit $d \in \ob D$, on a alors
\[ 
	R( d) \overset { \eta_{R( d) }  } { \longmapsto} RLR( d) \overset { R( \epsilon_d)   } { \longmapsto} R( d) 
\]
Si $L : C \leftrightarrow D : R$ est une adjonction avec transformations naturelles associees $\eta: \id_C \to RL$ et $\epsilon : LR \to \id_D$, alors on dit que $L$ est un adjoint a gauche de $R$ et $R$ est un adjoint a droite de $L$.\\
On notera alors $L \dashv R$.\\
\subsection{Caracterisation des Adjonctions}
\subsubsection{Preparation}
Soit $L:C \leftrightarrow D:R$ un couple de foncteurs entre deux categories petites. On peut y associer deux autres foncteurs interessants
\begin{itemize}
	\item $D( L( -) ,-) : C^{op}\times D \to \ens$ 
	\item $C( -, R( -) ): C^{op}\times D  \to \ens$ 
\end{itemize}
qui sont definis comme suit
\begin{itemize}
\item Sur les objets, 
	\[ 
		\forall ( c,d)  \in \ob C^{op}\times \ob d \quad D( L( -) ,-) ( c,d) = D( L( c) , d) 
	\]

\item Sur les morphismes
	Soient $( f^{op}, g) \in \mor ( C^{op}( c,c') \times D( d,d') ) $.\\
	Donc $\exists f \in C( c',c)  $, on veut definir une application ensembliste
	\begin{align*}
	D( L( f^{op}) ,g) : D( L( c) ,d) \to D( L( c') ,d') 	
	\end{align*}

	
	
\end{itemize}
On peut resumer ceci dans le diagramme
\[ 
	L( c') \overset { L( f) } { \longmapsto} L( c) \overset { h} { \longmapsto} d \overset { g} { \longmapsto} d'
\]
Ainsi, $D( L( f^{op},g) ) \coloneqq g\circ h \circ L( f) : L( c') \to d'$ .\\
\textbf{ Est-ce que ce choix definit bien un foncteur?} \\
\begin{itemize}
	\item Identites: Pour $h:L( f) \to d\in C( L( f) ,d) $ Si $( \id_c^{op},\id_d) \in \mor ( C^{op}\times D) $ alors $D( L( \id_c^{op}) ,\id_d) ( h ) = \id_d\circ h\circ \id_{L_c}=h $.\\
		Donc
		\[ 
			D( L( \id_c^{op},\id_d) ) =\id_{D( L( c) ,d) } 
		\]
	
	\item Considerons 
		\[ 
			( c,d) \overset { ( f^{op},g) } { \longrightarrow} ( c',d') \overset { ( f'^{op},g') } { \longrightarrow} ( c",d") 
		\]
		et etudions
		\[ 
			D( L( c ),d) \overset {   D(L( f^{op}),g) } { \longrightarrow} D( L( c' ),d') \overset { D( L( f'^{op} ),g') } { \longrightarrow} D(L(  c" ),d") 
		\]
		On a donc, pour $h \in D( L( c) ,d) $ 
		\begin{align*}
			D( L( f^{op}) ,g)\circ D( L( f'^{op},g') ) ( h) = g'\circ g\circ h \circ L( f)\circ L( f')   \circ g = D( L( f'^{op}\circ f^{op}), g'\circ g) ( h) 
		\end{align*}

		
		
\end{itemize}

De maniere semblable, $\exists$ foncteur
\[ 
	C( -, R( -) ) : C^{op}\times D \to \ens
\]
defini sur les objets par
\[ 
	\forall ( c,d) \in \ob ( C^{op}\times D) \quad C( -, R( -) ) ( c,d) = C( c,R( d) ) 
\]
et $\forall ( f^{op},g) :( c,d) \to ( c',d')  $ , alors
\begin{align*}
	C( f^{op},R( g) ) : C( c,R( d) ) &\to C( c', R( d') ) \\
	( h: c\to R( d) ) &\to ( R( g) \circ h \circ f ) 
\end{align*}



			



			
\end{document}	
