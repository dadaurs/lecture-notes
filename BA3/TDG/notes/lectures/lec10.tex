\documentclass[../main.tex]{subfiles}
\begin{document}
\lecture{10}{Sun 31 Oct}{Introduction aux actions de groupe}
\section{Actions de groupe}
\begin{defn}[Action de groupe]
	Une action de groupe $G$ sur l'ensemble $X$ consiste en un homomorphisme de groupe
	\[ 
	\phi: G \to \aut X
	\]
	
\end{defn}
De maniere equivalente, une action de $G$ sur $X$ consiste en une application
\[ 
\phi^{\flat}: G \times X \to X
\]
tel que
\[ 
\phi^{\flat}( a\cdot b, x) = \phi^{\flat}( a, \phi^{\flat}( b,x) ) \text{ et } \phi^{\flat}( e,x) = x\forall a,b\in G, x \in X
\]
\begin{defn}[Points fixes]
	L'ensemble des points fixes par $a$ , note $X^{a}$ est le sous-ensemble de $X$ defini par
	\[ 
	X^{a}= \left\{ x\in X| ax=x \right\} 
	\]
	
	L'ensemble des points fixes de l'action $\phi$, note $( X,\phi)^{G}$ ou simplement $X^{G}$ est le sous-ensemble de $X$ defini par
	\[ 
	X^{G}= \left\{ x\in X| ax= x \forall a \in G \right\} = \bigcap_{a\in G} X^{a}
	\]
	
\end{defn}	
\begin{defn}[Orbite]
	Soit $\phi: G\to \aut X$ une action de groupe. Soit $x\in X$ 
	\begin{enumerate}
	\item L'orbite de $x$, note $O_x$ est le sous-ensemble de $X$ defini par
		\[ 
		O_x= \left\{ ax| a \in G \right\} 
		\]
		
	
	\item L'ensemble des orbites de l'action est 
		\[ 
			( X,\phi) _G= \left\{ O_x| x \in X \right\} 
		\]
		que l'on note souvent simplement $X_G$. 

	\item Le stabilisateur de $x$, note $G_x$ est le sous-groupe de $G$ defini par
		\[ 
		G_x= \left\{ a\in G| ax=x \right\} 
		\]
			
	\end{enumerate}
\end{defn}
\begin{itemize}
\item Soient $x,x'\in X$, si $O_x\cap O_{x'} \neq \emptyset$, alors $ O_x= O_{x'}$.\\
	Donc on peut decomposer $X$ en une reunion disjointe d'orbites, $\exists \overline{X}\subset X$ tel que
	\[ 
	X= \bigcup_{x\in \overline{X}} O_x
	\]
		
\item $\forall x \in X\exists$ bijection
	\[ 
	\faktor{G}{G_x}\to O_x: \overline{a}\mapsto a\cdot x
	\]
		

\item Equation de classe
	Soit $\phi: G \to \aut X$ une action de groupe. La bijection du point precedent donne une equation de clsse
	\[ 
	\# X < \infty \Rightarrow \#= \sum_{x\in \overline{X}}^{ } ( G: G_x) 
	\]
	
\item Soit $\phi:G\to \aut X $ une action de groupe, le lemme de Burnside exprime la relation entre orbites et ensembles de points fixes
	\[ 
	X = \bigcup_{x\in \overline{X}} O_x \Rightarrow \# \overline{X} = \frac{1}{\# G} \sum_{a\in G}^{ } \# X ^{a}		
	\]
	
\end{itemize}


\end{document}	
