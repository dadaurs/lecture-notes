\documentclass[../main.tex]{subfiles}
\begin{document}
\lecture{16}{Fri 10 Dec}{Classification des groupes Abeliens Finis}
\subsection{Classification des groupes abeliens finis}
\begin{thm}[Classifiaction]
	Si $A$ est un groupe abelien fini d'ordre $n$, alors pour tout nombre premier $p$ qui divise $n$, il existe une unique suite d'entiers
	\[ 
	r_{p,1} \geq r_{p,2} \ldots \geq r_{p,m_p} 
	\]
	tels que
	\[ 
	A= \prod_{p\in P_n} \faktor{\mathbb{Z}}{p^{r_{p,1} \mathbb{Z}} } \times \ldots \times faktor{\mathbb{Z}}{p^{r_{p,m_p} \mathbb{Z}} }
	\]
ou $P_n = \left\{ p \text{ premier } \vert p|n \right\} $ 	
	
\end{thm}
Il nous faut deux lemmes techniques necessaires pour la demonstration du theoreme.
\begin{lemma}
Si $A$ est un groupe abelien tel que $A= T_n( A) $ et $n=lm$  ou $( l,m) =1$, alors
\[ 
A = T_l( A) \oplus T_m( A) 
\]

\end{lemma}
\begin{proof}
$( l,m) =1$ implique $\exists r,s \in \mathbb{Z} \text{ tel que } rl + sm =1$.\\
Ainsi
\[ 
\forall a \in A, a = rla + sma \in T_m( A) + T_l( A) 
\]
Par ailleurs, si $a\in T_m( A) \cap T_l( A) $, alors
\[ 
la = 0 = ma
\]
d'ou $rla+ sma = 0$, donc $a=0$.\\
Donc $T_m( A) \cap T_l( A) = \left\{ 0 \right\} $, donc on a bien une somme directe.
\end{proof}
Une consequence du lemme ci-dessous est
\begin{lemma}
Si $A$ est un groupe abelien, tel que $A= T_n( A) $, alors $A= \bigoplus_{p \in P_n} A( p) $ 
\end{lemma}
\begin{proof}
On ecrit $n = p_1^{l_1}\ldots p_k^{k}$, ou $P_n = \left\{ p_1, \ldots, p_k \right\} $.\\
Par recurrence sur $k$, on conclut.
\end{proof}
\begin{proof}[du theoreme de classification]
	Notons que, si $A$ est abelien fini, et $|A| = n$ alors
	\[ 
	A= T_n( A) 
	\]
	Donc par le lemme precedent 
	\[ 
	A= \bigoplus_{p\in P_n} A( p) 
	\]
	On sait aussi que chaque $A( p) $ est un $p$ groupe abelien, puisque $|A( p) |< \infty $.\\
	Ainsi, il suffit de demontrer que si $A$ est un $p$ groupe abelien, il existe une suite unique $r_1 \geq \ldots \geq r_m$ tel que
	\[ 
	A \simeq \faktor{\mathbb{Z}}{p^{r_1}\mathbb{Z}}\times \ldots \times \faktor{\mathbb{Z}}{p^{r_m}\mathbb{Z}}
	\]
\subsubsection*{Existence}
Par recurrence sur $|A|$.\\
Si $|A| = 1$, on a fini.\\
Si $|A| = p $, alors $A = \faktor{\mathbb{Z}}{p \mathbb{Z}}$ \\
Supposons qu'une telle decomposition en produit de groupes cycliques existe pour tout $p$-groupe abelien $A$ tel que $|A| = p^{k}, \forall k < n$.\\
Soit $A$ un $p$-groupe abelien tel que $|A| = p^{n}$.\\
Soit maintenant $a\in A$ d'ordre maximal, $o( a_1) = p^{r_1}$, alors
\[ 
| \faktor{A}{\langle a_{1}\rangle}| = p^{n-r_1}< p^{n}
\]
Par l'hypothese de recurrence, il existe une suite $r_2 \geq \ldots \geq r_m$ tel que $\exists $ iso 
\[ 
\phi: \faktor{A}{\langle a_1\rangle}\to \faktor{\mathbb{Z}}{p^{r_2}\mathbb{Z}}\times\ldots
\]
Posons $ \overline{a_i}= \phi^{-1}( ( 0, \ldots, 1, \ldots) ) $ avec un $1$ a la $i-1$ eme place.\\
Alors $o( \overline{a_i}) = p^{r_i}\forall i$ et
\[ 
\faktor{A}{\langle a_1\rangle}\simeq \bigoplus_{i=2}^{m} \langle \overline{a_i}\rangle
\]
Par le lemme 4.5, il existe un representant de $ \overline{a_i}$ dont l'ordre est $p^{r_i}$.\\
Sans perte de géneralité, $o( a_i) = p^{r_i}$.\\
On affirme que 
\[ 
A = \bigoplus_{i=1}^{m} \langle a_i \rangle \simeq \prod_{i=1}^{m} \faktor{\mathbb{Z}}{p^{r_i}\mathbb{Z}}
\]
Considerons l'homomorphisme quotient $q: A \to \faktor{A}{\langle a_1\rangle}$.\\
$\forall b \in q$, il existe $s_2, \ldots, s_m$ tel que $ \overline{b}= q( b) = s_2 \overline{a_2}+ \ldots + s_m \overline{a_m}$ 	\\
Par consequent 
\[ 
\overline{b}= \overline{s_2a_2+ \ldots+ s_m a_m}
\]
Donc il existe $s_1\in \mathbb{Z}$ tel que
\[ 
b - s_2a_2- \ldots - s_m a_m = s_1a_2
\]
ie, $b = \sum_i s_i a_i$

Finalement on observe que $\langle a_i \rangle \cap \langle a_j \rangle = \left\{ 0 \right\} $ 
\subsubsection*{Unicite}
Recurrence sur $|A|$.\\
Si $|A| = 1, p$, on a un seul choix.\\
Ensuite, si
\[ 
	\faktor{ \mathbb{Z}}{p^{r_1}\mathbb{Z}}\times \ldots \times \frac{\mathbb{Z}}{p^{r_m}} \simeq \faktor{ \mathbb{Z}}{p^{s_1}\mathbb{Z}}\times \ldots \times \frac{\mathbb{Z}}{p^{s_m}}
\]
avec $r_1$ et $s_1$ maximal.\\
Alors $r_1= s_1$ car a gauche l'ordre maximal est $p^{r_1}$ et a droite, l'ordre maximal est $p^{s_1}$.\\
Et on a fini par recurrence.

\end{proof}
	
	


\end{document}	
