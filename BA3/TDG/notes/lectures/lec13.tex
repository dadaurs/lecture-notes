\documentclass[../main.tex]{subfiles}
\begin{document}
\lecture{13}{Fri 19 Nov}{Groupes Abeliens}
\section{Groupes Abeliens}
\subsection{Constructions categoriques dans $\ab$ }
Comment distinguer les groupes abeliens parmis tous les groupes?
\begin{lemma}
Un groupe $G$ est abelien si et seulement si sa multiplication $\mu: G\times G\to G$ est un homomorphisme
\end{lemma}
\begin{proof}
$\mu: G\times G\to G: ( a,b) \mapsto ab$.\\
Alors $\mu$ un homomorphisme de groupe $\iff \forall a,b,a',b'\in G$
\[ 
\mu( ( a,b) \times ( a',b') ) = \mu( a,b) \times \mu( a',b') 
\]
Donc
\begin{align*}
\mu( ( aa',bb') ) = aba'b' = aa'bb' = aba'b' \forall a,b,a',b'\in G
\intertext{ Ce qui est equivalent a demandenr que}
a'b= ba'\forall a',b\in G	
\end{align*}
$G$ est abelien.

\end{proof}
\subsection{Sommes directes}
Comment construire des coproduits dans $\ab$?
\begin{lemma}
Pour tout couple de groupes abeliens $A$ et $B$,
\[ 
A \oar i_1 A\times B \oal i_2 B
\]
ou $i_1( a) = ( a,0) $ et $i_2 ( b) = ( 0,b) $ pour tout $a \in A$ et tout $b \in B$ est un coproduit dans $\ab$.

\end{lemma}
\begin{proof}
Il suffit de verifier la propriete universelle.\\
Soient $f\in \ab( A,C) , g \in \ab( B,C) $, Comment definir $h \in \ab( A\times B, C) $ tel que $hi_1=f, hi_2=g$.\\
Observer que $hi_1 = f \iff h( a,0) = f( a) \forall a \in A$ et $h i_2 = g \iff h( 0,b) = g( b) \forall b \in B$.\\
Cependant,
\[ 
h( a,b) = h( a,0) + h( 0,b) = f( a) + g( b)
\]
Definissons donc $ h( a,b) = f( a) + g( b) $ 

\end{proof}
Pour distinguer les deux roles categoriques de $A\times B$ au lieu du produit, on le note $A\oplus B$ et on l'appelle la somme directe et on ecrira les elements comme une somme formelle $a+ b \in A\oplus B$.\\
Quelle est la relation entre la notion de $\oplus$ ci-dessus et celle de $\oplus$ de 2 sous-groupes d'un groupe abelien?\\
Parfois on considere la somme directe comme une operation interne a l'ensemble des sous-groupes d'un groupe abelien. Si $B$ et $C$ sont des sous-groupes de $A$, leur somme, notee $B+C$ est un sous-groupe de $A$.
\hr\\
Traduisons la propriete universelle de la somme directe de groupes abeliens
\begin{rmq}
Selon la propriete universelle du coproduit
\[ 
\forall f \in \ab( A,C), \in \ab( B,C), \exists! h \in \ab( A\oplus B, C) 
\]
tel que
\[ 
\ab( A\oplus B, C) \to \ab( A, C) \times \ab( B,C) :h \mapsto ( h\circ i_1, h\circ i_2) 
\]
est une bijection.\\
Ainsi l'existence de $\alpha$ ne depend que de l'existence de $i_1$ et $i_2$.\\
De plus, la propriete universelle de $A\oplus B$ nous donne un inverse a $\alpha$ 
\[ 
\beta( f,g) = f+g
\]

\end{rmq}
\begin{defn}[Produit quelconque d'ensembles]
	Soit $X$ un ensemble et soit $ \left\{ Y_x |x\in X \right\} \subset \ob \ens $ Le produit des $Y_x$ note $\prod_X Y_x$ est l'ensemble
	\[ 
	\left\{ \omega \in \ens( X, \bigcup Y_x) | \omega( x) \in Y_x\forall x \in X \right\} 
	\]
	
\end{defn}
Ceci se generalise naturellement aux groupes
\begin{defn}[Produit quelconque de groupes]
	Soit $\left\{ G_x| x \in X \right\} \subset \ob \gr$. Le produit des $G_x$, note $\prod G_x$ est le groupe dont le sous-ensemble sous-jacent est
	\[ 
	\left\{ \omega\in \ens( X, \bigcup G_x) | \omega( x) \in G_x\forall x \in X \right\} 
	\]
	muni de la mulitplication definie par

	\[ 
( \omega\cdot \omega' )	( x) = \omega( x) \cdot \omega'( x) 
	\]
\end{defn}
Notons que ici, les projections selon les coordonnees sont en particulier des homomorphismes.
\begin{rmq}
Le produit d'une famille quelconque d'ensembles ou de groupes verifie une propriete universelle qui generalise celle de la definition du produit de deux objets, notamment
Dans $\gr: \forall \left\{ f_{x'} \in \gr( H, G_{x'} ) | x'\in X \right\} $, 
\begin{align*}
\exists ! f \in \gr( H, \prod G_x ) 
\end{align*}
tel que $p_{x'} \circ f  = f_{x'} \forall x'$ 

\end{rmq}
Generalisons la notion de somme directe aux familles quelconques de groupes abeliens.
\begin{defn}[Somme directe quelconque]
	Soit $X$ un ensemble, et soit $ \left\{ A_x|x\in X \right\} \subset \ob\ab$. Une somme directe des groupes abeliens $A_x$ consiste en un groupe abelien $B$ muni d'homomorphismes $i_x: A_x\to B$ pour tout $x\in X$ tel que l'application
	\[ 
\ab( B,C) \to \prod_{x\in X} \ab( A_x, C) : h \to ( h\circ i_x)_{x\in X} 	
	\]
	
\end{defn}
\begin{propo}
Soit $X$ un ensemble, et soit $\left\{ A_x| x \in X \right\} \subset \ob \ab$. La somme directe des $A_x$ existe et est unique a isomorphisme pres.\\
\end{propo}
\begin{proof}
Posons $B = \left\{ \omega\in \prod_{x\in X} A_x | \# \left\{ x\in X| \omega( x) \neq 0 \right\} < \infty  \right\} $ et $\iota_x: A_{x'} \to B: a \mapsto \iota_{x'} (a ) $ ou 
\[ 
\iota_{x'} ( a) : X \to \bigcup_{x\in X} A_x: x \mapsto 
\begin{cases}
a: x= x'\\
0: x\neq x'
\end{cases}
\]
On veut definir un inverse a $\alpha: \ab( B,C) \to \prod_{x\in X} \ab( A_x,C) $.\\
On definit $\beta: \prod_{x\in X} \ab( A_x,C) \to \ab( B,C) $ par
\[ 
\beta( ( f_x) _{x\in X} ) = f 
\]
ou $f: B\to C: \omega\mapsto \sum_{x\in X}^{ }f_X( \omega( x) ) $ 
$f$ est un homomorphisme, car:
\[ 
f( \omega+\omega') = \sum_{x\in X}^{ }f_x ( ( \omega+\omega') ( x) ) = \sum_{x\in X}^{ }f_x( \omega( x) ) + \sum_{x\in X}^{ }f_x( \omega'( x) ) = f( \omega) + f( \omega') 
\]
Il reste a voir que c'est bien un inverse
\[ 
\prod_{x\in X} \ab( A_x,C) \oar \beta \ab( B,C) 
\]
\[ 
	( f_x)_{x\in X} \mapsto f \mapsto ( f\circ \iota_x)_{x\in X} 
\]
Donc 
\[ 
\forall a \in A_{x'} : f\circ \iota_{x'} ( a) = \sum_{x\in X}^{ }f_x( \iota_{x'} ( a) ( x) ) = f_{x' } ( a)  
\]
On veut montrer que $\beta\alpha= \id$ 
\[ 
\ab( B,C) \oar \prod_{x\in X} \ab( A_x,C) \oar \ab( B,C)
\]
On a donc
\begin{align*}
	\beta( ( f\circ\iota_x) _{x\in X} )( \omega) &= \sum_{x\in X}^{ }f\circ \iota_x( \omega( x) ) \\
\end{align*}


		
\end{proof}
							

\end{document}	
