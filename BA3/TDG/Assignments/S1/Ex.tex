\documentclass[11pt, a4paper]{article}
\usepackage[utf8]{inputenc}
\usepackage[T1]{fontenc}
\usepackage[francais]{babel}
\usepackage{lmodern}
\usepackage{amsmath}
\usepackage{amssymb}
\usepackage{amsthm}
\renewcommand{\vec}[1]{\overrightarrow{#1}}
\newcommand{\del}{\partial}
\DeclareMathOperator*{\sgn}{sgn}
\DeclareMathOperator*{\id}{Id}
\DeclareMathOperator*{\im}{Im}
\DeclareMathOperator*{\re}{Re}
\DeclareMathOperator*{\vol}{Vol}
\newcommand\norm[1]{\left\vert#1\right\vert}
\newcommand\ns[1]{\left\vert\left\vert\left\vert#1\right\vert\right\vert\right\vert}
\newcommand\Norm[1]{\left\lVert#1\right\rVert}
\newcommand\N[1]{\left\lVert#1\right\rVert}
\newcommand\abs[1]{\left\vert#1\right\vert}
\newcommand\inj{\hookrightarrow}
\newcommand\surj{\twoheadrightarrow}
\newcommand\ded[1]{\overset{\circ}{#1}}
\newcommand\sidenote[1]{\footnote{#1}}
\newcommand\eng[1]{\left\langle#1\right\rangle}
\newcommand\hr{
    \noindent\rule[0.5ex]{\linewidth}{0.5pt}
}

\newcommand{\incfig}[1]{%
    \def\svgwidth{\columnwidth}
    \import{./figures}{#1.pdf_tex}
}
\newcommand{\filler}[1][10]%
{   \foreach \x in {1,...,#1}
    {   test 
    }
}

\newcommand\contra{\scalebox{1.5}{$\lightning$}}
\makeatother
\def\@lecture{}%
\newcommand{\lecture}[3]{
    \ifthenelse{\isempty{#3}}{%
        \def\@lecture{Lecture #1}%
    }{%
        \def\@lecture{Lecture #1: #3}%
    }%
    \subsection*{\@lecture}
    \marginpar{\small\textsf{\mbox{#2}}}
}

\begin{document}
\title{Exercice 1.2}
\author{David Wiedemann}
\maketitle
\section*{I}
Pour montrer que $BG$ est une catégorie, il nous suffit de montrer que la composition est associative et que $BG$ admet une application identité.
\subsection*{Identité}
On a que $e \in G$ satisfait $ \forall a  \in G : a \cdot e = e \cdot a = a$ et ainsi $e$ satisfait les conditions pour être une identité en tant qu'élément de $\mor BG$ .\\
\subsection*{Associativité}
Soit $a,b,c\in G$, l'associativité dans $G$ donne que $( a\cdot b) \cdot c= a \cdot ( b \cdot c).$\\
Ainsi, l'associativité tient également quand on considère $a,b,c \in \mor BG$ et on en déduit que $BG$ est bien une catégorie.
\section*{II}
On explicite une bijection entre les 2 classes d'objets.\\
Soit 
\begin{align*}
	\phi: \gr( G,H) &\to \cat( BG,BH) \\
	f &\to ( \phi( f)_{\ob}, \phi( f) _{\mor} ) 
\end{align*}

défini par
$\phi ( f)_{\ob} ( \star_G) = \star_{H} $ et $\phi( f) _{\ob} ( a) = f( a) \forall a \in G $.\\
Montrons la bijectivité.\\

Pour l'injectivité, soit $\phi( f) ,\phi( g) \in \cat ( BG,BH) $ tel que $\phi( f) =\phi( g) $, ainsi $\forall a \in G, \phi( f) ( a) = \phi( f) ( a) \Rightarrow  f( a) =g( a) \Rightarrow f=g  $.\\
De plus, pour la surjectivité, soit $F \in \cat ( BG,BH) $, alors $\forall a,b \in \mor BG = G:  F( a\cdot b) = F( a) \cdot F( b) $ et $F( e_G )= e_H$, ainsi, en considérant $F$ comme une application ensembliste entre groupes ( en considérant seulement $F_{\mor} $ ), on obtient un morphisme de groupe et $\phi( F_{\mor} ) = F$.
\section*{III}
On considère $\Phi = ( \Phi_{\ob} , \Phi_{\mor} ) : \gr \to \cat $, on définit
\[ 
	\Phi_{\ob} ( G) = BG \text{ et } \Phi_{\mor} ( f) = \phi( f) 
\]
ou $\phi$ est défini comme dans la section II.\\
Il nous suffit donc de vérifier que $\Phi $ définit bien un foncteur.

On a 
\[ 
	\phi( f) _{\mor} ( e_{G}  ) = f( e_{G} ) = e_{H} 
\]
de plus
\[ 
	\forall a,b \in G= \mor BG\quad \phi( f) _{\mor} ( a \cdot b) = f( a\cdot b ) = f( a) \cdot f( b) =\phi( f) _{\mor} ( a) \cdot \phi ( f) _{\mor} ( b) 
\]








\end{document}
