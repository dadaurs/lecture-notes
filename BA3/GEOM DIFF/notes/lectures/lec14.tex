\documentclass[../main.tex]{subfiles}
\begin{document}
\lecture{14}{Wed 22 Dec}{Rudiments de Geometrie hyperbolique}
Rappel: La sphere unite  $S^{2} \subset \mathbb{R}^{3}$.\\
On a des cartes ( parametrisations) conformes classiques telles que
\begin{enumerate}
\item Carte de Mercator
\item Carte stereographique 
\end{enumerate}
\begin{propo}
Le disque de Poincarre est isometrique au demi-plan de Poincarre.
\end{propo}
\begin{proof}
On a une application bi-holomorphe $\phi: \mathbb{H}_+ \to \mathbb{D}^{2}$ 
\[ 
\phi( z) = \frac{z-i}{z+i}
\]
Il suffit de montrer que $w= \phi( z) $ implique
\[ 
\frac{|dz|}{\im z}= \frac{2 |dw|}{( 1-|w|^{2}) }
\]

\end{proof}
\subsection*{Digression: Groupe de Moebius}
En notation complexe, une similitude du plan euclidien $ \mathbb{R}^{2} \simeq \mathbb{C}$ s'ecrit 
\[ 
f( z) = az+ b = |a| e^{i\alpha} z + b
\]
ou
\[ 
f( z) = a \overline{z}+ b
\]
En ajoutant l'inversion $z \to \frac{1}{ \overline{z}}$ on obtient le groupe de Moebius.
\begin{defn}[Transformation de Moebius]
Une application de Moebius est une application de la forme
\[ 
f( z) = \frac{az + b}{cz+d}
\]
avec $ad- bc \neq 0 $.	
\end{defn}
On note $ Mob^{+}( 2) $ l'ensemble des transformations de Moebius et on les appelle aussi des Homographie.	
\begin{propo}
Les transformations de Moebius forment un groupe et l'application 
\[ 
GL_2( \mathbb{C}) \to Mob^{+}( 2) 
\]
est un homomorphisme de groupe.\\
On a la suite exacte
\[ 
0 \to \mathbb{C}\to GL_2( \mathbb{C}) \to Mob^{+}( 2) \to 1
\]

On note $PGL_2(  \mathbb{C}) = \faktor{ GL_2( \mathbb{C}) }{ \mathbb{C}^{*}}$ 

\end{propo}
Une homographie $f$ agit sur $ \hat{ \mathbb{C}}$ par $f( \infty ) = \frac{a}{c}, f( \frac{-d}{c}) = \infty $.

\begin{lemma}
La derivee complexe de $f( z) = \frac{az+b}{cz+ d}$ est 
\[ 
f'( z) = \frac{ad- bc}{( cz+ d) ^{2}}
\]

\end{lemma}
\begin{propo}
$ \mathbb{H}^{2}$ et $ \mathbb{D}^{2}$ sont isometriques
\end{propo}
\begin{proof}
On a que $\phi: H^{2}\to \mathbb{D}^{2}$ est une isometrie.
\end{proof}
\begin{propo}
Les homographies et anti-homographies reelles sont les isometries du demi-plan de Poincarre.
\end{propo}
\subsection{Droites hyperboliques ( dans $\mathbb{H}^{2}$ ) }
\begin{lemma}
La distance entre deux points de meme partie reelle dans $ \mathbb{H}^{2}$ est
\[ 
d( z_1,z_2) = | \log ( \frac{\im z_1}{\im z_2}) 
\]

\end{lemma}
\begin{proof}
Le chemin le plus court entre deux points $z_1$ et $z_2$ de $ \mathbb{H}^{2}$ est la droite verticale ou l'arc de cercle orthogonal a $ \del \mathbb{H}^{2}$ passant par $z_1$ et $ z_2$ 
\end{proof}
\begin{defn}
	Une droite hyperbolique de $ \mathbb{H}^{2}$ est une demi-droite verticale ou un demi-cerrcle orthogonal a $\del \mathbb{H}^{2}$ 
\end{defn}	


\[ 
\int_{ 0 }^{ \infty  } \frac{x^{2}}{\sin x
}
\]




\end{document}	
