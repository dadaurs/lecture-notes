\documentclass[../main.tex]{subfiles}
\begin{document}
\lecture{5}{Wed 20 Oct}{...}
\section{Courbes dans le plan oriente}
\begin{defn}
	On dit que deux bases d'un ev reel de dimension finie.\\
	On dit que deux bases ont la meme orientation si la matrice de changement de base a determinant positif.\\
	C'est une relation d'equivalence appellee une "classe d'orientation"
\end{defn}
Un espace vectoriel est oriente si on en a choisi une classe d'orientation.\\
L'orientation canonique de $ \mathbb{R}^{n}$ est celle associee a la base canonique.
\begin{defn}[Angle oriente]
	L'angle oriente entre deux vecteurs $\vec{a}$ et $\vec{b}$ dans un plan oriente est defini par $\hat { \vec{a},\vec{b}} = \pm \hat{ \vec{a},\vec{b}}$ avec le signe $+$ si $\left\{ \vec{a},\vec{b} \right\} $ est une base positive ( directe ) et $-$ sinon.
\end{defn}
\begin{defn}[L'operateur J ]
	On note $J = R_{+\frac{\pi}{2}} $ la rotation dans un plan oriente d'angle $+ \frac{\pi}{2}$.
\end{defn}
Dans une base orthonormee directe, on a $J = R_{+\frac{\pi}{2}} = \begin{pmatrix}
	0 & -1\\
	-1 & 0
\end{pmatrix} $ 
\begin{defn}[Produit exterieur]
	Le produit exterieur de deux vecteurs $\vec{a}$ et $\vec{b}$ dans un plan oriente est
	\[ 
	\vec{a}\wedge \vec{b} =\eng { \vec{a}, J\vec{b}} 
	\]
	
\end{defn}
\begin{rmq}
\begin{enumerate}
\item $a \wedge b = \N a \N b \sin \theta$, alors
\item Si on plonge $ \mathbb{R}^{2}\to \mathbb{R}^{3}$ 
	\[ 
	\vec{a}\wedge \vec{b} = \eng { \vec{a}\times \vec{b}, e_3} 
	\]
	
\end{enumerate}
\end{rmq}
\begin{defn}
	Si $\gamma: I \to \mathbb{R}^{2}$ une courbe de classe $C^{1}$, alors on definit le vecteur normal oriente a $\gamma$ 
	\[ 
		N^{or}( u) 
	\]
	par la condition que
	\[ 
		T_\gamma \wedge N^{or}>0 \iff \left\{ T, N^{or}  \right\} \text{ est orthonormee directe } 
	\]
\end{defn}
On definit la courbure orientee d'une courbe reguliere de classe $C^{2}$ par
\[ 
	k( u) = \kappa^{or}( u) = \frac{1}{V} \eng { \dot T, N^{or}} 		
\]
\begin{defn}
	On dit qu'un arc de courbe $\gamma$ de classe $C^{2}$ dnas un plan oriente est convexe si $k>0 $, concave si $k<0$.\\
	On dit que $\gamma$ est une "spirale" si la courbure est non nulle et monotone.\\
	Un point d'inflexion si la courbure change de signe en ce point.\\
	Un point est un "sommet" si la derivee de la courbure change de signe
\end{defn}
\begin{rmq}
	Le graphe de $f( x) $ a un point d'inflexion en $f"( x) =0$ et $f"( x) $ change de signe
\end{rmq}
\begin{defn}[Fonction angulaire]
	La fonction angulaire d'une courbe plane ( dans un plan oriente).\\
	Soit $\gamma: [ a,b] \to \mathbb{R}^{2}$ une courbe $C^{2}$.\\
	On appelle fonction angulaire de $\gamma$ la fonction
	\[ 
	\phi: [ a,b] \to \mathbb{R}
	\]
	verifiant 
	\begin{enumerate}
	\item $\phi$ est continue
	\item $\phi( u) = \hat{ \dot \gamma ( u) , \vec{a}} \mod 2\pi$ 
	\end{enumerate}
	
\end{defn}
\begin{rmq}
Si on prend $\vec{a}= e_1$, alors
\[ 
T = \frac{\dot\gamma}{\N { \dot\gamma }}
\]

\end{rmq}
\begin{propo}
La courbure orientee est la variation naturelle de la fonction angulaire.
\end{propo}
\begin{proof}
	On a $T= ( \cos\phi,\sin\phi) $, donc 
	\[ 
		N = ( -\sin\phi,\cos\phi) 
	\]
	et de meme
	\[ 
	k= \frac{1}{V}\eng { \dot T, N} 
	\]
	
\end{proof}
\begin{thm}[Theoreme fondamental des courbes planes]
	Soit $k: [ 0,L] \to \mathbb{R}$ une fonction continue, alors il existe une courbe $\gamma: [ 0,L] \to \mathbb{R}^{2}$, de classe $C^{2}$ de courbure $k( s)$ et de vitesse 1.\\
	Cette courbe est unique a isometrie directe pres.
\end{thm}
\begin{proof}
\subsubsection*{Existence}
La fonction $k: [ 0,L] \to \mathbb{R}$ est donnee. On pose 
\[ 
	\phi( s) = \int_{ 0 }^{ s } k( u) du
\]
Puis on pose $T( s) = ( \cos\phi( s) ,\sin( \phi( s) ) ) $, on a donc $N( s) = ( -\sin( \phi( s) ) ) , \cos\phi( s) $.\\
On pose encore
\[ 
	X( s) = \int_{ 0 }^{ s }\cos( \phi( s) ) ds, Y( s) = \int_{ 0 }^{ s }\sin( \phi( s) ) ds
\]
On pose enfin $\gamma( s) = ( X( s) , Y( s) ) $.\\
L'existence est donc prouvee.\\
L'unicite vient de $ \frac{d\phi}{ds}=k$ 
\end{proof}
\begin{defn}
	On dit qu'une courbe $\gamma: [ a,b] \to \mathbb{R}^n$ est une courbe fermee de classe $C^{k}$ si $\gamma( a) =\gamma( b)$ et les derivees coincident.
\end{defn}	
Une telle courbe s'appelle aussi courbe periodique, car on peut l'etendre a 
\[ 
\gamma: \mathbb{R}\to \mathbb{R}^{2}
\]
\begin{thm}[Theoreme des 4 sommests]
	Toute courbe plane $C^{2}$-fermee admet au moins 4 changement de signes  de $ \frac{dk}{du}$ 	
\end{thm}
\begin{proof}
On montre le resultat dans le cas d'une courbe convexe.\\
Soit $\gamma: [ a,b] \to \mathbb{R}^{2}$ une courbe $C^{2}$-convexe. Alors on a $k( a) =k( b) $ et on note $\gamma( s) = ( X( s) ,Y( s) ) $.\\
On suppose $V_\gamma( s) =1$, alors on a
\[ 
\frac{dT}{ds}= kN
\]
avec 
\[ 
	T= ( \dot x,\dot y) , N = JT= ( -\dot y, \dot x) 
\]
Ainsi,
\[ 
\ddot x= - k \dot y, \ddot y = k \dot x
\]
On a alors
\[ 
	\int_{ a }^{ b }  \dot k ( s) = 0
\]
et
\[ 
	\int_{ a }^{ b } k( s) \dot y ( s) ds = - \int_{ a }^{ b }\ddot  y ( s) ds =0
\]
et
\[ 
	\int_{ a }^{ b }k( s) \dot x ( s) ds =0
\]
Supposons que $\gamma: [ a,b] \to \mathbb{R}^{2}$ est $C^{2}$-fermee avec deux changements de signe de $\dot k ( s) $, par exemple $\dot k ( s) >0$ sur $ ( a,c) $ et $\dot k ( s) <0$ sur $ ( c,b) $. Notons $p= \gamma( a) = \gamma( b) , q = \gamma( c) $, soit
\[ 
	h( x,y) = Ax + By+C=0
\]
l'equation de la droite par $p$ et $q$.\\
Alors le signe de $\dot k ( s)( Ax+By +C)  $ est constant.\\
Mais 
\begin{align*}
0<	\int_{ a }^{ b } f( s) ds &= A \int_{ a }^{ b }\dot k x( s) ds + B \int_{ a }^{ b }\dot k y( s) ds + C \int_{ a }^{ b }\dot k ds=0\\
\end{align*}
Contradiction.			
		
\end{proof}
\section{Surfaces}
\subsection{Le concept de variete}
\begin{defn}[Variete]
	Une variete topologique de dimension $n\in \mathbb{N}$ est un espace topologique $M$ tel que
	\begin{enumerate}
	\item Chaque point $p\in M$ admet un voisinage homeomorphe a un ouvert de $ \mathbb{R}^n$ 
	\item L'espace topologique $M$ est separe ( de Hausdorff) et admet une base denombrable d'ouvert
	\end{enumerate}
\end{defn}
\begin{defn}[Surface topologique]
	Une surface topologique est une variete de dimension 2.
\end{defn}
	
				

								
\end{document}	
