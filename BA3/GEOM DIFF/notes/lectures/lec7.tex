\documentclass[../main.tex]{subfiles}
\begin{document}
\lecture{7}{Wed 03 Nov}{Varietes (enfin)}
\subsection{Exemples de Sous-varietes}
\begin{enumerate}
\item Une sous-variete de dimension $0$ de $ \mathbb{R}^n$ est un sous-ensemble discret ( tous les points sont isoles) 
\item Un ouvert $U \subset \mathbb{R}^n$ est une sous-variete de dimension $n$.
\item L'ensemble vide est une sous-variete de dimension $n$ $\forall n$
\end{enumerate}
\begin{thm}
	Soit $f:U \subset  \mathbb{R}^m\to \mathbb{R}^n$ une application $C^{k}$ et de rang constant $=r$.\\
	\begin{itemize}
	\item L'ensemble des points 
		\[ 
		M= \left\{ x\in U | f( x) =0 \right\} \subset \mathbb{R}^m
		\]
		est une sous-variete differentiable de classe $C^{k}$ de dimension $m-r$.
	
	\item $\forall p\in U$, il existe un voisinage $V \subset U$ de $p$ tel que $N= f( V) $ est une sous-variete	
	\end{itemize}

\end{thm}
\begin{defn}[Groupe de Lie]
	Un groupe de Lie est une variete differentiable $G$  tel que la multiplication et l'inverse sont des operations bien definies.
\end{defn}
\subsection{Sur les differentielles et gradients}
Que vaut la differentielle  $dx_i$ ?
On a que
\[ 
dx_i\vert_p( h) =  x_i( p+h) -x_i( p) = p_i+h_i - p_i = h_i
\]
On a que $dx_i ( e_j) = \delta_{ij} $ , donc $ \left\{ dx_i,\ldots, dx_n \right\}\in ( \mathbb{R}^n) ^{*} $ est la base duale canonique.

	


\end{document}	
