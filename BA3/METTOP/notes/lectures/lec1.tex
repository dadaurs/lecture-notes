\documentclass[../main.tex]{subfiles}
\begin{document}
\lecture{1}{Tue 21 Sep}{Introduction}
\section{Sets}
\begin{itemize}
\item set: list of elements
\item notion of equivalence: bijection
\item can we classify up to eq classes?
\end{itemize}
\subsection{Finite sets}
\begin{defn}[Finite sets]
	We call a set finite if $\exists n \in \mathbb{N}, A \isom [ n] $ and we say the size of $A$ is $n$.
\end{defn}
Why is size well defined? $\to$ proof in notes.\\
\subsection{Infinite sets}
\begin{defn}[Infinite set]
	A set is contable if $A$ is finite or in bijection with $\mathbb{N}$ 
\end{defn}
We can classify $\mathbb{N}< \mathbb{R}$, a similar argument gives that $\mathcal{P}( \mathbb{R}) > \mathbb{R}$ 
How to decide if $A \isom B$ ?
\begin{thm}[Schroeder-Bernstein]
	If $\exists f: A \inj B, g: B\inj A$, $A\isom B$.
\end{thm}
\end{document}	
