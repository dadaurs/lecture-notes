\documentclass[../main.tex]{subfiles}
\begin{document}
\lecture{5}{Thu 07 Oct}{...}
\subsection{Logarithme}
Moralement, on aimerait definir le logarithme comme "l'inverse" de l'exponentielle.\\
Dans les reel, c'est ainsi qu'on avait procede, mais la difference, c'etait que la fonction exponentielle etait bijective.\\
Ici, on a que la fonction exponentielle $ \mathbb{C}\to \mathbb{C}^{*}$ est surjective.\\
Du coup on aurait envie de definir le $\log$ sur $\mathbb{C}^{*}$, mais la fonction exponentielle n'est pas injective.\\
En fait, cela fait qu'on ne peut pas definir une fonction $\log$ qui soit continue sur $\mathbb{C}^{*}$. Si on essaie de poser
\begin{align*}
	\log \exp ( a+ib) &= a+ib\\
	\log e^{a}  e^{ib} = \log |w| + i \arg w
\end{align*}
Comment choisir $\arg w$.\\
\begin{defn}
Une determination du logarithme est une fonction
\[ 
L:U \to \mathbb{C}
\]
ou $U$ est un ouvert de $\mathbb{C}$ tel que $ e^{L( z) } =z$ 
\end{defn}
\begin{rmq}
	Sur $ \mathbb{C}\setminus \mathbb{R}_-$, on a une determination de l'argument et du log: $\forall z \in \mathbb{C}\setminus \mathbb{R}_-$, on prend $arg w$ dans $ ( -\pi, \pi) $.
\end{rmq}
			
\begin{propo}
Il n'existe pas de determination continue du logarithme sur $ \mathbb{C}^{*}$ 
\end{propo}
\begin{proof}
Tous les problemes viennent du fait qu'on fait un tour autour de l'origine.\\
Montrons qu'il n'en existe pas sur $ \mathbb{S}^{1}$.\\
Supposons qu'on ait une telle determination du log.\\
Posons $u : \mathbb{R}\to \mathbb{C}$ definie par $u^{\theta}= f( e^{i\theta} ) $. On a $u ( \theta) -\theta = 2\pi in \theta$ puisque $ e^{u( \theta) } - \theta$ \\
Donc
\[ 
	\im u ( \theta) = \arg ( e^{i\theta}) + 2 \pi \mathbb{Z}
\]
Cependant
\[ 
	u( \theta + 2 \pi ) = \theta + 2 \pi + n  =u( \theta) 
\]
	

\end{proof}
\section{Fonctions holomorphes}
On souhaite generaliser la notion de derivee aux fonctions complexes.\\
Une possibilite est de voir le plan $ \mathbb{R}^{2}$ et en utilisant les notions de calcul differentiel sur $ \mathbb{R}^{2}$ \\
La notion d'holomorphie, c'est celle d'etre derivable au sens d'une variable complexe, et on verra que c'est une notion beaucoup plus forte que celle d'etre differentiable au sens de $ \mathbb{R}^{2}$, mais suffisamment naturelle pour etre verifiee dans beaucoup de cas
\begin{defn}[Fonction Holomorphe]
	Soit $f: U \to \mathbb{C}$ ou $U$ est un domaine de $ \mathbb{C}$. On dit que $f$ est holomorphe en $z \in U$ s'il existe une limite notee $ f'( z) \in \mathbb{C}$ si la limite suivante existe
	\[ 
		\lim_{h \to 0} \frac{f( z+h) - f( z) }{h}
	\]
	ou la limite est prise au sens  complexe.\\
	Formellement, cela veut dire $ \forall \epsilon >0 \exists \delta>0$ tel que si $ h \in D( 0, \delta) \setminus \left\{ 0 \right\} $ on a 
	\[ 
		| \frac{f( z+h) -f( z) }{h}- f'( z) 	| \leq \epsilon
	\]
Une autre maniere d'ecrire cela 
\[ 
	f( z+h)  = f( z) + f'( z) h + o( h) 
\]
ou $ o( h) $ est telle que $|o( h) /h| \to 0$.\\
\end{defn}
Comment comprendre ca en termes de derivees partielles en faisant l'identification $ \mathbb{R}^{2}\simeq \mathbb{C}$?\\
Dire que la fonction a un developpment de Taylor au 1er ordre pour une fonction $ \mathbb{R}^{2}\to \mathbb{R}^{2}$ 
\begin{align*}
\begin{pmatrix}
	f_1 ( x_1 + h_1, x_2+h_2) \\ f_2( x_1+h_1, x_2+h_2) 
\end{pmatrix} =
\begin{pmatrix}
	\del_1 f_1( x_1,x_2) & \del_2 f_1( x_1,x_2) \\
	\del_1f_2( x_1,x_2) & \del_2 f_2( x_1,x_2) 
\end{pmatrix} 
\begin{pmatrix}
h_1\\h_2
\end{pmatrix} + o ( h_1,h_2) 
\end{align*}
Donc la matrice $ Df_{x_1,x_2} $ doit etre la matrice de la composition d'une rotation et d'une homothetie.
Donc la contrainte d'etre differentiable au sens complexe est equivalence a celle de demander d'etre differentiable au sens de deux variables relles et d'avoir que la jacobienne soit de la forme
\[ 
\begin{pmatrix}
	A & B \\
	-B & A
\end{pmatrix} 
\]
donc
\begin{align*}
\del_1 \re f = \del 2 \im f \text{ et } \del_1\im f = - \del 2 \re f
\end{align*}
\begin{defn}
On dit que $f$ satisfait les equations de Cauchy-Riemann si 
\begin{align*}
\del_1 \re f = \del 2 \im f \text{ et } \del_1\im f = - \del 2 \re f
\end{align*}
\end{defn}
\begin{propo}
$f$ est holomorphe en $z= x_1 + i x_2$ $\iff$ $f$ est derivable au sens de $ \mathbb{R}^{2}$ et satisfait les equations de Cauchy-Riemann.
\end{propo}
\begin{defn}
	On dit que $f$ est holomorphe sur un ouvert $ U \subset \mathbb{C}$ si elle est $C^{1}$ ( au sens de $ \mathbb{R}^{2}$ ) et qu'elle est holomorphe en tout point de $ U$ 	
\end{defn}
\subsection{Analytique $\Rightarrow$ Holomorphe}
\begin{propo}
	Si
	\begin{align*}
		f( z) = \sum a_k ( z-z_*) ^{k}
	\end{align*}
	a comme rayon de convergence $ \rho$ , alors $f$ est holomorphe sur $D( z_*, \rho) $ et $f'$ est donee par la serie entiere
	\[ 
		f'( z) = \sum k a_k ( z-z_*) ^{k-1}
	\]
	qui a aussi comme rayon de convergence $\rho$.
\end{propo}
\begin{proof}
La serie qui donne la derivee converge avec rayon de convergence $\rho$. Maintenat, ce qu'il nous faut, c'est de montrer que
\[ 
	\lim_{h \to 0} \frac{f( z+h) - f( z) - hf'( z) }{h}=0
\]
Supposons $z_*=0$.
\begin{align*}
	\sum_{k=0}^{ \infty } \frac{a_k ( z+h) ^{k}- a_k z^{k}}{h}-h k z^{k-1}\to 0\\
&	\sum_{k=1}^{ \infty } \frac{a_k [ ( z+h) ^{k}- z^{k}- kh z^{k-1}] }{h}\\
&= \sum_{k=1}^{ \infty }\frac{a_k [ h [ ( z+h^{k-1})+ ( z+h) ^{k-2}+ \ldots + z^{k-1} ] ] - kh z^{k-1}}{h}
\end{align*}
On va montrer que $\forall \epsilon>0$, on peut rendre la queue de la serie plus petite que $\frac{\epsilon}{2}$ en allant assez loin dans la serie et qu'ensuite, pour le $N$ fixe qui sortira, on pourra prendre $h$ assez petit pour que les $N$ premieres termes soient plus petits que $ \frac{\epsilon}{2}$.\\
Notons que pour $mh$ suffisamment petit ( il existe $\delta_1>0$ tel que si $h \in D( 0, \delta_1), z +h \in D( 0, \rho) $ ) et du coup on aura la convergence de
\[ 
\sum_{k=1}^{ \infty }a_k [ [ |z+h|^{k-1}+ \ldots + |z^{k-1}|]+ k |z|^{k-1} ] 
\]
vu que
\[ 
	[ |z+h|^{k-1}+ \ldots] \leq  k ( |z|+|h|) ^{k}
\]




\end{proof}

		

\end{document}	
