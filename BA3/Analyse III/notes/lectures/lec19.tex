\documentclass[../main.tex]{subfiles}
\begin{document}
\lecture{19}{Mon 29 Nov}{Applications conformes}
\section{Applications conformes}
\begin{defn}[Fonction Holomorphe]
	Une fonction holomorphe bijective entre deux domaines.	
\end{defn}
\subsection{Applications conformes de $ \mathbb{D}\to \mathbb{D}$ }
Une famille d'applications conformes $ \mathbb{D}\to \mathbb{D}$:\\
Pour $\alpha\in \mathbb{D}$, notons $\phi_\alpha= \frac{z-\alpha}{ 1- \overline{\alpha}z}$.\\
Montrons que $\phi_\alpha  $ est une bijection.\\
Voyons que $\phi_\alpha	$ envoie $\del \mathbb{D}\to \del \mathbb{D}$.\\
Pour $\beta\in \mathbb{R}$
\begin{align*}
| \frac{ e^{i\beta} - \alpha}{1- \overline{\alpha e^{-i\beta}} }| &= | \frac{ e^{i\beta} -\alpha}{1- \alpha e^{i-\beta} }|=1	\\
\intertext{Par le principe du maximum, $\phi_\alpha$ envoie $ \mathbb{D}\to \mathbb{D}$ }
\end{align*}
On verifie de plus que l'inverse de $\phi_\alpha$ est $\phi_{-\alpha} $ 
Pour $\alpha\in \mathbb{D}$ et $\theta \in \mathbb{R}$, on note
\[ 
\phi_{\alpha,\theta} ( z) = e^{i\theta} \phi_\alpha( z) 
\]
Voyons pourquoi les $\phi_{\alpha,\theta} $ sont les seules applications conformes $ \mathbb{D}\to \mathbb{D}$.\\
\begin{proof}
Idee: si $\phi: \mathbb{D}\to \mathbb{D}$ est conforme, posons $\alpha= \phi( 0) $, alors $\tilde\phi= \phi_\alpha \circ \phi( 0) =0$.\\
De plus, si $\theta= \arg( \tilde\phi'( 0) ) $, alors 
\[ 
\tilde\phi= \phi_{\alpha,-\theta} \circ\phi: \mathbb{D}\to \mathbb{D}
\]

\end{proof}
\begin{thm}
Pour n'importe quel domaine simplement connexe $\Omega$, il existe une application conforme $ \Omega\to \mathbb{D}$.
\end{thm}
\begin{proof}
L'idee est de trouver unefonction injective $\phi$ qui envoie $z\to 0$, $\phi'( z) >0$ et l'optimiser pour la rendre surjective.\\
Posons $\Sigma_{\Omega,z_*} $ l'ensemble des fonctions holomorphes injectives qui envoient $z_*\to 0$
\begin{lemma}
$\Sigma_{\Omega,z_*} \neq \emptyset$ 
\end{lemma}
\begin{proof}
Soit $w_0\in \mathbb{C}\setminus \Omega$, comme $z\to z-w_0$ ne s'annule pas, il existe $f( z) = \sqrt{z-w_0} $, il existe $f( z) = \sqrt{z-w_0}$. Soit $\tilde\Omega= f( \Omega) $.\\
Montrons qu'il existe $\tilde \omega\in \mathbb{C}$ tel que $d( \tilde \omega, \tilde \Omega) >0$.\\
En effet, $\not\exists z_1,z_2$ tel que $f( z_1) = f( z_2) $ et $\not\exists z_1\neq - z_2$ tel que $f( z_1) = - f( z_2) $.\\
Donc
\[ 
f( \Omega) \cap -f(  \Omega) = \emptyset
\]

Soit $w\in \mathbb{C}\setminus \Omega$. Soit $\psi( z) = \sqrt{z- w} $.\\
On a $\psi( \Omega) \cap - \psi( \Omega) $.\\
Par le theoreme de l'application ouverte $\psi( \Omega) $ est ouvert et donc $\exists b \in \psi( \Omega) $ et $r( 0, |b|) $ tel que $D( b,r) \subset \psi( \Omega) $. Mais alors $D( -b,r) \subset \mathbb{C}\setminus\psi( \Omega) $ .\\
Donc $z\to \frac{r}{\psi( z) +b}$ est injective et envoie $\Omega\to \mathbb{D}$.\\
En postcomposant avec une application de Moebius on arrive a une fonction dans $\Sigma_{\Omega, z_*} $ 	
\end{proof}

\end{proof}



\end{document}	
