\documentclass[../main.tex]{subfiles}
\begin{document}
\lecture{9}{Thu 21 Oct}{...}
\begin{thm}
	$\ind( \gamma,z) \in \mathbb{Z}$ 
\end{thm}
\begin{proof}
Montrons que 
\[ 
	\exp( \int_{ \gamma }^{  }\frac{d\zeta}{\zeta-z}) =1
\]
Posons $\phi( t) = \exp ( \int_{ 0 }^{ t }\frac{\gamma'( s) }{\gamma( s) -z}ds) $ 

On a donc
\[ 
	\del_t \log\phi( t) = \frac{\gamma'( t) }{\gamma( t) -z}
\]
De meme
\[ 
	\del_t \log( \gamma( t) -z) = \frac{\gamma'( t) }{\gamma( t) -z}
\]
Donc
\[ 
	\del_t \log ( \frac{\phi( t) }{\gamma( t) -z}) =0
\]
Si $\gamma$ n'est pas derivable sur un ensemble $S \subset [ 0,1] $ fini, la conclusion est la meme car si $f'( t) =0\forall t \in [ 0,1] \setminus S$ 35 $S$ fini, $f$ constante.
\end{proof}
\subsection{$\log$ et racines}
\begin{propo}
Soit $U$ un domaine simplement connexe qui ne contient pas 0 et $z_*\in U$ et $W_*$ tel que $ e^{W_*} = z_*$.\\
Alors la fonction $L: U \to \mathbb{C}$ definie par
\[ 
	L( z) = W_* + \int_{ z_* }^{ z }\frac{1}{\zeta}d\zeta
\]
est telle que
\[ 
	\exp( L( z) ) =z\forall z \in U
\]

\end{propo}
\begin{proof}
En $z_*$ , c'est bon.\\
On va essaier de montrer que sur un petit voisinage de $W_*$, $L\circ\exp=\id$.\\
On a 
\begin{align*}
	( L( e^{z}) )' = \frac{1}{e^{z}}e^{z}=1
\end{align*}
Si $w$ est dans un petit voisinage de $W_*$.\\
Donc,
\[ 
	\exp( L( \exp( w) ) ) = \exp ( w) 
\]
Comme $\exp$ est bijective dans un petit voisinage de $W_*$, on a $\exp\circ\log=\id$.\\
Dans la section suivante, on verra que holomorphe implique analytique et le principe des zeros isoles s'applique donc a la fonction $\exp\circ\log-\id=0$ 
\end{proof}
Avec un log on peut definir des racines.\\
Soit $U$ simplement connexe qui ne contient pas 0, alors pour tout $n$, il existe $n$ fonctions holomorphes $r_n: U\to \mathbb{C}$ tel que
\[ 
	( r_n)^{n}( z) =z
\]
\begin{proof}
	Prendre $\exp \frac{1}{n}L( z) $ 
\end{proof}
\begin{propo}
	Soit $U$ simplement connexe et $f:U\to \mathbb{C}$, avec $f( z) \neq 0 \forall z $, alors il existe $ L_f: U \to \mathbb{C}$ holomorphe tel que
	\[ 
		\exp( L_f ) 	= f( z) 
	\]
	et soit $z_*\in U$ et $l_*$ tel que $ e^{l_*} =z_*$ 
	
\end{propo}
\begin{proof}
On pose 
\[ 
	L_f( z) = l_* + \int_{ z_* }^{ z } \frac{f'( \zeta) }{f( \zeta) }d\zeta
\]
	
\end{proof}
\begin{rmq}
	Cela couvre des cas ou $f( U) $ pourrait ne pas etre simplement connexe.
\end{rmq}


	



\end{document}	
