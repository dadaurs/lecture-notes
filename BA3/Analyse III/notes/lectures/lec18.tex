\documentclass[../main.tex]{subfiles}
\begin{document}
\lecture{18}{Mon 22 Nov}{Theoreme des Nombres premiers preuve}
\begin{proof}[1]
On veut montrer que si $\phi( x) \sim x$, alors $\pi( x) \sim \frac{\phi( x) }{\log x}$
\[ 
\iff \pi( x) \log x \sim \phi( x) 
\]
On a clairement
\[ 
\pi( x) \log x \geq \phi( x) 
\]
Soit $\epsilon>0$, on a
\[ 
\phi( x) \geq  \sum_{x^{1-\epsilon} \leq p \leq x  }^{ }\log p \geq \sum_{x^{1-\epsilon} \leq p \leq  x}^{ } \log ( x^{1-\epsilon}) = ( 1-\epsilon) ( \pi( x) -\pi( x^{1-\epsilon}) ) \log x
\]
Pour tout $x, \pi( x^{1-\epsilon}) \leq x^{1-\epsilon}$ et $\pi( x) \geq \frac{\phi( x) }{\log x}$. Donc si $\phi( x) \sim x$ quand $x\to \infty $, alors
\[ 
\pi( x^{1-\epsilon}) / \pi( x) \oar { x\to \infty } 0
\]
et donc
\[ 
\phi( x) \geq ( 1-\epsilon^{2}) \pi( x) \log x
\]
	

\end{proof}
\begin{proof}[2]
Prouvons le par l'absurde.\\
Montrons qu'il n'est pas possible de trouver $\lambda>1$  une suite $a_n\to \infty $ tel que $\phi( a_n) \geq a_n $ et $a_{n+1} \geq \lambda a_n$  ou $\mu<1$ et une suite $b_n\to \infty $ avec 
\[ 
\phi( b_n) \leq \mu b_n \text{ et } b_{n+1} \geq \frac{1}{\mu}b_n	
\]
Si la suite $a_n$ existait, montrons que $I_1$ diverge.	
Montrons que
\begin{align*}
\int_{ a_n }^{ \lambda a_n }	\frac{\phi( x) -x}{x^{2}}dx \geq \int_{ a_n }^{ \lambda a_n } \frac{\phi( a_n) -x}{x^{2}}dx\\
\geq \int_{ a_n }^{ \lambda a_n } \frac{\lambda a_n -x}{x^{2}}\\
= \int_{ 1 }^{ \lambda} \frac{\lambda-t}{t^{2}} dt \geq \frac{1}{\lambda^{2}}\int_{ 1 }^{ \lambda } ( \lambda-t) dt = \frac{\lambda-1}{2\lambda^{2}}>0
\end{align*}

\end{proof}
Pour le lemme 3, commencons par un petit lemme intermediaire
\begin{lemma}
$\frac{\phi( x) }{x}$ est bornee.
\end{lemma}
\begin{proof}
Pour ce faire, etudions 
\[ 
\phi( 2n) -\phi( n) = \log \prod_{n \leq p \leq 2n} p \leq \log \binom { 2n } n \leq \log ( 1+1)^{n}= \log ( 2 ) 2n
\]
Pour $x>1$ , $\phi( x) \geq \phi( \lfloor x \rfloor) $ 
\begin{align*}
\phi( 2x) \leq \phi( 2 \lfloor x \rfloor) )+ \log ( 2 \lfloor x \rfloor +1) 
\end{align*}
Donc
\begin{align*}
	\phi( 2x) -\phi( x) &\leq \phi( \lfloor x \rfloor ) - \phi( \lfloor x \rfloor ) + \log ( 2x+1) \\
			    & \leq 2 \log 2 \lfloor x \rfloor + \log ( 2x+1 )\\
			    & \leq 2 Cx
\end{align*}
Donc
\begin{align*}
\phi( x) - \phi(  \frac{x}{2^{n}}) \leq C( \frac{x}{2}+ \frac{x}{4}+ \ldots) \leq Cx
\end{align*}



\end{proof}
\begin{proof}[3]
Repose sur la transformation de Laplace.
\begin{defn}[Transformee de Laplace]
	Soit $f: \mathbb{R}_+ \to \mathbb{C} $ une fonction bornee et continue par morceaux sur $[0, \infty )$.\\
	La transformee de Laplace de $f$, notee $ \mathcal{L}( f) $ est definie sur $ \mathbb{H}_0$ par
	\[ 
	\mathcal{L} f( z) = \int_{ 0  }^{ \infty  } f( z) e^{- tz} dt
	\]
		
\end{defn}
\begin{thm}
Si $f$ est bornee et que $ \mathcal{L}f$ s'etend en une fonction meromorphe sur $ \mathbb{H}_{-\delta} $ pour $\delta>0$ sans poles sur $ \mathbb{H}_0$, alors 
\[ 
\int f( t) dt
\]
converge et vaut $\mathcal{L}f( 0) $ 
\end{thm}
Le lemme 3 en suit ( en prenant la transformee de Laplace ).
\end{proof}

	


\end{document}	
