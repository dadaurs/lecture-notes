\documentclass[../main.tex]{subfiles}
\begin{document}
\lecture{22}{Thu 09 Dec}{Analyse Vectorielle}
\section{Analyse Vectorielle}
But: Comprendre le gradient, le rotationnel, la divergence, le laplacien et les theoremes les reliant.\\
\begin{defn}
Soit $U \subset \mathbb{R}^n$ un domaine.\\
Des fonctions $C^1$ $\phi: U \to \mathbb{R}$ est apellee champ scalaire et $X: U\to \mathbb{R}^m$ un champ vectoriel.
\end{defn}
\subsection{Integrales curvilignes et circulation}
Soit $U$ un domaine, $\phi:U\to \mathbb{R}$ un champ, et $\gamma [ 0,1] \to U$ $C^{1}$ par morceaux, continue, derivees bornees.
\begin{defn}[Integrale curviligne]
	L'integrale curviligne de $\phi$ le long de $\gamma$ 
	\[ 
	\int_{ \gamma }^{  }\phi ds
	\]
	est par definition
	\[ 
	\int_{ 0 }^{ 1 } \phi( \gamma( s) ) \N { \gamma'( s) } ds
	\]
	En particulier
	\[ 
	l( \gamma) = \int_\gamma 1 ds
	\]
	
	
\end{defn}
\begin{rmq}
Une reparametrisation ne change pas la valeur de l'integrale.
\end{rmq}
\begin{defn}[Circulation]
	Soiit $X:U\to \mathbb{R}^n$ un champ vectoriel et $\gamma$ une courbe.\\
	La circulation de $X$ le long de $\gamma$ est defini par
	\[ 
	\int_{ \gamma }^{  }X \cdot ds = \int_{ 0 }^{ 1 }X( \gamma( s) ) \cdot \gamma'( s) ds
	\]
	
\end{defn}
\begin{rmq}
Cette definition est aussi invariante par reparametrisation.
\end{rmq}
\begin{defn}[Gradient]
	Le gradient d'un champ scalaire $\phi$ est donne par
	\[ 
	\nabla \phi = ( \del_1\phi,\ldots,\del_n\phi) 
	\]
	    	
\end{defn}
\begin{propo}
Circulation d'un gradient.
\[ 
\int_{ \gamma }^{  }\nabla\phi\cdot ds = = \phi( \gamma( 1) ) -\phi( \gamma( 0) ) 
\]
\end{propo}
\begin{proof}
	En supposant $\gamma C^1$ 
\[ 
\int_{ 0 }^{ 1 } \nabla\phi( \gamma( s) ) \cdot \gamma'( s) ds = \int_{ 0 }^{ 1 } \frac{d}{ds}( \phi\circ\gamma ) ds = \phi( \gamma( 1) ) -\phi( \gamma( 0) ) 
\]

\end{proof}
On dit qu'un champ vectoriel derive d'un potentiel ( scalaire) s'il existe $\phi$ tel que
\[ 
\nabla\phi= X
\]
\begin{propo}
Si $X:U\to \mathbb{R}^n$ est un champ vectoriel tel que $\forall \gamma$ un lacet
\[ 
\int_{ \gamma }^{  }X ds =0
\]
Alors $X$ derive d'un potentiel.
\end{propo}
\begin{proof}
Soit $x_*\in U$, posons pour $x\in U$ 
\[ 
\phi( x) = \int_{ \gamma} X ds
\]
ou $\gamma$ est un chemin de $x_*$ vers x.\\
Cette integrale ne depend pas du choix de $\gamma$.\\
Montrons que $\nabla \phi = X$.\\
$\forall v \in \mathbb{R}^n$ 
\[ 
\nabla\phi( x) \cdot v = X( x) \cdot v
\]
On a donc
\[ 
\lim_{t \to  + \infty} \frac{1}{t}\int_{ x }^{ x+tv } X ds
\]
On parametre le segment de maniere affine
\[ 
= \frac{1}{t} X( x+stv)\cdot tv= X( x+stv) \cdot v\to X( x) \cdot v	
\]
\subsection{Rotationel}
Pour $X:U \subset \mathbb{R}^{2}\to \mathbb{R}^{2}$ 
\[ 
\nabla\times X: U \to \mathbb{R} : x \to \del_1X_2- \del_2X_1
\]






\end{proof}


	
\end{document}	
