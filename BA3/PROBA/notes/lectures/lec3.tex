\documentclass[../main.tex]{subfiles}
\begin{document}
\lecture{3}{Wed 06 Oct}{Measurable maps}
\subsection{Borel $\sigma$-algebra}
\begin{itemize}
	\item Cannot define shift-invariant probability measure on $ ( [ 0,1] , \mathcal{P}( [ 0,1] ) ) $.
	\item What $\sigma$-algebra to choose on $( X,\tau) $?
	\item Want to know the siize of all open-sets
\end{itemize}
\begin{defn}[Borel sigma-algebra]
	On $( X,\tau) $ the borel $\sigma$-algebra $ \mathcal{F}_{\tau} $ is the smallest $\sigma$-algebra containing $\tau$.
\end{defn}
This is well defined because, given a collection of $\sigma$-algebras, their intersection is too.
\subsection*{Two nice properties}
\begin{itemize}
\item Continuous functions on a Borel $\sigma$-algebra are also measurable.
	\begin{proof}
		Suffices to check that $f^{-1}( U) \in \mathcal{F}_{\tau_1} $ for $U\in \tau_2$ but this is immediate since $f$ is continuous.
	\item In $( \mathbb{R}^{n},\tau_E )$, the Borel $\sigma$-algebra $ \mathcal{F}_E$ is generated by $( a_1,b_1) \times ... \times( a_n,b_n) $. $ \mathcal{F}_E$ is the smallest $\sigma$-algebra containing open intervalls.
	\end{proof}
	
\end{itemize}
\subsection{Probability Measures on $\mathbb{R}^{n}$ }
\begin{thm}[Existence of Lebesgue-measure]
	There exists a unique measure $\lambda$ on $ ( \mathbb{R}^n, \mathcal{F}_E) $ s.t. $ \lambda( ( a_1\times b_1) \times \ldots \times ( a_n,b_n) ) = \prod_i |b_i-a_i|$ 
\end{thm}
\begin{thm}[Uniforme Measure]
	There exists a unique $ \mathbb{P}$ measure on $( [ 0,1]^{n}, \mathcal{F}_E) $ with the same property.
\end{thm}
Both $\lambda$ and $ \mathbb{P}$ are shift-invariant in fact only shift invariant measures on $ \mathbb{R}$ ( up to a constant) 
\begin{proof}
	Consider the case of $ ( \mathbb{R}^n, \mathcal{F}_E) $ and $ f_r: x \to x+\tau,\tau \in \mathbb{R}^n$.
	\begin{itemize}
	\item $f_r$ continuous $\Rightarrow$ measurable
	\item $\tilde { \mathbb{P}} ( A) = \mathbb{P}( f^{-1}( A) ) $ is a probability measure
	\item All boxes have the same measure
	\end{itemize}
	
\end{proof}
\subsection{Probability measures on $ ( \mathbb{R}, \mathcal{F}_E) $ }
We saw that we can put a uniform measure on $ [ 0,1] $.\\
All probability measures on $ ( \mathbb{R}, \mathcal{F}_E) $ 
\begin{enumerate}
\item $ \mathbb{P}: \mathcal{F}_E\to [ 0,1] $ 
\item These are actually only characterized by $ \mathbb{P}( ( - \infty,x ) ) $ 
\end{enumerate}
\begin{defn}[Cumulative distribution function]
	$F: \mathbb{R}\to [ 0,1] $ is called a c.d.f if
	\begin{itemize}
	\item $F$ is non-decreasing
	\item $F( x_n) \to 0 $ then $ x_n \to - \infty $ 
	\item $F( x_n) \to 1$ if $x_n \to 1$ 
	\item $F$ is right-continuous.
	\end{itemize}
	
\end{defn}
\begin{thm}
	Given a probability measure $ \mathbb{P}$ on $ ( \mathbb{R}, \mathcal{F}_E) $ , then $f( x) \coloneqq \mathbb{P}( ( - \infty ,x) ) $ is a c.d.f\\
	Given a c.d.f, there exists a  unique probability measure s.t. $ \mathbb{P}( - \infty ,x) = F( x) $ 
\end{thm}
\begin{proof}
	Given $ \mathbb{P}$ on $ ( \mathbb{R}, \mathcal{F}_E) $.\\
	Let's show that $F( x) = \mathbb{P}( ( - \infty , x) ) $ is a c.d.f.
	\begin{itemize}
		\item $x<y\quad F( x) = \mathbb{P}( ( - \infty ,x) ) \leq \mathbb{P}(  - \infty ,y) =F( y) $
		\item $x_n\to - \infty \quad F( x_n) = \mathbb{P}( - \infty ,x_n) \to \mathbb{P}(  \bigcap_{n} ( - \infty , x_n) ) =0 $  
		\item $x_n \to \infty \Rightarrow F( x_n ) \to 1$ is similar
		\item Also for right continuous $x_n \to x$ , we have that $ [ x_n, \infty ) \subset [ x_{n+1} , \infty )  $ 
	\end{itemize}
How do we construct $ \mathbb{P}$ given $F$ ?\\
Trick using push-forward measure.\\
Define $f: ( 0,1) \to \mathbb{R}$, define
\[ 
	f( x) = \inf_{y \in \mathbb{R}} \left\{ F( y) \geq x \right\} 
\]
Define $ \mathbb{P}( A) \coloneqq \mathbb{P}_U ( f^{-1}( A) )\forall A \in \mathcal{F}_E $ 
Why is $f$ measurable?\\
If $f$ is increasing $\Rightarrow$ $f$ is measurable

	
\end{proof}





\end{document}	
