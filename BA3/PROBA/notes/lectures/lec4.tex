\documentclass[../main.tex]{subfiles}
\begin{document}
\lecture{4}{Wed 13 Oct}{...}
Each c.d.f gives rise to a unique $ \mathbb{P}$.\\
A priori $ \mathbb{P}_1 = \mathbb{P}_2$ means $\forall F \in \mathcal{F}_E \mathbb{P}_1( F) = \mathbb{P}_2( F)  $.\\
We show that it suffices to show that $ \mathbb{P}_1(  ( - \infty ,x] )= \mathbb{P}_2 ( ( - \infty ,x] ) \forall x \in \mathbb{R} $.
\begin{lemma}
	Given $ ( \mathbb{R}, \mathcal{F}_E, \mathbb{P}) $ then $ \forall B \in \mathcal{F}_E, \forall \epsilon>0$ one can find disjoint intervals $I_1,\ldots, I_n$ s.t. $ \mathbb{P}( B \Delta ( I_1\cup \ldots\cup I_n) ) < \epsilon$ 
\end{lemma}
\begin{proof}
Consider the collection $H$ of all subsets $H \in \mathcal{F}_E$ s.t. the property above holds.\\
We know that $H$ contains all intervalls, hence $\sigma( H) = \mathcal{F}_E$.\\
So we only need to show that $H$ is a $\sigma-$algebra
\begin{enumerate}
	\item $\emptyset \in H$ : Know that $\forall x ( - \infty , x] \in H$ 
	\item If $B \in H \Rightarrow B^{C}\in H$.\\
		Given $\epsilon>0,$ choose $I_1,\ldots, I_n$ s.t. $ \mathbb{P}( B\Delta ( I_1\cup\ldots) )< \epsilon $, but $ ( B\Delta A)= B^{C}\Delta A^{C}$ , hence
		\[ 
			\mathbb{P}( B^{C}\Delta (I_1\cup \ldots ) ) < \epsilon
		\]
		
	\item $H_1,\ldots\in H$, we want $\bigcup_i H_i \in H$ 
		$\exists n \in \mathbb{N}$ 
		\begin{align*}
			\mathbb{P}(  ( \bigcup_{i=0}^{m} H_i) \Delta (\bigcup_{i} H_i ) ) < \frac{\epsilon}{2}		
		\end{align*}
	$\forall i = 1,\ldots,m$ , we have disjoint $I_{i,1} ,\ldots, I_{i,m_i}$ s.t.
	\[ 
		\mathbb{P}( H_i \Delta( I_{i,1} \cup\ldots) ) < \frac{\epsilon}{2m}
	\]
	Now use that
	\[ 
		( \bigcup_{i=1}^{m}H_i) \Delta ( \bigcup_{i=1} ^{m}\bigcup_{j=1}^{m_i} I_{i,j}  ) \subseteq \bigcup_{i=1} ^{m} ( H_i\Delta \bigcup_{j=1} ^{m_i}I_{i,j} ) 
	\]
	Finally, we can write a finite union of disjoint intervals
\end{enumerate}

\end{proof}
\begin{crly}
	$\mathbb{P}_1, \mathbb{P}_2$ probability measure on $ ( \mathbb{R}, \mathcal{F}_E) $, then $ \mathbb{P}_1= \mathbb{P}_2$ as soon as
	\[ 
		\mathbb{P}_1( ( - \infty, x] ) = \mathbb{P}_2 ( ( - \infty , x] ) 
	\]
	or
	\[ 
		\mathbb{P}_1 ( x,y) = \mathbb{P}_2 ( x,y) 
	\]
	
\end{crly}
\begin{proof}
	Notice $ ( - \infty , x) $ can be written as
	\[ 
		( - \infty , x) = ( \bigcup_n ( x,x+n)  )^{C}
	\]
	So it suffices to prove the first point.\\
	Observe, for all intervalls $ \mathbb{P}_1 ( I) = \mathbb{P}_2 ( I) $ since
	\[ 
		\mathbb{P}_i ( y,x) = \mathbb{P}_i ( - \infty , x) - \mathbb{P}_i ( - \infty , y) 
	\]
	The condition holds for $B$ if $\forall \epsilon>0$, we can pick $I_1,\ldots, I_n$ s.t.
	\[ 
		\mathbb{P}_1( B\Delta ( I_1\cup\ldots) ) < \epsilon
	\]
	and 
	\[ 
		\mathbb{P}_2( B\Delta( I_1\cup \ldots) ) <\epsilon
	\]
	So we need to check again that this is  a $\sigma-$ algebra and we are done.\\
	Now we can conclude that
	\[ 
		| \mathbb{P}_1( B) -\mathbb{P}_1( I_1\cup \ldots) | = | \mathbb{P}_1( B) - \mathbb{P}_2 (I_1\cup\ldots	 ) 	| <\epsilon
	\]
	and
	\begin{align*}
	| \mathbb{P}_2( B) -\mathbb{P}_1( I_1\cup \ldots) | = | \mathbb{P}_2( B) - \mathbb{P}_2 (I_1\cup\ldots	 ) 	| <\epsilon
	\end{align*}
\end{proof}
An abstract uniqueness result follows from a similar strategy.
\begin{thm}[Dynkin]
	$\mathbb{P}_1$ and $\mathbb{P}_2$ two probability measures on $( \Omega, \mathcal{F}) $, suppose $ \mathbb{P}_1( H) = \mathbb{P}_2( H) $ for all $H \in \mathcal{H} \subset \mathcal{F}$ and
	\begin{itemize}
		\item $\sigma( H) = \mathcal{F}$ 
		\item $H_1\in \mathcal{H}, H_2\in \mathcal{H} \Rightarrow H_1\cap H_2 \in \mathcal{H}$ 
	\end{itemize}
Then $ \mathbb{P}_1= \mathbb{P}_2$ 	
\end{thm}
\subsection{Probability measures on $ \mathbb{R}^n$ }
\begin{defn}[Joint c.d.f.]
	$F: \mathbb{R}^n\to [ 0,1] $ 
	\begin{itemize}
	\item $F$ non-decreasing in each coordinate
	\item $ F( x_1,\ldots,x_n) \to 1$ if all $x_i \to -\infty $ 
	\item right-continuous
	\end{itemize}
	
\end{defn}
\begin{thm}
	Joint c.d.f $\iff$  $\mathbb{P}$ on $ ( \mathbb{R}^n, \mathcal{F}_E) $ 	
\end{thm}
\subsection{Product probability measures on $ \mathbb{R}^n, \mathbb{R}^{ \mathbb{N}}$ }
\begin{itemize}
\item Related to independence
\item Natural mathematically
\end{itemize}
\subsection*{2 steps}
\begin{itemize}
\item product $\sigma$-algebra
\item product measure
\end{itemize}
\subsubsection{Product $\sigma$-algebra}
\begin{defn}[Product algebra ]
	Let $ ( \Omega_i, \mathcal{F}_i)_{i \geq 1} $ measurable spaces, then the product $\sigma$-algebra $ \mathcal{F}_\pi$ on $ \prod_i \Omega_i$ is the $\sigma$-algebra generated by sets $F= E_1\times \ldots\times E_n\times \Omega_{n+1} \times\ldots$, $E_i \in \mathcal{F}_i$ 
\end{defn}
\begin{rmq}
\begin{itemize}
\item Projections are measurable
\item In fact, product $\sigma$-algebra s.t. all projections are measurable
\end{itemize}

\end{rmq}
Notice on $ \mathbb{R}^{n}$, we now have two ways to define a $\sigma-$algebra.
\begin{itemize}
	\item Take $ ( \mathbb{R}^n, \tau_E) $ and induce a Borel $\sigma$-algebra
	\item Take $n$ copies of $ ( \mathbb{R}, \mathcal{F}_E) $ and consider $\mathcal{F}_\pi $ on $ \mathbb{R}^n$ 
\end{itemize}
\subsection{Product measures}
\begin{defn}
	Given $ ( \Omega_i, \mathcal{F}_i, \mathbb{P}_i) _{i \geq 1} $ 	probability spaces $ \mathbb{P}_\pi$ on $( \prod_i \Omega_i, \mathcal{F}_{\pi} ) $ is called the product measure of $ \mathbb{P}_i$.\\
	If $\forall n \geq 1$, all sets $E= E_1\times E_2 \times \ldots \times E_n\times \Omega_{n+1} \times \ldots$ 
	\[ 
		\mathbb{P}_{\pi} ( E) =\prod_{i=1}^{n}\mathbb{P}_i ( E_i) 
	\]
	
\end{defn}






\end{document}	
