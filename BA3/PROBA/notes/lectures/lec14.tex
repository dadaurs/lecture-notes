\documentclass[../main.tex]{subfiles}
\begin{document}
\lecture{14}{Wed 22 Dec}{Central limit theorem}
\begin{proof}[of SLLN]
	We suppose $ \mathbb{E} X_1^{4}< C$.\\
Recall that $ \mathbb{P}( |X_n- X| > \epsilon) \leq \frac{\epsilon }{n}$.\\
Instead of $ \mathbb{E}( S_n- \mathbb{E}X_1)^{2} $, we look at 
$ \mathbb{E}( S_n- \mathbb{E}X_1)^{4}$.\\
Define
\[ 
A_{ij}  = \mathbb{E}( ( X_i - \mathbb{E}X_i)^{2} ( X_j - \mathbb{E}X_j)^{2}) 
\]
We can now apply Cauchy-Schwarz to see that $A_{ij} \leq  C'$.\\
By Markov 
\[ 
\mathbb{P}( |S_n- \mathbb{E}X_i| > \epsilon) = \mathbb{P}( |S_n - \mathbb{E}X_1|^{4}> \epsilon^{4}) \leq \frac{c}{n^{2}}
\]
Hence the event $ \left\{ |S_n- \mathbb{E}X_1| > \epsilon \right\} $ can only happen finitely many times.


\end{proof}
\subsection{Central limit theorem}
\begin{thm}[Central limit theorem]
	Let $X_1,\ldots, $ be i.i.d. random variables with $ Var X_1= \sigma^{2}$, then
	\[ 
	\frac{ \sum_{i=1}^{ n}( X_i - \mathbb{E}X_i) }{ \sqrt{n} } \to \mathcal{N}( 0, \sigma^{2}) 
	\]
	
\end{thm}
Observe that we can assume $\mathbb{E}X_i = 0$ by adding a constant and $ Var X_1 =1$ by scaling.
\begin{proof}
The idea is that swapping $X_i$ for $ \mathcal{N}( 0,1) $ shouldn't influence the sum.\\
We will show we can swap all of them.\\
We prove it under the assumption that $ \mathbb{E}|X_i| ^{3}< \infty $.\\
\begin{propo}[Lindeberg Exchange principle]
Given a sequence of random variables $X_1, \ldots$ with 0 mean and variance 1, then
\[ 
S_n = \frac{ \sum_{i=1}^{ n}X_i}{ \sqrt{n} }, |g|,|g'|, |g"|, |g'''| < C \text{ then }    \mathbb{E}( g( S_n) - g( X) ) \to 0
\]

\end{propo}
Lindeberg exchange principle implies CLT.

\end{proof}

	
\end{document}	
