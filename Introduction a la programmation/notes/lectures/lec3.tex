\documentclass[../main.tex]{subfiles}
\begin{document}
\lecture{3}{Thu 24 Sep}{Programmation:Variables et Expressions}
\section{Etude de cas}

\textbf{ But }
\begin{itemize}
	\item Resumer ce qu'il faut retenir
	\item Etude de cas ( simple ici)
	\item Complements de cours
	\item Repondre a des questions
\end{itemize}
\subsection{Le langage C++}
Cest un language oriente-objet, compile et fortement type.\\
Parmi les avantages de C++:
\begin{itemize}
	\item tres utilise
	\item compile, applications efficaces
	\item typage fort
\end{itemize}
A retenir:
\begin{itemize}
	\item variable = representation interne d'une donnee du probleme traite
	Une entite conceptuelle representee dans le programme.
	\item Chaque valeur est typee
	\item Les principaux types sont intt, double et bool
\end{itemize}
\subsection{Valeurs litterales}
\begin{itemize}
	\item valeurs litterales de type int: 1,12, etc
	\item double : 1.23, ...
	\item valeurs litterales de type char: 'a', '!', ...
	\begin{lstlisting}
	char x('B'); \end{lstlisting}
	
\end{itemize}
\subsection{Expression}
\begin{itemize}
	\item un calcul a faire
	\item expression= combinaison d'expressions, valeurs, a l aide d'operateurs
	\item toute expression \textit{fait} quelque chose et \textit{vaut} quelquechose.

\end{itemize}

\subsection{\texttt{auto}}

En c++11, on peut laisse rle compilateur deviner le type d'une variable grave au mot-cle \texttt{auto}.\\
Par exemple:\\
\texttt{auto val(2);}\\
\texttt{auto j(2*i+5);}\\
Conseil: ne pas en abuser!\\
N'utilisez \texttt{auto} que dans les cas ``techniques'' qui seront vus plus tard.

\subsection{ \texttt{const} et \texttt{constexpr} }
Par defaut, les variables sont modifiables.\\
si une variable est initialisee avec \texttt{const} mais elle est modifiable par pointeur.\\
\texttt{constexpr} est pas modifiable, doit etre connue au moment de la compilation et n'est pas modifiable par pointeur.


\end{document}	
