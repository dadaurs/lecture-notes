\documentclass[../main.tex]{subfiles}
\begin{document}
\lecture{2}{Fri 18 Sep}{Calcul et Algorithmes I}
\chapter{Algorithmes}
\section{Formalisation d'algorithmes}
\begin{defn}[Algorithme]
	Un algorithme est une description abstraite des etapes conduisant a la solution d'un probleme
\end{defn}

\begin{exemple}
	\textbf{Probleme:}\\
	Trouver la valeur maximale dans une liste\\
	Une liste c'est un element du produit cartesien de $E^{n}$, $n$ taille de la liste.\\
	On pourrait ordonner la liste et retourner le dernier element.\\
	\begin{itemize}
		\item comparer les elements de la liste entre eux\\
		\item a chaque fois prendre le plus gd.\\
		\item au fur et a mesure
	\end{itemize}
	
\end{exemple}

\subsection{Methodologie}
But: Pour un probleme, trouver une sequence d'actions permettant de produire une solution acceptable en un temps raisonnable.
\begin{itemize}
	\item Bien identifier le probleme
		\begin{itemize}
			\item Quelle question?\\
			\item Quelles entrees?\\
			\item Quelles sorties?
		\end{itemize}
	\item Trouver un algorithme correct?\\
		verifier qu'il est effectivement correct, qu'il se termine dans tous les cas.\\
	\item trouver l'algorithme le plus efficace possible
\end{itemize}
\subsection{Qu'est-ce qu'un algorithme?}
Moyen pour un humain de representer la resolution d'un probleme donne\\
\begin{defn}[Algorithme]
	Composition d'un ensemble fini d'operations elementaires bien definies operant sur un nombre fini de donnees et effectuant un traitement bien defini:
	\begin{itemize}
		\item suite finie de regles a appliquer\\
		\item dans un ordre determine\\
		\item a un nombre fini de donnees
\end{itemize}
Un algorithme peut etre
\begin{itemize}
	\item sequentiel: operations s'executent en sequence\\
	\item parallele: certaines de ses operations s' executent en parallele: simultanement\\
	\item reparti: certaines de ses operations s executent sur plusieurs machines.
\end{itemize}
\end{defn}





\end{document}	
