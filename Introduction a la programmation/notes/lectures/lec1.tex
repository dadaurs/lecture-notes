\documentclass[../main.tex]{subfiles}
\begin{document}
\lecture{1}{Fri 18 Sep}{Partie theorie}
\chapter{Information, Calcul et Communication}
\section{Introduction}

Objectifs:\\
\begin{itemize}
	\item Vous convaincre de l'importance de ce cours\\
	\item insister sur le role de l'informatique
\end{itemize}

Presenter l'info en tant  que discipline scientifique.\\
Fonde sur 3 grands principes fondamentaux:
\begin{itemize}
	\item representation discrete du monde\\
	\item representation entachee d'erreurs, mais controlee\\
	\item variabilite de la difficulte des problemes et des solutions 
\end{itemize}
\subsection{A quoi ca sert?}
\begin{itemize}
	\item la simulation/ l'optimisation\\
	\item l'automatisation\\
	\item Gestions de donnees
\end{itemize}
\subsubsection{Calcul scientifique}
\begin{itemize}
	\item Utilisation: simulation de systemes complexes\\
	\item Exigences: grande puissance de calcul.
\end{itemize}
\subsubsection{La conduite de processus}
\begin{itemize}
	\item Utilisation: tres nombreuses applications: pilotage/surveillance de processus industriels\\
	\item Exigences: necessite d'un faible encombrement, consommation reduite, d un cout minimum et d'une grande fiabilite
\end{itemize}
\subsubsection{La gestion d'information}
\begin{itemize}
	\item Utilisation: gestion de systemes bancaires ou boursiers, commerce electronique, fichiers de police\\
	\item Exigences: importantes de capacite, traitement efficace, controle de processus
\end{itemize}

\section{Plan du cours}
\begin{enumerate}
	\item Fondement du calcul\\
	\item Calcul et algorithme\\
	\item Strategies de calcul\\
	\item theorie du calcul
\end{enumerate}
\begin{enumerate}
	\item Information et communication\\
	\item Echantillonage\\
	\item Reconstruction\\
	\item Entropie et information\\
	\item Compression des messages/donnees
\end{enumerate}
\begin{enumerate}
	\item Fondements des systems\\
	\item Architecture des ordinateurs\\
	\item Stockage et reseaux
\end{enumerate}
\begin{enumerate}
	\item Secureite informatique\\
	\item RSA\\
	\item Problemes sociaux
\end{enumerate}





\end{document}	
