\documentclass[../main.tex]{subfiles}
\begin{document}
\lecture{11}{Tue 20 Oct}{Espaces Vectoriels 2}
Soit $V$ un $K$-ev de dimension finie et $G= \left\{ e_1, \cdots, e_n \right\} $ une famille de vecteurs.
\begin{align*}
CL_G: K ^{d} \to V\\
( x_1, \cdots, x_d)  \to x_1.e_1+x_2e_2+\cdots + x_d e_d
\end{align*}
$CL_G$ est linéaire, suit du critere de combinaison linéaire.\\
Dire que $G$ est génératrice $\iff CL_G$  est surjective, donc que $CL_G ( K^{d}) = V$.
\subsection{Famille Libre}
\begin{defn}[Famille Libre]
	Soit $ \mathcal{F}= \left\{ e_1, \ldots, e_d \right\} \subset V $ et définissons
	\[ 
	CL_{ \mathcal{F}} : K^{d} \mapsto V
	\]
	une application pas forcément surjective.\\
	Si cette application est injective, alors la famille $ \mathcal{F}$ est libre.
\end{defn}
Comme $CL_{\mathcal{F}} $ est linéaire, $CL_{\mathcal{F}} $ est injective si et seulement si
\[ 
\ker CL_{\mathcal{F}} = \left\{ 0_V \right\} 
\]
Donc $\vec{x} = \left( x_1, \ldots, x_n\right) $ ssi
\[ 
\sum_i x_i e_i = 0
\]
\begin{defn}
	Un sous-ensemble fini $\mathcal{F} = \left\{ e_1, \ldots, e_d \right\} \subset V$ d'un espace vectoriel forme une famille libre de $V$ si et seulement si pour tous $x_1, \ldots, x_d \in K$ 
	\[ 
	\sum_i x_i e_i = 0_V \implies x_1= \cdots = x_d = 0
	\]
	Une famille $\mathcal{F}$ qui n'est pas libre est dite liée.
\end{defn}

\begin{propo}
Une famille à $d$ éléments $ \mathcal{F}= \left\{ e_1, \cdots, e_d \right\} \subset V$ est liée si et seulement si il existe $i \in \left\{ 1,\cdots, d \right\} $ tel que $e_i$ peut s'exprimer comme combinaison linéaire des autres éléments de $ \mathcal{F}$ 
\[ 
	e_i \in CL( \mathcal{F}\setminus \left\{ e_i \right\} ) = CL \left( e_j, j\neq i\right) 
\]
\end{propo}
\begin{proof}
	Supposons $\mathcal{F}$ est liée, il existe $ ( x_1,\cdots, x_d) \neq 0_V$ tel que 
	\[ 
	x_1 e_1 + \cdots + x_d e_d = 0_V
	\]
	un des $x_i \neq 0_K$ on peut suposer sans perte de géneralité que $x_d \neq 0$, donc
	\[ 
- x_d e_d = x_1e_1 + \cdots + x_{d-1} e_{d-1} 	
	\]
Or $x_d\neq 0$ donc innversible, on obtient donc
\[ 
x  ( x_{d} ) ^{-1} \in K \setminus \left\{ 0 \right\} 
\]

Donc
\begin{align*}
e_d = \frac{x_1}{-x_d} e_1 + \cdots  + \frac{x_{d-1} }{-x_d}e_d	
\end{align*}

Si $e_d \in CL \left( \left\{ e_1, \cdots, e_{d-1}  \right\}  \right) $, avec
avec
\[ 
e_d = y_1e_1 + \cdots + y_{d-1} e_{d-1} , y_i \in K
\]
Donc 
\[ 
0_V = y_1e_1 + \cdots + y_{d-1} e_{d-1}  -e_d \neq 0
\]

\end{proof}
\begin{thm}
	Soit $V$ un espace vectoriel non-nul de dimension $d$ et $\mathcal{F} = \left\{ v_1,\cdots,v_f \right\}\subset V $ une famille finie et libre, alors $f\leq d$
\end{thm}
\begin{proof}
Par récurrence sur d.\\
Supposons que l'espace est engendré par un élément $K$.
\[ 
d=1 \quad V= K.e,\quad e\neq 0
\]
Montrons que $ \mathcal{F} = \left\{ v_1, \cdots, v_f \right\} \subset V = K.e$ avec $ v_i = x_i .e \quad f\geq 2$
Comme $v_1 \neq v_2, x_1.e = v_1, x_2.e = v_2$, alors $x_1$ ou $x_2 \neq 0_k$.\\
Supposons $x_1\neq 0$, alors $v_2= x_2.e = \frac{x_2}{x_1}.x_1.e$ \\
Alors $\mathcal{F}$ est liée car $v_2$ est cl de $v_1$.\\
Dimesions $\dim V = d \geq 2$ et on suppose le résultat démontré en dimension $\leq d-1$.\\
Soit $\mathcal{F}= \left\{ v_1,\cdots, v_f \right\} \subset V$ avec $f \geq d+1$, on veut montrer que $\mathcal{F}$ est liée.\\
Soit $G= \left\{ e_1,\cdots, e_d \right\} $ une famille génératrice de $V$ pour $i= 1,\cdots, f$ 
\[ 
v_i = x_{i,1} e_1 + \cdots + x_{i,d} e_d
\]
avec $x_{i,j} j\leq d$ dans $K$.\\
Comme $f> d\geq 1$, il existe $x_{i,j} \neq 0_K$.\\
Quitte à permuter les $e_j$ et les $v_i$ on peut supposer que
\[ 
x_{f,d}  \neq 0_K
\]
On pose : $i\leq f$
\[ 
v'_i  := v_i - \left( \frac{ x_{i,d}  }{x_{f,d} }. v_f \right) 
\]

Si $i=f\quad v'_f = v_f - \frac{x_{f,d} }{x_{f,d} }v_f = 0_V$.\\
Posons 
\[ 
	v'_i = x'_{i,1} e_1 + \cdots + x'_{i,d-1} e_{d-1}  + ( x_{i,d} - \frac{x_{i,d} }{x_{f,d} }x_{f,d}  ) e_d
\]
On a construit $f-1$ vecteurs $\mathcal{F}' =\left\{ v'_1, v'_2,\cdots, v'_{f-1}  \right\} $ qui sont contenus dans l'espace vectoriel
\[ 
	V' = CL ( \left\{ e_1,\cdots, e_{d-1}  \right\} ) \subset V
\]
Or
\[ 
\dim V' \leq d-1 \text{ comme  } f-1 > d-1
\]
la famille $\mathcal{F}'$ est liée par hypothèse de récurrence.\\
Donc l'un des $v'_i$ est CL des autres $v'_{i'} i' \neq i, $ 
On peut supposer que c'est $v'_1$ 
\[ 
	v'_1 = y_2 v'_2 + \cdots + y_{f-1} v'_{f-1} 
\]
Or
\[ 
	v'_1 = v_1 - \frac{x_{1,d} }{x_{f,d} }v_f = y_2 ( v_2 - ( ) v_f) + \cdots + y_{d-1} ( v_{d-1} - (  ) .v_f) 
\]
Donc
\[ 
	v_1 = y_2 (  v_2 - ( ) v_f) + \cdots + y_{d-1} ( v_{d-1} - ( ) v_f)  + \frac{x_{1d} }{x_{f,d} }v_f 
\]
Donc $v_1$ est cl de $v_2, \cdots, v_f$, donc $\mathcal{F}$ est liée.



\end{proof}
\begin{crly}
$\dim K^{d}= d$
\end{crly}
\begin{proof}
On sait que pour $K^{d}$, la base canonique 
\[ 
B_{d} ^{0}= \left\{ e_1^{0}, \cdots, e_{d} ^{0} \right\}  
\]
est génératrice, donc $\dim K^{d}\leq d$.\\
Est libre : $d \leq \dim K^{d}$
\end{proof}
\subsection{Bases}
\begin{defn}
	Soit $V$ un espace vectoriel de dimnesion finie. Une famille $\mathcal{B} = \left\{ e_1, \cdots, e_d \right\} $ est une base de $V$ si l'une des contions equivalentes suivantes est vérifiée:
	\begin{enumerate}
	\item $\mathcal{B}$ est génératrice et libre
	\item L'application combinaison linéaire de $\mathcal{B}$ 
		\[ 
		CL_{\mathcal{B}} : K^{d} \to V
		\]
		est un isomorphisme.
	\item Pour tout $v\in V$ il existe un unique uplet $ ( x_1,\cdots, x_d) \in K^{d}$ tel que $v$ s'écrit sous la forme
		\[ 
		v= x_1e_1 + \cdots + x_d e_d
		\]
		
	\end{enumerate}
	
\end{defn}
\begin{rmq}
\[ 
|\mathcal{B}| = \dim V
\]
Une base à travers l'isomorphisme $CL_{\mathcal{B}} $ permet d'identifier un espace vectoriel abstrait $V$ avec un espace vectoriel concret $K^{d}$.

\end{rmq}
\begin{thm}
Soit $V$ un $K$-espace vectoriel de dimension $d = \dim V \geq 1$ alors $V$ possède une base $\mathcal{B}$ et on a donc un isomorphisme de $K$-ev
\[ 
V \simeq K ^{d}
\]
Plus précisément
\begin{enumerate}
\item Soit $ \mathcal{K} \subset V$  une famille génératrice alors $\mathcal{K}$ contient une base de $V$. Si de plus $ |\mathcal{K}| =d$, alors $ \mathcal{K}$ est une base.
\item Si $\mathcal{L}\subset V$ est lire alors $\mathcal{L}$ est contenue dans une base de $V$. Si $| \mathcal{L}| =d$, alors $\mathcal{L}$ est une base.
\end{enumerate}

\end{thm}
\begin{proof}
Soit $G$ une famille génératrice
\[ 
|G| = d' \geq d = \dim V
\]
Soit $B\subset G$ une famille génératrice de $G$ de taille minimale parmi les familles génératrices contenues dans $G$.\\
$\mathcal{B}$ est libre ( et est donc une base) 
\begin{align*}
G = \left\{ e_1, \cdots e_n \right\} 
\end{align*}
Supposons que $\mathcal{B}$ est liée, alors il existe $e_{|B|} $ qui est cl de $ \left\{ e_1, \cdots e_{|B|-1}  \right\} $ \\
Mais alors 
\[ 
	V= CL( \mathcal{B}) = CL( \left\{ e_1,\cdots, e_{|B|}  \right\} ) 
\]
mais comme $e_{|B|}$ est cl de $ \left\{ e_1, \cdots e_{|B|-1}   \right\} $
\[ 
	CL( \left\{ e_1,\cdots, e_{|B|-1}  \right\} ) \supset \left\{ e_1,\cdots, e_{|B|-1} , e_{|B|}   \right\} 
\]
Ca contredit la minimalité de $\mathcal{B}$. Donc $\mathcal{B}$ est libre et c'est une base.

\end{proof}










\end{document}	
