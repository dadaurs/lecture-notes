\documentclass[../main.tex]{subfiles}
\begin{document}
\lecture{1}{Mon 14 Sep}{Le language des Ensembles}
\chapter{Le language des Ensembles}

Le terme ``Algebre'' est derive du mot arabe al-jabr tire du tire d'un ouvrage.

Al-jabr signifie restoration.

Par exemple:
$2x -4 = 0$
Ce qu'on veut c'est trouver $x$.
Il faut donc transformer cette egalite en effectuant des operations de part et d'autres de l egalite.
\begin{align*}
	2x &= 4 &\vert +4\\
	x &= \frac{4}{2} = 2  &\vert :2
\end{align*}



Le but de l'ouvrage etait de resoudre des soucis administratifs, comment partager des champs etc.\\

Le but c'est d'introduire les espaces vectoriels a partir de 0.\\

Il y aura besoin d'introduire des groupes, anneaux, corps ( anneaux particuliers), modules et des ensembles.\\

Il faut donc commencer avec les objets les plus simples, i.e. les groupes.\\
Ici, on introduit de maniere moins rigoureuse qu'avec les systemes algebriques.

\section{Notations}

\begin{itemize}
	\item "Il existe" $\exists$, "Il existe un unique"  $\exists !$\\
	\item "Quelque soit", "Pour tout",  $\forall$\\
	\item "Implique",  $\Rightarrow$ \\
	\item "est equivalent" $\iff$, ou ``ssi''\\
	\item "sans perte de generalite" ``spdg'', ``wlog''\\
	\item "on peut supposer" ``ops, wma''
	\item "tel que" $\vert$
\end{itemize}

On ne va pas parler de logique mathematique dans ce cours, ni de definition rigoureuse des ensembles

\section{Ensembles}
Un ensemble est une collection d'elements ``appartenant'' a $E$
\[ 
e \in E
\]

\subsection{Exemples}
\begin{itemize}
	\item $\emptyset$ ne contient aucun element\\
	\item $ \mathbb{N} = \{ 0,1,2 \}$\\
	\item $\mathbb{Z} = \{-2,-1,0,1,2\}$\\
	\item $\mathbb{Q} = \{ \frac{p}{q} \vert p,q \in \mathbb{Z}, q \neq 0 \}$
	\item $\mathbb{R}$
\end{itemize}
\section{Sous-Ensembles}
Un sous-ensemble $A$ d'un ensemble $E$ est un ensemble t.q. tout element de $A$ appartient a $E$.
Formellement:
\[ 
	a \in A \Rightarrow a \in E
\]

L'ensemble vide est un sous-ensemble de $E$ pour tout ensemble $E$.\\
\[ 
\emptyset \subset E \forall E
\]


Deux ensembles $E$ et $F$ sont egaux si ils ont les memes elements, ssi
E est inclus dans F et F est inclus dans E ( regarder  notations)
\[ 
E \subset F \land F \subset E \Rightarrow E = F
\]

\section{$\mathcal{P}(E)$ l'ensemble des sous-ensembles}
C'est l'ensemble des $A \in E$ 

Remarque: L'ensemble de TOUS les ensembles n'est pas un ensemble et c'est du au paradoxe de Russell (Logicien anglais)
Si c'etait le cas, on considererait
\[ 
	Ncont = \{ \text{ L'ensemble des $E$ tq $E$ n'est pas contenu dans lui meme. } \}
\]
Cet ensemble Ncont est-il contenu dans lui meme ou pas?\\
\subsection{ Exercice }
Ncont est il contenu dans lui meme ou pas? \contra

\section{Operations sur les ensembles}
\[ 
A,B dans E
\]
\[ 
A \cup B = \{ e \in E tq e \in A \intertext{ ou bien } e \in B \}
\]
\[ 
	A \cap B = \{ e \in E \vert e \in A  e \in B \}
\]

Difference: $A-B$ ou $A\setminus B$
 \[ 
	 = \{A \in A \land \not\in B \}
\]
Difference symmetrique:
\[ 
	A \Delta B = ( A -B) \cup ( B-A)

\]
Si $A \cap B $


$ A_1, \ldots, A_{n} \subset E $ $n \geq 1$ \\


\section{$\times$: Produit cartesien}

Si $A$ et $B$ sont des ensembles

\[ 
	A\times B = \{ ( a,b), a \in A b \in B\}
\]
($\neq ( b,a)$)

On peut bien sur iterer
\[ 
	A_1 \times \ldots \times A_{n}  = \prod_{i=1}^{n} A_{i}  = \[ (a_1,a_2,\ldots,a_{n}  avec $a_i \in A_{i} $)
\]
On peut noter une grande reunion ainsi:
\begin{align*}
	A_1 \cup A_2 \cup \ldots \cup A_{n} &= A_1 \cup ( A_2 \cup \ldots \cup A_{n} )\\
					    &= \{ e \in E \vert \exists i \in \{1,\ldots,n\} \text{avec} e \in A_{i} \}\\
					    &= \bigcup_{i=1}^{n} A_{i} 
\end{align*}


$ ( a_1, \ldots, a_{n} ) \in A_1 \times \ldots \times A_{n} $ s'appelle $n$-uplet.

\section{Applications entre ensembles}
%\begin{definition}[Application lineaire]\index{Application lineaire}\label{def:application_lineaire}
	
Soient $X$ $Y$ deux ensembles.\\

Une application (fonction) $f$ est la donnee pour chaque element $x \in X$ (L'espace de depart) d'un element $f(x) \in Y$ (l'espace d'arrivee)
%\end{definition}

\[ 
f: X \rightarrow Y
\]
\subsection{Graphe}
Se donner une application
\[ 
	f: X \rightarrow Y
\]
equivaut a se donner un graphe $G$ (graphe de $f$)

\[ 
	G \subset X \times Y = \{(x,y) \vert x \in X y \in Y\}
\]
tq pour $x_0 \in X$ l'ensemble des elements du graphe $G$ de la forme $(x_0,y)$ possede exactement un element $(x_0,y_0)$.
$y_0= f(x_0)=$ l'image de $x_0$ par l'application $f$.

On associe simplement au premier element un autre element.




\section{Composition/Associativite}
Soient
\[ 
f: X \rightarrow Y
\]
\[ 
g: Y \rightarrow Z
\]

\begin{marginfigure}
    \incfig{schemacomposition}
    \caption{Schema de la composition de 2 applications}
    \label{fig:schemacomposition}
\end{marginfigure}
\begin{align*}
	g \circ f : X \longrightarrow Z \vert x \in X &\longrightarrow f(x) \in Y\\
						      &\longrightarrow g(f(x)) \in Z
\end{align*}

Cette application s'appelle la composee de $f$ et $g$.

\subsection{Associativite}
\[ 
f:X \longrightarrow Y
\]
\[ 
g:Y \longrightarrow Z
\]
\[ 
h:Z \longrightarrow W
\]
Alors
\[ 
	( g \circ f): X \longrightarrow Z \huge{ \circ } h: Z \longrightarrow W
\]
\[ 
	\Rightarrow h \circ ( g \circ f)
\]

\[ 
	f: X \longrightarrow Y \huge{\circ} h \circ g : Y \longrightarrow W
\]

On a que 

\begin{thm}[Composition de fonctions]\index{Composition de fonctions}\label{thm:composition_de_fonctions}
	
\[ 
	h \circ ( g \circ f) = ( h \circ g) \circ f = h \circ g \circ f
\]
\end{thm}
\begin{proof}
\[ 
	h \circ ( g \circ f) : x \longrightarrow h((g\circ f) ( x))
\]
\[ 
	= h(g(f(x))) \in W
\]
\[ 
	( h\circ g) \circ f : x \longrightarrow ( h\circ g) ( f(x))
\]
\[ 
	h(g(f(x))) \in W
\]



\end{proof}
\section{Image,Preimage}
\[ 
f: X \longrightarrow Y
\]
A l'application $f$ sont associes deux applications impliquant $\mathcal{P}(X), \mathcal{P}(Y)$.
\begin{itemize}
	\item $Im(f): \mathcal{P}(X) \longrightarrow \mathcal{P}(Y)$\\
		$A \subset X  \longrightarrow  Im(f)(A)=f(A)$\\
		C'est ce qu'on appelle l'image de $A$ par $f$ 
		\[ 
			= \{ f(a) \in Y \vert a \in A\} \subset Y \in \mathcal{P}(Y)
		\]
		L'image de $f$ $Im(f) := f(X) = \{f(x) \in Y \vert x \in X\}$

	\item Preimage de  $f$: $Preim(f)$:
		\begin{align*}
			Preim(f): \mathcal{P}(Y) &\longrightarrow \mathcal{P}(X)\\
			B &\longrightarrow Preim(f)(B) = f^{-1}(B)
			  &= \text{preimage de l'ensemble $B$ par $f$.}
		\end{align*}
		\[ 
			f^{-1}(B) = \{x \inn X \vert f(x) \in B\}
		\]
\end{itemize}

\textbf{\underline{Exemples}}\\

\[ 
	f_1(\{1,2\}) = \{2,4\}
\]

\[ 
	f_1^{-1}(\{1,2,3,4\}) = \{1,2,3,4\}
\]




















\end{document}	
