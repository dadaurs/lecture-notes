\documentclass[../main.tex]{subfiles}
\begin{document}
\lecture{25}{Tue 08 Dec}{Determinant d'une matrice}
\begin{crly}
Soit $\Lambda$ une forme multilineaire en $n$ variables sur $V$, alors
\[ 
	\Lambda_{sgn} = \sum_{\sigma\in S_n} sgn( \sigma) \Lambda_{|\sigma} 
\]
est alternee.
\end{crly}
\begin{thm}
	L'espace $Alt^{d}( V;K) $ est de dimension 1 exactement et on a 
	\[ 
		Alt^{d}( V;K) = K ( e_1^{*}\otimes \ldots \otimes e_d^{*}) _{sgn} 
	\]
	
\end{thm}
\begin{proof}
	Soit 
	\[ 
	\Lambda= e_1^{*}\otimes \ldots \otimes e_d^{*}
	\]
	et $\Lambda_{|sgn} $ la forme correspondante symmetrisee.\\
	Montrons qu'elle est non nulle.
	On montre deux methodes.
	 Calculons
		\begin{align*}
			\Lambda_{|sgn} ( e_1, \ldots, e_d) = \sum_{\sigma\in S_n} sgn( \sigma) e_{\sigma( 1) } ^{*}( e_1) \ldots e_{\sigma( d) } ^{*}( e_d) \\
			&= sgn( \id)
		\end{align*}
		Donc la forme est non-nulle.
	

\end{proof}
\begin{defn}
	La forme alternee $( e_1^{*}\otimes \ldots \otimes e_d^{*}) $ est appellee le determinant de $V$ relatif a la base $B= \left\{ e_1, \ldots, e_d \right\} $ et est notee $\det_B$. C'est l'unique forme lineaire alternee satisfaisant
	\[ 
		\Lambda( e_1, \ldots, e_d) = 1
	\]
	
\end{defn}
\begin{propo}
On a la formule suivante
\[ 
	det_B( v_1, \ldots, v_d) = \sum_{\sigma\in S_d} sgn( \sigma) \prod_{i=1}^{d}x_{i\sigma( 1) } = \sum_{\sigma \in S_d} x_{1\sigma( 1) } \ldots x_{d\sigma( d) } 
\]
 
\end{propo}
\begin{proof}
	 \begin{align*}
		 det_B( v_1, \ldots, v_d) 
		 = \sum_{\sigma\in S_d} sgn( \sigma) e_{\sigma( 1) } ^{*}\ldots
	\end{align*}
\end{proof}
Soit $\phi:  V \to V$ et $\phi^{*}( \Lambda) \in Alt^{n}( V;K) $
tel que
\[ 
	\phi^{*}( \Lambda) ( v_1, \ldots, v_n) = \Lambda( \phi( v_1) , \ldots) 
\]
\begin{defn}
Le determinant de $\phi$ est le scalaire verifiant
\[ 
	\phi^{*}( det_B) = det( \phi) det_B
\]

\end{defn}
\begin{thm}
	Soit $\phi:V \to V$ un endomorphisme. Pour tout $\Lambda\in Alt^{d}( V,K)$, on a
	\[ 
		\phi^{*}( \Lambda) = det( \phi) \Lambda
	\]
	En particulier $det( \phi) $ ne depend pas du choix de la base $B$.\\
	L'application $det$ a les proprietes suivantes
	\begin{enumerate}
	\item Homogeneite: soit $\lambda\in K$, alors
		\[ 
			det( \lambda\phi) = \lambda^{d}det( \phi) 
		\]
		
	\item Multiplicativite: on a
		\[ 
			det( \psi\circ\phi) = det( \phi) det( \psi) 
		\]
		
	\item Critere d'inversibilite: on a
		\[ 
			det( \phi) \neq 0 \iff \phi \in GL( V) 	
		\]
		
	\item Morphisme: L'application 
		\[ 
			det: GL( V)  \to K^{\times}
		\]
		est un morphisme de groupes. En particulier $det( Id_V) =1$.
	\end{enumerate}
	
\end{thm}
\begin{proof}
	Soit $\Lambda \in Alt^{d}( V;K) $, alors $\Lambda = \lambda det_B$.\\
	On a 
	\[ 
		\phi^{*}(\Lambda ) = \phi^{*}( \lambda det_B) = \lambda \phi^{*}( det_B) 
	\]
	
De meme
\begin{align*}
	\phi^{*}( \lambda det_B) ( v_1, \ldots, v_d ) = \lambda det( \phi( v_1) , \ldots) 
\end{align*}
\begin{enumerate}
\item 
	\begin{align*}
		&det( \lambda \phi) \\
		&= ( \lambda. \phi) ^{*}( \Lambda) ( v_1, \ldots) \\
		&= \lambda^{d} \phi^{*}( \Lambda) ( v_1, \ldots)\\
		&= \lambda^{d}det( \phi) \Lambda( v_1, \ldots) 
	\end{align*}

\item Soit $\Lambda \in Alt^{( d) }( V;K) $.
	\begin{align*}
		( \psi \circ \phi) ^{*}( \Lambda) = det( \psi \circ \phi) \Lambda\\
		= \Lambda( \psi\circ \phi(v_1 ) ,\ldots) \\
		= \psi^{*}( \Lambda) ( \phi( v_1) ,\ldots) \\
		= det \psi \Lambda( \phi( v_1) , \ldots) \\
		= det \psi det \phi \Lambda( v_1, \ldots) 
	\end{align*}
	
\item Si $\phi$ est inversible
	\[ 
	\phi \circ \phi^{-1}= \id
	\]
	Donc
	\[ 
	det \phi \circ \phi^{-1} = 1 
	\]
	Donc $det \phi \neq 0$.

\item Si $det \phi \neq 0$.\\
	On va montrer que si $\phi$ n'est pas inversible $det \phi = 0 $.\\
	Si  $ \left\{ \phi( e_1) , \ldots \right\} $ est liee, donc 
	\begin{align*}
		det_B ( \phi( e_1) , \ldots) = 0
	\end{align*}
	
\item Morphisme resulte du critere d'inversibilite et de multiplicativite.
	
\end{enumerate}

\end{proof}




	


\end{document}	
