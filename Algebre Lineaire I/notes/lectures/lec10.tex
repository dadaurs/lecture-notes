\documentclass[../main.tex]{subfiles}
\begin{document}
\lecture{10}{Mon 19 Oct}{EV}
\section{Espaces Vectoriels}

\begin{defn}[Espace Vectoriel]
	Soit $K$ un corps, in $K$-espace vectoriel $V$ est simplement un $K$-module.\\
	Les éléments de $V$ sont appelés vecteurs de $V$.

\end{defn}
\begin{exemple}
$\mathbb{Q}^{d}, \mathbb{R}^{d}, \mathbb{C}^{d}, d \geq 1$\\
Espaces de fonctions
\[ 
	\mathcal{F}(X;\mathbb{R}) \simeq \mathbb{R}^{X}
\]
Plus généralement, si $V$ est un $K$-ev
\[ 
	\mathcal{F}(X;V)=V^{X} \text{ est un $K$-ev } 
\]


\end{exemple}
\begin{defn}[Produit]
Si $V$ et $W$ sont des $K$-ev
\[ 
	V\times W = \left\{ ( v,w), v\in V, w \in W \right\} 
\]

\end{defn}
\begin{defn}
	Soit $V$ un $K$-espace vectoriel, un sous-espace vectoriel ( SEV) de $V$ est un sous-$K$ module $W \subset V$
\end{defn}
\begin{propo}[Critere de SEV]
Un sous-ensemble $U\subset V$ d'un $K$-ev est un sev
si
\[ 
\forall \lambda\in K, \vec{v}, \vec{v'} \in U \Rightarrow \lambda \vec{v} + \vec{v'} \in U
\]

\end{propo}
\begin{exemple}
\begin{itemize}
\item $ \left\{ 0_V \right\} \subset V$
\item $e \in V\quad K.e = \left\{ \lambda.e\quad \lambda \in K \right\} \subset V$ est un SEV.
\end{itemize}

\end{exemple}
\begin{defn}
Soient $V$ et $W$ deux $K$-espaces vectoriels, un morphisme $\phi:V\to W$ de $K$-modules est appelé une application $K$-linéaire.
\end{defn}
\begin{propo}[Critere d'application linéaire]
	Une application entre espaces vectoriels $\phi: V \to W$ est linéaire ssi
	\[ 
		\forall \lambda \in K, \vec{v}, \vec{v'} \in V, \phi(\lambda.\vec{v} +\vec{v'}) = \lambda\phi(\vec{v})+\phi(\vec{v'})
	\]
	

\end{propo}
\begin{proof}
C'est un cas particulier du critere de morphisme de modules.
\end{proof}
\begin{propo}
Le noyau et l'image d' une application linéaire est un sev
\end{propo}
\begin{proof}
C'est un cas particulier du critere de morphisme de modules.
\end{proof}
\begin{propo}
$\phi$ une application linéaire. $\phi$ injective ssi
\[ 
\ker \phi = \left\{ 0 \right\} 
\]
\end{propo}
\begin{defn}[Notations]
On notera
\[ 
	Hom_{K-ev} ( V,W), Isom_{K-ev } ( V,W), Aut_{K-ev } ( V)= GL(V)
\]
Les ensembles des applications bijectives.
\end{defn}

\begin{defn}
Une forme linéaire sur $V$ est une application linéaire a valeurs dans $K$ 
\[ 
l: V\mapsto K.
\]
On note l'ensemble des formes linéaires
\[ 
	V^{*} := End_{K-ev } ( V,K)
\]
C'est le dual.
\end{defn}
\begin{propo}
Soit $l:V \mapsto K$, si $l\neq 0_K$, alors $l$ est surjective
\[ 
	l(V)=K.
\]

\end{propo}
\begin{proof}
Comme $l\neq 0_K$, il existe 
\[ 
	v\in V \text{ tel que } l(v)=x \neq 0_K
\]
Soit $y\in K$, on cherche $v'$ tel que $l(v')=y$.\\
Comme $x\neq 0_K$, $x$ est inversible d'inverse
$ x^{-1} $
soit $v'= y.x^{-1}.v$, on a
\[ 
	l(v')=l(y.x^{-1}.v) = y.x^{-1}.l(v)= y.x^{-1}.x=y
\]


\end{proof}
\subsection{Familles génératrices}
\begin{defn}
Soit $ \mathcal{F}\subset V$ un sous-ensemble, on note
\[ 
	\eng{ \mathcal{F}} = Vect( \mathcal{F}) = CL_K( \mathcal{F})
\]
le sous-espace vectoriel engendre par $ \mathcal{F}$.
\end{defn}
\begin{defn}
Soient $X,Y\subset V$ des sev d' un espace vectoriels. Leur somme $X+Y\subset V$ est
\[ 
	X+Y = \eng{X\cup Y} \subset V
\]
est le sev engendré par les vecteurs de $X$ et de $Y$.
\end{defn}
\begin{lemma}
On a 
\[ 
X+Y = \left\{ x+y, x\in X, y \in Y \right\} 
\]

\end{lemma}
\begin{proof}
Il suffit de montrer que $ \left\{ x+y ,    x\in X, y\in Y \right\} $ est un sev.\\
En effet, si c'est le cas, il contient $X,Y$, il contient donc $X\cup Y$ et donc il contient $ \eng{X\cup Y}= X+Y$.\\
De plus, comme $\eng{X\cup Y}$ contient tout élément $x\in X$ et tout élément $y\in Y$, il contient $x+y$ ( car c'est un sev )
\[ 
	\Rightarrow \eng{X\cup Y} = \left\{ x+y\quad x\in X, y \in Y \right\} 
\]
Soit $\lambda\in K, x+y $ et $ x'+y'\in \left\{ u+v\quad u\in X, v\in Y \right\} $.
\begin{align*}
\lambda(x+y) + ( x'+y')= \lambda x +\lambda y + x' + y'\\
= ( \lambda x + x') + ( \lambda y + y') \in \left\{ u+v, u \in X, v \in Y \right\} 
\end{align*}
	
\end{proof}
\begin{defn}[Notations]
	Si $X\cap Y$, on dit que $X$ et $Y$ sont en somme directe et on ecrit
	\[ 
	X \oplus Y \subset V
	\]
	pour leur somme.Si
	\[ 
	X \oplus Y =V
	\]
	on dit que $V$ est somme directe de $X$ et $Y$.

\end{defn}

\begin{propo}
Soit $X$ et $Y$ en somme directe. Soit $W=X\oplus Y$, alors $w \in W$ s'écrit comme combinaison linéaire unique de $x\in X$ et $y\in Y$
\end{propo}
\begin{proof}
Supposons $w=x+y= x'+y'$, alors
 \begin{align*}
\Rightarrow x+y=x'+y'\\
\Rightarrow X\ni x-x' = y'-y \in Y
\end{align*}
Donc $x-x'=y'-y =0$
\end{proof}

\begin{defn}[Famille génératrice]\index{Famille génératrice}\label{defn:famille_generatrice}
	Soit $V$ un $K$-ev. Un sous-ensemble $\mathcal{F}\in V$ est une famille génératrice si
	\[ 
		Vect( \mathcal{F}) = V
	\]
	ie. tout élément $v \in V$ peut s'écrire sous la forme d' une combinaison linéaire
	\[ 
	v= \sum_{i=1}^{ n}x_i e_i
	\]
	
\end{defn}

\begin{defn}[Espace vectoriel fini]\index{Espace vectoriel fini}\label{defn:espace_vectoriel_fini}
	Un $K$-espace vectoriel non-nul est dit de dimension finie si il est de type fini comme $K$-module: si il exist un ensemble $\mathcal{F}$ fini tel que
	\[ 
		V= Vect( \mathcal{F})
	\]
	La dimension de $V$ est définie comme le minimum du cardinal de toutes les familles génératrices finies de $V$ 
	\[ 
		\dim_K(V)= \min_{\mathcal{F} \text{ genératrice } } | \mathcal{F}|
	\]
	Par convention, la dimension de l'espace vectoriel nul $ \left\{ 0_V \right\} $ est
	\[ 
		\dim_K( \left\{ 0_K \right\} )=0
	\]
	On peut prendre la famille vide comme famille génératrice
\end{defn}
\begin{thm}
	Tout $K$-espace vectoriel de dimension finie est linre,c'est a dire isomorphe a $K^{d}$ pour un certain $d \geq 0$
\end{thm}
\begin{rmq}
	$d=\dim_K(V)$
\end{rmq}
\begin{rmq}
	On verra à la fin ce qui arrive aux espaces vectoriels qui ne sont pas de dimension finie.
\end{rmq}



















\end{document}	
