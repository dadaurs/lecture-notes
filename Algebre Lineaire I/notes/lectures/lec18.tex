\documentclass[../main.tex]{subfiles}
\begin{document}
\lecture{18}{Mon 16 Nov}{Corps des Nombres Complexes}
\section{Le Corps des Nombres Complexes}
Prenons $K= \mathbb{R}$ et $\mathcal{M} = M_2( \mathbb{R}) $. Soit $I$ la matrice
\[ 
I = 
\begin{pmatrix}
	0 & -1\\
	1 & 0
\end{pmatrix}
\]
\begin{defn}
L'espace des nombres complexes $\mathbb{C}$ est le sous-espace vectoriel engendre par $\id$ et $I$ 
\[ 
\mathbb{C} = \mathbb{R}.\id + \mathbb{R}.I
\]

\end{defn}
\begin{thm}
L'espace des nombres complexes est de dimension 2 et $ \left\{ \id, I \right\} $ en forme une base.\\
De plus $\mathbb{C}$ est une sous-algebre commutative de $M_2( \mathbb{R}) $ et est en fait un corps. Le corps des nombres reel s'injecte dans $\mathbb{C}$ via l'application
\[ 
x \in \mathbb{R} \mapsto x. \id \in \mathbb{C}
\]
( les nombres reels s'identifient aux matrices scalaires).
\end{thm}
\begin{proof}
La famille $ \left\{ \id, I \right\} $ est libre $\Rightarrow$ base de $\mathbb{C}$.\\
$\mathbb{C}$ est un sev de  $M_2( \mathbb{R}) $, pour montrer que $\mathbb{C}$ est un sous-anneau de $M_2( \mathbb{R}) $, il suffit de montrer que $\mathbb{C}$ est stable par produit.

\begin{rmq}
$I^{2} = - \id$ \\
En particulier, $I$ est inversible et
\[ 
	I^{-1} = -I = ^{t}I
\]

\end{rmq}
Soit  $z= x \id + y I$ et $z'=x'\id + y'I$.
\begin{align*}
	z.z' &= xx'\id + x'yI + xy'I - yy'\id\\
	     &= ( xx'-yy') \id + ( xy' + yx') I
\end{align*}
Donc $\mathbb{C}$ est un sous-anneau de $M_2( \mathbb{R}) $.\\
Montrons que $\mathbb{C}$ est un corps.\\
$\mathbb{C}$ est un corps: $0_2 = 0_{\mathbb{C}}\neq \id = 1_{\mathbb{C}}  $.\\
Il reste a montrer que $z \in \mathbb{C}\setminus 0_{\mathbb{C}} $ est inversible.
\begin{align*}
	z^{2} - 2xz &= ( x^{2}-y^{2}) \id + 2xy I - 2x( x\id+ yI) \\
		    &= -( x^{2} + ^{2}) \id
\end{align*}
Or $z \neq 0_2 \iff ( x,y) \neq ( 0,0) \iff x^{2}+y^{2}\neq 0$.\\
Et donc
\[ 
	\id = \frac{-1}{x^{2}+y^{2}}( z^{2}-2xz) 
\]
Donc
\[ 
	z^{-1}= \frac{-1}{x^{2}+y^{2}}( z-2x) 
\]
On trouve, en developpant que
\[ 
\frac{1}{x^{2}+y^{2}}  \quad ^{t}z = z^{-1}
\]
\end{proof}
\begin{rmq}
On peut identifier $\mathbb{R}$ avec l'algebre $\mathbb{R}\id$ des matrices scalaires
\[ 
x\in \mathbb{R}\to x \id
\]

\end{rmq}
\begin{defn}
Le reel $x$ est appele partie reelle de $z$ et le reel $y$ est la partie imaginaire de $z$ 
\[ 
x= \re z, y = \im z
\]
Dans la notation matricielle, la transposition $z \mapsto ^{t}z$ envoie
\[ 
x+iy \mapsto x-iy
\]
Avec la notation simplifiee, on note
\[ 
	\bar{z} = x-iy
\]
et s'appelle la conjugaison complexe de $z$. On a alors
\[ 
	z.\bar{z} = x^{2}+y^{2}\geq 0
\]
Le nombre $( z.\bar{z}) ^{\frac{1}{2}}$ se note
\[ 
	|z| = ( x^{2}+y^{2})^{\frac{1}{2}}
\]
et s'appelle le module de $z$. On a donc
\[ 
	z\bar{z} = |z|^{2}
\]

\end{defn}
\begin{propo}
On a les proprietes suivantes
\begin{enumerate}
\item Les applications partie reelle et imaginaire sont lineaires
\item La conjugaisonn complexe est un automorphisme du corps $\mathbb{C}$.\\
	De plus $\bar{\bar{z}} = z$ et on a 
	\[ 
		\bar{z} = z \iff z= x \in \mathbb{R}			
	\]

\item Le module $z \mapsto |z|$ est multiplicatif:
	\[ 
	|z.z'| = |z|.|z'|
	\]
	et on a 
	\[ 
	z=0 \iff |z| = 0
	\]
	
	
\end{enumerate}

\end{propo}
\begin{proof}
$\re,\im$ sont lineaires car ce sont les formes lineaires 1ere et 2eme coordonnees de $z\in \mathbb{C}$ dans la base $ \left\{ \id, I \right\} $.\\
De meme $z \mapsto \bar{z}$ est lineaire.\\
etc
\end{proof}
On remarque que $|\bullet|$ est un morphisme de groupe multiplicatif

\begin{propo}
On a un isomorphisme de groupes
\[ 
	pol: \mathbb{C}^{\times}\simeq \mathbb{R}_{>0} \times \mathbb{C}^{( 1) }
\]
donne par
\[ 
	z \in \mathbb{C}^{\times} \mapsto pol( z)  = ( |z|, \frac{z}{|z|}) 
\]

\end{propo}
\begin{proof}
On a que
\[ 
	\frac{z.z'}{|z.z'|}= ( \frac{z}{|z|}) ( \frac{z'}{|z'|}) 
\]
et 
\[ 
	| \frac{z}{|z|}| = \frac{( x^{2}+y^{2})^{\frac{1}{2}} }{|( x^{2}+y^{2}) ^{\frac{1}{2}}|}
\]

Donc l'application $pol$ est un morphisme.\\

\end{proof}









\end{document}	
