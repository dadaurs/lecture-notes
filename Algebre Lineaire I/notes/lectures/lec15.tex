\documentclass[../main.tex]{subfiles}
\begin{document}
\lecture{15}{Tue 03 Nov}{Matrices}
\subsection{Rang d'une Matrice}
On a déjà défini le rang d'une application linéaire.
\begin{defn}[Rang d'une matrice]\index{Rang d'une matrice}\label{defn:rang_d_une_matrice}
	Soit $M \in M_{d'\times d} ( K) $.\\
	Le rang de $M$ 
	\[ 
		rg( M) = \text{ dimension de l'espace engendré par les colonnes de $M$ dans l'espace $Col_{d} ( K) $ } 
	\]
	
\end{defn}
\begin{propo}
Soit $\phi: V \to W$ et $M= mat_{B'B} \phi$, alors
\[ 
	rg( M) = rg( \phi) = \dim \phi( V) 
\]

\end{propo}
\begin{proof}
	$M$ est formée de colonnes dont les coordonnées sont celle des $\phi( e_j) j \leq d$ dans la base $B'= \left\{ f_i, i \leq d' \right\} $.
\end{proof}
\begin{rmq}
	Le rang de $M$ est $\leq min( d,d')  $.
	\[ 
	\text{ rang de  } M= \text{ dimension d'un espace engendré par $d$ vecteurs } 
	\]
	Sa dimension sera toujours $\leq d$.\\
	Cet espace est contenu dans $Col_{d'} ( K) $ qui est de $\dim d'$
\end{rmq}
\subsection{Transposition}
Soient $\phi: V \to W$ et $\phi^{*}: W^{*}\to V^{*}$.
\begin{thm}
	Soit $( m_{ij} ) = Mat_{B,B'} ( \phi) , ( m_{ij} ^{*}) = Mat_{B',B} ( \phi^{*}) $. Alors on a
	\[ 
	m_{ij} = m_{ji}^{*}
	\]
	On dit que $mat( \phi^{*}) $ est la transposée de $mat( \phi) $ et on la note $^{t}mat( \phi) $
\end{thm}
\begin{defn}
	La transposition est l'application des matrices $d'\times d$ vers les matrices $d\times d'$ définie par
	\[ 
		^{t}\bullet: ( m_{ij})_{i\leq d', j \leq d}  \mapsto ( m_{ji} ) _{j\leq d, i \leq d'} 
	\]
	
\end{defn}
\begin{propo}
La transposition est 
\begin{enumerate}
\item Linéaire
\item Involutive: $^{t}( ^{t}M) = M$ 
\item Multiplicativité: pour  deux matrices $M$ et $N$, alors on a
	\[ 
		^{t}(M.N ) = ^{t}N.^{t}M
	\]
	
\end{enumerate}

\end{propo}
\begin{proof}
	Il suffit de montrer que
	\[ 
	\phi: U \mapsto V \quad \psi: V \to W
	\]
	et 
	\[ 
	\psi\circ\phi: U \mapsto W
	\]
	Alors 
	\[ 
		( \psi\circ\phi) ^{*} = \phi^{*}\circ \psi^{*}
	\]
On a
\[ 
	( \psi\circ\phi) ^{*}: l'' \to l'' \circ \psi\circ\phi = ( l''\circ\psi) \circ\phi = \phi^{*}( l''\circ\psi) = \phi^{*}( \psi^{*}( l'') ) 
\]

\end{proof}
\begin{propo}
	Soit $M \in M_{d'\times d} ( K) $ on a
	\[ 
		rg( M) = rg( ^{t}M) .
	\]
	Soit $\varphi \in Hom( V,W)$, on a
	\[ 
		rg( \phi) = rg( \phi^{*}) 	
	\]
	
\end{propo}
\begin{proof}
Soit une base $B= \left\{ e_1,\ldots, e_d \right\}\subset V $, alors on a
\[ 
	\phi( B) = \left\{ \phi( e_1) ,\ldots, \phi( e_d)  \right\} 
\]
engendre $\phi( V) $ si $\dim \phi( V) = r = rg \phi$ oon peut extraire de $\phi( B) $ une base de $\phi( V) $.\\
Supposons que cette base soit
\[ 
	\left\{ \phi( e_1)=f_1, \ldots , \phi( e_r) = f_r \right\} \subset W
\]
C'est une famille libre de $W$.\\
On peut la compléter pour former une base de $W$:
\[ 
B' = \left\{ f_1,\ldots, f_r,f_{r+1} ,\ldots, f_{d'}  \right\} 
\]
On regarde 
\[ 
B' = \left\{ f_1,\ldots, f_r, f_{r+1} ,\ldots, f_{d'}  \right\} 
\]
et on lui associe
\[ 
B'^{*}= \left\{ f_1^{*}, \ldots, f_d^{*} \right\} 
\]
On considère donc 
\[ 
\phi^{*}: W^{*}\to V^{*}
\]
On a
\[ 
	rg\phi^{*}= \dim \phi^{*}( W^{*}) = \dim \eng{ \phi^{*}( f_1^{*}) , \ldots}
\]
	On va montrer que 
	\[ 
		\left\{ \phi^{*}( f_1^{*}) , \ldots \right\} \subset V^{*}
	\]
est libre.\\
Soient $x_1, \ldots, x_r\in K$ tel que
\[ 
	x_1\phi^{*}( f_1^{*}) + \ldots = 0_K
\]
On a que $\forall j \leq d$ et $\forall e_j \in B$
\begin{align*}
&( x_1\phi^{*}+ \ldots) ( e_{j} ) \\
&= x_1 \phi^{*}( f_1^{*}) e_j + \ldots = x_j f_j^{*}( \phi( e_j) ) = x_{j}  f_j^{*}( f_j) = x_{j} 
\end{align*}
Et donc
\[ 
	rg( \phi^{*}) \leq rg( \phi) 
\]
Donc pour toute matrice $M$, on a $rg( ^{t}M) \geq rg( M) $.\\
Donc en particulier
\[ 
	rg( ^{t}( ^{t}( M) ) ) \geq rg( ^{t}( M) ) 
\]
Et donc 
\[ 
	rg( M) = rg( ^{t} M ) 
\]
Et donc
\[ 
	rg( \phi) = rg( ^{*}\phi) 
\]




\end{proof}
\subsection{Les matrices carrées}
On note
\[ 
	M_{d} ( K) = M_{d\times d} ( K) 
\]
On remarque que la multiplication des matrices induit sur les matrices carrées de taille $d$ une loi de composition interne.\\
Cette loi est
\begin{itemize}
\item distributive
\item associative
\item $Id_d$ est neutrre pour la multiplication
\item $0_d= 0_{d\times d} $ est absorbante
\end{itemize}
Donc
\[ 
	( M_{d} ( K) ,+,\cdot) 
\]
est un anneau non-commutatif.\\
Et de plus comme $M_d( K) $ est un $K$-ev, donc
\[ 
	M_d( K) 
\]
est une $K$-algebre.\\
Soit $V$ de $\dim d$, $B=$ base et $B'=B$ une base
\begin{align*}
	End( V) = Hom( V,W) &\to M_d( K) \\
	\phi&\to mat_{B,B} ( \phi) 
\end{align*}
$mat_{B,B} ( \bullet) $ est un isomorphisme de $K$-ev mais c'est également un isomorphisme d'anneaux.\\
On a
\[ 
	mat_{B,B} ( \psi\circ\phi) = mat_{B,B} ( \psi) \cdot mat_{BB} ( \phi) 
\]






	



\end{document}	
