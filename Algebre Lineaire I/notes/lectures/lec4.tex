\documentclass[../main.tex]{subfiles}
\begin{document}
\lecture{4}{Mon 28 Sep}{Groupes et Anneaux}
\begin{thm}
	Soit $A\subset G$ un ensemble,si $A=\emptyset$ alors $\eng{A}= \left\{ e_G \right\}$, sinon on pose
	\[ 
	A^{-1}= \left\{ g^{-1}, g\in A \right\} \subset G
	\]
	l'image de $A$ par l'inversion alors
	\[ 
		\eng{A} = \left\{ g_1\star\ldots\star g_n, g_i \in A \cup A^{-1} \right\} 
	\]
	En d'autres termes, $\eng{A}$ est l'ensemble des elements de $G$ qu'on peut former en multipliant ensemble des elements de $A$ et de son invers $A^{-1}$ de toutes les manieres possibles.
\end{thm}
\begin{proof}
	Pour montrer que c'est $\eng{A}$, on procede par double inclusion.\\
	$\supset$ : soit $H\subset G$ un ssgpe tq
	\[ 
	A\subset H \subset G
	\]
	Alors commme $H$ est stable par $\bullet^{-1}$	
	\[ 
	A^{-1}\subset H^{-1} =H
	\]
	Donc, $A\cup A^{-1}\subset H$ comme $H$ est stable par $\star$, si $ g_1,\ldots,g_n\in A\cup A^{-1}$
	Le produit $ g_1\star g_2\star\ldots\star g_n \in H$
	
	Donc $ \left\{ g_1\star g_2\star \ldots \star g_n, g_i \in A\cup A^{-1} \right\} \subset H $
	et donc $ \left\{ g_1\star g_2\star \ldots \star g_n, g_i \in A\cup A^{-1} \right\} \subset \bigcap_{A\subset H} H \subset \eng{A} $\\
	$\subset$ : il suffit de mq $ \left\{ \ldots \right\} $ et un sous groupe de $G$.
	En effet, $ \left\{ g_1\star \ldots\star g_n, n\geq 1, g_i \in A\cup A^{-1} \right\} \supset A $\\
	Critere de ss-groupe:\\
	a) Soit $g\in A \Rightarrow g^{-1} \in A^{-1}, g\star g^{-1} = e_G \in \left\{ g_1\star\ldots\star g_n, \ldots \right\} $\\
	b)Soit $g=g_1\star g_2\star\star\ldots \star g_n$ et$g'=g'_1\star g'_2\star\star\ldots \star g'_n$\\
	\[ 
	n,n'\geq 1, g_i, g'_j \in A \cup A^{-1}
	\]
	Alors
	\begin{align*}
	g\star g' = g_1\star\ldots\star g_n\star g'_1\ldots g'_n \in \left\{ \ldots \right\} 	
	\end{align*}
	c) soit $g=g_1\star\ldots\star g_n$comme ci-dessus
	\[ 
	g^{-1}= g_n^{-1}\star g_{n-1} ^{-1}\star\ldots\star g_1^{-1} \in \left\{ \ldots \right\} 
	\]
	$ \left\{ \ldots \right\} $ est un sousgroupe de $G$ contenant $A$ donc il contient $\eng{A}$.
\end{proof}
\subsection{Morphismes de Groupes}
\begin{defn}[Morphisme de Groupe]\index{Morphisme de Groupe}\label{defn:morphisme_de_groupe}
	Soient $(G,\star)$ et $(H,\bullet)$ deux groupes, un morphisme de groupes $\phi: G \to H$ est une application telle que
	\[ 
		\forall g,g'\in G, \phi(g\star g')=\phi(g) \bullet \phi(g')
	\]
\end{defn}
\begin{thm}
Soit $\phi: G\to H$ un morphisme de groupes alors
\begin{enumerate}
	\item $\phi(e_G)= e_H$
	\item $\forall g \in G, \phi(g^{-1})=\phi(g)^{-1}$
	\item $\forall g,g' \in G, \phi(g\star g') = \phi(g) \bullet\phi(g')$
\end{enumerate}
\end{thm}
\begin{proof}
Il suffit de demontrer 1 et 2, 3 est vrai par definition.\\
1)\\
Soit $g\in G,\phi(g)=\phi(g\star e_G)= \phi(g) \bullet \phi(e_G)$.\\
Donc $\phi(g) = \phi(g) \star\phi(e_G)$ et donc
\begin{align*}
h = h\bullet \phi(e_G)\\
h^{-1}\bullet h = h^{-1}\bullet h \bullet \phi(e_G)
\end{align*}
2)\\
 
\begin{align*}
	\phi(g) \bullet \phi(g)^{-1} &= e_H\\
	\phi(g) \bullet \phi(g^{-1}) &= \phi(g\star g^{-1})\\
				     &= \phi(e_G) = e_H
\end{align*}
On conclut en utilisant l'unicite de l'inverse
\[ 
	\phi(g^{-1})=\phi(g)^{-1}
\]



 

\end{proof}
\begin{defn}[Notations]
\begin{itemize}
	\item $Hom_{Gr} (G,H)$ l'ensemble des morphismes de groupe entre $G$ et $H$.
	\item $End_{Gr} (G) = Hom _{Gr} ( G,G)$ les endomorphismes du groupe $G$.
	\item $Isom_{Gr} ( G,H)$ l'ensemble des morphismes bijectifs
	\item $Aut_{Gr} ( G)= Isom_{Gr} ( G,G)$ l'ensembles des automorphismes du groupe $G$.
\end{itemize}
\end{defn}
\begin{exemple}
\begin{itemize}
\item 
	\begin{align*}
	e_H:
	\begin{cases}
	G\to H\\
	g \to e_h
	\end{cases}
	\end{align*}
	
\item Soit $g\in G$ 
	\begin{align*}
	\exp_G: 
	\begin{cases}
	\mathbb{Z} \to G\\
	n\to g^{n}
	\end{cases}
	\end{align*}
\end{itemize}
\item Si $G$ est commutatif note additivement
	\begin{align*}
	\bullet . g:
	\begin{cases}
	\mathbb{Z} \to G\\
	n \to n.g
	\end{cases}
	\end{align*}

\item Conjugaison dans un groupe: $(G, .)$
	\begin{align*}
	h \in G\\
	Ad_h: 
	\begin{cases}
	G \to G\\
	g\to h.g.h^{-1}
	\end{cases}
	\end{align*}
	\begin{proof}
	On veut montrer que $\forall g,g' \in G$ 
	\[ 
		Ad_h(g.g') = Ad_h(g).Ad_h(g')
	\]
	\begin{align*}
		Ad_h(g).Ad_h(g') &= (h.g.h^{-1} ). ( h.g.h^{-1})\\
				 &= h.g.h^{-1}.h.g'.h^{-1}\\
				 &= h.g.e_G.g'.h^{-1}
				 &= h.g.g'.h^{-1}  = Ad_h(g.g')
	\end{align*}
	Terminologie:\\
	\[ 
		Ad_h(g) = h.g.h^{-1} 	
	\]
	Le conjugue de $g$ par $g$.
	
	
	
	\end{proof}
\end{exemple}
\begin{rmq}
$Ad_h:G\to G$ est bijectif.
$Ad_h$ admet une application reciproque qui est $Ad_h^{-1}$
\end{rmq}
\begin{proof}
$Ad_{h^{-1}} \circ Ad_h  ?= Id_G$\\
$ Ad_h \circ Ad_{h^{-1}}  ?= Id_G$\\
Il suffit de montrer le premier.
\begin{align*}
	Ad_{h^{-1}} \circ Ad_h(g) &= h^{-1}.(h.g.h^{-1}).h\\
&= h^{-1}.h.g.h^{-1}.h\\
=g = Id_G(g)
\end{align*}
car $ ( h^{-1}) ^{-1}=h$
\end{proof}
$\forall h \in G$, 
\[ 
	Ad_h \in Aut_{Gr} ( G)
\]
\begin{propo}
	Soient $(G,\star), ( H,\ast), ( K,\bullet ))$ des groupes et $\phi:G\to H$ et $\psi: H\to K$ des morphismes de groupes alors la composee $\psi \circ \phi: G\to K$ est un morphisme de groupes
\end{propo}
\begin{proof}
On veut montrer que 
\[ 
	\psi \circ \phi ( g \star g') =? \psi \circ \phi(g) \bullet \psi \circ(g')
\]
on a:
\begin{align*}
	\psi \circ \phi(g\star g') &= \psi ( \phi(g\star g'))\\
				   &= \psi ( \phi(g) \ast \phi(g'))\\
				   &=\psi(\phi(g)) \bullet \psi(\phi(g'))
\end{align*}

\end{proof}
\begin{propo}
Soit $\phi: G \to H$ un morphisme de groupe bijectif alors l'application reciproque $\phi^{-1}$ est un morphisme bijectif.
\end{propo}
\begin{proof}
	Soit $\phi: G \to H$ un morphisme de groupe bijectif ( en tant qu'application), on veut montrer que $\phi^{-1}:H \to G$ verifie
	\[ 
		\phi^{-1}(h\star h') =? \phi^{-1}(h) \ast \phi^{-1}(h'), \forall h,h' \in H
	\]
	On calcule
	\begin{align*}
		\phi ( \phi^{-1}(h) \ast \phi^{-1}(h')) &= \phi(\phi^{-1}(h)) \star \phi(\phi^{-1}(h'))\\
							&= h \star h'\\
							\Rightarrow \phi^{-1}(h) \ast \phi^{-1}(h')
	\end{align*}
	est un antecedent de $h\star h'$ mais le seul antecedent de  $h\star h'$ c'est $\phi^{-1}(h\star h')$\\
	$\Rightarrow \phi^{-1}(h) \ast \phi^{-1}(h') = \phi^{-1}(h\star h')$ 
	
	

\end{proof}
\begin{defn}[Groupes Isomorphes]\index{Groupes Isomorphes}\label{defn:groupes_isomorphes}
	Soient $G$ et $H$ deux groupes si
	\[ 
		Isom_{gr} ( G,H) \neq \emptyset
	\]
	On dit que $G$ et $H$ sont isomorphes ( comme groupes)
	\[ 
	G \simeq _{Gr} H
	\]
	et si $Isom_{gr} ( G.H) \neq \emptyset$ alors
	$Isom_{Gr} ( H,G) \neq 0, H \simeq_{Gr} G$
\end{defn}
La relation ``etre isomorphe'' dans la categorie des groupes est une relation d'equivalence :
\begin{itemize}
	\item $G\simeq_{Gr} G$ ( $Isom_{Gr(G,G) \ni Id_G} $)
	\item Si $G\simeq _{Gr} H \Rightarrow  H \simeq_{Gr} G$
	\item Si $G \simeq_{Gr} H$ et $H\simeq_{Gr} K \Rightarrow  G \simeq_{Gr} K$
\end{itemize}
\begin{exemple}
Le groupe des automorphismes d'un groupe
\[ 
	Aut_{Gr} ( G) = Isom_{Gr} ( G,G) \subset Bij(G)
\]

\end{exemple}
\begin{thm}
	$Aut _{Gr} ( G)$ est un sous-groupe de $\left(Bij(G), \circ, Id_G, \bullet^{-1}\right)$
\end{thm}
\begin{proof}
	Si $\phi$ et $\psi \in Isom_{Gr} ( G,G)$, alors $\psi\circ\phi$ est un morphisme et $\psi\circ\phi$ est bijectif
	$\Rightarrow \in Isom_{Gr} ( G,G)$ \\
	Si $\phi \in Isom_{Gr} ( G,G) \cup Bij(G,G)$ alors $\phi^{-1}$ est un morphisme donc
	\[ 
		Isom_{Gr} ( G,G) = Aut_{Gr} ( G)
	\]
	
\end{proof}




	





		

		
	





\end{document}	
