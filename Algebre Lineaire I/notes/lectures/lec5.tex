\documentclass[../main.tex]{subfiles}
\begin{document}
\lecture{5}{Tue 29 Sep}{Noyau et Image}

\section{Noyau et Image}

\begin{propo}
	Soit $\phi \in Hom_{Gr} ( G,H) $ un morphisme de groupes.
	\begin{itemize}
		\item Soit $K\subset G$ un sous groupe alors $\phi(K) \subset H$ est un sous-groupe. En particulier l'imaged de $\phi$,
			\[ 
				Im(\phi) = \phi(G)
			\]
			
		\item Soit $L \subset H$ un sous-groupe de $H$, alors l'image inverse
			\[ 
				\phi^{-1}(L) = \left\{ g \in G, \phi(g) \in L \right\} \subset G
			\]
			est un sous-groupe de $G$. En particulier, $\phi^{-1}( \left\{ e_H \right\})$ est un sous-groupe
	\end{itemize}
	
\end{propo}
\begin{proof}
Soit $K \subset G$ un sous-groupe.\\
Soit
\[ 
	h,h' \in \phi(K)
\]
On veut montrer que $h\star h'^{-1}\in \phi(K)$.\\
Il existe $k,k' \in K$ tel que $\phi(k)=h, \phi(k')=h'$ 
\begin{align*}
	h \star h'^{-1} &= \phi(k) \star \phi(k')^{-1}\\
			&=\phi(k) \star \phi(k'^{-1})\\
			&= \phi(k\ast k'^{-1}), \text{     } k\ast k'^{-1} \in K
\end{align*}
car $K$ sous-groupe.\\
\[ 
	h \star h'^{-1} \in \phi(K)
\]
Soit $L \subset H$ un sous-groupe, on veut montrer que
\[ 
	\phi^{-1}(L) \subset G
\]
est un sous-groupe
Soient $g,g' \in \phi^{-1}(L)$, alors $\phi(g) = h \in L, \phi(g') = h' \in L$\\
 \[ 
	 g\star g'^{-1} \in \phi^{-1}(L)?
\]
on a
\begin{align*}
	\phi(g \star g'^{-1}) &= \phi(g) \ast \phi(g')^{-1}\\
			      &=h \ast h'^{-1} \in L \text{ car $L$ sous-groupe } 
\end{align*}
\end{proof}
\begin{defn}
	Le sous-groupe $\phi^{-1}( \left\{ e_H \right\} )$ s'appelle le noyau de $\phi$ et est note
	\[ 
		\ker(\phi) = \phi^{-1}( \left\{ e_H \right\} ) = \left\{ g\in G, \phi(g) = e_H \right\} 
	\]
	L'importance du noyau vient du fait qu'il permet de tester facilement si un morphisme est injectif.
\end{defn}
\begin{thm}[Critere d'injectivite]\index{Critere d'injectivite}\label{thm:critere_d_injectivite}
	Soit $\phi \in Hom_{Gr} ( G,H)$ un morphisme de groupes alors les proprietes suivantes sont equivalentes
	\begin{itemize}
	\item $\phi$ est injectif
	\item $\ker(\phi) = \left\{ e_G \right\} $
	\end{itemize}
	
\end{thm}
\begin{proof}
	$1 \rightarrow 2$\\
	si $\phi $ est injectif, l'image reciproque de $ \left\{ e_H \right\} $ possede au plus un seul element. Mais comme $\phi$ est un morphisme $\phi(E_G) = e_H \Rightarrow \phi^{-1}( \left\{ e_H \right\} ) = \left\{ e_G \right\} $\\
	$2 \rightarrow 1$\\
	On se donneun $h \in H$ et on veut montrer que $\phi^{-1}( \left\{ h \right\} ) = \left\{ g \in G, \phi(g) =h \right\} $ n'a pas plus d'un element.\\
	Si $\phi^{-1}( \left\{ h \right\} ) = \emptyset $ OK\\
	Si $\phi^{-1}( \left\{ h \right\} ) \neq \emptyset$, soient $g,g' \in \phi^{-1}( \left\{ h \right\} )$ on veut montrer que $g=?g'$.\\
	Par definition,  $\phi(g) = \phi(g') = h$ 
	\begin{align*}
		\phi(g) \ast \phi(g')^{-1} &= e_H\\
				      &= \phi(g \ast g'^{-1}) \text{ car $\phi$ morphisme } 
	\end{align*}
	Donc, $g \ast g'^{-1} \in \ker(\phi) = \left\{ e_G \right\} $,
	\[ 
	\Rightarrow  g \ast g'^{-1} = e_G \Rightarrow g = g'
	\]
	
	
\end{proof}

\begin{exemple}
Ordre d'un element\\
Soit $g \in G$ groupe
\[ 
\exp_g: \mathbb{Z} \to G\\
n \in (  \mathbb{Z}, +) \to g^{n} \in G
\]
est un morphisme de groupes.
\[ 
 \ker(\exp_g) \subset \mathbb{Z}
 q \cdot \mathbb{Z}, q \in \mathbb{Z}
\]
Si $q=0$, $\ker(\exp_q) = \left\{ 0 \right\} $
\begin{align*}
	\Rightarrow \mathbb{Z} &\to G\\
	n &\to g^{n} \text{ est injective } 
\end{align*}
$\mathbb{Z}$ est isomorphe a $g^{\mathbb{Z}} (\mathbb{Z} \simeq g^{\mathbb{Z}})$
\[ 
G \supset g^{\mathbb{Z}} \simeq \mathbb{Z}
\]
donc $g$ est d'ordre infini.\\
Si $q>0$, alors 
\[ 
g^{\mathbb{Z}} = \left\{ g^{0} = e_G, g, g^{2}, \ldots, g^{q-1} \right\} 
\]
est un sous-groupe de cardinal $q$ ( a demontrer en exercice)
et donc $G$ contient un sous-groupe d'ordre $q$ 
\[ 
	q := \text{ ordre de $g$ } = ord(g)
\]
q est le plus petit entier $>0$ tel que
\[ 
g^{q} = e_G
\]




\end{exemple}
\begin{exemple}[Conjugaison]
$G \ni h$ 
\begin{align*}
Ad_h: g \to h .g .h^{-1}
\end{align*}

On a montrer que $Ad_h \in Aut_{Gr} ( G)$

\end{exemple}
On considere l'application
\[ 
	h \in G \to Ad_h \in Aut_{Gr} ( G)
\]
Cette application est un morphisme de groupes:\\
On doit verifier que: $\forall h,h' \in G$ 
\[ 
	Ad_{h.h'} =Ad_{h} \circ Ad_{h'}
\]
On veut montrer que pour tout $g\in G$ 
\[ 
	Ad_{h.h'} = Ad_h(Ad_{h'} ( g))
\]
\begin{align*}
	h.h'.g.(h.h')^{-1} &=  h.h'.g .h'^{-1} .h^{-1}\\
			   &= h.(h'.g.h'^{-1}).h^{-1}\\
			   &= Ad_h(Ad_{h'} ( g))\\
	\ker ( Ad ) &= \left\{ h \in G \vert Ad_h = Id_G \right\} \\
		    &= \left\{ h \in G \vert \forall g \in G Ad_h(g) = g \right\} \\
		    &= \left\{ h \in G \vert \forall g \in G, h.g.h^{-1}=g\right\}\\
	h.g.h^{-1} =g & \iff h.g= g.h
\end{align*}
On dit que $h$ commute avec $g$.
\begin{align*}
	\ker(Ad) &= \left\{ \text{ l'ensemble des $h$ dans $G$ qui commutent avec tous les elements de de $G$ }  \right\} \\
&= \text{ Centre de  } G\\
&= Z(G) = Z_{G} 
\end{align*}
$Z_G$ est un groupe commutatif de $G$ 
\begin{exemple}[Translation]

Soit $h \in G$ la translation a gauche par $h$ 
\begin{align*}
t_h:
\begin{cases}
G \to G\\
g \to h.g
\end{cases}
\end{align*}
Attention $t_h$ n'est pas un morphisme de groupes, car l'element neutre ne va pas sur lui meme ( sauf si $h = e_G, t_h = t _{e_G} = Id_G $)\\
Par contre $t_h$ est bijective de reciproque $t_{h^{-1}} $\\

$ t_{\bullet} :h \in G \to t_h \in Bij(G)$ 
est un morphisme de groupe injectif, l'image s'appelle le groupe des translations ( a gauche) de $G$.\\
Donc $G \simeq t_G \subset Bij(G)$ \\
Tout groupe $G$ abstrait peut s'identifier ( est isomorphe) a un sous-groupe d'un groupe de bijections d'un ensemble.

\end{exemple}
\section{Anneaux}
\begin{defn}[Anneaux]\index{Anneaux}\label{defn:anneaux}
	Un anneau $(A,+,.,1_A)$  est la donee, d'un groupe commutatif $(A,+)$ ( note additivement)d'element neutre note $0_A$, d'une loi de composition interne ( dite de multiplication) 
	\begin{align*}
	\bullet.\bullet
	\begin{cases}
	A\times A \to A\\
	( a,b) \to a.b
	\end{cases}
	\end{align*}	
	et d'un element unite $1_A \in A$ ayant les proprietes suivantes
	\begin{enumerate}
	\item Associativite de la mutliplication
		\[ 
			\forall a,b,c \in A, ( a.b).c = a.(b.c) = a.b.c
		\]
		
	\item Distributivite
		\[ 
			\forall a,b,c \in A ( a+b).c = a.c + b.c, c.(a+b) = c.a + c.b
		\]
		
	\item Neutralite de l'unite
		\[ 
		\forall a \in A, a. 1_A = 1_A.a = a
		\]
		Un anneau est dit commutatif si de plus la multiplication est commutative
		\[ 
		\forall a,b \in A , a.b = b.a
		\]
		
	\end{enumerate}
\end{defn}
\begin{lemma}
Pour tout $a,b \in A$, on a
\[ 
0_A.a = a.0_A = 0_A
\]
On dit que l'element neutre de l'addition $0_A$ est absorbant. Pour l'oppose, on a
\[ 
	( -a).b = - ( a.b) = a.(-b)
\]

\end{lemma}
\begin{proof}
$\forall a \in A$\\
 \begin{align*}
	 a = a.1_A &= a.(1_A + 0_A)\\
		   &= a. 1_A + a.0_A\\
		 0_A = a.0_A
 \end{align*}

\end{proof}
\begin{exemple}
\begin{itemize}
\item L'anneau nul: $ \left\{ 0 \right\} $
\item $\mathbb{Z}, ( \mathbb{Q}, +, \bullet), ( \mathbb{R}, +,\bullet)$ 
\item $ \mathcal{F}(X,\mathbb{R})$ des fonctions d'un ensemble $X$ a valeurs dans $\mathbb{R}$.
	\[ 
		+: f+g : x \in X \to f(x) + g(x) = ( f+g)(x)
	\]
	\[ 
		0_{\mathcal{F}(X,\mathbb{R})} : x \to 0 \in \mathbb{R}
	\]
	\[ 
		1_{\mathcal{F}(X,\mathbb{R}): x \to 1 \in \mathbb{R}} 
	\]
	$ \left( \mathcal{F}(X,A), + , \bullet \right)$ est un anneau ( commutatif si $A$ commutatif) generalisation du cas des fonctions reelles

\item $\mathbb{R}[x] = \left\{ P(x) = a_0 + a_1x + \ldots + a_d x^{d}, a_0,a_1, \ldots, a_d \in \mathbb{R}, d \geq 0 \right\} $

\item $A[x] = \left\{ P(x) = a_0+ a_1x+ \ldots + a_d x^{d}, a_0, \ldots a_d \in A d \geq 0 \right\} $\\
	Anneau des polynomes a coefficients dans $A$.
\item $(M,+)$ un groupe commutatif
	\[ 
		End(M) = End_Gr(M) = Hom_{Gr} (M,M)
	\]
	\begin{align*}
		+: \psi, \phi \in End(M)\\
		\phi + \psi : m \to 
		 \phi(m) + \psi(m)
	\end{align*}
	Soient $\phi, \psi \in End(M)$ 
	\begin{align*}
		\phi \circ \psi \in End(M)\\
		0_{End(M)} : m \in M \to 0_M \in M\\
		1_{End(M)} : Id_M : m \in M \to m \in M	
	\end{align*}
	$\left( End(M), +, \circ, 0_M, Id_M\right)$ est un anneau
\end{itemize}
\end{exemple}









\end{document}	
