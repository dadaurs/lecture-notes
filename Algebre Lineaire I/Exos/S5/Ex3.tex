\documentclass[11pt, a4paper, twoside]{article}
\usepackage[utf8]{inputenc}
\usepackage[T1]{fontenc}
\usepackage[francais]{babel}
\usepackage{lmodern}

\usepackage{amsmath}
\usepackage{amssymb}
\usepackage{amsthm}

\begin{document}
\title{Série 5, Exercice 3}
\author{David Wiedemann}
\maketitle
\section*{1}
Supposons que $a^{n}\neq a^{m}, \forall n, m \in \mathbb{N}, n\neq m$, alors la suite définie par
\[ 
a_0,a_1,a_2,a_3,\ldots
\]
est contenue dans $A$, mais possède une infinité d'éléments distincts, ce qui contredit l'hypothèse que $A$ est fini.\\
Donc il existe, $m,n \in \mathbb{N} $, tel que $a_n=a_m$
\section*{2}
Si $a^{m}=1_A$, alors $a^{m}-1_A = 0_A$ et on a fini.\\
Supposons donc $n>m$ et $a^{n}=a^{m}\neq 1_A$.\\
On a que $n-m>0$, donc $a^{n-m}$ est bien défini.\\
\[ 
a_n = a^{n} = a^{n-m} a^{m} = a^{m} = a_m
\]
Donc 
\[ 
a^{n-m}a^{m} = a^{m} \Rightarrow a^{n-m} = 1_A.
\]
\section*{3}
Donc, pour tout $a \in A$, il existe$ n \in \mathbb{N}$ tel que
\[ 
a^{n}= 1_A
\]
Alors l'inverse de l'élément $a$ est l'élément
\[ 
a^{n-1}
\]
Donc l'ensemble $A\setminus \left\{ 0_A \right\} $ est stable par inversion, donc $A$ est un corps.
\end{document}
