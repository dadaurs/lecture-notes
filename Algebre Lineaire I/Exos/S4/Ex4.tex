\documentclass[11pt, a4paper, twoside]{article}
\usepackage[utf8]{inputenc}
\usepackage[T1]{fontenc}
\usepackage[francais]{babel}
\usepackage{lmodern}

\usepackage{amsmath}
\usepackage{amssymb}
\usepackage{amsthm}
\begin{document}
\title{Série 4}
\author{David Wiedemann}
\maketitle
\section*{1}
Il suffit de résoudre le système d'équations.\\
On résout par substitution.\\
On a donc que $ad = \pm 1$
\begin{align*}
\begin{cases}
ax+ cy = m\\
bx + dy = n
\end{cases}
\begin{cases}
adx + dcy = dm\\
-bcx - dcy = -cn
\end{cases}\\
\text{ On peut additionner les deux équations, en simplifiant on obtient }\\
x  = \frac{dm-cn}{\Delta} =\pm( dm - cn )
\end{align*}
De même, on obtient que
\begin{align*}
	\begin{cases}
	ax+ cy = m\\
	bx + dy = n
	\end{cases}
\begin{cases}
	-abx -bcy = -bm\\
	abx + ady = an	
\end{cases}
\text{ et donc } 
y = \frac{an-bm}{\Delta}= \pm( an-bm )
\end{align*}
Cette réponse marche aussi si $b=0$.

\section*{2}

Ce résultat suit directement du fait que $\mathbb{Z}$ est un anneaux, et donc stable par la multiplication, l'addition et la soustraction.\\
Par hypothèse, $a, b,c,d \in \mathbb{Z}$, donc $\pm (an-bm)$ et $\pm ( dm-cn) \in \mathbb{Z}$ et donc $(x,y) \in \mathbb{Z}^{2}$
\section*{3}
Celà suit directement de la partie 1, en effet, soit $(m,n) \in \mathbb{Z}^{2}$, alors, en résolvant le  système d'équations de la partie 1, on peut trouver des coefficients $x, y \in \mathbb{Z}$ tel que 
\[ 
	x(a,b) + y(c,d) = ( n,m)
\]
Donc tout élément de $(n,m) \in \mathbb{Z}^{2}$ s'exprime comme combinaison linéaire de $(a,b)$ et $(c,d)$, donc
$\langle \left\{ ( a,b), ( c,d) \right\} \rangle = \mathbb{Z}^{2} $

\section*{4}
Car $\Delta \neq 0$, on peut toujours encore admettre les solutions données dans la partie 1.\\
Supposons, par l'absurde que, 
\[ 
\langle \left\{ ( a,b), ( c,d) \right\} \rangle = \mathbb{Z}^{2} 
\]
et que $\Delta =ad - bc \neq \pm 1$.\\
Donc, $\forall (m,n) \in \mathbb{Z}^{2}$, il existe $x,y \in \mathbb{Z}$ tel que
$x\cdot ( a,b) + y \cdot ( c,d)= (m,n)$.\\
Par la partie 1, on a que 
\[ 
x = \frac{dm-cn}{\Delta} \text{ et }  y = \frac{an-bm}{\Delta}
\]
Pour que $x, y \in \mathbb{Z}$, il faut que $\Delta | dm -cn$ et que $\Delta | an-bm$ et ceci pour toutes les valeurs de $m$ et $n$, donc en particulier:
\begin{align*}
	\Delta &| d, \quad ( m=1, n=0)\\
	\Delta &| c, \quad ( m=0, n=-1)\\
	\Delta &| a, \quad ( m=0,n=1)\\
	\Delta &| b, \quad ( m=-1,n=0)
\end{align*}
Donc, il existe $a', b', c', d' \in \mathbb{Z}$ tel que
a
\begin{align*}
	c = \Delta \cdot c', \quad a = \Delta \cdot a'\\
	b = \Delta \cdot b', \quad d = \Delta \cdot d'
\end{align*}
Donc 
\begin{align*}
	\Delta &= ad -bc\\
	\Delta &= \Delta^{2}(a'd' - b'c')\\
	1 &= \Delta(a'd' - b'c')\\
	\frac{1}{\Delta} &= (a'd' - b'c')
\end{align*}
Or, par hypothèse, $\Delta \neq 1,-1,0$ et donc $a'd' - b'c'$ n'est pas entier.\\
Donc, $\langle\{(a,b),(c,d) \}\rangle$ n'engendre pas $\mathbb{Z}^{2}$






	
\end{document}
