\documentclass[11pt, a4paper]{article}
\usepackage[utf8]{inputenc}
\usepackage[T1]{fontenc}
\usepackage[francais]{babel}
\usepackage{lmodern}

\usepackage{amsmath}
\usepackage{amssymb}
\usepackage{amsthm}

\begin{document}
\title{Série 1, Exercice 6}
\author{David Wiedemann}
\maketitle
	$\bullet$ Clairement le numérateur est un entier positif, il suffit donc de montrer qu'il est pair.\\
		On distingue donc les cas.\\
		\begin{itemize}
			\item Supposons $m$ pair et $n$ pair, alors:
		\[ 
			m=2i \text{ et } n=2k
		\]

		
		Donc
		\begin{align*}
			( m+n)^{2} + m + 3n&= ( 2i+ 2k)^{2} + 2i + 6k\\
					   &= 2 ( 2 ( i+k)^2 + i + 3k)
		\end{align*}
		Donc le numérateur est pair.\\
	\item Supposons $m$ pair et $n$ impair, alors
		\[ 
			m = 2i \text{ et } n = 2k +1
		\]
		Donc 
		\begin{align*}
			( m+n)^{2} + m + 3n&= ( 2i+ 2k + 1)^{2} + 2i + 6k + 3\\
					   &= (2i + 2k)^{2} + 2(2i+ 2k) + 1 + 2i + 6k + 3\\
					   &= (2i + 2k)^{2} + 2(2i+ 2k)  + 2i + 6k + 4\\
					   &= 2( 2(i + k)^{2} + (2i+ 2k)  + i + 3k + 2 )\\
		\end{align*}
		Donc le numérateur est pair.\\
	\item Supposons $m$ impair et $n$ pair, alors
		\[ 
		m=2i + 1 \text{ et } n=2k
		\]
		Donc
		\begin{align*}
		( m+n)^{2} + m + 3n&= ( 2i+ 2k + 1)^{2} + 2i + 6k + 1\\
				&= (2i + 2k)^{2} + 2(2i+ 2k) + 1 + 2i + 6k + 1\\
				&= (2i + 2k)^{2} + 2(2i+ 2k) + 2i + 6k + 2\\
				&= 2( 2(i + k)^{2} + (2i+ 2k)  + i + 3k + 1 )\\
		\end{align*}
	\item Finalement, supposons $m$ impair et $n$ impair

		\[ 
			m=2i + 1 \text{ et } n=2k +1
		\]
		Donc
		\begin{align*}
		( m+n)^{2} + m + 3n&= ( 2i+ 2k + 2)^{2} + 2i + 6k + 4\\
				   &= 2 (2 ( i+k+1) + i + 3k +2)
		\end{align*}
		\end{itemize}
		$\bullet$ Par la définition de $D_k$, $(m,n) \in D_k$ implique que $(k-n,n)\in D_k$ et $0\leq n\leq k$.\\
		Supposons donc $(m,n) \in D_k$, on a:
		\begin{align*}
			C(m,n)=C(k-n,n)&= \frac{1}{2} \cdot \left( (k-n+n)^{2} + k-n + 3n\right)\\
				       &= \frac{1}{2} \cdot \left( k^{2} + k + 2n\right)\\
				       &= \frac{k^{2}+k}{2} + n
		\end{align*}
		Donc si $n=0$, $C(m,n)= \frac{k^{2}+k}{2}$ et si $n=k$, $C(m,n)= \frac{k^{2}+k}{2}+k$, donc les valeurs de $C(m,n)$ sont comprises entre $\frac{k^{2}+k}{2}$ et  $\frac{k^{2}+k}{2}+k$.\\
		$\bullet$ On est maintenant prêt à montrer la bijectivité de $C: \mathbb{N}^{2} \to \mathbb{N}$.
		Pour ceci, on va procéder par étapes:
		\begin{enumerate}
			\item Montrer que $C:D_k \to \{\frac{k^{2}+k}{2},\ldots,\frac{k^{2}+k}{2}+k\}$ est bijective.\\
			\item Montrer que $ D_i \cap D_k= \emptyset$, si $i\neq k$.\\
			\item Montrer que $\bigcup_{i=0}^{+\infty} D_i = \mathbb{N}^{2}$.\\
			\item Montrer la bijectivité de $C: \mathbb{N}^{2} \to \mathbb{N}$.
		\end{enumerate}
		Pour le point 1.\\
		Trouver un inverse pour $C:D_k \to \{\frac{k^{2}+k}{2},\ldots,\frac{k^{2}+k}{2}+k\}$ est facile, soit $a \in \{\frac{k^{2}+k}{2},\ldots,\frac{k^{2}+k}{2}+k\}$, alors
		\begin{align*}
			C^{-1}: \left\{\frac{k^{2}+k}{2},\ldots,\frac{k^{2}+k}{2}+k\right\} &\to D_k\\
			a &\to \left(a-\frac{1}{2}(k^{2}+k),k+\frac{1}{2}(k^{2}+k)-a\right)
		\end{align*}
		Clairement, cette application est bijective car $k$ est constante.\\
		Pour le point 2.\\
		Par l'absurde, supposons que $\exists (m,n) \in D_k$ et $(m,n) \in D_i$.\\
		Donc $m+n=i$ et $m+n=k$, donc $i=k$, ce qui est une contradiction à l'hypothèse.\\
		Pour le point 3.\\
		On montre la double inclusion.\\
		L'inclusion de gauche à droite est triviale.\\
		Supposons donc $(m,n) \in \mathbb{N}^{2}$. On pose $m+n=i, i \in \mathbb{N}$, donc $m=i-n$.
		\[ 
			(m,n) = (i-n,n) \in D_i
		\]
		On en déduit $\bigcup_{i=0}^{+\infty} D_i = \mathbb{N}^{2}$
		

		



		
		
		

\end{document}
