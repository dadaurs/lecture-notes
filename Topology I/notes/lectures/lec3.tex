\documentclass[../main.tex]{subfiles}
\begin{document}
\lecture{3}{Mon 17 Oct}{Singular homology}
Goal: Find a way to organise the information in $Sing_n( X) $!
\begin{enumerate}
\item Relate $Sing_n( X) $ for different $n$ to each other
\item Linearize !
\end{enumerate}
We'll call $Sing_n( X) $ the $n$-th component of the singular set.\\
We think of the edges of the simplices as being ordered.\\
There are maps $\Delta^{1}\to \Delta^{n}$ which are inclusions into the edges.\\
In fact, for every subset $S \subset \left\{ 0,\ldots, n \right\} $, there is a continuous injective map $\Delta^{k}\to \Delta^{n}$, where $k = |S|$.\\
Now, for any $k< n$, we have restriction maps $Sing_n( X) \to Sing_k( X) $.\\
Define the category $\Delta_{inj} $, whose onjects are $ [ n] $ for every $n \in \mathbb{N}$ and whose morphisms $[k] \to [ n] $ are order preserving injective maps.\\
The composition is just the composition of maps.\\
For $X$ a fixed topological space, we get a contravariant functor $Sing_{ \cdot} ( X) : \Delta_{inj} \to \Set$.\\
Given $\alpha: [ k] \to [ n] $ an injective order preserving map, we get 
\[ 
Sing_n( X) \to Sing_k( X) 
\]
with precomposition by $\alpha$.
\begin{lemma}
	$\Delta_{inj} $ can also be described as the category with objects $ [ n] $ and generated by maps $d^{i}:[n] \to [ n+1] $ subject to the relations
	\[ 
	d^{j}d^{i}= d^{i}d^{j-1}
	\]
	for $0 \leq i < j \leq n$ 
\end{lemma}
\begin{proof}[Sketch]
This relation is indeed satisfied in $\Delta_{inj} $ 
\[ 
\left\{ 0< \ldots <n-2 \right\} \xto { d^{i} } \left\{ 0< \ldots<n-1 \right\} \xto { d^{j}  }\left\{ 0<\ldots<n \right\} 
\]
Here
\[ 
k \mapsto
\begin{cases}
k, k \leq i-1\\
k+1, k \geq i
\end{cases}
\mapsto
\begin{cases}
k, k \leq i-1\\
k+1, k+1 \leq j\\
k+2, k+2 \geq j+1
\end{cases}
\]
One can compute that the composition $d^{i}d^{j-1}$ gives the same map.\\
What remains to show is that, subject to these relations, any order preserving injective map can be written as a composition of maps $d^{i}$.\\
If $\alpha$ is missing $ i_1< i_2< \ldots< i_{n-k} $, then $\alpha$ can be written as
\[ 
\alpha= d^{i_{n-k } } d^{i_{n-k-1}}\ldots d^{i_1}
\]
\end{proof}
We'll call $d^{i}$ the $i$-th coface map.\\
A contravariant functor $\Delta_{inj} \to \Set$ is called a semi-simplicial set.
\begin{defn}[Singular Chain Complex]
	A (non-negatively graded) singular chain complex of a space $X$ has as chain groups
	\[ 
		S_n X = \mathbb{Z}\eng{Sing_n( X) }
	\]
and differentials $\delta_n :S_n( X) \to S_{n-1} ( X) $ defined on generators as
\[ 
\del_n\left( \sigma: \Delta^{n}\to X\right) \mapsto \sum_{i=0}^{ n}( -1)^{i}\sigma\circ d^{i}
\]
\end{defn}
\begin{lemma}
The singular chain complex of a space is a chain complex.
\end{lemma}
\begin{proof}
By linearity, it is enough to check this on generators $\sigma: \Delta^{n}\to X$.\\
\begin{align*}
\delta_{n-1} \delta_n \sigma &= \delta_{n-1} \left( \sum_{i=0}^{ n} ( -1)^{i} \sigma\circ d^{i}\right) \\
&= \sum_{i=0}^{ n} ( -1)^{i} \sum_{j=0}^{n-1} ( -1)^{j}\sigma\circ d^{i}\circ d^{j}\\
&= \sum_{i=0}^{ n} \sum_{j=0}^{ n-1}( -1) ^{i+j} \sigma \circ d^{i}\circ d^{j}\\
&= \sum_{0 \leq j < i \leq n}^{ } ( -1) ^{i+j} \sigma\circ d^{i}\circ d^{j}\\
& + \sum_{0 \leq i \leq j \leq n-1}^{ } ( -1) ^{i+j} \sigma\circ d^{i}\circ d^{j} \\
&= \sum_{0 \leq j < i \leq n}^{ } ( -1) ^{i+j} \sigma\circ d^{i}\circ d^{j}\\
& + \sum_{0 \leq i < j' \leq n-1}^{ } ( -1) ^{i+j'-1} \sigma\circ d^{j'}\circ d^{i} \\
&= 0
\end{align*}
\end{proof}
\begin{lemma}
We get a functor from chain complexes with chain maps to graded abelian groups, which is just taking homology.
\end{lemma}
\begin{defn}[Singular Homology]
	The singular homology $H_{\bullet} X$ (with integer coefficients) on a space $X$ is the homology of the singular chain complex.
\end{defn}



\end{document}	
