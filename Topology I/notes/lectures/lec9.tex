\documentclass[../main.tex]{subfiles}
\begin{document}
\lecture{9}{Mon 07 Nov}{Conclusion of Excision}
We proved last week that $S_\ast^{ \left\{ X\setminus B, A \right\} X \to S_\ast X}$ is in fact a quasi-isomorphism.\\
In homework, we proved that any quasi-isomorphism between chain complexes of levelwise free abelian groups is a chain homotopy.\\
With this, we can now prove excision
\begin{proof}
There are maps between the two following SES
\[ 
0 \to S_\ast( A\setminus B) \to S_\ast( X\setminus B) \to S_\ast( X\setminus B, A \setminus B) \to 0
\]
\[ 
0 \to S_\ast(B) \to S_\ast( X) \to S_\ast( X, B) \to 0
\]
which are induced by inclusion.\\
There are maps $S_\ast A \to S_\ast^{U}X$ and $S_\ast( X\setminus B) \to S_\ast^{U}( X) $ and a quotient map $S_\ast^{U}X \to S_\ast( X\setminus B,A\setminus B) $.\\
As the map of chain complexes is a quasi-isomorphism, they are all chain homotopic and thus tensoring with $G$ preserves the quasi-isomorphism.\\
Applying the LES in homology, we get maps between the two following short exact sequences
\[ 
	H_n( A;G) \to H_n( S_\ast^{ \left\{ X\setminus B, A \right\} }( X) \otimes G)  \to H_n( X\setminus B, A\setminus B;G) \xto{\del} H_{n-1} ( A;G) \to H_{n-1} ( S_\ast^{ \left\{ X\setminus B,A \right\} } ( X)\otimes G) 
\]
\[ 
	H_n( A;G) \to H_n( X;G) \to H_n( X,A;G) \xto{\del} H_{n-1} ( A;G) \to H_{n-1} ( X;G) 
\]
Now, applying the five lemma, we see that $H_n( X\setminus B, A\setminus B;G) \simeq H_n( X,A;G) $ 
\end{proof}
It remains to show additivity.\\
Let $\left\{( X_i,A_i) \right\}_{i \in I} $ be a family, we need to show that the inclusions induce a map
\[ 
\bigoplus_{i\in I} H_n( X_i,A_i) \to H_n \left( \coprod_{i\in I} X_i, \coprod_{i \in I} A_i\right) 
\]
Notice that the homology functor $H_n: Ch_{ \geq 0} \to gr\ab$ commutes with arbitrary direct sums.\\
Furthermore, from the definition, it is clear that $S_n( \coprod_i X_i,\coprod_i A_i;G) \simeq \bigoplus_i S_n( X_i,A_i;G) $.\\
It is easy to check that $H_n( pt; \mathbb{Z}) = 0$ for all $ n \neq 0$, tensoring the chain complex (of singular chains of a point)
\[ 
	\ldots \xto{\id} \mathbb{Z}\xto{0} \mathbb{Z} \xto { \id} \mathbb{Z}\ldots
\]
with $G$, it is clear that $H_n( pt; G) =0\forall n \geq 1$ and $H_0( pt, G) = G$ 


		
\end{document}	
