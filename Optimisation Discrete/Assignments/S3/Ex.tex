\documentclass[11pt, a4paper]{article}
\usepackage[utf8]{inputenc}
\usepackage[T1]{fontenc}
\usepackage[francais]{babel}
\usepackage{lmodern}
\usepackage{amsmath}
\usepackage{amssymb}
\usepackage{amsthm}
\renewcommand{\vec}[1]{\overrightarrow{#1}}
\newcommand{\del}{\partial}
\DeclareMathOperator*{\sgn}{sgn}
\DeclareMathOperator*{\id}{Id}
\DeclareMathOperator*{\im}{Im}
\DeclareMathOperator*{\re}{Re}
\DeclareMathOperator*{\vol}{Vol}
\newcommand\norm[1]{\left\vert#1\right\vert}
\newcommand\ns[1]{\left\vert\left\vert\left\vert#1\right\vert\right\vert\right\vert}
\newcommand\Norm[1]{\left\lVert#1\right\rVert}
\newcommand\N[1]{\left\lVert#1\right\rVert}
\newcommand\abs[1]{\left\vert#1\right\vert}
\newcommand\inj{\hookrightarrow}
\newcommand\surj{\twoheadrightarrow}
\newcommand\ded[1]{\overset{\circ}{#1}}
\newcommand\sidenote[1]{\footnote{#1}}
\newcommand\eng[1]{\left\langle#1\right\rangle}
\newcommand\hr{
    \noindent\rule[0.5ex]{\linewidth}{0.5pt}
}

\newcommand{\incfig}[1]{%
    \def\svgwidth{\columnwidth}
    \import{./figures}{#1.pdf_tex}
}
\newcommand{\filler}[1][10]%
{   \foreach \x in {1,...,#1}
    {   test 
    }
}

\newcommand\contra{\scalebox{1.5}{$\lightning$}}
\makeatother
\def\@lecture{}%
\newcommand{\lecture}[3]{
    \ifthenelse{\isempty{#3}}{%
        \def\@lecture{Lecture #1}%
    }{%
        \def\@lecture{Lecture #1: #3}%
    }%
    \subsection*{\@lecture}
    \marginpar{\small\textsf{\mbox{#2}}}
}

\begin{document}
\title{Assignment 3}
\author{David Wiedemann}
\maketitle
\section{Problem Set 7, Exercise 3}
To each edge $e\in E$, we associate a number $a( e ) \in \left\{ 0,1 \right\}  $ 
We will say that $e$ is in $M$ if $a( e ) =1$ and that $e$ is not in $M$ if $a( e  ) =0$.\\
Now, for each vertex $v\in V$, we denote by $E( v) \subset E$ the set of all edges containing $v$.\\
The condition for $M$ to be a matching then reads as $\sum_{ e \in E( v) } a( e )  \leq 1\forall  v \in V$.\\
Thus, the problem of finding a matching with a maximal number of edges reads as
\begin{align*}
	\max & \sum_{ e \in E} a( e  ) \\
	\text{ s.t. } & \sum_{e \in E( v) } a( e ) \leq 1\forall v \in V\\
		      &a( e  ) \in \left\{ 0,1 \right\} 
\end{align*}
\section{Every tree has at least one leaf vertex}
Recall that a tree is a connected graph that contains no cycles and that a leaf is a vertex of degree 1.\\
Let $T= ( V,E) $ be a (finite) tree.\\
Suppose by contradiction that $T$ does not contain any leaf, we construct a path as follows.
We pick some vertex $ v_1\in V$, as $T$ is connected, there is some vertex $v_2$ connected to $v_1$.\\
As $v_2$ is not a leaf it is at least of degree 2, so there is a vertex $v_3$ different from $v_1$ which is connected to $v_2$.\\
Proceeding inductively in this way, we get a path $v_1, v_2,\ldots, $ which always satisfy $v_{n+1} \neq v_{n-1} $ (ie. we aren't going back and forth).\\
As the set $V$ is finite, this path must cross itself again at some point, ie. there is some index $j$ and some $i< j-1$ satisfying $v_i = v_j$.\\
We may pick $j$ minimal among all indexes with this property, then the walk $v_i, v_{i+1} , \ldots, v_{j-1} , v_j =v_i$ is in fact a cycle (as it doesn't intersect itself by our minimality assumption on $j$).\\
But this contradicts the fact that $T$ is a tree.
\section{Problem Set 9, Exercise 2.a)}
Let $x$ be a basic solution for $ \mathcal{P}$, then there exists a basis $\beta$ of $n$ elements such that 
\[ 
A_\beta x_\beta = b
\]
and such that for every index $i\notin \beta$, $x_i=0$.\\
As the set of collumn vectors of $A_\beta$ is linearly independent, we know that $A_\beta$ is invertible and thus by hypothesis $\det A_\beta = \pm 1 $.\\
Cramer's rule tells us that for all $j\in \beta$, the value of $x_j$ is given by
\[ 
x_j = \frac{ \det A_\beta^{j}}{ \det A_\beta}
\]
where $A_\beta^{j}$ denotes the matrix $A_\beta$ where we've replaced the $j$-th collumn with the vector $b$.\\
But then $\det A_\beta^{j}$ is an integer and as $\det A_\beta= \pm 1$, we conclude that in fact $x$ is a vector with integer entries.




\end{document}
