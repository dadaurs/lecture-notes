\documentclass[11pt, a4paper]{article}
\usepackage[utf8]{inputenc}
\usepackage[T1]{fontenc}
\usepackage[francais]{babel}
\usepackage{lmodern}
\usepackage{amsmath}
\usepackage{amssymb}
\usepackage{amsthm}
\DeclareMathOperator{\col}{col}
\renewcommand{\vec}[1]{\overrightarrow{#1}}
\newcommand{\del}{\partial}
\DeclareMathOperator*{\sgn}{sgn}
\DeclareMathOperator*{\id}{Id}
\DeclareMathOperator*{\im}{Im}
\DeclareMathOperator*{\re}{Re}
\DeclareMathOperator*{\vol}{Vol}
\newcommand\norm[1]{\left\vert#1\right\vert}
\newcommand\ns[1]{\left\vert\left\vert\left\vert#1\right\vert\right\vert\right\vert}
\newcommand\Norm[1]{\left\lVert#1\right\rVert}
\newcommand\N[1]{\left\lVert#1\right\rVert}
\newcommand\abs[1]{\left\vert#1\right\vert}
\newcommand\inj{\hookrightarrow}
\newcommand\surj{\twoheadrightarrow}
\newcommand\ded[1]{\overset{\circ}{#1}}
\newcommand\sidenote[1]{\footnote{#1}}
\newcommand\eng[1]{\left\langle#1\right\rangle}
\newcommand\hr{
    \noindent\rule[0.5ex]{\linewidth}{0.5pt}
}

\newcommand{\incfig}[1]{%
    \def\svgwidth{\columnwidth}
    \import{./figures}{#1.pdf_tex}
}
\newcommand{\filler}[1][10]%
{   \foreach \x in {1,...,#1}
    {   test 
    }
}

\newcommand\contra{\scalebox{1.5}{$\lightning$}}
\makeatother
\def\@lecture{}%
\newcommand{\lecture}[3]{
    \ifthenelse{\isempty{#3}}{%
        \def\@lecture{Lecture #1}%
    }{%
        \def\@lecture{Lecture #1: #3}%
    }%
    \subsection*{\@lecture}
    \marginpar{\small\textsf{\mbox{#2}}}
}

\begin{document}
\title{Assignment 2}
\author{David Wiedemann}
\maketitle
\section*{Problem Set 4, Exercise 1.a, 1.b}
\subsection*{1.a)}
By definition of the reduced cost, we get that
\begin{align*}
\overline{c}_i = c_i - c_B^{T}	B^{-1} \col_j( A) 
\end{align*}
Hence, the reduced cost vector in the basis $B$  is of the form
\[ 
\overline{c}_B^{T} = c_B^{T}- c_B^{T}B^{-1} \col_B( A) = c_B^{T}-c_B^{T}B^{-1}B = 0
\]
Hence, if $x_i$ is in the basis, then $\overline{c}_i x_i = 0$.\\
If $x_i$ is not in the basis, then by definition, $x_i=0$, which implies $\overline{c}_i x_i = 0 \forall i$ 

%First, suppose that $x_i\neq 0$, then note that the column basis contains the $i$'th vector.\\
%We get that for $\beta$ our current basis and $B= \col_\beta( A) $ the associated matrix
%\[ 
%B^{-1}\col_i (A) = e_i = 
%\begin{pmatrix}
%0\\
%\vdots\\
%1\\
%\vdots\\
%\end{pmatrix} 
%\]
%where the $1$ appears in the $i$-th position.\\
%Hence
%\[ 
%\overline{c}_i = -c_\beta^{T} B^{-1}\col_i( A) + c_i = - c_{\beta} ^{T} e_i + c_i = -c_i + c_i =0.
%\]
%Now suppose that $\overline{c}_i\neq 0$, and suppose $x_i \neq 0$ also, then $x_i$ is in the column basis which, by the computation above implies that $ \overline{c}_i=0$, hence $x_i =0$, concluding the proof.
\subsection*{1.b)}
Since $d$ is a feasible direction, in particular, we get that $A\cdot d =0$.\\
Indeed, for some $\theta >0$ we have that
\[ 
A ( x+ \theta d) =b \implies Ax + \theta A d = b \implies \theta Ad = 0 \implies Ad = 0
\]

Hence
\[ 
\overline{c} \cdot d = ( c^{T}-c_B^{T}B^{-1}A) \cdot d = c^{T}\cdot d - c_B^{T}B^{-1}Ad = c\cdot d
\]



\section*{Problem Set 5, Exercise 1.a, 1,b}
\subsection*{1.a)}
The data of the program is
\begin{align*}
c= \begin{pmatrix}
3\\1\\4
\end{pmatrix}  , b = \begin{pmatrix}
1\\1\\1\\1
\end{pmatrix} 
\intertext{and}
A=
\begin{pmatrix}
	2 & 1 &1\\
	1& -1 & 2\\
	-1 &-1 &1\\
	-2 & 1 &-1
\end{pmatrix} 
\end{align*}
Hence we first turn it into a minimization problem by multiplying the cost by $-1$ 
\begin{align*}
	P'=\min  -3x_1-x_2-4x_3&\\
	\text{s.t. }  2x_1+x_2 + x_3 &\leq 1\\
	x_1-x_2 + 2x^{3} &\leq 1\\
	-x_1-x_2+x_3 &\leq 1\\
	-2x_1 + x_2 - x_3 &\leq 1\\
\end{align*}

Hence our dual program will be
\begin{align*}
	P'=\min \quad b\cdot \lambda&\\
			    \text{s.t. }  \lambda^{t}A&= -c^{T}\\
						       \lambda &\geq 0
\end{align*}
which when written out is
\begin{align*}
	P' = \min \lambda_1+ \lambda_2 + \lambda_3+\lambda_4&\\
	\text{ s.t. } \begin{pmatrix}
		\lambda_1 &\lambda_2 &\lambda_3&\lambda_4
	\end{pmatrix}  \cdot
	\begin{pmatrix}
	2 & 1 &1\\
	1& -1 & 2\\
	-1 &-1 &1\\
	-2 & 1 &-1
	\end{pmatrix} 
							    &= \begin{pmatrix}
								    -3 &-1&-4
							    \end{pmatrix}  \\
	\lambda_i &\geq 0\forall i
\end{align*}
\subsection*{1.b)}
We use the procedure described in the course, our starting program is
\begin{align*}
	\max \quad x_1 + 4x_2 - 2 x_3 - x_4&\\
	3x_1 + x_3 + 2x_4 &=6\\
	-x_1 + 2x_2 - x_3 - x_4 &= 2\\
	x_i &\geq 0
\end{align*}
Setting $c= \begin{pmatrix}
	1 & 4 & -2 & -1
\end{pmatrix} $, $ b= \begin{pmatrix}
 6 &2
\end{pmatrix} $ and
\[ 
A = 
\begin{pmatrix}
	3 & 0 & 1 & 2 \\
	-1 & 2 & -1  & -1
\end{pmatrix} 
\]
This gives us a dual program
\begin{align*}
	\min \quad b\cdot \lambda &\\
	\text{ s.t. } A^{T}\lambda \leq c
\end{align*}
Which, when written out yields
\begin{align*}
	\max\quad 6\lambda_1 + 2 \lambda_2&\\
	\text{ s.t. } 3\lambda_1 - \lambda_2 & \geq 1\\
	2\lambda_2 &\geq 4\\
	\lambda_1 - \lambda_2 &\geq -2\\
	2\lambda_1 - \lambda_2 &\geq  -1
\end{align*}
as the dual program.











\end{document}
