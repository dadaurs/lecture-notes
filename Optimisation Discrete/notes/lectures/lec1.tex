\documentclass[../main.tex]{subfiles}
\begin{document}
\lecture{1}{Tue 22 Feb}{Introduction}
\section{What is optimization?}
Given $f:X\to \mathbb{R}$ find
\[ 
\max \left\{ f( x) :x\in S \right\} \text{ where $S \subset \mathbb{R}^n $  } 
\]
$f$ is called the objective function and $S$ the feasible region.\\
The answer will be a point $x_0 \in S$ such that $f( x_0) \geq f( x) $ for all other $x\in S$.\\
A solution is anything you can plug into $f$.\\
A feasible solution is something in $S$ and the $x_0$ would be called an optimal solution.\\
We consider feasible regions described by intersection of equalities and inequalities.\\
A constraint is "active" or "tight" if $f$ satisfies it with equality.
\subsection{New lagrange multipliers}
We should try to satisfy $\nabla f= \sum_i \lambda_i \nabla g_i$ where $\lambda_i=0$ whenever $g_i$ is not active.
Feasible regions split into nice ones and not so nice ones.\\
A region $S$ is convex if $\forall x,y \in S$ and all $\lambda\in [ 0,1] $, then $\lambda x+ ( 1-\lambda) y\in S$ 
\end{document}	
