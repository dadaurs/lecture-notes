\documentclass[../main.tex]{subfiles}
\begin{document}
\lecture{15}{Thu 28 Apr}{Pointwise convergence of fourier series}
\begin{defn}
	Let $\alpha\in ( 0,1) $ and $\Omega$ be a bounded set in $ \mathbb{R}^{n}$, then the space of Holder continuous functions is
	\[ 
	\left\{ f\in C^{0,\alpha}( \Omega) : \sup_{x\,y \in \Omega} \frac{|f( x) -f( y) |}{|x-y|^{\sigma}} \right\} 
	\]
	
\end{defn}
\begin{thm}[Dirichlet]
	\begin{itemize}
	\item If $f$ is 1-periodic and piecewise $C^{1}$, then $\forall x $ 
		\[ 
			F_n f( x) \to \frac{1}{2}(  f^{+}( x) + f^{-}( x)  )
		\]
	
	\item If $f$ is 1-periodic and $C^{0,\alpha}( -1,2) $, then
		\[ 
		\forall  x F_nf( x) \to f( x) 
		\]
		
	\item If $f\in L^{1}( \faktor { \mathbb{R}} { \mathbb{Z}} ) $ such that for $a \in \mathbb{R} \exists M( a) >0$ and $\delta( a) >0$ such that
		\[ 
		|f( t+a) - f^{+}( a) |, |f( a-t) - f^{-}( a) | \leq M( a) t^{a}\forall t \in ( 0,\delta( a) ) 
		\]
	Then $ F_nf( a) \to \frac{f^{+}( a)+ f^{-}( a) 	}{2}$ 
	\end{itemize}
\end{thm}
\begin{rmq}
Note that point 3 trivially implies point 2.\\
In fact $3$ implies 1 as well since
\begin{align*}
f( a+t) = f( a) ^{+} + f'( a)+ + o( t) \\
f( a-t) = f( a)^{-} + f'( a)^{-}t + o( t) 	
\end{align*}
But then
\[ 
|f( t+a) - f( a)^{+}| \leq  (  |f'( a)^{+}|+ 1) |t|
\]
\end{rmq}
Kolmogorov showed that
\begin{enumerate}
\item $\exists f$ continuous such that $F_nf ( x) \not\to f( x) $ 
\item $\exists f \in L^{1}$ such that the fourier series diverges everywhere.
\item Carleson also show that if $f\in L^{p}(  \faktor { \mathbb{R}} { \mathbb{Z}} ) $ then $F_n f( x) \to f( x)  $ almost everywhere.
\end{enumerate}
\begin{proof}
Recall that $F_N f( x) = \sum_{-N}^{ N} \hat{f}( n) e^{i 2 \pi n x } = \sum_{-N}^{ N} \int_{  }^{  }f( y)  e^{i 2 \pi n ( x-y) } dy $.\\
Then
\begin{align*}
&= \int_{  }^{  } f( y)  \sum_{-N}^{ N } e^{i 2 \pi n ( x-y) } dy\\
&= f \ast D_N( x) 
\intertext{where}
	D_N( y) &= \sum_{-N}^{ N} ((  e^{i 2 \pi y} )^{n}) 
	&=  \frac{ \sin( \pi y ( 2N+1) ) }{\sin \pi y}
\end{align*}
Now let $a \in [ 0,1] $ and define
\[ 
M=
\begin{cases}
f( a) \text{ if continuous at $a$  } \\
\frac{1}{2}( f^{+}( a) + f^{-}( a) ) \text{ if not } 
\end{cases}
\]
Then
\begin{align*}
F_n f( a) - M &= \int_{ - \frac{1}{2} }^{ \frac{1}{2} } ( f( y)-M )  \frac{\sin ( \pi y( 2N+1) ) }{\sin \pi y}
\end{align*}
\begin{lemma}[Riemann-Lebesgue]
Let $f \in L^{1}( \mathbb{R}) $ and
\[ 
\hat{f} ( \xi) = \int_{  }^{  } f( x) e^{- i 2\pi \xi x} dx
\]
then $\lim_{\xi \to  + \infty} \hat{f}( \xi) = 0$ 
\end{lemma}
Indeed, taking the imaginary part in  Riemann-Lebesgue
\[ 
\int_{   }^{  }	F( x)  \sin ( 2\pi \xi x) dx \to 0
\]
We apply  this to Riemann-Lebesgue, we apply it to $F= \phi_a, \xi = N +\frac{1}{2}$ 
We check $\phi_a \in L^{1}$, we do it assuming $f$ continuous
\begin{align*}
\int_{ -\frac{1}{2} }^{ \frac{1}{2} } |\phi_a| = \int_{ - \frac{1}{2} }^{ \frac{1}{2} } \frac{f( a+y) - f( a) }{ \sin \pi y} \leq  \int_{ -\frac{1}{2} }^{ \frac{1}{2} } \frac{M y}{cy} = \frac{t^{\alpha}}{\alpha} |_{-\frac{1}{2}} ^{\frac{1}{2}}
\end{align*}





\end{proof}



\end{document}	
