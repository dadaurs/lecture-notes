\documentclass[../main.tex]{subfiles}
\begin{document}
\lecture{13}{Wed 13 Apr}{Fourier Analysis}
\subsection{Periodic Functions}
\begin{defn}[Periodic function]
	Let $L>0$, $f: \mathbb{R}\to \mathbb{R}$ is $L$-periodic if $f( x+L ) = f( x) $.
\end{defn}
We look for $f,g\in L^{p}( [ 0,1] ) $ and $1$-periodic such that
\[ 
	\N{f-g}_{L^{p}} = ( \int_{ 0 }^{ 1 }|f-g|^{p}) ^{\frac{1}{p}}
\]
For $p=2$ there is an associated scalar product given by
\[ 
\langle f,g\rangle = \int_{ 0 }^{ 1 } f \overline{g} dx
\]
\begin{defn}[Space of periodic functions]
	$C^{0}(  \faktor { \mathbb{R}} { \mathbb{Z}} , \mathbb{C}) $ is the space of continuous $1$-periodic functions.
\end{defn}
\begin{lemma}[Basic properties]
\begin{itemize}
\item If $f\in C^{0}( \faktor { \mathbb{R}} { \mathbb{Z}} , \mathbb{C}) $, then $f$ is bounded.
\item $ \mathbb{C}^{0}( \faktor { \mathbb{R}} { \mathbb{Z}} , \mathbb{C}) $ is a vector space and an algebra.
\item The space is is closed under uniform limits.
\end{itemize}
\end{lemma}
\subsection{Trigonometric polynomials}
\begin{defn}
	$\forall m\in \mathbb{Z}$, the character with frequency $n$ is
	\[ 
	e_n( x) = e^{i 2 \pi n x } 
	\]
\end{defn}
\begin{defn}[Trigonometric polynomial]
	An element $f\in C^{0}( \faktor { \mathbb{R}} { \mathbb{Z}} , \mathbb{C}) $ is a trigonometric polynomial if
	\[ 
	f( x) = \sum_{-N}^{ N} c_n e^{i 2\pi n x } 
	\]
	for some $N \geq 0$.
\end{defn}
\begin{lemma}
	The family of $ \left\{ e_n \right\} $ is an orthonormal system, ie.
	\[ 
	\langle e_n, e_m \rangle  =  \delta_{nm} 
	\]
	
\end{lemma}
The proof is an exercise.
\begin{crly}
Let $f= \sum_{-N}^{ N}c_n e_n$, then
\[ 
c_n = \langle f, e_n\rangle
\]
and 
\[ 
\sum_{}^{ } c_n^{2} = \N { f}_{L^{2}} ^{2}
\]


\end{crly}
\begin{proof}
\[ 
\langle f, e_m\rangle \langle \sum c_n e_n, e_m\rangle = c_m
\]
And thus also
\[ 
\N { f} _{L^{2}} ^{2} = \sum_{}^{ } |\langle f, e_m\rangle|^{2} 
\]


\end{proof}

\begin{defn}[Fourier Coefficients]
	Let $f$ be some periodic function, then the $n$-th fourier coefficient is
	\[ 
	\hat{f}( n) = \langle f, e_n\rangle
	\]
	

\end{defn}
\begin{crly}
Let $f$ be a trigonometric polynomial, then
\[ 
f= \sum_{-N}^{ N} \langle f, e_n\rangle e_n
\]

\end{crly}
\subsection{Periodic convolutions}
\begin{thm}[Weierstrass approximation]
	Let $f \in C^{0}(  \faktor { \mathbb{R}} { \mathbb{Z}} , \mathbb{C}) $ and $\epsilon>0$, then $\exists P$ a trigonometric polynomial such that
	\[ 
	\N { f_n -P} \leq  \epsilon
	\]
	
\end{thm}
\begin{defn}[Convolution of periodic functions]
	Let $f,g\in C^{0}(  \faktor { \mathbb{R}} {\mathbb{Z} } , \mathbb{C}) $, then the periodic convolution is 
	\[ 
	f\ast g ( x) = \int_{ 0 }^{ 1 } f( y) g( x-y) dy
	\]
		
\end{defn}
\begin{rmq}
	Let $ f \in C^{0}(  \faktor { \mathbb{R}} { \mathbb{Z}} , \mathbb{C}) \cap L^{1}( \mathbb{R}) \implies f=0$ 
\end{rmq}
\begin{lemma}[basic properties]
Let $f,g \in C^{0}(  \faktor { \mathbb{R}} { \mathbb{Z}} , \mathbb{C}) $, then
\begin{itemize}
\item $f\ast g$ is closed in $ C^{0}(  \faktor { \mathbb{R}} { \mathbb{Z}} , \mathbb{C}) $ 
\item $f\ast g = g \ast f$ 
\item $( f+g) \ast h = f\ast h + g \ast h$ 
\end{itemize}
\end{lemma}
\begin{rmq}
	\begin{align*}
	f \ast e_n = \hat{f}e_n
	\end{align*}
	since
	\[ 
	f \ast e_n = \int_{ 0 }^{ 1 } f( y) e^{i 2 \pi n ( x-y) } dy = e^{i 2 \pi n x } \hat{f}( n)
	\]
	
\end{rmq}
\begin{defn}
	Let $\epsilon>0, \delta \in ( 0,\frac{1}{2}) $, we say that $f \in C^{0}(  \faktor { \mathbb{R}} { \mathbb{Z}} , \mathbb{C}) $ is a periodic $\epsilon-\delta$ approximation of the identity if
	\begin{enumerate}
	\item $f \geq 0, \int_{0  }^{1  }f = 1$ 
	\item $f( x) < \epsilon \forall x \in [ \delta, 1- \delta] $ 
	\end{enumerate}
	
\end{defn}
\begin{lemma}
$\forall \epsilon- \delta$ there exists a trigonometric polynomial $P$ which is an $\epsilon-\delta$ approximation of the identity.
\end{lemma}
\begin{proof}
Consider $F_N( x) = \frac{1}{N} ( \sum_{}^{ } e_n) ^{2}$, and the rest follows from an exercise.
\end{proof}
\begin{proof}[Of Weierstrass]
Let $f\in C^{0}(  \faktor { \mathbb{R}} { \mathbb{Z}} , \mathbb{C}) $ and $\epsilon>0$.\\
$f$ is bounded and uniformly continuous hence there exists $M$ such that
\[ 
|f( x) | \leq M
\]
$\exists \delta$ such that $ |f( x) - f( y) | < \epsilon \forall |x-y| < \delta$.\\
By lemma, let $P$ be a trigonometric polynomal satisfying the $\epsilon-\delta$ condition.\\
Then $f \ast P$ is a trigonometric polynomial.\\
We claim that $ \N { f- f\ast P} _{L^{ \infty }} < \epsilon$.\\
Indeed
\begin{align*}
	|f( x) - f\ast P( x) | &= |f( x) - \int_{ 0 }^{  1}P( y)  f( x-y) dy|\\
	&= | \int_{ 0 }^{ 1 } ( f( x) - f( x-y) ) P( y) dy|\\
	&\leq  \int_{ 0 }^{ 1 } |f( x) - f( x-y) | P( y) dy\\
	& \leq  \int_{ \delta }^{ 1- \delta } 2 \max f \epsilon + \int_{ [ 0,\delta] \cup [ 1- \delta, 1]  }^{  } \epsilon P( y) dy \leq  2 M \epsilon + \epsilon
\end{align*}
		
\end{proof}




\end{document}	
