\documentclass[../main.tex]{subfiles}
\begin{document}
\lecture{6}{Wed 16 Mar}{Dominated Convergence Theorem}
\subsection{Integration of signed functions}
\begin{defn}
	$f: \Omega\to \left[ - \infty , \infty \right]  $ is absolutely integrable if
	\[ 
	\int_{ \Omega }^{  } |f| < \infty 
	\]
\end{defn}
\begin{defn}[Integral of a function]
	  Let $f$ be an absolutely integrable function, then
	  \[ 
	  \int_{ \Omega }^{  } f = \int_{ \Omega }^{  }f^{+} - \int_{ \Omega }^{  }f^{-}
	  \]
	  	
\end{defn}
\begin{rmq}
\[ 
| \int_{ \Omega }^{  } f | \leq  \int_{ \Omega }^{  }|f|
\]

\end{rmq}
\begin{propo}[Basic properties]
Let $f,g$ be absolutely integrable functions
\begin{itemize}
\item $\forall c \in \mathbb{R}$, $cf$ is absolutely integrable and $ \int_{ \Omega }^{  }cf = c \int_{ \Omega }^{  }f$ 
\item $f+g$ is absolutely integrable and $ \int_{ \Omega }^{  } f+g = \int_{ \Omega }^{  }f+ \int_{ \Omega }^{  }g$ 
\item  If $f=g$ almost everywhere then $ \int_{ \Omega }^{  }f = \int_{ \Omega }^{  }g$ 
\end{itemize}
\end{propo}
\begin{thm}[Dominated Convergence Theorem]
	Let $f_1,f_2, \ldots: \Omega \to [ - \infty , \infty ] $ be measurable functions. Assume $f_n\to f$ almost everywhere and such that $|f_m( x) | \leq F( x) \forall m,x \in \Omega$ where $F$ is absolutely integrable.\\
	Then 
	\[ 
	\lim_{n \to  + \infty} \int_{  }^{  } f_n = \int_{  }^{  }f
	\]
	
\end{thm}
\begin{rmq}
With the same assumptions, we can conclude that
\[ 
\lim_{n \to  + \infty} \int_{  }^{  } |f_n -f| = 0
\]
Indeed, apply the theorem to $g_n = |f_n -f|$.\\
Then $|g_m| \leq |f_n| + |f| \leq  2F $ .\\
Similarly, let $f_m$ be such that the above condition holds, then $ \int_{  }^{  }f_n\to \int_{  }^{  }f$, since
\[ 
| \int_{  }^{  }f_n - \int_{  }^{  }f | = | \int_{  }^{  }f_n -f | \leq \int_{  }^{  } |f_n-f|\to 0
\]


\end{rmq}
\begin{proof}
By assumption $|f_n| \leq F$, hence $|f| \leq F$.\\
Apply Fatou to $F( x) + f_n( x) $, we get
\[ 
 \int_{ \Omega }^{  } F+f \leq  \liminf \int_{  }^{  } F+f_n \leq  \liminf \int_{  }^{  }f_m + \int_{  }^{  } f_n
\]
Now we apply Fatou to $F- f_n \geq 0$, we get
\begin{align*}
\int_{ \Omega }^{  } F - \int_{ \Omega }^{  } f \leq  \liminf \int_{ \Omega }^{  } F - f_n 
\intertext{Which in turn implies that}
\int_{ \Omega }^{  } f \leq \liminf_{n \to \infty } \int_\Omega f_n
\end{align*}
We now apply the same trick to $F-f_n $, noticing again this family of functions is non-negative
\begin{align*}
\int_{ \Omega }^{  } F-f \leq  \liminf_{n \to \infty } \int_{ \Omega }^{  } F-f_n\\
\int_{ \Omega }^{  } f \geq   \limsup_{n \to \infty } \int_{ \Omega }^{  } f_n\\
\end{align*}


Which implies the limit $ \int_{  }^{  }f_n$ exists and is equal to $ \int_{  }^{  }f$  
\end{proof}
\begin{rmq}[Differentiation under the integral]
	Let $f: \Omega\times \mathbb{R}\to \mathbb{R} \cup \left\{ \pm \infty  \right\} $ be measurable such that
	\begin{itemize}
	\item $\del_t f( x,t) $ for almost every $x$ and every $t$ 
	\item $| \del_t f( x,t) | \leq h( x) $ where $h( x) $ is an absolutely integrable function, then
		\[ 
		\frac{d}{dt} \int_{  }^{  }f( x,t) dx = \int_{  }^{  } \del_t f( x,t) 
		\]
		
	\end{itemize}
	\begin{proof}
	Indeed
	\[ 
	\frac{d}{dt} \int_{  }^{  }f( x,t) = \lim_{h \to 0} \int_{  }^{  } \underbrace{\frac{f( x,t+h) - f( x,t) }{h}}_{ \to \del_t f( x,t) }
	\]
	Now notice that
	\[ 
	| \frac{f( x,t+h) -f( x,t) }{h}| \leq | \int_{  }^{  } \del_t f( x,t+hs) ds| \leq  h( x) 
	\]
	
	\end{proof}
\end{rmq}
\begin{defn}
	Let $\Omega \subset  \mathbb{R}^m$, $f$ a function ( not necessarily measurable).\\
	The upper and lower Lebesgue integrals
	\[ 
		\overline{\int_\Omega} f = \inf \left\{ \int_{  }^{  }g: g \text{ measurable }, g \geq f \right\} 
	\]
	and similarly the lower integral.
	\[ 
		\underline{\int_\Omega} f = \inf \left\{ \int_{  }^{  }g: g \text{ measurable }, g \leq  f \right\} 
	\]
\end{defn}
\subsection{Comparison with Riemann Integral}
\begin{thm}[Lebesgue generalizes Riemann]
	Let $I \subset \mathbb{R}$ be an interval, $f: I \to \mathbb{R}$ be Riemann integralble, then $f$ is absolutely integrable and
	\[ 
	\int_{ I }^{  }f dx = \text{ Riemann integral of $f$ on $I$  } 
	\]
		
\end{thm}
\begin{proof}
$f$ is Riemann integrable if $\forall \epsilon>0$ there exists $p$ a partition of $I$ such that 
\[ 
A- \epsilon \leq \sum_{}^{ } |J| \inf_{x\in J} f \leq  \sum_{J\in P}^{ } |J| \sup f \leq A + \epsilon
\]
Since $f_\epsilon^{-} \leq f \leq  f_\epsilon^{+}$ 
\[ 
	A-\epsilon \leq \int_{  }^{  } f_\epsilon^{-} \leq  \underline{\int} f \leq  \overline{\int} f \leq  \int_{  }^{  }f_\epsilon^{+} \leq A+\epsilon
\]
Letting $\epsilon \to 0$ yields the result.\\
Indeed let $f_m^{\pm}$ be such that $f_m^{-} \leq  f \leq f_m^{+}$ 
\[ 
\underline{ \int }_{  }^{  }f-\frac{1}{m} \leq  \int_{  }^{  } f_m^{+ } \leq \overline { \int} f + m
\]
Setting $F^{-}= \sup f_m^{-}, F^{+}= \inf f_m^{+}$ are measurable.\\
$F^{-} \leq  f \leq F^{+}$ 

			
\end{proof}
\subsection{Fubini's Theorem}
\begin{thm}[Fubini-Tonelli]
Let $f: \mathbb{R}^n\times \mathbb{R}^m\to \mathbb{R}$. Assume $f \geq 0$ or $f$ absolutely integrable, then
\begin{itemize}
\item for almost every $x$ , $f( x, \cdot) $ is measurable and
	\[ 
	x\mapsto \int f( x,y) dy
	\]
	is measurable
\item For almost every $y$ , $f( \cdot, y) $ is measurable and
	\[ 
	y\mapsto \int f( x,y) dy
	\]
	is measurable
\end{itemize}

\[ 
\int_{ \mathbb{R}^n\times \mathbb{R}^m }^{  } f dx dy = \int_{ \mathbb{R}^n }^{  } \left( \int_{ \mathbb{R}^m }^{  } f( x,y) dy  \right)dx
\]

\end{thm}




\end{document}	
