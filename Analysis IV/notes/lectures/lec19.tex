\documentclass[../main.tex]{subfiles}
\begin{document}
\lecture{19}{Thu 12 May}{Applications of Fourier Transforms to PDE's}
\section{Fourier Transforms and PDE's}
\begin{exemple}
\begin{enumerate}
\item Let $u: \mathbb{R}\to \mathbb{R}$, we search for the solutions to $\del_{xx} u =0$ 
\item Ordinary Differential equations
\item 
	\[ 
	\del_{xx} u + \del_{yy} u =0 \quad \text{ Laplace equation } 
	\]


\item Poisson equation
	\[ 
	\Delta u = f
	\]
	for some given function $f$ 
\item Heat equation
	\[ 
	\del_t u - \Delta u =0
	\]
	
\item Wave Equation
	\[ 
	\del_{tt} u - \Delta u =0
	\]
	
\item Finding $u$ such that
	\[ 
	\nabla u = f
	\]
	
\item Burger equation
\[ 
\del_t u - \del_{xx} ( u^{2}) =0
\]

\end{enumerate}
\end{exemple}
\begin{defn}[PDE]
	A partial differential equation ( PDE) is finding  a function $u$ such that
	\[ 
	F( x, u( x) , \nabla u( x) , \ldots, \nabla^{k} u( x) ) =0
	\]
	where $x \in \Omega \subset  \mathbb{R}^n$, $u: \Omega \to \mathbb{R}^{N}$ 
\end{defn}
\begin{rmq}
The laplace, heat and wave equations are linear PDE's, namely, if $u,v$ are solution, then $ \alpha u + \beta v$ is a solution.	
\end{rmq}
Typically, PDE's come with boundary conditions and/or initial conditions


\end{document}	
