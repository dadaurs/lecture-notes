\documentclass[../main.tex]{subfiles}
\begin{document}
\lecture{11}{Thu 31 Mar}{Lusin's theorem}
\begin{thm}[Lusin's theorem]
	Let $\Omega$ be a measurable set, $m( \Omega) < \infty $ and $f: \Omega \to \mathbb{R}$ measurable/\\
	Then $\forall \epsilon>0$ $\exists F_\epsilon \subset \Omega$ closed s.t. $m( \Omega \setminus F_\epsilon) \leq \epsilon$ such that $f|_{F_\epsilon} $ is continuous.
\end{thm}
\begin{rmq}
$F_\epsilon$ cannot be taken open.
\end{rmq}
\begin{proof}
Using approximation of $L^{1}$ functions with smooth functions, $\exists f_n \to f 1_{ \left\{ |f| \leq M \right\} } \in L^{ \infty }( \Omega) \subset L^{1}( \Omega) $ .\\
And we choose $M$ s.t. $m(  \left\{ |f| >M \right\} ) < \frac{\epsilon}{4}$.\\
Now we can apply Egoroov to make the convergence uniform.\\
Let $C_{\frac{\epsilon}{4}} \subset \Omega$ s.t. $f_n\to f$ uniformly in $C_\epsilon$ and $m( \Omega\setminus C_{\frac{\epsilon}{4}} ) \leq \frac{\epsilon}{4}$.\\
These functions converge $ f_n |_{C_{\frac{\epsilon}{4}} } \to f 1_{ \left\{ |f| < M \right\} } $ uniformly on $C_{\frac{\epsilon}{4}} $, hence $f 1_{|f| <M} $ is continuous on $C_{\frac{\epsilon}{4}} $ hence $f$ is continuous on $C_{\frac{\epsilon}{4}} \cap \left\{ |f| <M \right\} $ 
\end{proof}

\begin{thm}[Borel sets are strictly included in Measurable sets]
	\begin{itemize}
	\item There exist non-measurable sets.
	\item There exists a Borel set which is not Borel.
	\end{itemize}
	
\end{thm}
\begin{proof}
$\forall x \in \mathbb{R}$ consider the coset $x+\mathbb{Q}$.\\
Note that $x+ \mathbb{Q} \cap [ 0,1] \neq \emptyset$.\\
\[ 
	( x+ \mathbb{Q}) \cap ( y+ \mathbb{Q}) = 
	\begin{cases}
	x+ \mathbb{Q} \text{ if } x-y \in \mathbb{Q}\\
	\emptyset \text{ if not } 
	\end{cases}
\]
Consider $ \faktor { \mathbb{R} } { \mathbb{Q}} $.\\
By the axiom of choice, pick $x_A \in A\cap [ 0,1] \forall A \in \faktor { \mathbb{R}} { \mathbb{Q}} $.\\
The Vitali set is $ \left\{ x_A: A \in \faktor { \mathbb{R}} { \mathbb{Q}}  \right\} $.\\
Define
\[ 
X= \bigcup_{ q\in [ -1,1] \cap \mathbb{Q}} q+ V
\]
Notice that $X \in [ -1,2] $ and $X \supset [ 0,1] $.\\
Indeed, let $y \in [ 0,1] $ and let $x_A$ be a representative, then $|y-x_A|$ is rational and smaller than $1$.\\
If the Vitalli set was measurable, then $X$ would be measurable, then
$ 1 \leq m( X) \leq 3$ .\\
If $q_1,q_2$ are two different rationals, then $q_1+V \cap q_2+V = \emptyset$.\\
Indeed, if there is a point of intersection, then $q_1+v_1 = q_2 +v_2$ then $v_1 \sim v_2$ which is a contradiction as the Vitalli set has one element of each coset.\\
Then $m( X) = \sum_q m( q+V) = \sum_q m( V) $ but then either $m( V) = 0 $ which is a contradicition or $m( V) >0,$ then $ m( X) = \infty $ 


\end{proof}

			

\end{document}	
