\documentclass[../main.tex]{subfiles}
\begin{document}
\lecture{21}{Wed 25 May}{Laplace Equation}
In our solution to the heat equation, we saw that a finite sum
\[ 
\sum_{}^{ } a_n \sin ( n \pi x ) e^{- ( np)^{2}t} 
\]
is a solution, but we want to take infinite linear combinations, indeed, infinite sums still are (formally) solutions.\\
Finally, we want to respect the initial condition (at $t=0$ ), this can be rewritten as 
\begin{align*}
u( x,0) = \sum_{}^{ } a_n sin( n\pi x ) = f( x) 
\end{align*}
Recognizing the fourier expansion, we can find
\[ 
a_n = 2 \int_{ 0 }^{ 1 } f( x) \sin( \pi n  x) dx
\]

Until now, we've derived this result formally (ie. with no respect for actual convergence), let us now prove a rigorous theorem, first we state a general proposition though
\begin{propo}
Let $\Omega \subset \mathbb{R}^{2}$ open and $ \left\{ u_N \right\} \subset C^{1}( \Omega, \mathbb{R}) $.\\
If $u_N \to u$ pointwise and $\nabla u_N \to v$ locally uniformly in $ \Omega$, then 
\[ 
u\in C^{1}( \Omega) \text{ and } \nabla u = v 
\]

\end{propo}
\begin{proof}
First, notice that $v$ is continuous, furthermore
\begin{align*}
u_N( x+h e_i) - u_N( x) = h \int_{  }^{  }\del_i u_N( x,t+e_i ) dt
\intertext{Taking the limit}
u( x+h e_i) - u( x) = h \int_{ 0 }^{1  }v_i ( x+ t h e_i) dt
\intertext{Now}
\lim_{h \to 0}  \frac{ u( x+he_i) - u( x) }{h } = \lim_{h \to 0} \int_{ 0 }^{ 1 } v_i( x+ th e_i) dt = v_i ( x) 
\end{align*}


\end{proof}


\begin{thm}
Let $f\in C^{4}$, then $u( x,t) $ defined as above satisfies the following 
\begin{enumerate}
\item $u\in C^{2}(  ( 0,1) \times [ 0, \infty ) ) $ and solves the heat equation.
\item $ \lim_{x \to 0 \text{ or } 1} u( x,t) = 0$ 
\item $ f( x) = \lim_{t \to  0} u( x,t) \forall x  $ 
\end{enumerate}

\end{thm}
\begin{proof}
\begin{itemize}
\item Let $u_N( x) = \sum_{n=0}^{ N} a_n \sin( n \pi x) e^{- ( n\pi)^{2} t } \in C^{ \infty }(  ( 0,1) \times [  0, \infty ) ) $ then
	\begin{align*}
	\del_t u_N = \sum_{n=1}^{ N}a_n \sin( n \pi x) ( -n^{2}\pi^{2})  e^{ - ( n\pi)^{2} t  } 
	\end{align*}
Recall that, if $f\in L^{1}$, then $\sup |a_n| \leq 2 \N { f}_{L^{1}}  $ 
Furthermore, as $N \to \infty $, $\del_t u_N \in C^{ \infty }$ converges locally uniformly for $ t>0$ to
\[ 
- \sum_{n=1}^{ \infty }a_n \sin( n \pi x ) n^{2} \pi^{2} e^{- n ^{2}\pi^{2} t} 
\]
Similarly, $\del_x u_N \in C^{ \infty }$ and converges locally uniformly to
\[ 
\sum_{n=0}^{ \infty }a_n ( n\pi)  \sin( n\pi x) e^{- n^{2}\pi^{2}t} 
\]
To prove that the limite as $N\to \infty $ still converges, using the proposition, we get $u \in C^{1}$.\\
Applying the proposition again to $ \del_x u$ inductively we in fact fidn that $u \in C^{ \infty }( ( 0,1) \times ( 0, \infty ) ) $ and 
\[ 
	\del_t^{l}\del_x ^{m} u = \sum_{n=1}^{ \infty }a_n g_n(x)  ( n\pi ^{m}) e^{ - ( \pi n)^{2} t}  ( - \pi n)^{2l}
\]
where $g_n$ is equal to either $\sin$ or $\cos$.
\item Now, since $u \in C^{1}_x ( 0,1) $ and $\del_x u( \cdot, t) $ is bounded, thus $u \in C^{0}_x ( [ 0,1] ) $.\\
	Lets compute
	\begin{align*}
	|u( x,t) - f( x)| \leq  \sum_{}^{ }|a_n| |\sin( n\pi  x) | \left( e^{- n \pi^{2} t } -1\right)|
	\intertext{The term in parentheses goes to $0$ as $t\to 0$ }
	\leq \pi^{2} \sum_{}^{ } a_n n^{2} t
	\intertext{But $a_n$ decays with order $ \frac{1}{n^{4}}$ thus this series converges to 0 as $ t \to 0$ }
	\end{align*}
	


	
\end{itemize}

\end{proof}
\subsection{The laplace equation in a box}
The Laplace equation reads
\[ 
\begin{cases}
\Delta u = 0 \text{ in  } ( 0,L) \times ( 0,M) \\
u( x,0) = \alpha( x)  \text{ and } u( x,M) = \beta( x) \\
u( 0,y) = \gamma( y) \text{ and  } u( L,y) =  \delta( y) 
\end{cases}
\]
Using separation of variables, we get
\begin{align*}
	\phi''( x) \psi( y) &= - \phi( x) \psi''( y) \\
	\frac{\phi''( x) }{\phi( x) }= - \frac{\psi''( y) }{\psi( y) }
\end{align*}
Thus, we get $ \phi''= \lambda\phi$ and $ \psi'' = - \lambda \psi$.\\
Now, if $ \lambda< 0$ we get $ \phi( x) = \alpha \sin( \sqrt{ - \lambda} x) + \beta \cos(  \sqrt{- \lambda} x) $ and if $ \lambda > 0$ we get $ \phi( x) = \gamma e^{ \sqrt{\lambda} x} + \delta e^{ - \sqrt{ \lambda} x} $ .\\
If $ \lambda>0$ we can write $\phi$ as
\[ 
	( \gamma+ \delta)  \cosh (  \sqrt{\lambda} x) + ( \gamma- \delta) \sinh ( \sqrt{ \lambda} x) 
\]
Overall, solutions will be of the form 
\begin{align*}
\left( \alpha \sinh(  \sqrt{- \lambda} x) + \beta \cosh (  \sqrt{- \lambda} x) \right) \left(  \gamma' \cosh (  \sqrt{- \lambda } y) + \delta \sinh(  \sqrt{ - \lambda } y) \right) 
\end{align*}
We can now split this into two simpler problem, suppose we can solve
\[ 
\begin{cases}
\Delta v = 0 \\
v( x,0) = \alpha( x) \text{ and } v( x,M) = \beta( x) \\
v( 0, y) =0 \text{ and } v( L,y) = 0
\end{cases}
\]
as well as
\[ 
\begin{cases}
\Delta w = 0\\
w( x,0) = 0 \text{ and } w( x,M) = 0 \\
v( 0, y) = \gamma( y)  \text{ and } v( L,y) = \delta( y) 
\end{cases}
\]
Then $u$, the solution of the original problem is $u = v+w$.\\
Let's solve the first problem by separation of variables, writing $u = \phi( x) \psi( y) $, we get
\[ 
\frac{\phi''( x) }{\phi( x)} = - \frac{\psi''( y) }{\psi ( y) }= \lambda
\]
and the initial condition implies $ \phi( 0) = \phi( L)  = 0$, thus $ \lambda = - ( \frac{n \pi}{L}) ^{2}$, thus $\phi= \alpha_n \sin (  \frac{n \pi}{L} x) $ and $ \psi_n ( x) = \xi_n \cosh (  \frac{n \pi}{L}y) +  \eta_n \sinh(  \frac{n \pi }{L}y) $.\\
Thus, the general solution is
\[ 
\sum_n \left[ \xi_n \cosh(  \frac{n \pi}{L} y) + \eta_n \sinh(  \frac{n \pi}{L} y)  \right] \sin ( \frac{n \pi}{L}x) 
\]
Now we impose our boundary solutions $ \alpha( x) = v( x,0) = \sum_{}^{ }\xi_n \sin(  \frac{n \pi}{L} x) $ and so we compute the fourier coefficients of $\alpha$ (in sines), which are given by
\[ 
\xi_n = \frac{2}{L} \int_{ 0 }^{ L }  \alpha( x) \sin ( \frac{n \pi}{L}x) dx
\]
Similarly, we find
\[ 
\eta_n = \frac{1}{\sinh(  \frac{2\pi }{L}M) } \left[ \frac{2}{L }\int_{ 0 }^{ L }\beta( x) \sin( \frac{n \pi}{L}x) dx - \xi_n \cosh (  \frac{n \pi}{L}M) \right] 
\]
\begin{propo}
If $\alpha,\beta \in L^{1}$, then $ w \in C^{ \infty }( ( 0,L) \times ( 0,M) ) $ and $\Delta w=0$ 
\end{propo}










\end{document}	
