\documentclass[../main.tex]{subfiles}
\begin{document}
\lecture{4}{Wed 09 Mar}{Lebesgue Integration}
\subsection{Lebesgue integration}
\begin{defn}[Simple functions]
	A measurable function $f: \Omega \subset \mathbb{R}^n \to \mathbb{R}$ is simple if ( $\Omega$ is measurable)  
	\begin{enumerate}
	\item $ f( \Omega) $ is a finite set
	\item $\exists c_1, \ldots, c_n \in \mathbb{R} $ and $E_1,\ldots, E_n \subset \Omega$ measurable s.t.
		\[ 
		f= \sum_{i=1}^{ n}c_i 1_{E_i} 
		\]
		
	\end{enumerate}
	
\end{defn}
\begin{proof}
Clearly $ \left\{ c_1,\ldots, c_n \right\} = f( \Omega) $, conversely, if $f( \Omega) = \left\{ c_1,\ldots, c_n \right\}$, define $E_i = f^{-1}( c_i) $ 
\end{proof}
\begin{rmq}
Note that simple functions are vector spaces
\end{rmq}
\begin{lemma}
Let $f: \Omega\to \mathbb{R}_{ \geq 0} $ be measurable. Then $\exists$ an increasing sequence $ \left\{ f_n \right\} $ converging pointwise to $f$ 
\end{lemma}
\begin{proof}
Define $f_n( x) = \sup_j \left\{ 2^{-n}J \leq \min ( f( x) , 2^{n})  \right\}  $.\\
\end{proof}
\begin{defn}
	Let $f:\Omega\to \mathbb{R}_{ \geq 0} $ be a simple function, then the lebesgue integral of $f$ is
	\[ 
	\int_\Omega f dx = \sum_{ \lambda \in f( \Omega) , \lambda \geq 0}^{ } \lambda \mu \left\{ x\in \Omega: f( x) = \lambda \right\} 
	\]
Note this definition works for general measures.	
\end{defn}
\begin{rmq}
Let $f= \sum_{i}^{ } c_i 1_{E_i} $, then 
\[ 
\int_{ \Omega }^{  } f dx = \sum_{i}^{ } c_i \mu( E_i) 
\]

The integral may be infinite.
\end{rmq}
\begin{defn}[Almost everywhere]
	A property $P( x) $ holds almost everywhere if $P( x) $ holds for every $x$ except a set of measure $0$.
\end{defn}
\begin{propo}[Properties of simple functions]
Let $f,g : \Omega\to \mathbb{R}_{ \geq 0} $ be simple functions
\begin{enumerate}
\item $0 \leq  \int_\Omega f  \leq  \infty $ and $ \int_{ \Omega }^{  }f = 0 \iff f\equiv 0$ almost everywhere.
\item $ \int_{ \Omega }^{  } f+g d\mu = \int_{ \Omega }^{  }f d\mu + \int_{ \Omega }^{  }g d\mu$ 
\item $ \lambda\int_{ \Omega }^{  } f d\mu =c \int_{ \Omega }^{  } f$ 
\item if $f \leq g$, then $\int_{ \Omega }^{  }f + \int_{ \Omega }^{  }g$ 
\end{enumerate}

\end{propo}
\begin{defn}[Lebesgue Integral of non-negative function]
Let $f: \Omega \subset \mathbb{R}^n\to \mathbb{R}_{ \geq 0} $ be measurable, we define 
\[ 
\int_{\Omega} f \coloneqq \sup \left\{ \int_{ \Omega }^{  }s dx: s \leq f, s \text{ simple }  \right\} 
\]

\end{defn}
\begin{rmq}
In fact, if $f$ is simple both definitions are compatible.
\end{rmq}
\begin{propo}
Let $f,g : \Omega \to \mathbb{R}_{ \geq 0} $ be measurable
\begin{itemize}
\item $ 0 \leq \int_{ \Omega }^{  }f \leq \infty $ and $ \int_{ \Omega }^{  }f = 0 \iff f=0$ a.e.
\item $ \int_{ \Omega }^{  } cf = c \int_{ \Omega }^{  }f$ 
\item If $f \leq g$ then $ \int_{ \Omega }^{  } f \leq \int_{ \Omega }^{  } g$ 
\item If $f=g$ a.e. then $ \int_{ \Omega }^{  }f = \int_{ \Omega }^{  }g$ 
\item if $\Omega' \subset \Omega$, then $ \int_{ \Omega' }^{  }f = \int_{ \Omega }^{  }( f 1_{\Omega'} ) $ 
\end{itemize}
We will prove additivity later on
\end{propo}
\begin{thm}[Lebesgue Monotone convergence theorem]
	Let $\Omega \subset \mathbb{R}^n$ be a measurable set and take $f_n$ an increasing sequence of functions converging pointwise to $f$.\\
Then
\[ 
\int_{ \Omega }^{  } f = \lim_{m \to  + \infty}  \int_{ \Omega }^{  }f_n
\]

\end{thm}
\begin{proof}
By definition $f( x) = \lim_{n \to  + \infty} f_n( x) = \sup_n f_n( x) $ ( since the $f_n$ are increasing).\\
Using the propositions above, we have that
\[ 
\int_{ \Omega }^{  }\sup_m f_m \geq \int_{ \Omega }^{  }f_m \quad \forall m
\]
Hence $ \int_\Omega f \geq \sup \int_{ \Omega }^{  } f_m$.\\
We claim $ \int_{ \Omega }^{  }\sup f_m \leq \sup \int_{ \Omega }^{  }f_m$.\\
It suffices to show that $\forall \epsilon$ 
\[ 
	( 1-\epsilon) \int_{ \Omega }^{  } s \leq \sup_m \int_{ \Omega }^{  }f_m \quad \forall s \leq \sup f_m \text{ simple  } 
\]
Indeed, note that $\forall x \in \Omega\exists N \coloneqq N( x) $ s.t. $f_N( x) \geq ( 1-\epsilon)  s( x)  $.\\
Let $E_n = \left\{ x\in \Omega:f_n \geq ( 1-\epsilon) s \right\} $.\\
Since $ f_n$ is increasing, $E_1 \subset E_2 \ldots$ and $ \bigcup E_i = \Omega$, hence we get
\[ 
	( 1-\epsilon) \int_{ E_m }^{  } s = \int_{ E_m }^{  }( 1-\epsilon) s \leq \int_{ E_m }^{  }f_N \leq \int_{ \Omega }^{  } f_n
\]
Taking the sup yields
\[ 
\sup_n ( 1-\epsilon) \int_{ E_n }^{  }s \leq \sup_n \int_{ \Omega }^{  }f_n
\]
Hence, we only need to show that the left hand side equals $ ( 1-\epsilon) \int_\Omega s$.\\
Indeed, the inequality $\sup_n ( 1-\epsilon) \int_{ E_n }^{  }s \leq  ( 1-\epsilon) \int_{ \Omega }^{  }s$.\\
For the other inequality, write $s= \sum 1_{F_j } c_j $, then 
\[ 
\int_{ E_n }^{  }s = \int_{ \Omega }^{  } \sum c_j 1_{E_n\cap F_j} 
\]




\end{proof}
	






\end{document}	
