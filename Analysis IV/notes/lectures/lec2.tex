\documentclass[../main.tex]{subfiles}
\begin{document}
\lecture{2}{Thu 24 Feb}{Existence of Lebesgue Measure}
\begin{crly}
$m^{*}(B)= vol( B) $ for every open box $B$.
\end{crly}
\begin{proof}
For one direction, we use monotonicity,
$m^{*}( B) \leq m^{*}( \overline{B}) = vol( B) $.\\
Furthermore, set $ B= \prod ( a_i,b_i) $, then for $\epsilon>0$, we get
\[ 
\prod [ a_i+\epsilon,b_i-\epsilon] \subset \prod_i ( a_i,b_i) \implies m^{*}( \prod [ a_i+\epsilon,b_i-\epsilon]  ) \leq \prod_i ( b_i-a_i) 
\]

\end{proof}
\begin{exemple}
	\begin{itemize}
	\item 
$m^{*}( \mathbb{R}) = \infty $ since by monotonicity, we get $m^{*}( \mathbb{R}) \geq m^{*}( [ 0,N] ) >N$ 
\item $m^{*}( \mathbb{Q}) =0 $ since
	\[ 
	m^{*}( \mathbb{Q}) \leq m^{*}( \left\{ q \right\} ) =0
	\]
	Which proves that the reals are uncountable.
	\end{itemize}
	
\end{exemple}
\subsection{Measurable sets ( again) }
We want to know whether $\forall A,E \subset \mathbb{R}^m$, the inequality
\[ 
m^{*}( A) \leq m^{*}( A\cap E) + m^{*}( A \setminus E) 
\]
generalises to an equality?\\
The inequality follows directly from countable subadditivity.
In fact equality does not hold in general.
\begin{defn}[Lebesgue Measurable set]
A set $E \subset \mathbb{R}^m$ is Lebesgue measurable if 
\[ 
m^{*}( A) = m^{*}( A\cap E) + m^{*}( A\setminus E) \forall A \subset \mathbb{R}^n
\]
Then the lebesgue measure of $E$ is defined as
\[ 
m( E) \coloneqq m^{*}( E) 
\]

\end{defn}
Note that, according to this definition, $\emptyset, \mathbb{R}^n$ are both measurable.
\begin{lemma}
Half-spaces are measurable
\end{lemma}
The proof is given as an exercise.\\
We now establish a few basic facts about measurable sets.
\begin{lemma}
\begin{itemize}
\item The complement of a measurable set is measurable
\item The translation of a measurable set is measurable, ie. $E$ measurable, $x\in \mathbb{R}^n$ implies $E+x$ measurable
\item Finite unions of measurable sets is measurable. ( as well as the intersection) 
\item Open ( as well as closed)  boxes are measurable.
\item If the outer measure of a set is $0$, then $E$ is measurable.	
\end{itemize}
\end{lemma}
\begin{proof}
	\begin{itemize}

	\item 
		\[ 
		m^{*}( A) =m^{*}( A\cap E^{c^{c}}) + m^{*}( A\cap E^{c}) 
		\]
	\item 
		Given $A$ a set and $x\in \mathbb{R}^n$, we get
		\[ 
		m^{*}( A-x) = m^{*}( A-x \cap E) +m^{*}( ( A-x) \cap E^{c}) = m^{*}( A\cap E+x) + m^{*}( A\cap E^{c}+x) = m^{* }( A) 
		\]

	\item 
		\[ 
		m^{*}( A) =m^{*}( A\cap ( E_1\cup E_2) ) + m^{*}( A\cap ( E_1\cup E_2)^{c} ) 
		\]
		
	\item Consider the union of two setsWe now bound $m^{*}( A) $ by below ( the upper bound is always true) 
		\[ 
		m^{*}( A) = m^{*}( A\cap E_1) +m^{*}( A\cap E_1^{c})
	\]
	\[ 
		=m^{*}( A\cap E_1\cap E_2) +m^{*}( A\cap E_1\cap E_2^{c}) +m^{*}( A\cap E_1^{c}\cap E_2) +m^{*}( A\cap E_1^{c}\cap E_2^{c})
	\]
	\[
		\geq m^{*}( A\cap ( E_1\cup E_2) ) + m^{*}( A\cap ( E_1\cup E_2)^{c} ) 
	\]
		The general result follows immediatly by induction on the number of sets.
	\item We get that
		\[ 
		m^{*}( A) \geq m^{*}( A\cap E) + m^{*}( A\cap E^{c}) 
		\]
		
	\item We write boxes as intersections of halfspaces
		
	\end{itemize}
	
\end{proof}
Now we want to show that the lebesgue measure is countably additive.
\begin{propo}
If $( E_j )_{j\in \mathbb{N}} $ are measurable disjoint sets, then $\bigcup_{i\in \mathbb{N}} E_i$ is measurable and
\[ 
m^{*}( \bigcup_{j\in \mathbb{N}} E_j) = \sum_{j=1}^{ \infty } m^{*}( E_j) 
\]

\end{propo}
The proof depends on a lemma
\begin{lemma}
Let $E_1,\ldots,E_n$ be measurable disjoint sets, $A \subset \mathbb{R}^m$,then
\[ 
m^{*}( A\cap ( \bigcup E_j) ) = \sum_{j=1}^{ n}m^{*}( A\cap E_j) 
\]
As a consequence of this, we get finite additivity.
\end{lemma}
\begin{proof}
For $n=2$ , we get
\[ 
m^{*}( A\cap ( E_1\cup E_2) ) =m^{*}( A\cap ( E_1\cup E_2) \cap E_1) + m^{*}( A\cap ( E_1\cup E_2) \cap E_1^{c}) 
\]
\[ 
=m^{*}( A\cap E_1) + m^{*}( A\cap E_2) 
\]
and the general case follows by induction.

\end{proof}
\begin{crly}
$E \subset F$ measurable implies $F\setminus E$ is measurable and
\[ 
m^{*}( F\setminus E) = m( F) -m( E) 
\]

\end{crly}
\begin{proof}
The set is trivially measurable since $F\setminus E = F\cap E^{c}$ 	
Using the lemma above, we get
\[ 
m^{*}( F) = m^{*}( E) +m^{*}( F\setminus E) 
\]

\end{proof}
We can now prove countable additivity
\begin{proof}
Let $E= \bigcup_{j=1} ^{ \infty } E_j$.\\
We claim that $\forall A$ 
\[ 
m^{*}( A) \geq m^{*}( A\cap E) + m^{*}( A\setminus E) 
\]
Indeed note that
\[ 
m^{*}( A\cap E) \leq \sum_{j=1}^{ \infty }m^{*}( A\cap E_J) = \sup_N \sum_{j=1}^{ N} m^{*}( A\cap E_j) 		
\]

Set $F_n = \bigcup_{j=1}^{N}E_j$, by the lemma, the finite sum above is 
\[ 
\sup_N \sum_{j=1}^{ N} m^{*}( A\cap E_j) =m^{*}( A\cap F_N) 
\]
Since $F_N \subset E$,
\[ 
m^{*}( A\setminus E) \leq m^{*}( A\setminus F_N) 
\]
Then
\[ 
m^{*}( A\cap E) +m^{*}( A\setminus E) < \sup_N m^{*}( A\cap F_N) + \underbrace{m^{*}( A\setminus E)}_{ \leq m^{*}( A\setminus F_N) }< \sup_N m^{*}( A) 	
\]
This proves that $m( E) \geq \sup_N m( F_N) = \sup_N \sum_{j=1}^{ N}m( E_j) = \sum_{j=1}^{ \infty }m( E_J) $ 

\end{proof}
\begin{lemma}[Lebesgues sets are a sigma-algebra]
If $( E_J )_J\in \mathbb{N}$ are measurable, then $\bigcup E_j$ and $\bigcap E_j$ are measurable.
\end{lemma}
\begin{proof}
\[ 
E_1\cup \ldots  = E_1 \cup ( E_2\setminus E_1) \cup ( E_3\setminus ( E_1\cup E_2) ) \ldots
\]
and the property about intersections follows from $\bigcap E_J = ( \bigcup E_J^{c})^{c}$
\end{proof}
\begin{lemma}[Open sets are measurable]
Every open set is measurable 
\end{lemma}
\begin{proof}
By an exercise, every open set is a countable union of open boxes and a countable union of measurable sets is countable by the lemma above.
\end{proof}
\subsection{A glimps on abstract measure theory and theoretical foundations of probability}
The idea of Lebesgue was to fix the measure of boxes and then extend the measure to the sigma algebra of measurable sets.
\begin{thm}[Caratheodory theorem]
	Given a set $\Omega$, $\mathcal{G}$ an algebra ( finite union of boxes), $A$ the smallest algebra containing $ \mathcal{G}$.\\
	Let $m_0: \mathcal{G}\to [ 0, \infty ] $ be a function s.t. $m( \emptyset) =0, m_0( \bigcup_{i=1}^{ \infty }A_i) = \sum_{m=1}^{ \infty } m_0( A_m) $ if $A_m \in \mathcal{G}, A_m$ disjoint and $ \bigcup A_m\in \mathcal{G}$ \\
	Then $\exists $ a measure on $A$ such that $m\vert_{ \mathcal{G}} = m_0$ and, if the measure of $m_0( \Omega) < \infty \implies m$ is unique.
\end{thm}
Furthermore
\begin{thm}
	Every probability $\mathbb{P}$ on $ \mathbb{R}^n$ gives rise to a cumulative distribution function, conversely, every cdf gives rise to a ( unique) probability measure.
\end{thm}
\subsection{The cantor set}
\begin{defn}[Cantor set]
	Consider $ [ 1,1] $, define $P_0= [ 0,1] $, $P_1= [ 0,\frac{1}{3}, ] \cup [ \frac{2}{3},1] $ and keep going.\\
By definition $P_0 \supset P_1 \ldots$, the cantor set is the intersection of all of them.
\end{defn}
There are a few nice properties of the cantor set
\begin{thm}
\begin{enumerate}
\item $P$ is compact
\item $m^{*}( P) =0$ 
\item $P$ is uncountable
\item $P$ is perfect\footnote { No point in $p$ is isolated.}  and has empty interior.
\end{enumerate}

\end{thm}











\end{document}	
