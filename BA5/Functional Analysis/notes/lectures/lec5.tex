\documentclass[../main.tex]{subfiles}
\begin{document}
\lecture{5}{Wed 26 Oct}{Examples of function spaces}
\section{Function Spaces}
\begin{defn}
	Let $( X,p) $ be a normed space and $T\neq \emptyset$ a set, then the set 
	\[ 
	B( T,X) = \left\{ f:T\to X : \sup_{t\in T} \N { f( t) } _X < \infty \right\} 
	\]
	is the set of bounded functions on $T$.
\end{defn}
\begin{lemma}
$B( T,X) $ is a normed space, it is complete if $X$ is complete.
\end{lemma}
\begin{lemma}
If $( T,\tau) $ is a topological space, then $C_b( T,X) = C^{0}( T,X) \cap B( T,X) $ is a closed linear subspace of $B$.\\
\end{lemma}
\begin{rmq}
$C^{0}( T,X) $ is a metric space but not a normed space.\\
We can still metrize uniform convergence by letting
\[ 
d( f,g) =\min (\N { f-g} _{ \infty } , 1) 
\]
\end{rmq}
\begin{lemma}
The continuous image of a compact set is compact.\\
If $f \in C^{0}( T,X) $ , $K \subset T$ compact and $X$ is normed, then $f( K) $ is bounded.\\
If $X= \mathbb{R}, K \neq \emptyset,f( K) $ has a maximum and minimum.
\end{lemma}
Now, let $K \subset T$ be compact and $\mathbb{K}= \mathbb{R}, \mathbb{C}$.\\
Consider $C^{0}( K, \mathbb{K}) $ 
\begin{defn}
	A set $\mathbb{A} \subset C^{0}( K, \mathbb{K}) $ is a subalgebra of $C^{0}$ if it is a linear subspace and closed under multiplication.\\
	It separates points if $\forall a,b \in K \exists f \in \mathbb{A}$ such that $f( a) \neq f( b) $.
\end{defn}
\begin{thm}[Stone-Weierstrass]
	If $ \mathbb{A}\subset C^{0}( K, \mathbb{R}) $ is a subalgebra that separates points, then either $ \overline{ \mathbb{A}}= C^{0}$ or $\exists x_0 \in K$ such that
	\[ 
		\overline{ \mathbb{A}} = \left\{ f\in C^{0}: f( x_0) =0 \right\} 
	\]
	
\end{thm}
\begin{lemma}
If $ \overline{\mathbb{A}}$ is a subalgebra, then $f \in \overline{\mathbb{A}}\implies |f|\in \overline{\mathbb{A}}$, then $ \min( f,g) , \max( f,g) \in \overline{\mathbb{A}}$.
\end{lemma}

The proof of this is an exercise.
\begin{lemma}
Assum $\forall x \in K \exists g \in \mathbb{A}$ such that $g( x) \neq 0$.\\
$\forall a,b \in K, \forall \lambda,\mu\in \mathbb{K},\exists g \in \mathbb{A}$ such that $g(a) =\lambda, g( b) = \mu$.
\end{lemma}
\begin{proof}
Let $h \in \mathbb{A}$ be such that $h( a) = h( b) $.\\
Let $g( x) = \alpha h( x) + \beta h^{2}( x) $, solving for $g( \alpha) = \lambda$ and $g( \beta) = \mu$, we get a linear system with determinant $h( a) h^{2}( b) - h^{2}( b) h( a) = h( a) h( b) ( h( b) -h( a) ) $ \\
Suppose $h( a) = 0$, consider $g_a\in \mathbb{A} $ such that $g_a( a) \neq 0$ and let $G( x) = \alpha g_\alpha( x) + \beta h( a) $, again solving $G(a	) = \lambda, G( b) = \mu  $, we get a linear system with non-zero determinant.
\end{proof}
We can now prove the Stone-Weierstrass theorem.
\begin{proof}
Fix $ f\in C^{0}$.\\
Pick $a \in K$ such that $\exists g \in \mathbb{A}, g( a) \neq 0$, then $g( x) \frac{f( a) }{g( a) }\in \mathbb{A} $ and is equal to $f$ at $a$.\\
Assume $\forall a \in \mathbb{K} \exists g_a \in \mathbb{A}$ such that $g_a( a) \neq 0$.\\
Fix $F\in C^{0}( K) $ and $\epsilon>0$.\\
We need to show that $\exists f\in \overline{\mathbb{A}}$ such that $ \N { F- f}_{ \infty } < \epsilon$.\\
\begin{enumerate}
\item $\forall a \in K \exists f_a \in \overline{\mathbb{A}}$ such that $F( a) = f_a( a) $ and $F( y)  < f_a ( y) +\epsilon$.\\
	We prove this later on.
\item Given 1, notice that $\forall a \in K$, let $V_a = \left\{ y \in K, f_a( y) < F( y) + \epsilon  \right\} $.\\
	Then for everry $\alpha \in V_a$, $V_a$ is open and $K= \cup V_a$ and we can refine this to a finite cover.\\
	Let $g= \min \left\{ f_{a_1} ,\ldots, f_{a_n}   \right\} $.\\
	Then $g \geq F-\epsilon; g \leq  f_a \leq  F+ \epsilon$ on $V_{a_i} \implies |F-g | \leq \epsilon$ 
\end{enumerate}
Now, we prove step 1.\\
Fix $a\in K$.\\
$\forall b \neq a \exists f_{ab} \in \overline{\mathbb{A}}$ such that $f_{ab} ( a) = F( a) , f_{ab} ( b) = F( b) $.\\
Let $V_{ab} = \left\{ y \in K: F( y) < f_{ab} ( y) + \epsilon \right\} $.\\
Let $b_1, \ldots b_M$ be such that $K = \bigcup_{i=1}^{M} V_{ab_i} $.\\
Let $f_a= \max \left\{ f_{ab_1},\ldots, f_{ab_M}   \right\} \in \overline{\mathbb{A}}$.\\
Then $f_a > F - \epsilon$ on $K$.\\

Assume now $\exists x_0$ such that $f\in \mathbb{A}\implies f( x_0) =0$.\\
Let $ \tilde{ \mathbb{A}} = \mathbb{A}+ \mathbb{R}$, then $ \overline{\tilde{\mathbb{A}}}= C^{0}$.\\
Let $F\in C^{0}, F( x_0) =0$, then $\exists f_\epsilon \in \mathbb{A}, \lambda_\epsilon \in \mathbb{R}$ such that $| f_\epsilon + \lambda_\epsilon - F|_{ \infty } < \epsilon$.\\

But $F( x_0) = f_\epsilon( x_0) = 0 \implies |\lambda_\epsilon| < \epsilon\implies | f_\epsilon - F|_{ \infty } < 2 \epsilon$ 
\end{proof}
\begin{thm}
	If $ \mathbb{A} \subset C^{0}( K, \mathbb{C}) $ is a subalgebra, separates points and $f\in \mathbb{A}\implies \overline{f} \in \mathbb{A}$.\\
	Then either $ \overline{\mathbb{A}}= C^{0}$ or $ \overline{\mathbb{A}}= C^{0}\cap \left\{ f( x_0) =0 \right\}.$ 
\end{thm}
\begin{proof}
$ \mathbb{A}_{ \mathbb{R}} = \mathbb{A}\cap C^{0}( K, \mathbb{R}), f \in \mathbb{A}\implies \re f, \im f \in \mathbb{A}_{ \mathbb{R}} $.
\end{proof}
\begin{defn}
If $\Omega \subset \mathbb{R}^{d}$ is open, then $C^{k}( \Omega, \mathbb{K})= \left\{ f : \Omega\to \mathbb{C}| f \text{ k-times continuously differentiable. }  \right\}  $.\\
We define $C^{k}(  \overline{\Omega}, \mathbb{K})= \left\{ f \in C^{k}( \Omega, \mathbb{K})| f \text{ continuously extends to the boundary. }   \right\}  $ 
\end{defn}
\begin{lemma}
$ C^{k}_b = \left\{ f \in C^{k}: D^{\alpha}f \in B \forall |\alpha| \leq k \right\} $ is a Banach space.
\end{lemma}
\begin{defn}[Hoelder Continuity]
	Let $f: \Omega\to \mathbb{K}$, $\alpha \in ( 0,1] $ , then
	\[ 
		[ f]_{\alpha} = \sup _{x,y \in \Omega, x\neq y} \frac{|f( x) - f(y )| }{|x-y|^{\alpha}}
	\]
	\[ 
	C^{k,\alpha}= \left\{ f \in C^{k}: D^{\beta}f \in C^{\alpha }\forall |\beta| \leq k \right\} 
	\]
	
\end{defn}







\end{document}	
