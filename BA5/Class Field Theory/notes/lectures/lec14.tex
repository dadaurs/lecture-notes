\documentclass[../main.tex]{subfiles}
\begin{document}
\lecture{14}{Mon 28 Nov}{Class Formations (or something)}
\begin{thm}[General Law of Reciprocity] 
	For every normal extension $L|K$, we get an isomorphism
	\[ 
	\theta_{L|K} : G_{L|K}^{\ab}\simeq H^{-2}( G_{L|K} , \mathbb{Z}) \to H^{0}( L|K) = \faktor{A_K}{N_{L|K} A_L}	
	\]
	This isomorphism is also called the Nakayma map.
\end{thm}
Recall that we have
\[ 
\theta_{L|K} ( \sigma G_{L|K}') = \left[ \prod_{\tau \in G_{L|K} } u( \tau,\sigma) \right] \cdot N_{L|K} A_L
\]
for $[u] = u_{L|K} \in H^{2}( L|K) $.\\
The inverse isomorphism is called the reciprocity.\\
We define the norm residue symbol by lifting $\theta_{L|K}^{-1}$ to $A_k $ and we get an exact sequence
\[ 
	1\to N_{L|K} A_L \to A_k \xto{( \ast,L|K) } G_{L|K}^{\ab}\to 1
\]
Ie., an element $a\in A_K$ is a norm from $L$ iff $( A,L|K) =1$.
\begin{lemma}
	Let $L|K$ be a normal extension, take $a\in A_k$ and set $[a] = a N_{L|K} A_L\in H^{0}( L|K) $, then
	\[ 
		\chi( ( a,L|K) ) = inv_{L|K} ( [ a] \cup \delta \chi) \in \frac{1}{[L:K]} \mathbb{Z} /\mathbb{Z}
	\]
	for every character $\chi\in \chi( G_{L|K}^{\ab}) = H^{1}( G_{L|K} , \mathbb{Q}/\mathbb{Z}) $.
\end{lemma}
\begin{thm}
	Let $N \supset L \supset K$ a normal tower, then $\pi\circ( ,N|K)= ( , L|K)  $ where $\pi:G_{N|K}^{\ab}\to G_{L|K}^{\ab}$.
	
\end{thm}
\begin{defn}[Norm group]
	A subgroup $I \subset A_k$ is called a normm group if there is a normal extension $L|K$ so that $I= N_{L|K} A_L$.
\end{defn}
\begin{thm}
		Let $L|K$ be a normal extension and let $L^{\ab}$ be the maximal abelian extension of $K$ contained in $L$.\\
		Then $N_{L|K} A_L= N_{L^{\ab}|K} A_{L^{\ab}} $.
\end{thm}
\begin{proof}
\[ 
A_K /N_{L|K} A_L\simeq G_{L|K} ^{\ab}\simeq G_{L^{\ab}|K} \simeq A_K /N_{L^{\ab}|K} A_{L^{\ab}} 
\]

\end{proof}
\begin{crly}
	The index $[A_K:N_{L|K} A_L] $ divides $[L|K]$ with equality if and only if $L|K$ is abelian.
\end{crly}
\begin{thm}
The map $L\mapsto I_L= N_{L|K} A_L$ gives an inclusion reversing isomorphism between abelain extensions $L|K$ and norm groups in $A_K$.
\end{thm}



\end{document}	
