\documentclass[../main.tex]{subfiles}
\begin{document}
\lecture{18}{Mon 12 Dec}{Adeles and Ideles}
\section{Adeles and Ideles}
The goal is to find the right module to construct the global class formation.\\
Let $K$ be a number field.\\
We call $v_p$ a finite place of $K$.\\
If $v: K \to \mathbb{R}$ is a real embedding of $K$, we can let $|x|_v = |v( x) |$ and then $K_v = \mathbb{R}$.\\
We call $v$ a real infinite place.\\
We can do the same thing for (pairs of)  complex embeddings $v, \overline{v}: K \to \mathbb{C}$  and $|x|_v = |v( x) |^{2}$ and we call these complex infinite places.\\
These are all the places of $K$.
\begin{lemma}
For $x\in K^{\times}$, we have
\[ 
\prod_{v} |x|_v = 1 
\]
the product is taken over all places.
\end{lemma}
From now on, $ \mathcal{S}$ is a finite set of places of $K$ containing $S_{ \infty } = \left\{ v| \infty  \right\} $.\\
We have the $s$-unis $K^{S}= \left\{ a\in K^{\times}| |a|_v =1\forall v \notin S \right\} $.\\
In particular, $K^{S_{ \infty } }= \O_K^{\times} \subset \O_K$.\\
By Dirichlets unit theorem says that this is finitely generated and of rank $\# S_{ \infty } -1$ and in fact $K^{S}$ is finitely generated and of rank $\# S-1$.\\
To unify notation, we set 
\[ 
\O_v
\begin{cases}
\O_{k_p} \text{ if  } v=v_p \text{ is finite } \\
K_v \text{ if } v| \infty 
\end{cases}
\]
and
\[ 
U_v^{( n) } =
\begin{cases}
U_{k_\beta}^{( n) } \text{ if } v= v_p \text{ is finite } \\
\mathbb{R}_+ \text{ if $v$ is a real place } \\
\mathbb{C}^{\times} \text{ if } v \text{ is a complex place } 
\end{cases}
\]
The class group
\[ 
\mathcal{C}_K = \faktor{\mathcal{J}_k} { \mathcal{P}_k }
\]
Set
\[ 
\mathbb{A}^{S}_K= \prod_{v \in S} K_v \times \prod_{v\notin S} \O_v
\]
This is a locally compact group with componentwise addition.\\
The adeles are defined by
\[ 
\mathbb{A}_K = \bigcup_{S} \mathbb{A}_K^{S} \subset \prod_v K_v
\]
\begin{defn}[Ideles]
	We define the Ideles by
	\[ 
	\mathbb{A}_K^{\times,S}= \prod_{v \in S} K^{\times}_v \times \prod_{v \notin S} \O_v^{\times} \subset \prod_v K_v^{\times}.
	\]
This is locally compact wrt the product topology.\\
	We again set
	\[ 
	\mathbb{A}_K^{\times}= \bigcup_S \mathbb{A}_K^{\times,S}
	\]
We view $K^{\times}\subset \mathbb{A}_K^{\times}$ by embedding it diagonally.	\\
The idele class group is 
\[ 
\mathbb{C}_K = \faktor{\mathbb{A}_K^{\times}}{K^{\times}}
\]

\end{defn}
\begin{thm}
	THe quotient
	\[ 
	\faktor{\mathbb{A}_K^{\times}}{\mathbb{A}_K^{\times, S_{ \infty } }}\simeq \mathcal{J}_K
	\]
	and 
	\[ 
	\faktor{ \mathbb{A}_K^{\times}}{\mathbb{A}_K^{\times,S_{ \infty } } K^{\times}}\simeq \mathcal{C}_K
	\]
	
\end{thm}
\begin{thm}
	We have
	\[ 
	\mathbb{A}_K^{\times}= \mathbb{A}_K^{\times,S}\cdot K^{\times}
	\]
	and
	\[ 
	\mathbb{C}_K^{\times}= \faktor{( \mathbb{A}_K^{\times,S}\cdot K^{\times}) }{K^{\times}}= \mathbb{A}_K^{\times,S} /K^{S}
	\]
	for some sufficiently large set $S$.
\end{thm}
\begin{rmq}
\begin{enumerate}
\item $ |a|_{ \mathbb{A}} = \prod_v |a_v|_v$ 
\item $ \mathbb{A}_K^{1}= \left\{ a\in \mathbb{A}_K^{\times}| |a|_{\mathbb{A}} =1 \right\} $ 
\end{enumerate}

\end{rmq}



\end{document}	
