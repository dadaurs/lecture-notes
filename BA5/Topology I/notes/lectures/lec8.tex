\documentclass[../main.tex]{subfiles}
\begin{document}
\lecture{8}{Wed 02 Nov}{still excision}
I missed the first half of the lecture because I overslept, so there is a part missing here.
\begin{lemma}
The map of chain complexes $S^{U}( X) \to S( X) $ is a quasi-isomorphism
\end{lemma}
\begin{proof}
Let $ \sum_{k}^{ } a_k \sigma_k\in S_N X $ be in the kernel of $\delta_n$, ie. $ [ \sum_k a_k \sigma_k] $ representes an element in $H_n( X) $.\\
To see surjectivity, notice that
\[ 
Sd( \sum_k a_k \sigma_k) = \sum_k a_k\sigma_k - h( \delta( \sum a_k \sigma_k) ) - \delta h( \sum a_k \sigma_) 
\]
The middle term is 0 by our hypothesis, and we see that $Sd( \sum_k a_k\sigma_k) $ is representing the same element in homology.\\
We can apply $Sd$ arbitrarily many times and hence $Sd^{l}( \sum_k a_k\sigma_k) $ is too.

Now, for every $k$, $\sigma_k^{-1}A^{\circ}, \sigma_k^{-1}( X\setminus \overline{B}) $ forms an open covering of $\Delta_n$.\\
There exists a Lebesgue Number $\epsilon_k>0$ for this open cover.\\
There is an $l_k \in \mathbb{N}$ such that $( \frac{n}{n+1})^{l_k}< \frac{\epsilon_k}{ \sqrt{2} }$.\\
Now, for any simplex $\tau$ in the barycentric subdivision of $\Delta_n$  $diam \tau \leq \epsilon$.\\
Now $( \sigma_k)_\ast ( \tau)  \subset X\setminus \overline{B}$ or $ \subset A^{\circ}$.\\
Setting $l= \max_k l^{_k}$, we see that $Sd^{l}( \sum a_k \sigma_k) $ is a preimage for $\sum a_k \sigma_k$.\\
For injectivity, let $\sum a_k \sigma_k$ be an element of $S^{U}_\bullet X$ in $\ker \delta$ which is in the image of $\delta$.\\
Let $\sum b_j \sigma_j $ be such that $\delta( \sum b_j \tau_j ) = \sum a_k \sigma_k $.\\
There exists an $m$ such that $Sd^{m}( \sum b_j \tau_j )\in S^{U}X $, now $\delta Sd^{m}( \sum b_j \tau_j ) = Sd^{m}( \sum a_k \sigma_k ) $.\\
Thus, $\sum a_k \sigma_k$ is a boundary.\\
This concludes the proof.
\end{proof}
We can now use that a quasi-isomorphism of complexes of free abelian groups has a homotopy inverse.

\end{document}	
