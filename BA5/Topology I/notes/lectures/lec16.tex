\documentclass[../main.tex]{subfiles}
\begin{document}
\lecture{16}{Wed 30 Nov}{Cellular homology coincides with it's ordinary homology theory}
\begin{thm}
	Given an ordinary homology theory $( h_\ast,\del_\ast) $ and a relative CW-complex $( X,A) $ such that either $h_\ast=H_\ast^{sing}$ or $( X,A) $ is finite dimensional then
	\[ 
	H_k^{cell}( X,A) \simeq h_k( X,A) 
	\]
	
\end{thm}
\begin{lemma}
\begin{enumerate}
\item $h_k( X_{n+1} ,X_n) =0$ if $k\neq n+1$ 
\item $h_k( X_n,A) =0 $ if $k>0$ 
\item If $k \leq l \leq n$,  $h_k( X_n,X_l) \to h_k( X_{n+1} ,X_l) $ is an isomorphism.
\end{enumerate}

\end{lemma}
\begin{proof}
	\begin{enumerate}
	\item As $X_n \subset X_{n+1} $ is a NDR, the map $h_k( X_{n+1} ,X_n) \to h_k ( X_{n+1} /X_n, X_n/X_n) \simeq h_k ( \bigvee_{i \in I_{n+1} S^{n+1},pt} ) =0$ 
	\item By induction, for $n=-1$, $h_k( A,A) =0$, for $n=0, h_k( X_0,A) = h_k( A\coprod_{i\in I_0} pt, A) = 0$.\\
		For the induction step, consider the LES of the triple
		\[ 
		h_k( X_{n-1} ,A) \to h_k( X_n,A) \to h_k( X_n,X_{n-1} ) 
		\]
		The leftmost and rightmost terms are 0
	\item By induction, the case $n=l$ 
		\[ 
		0=h_k( X_l,X_l) \to h_k( X_{l+1} , X_l) =0
		\]
		For the induction step, $h_k( X_n,X_l )\xto{\sim}  h_k( X_{n+1} ,X_l) \to h_k( X_{n+2} , X_l) $ 	
		
		The LES for the triple $( X_{n+2} ,X_n,X_l) $ looks like
		\[ 
		h_{k+1} ( X_{n+2} ,X_n) \to h_k(  X_n,X_l) \to h_k( X_{n+2} ,X_l) \to h_k( X_{n+2} ,X_n) 
		\]
		The LES of the triple $( X_{n+2} ,X_{n+1} ,X_n) $ 
		\[ 
		0=h_k( X_{n+1} ,X_n) \to h_k( X_{n+2} ,X_n) \to h_k( X_{n+2} ,X_{n+1} ) =0
		\]
		Hence $h_k( X_{n+2} ,X_n) =0$.\\
		Thus, the rightmost and leftmost terms in the sequence
		\[ 
		h_{k+1} ( X_{n+2} ,X_n) \to h_k(  X_n,X_l) \to h_k( X_{n+2} ,X_l) \to h_k( X_{n+2} ,X_n) 
		\]
		vanish and we get the desired isomorphism.
	\end{enumerate}
\end{proof}
We can now prove that homology and cellular homology coincide on nice enough CW-complexes.
\begin{proof}
Recall that the cellular homology was defined as the homology of
\[ 
	h_{k+1} ( X_{k+1} ,X_k) \xto{\del} h_k( X_k,X_{k-1} ) \xto{\del} h_{k-1} ( X_{k-1} ,X_{k-2} )
\]

\end{proof}


\end{document}	
