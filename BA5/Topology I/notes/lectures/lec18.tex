\documentclass[../main.tex]{subfiles}
\begin{document}
\lecture{18}{Mon 12 Dec}{homotopy theory}
Consider a pointed pair of topological spaces $( X,A, a_0) $, ie. $a_0\in A \subset X$.\\
\begin{defn}[Homotopy group]
	The $n$-th homotopy group of a pointed space $( X,x_0) $ 
	\[ 
	\pi_n( X,x_0) = \left\{ \text{ cont. maps $( S^{n},e_1) \to ( X,x_0) $  }  \right\} / \text{ base point preserving homotopy } 
	\]
Alternatively, one could define
\[ 
\pi_n( X,x_0) = \left\{ \text{ cont maps } ( D^{n},S^{n-1}) \to ( X,x_0)   \right\} / \text{ homotopy of pairs. } 
\]

\end{defn}
\begin{defn}[Relative homotopy groups]
	The $n$-th relative homotopy group of a pointed pair of topological spaces $ ( X, A,a_0) $ is a 
	\[ 
	\pi_n( X,A,x_0) = \left\{ ( D^{n},S^{n-1}, a_0)\to ( X,A,a_0)   \right\} / \text{ homotopy of triples } 
	\]
	
\end{defn}
\subsection{Group structure on $\pi_n( X,A,a_0) $ }
We let $J^{n-1}= \del [ 0,1]^{n}\setminus [ 0,1]^{n-1}\times \left\{ 1 \right\} \subset \del [ 0,1]^{n}$.\\
It is clear that $\pi_n( X,A,x_0) \simeq \left\{ ( [ 0,1]^{n}, J^{n-1}, x_0)\to ( X,A,x_0)   \right\} / \text{ homotopy of pairs } $ 
Given two maps $f,g: [ 0,1]^{n}\to X$ (of pairs) , we want to define their concatenation.\\
\begin{defn}
	For any two $f,g : [ 0,1]^{n}\to X$ representing elements in $\pi_n( X,A,a_0) $ for $1 \leq i \leq n-1$ ( $n \geq 2$ ), we define
	\[ 
	[ f+_i g] \in \pi_n( X,A,a_0) 
	\]
as represented by the map
\[ 
	( t_1,\ldots,t_n) \mapsto
	\begin{cases}
	f( t_1,\ldots, 2t_i, \ldots,t_n) \\
	g( t_1,\ldots, 2t_i -1, t_n) 
	\end{cases}
\]
\end{defn}
\begin{lemma}
Any $+_i$ is a well defined binary operation $\pi_n( X,A,a_0) $ and all of them are defining group structures on $\pi_n( X,A,a_0) $ 
\end{lemma}
\begin{propo}
For $n \geq 3$, all the binary operations $+_i$ coincide and are abelian.\\
In the non-relative case, $\pi_2( X,x_0) $ is also abelian.
\end{propo}
\begin{lemma}
Let $S$ be a set with two binary operations $\ast_1,\ast_2: S\times S \to S$ unital with two-sided units $ e_1, e_2$ satisfying 
\[ 
	( a\ast_1 b) \ast_2 ( c\ast_1 d) = ( a\ast_2 b) \ast_1 ( c\ast_2 b) 
\]
for all $a,b,c,d\in S$.\\
Then the two units coincide, $\ast_1= \ast_2$  are commutative and associative.

\end{lemma}


\end{document}	
