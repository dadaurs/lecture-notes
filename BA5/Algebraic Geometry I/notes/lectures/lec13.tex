\documentclass[../main.tex]{subfiles}
\begin{document}
\lecture{13}{Fri 25 Nov}{Proj construction}
\section{The proj construction}
\subsection{Algebraic preliminaries}
\begin{defn}[Graded rings]
	\begin{enumerate}
	\item A $(\mathbb{N})-$ graded ring is a ring $A$ together with a decomposition $A= \bigoplus_{d=0}^{ \infty }A_d$ of it's underlying additive group such that $A_d\cdot A_{d'} \subset A_{d+d'} $.
	\item A graded morphism of graded rings is a ring morphism $\phi:A\to B$ such that $\phi( A_d) \subset B_d\forall d \geq 0$ 
	\item A graded module over a graded ring is an $A$-module $M$ together with a decomposition $M= \bigoplus_{i=0}^{ \infty } M_d $ which is compatible with the grading from $A$.
	\item A graded morphism of graded modules is what you'd expect
	\item The irrelevant ideal of a graded ring is $A_+ = \bigoplus_{d=1}^{ \infty } A_d $.
	\item An element $a\in A$ is called homogeneous if $a\in A_d$ for some $d \geq 0$
	\item An ideal $I \subset A$ is homogeneous if $I= \bigoplus_d ( I\cap A_{d} ) $ 
	\item The homogenization of an ideal $I \subset A$ 
	\end{enumerate}
\end{defn}
\begin{exemple}
For any ring $A_0, A= A_0 [ x_0,\ldots, x_n] $ becomes a graded ring if we set $A_d$ to be the subset of polynomials of degree $d$.
\end{exemple}
\begin{lemma}
Let $A$ be a graded ring, then
\begin{enumerate}
\item $A_0$ is a subring
\item If $\p \subset A$ is a prime, then so is $\p^{h} \subset A$ 
\item An ideal is homogeneous iff it can be generated by homogeneous elements
\item A homogeneous ideal $I \subset A$ is prime iff $\forall a,b \in A$ homogeneous such that $ab \in I, $ either $a\in I $ or $b \in I$ 
\item Being homogeneous is stable under taking radicals
\end{enumerate}
\end{lemma}
\begin{defn}[Localizations of graded rings]
	Let $A= \bigoplus_{d=0}^{ \infty }A_d$ be a graded ring and $M$  a graded $A$-module
	\begin{enumerate}
	\item Let $S \subset A$ be a multiplicatively closed subset consisting of homogeneous elements then the homogeneous localization of $M$ at $S$ is $( S^{-1}M)_0= \left\{ \frac{a}{b}\in S^{-1}M| \deg a= \deg b \right\} $ 
	\item If $a\in A$ is homogeneous, then $S= \left\{ a^{i} \right\} $ consists of homogeneous elements and we let $M_{( a) } = ( S^{-1}M)_0$ is the homogeneous localization of $M$ at $a$.
	\item If $p \subset A$ is a homogeneous prime ideal, we set $S= \left\{ a\in A\setminus p| a \text{ homogeneous }  \right\} $ and let $M_{( p) } = ( S^{-1}M)_0$ be the homogeneous localization of $M$ at $p$.
	\end{enumerate}
\end{defn}
\subsection{The projective spectrum}
\begin{defn}
	Let $A$ be a graded ring.
	\begin{enumerate}
	\item The projective spectrum of $A$ is $Proj A= \left\{ p \in \spec A | p \text{ homogeneous, } A_+ \not \subset p \right\}$ 
	\item The Zariski topology on $Proj A$ is defined by defining the closed subsets $V_+( I) = \left\{ p \in Proj A| I \subset p \right\} $ for subsets $I \subset A$.
	\item The principal open subsets associated to a homogeneous element $a\in A_+$ is $D_+( a)= Proj( A) \setminus V_+( \left\{ a \right\} ) = \left\{ p \in Proj A | a \notin p \right\}  $.
	\end{enumerate}
\end{defn}
\begin{lemma}
The following identities hold
\begin{enumerate}
\item For any subset $I \subset A$, we have $V_+( I) = V_+( ( I) )= V_+( ( I)^{h})  $ 
\item For any two subsets $I,J \subset A$,
	\[ 
	V_+( I) \cup V_+( J) = V_+ ( I \cap J) 
	\]
	
\item For abritrarily many $J_i$ 
	\[ 
	\cap_i V_+( J_i) = V_+( \sum_i J_i) 
	\]
	
\item 
	\[ 
	V_+( A_+ )= V_+ ( A) = \emptyset
	\]
	
\item 
	\[ 
	V_+( ( 0) ) = Proj A
	\]

\item If $a= \sum a_i \in A$ is an element and the $a_i$ are homogeneous of degree $i$, then $D( a) \cap Proj( A) = ( D( a_0)\cap Proj A ) \cup ( \bigcup D_+( a_i) ) $.
\item If $a= a_0\in A$, then $D( a_0) \cap Proj A= \bigcup_{b \in A_d, d \geq 1} D_+( a_0b) $ 
\item In particular, the Zariski topology on $Proj A$ is a topology by the above properties
\end{enumerate}

\end{lemma}
\begin{lemma}
Let $I,J$ be homogeneous ideals in a graded ring, then $V_+( I) \subset V_+( J) \iff J \cap A_+ \subset \sqrt{I} $.\\
In particular, if $a,b \in A_+$ are homogeneous, then $D_+( b) \subset D_+( a) \iff D( b) \subset D( a) $.
\end{lemma}
\begin{proof}
If $J \cap A_+ \subset \sqrt{I} $ and let $\p \in V_+( I) \implies I \subset \p \implies \sqrt{I} \subset p\implies J A_+ \subset   J\cap A_+ \subset \p \implies p \in V_+( J) $ \\
Assume $V_+( I) \subset V_+( J) $.\\
Let $\p \in \spec A$ with $I \subset \p$, then $I \subset \p^{h}$.\\
If $A_+ \not \subset p^{h}$, then $p^{h}\in V_+( I) \subset V_+( J) \implies J \subset p^{h}\implies J \subset p$.\\
If $A_+ \subset p^{h}$, then $ J \cap A_+ \subset p$.\\
Finally, $D_+( b) \subset D_+( a) \iff V_+( a) \subset V_+( b) \iff ( b) \cap A_+ \subset \sqrt{( a) } \iff ( b) \subset \sqrt{( a) } \iff D( b) \subset D( a) $ 
\end{proof}
\begin{crly}
Let $A$ be a graded ring, then $Proj A = \emptyset $ iff $A_+ \subset \sqrt{0} $.
\end{crly}
\subsection{The structure sheaf on $Proj A$ and quasi-coherent sheaves}
If $M$ is a graded $A$-module and $D_+( b) \subset D_+( a) ( \iff D( b) \subset D( a) ) $, then there is a map $M_{( a) } \to M_{( b) } $ which commutes with the maps $M_{( a) } \to M_a$ 
\begin{lemma}
If $A= \bigoplus_{d=0}^{ \infty } A_d$  a graded ring, $M= \bigoplus_{d=0}^{ \infty }M_d$ is a graded $A$-module and $a,b \in A_+$ are homogeneous with $D_+( b) \subset D_+( a) $.\\
There is a map $M_{( a) \to M_a\to M_b} $ which factors through an $A_{( a) } $ linear map $M_{( a) } \to M_{( b) } $ via the map $( M_{( a) } )_{b^{\deg a}/a^{\deg b}} \simeq M_{( b) } $.
\end{lemma}
\begin{proof}
$D( b) \subset D( a) \implies \exists n>0,  c \in A$ such that $b^{n}= ac$.\\
Since $b^{n}$ is homogeneous and $a$ is homogeneous, we can choose $c$ homogeneous.\\
Take $\frac{e}{a^{m}}\in M_{( a) } $ with $\deg e = m \deg a$.\\
Then, in $M_b$, we have
\[ 
\frac{e}{a^{m}}= \frac{c^{m}e}{c^{m}a^{m}}= \frac{c^{m}e}{b^{nm}}\in M_{( b) } 
\]
\end{proof}

In particular, we get homeomorphisms $\phi_a: D_+( a) \to \spec A_{(a) } $ 
\begin{proof}
Note that $\phi_a( p) = p A_a \cap A_{( a) } $.\\
Define $\psi_a: \spec A_{( a) } \to D_+( a) $ by sending $q$ to $\bigoplus_{d} \left\{ b \in A_d | \frac{b^{\deg a}}{a^{d}} \in q \right\} $ 
\end{proof}
\begin{defn}
	Let $A$ be a graded ring, $M$ a graded $A$-module.\\
	Let $X= Proj A$.
	\begin{enumerate}
		\item The structure sheaf $\O_X$ on $X$ is the unique sheaf of rings on $X$ wich agrees with $\widetilde{A_{( a) } }$ on $D_+( a) $ 
	\end{enumerate}
	
\end{defn}






\end{document}	
