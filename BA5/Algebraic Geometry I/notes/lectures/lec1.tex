\documentclass[../main.tex]{subfiles}
\begin{document}
\lecture{1}{Mon 10 Oct}{Intro}
\section*{Quick Motivation}
We study schemes.\\
These are objects that "look locally" like $( \spec A, A ) $.\\
Examples include
\begin{itemize}
\item $A$ itself
\item Varieties in affine or Projective 
\end{itemize}
\section{Presheaves and Sheaves}
\subsection{Presheaves}

Let $X$ be a topological space.
\begin{defn}[Presheaf]
	Let $C$ be a category. A presheaf $\mathcal{F}$ of $C$ on $X$ consists of 
	\begin{itemize}
	\item $\forall U \subset X$ open, an object in $C$ $\mathcal{F}( U) $ 
	\item $\forall V \subset U \subset X$ open, a morphism $\rho_{U,V} : \mathcal{F}( U) \to \mathcal{F}( V) $ 
	\end{itemize}
	such that 
	\begin{itemize}
	\item $\forall U$ open, $\rho_{U,U} $ is the identity on $\mathcal{F}( U) $ 
	\item Restriction maps are compatible
		\[ 
			\forall W \subset V \subset U \subset X
		\]
		open, we have $\rho_{U,V} \circ \rho_{V,W} = \rho_{U,W} $ 
	\end{itemize}
	
\end{defn}
\begin{rmq}
Usually, $C = Set, Ab, Ring, etc.$ 
\end{rmq}
In particular, we usually assume the objects in $C$ have elements.
\begin{rmq}
\begin{itemize}
\item Elements of $\mathcal{F}( U) $ are called sections of $\mathcal{F}$ over $U$.
\item $\mathcal{F}( U) $ is called the space of sections of $\mathcal{F}$ over $U$ 
\item Elements of $\mathcal{F}( X) $ are called global sections.
\item There are alternative notations for $\mathcal{F}( U) $ : $ \Gamma( U,F) $ or $H_0( F) $ 
\item The $\rho_{UV} $ are called restriction maps, for $s\in \mathcal{F}( U) $, we write $s|_V \coloneqq  \rho_{UV} ( s) $ and is called restriction of $s$ to $V$.
\end{itemize}
\end{rmq}
\begin{exemple}
\begin{itemize}
	\item For any object $A$ in $C$, we define the constant presheaf $ \underline{A}'$ defined by $\underline{A}'( U) = A $ and with restriction maps the identity.
	\item The presheaf of continuous functions: $C^{0}$.\\
		We define $C^{0}( U) \coloneqq \left\{ f:U\to \mathbb{R}| f \text{ continuous }  \right\} $ and the restriction maps are the natural restrictions.
	\item More generally, if $\pi:Y\to X$ is continuous, we can look at the presheaf of continuous sections of $\pi$, here
		\[ 
		\mathcal{F}_\pi( U) \coloneqq  \left\{ s:U \to Y | s \text{ continuous } \pi \circ s = \id \right\} 
		\]
		This example is universal in a certain sense
\end{itemize}
\end{exemple}
\begin{rmq}
Define the category $\Ouv_X $ with
\begin{itemize}
\item objects $U \subset X$ open subsets
\item morphisms $U \to V$ are either empty or the inclusion $U \to V$ if $U \subset V$ 
\end{itemize}
Then a presheaf of $C$ on $X$ is just a contravariant functor $ \Ouv_X^{op}\to C$ 
\end{rmq}
\begin{defn}[Morphism of presheaves]
	A morphism $\phi: \mathcal{F}_1 \to \mathcal{F}_2$ of presheaves on $X$ consists of a collection of morphisms $\rho( U) : \mathcal{F}_1( U) \to \mathcal{F}_2( U) $ which are natural.
\[\begin{tikzcd}
	{\mathcal{F}_1(U)} & {\mathcal{F}_2(U)} \\
	{\mathcal{F}_1(V)} & {\mathcal{F}_2(V)}
	\arrow[from=1-1, to=2-1]
	\arrow[from=1-2, to=2-2]
	\arrow["{\rho(U)}", from=1-1, to=1-2]
	\arrow["{\rho(V)}"', from=2-1, to=2-2]
\end{tikzcd}\]
\end{defn}
\begin{exemple}
\begin{itemize}
	\item Every morphism of objects $A \to B$ in $C$ yields a morphism $ \underline{A}' \to \underline{B}'$ 
	\item If $X = \mathbb{R}^{n}$, let $C^{ \infty }$ be the presheaf of smooth functions, then for every open $U$, there is an inclusion $C^{ \infty }( U) \to C^{0}( U) $ and these inclusions induce a morphism of sheaves $C^{ \infty }\to C^{0}$ 

	\item If $Y_2\xto{\pi_2}  Y_1 \xto{\pi_1} X$ are continuous, we get $\rho: \mathcal{F}_{\pi_1\circ\pi_2} \to \mathcal{F}_{\pi_1} $ by mapping a section $s \in \mathcal{F}_{\pi_1\circ\pi_2} ( U) \to \pi_2\circ s$ 
\end{itemize}
\end{exemple}
\begin{rmq}
There is an equivalence of categories 
\[ 
\text{ Presheaves of $C$ on $X$  }  \simeq Fun( \Ouv_X^{op},C) 
\]
\end{rmq}
\subsection{Sheaves}
\begin{defn}[Sheaf]
	Let $C = \Set, \Ab, \ring$ .\\
	A sheaf $\mathcal{F}$ of $\mathcal{C}$ on $X$ is a presheaf such that $\forall U \subset X $ open and all open covers $U = \bigcup_{i \in I} U_i$ 
	\begin{itemize}
	\item $\forall s,t\in \mathcal{F}( U) $ , if $s|_{U_i} = t|_{U_i} $ $\forall i \in I$ then $s=t$
	\item  $\forall \left\{ s_i \right\} $ with $s_i\in \mathcal{F}( U_i) $ and $s_i|_{U_i \cap U_j} =s_j |_{ U_i \cap U_j} $ $\forall i,j\in I$, then $\exists s \in \mathcal{F}( U) $ such that $s|_{U_i} = s_i$ 
	\end{itemize}
\end{defn}
Condition 1 is called locality and condition 2 is the gluing condition.
\begin{rmq}
	\begin{itemize}
		\item The section $s$ of the gluability condition is unique by the locality condition.
	\item If $C$ has products, then a presheaf is called a sheaf if 
		\[ 
		\mathcal{F}( U) \to\prod_{i\in I} \mathcal{F}( U_i) \rightrightarrows  \prod_{i,j \in I} \mathcal{F}( U_i \cap U_j) 
		\]
		is an equalizer diagram
		Here the first map is induced by the maps $s_i: \mathcal{F}( U) \to \mathcal{F}( U_i)  $, the two second maps are induced by, for each pair $i,j \in I$ the restrictions $\rho_{U_i, U_i\cap U_j} $ resp. $\rho_{ U_j, U_i\cap U_j} $ 
		
	\end{itemize}
\end{rmq}
\begin{exemple}
\begin{enumerate}
\item If $\mathcal{F}$ is a sheaf , let $U \emptyset  \subset X$ and $I = \emptyset$, then $\mathcal{F}( \emptyset ) $ contains at most one element
\item $C^{0}$ ( and $C^{ \infty }$ if $X = \mathbb{R}^{n}$ ) are sheaves since $\forall U \subset X$ open
	\begin{itemize}
	\item Two continuous functions $f,g:U \to \mathbb{R}$ that coincide on an open cover are equal
	\item Given an open cover $U= \bigcup_{i \in I} U_i$ and $f_i: U_i \to \mathbb{R}$, the function $f:U \to \mathbb{R} $ defined in the obvious way is continuous ( resp. smooth) because continuity ( resp. smoothness) is local.
	\end{itemize}
\end{enumerate}
\end{exemple}
\begin{defn}[Morphisms of sheaves]
	A morphism of sheaves $\rho: \mathcal{F}_1 \to \mathcal{F}_2$ is a morphism of the underlying presheaves.
\end{defn}
\begin{rmq}
\begin{itemize}
\item $PSh_C( X) $ is the category of presheaves of $C$ on $X$ 
\item $Sh_C( X) $ is the category of sheaves of $C$ on $X$ 
\end{itemize}
If $C= Ab$, we drop the index.
\end{rmq}
\begin{rmq}
There is a forgetful functor $Sh_C( X) \to PSh_C( X) $.\\
By definition, this functor is fully faithful
\end{rmq}
\subsection*{Recall}
Let $A$ be a commutative ring ( with 1), then $\Spec A$ is the set of prime ideals of $A$.\\
The closed subsets of the Zariski topology on $\Spec A$ are of the form $V( M) = \left\{ p \in \Spec A | M \subset p \right\} $.\\
A basis of this topology is given by $D( a) = \left\{ p \in \Spec A | a \notin p  \right\} $, here $a\in A$ 
\begin{defn}[Natural sheaf on Spec A]
	Let $A$ be a ring and $X= \Spec A$, then the structure sheaf $ \O_X$ on $X$ is defined by
	\[ 
	\O_X( U) = \left\{ s: U \to \coprod_{p \in \Spec A}  A_p | s \text{ satisfies i and ii }  \right\} 
	\]
	where
	\begin{enumerate}
		\item $\forall p\in U, s( p) \in A_p$
		\item $\forall p \in U, \exists a,b \in A$ and $V \subset U$ open with $p \in V \subset D( b) $ with $s( q) = \frac{a}{b}\in A_q \forall q\in V$ 
	\end{enumerate}
	and $\rho_{UV} $ are simply the ( pointwise) restrictions.
\end{defn}
\begin{rmq}
$\O_X$ is a sheaf of rings:
\begin{itemize}
\item $\O_X( U) $ is a ring with pointwise multiplication and addiiton
\end{itemize}

\end{rmq}












		
\end{document}	
