\documentclass[../main.tex]{subfiles}
\begin{document}
\lecture{2}{Tue 22 Sep}{Applications entre ensembles}
\section{Applications entre ensembles}
Plus complet dans les notes de cours.\\
\begin{defn}[Formalisation des applications]\index{Formalisation des applications}\label{defn:formalisation_des_applications}
	Soit $A, B$ deux ensembles, alors 
	\[ 
	\phi : A \to B
	\]
	On la definit comme un sous-ensemble du produit cartesien:
	\begin{align*}
		\Gamma_\phi &\subseteq A \times B\\
		\forall a &\exists ! b: ( a,b) \in \Gamma_\phi
	\end{align*}
	Une maniere de penser d'une application est comme une machine qui prend $a$ et qui sort $b$, la machine aura un fonctionnement deterministe.
\end{defn}
\begin{propr}[Propriete des applications]\index{Propriete des applications}\label{propr:propriete_des_applications}
	Soit $\phi: A \to B$
	\begin{enumerate}
		\item injective:
			\[ 
				\phi(a) = \phi(b) \iff a=b
			\]
		\item surjective
			\[ 
				\forall b \in B \exists a : \phi(a)=b
			\]
		\item bijective $\iff$ injective et bijective\\
			L'inverse
			\begin{align*}
				\phi^{-1}: B &\to A
				\iff \phi(a) &= b
			\end{align*}
			
		\item Image
			\[ 
				\phi(A) = \left\{ \phi(a) \vert a \in A \right\} \subseteq B
			\]
		\item $\phi: A \to B, \xi :B \to C$, alors
			\[ 
				( \xi \circ \phi )(a) = \xi(\phi(a))
			\]
			\sidenote{L'ordre est etrange.}
	\end{enumerate}
\end{propr}
\begin{propo}[Surjectivite de la composition]\index{Surjectivite de la composition}\label{propo:surjectivite_de_la_composition}
	(i) $\xi$ surjectif\\
	(ii) $\phi$ pas necessairement $\iff$ il existe un contre exemple.
\end{propo}
\begin{proof}
	(i) $\forall c \in C: \exists a: \xi(\phi(a))=c$\\
	Donc $\exists b :=\phi(a) \Rightarrow \xi(b)=c$\\
	(ii) 
\begin{figure}[ht]
    \centering
    \incfig{contre-exemple-injectivite}
    \caption{contre-exemple injectivite}
    \label{fig:contre-exemple-injectivite}
\end{figure}
\end{proof}
\subsection{Relations d'equivalence}
\begin{figure}[ht]
    \centering
    \incfig{schema-relation-d-equivalence}
    \caption{schema relation d equivalence}
    \label{fig:schema-relation-d-equivalence}
\end{figure}
\begin{defn}[Relations d'equivalence]\index{Relations d'equivalence}\label{defn:relations_d_equivalence}
	Une relation d'equivalence de $A$ est un sous ensemble du produit $R \subseteq A \times A$ tq.
	\begin{enumerate}
		\item (identite) $\forall a \in A: ( a,a) \in R	$\\
		\item ( reflexivite): $(a,b) \in R \iff ( b,a) \in R$ \\
		\item ( transitivite): $(a,b) \in R, ( b,c) \in R \Rightarrow ( a,c)\in R$.
	\end{enumerate}
\end{defn}
\begin{exemple}[Exemple de transitivite]
$A= \mathbb{Z}$, alors:
\[ 
	R \subseteq \mathbb{Z}\times \mathbb{Z}: ( a,b) \in R \iff m \vert a-b
\]
\begin{enumerate}
	\item $(a,a) \in R: m \vert a-a$.\\
	\item $( a,b) \in R \Rightarrow ( b,a) \in R$

		\[ 
			\Rightarrow m \vert a-b \text{  } m \vert b-a = -(a-b)
		\]
	Ce qui est equivalent.\\
\item $(a,b) \in R, ( b,c) \in R \Rightarrow ( a,c) \in R$\\
	\[ 
		m \vert a-b , m | b-c \Rightarrow m \vert ( a-b)+ ( b-c) = a-c
	\]
	
\end{enumerate}
\end{exemple}
\begin{defn}[Classes d'equivalence]\index{Classes d'equivalence}\label{defn:classes_d_equivalence}
	Soir $R \subseteq A \times A$ rel. d'equivalence. et $a \in A$.\\
	La classe d'equivalence de $a$ est
	\[ 
		R_a= \left\{ b \in A \vert ( a,b) \in R \right\} 
	\]
\end{defn}
\begin{defn}[L'ensemble quotient]\index{L'ensemble quotient}\label{defn:l_ensemble_quotient}
	L'ensemble quotient de $R$ :
	\[ 
	A / R = \left\{ R_a \vert a \in A \right\} \subseteq 2^{A}
	\]
	
\end{defn}
\begin{exemple}[Cas de relation d'equivalence]\index{Cas de relation d'equivalence}\label{exemple:cas_de_relation_d_equivalence}
	$m=3$ et $R$ la relation d'equivalence precedente.
	\[ 
	A= \mathbb{Z} = \left\{ -2,-1,0,1,2 \right\} 
	\]
	Alors:
	\begin{align*}
		R &\supseteq (0.3)\\
		  & ( 1,4)\\
		  & ( 1,7)\\
		  & ( 11,8)
	\end{align*}
	$R_a= \left\{ b \in A \vert ( a,b) \in R \right\} = \left\{ b \in \mathbb{Z} \vert 3 \vert a-b \right\}$
	Pour le cas $a=1$, on a:
	\begin{align*}
	R_1= \left\{ \ldots, -5,-2,1,4,7,\ldots \right\} = 1 + 3 \mathbb{Z}\\
	R_0= 3 \mathbb{Z}\\
	R_2 = \left\{ \ldots,-4,-1,2,5,\ldots \right\} \\
	A / R = \left\{ 3\mathbb{Z}, 3\mathbb{Z}+1, 3\mathbb{Z}+2 \right\} 
	\end{align*}
	En general, pour m arbitraire
	\[ 
		A / R = \left\{ m \mathbb{Z}, m \mathbb{Z} +1, \ldots, m\mathbb{Z} + (m-1) \right\} 
	\]
\end{exemple}
\subsection{Cardinal d'un ensemble}
La question generale est: comment mesure-t'on la taille d'un ensemble ( meme pour des ensembles infinis)?\\
\begin{defn}[Cardinal d'un ensemble]\index{Cardinal d'un ensemble}\label{defn:cardinal_d_un_ensemble}
	\begin{enumerate}
		\item $A$ et  $B$ ont le meme cardinal si il existe $\phi: A \to B$ bijection, on note $\abs{A} = \abs{B}$\\
		\item $A$ a un cardinal plus petit que $B$ si $\exists$ une injection 
			\[ 
			\psi: A \inj B
			\]
			On note $\abs{A} \leq \abs{B}$.

			Par exemple, il n'existe pas de bijection de $\mathbb{Z}$ a $\mathbb{R}$, par contre il existe une injection $\mathbb{Z} \inj \mathbb{R}$ donc $\abs{\mathbb{Z}} < \abs{\mathbb{R}}$. On dit quue $\abs{\mathbb{Z}} = \omega_0 = \aleph_0$ et on note $\abs{R}= \kappa$
	\end{enumerate}
\end{defn}
\begin{exemple}
	On veut montrer que $\abs{\mathbb{N}} = \abs{\mathbb{Z}}$ et 
	\[ 
		\phi: \mathbb{Z} \to \mathbb{N}
	\]
	\begin{align*}
	\phi :
	\begin{aligned}
		0 \leq x &\mapsto 2x\\
	0>x &\mapsto -2x-1
	\end{aligned}
	\end{align*}
	Devoir: montrer que $\phi$ est une bijection.
\end{exemple}
\begin{thm}[Cantor-Bernhard-quelquechose]
	$\abs{A} \leq \abs{B}, \abs{B} \leq \abs{A}$ alors $\abs{A} = \abs{B}$.
	Autrement dit:
	\[ 
	f: A \inj B, B \inj A \Rightarrow \exists bij A \mapsto B
	\]
	
\end{thm}
\begin{lemma}
	
	Si il existe
	\begin{align*}
	X \subseteq A\\
	X = A \setminus g(B\setminus f(X))
	\end{align*}
	Alors il existe une bijectin $A \mapsto B$
\end{lemma}
\begin{proof}
\begin{align*}
	Y_A := X \setminus A = g(Y)\\
X_B = f(X)\\
Y = B \setminus f(x)
\end{align*}

\begin{figure}[ht]
    \centering
    \incfig{preuve-fonction-bizarre}
    \caption{preuve fonction bizarre}
    \label{fig:preuve-fonction-bizarre}
\end{figure}
Union disjointe $ B =Y \bot X_B$
\end{proof}
\begin{proof}
$f: A \inj B$ et  $g: B \inj A$.\\
Il faut : $X$ tq:
\[ 
	X = A \setminus g(B \setminus f(x)) = H(X)
\]

$X  \subseteq Z  \Rightarrow f(X) \subseteq f(Z)$

\begin{align*}
	&\Rightarrow B \setminus f(x) \supseteq B \setminus f(Z)\\
	&\Rightarrow g(B\setminus f(x)) \supseteq g(B \setminus f(Z))\\
	&\Rightarrow A \setminus g(B\setminus f(x)) \supseteq g(B \setminus f(Z))\\
	&\Rightarrow  A \setminus g(B\setminus f(Z)) \subseteq A \setminus g(B\setminus f(x))\\
	&\Rightarrow H(X) \subseteq H(Z)
\end{align*}
Soit $W = \bigcap_{X \subseteq A, \text{  } H(X) \subseteq X}X$ 
Lire les notes pour voir que $W = H(W)$

\end{proof}







\end{document}	
