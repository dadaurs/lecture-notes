\documentclass[../main.tex]{subfiles}
\begin{document}
\lecture{4}{Tue 06 Oct}{mardi moitie}

\section{Théorie des Groupes}
\subsection{Groupe symmétrique de n}
Le groupe $Bij(X)$ pour $X= \left\{ 1,\ldots,n \right\} \to S_n$
\[ 
\sigma \in S_n \to
\begin{pmatrix}
	1 & 2 &\ldots n\\
	\sigma(1) & \sigma(2) &\ldots & \sigma(n)
\end{pmatrix}
\]
La multiplication ( loi de composition) est simplement la composition des applications, attention le groupe n'est pas abélien.
\[ 
\begin{pmatrix}
	1 &2 &3\\
	2 &1 &3
\end{pmatrix}
\cdot
\begin{pmatrix}
	1 & 2 &3\\
	2 &3 &1
\end{pmatrix}
= 
\begin{pmatrix}
	1 &2 &3\\
	1 & 3 &2
\end{pmatrix}
\]
Dans l'autre sens:
\[ 
\begin{pmatrix}
	1 & 2 &3\\
	2 &3 &1
\end{pmatrix}
\cdot
\begin{pmatrix}
	1 &2 &3\\
	2 &1 &3
\end{pmatrix}
= 
\begin{pmatrix}
	1 &2 &3\\
	3 &2 &1
\end{pmatrix}
\]
Les autres exemples seront contruit par une relation d'équivalence, on note
\[ 
G / R
\]
Question:\\
Quand est-ce que $G /R$ est-il un groupe?\\
Construction:\\
\[ 
	[ g]= R_g= \left\{ h \in G | ( g,l)\in R \right\} 
\]
la classe de $G$.\\
Multiplication sur $\frac{G}{R}$ \\
Soit $x,y \in G/R$, alors
\[ 
	x = [ g] ,y \in [ f]
\]


\end{document}	
