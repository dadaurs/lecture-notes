\documentclass[../main.tex]{subfiles}
\begin{document}
\lecture{7}{Tue 27 Oct}{theorie des groupes}
Supposons $o( g) \neq \infty $, alors on note
\[ 
	|\eng{g}| = 0( g) 
\]

On remarque que
\[ 
	g^{n}= ( g^{0( g) })^{r} \cdot g^{s}
\]
\begin{propo}
$\phi: G \to H$ avec 
\[ 
\ker \phi = \left\{ e  \right\} 
\]
Alors $\phi$ injectif.
\end{propo}
\begin{proof}
Soient $g,h \in G$.\\
Supposons $\phi( g) = \phi( h) $. On a 
\[ 
	\phi( g^{-1}h) = \phi( g^{-1})  \phi( h)  = \phi( g) ^{-1} \phi( h) = e
\]
Donc $g^{-1}h \in \ker \phi$
Donc
\begin{align*}
g^{-1}h= e\\
g= h
\end{align*}
\end{proof}
\subsection{L'homomorphisme $\sgn$ }
Rappelons que un cycle $\sigma \in S_n$ tel que $\exists a_1, \ldots, a_r$ éléments différents de 
\[ 
\left\{1,\ldots,n \right\} 
\]
\begin{align*}
	\sigma( a_1) = a_2\\
	\vdots\\
	\sigma( a_{r-1} ) = a_r\\
	\sigma( a_r) = a_1\\
	\sigma( i) = i \text{ sinon } 
\end{align*}
\begin{propo}
Soit $\sigma \in S_n$, avec $\sigma$ un produit de cycles disjoints de taille $\geq 2$.\\
Cette décomposition est unique modulo l'ordre des cycles
\end{propo}
\begin{propo}
$\sigma \in S_n$, alors on peut ecrire $\sigma$ comme un produit de transpositions.
\end{propo}
\begin{proof}
Il suffit de poser 
\[ 
	\sigma = ( a_1 \ldots a_r) 
\]
On peut écrire $\sigma$ comme produit de transpositions.\\
Induction sur $r$\\
\begin{itemize}
\item $r=2$ 
\end{itemize}
On peut simplement envoyer chaque élément sur son prochain.
\end{proof}
\begin{defn}[$\sgn$ ]
\[ 
\sgn : S_n \to \left\{ -1,1 \right\} 
\]
On dénote
\[ 
	\sgn \sigma = ( -1) ^{ |\left\{ ( i,j)  \in \mathbb{N}^{2} | 1 \leq j < i \leq n, \sigma( j) < \sigma( i)  \right\}| } 
\]


\end{defn}

\begin{lemma}
$\sigma \in S_n$ et
\[ 
1 \leq r < s \leq n
\]
On définit
\[ 
	\tau = \sigma( rs) 
\]
Alors
\[ 
	\sgn( \tau) = - \sgn( \sigma) 
\]

\end{lemma}
\begin{proof}
	Si on applique $( rs) $ les autres éléments restent les mêmes.\\
	On a donc que
	\[ 
		\tau( r) = \sigma( s) \text{ et } \tau( s) = \sigma( r) 
	\]
	
\begin{itemize}
\item $i\neq j \notin \left\{ r,s \right\} $ implique
	\begin{align*}
		\sigma( i) = \tau( i) \\
		\sigma( j) = \tau( j) 
	\end{align*}
Pas de changement.
\item Soit $j < r$ ou $j>s$, alors
	Donc $j$ et $r$ sont en inversion pour $\sigma$ si et seulement si $j$ et $s$ sont en inversions pour $\tau$
Donc il n'y a pas de contribution au nombre de paires d'inversions.
\item Si $r< j < s$, alors $r$ et $j$ pour $\tau$ est le même que $s$ et $j$ pour $\sigma$. ( car $\tau( r) = \sigma( s)$ et $\tau( j) = \sigma( j) $) 
\end{itemize}
et $s$ et $j$ sont en inversion pour $\tau$ si et seulement si $r$ et $j$ ne sont pas en inversion pour  $\tau$ \\
Si  $r$ et $s$ sont en inversion pour $\tau$ si et seulement si $r$
et $s$ ne sont pas en inverson pour $\sigma$
\end{proof}
\begin{crly}
$\sgn$ est un homomorphisme
\end{crly}
\begin{proof}
$\sigma, \tau \in S_n$, alors
\[ 
	\sgn( \sigma\tau) = ( -1) ^{r+s} = \sgn \sigma \sgn \tau
\]

\end{proof}







\end{document}	
