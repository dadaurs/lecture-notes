\documentclass[11pt, a4paper, twoside]{article}
\usepackage[utf8]{inputenc}
\usepackage[T1]{fontenc}
\usepackage[francais]{babel}
\usepackage{lmodern}

\usepackage{amsmath}
\usepackage{amssymb}
\usepackage{amsthm}
\usepackage{faktor}
\DeclareMathOperator*{\Hom}{Hom}
\usepackage{import}
\usepackage{pdfpages}
\usepackage{transparent}
\usepackage{xcolor}

\newcommand{\incfig}[2][1]{%
    \def\svgwidth{#1\columnwidth}
    \import{./figures/}{#2.pdf_tex}
}

\pdfsuppresswarningpagegroup=1

\begin{document}
\title{Série 6}
\author{David Wiedemann}
\maketitle
\section*{1}
Soit $g \in G$ un élément satisfaisant $g^{2}= e_G$, où $e_G$ dénote l'élément neutre.\\
On construit alors l'application de $\faktor { \mathbb{Z}} { 2\mathbb{Z}}$ vers $G$.\\
\begin{align*}
	\phi_g: & 
	\begin{array}{ll}
	\quad [ 0 ] \to e_G\\
	\quad [ 1 ] \to g
	\end{array}
\end{align*}
Montrons que tous les homomorphismes de $\faktor { \mathbb{Z}} { 2\mathbb{Z}}$ vers $G$.\\
En effet, supposons qu'il existe un homomorphisme $\phi$ tel que $\phi( 1) ^{2}\neq e_G$, alors 
\[ 
	e_G = \phi( [ 0] ) =	\phi( [ 1 ]+[ 1 ])  = \phi( [ 1] ) ^{2} \neq e_G
\]
Ce qui est une contradiction. Donc tous les homomorphismes sont de la forme $\phi_g$.\\
Dénotons par $G_2= \left\{ g \in G | g^{2} = e_G \right\} $, alors on définit l'application
\begin{align*}
	G_2 &\mapsto \Hom\left( \faktor { \mathbb{Z}} { 2\mathbb{Z}},G\right)  \\
	g &\mapsto \phi_g\\
\end{align*}
\section*{2}
Posons, comme dans l'exercice 2, $G=H=\faktor { \mathbb{Z}} { 2\mathbb{Z}}$.\\
On sait que le groupe $\left( \faktor { \mathbb{Z}} { 8\mathbb{Z}} \right)^{\times} $ contient $4$ éléments:
\[ 
1,3,5,7
\]
Clairement, 1 sera l'élément neutre, les 3 éléments restants sont tous des 2-torsions. Donc par la section 1, les seuls morphismes de $\left( \faktor { \mathbb{Z}} { 8\mathbb{Z}} \right)^{\times}$ sont de la forme $\phi_g$.\\
Définissons donc $\alpha = \phi_3$ et $\beta= \phi_5$.\\
Car $( \faktor { \mathbb{Z}} { 8\mathbb{Z}} )^{\times}$ et $\faktor { \mathbb{Z}} { 2\mathbb{Z}} $ sont des groupes abéliens, les hypothèses de l'exercice 2 s'appliquent et il existe un unique homomorphisme $\phi$ de
\begin{align*}
\faktor { \mathbb{Z}} { 2\mathbb{Z}} \times \faktor { \mathbb{Z}} { 2\mathbb{Z}} \mapsto ( \faktor { \mathbb{Z}} { 8\mathbb{Z}} )^{\times}
\end{align*}
Il ne reste plus qu'à vérifier qu'il s'agit d'un morphisme bijectif.\\
Si on traduit l'exercice 2 aux conditions de cet exercice, on obtient un diagramme tel que celui-ci:\\
\begin{figure}[h]
    \centering
    \incfig{diagramme}
\end{figure}



Dans l'exercice 2, on a trouvé que $\phi(g,h) = \phi_3( g) \cdot \phi_5( h)$ où $g \in G, h \in H$. \\
Pour montrer la bijectivité de $\phi$, on montre la surjectivité et l'injectivité.\\
Clairement, chaque élément de $( \faktor { \mathbb{Z}} { 8\mathbb{Z}} )^{\times}$ est atteint, en effet:
\begin{align*}
	\phi(0,0) = \phi_3(0 )  \cdot \phi_5( 0) = 1\\
	\phi(1,0) = \phi_3(1 )  \cdot \phi_5( 0) = 3\\
	\phi(0,1) = \phi_3(0 )  \cdot \phi_5( 1) = 5\\
	\phi(1,1) = \phi_3(1 )  \cdot \phi_5( 1) = 7\\
\end{align*}

De la liste ci-dessus, il suit également que $\phi$ est injective.








\end{document}
