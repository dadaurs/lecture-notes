\documentclass[11pt, a4paper]{article}
\usepackage[utf8]{inputenc}
\usepackage[T1]{fontenc}
\usepackage[francais]{babel}
\usepackage{lmodern}

\usepackage{amsmath}
\usepackage{amssymb}
\usepackage{amsthm}
\newtheorem{theorem}{Théorème}
\newtheorem{lemma}{Lemme}
\begin{document}
\title{Série 4}
\author{David Wiedemann}
\maketitle
On dénotera par $e_{S_n} $ l'élément neutre de $S_n$.
On utilisera le résultat démontré dans l'exercice 2 partie 3:
\[ 
	\forall S \in S_n, S \neq e_{S_n}  \implies \exists \sigma_1,\ldots,\sigma_k \text{ des permutations disjointes tel que } S = \prod_{i=1} ^{k} \sigma_i
\]

\begin{theorem}
Un cycle de longeur $k$ est d'ordre $k$.
\end{theorem}
\begin{proof}
 	Soit $i_j$ le $j$-ième élément d'un cycle $\sigma$ de longueur $k$.\\
	On voit que
	\[ 
	\sigma^{k-j} i_j = i_k
	\]
	Et par définition d'un cycle, on a donc que 
	\[ 
	\sigma^{k-j+1} i_j = i_1
	\]
	Et donc 
	\[ 
	i_j =\sigma^{j-1} i_1 = \sigma^{j-1} \sigma^{k-j+1} i_j = \sigma^{k} i_j
	\]
	Et ceci est valable pour tout $i_j$ dans le support de $\sigma$.	
\end{proof}

\begin{theorem}
	Soit $S \in S_n, S \neq e_{S_n} $ une permutation d'ordre $o(S)=k$, par exercice 2, partie 3, on peut poser que
	\[
		S = \prod_{i=1} ^{m} \sigma_i
	\]
	avec $\sigma_i, 0<i\leq m$ des cycles disjoints.\\
	Alors,
	\[ 
		o(\sigma_i) \leq k \quad \forall 0< i \leq m
	\]
\end{theorem}
\begin{proof}
	Supposons qu'il existe $\sigma_m$ tel que $n=o(\sigma_m) > k$, alors
	 \[ 
		 S^{k}= \left(\prod_{i=1} ^{m} \sigma_i\right)^{k} = \prod_{i=1} ^{m} \sigma_i^{k}
	\]
	où la deuxième égalité suit du fait que un produit de cycles disjoints commute (exercice 2 partie 2).\\
	Or $\sigma_m^{k} \neq e_{S_n}$ car par définition $n$ est le plus petit entier tel que $\sigma_m^{n}=e_{S_n}$ et $k<n$.\\
	Etant donné que $\sigma_m$ est disjoint des autres cycles, on en déduit que 
	\[ 
\prod_{i=1} ^{m} \sigma_i^{k} \neq e_{S_n} 
	\]
\end{proof}
Grâce à ce théorème, on sait que pour énumérer toutes les permutations d'ordre $n$, il suffit de considérer le produit de cycles dont l'ordre est inférieur ou égal à $n$.\\
On sait, par l'exercice 2.5 que l'ordre d'un élément de $S_n$ est borné par $n!$, il n'y a donc qu'un cas fini de possibilités à considérer.\\
Ramenons nous maintenant à $S_4$.\\
On désignera par $\binom{\bullet}{\bullet}$ les coefficients binomiaux.
\begin{itemize}
\item Le nombre d'éléments dans $S_4$ dont l'ordre est 1, est clairement 1, c'est l'identité.\\
\item Le nombre d'éléments dont l'ordre est 2 est:
	\[ 
		\binom{4}{2} + \binom{4}{2} \cdot \frac{1}{2} = 9
	\]

En effet, soit $S \in S_4$ d'ordre 2, alors on distingue 2 cas:\\
Si $S = \sigma$ un cycle de longueur 2, alors il y a clairement $\binom{4}{2}$ manières de choisir les deux éléments du cycle, et par le théorème 1, $\sigma$ est d'ordre 2.\\
Si $S= \sigma_1 \cdot \sigma_2$, avec $\sigma_1,\sigma_2$, deux cycles d'ordre 2, alors il y a $\binom{4}{2}$ manières de choisir la première permutation et $\binom{2}{2}$ manières de choisir la deuxième.\\
Il faut diviser le deuxième coefficient binomial car l'ordre dans lequel on choisit les cycles n'importe pas, ceci suit de la commutativité de cycles disjoints.\\
Il est évident que la composition de deux cycles d'ordre 2 est d'ordre 2.
\item Le nombre d'éléments dont l'ordre est 3 est:
	\[ 
		\binom{4}{3} \cdot 2 = 8
	\]

Si une permutation $S$ est d'ordre 3, elle peut être représentée par un seul cycle de longueur 3, et donc d'ordre 3.\\
Il y a clairement $\binom{4}{3}$ manières de choisir les trois éléments du cycle.\\
Il y a $\frac{3!}{3}=2$ manières d'arranger les trois termes choisis pour qu'ils forment des permutations distinctes.

\item Le nombre d'éléments dont l'ordre est 4 est:
	\[ 
		\binom{4}{4} \cdot 6 = 6
	\]
A nouveau, si $S \in S_4$ une permutation d'ordre 4, elle peut être représentée par un cycle unique de longueur 4, et donc d'ordre 4.\\
Il y a, clairement $\binom{4}{4}$ manières de choisir les éléments.\\
Il y a $\frac{4!}{4}=6$ manières d'arranger ces 4 éléments pour qu'ils forment des cycles différents.
\end{itemize}
On peut maintenant facilement vérifier que
\[ 
1 + 6 + 8 + 6 = 24
\]
On a donc bien énuméré toutes les possibilités.






\end{document}
