\documentclass[11pt, a4paper, twoside]{article}
\usepackage[utf8]{inputenc}
\usepackage[T1]{fontenc}
\usepackage[francais]{babel}
\usepackage{lmodern}

\usepackage{amsmath}
\usepackage{amssymb}
\usepackage{amsthm}
\newtheorem{theorem}{Théorème}
\newtheorem{lemma}{Lemme}
\begin{document}
\title{Série 4}
\author{David Wiedemann}
\maketitle
On dénotera par $e_{S_n} $ l'élément neutre de $S_n$.
On utilisera le résultat démontré dans l'exercice 2 partie 3:
\[ 
	\forall S \in S_n, S \neq e_{S_n}  \implies \exists \sigma_1,\ldots,\sigma_k \text{ des permutations disjointes tel que } S = \prod_{i=1} ^{k} \sigma_i
\]

\begin{theorem}
	Soit $S \in S_n, S \neq e_{S_n} $ un cycle d'ordre $o(S)=k$, on pose que
	\[
		S = \prod_{i=1} ^{m} \sigma_i
	\]
	Alors,
	\[ 
		o(\sigma_i) \leq k \quad \forall 0< i \leq m
	\]
\end{theorem}
\begin{proof}
	Supposons qu'il existe $\sigma_m$ tel que $n=o(\sigma_m) > k$, alors
	 \[ 
	S^{k}= \prod_{i=1} ^{m} \sigma_i^{k} \neq 0
	\]
	en effet, $\sigma_m^{k} \neq e_{S_n}$ car par définition $n$ est le plus petit entier tel que $\sigma_m^{n}=e_{S_n}$ et $k<n$.\\
	Etant donné que $\sigma_m$ est disjoint des autres cycles, on en déduit le théorème.
\end{proof}
On sait, par l'exercice 2.5 que l'ordre d'un élément de $S_n$ est borné par $n!$.\\




\end{document}
