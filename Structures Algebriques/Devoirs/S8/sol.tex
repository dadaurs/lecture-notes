\documentclass[11pt, a4paper, twoside]{article}
\usepackage[utf8]{inputenc}
\usepackage[T1]{fontenc}
\usepackage[francais]{babel}
\usepackage{lmodern}

\usepackage{amsmath}
\usepackage{amssymb}
\usepackage{amsthm}
\begin{document}
\title{Série 8}
\author{David Wiedemann}
\maketitle
\section*{1}
Montrons la double implication.\\
Supposons d'abord que $H$ est un sous-groupe normal de $G$, alors
\[ 
\forall h \in H, g \in G\quad ghg^{-1} \in H
\]
Montrons que ceci implique l'égalité 
\[ 
	H = \bigcup_{h \in H} C( h) 
\]
où on dénote par $C( h) $ la classe de conjugaison de $h$.\\
Montrons la double inclusion de ces ensembles.\\

Supposons qu'il existe $h \in H$ avec  $C( h) \not\subset H $, ceci implique que $khk^{-1} \in C( H) $ mais $khk^{-1} \notin H$, donc $H$ n'est pas un sous-groupe normal ce qui est une contradicition.\\
Montrons l'inclusion dans l'autre sens, comme montré dans l'exercice 3.2 de la même série, tout élément est dans sa propre classe de conjugaison, et donc
 \[ 
\forall	 h \in H, h \in C( h) 
\]
Montrons maintenant l'implication dans l'autre sens.\\
Supposons 
\[ 
	H = \bigcup_{h \in H} C( h) 
\]
Montrons que $H$ est un sous-groupe normal ( on suppose par hypothèse que c'est un sous-groupe).\\
Soit $g \in G$ et $h \in H$, il suffit de montrer que 
\[ 
ghg^{-1}\in H
\]
pour montrer que $H$ est normal.\\
Or, en effet, $ \forall g \in G, \forall h \in H, ghg^{-1}\in C( h) $ et l'élément est donc dans $\bigcup_{h \in H} C( h) $ et donc dans $H$.\\
On en déduit que $H$ est un sous-groupe normal.
\section*{2}
On sait que tout sous-groupe de $S_3$ est cyclique, donc en particulier un sous-groupe normal.\\
De plus on sait que tout élément de $S_3$ s'écrit comme un cycle d'ordre 2 ou d'ordre 3.\\
Supposons qu'il existe un sous-groupe normal $H$ de $S_3$ qui est engendré par un élément d'ordre $2$ \\
Posons $H = \langle ( ab) \rangle$,  avec $a,b \in \left\{ 1,2,3 \right\} $.\\
On vérifie alors facilement que
\[ 
	( bc) ( ab) ( bc) = ( ac) 
\]
ici, $c$ est un élément dans $ \left\{ 1,2,3 \right\} $ et qui est différent de $a$ et de $b$ 

et que donc $H$ n'est pas clos sous l'opération de conjugaison.\\
Considérons donc le sous-groupe engendré par un cycle d'ordre 3: $( abc) $.\\
Il n'existe que deux cycles d'ordre 3 dans $S_3$ : $( 123) $ et $( 132) $, on remarque que le sous-groupe engendré par $( 123) $ contient 3 éléments: $ \left\{ Id, ( 123) ,( 132)\right\} $, vérifions que $\langle ( 123) \rangle$ est normal.\\
Car $ \langle( 123) \rangle$ forme un sous-groupe, il ne faut pas verifier la conjugaison par des cycles d'ordre 3.\\
Pour la conjugaison par des cycles d'ordre 2, on a:
\begin{align*}
	( 12) ( 123) ( 12) = ( 132) \\
	( 13) ( 123) ( 13) = ( 132) \\
	( 23) ( 123) ( 23) = ( 132)
\end{align*}
et
\begin{align*}
	( 12) ( 132) ( 12) = ( 123) \\
	( 13) ( 132) ( 13) = ( 123) \\
	( 23) ( 132) ( 23) = ( 123)
\end{align*}
Donc le sous-groupe engendré par $( 123) $ est un sous-groupe normal. Or on sait que tous les sous-groupes de $S_3$ sont cycliques et que un sous-groupe engendré par un élément d'ordre 2 n'est pas normal.\\
Donc $\langle ( 123) \rangle = \langle ( 132) \rangle$ est le seul sous-groupe normal de $S_3$.\\
\section*{3}
Dans l'exercice 5, on a montré que tous les sous-groupes de $A_4$ sont engendrés par, soit un cycle d'ordre 3 ou un produit de deux cycles d'ordre 2 disjoints.\\
Montrons d'abord que un sous-groupe normal de $A_4$ ne peut pas contenir de trois cycle.
Soit $H$ un sous-groupe normal et $( abc)\in H $ un cycle d'ordre 3.
\[ 
	( cd)( ab) (abc ) ( ab) ( cd) = ( adb) 
\]
Donc $H$ contient deux trois-cycles qui ne fixent pas le même élément, et donc par l'exercice 5.3, $H$ est le sous-groupe trivial $A_4$.\\
On en déduit que un sous-groupe normal de $A_4$ ne peut que contenir des produits de cycles d'ordre 2.\\
Notons donc $F$ le sous-groupe des produits de deux 2-cycles disjoints.\\
On a montré que $F$ forme un sous-groupe de $A_4$ dans l'exercice 5.\\
Etant donné qu'il forme un sous-groupe, pour montrer que $F$ est normal, il faut seulement montrer que la conjugaison par des éléments n'appartenant pas au groupe est close.\\
Les éléments n'appartenant pas au groupe sont des 3-cycles.\\
Soit $( ab) ( cd) \in H$, alors
\begin{align*}
	( abc) ( ab) ( cd) ( acb) = ( ad) ( bc)\\
	( acb) ( ab) ( cd) ( abc) = ( ac) ( bd)\\
	( abd) ( ab) ( cd) ( adb) = ( ac) ( bd)\\
	( adb) ( ab) ( cd) ( abd) = ( ad) ( bc)\\
	( bcd) ( ab) ( cd) ( bdc) = ( ac) ( bd)\\
	( bdc) ( ab) ( cd) ( bcd) = ( ad) ( bc)\\
	( acd) ( ab) ( cd) ( adc) = ( ad) ( bc)\\
	( adc) ( ab) ( cd) ( acd) = ( ac) ( bd) 
\end{align*}
Donc $H$ est stable par conjugaison dans $A_4$.\\
Les éléments de $H$ sont:
\[ 
	H = \left\{ Id, ( 12)( 34) , ( 13) ( 24), ( 14)( 23)  \right\} 
\]
Et on sait que $H$ est le seul sous-groupe normal de $A_4$ par l'exercice 5.










\end{document}
