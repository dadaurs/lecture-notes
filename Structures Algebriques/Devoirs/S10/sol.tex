\documentclass[11pt, a4paper]{article}
\usepackage[utf8]{inputenc}
\usepackage[T1]{fontenc}
\usepackage[francais]{babel}
\usepackage{lmodern}

\usepackage{amsmath}
\usepackage{amssymb}
\usepackage{amsthm}
\DeclareMathOperator*{\om}{Hom}
\DeclareMathOperator*{\aut}{Aut}
\DeclareMathOperator*{\id}{Id}

\newtheorem{thm}{Theorème}



\begin{document}
\title{Série 10}
\author{David Wiedemann}
\maketitle
\section*{1}
Montrons que l'application est injective.\\
En effet, soient $\phi, \psi \in \om(G,H ) $ et supposons que
\[ 
	( \phi( g_1) , \ldots, \phi( g_r) ) = ( \psi( g_1) , \ldots, \psi( g_r) )
\]
Soit $a \in G$, car $G= \langle g_1, \ldots, g_r \rangle$, on sait qu'il existe des $a_1, \ldots, a_r$ que $a= \prod_i g_i^{a_i}$, on a alors que
\[ 
	\phi( a) = \phi\left( \prod_i g_i ^{a_i}\right) = \prod_i \phi( g_i ^{a_i} )  = \prod_i \phi( g_i) ^{a_i} = \prod_i \psi( g_i) ^{a_i}= \prod_i \psi\left( g_i^{a_i}\right) = \psi\left( \prod_i g_i ^{a_i}\right) = \psi( a) 
\]
On en conclut que $\phi= \psi$, et donc l'application est injective.
\section*{2}
Par l'exercice 2, on sait que $S_3 = \langle ( 12) , ( 23) \rangle$, et donc, en particulier, $S_3 = \langle ( 12) , ( 23), ( 13)\rangle $.\\
On en déduit
\[ 
	\forall \Sigma \in S_3, \exists a,b,c \in  \left\{ 0,1 \right\}  \Sigma = ( 12) ^{a} ( 23) ^{b} ( 13) ^{c}
\]
Montrons qu'un automorphisme de $\aut( S_3) $ peut seulement permuter ces 3 2-cycles.\\
On sait que $S_3$ est seulement composé de deux-cycles et de trois-cycles.\\
Supposons d'abord qu'il existe $\phi \in \aut( S_3) $ tel que
\[ 
	\phi( ( ab) ) = ( abc) 
\]
Alors, on voit que
\[ 
	\id = \phi( \id) = \phi( ( ab) ( ab) ) = \phi( ( ab) ) \phi( ( ab )) = ( abc) ( abc) = ( acb) 
\]
Ce qui est une contradiction.\\
Le raisonnement est le même si on suppose que $\phi( ( ab) ) = ( acb) $.\\
Car un produit de deux 2-cycles dans $S_3$ est un 3-cycle, on a montré que $\phi( ( ab )) $ doit être un autre 2-cycle.\\
Car $\phi$ est un automorphisme, $\phi$ doit permuter les 2-cycles.
On définit donc l'application
\[ 
	\Psi: \phi\in \aut( S_3) \to 
	\begin{pmatrix}
		( 12) & ( 23) & ( 13) \\
		\phi( 12) & \phi( 23) & \phi( 13) \\
	\end{pmatrix}
\]

On voit que $\Psi( \id_{S_3} )= \id$, où $\id_{S_3} $ est l'automorphisme identité de $S_3$ et $\id$ est la permutation identité de $S_{( 12) ,( 23) , ( 13) } $.\\
Posons $\alpha,\beta\in \aut( S_3) $, montrons que $\Psi( \alpha\circ \beta) = \Psi( \alpha)\cdot \Psi( \beta) $.\\
Par la partie 1, il suffit de considérer l'image des éléments engendrant le groupe.\\
Par définition de $\Psi$, on obtient
\begin{align*}
	\Psi ( \alpha \circ \beta) &=  
\begin{pmatrix}
	( 12) & ( 23) & ( 13)\\
	\alpha \circ \beta ( ( 12) ) & \alpha \circ \beta ( ( 23) ) & \alpha \circ \beta ( ( 13) ) 
\end{pmatrix}\\
				   &= 
				   \begin{pmatrix}
	( 12) & ( 23) & ( 13)\\
	\alpha ( ( 12) ) &\alpha ( ( 23) ) &  \alpha ( ( 13) ) 
				   \end{pmatrix}\cdot
				   \begin{pmatrix}
	( 12) & ( 23) & ( 13)\\
	\beta ( ( 12) ) &\beta ( ( 23) ) &  \beta ( ( 13) ) 
				   \end{pmatrix}\\
				   &= \Psi( \alpha) \cdot \Psi( \beta) 
\end{align*}
Où, la deuxième égalité suit par définition de la composition de permutations.\\
Montrons que cette application est injective.\\
Pour ceci, on montre que le $\ker$ de l'application est réduit à l'identité.\\
En effet, supposons que $\phi\in \aut( S_3) $ et $\Psi( \phi) = \id$, alors, pour tout $\sigma\in S_3$, il existe $x,y,z \in \left\{ 0,1 \right\} $ et $a,b,c \in \left\{ 1,2,3 \right\} $ avec $a,b,c$ deux-à-deux différents tel que
\[ 
\phi( \sigma) = \phi( ( ab)^{x}( bc) ^{y} ( ac) ^{z} ) = ( ab) ^{x} ( bc)^{y} ( ac) ^{z}	
\]
Donc $\ker \Psi = \left\{  \id_{S_3}\right\}$, et donc $\Psi$ est injective.
\section*{3}
Grace à la partie 2, il suffit de montrer que la conjugaison par un cycle correspond à la permutation de deux-cycles.\\
Vérifions d'abord la conjugaison par un deux-cycle.
\begin{align*}
	( ab) ( ab) ( ab) &= ( ab) \\
	( ab) ( bc) ( ab) &= ( ac) \\
	( ab) ( ac) ( ab) &= ( bc) 
	\end{align*}
De manière plus générale, notons $C_{( ab) } $ l'opération de conjugaison par le 2-cycle $( ab) $, et posons $( cd) $ et $( ef) $ deux 2-cycles. On voit
\[ 
	C_{( ab) } ( ( cd) ( ef) ) = ( ab) ( cd) ( ef) ( ab) = ( ab) ( cd) ( ab) ( ab ) ( ef) ( ab) = C_{( ab) } ( ( cd )) C_{( ab) } ( ( ef) ) 
\]

Si l'on conjugait par un 3-cycle, il suffit d'ecrire le 3-cycle comme un produit de deux 2-cycles, pour un 3-cycle $( abc) $ et une permutation $\sigma \in S_3$, on a alors
\[ 
	C_{( abc) } ( \sigma) = ( abc) \sigma ( acb) = ( ab) ( bc) \sigma ( bc) ( ab) = C_{( ab) } \circ C_{( bc) } ( \sigma) 
\]

Montrons que l'opération de conjugaison est un automorphisme.\\
Prenons $C_\sigma$, la conjugaison par $\sigma \in S_3$.\\
Il est clair que $C_{\sigma} ( \id) = \id $, et par les vérifications ci-dessus, il est également clair que
\[ 
	C_\sigma( \alpha \cdot \beta) = C_\sigma ( \alpha) \cdot C_\sigma( \beta) 
\]
Car $C_\sigma: S_3 \mapsto S_3$, par l'exercice 1 de la série 5, il suffit de montrer que $C_\sigma$ est injective pour montrer qu'elle est bijective.\\
Montrons que le $\ker$ de l'application est réduit à l'identité.\\
Prenons $\alpha \in S_3$, et supposons 
\[ 
	C_\sigma( \alpha) = \id
\]
Alors, on a
\[ 
\sigma \cdot \alpha \cdot \sigma^{-1} = \id \Rightarrow \sigma \cdot \alpha = \sigma \Rightarrow \alpha= \id
\]
Donc l'opération de conjugaison est un morphisme.\\




On remarque donc que la conjugaison par un deux cycle est une permutation des deux-cycles qui forment la transposition.\\
On a donc également une injection entre l'ensemble des conjugaisons et l'ensemble des automorphismes de $S_3$.\\
On a donc trouvé une injection allant de l'ensemble des automorphismes de $S_3$ vers l'ensemble des conjugaisons et une injection allant de l'ensemble des conjugaisons vers l'ensemble des automorphismes.\\
Car toutes les conjugaisons sont des automorphismes, on en conclut que tous les automorphismes de $S_3$ sont des conjugaisons.



\end{document}
