\documentclass[11pt, a4paper, twoside]{article}
\usepackage[utf8]{inputenc}
\usepackage[T1]{fontenc}
\usepackage[francais]{babel}
\usepackage{lmodern}

\usepackage{amsmath}
\usepackage{amssymb}
\usepackage{amsthm}
\newcommand\hr{
    \noindent\rule[0.5ex]{\linewidth}{0.5pt}
}
\begin{document}
\title{Série 5, Exercice 6}
\author{David Wiedemann}
\maketitle
Soit $A$ et $B$ deux groupes d'ordre 3.\\
Regardons la loi de composition $\star$ sur $A$, on va montrer que la manière de construire cette loi de composition est unique.\\

Posons $A= \left\{ e_A, a, a' \right\} $, on dénote par $e_A$ l'élément neutre.\\
Clairement $e_A \star a = a, e_A \star a'= a', a \star e_A = a, a' \star e_A = a'$ car sinon $e_A$ ne serait pas l'élément neutre.\\
Supposons maintenant que $a \star a' \neq e_A$.\\
$\quad$ Si $a \star a' = a$, alors ceci force $a \star a = e_A $ et $a' \star a' = e_A $ car sinon $a$ est $a'$ non-inversible.\\
$\quad$ Ce qui implique que
\begin{align*}
a \star a \star a' \star a' = e_A\\
a \star e_A \star a' = e_A\\
a \star a' = e_A
\end{align*}
Ce qui est une contradiction à l'hypothèse.\\

On trouve de la même manière que $a \star a' =a'$ contredit l'hypothèse que $a\star a'\neq e_A$.\\

Donc $a \star a' = e_A$.\\




Finalement, on considère $a \star a$, supposons que $a \star a= e_A$, alors $a$ admet deux inverses $a$ et $a'$ ce qui est une contradiciton.\\
Supposons donc que $a \star a = a$, alors $a \star a \star a' = a \star a'$ et donc $a = e_A$ ce qui est une contradiciton. Ceci force donc $a \star a = a'$.\\
Avec le même raisonnement, on montre que $a' \star a' = a$.\\
\hr\\
Donc, il y a une manière unique de construire une loi de composition sur un groupe d'ordre 3.\\
Afin de mieux visualiser la chose on peut représenter la loi de composition qu'on vient de construire avec un tableau.
\begin{align*}
&\begin{array}{l|lll}
\star & e_A  & a  & a' \\
\hline
e_A  & e_A  & a  & a' \\
a  & a  & a' & e_A  \\
a' & a' & e_A  & a  
\end{array}\\
&\text{ et de même pour un groupe $B= \left\{ e_B, b,b' \right\}$ d'ordre 3 \footnote{On dénote par $e_B$ à nouveau l'élément neutre.}} \\
&\begin{array}{l|lll}
\times & e_B  & b  & b' \\
\hline
e_B  & e_B  & b  & b' \\
b  & b  & b' & e_B  \\
b' & b' & e_B  & b  
\end{array}
\end{align*}
Par la construction ci-dessus, on sait que la loi de composition sur $B$ est unique.\\
On définit donc la bijection $\phi$ de $A$ vers $B$
\begin{align*}
\phi: 
	x \in A \to 
	\begin{cases}
	e_B \text{ si  } x = e_A\\
	b \text{ si } x = a\\ 
	b' \text{ si } x = a'\\ 
	\end{cases}
\end{align*}
Il reste à vérifier qu'il s'agit d'un morphisme.\\
On remarque que les tableaux ci-dessus sont symmétriques, cette symmétrie implique que $( A, \star) $ et $( B, \times) $ sont des groupes symmétriques.\\
Cette observation nous permet de réduire le nombre de cas à vérifier pour montrer que $\phi$ est un isomorphisme.\\
Pour montrer que $\phi$ est un morphisme, on vérifie toutes les combinaisons possibles.
\begin{align*}
	\phi( e_A \star a) &= e_B \times b = \phi( e_A) \times \phi( a) \\
	\phi( e_A \star a') &= e_B \times b' = \phi( e_A)  \times \phi( a')\\
	\phi( a' \star a) &= \phi(a \star a') = \phi( e_A)  = b' \times b = e_B=\phi( a') \times \phi( a) = \phi( a) \times \phi( a') \\
	\phi(a \star a) &= \phi( a') = b' = b \times b = \phi( a) \times \phi( a)\\
	\phi( a' \star a') &= \phi( a) = b = b' \times b' = \phi( a') \times \phi( a') 
\end{align*}





\end{document}
