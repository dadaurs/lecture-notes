\documentclass[11pt, a4paper, twoside]{article}
\usepackage[utf8]{inputenc}
\usepackage[T1]{fontenc}
\usepackage[francais]{babel}
\usepackage{lmodern}

\usepackage{amsmath}
\usepackage{amssymb}
\usepackage{amsthm}
\newcommand\hr{
    \noindent\rule[0.5ex]{\linewidth}{0.5pt}
}
\begin{document}
\title{Série 5, Exercice 6}
\author{David Wiedemann}
\maketitle
Soit $A$ et $B$ deux groupes d'ordre 3.\\
Regardons la loi de composition $\star$ sur $A$, la construction d'une loi de composition sur $B$.\\
Posons $A= \left\{ e_A, a, a' \right\} $, on dénote par $e_A$ l'élément neutre.\\
Clairement $e_A \star a = a, e_A \star a'= a', a \star e_A = a, a' \star e_A = a'$ car sinon $e_A$ ne serait pas l'élément neutre.\\
Supposons maintenant que $a \star a' \neq e_A$, alors $a $ n'admet pas d'inverse, donc $a \star a' = e_A$.\\
Le raisonnement est le même pour $a'$, en effet $a' \star a = e_A$.\\
Finalement, on considère $a \star a$, supposons que $a \star a= e_A$, alors $a$ admet deux inverses ce qui est une contradiciton.\\
Supposons donc que $a \star a = a$, alors $a \star a \star a' = a \star a'$ et donc $a = e_A$ ce qui est une contradiciton. Ce qui force $a \star a = a'$.\\
Avec le même raisonnement, on montre que $a' \star a' = a$.\\
\hr\\
Donc, il y a une manière unique de construire une loi de composition sur un groupe d'ordre 3.\\
On peut donc définir une table pour la composition sur $A$, ainsi
\begin{align*}
&\begin{array}{l|lll}
\star & e_A  & a  & a' \\
\hline
e_A  & e_A  & a  & a' \\
a  & a  & a' & e_A  \\
a' & a' & e_A  & a  
\end{array}\\
&\text{ et de même pour un groupe $B= \left\{ e_B, b,b' \right\}$ d'ordre 3 \footnote{On dénote par $e_B$ à nouveau l'élément neutre.}} \\
&\begin{array}{l|lll}
\times & e_B  & b  & b' \\
\hline
e_B  & e_B  & b  & b' \\
b  & b  & b' & e_B  \\
b' & b' & e_B  & b  
\end{array}
\end{align*}
Par la construction ci-dessus, on sait que la loi de composition sur $B$ est unique.\\
On définit donc la bijection $\phi$ de $A$ vers $B$
\begin{align*}
\phi: 
	x \in A \to 
	\begin{cases}
	e_B \text{ si  } x = e_A\\
	b \text{ si } x = a\\ 
	b' \text{ si } x = a'\\ 
	\end{cases}
\end{align*}
On vérifie facilement qu'il s'agit d'un isomorphisme.




\end{document}
