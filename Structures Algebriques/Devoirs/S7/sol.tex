\documentclass[11pt, a4paper, twoside]{article}
\usepackage[utf8]{inputenc}
\usepackage[T1]{fontenc}
\usepackage[francais]{babel}
\usepackage{lmodern}

\usepackage{amsmath}
\usepackage{amssymb}
\usepackage{amsthm}
\usepackage{faktor}
\newtheorem{thm}{Théorème}
\newcommand{\f}{\faktor}
\newcommand{\bbz}{\mathbb{Z}}
\DeclareMathOperator*{\pr}{pr}
\newcommand\hr{
    \noindent\rule[0.5ex]{\linewidth}{0.5pt}
}
\begin{document}
\title{Série 7}
\author{David Wiedemann}
\maketitle
\section*{1}
Supposons que $( m,n) =1$.\\
On utilise la propriété du produit universelle pour construire un morphisme entre $\f { \mathbb{Z}} { nm \mathbb{Z}} $.\\
Soit l'application $\alpha$ définie par
\[ 
\alpha : k \in \f { \mathbb{Z}} { nm\mathbb{Z}} \mapsto [ k] _n \in \f { \mathbb{Z}} { n\mathbb{Z}} 
\]
De même on définit l'application $\beta$ par
\[ 
\beta : k \in \f { \mathbb{Z}} { nm\mathbb{Z}} \mapsto [ k] _m \in \f { \mathbb{Z}} { m\mathbb{Z}} 
\]
où $[.]_m$ est la classe de congruence modulo m d' un élément.

Vérifions que ces applications sont linéaires.\\
On vérifie facilement que $\alpha$ et $\beta$ sont des morphismes, en effet 
\[ 
	\alpha( 0) = 0
\]
car 0 est congru à 0 modulo n.\\
Soit $a,b \in \f { \mathbb{Z}} { nm\mathbb{Z}} $, alors
 \[ 
	 \alpha( a + b)  = [ a+b]_n = [ a]_n + [ b]_n
\]
où la dernière égalité suit directement de la définition de classe d'équivalence modulo n.\\
On vérifie de la même manière que $\beta$ est linéaire.\\
Par la propriété du produit universel, il existe donc un morphisme $\phi$ de $\f { \mathbb{Z} } { nm\mathbb{Z}} $ vers $\f { \mathbb{Z}} { n \mathbb{Z}} \times \f { \mathbb{Z}} { m \mathbb{Z}} $.\\

De plus, on sait que $\pr_n \circ \phi = \alpha$ et $\pr_m \circ \phi =\beta$ où $\pr_n$ est la projection sur $ \f { \mathbb{Z}} { n \mathbb{Z}}$ et $\pr_m  $ la projection sur $\f { \mathbb{Z}} { m \mathbb{Z}}$.\\

Vérifions que $\phi$ est un isomorphisme.\\
Car
\[
\left|\f { \mathbb{Z}} { nm \mathbb{Z}}\right| = \left|\f { \mathbb{Z}} { n \mathbb{Z}} \times\f { \mathbb{Z}} { m \mathbb{Z}}\right| 
\]
Il suffit, par l'exercice 1 de la série 5, de vérifier que $\phi$ est injective.\\

Supposons que $\phi( k) =( 0,0) $ et montrons que ceci implique $k=0$.\\
Si $\phi( k) = (0,0) $, alors $\alpha( k) = 0 $ et $\beta( 0) $.
Donc $[k]_n = 0$ et $[k]_m= 0$, donc $k$ est un multiple de $n$ et de $m$.
Car $n$ et $m$ sont premiers entre eux, ceci implique qu'il existe $a$ tel que $k= anm$, donc $k=0$.\\

Supposons maintenant que $( n,m) \neq1$.\\
Supposons par l'absurde qu'il existe un isomorphisme $\phi$ entre $\f { \mathbb{Z}} { nm \mathbb{Z}} $ et $\f { \mathbb{Z}} { n\mathbb{Z}} \times\f { \mathbb{Z}} { m \mathbb{Z}} $.
On sait que les isomorphismes préservent les ordres des éléments, on a donc en particulier
\[ 
	o( 1) = nm.
\]
Ce qui implique
\[ 
	o(\phi( 1)) = nm
\]
Or, posons que $( n,m) = a$,  alors $a|nm$ et donc $\frac{nm}{a}$ est un entier.\\
Car $\frac{nm}{a}$ est un multiple de $n$, on trouve que pour tout $k \in \f { \mathbb{Z}} { n\mathbb{Z}}$ 
\[ 
\frac{nm}{a} \cdot [ k ]_n = [ 0 ]
\]
Cette égalité suit du fait que $\frac{nm}{a}$ contient tous les facteurs premiers de $n$ et est donc un multiple de $n$.\\
Par le même argument, $\forall k \in\f { \mathbb{Z}} { m\mathbb{Z}} $

\[ 
\frac{nm}{a} \cdot [ k ]_m = [ 0 ]
\]
Donc l'ordre de $ \phi( 1) $ est borné par $\frac{nm}{a}$ ce qui est une contradiction au fait que $\phi$ est bijective.\\
Donc, il ne peut pas y avoir d'isomorphismes si $( n,m) \neq 1$.
\section*{2}
\begin{thm}
Soit $A$ un groupe, et $B\simeq C$ deux groupes isomorphes, alors
\[ 
A \times B \simeq A \times C
\]
\end{thm}
\begin{proof}
	On sait qu'il existe un isomorphisme $\phi$ entre $B$ et $C$, on peut donc construire un isomorphisme ainsi
	\begin{align*}
	A \times B \mapsto A \times C\\
	(a,b) \mapsto ( a, \phi( b) ) 
	\end{align*}
La vérification que cette application est un isomorphisme est immédiate car $\phi$ est un isomorphisme.
\end{proof}
	

On procède par récurrence sur $r$.\\
On sait que le cas $r=2$ est vrai par la partie 1. ( Le cas $r=1$ est trivialement vrai)\\
Supposons donc vrai pour $r$ et montrons pour $r+1$.\\
On sait que $(n_r, n_i) =1$ et $( n_{r+1} , n_i) =1$ pour tout $0<i<r$. On en déduit que $( n_r \cdot n_{r+1} , n_i) =1$. Ceci suit directement de la décomposition en nombre premiers, en effet $n_{r+1} $ et $n_r$ ne partagent pas de facteurs avec les $n_i$ et donc $n_r \cdot n_{r+1} $ non plus.\\
En posant donc que $n_r\cdot n_{r+1}= p $, on trouve
\begin{align}
	\f { \mathbb{Z}} { ( n_1\ldots n_{r+1})\mathbb{Z}} \simeq \f { \mathbb{Z}} { ( n_1\ldots n_{r-1} p) \mathbb{Z}} &\simeq \prod_{i=1}^{r-1}\f { \mathbb{Z}} { n_i \mathbb{Z} }  \times \f { \mathbb{Z}} { p \mathbb{Z}} \\
							      &\simeq \prod_{i=1}^{r-1} \f { \mathbb{Z}} { n_i \mathbb{Z} }\times \f { \mathbb{Z}} { n_r n_{r+1} \mathbb{Z}} \\
							      &\simeq \prod_{i=1}^{r-1}\f { \mathbb{Z}} { n_i \mathbb{Z} }  \times \f { \mathbb{Z}} { n_r \mathbb{Z}} \times \f { \mathbb{Z}} { n_{r+1} \mathbb{Z}} \\
							      &\simeq \prod_{i=1}^{r+1}\f { \mathbb{Z}} { n_i \mathbb{Z}} 
\end{align}
où l'égalité (1) suit de l'hypothèse de récurrence, et l'égalité (3) suit du théorème 1.\\
La forme de l'isomorphisme ci-dessus est donné par la généralisation de l'isomorphisme donné en 1 à plusieurs entiers, c'est à dire
\[ 
	k \in \f { \mathbb{Z}} { ( n_1\ldots n_{r}) \mathbb{Z}} \mapsto ( [ k]_{n_1} , \ldots [k]_{n_r} ) 
\]













































































%Supposons que $( n,m) =1$, alors il existe $d_1, \ldots, d_a$ des nombres premiers tel que $\prod_{i=1}^{a}d_i = n$ et des nombres premiers $p_1,\ldots, p_b$ tel que $\prod_{i=1} ^{b}p_i = m$, avec $p_i \neq d_j \forall i, j\geq 0$\footnote{Il faut noter ici que des éléments $d_i$ et $d_j$ pourraient être les mêmes}.\\
%De plus, par l'unicité de la décomposition en nombres premiers, on sait que 
%\[ 
%\forall k \in \faktor { \mathbb{Z} } { nm \mathbb{Z}} \text{ tel que } k = \prod_{i \in D}  d_i \prod_{j \in P}  p_j
%\]
%où $P$ et $D$ sont des sous-ensembles de $\mathbb{N}$ dont les éléments maximaux ne dépassent pas $a$ respectivement $b$.\\

%Ainsi, on peut créer deux applications de $\faktor { \mathbb{Z} } { nm \mathbb{Z}} $ vers $\faktor { \mathbb{Z}} { n \mathbb{Z}} $ et $ \faktor { \mathbb{Z}} { m \mathbb{Z}} $ de la manière suivante
%\begin{align*}
	%\alpha:
	%\begin{array}{ll}
		%\faktor { \mathbb{Z} } { nm \mathbb{Z}} &\mapsto \faktor { \mathbb{Z}} { n \mathbb{Z}}  \\
	%k &\mapsto  \prod_{i \in D} d_i
	%\end{array}\\
	%\\
	%\beta:
	%\begin{array}{ll}
		%\faktor { \mathbb{Z} } { nm \mathbb{Z}} &\mapsto \faktor { \mathbb{Z}} { m \mathbb{Z}} \\
	%k &\mapsto  \prod_{i \in D} d_i
	%\end{array}\\
%\end{align*}
%De la propriété universelle des produits, il suit qu'il existe un homomorphisme $\phi$ entre $\faktor { \mathbb{Z} } { nm \mathbb{Z}} $ et $\f { \bbz}{n\bbz} \times \f {\bbz} { m\bbz} $. On sait que ce morphisme satisfait:
%\[ 
%\pr_{n} \circ \phi = \alpha \text{ et } \pr_m \circ \phi = \beta
%\]
%où $\pr_n$ et $\pr_m$ dénote les projections respectives sur $\f { \bbz} { n \bbz}  $ et $\f { \bbz } { m \bbz } $.\\

%Or on sait que
%\[ 
%\left|\f { \bbz}{n\bbz} \times \f {\bbz} { m\bbz}\right| = nm
%\]
%et que
%\[ 
	%\left|\f{\bbz}{nm\bbz}\right| = nm.
%\]
%Or, par l'exercice 1 de la série 5, on sait que $\phi$ injective implique $\phi$ bijective.\\
%Vérifions que $\phi$ est injective, donc que
%\[ 
%\ker \phi = \left\{ 0 \right\} 
%\]
%Soit $k \in \f{\bbz } { nm \bbz } $ non-nul, on sait que 
%\[ 
%k = \prod_{i \in D} d_i \prod_{j \in P} p_j
%\]
%Supposons, par l'absurde, que
%\[ 
	%\phi( k) = 0
%\]
%Alors, par la propriété universelle des produits, on a
%\[ 
%\prod_{i\in D} d_i \equiv n \mod n \text{ et } \prod_{i \in P} p_i \equiv m \mod m
%\]
%Ce qui implique 
%\[ 
%\prod_{i\in D} d_i \prod_{j \in P}  \equiv nm \mod nm 
%\]
%Et donc que l'élément $k$ est nul, ce qui contredit l'hypothèse.\\
%On a donc montré que le noyau de l'homomorphisme $\phi$ est réduit à 0, il suit que $\phi$ est injective et donc, par l'exercice 1 de la série 5, bijective.\\

%On a donc construit un homomorphisme bijectif entre $ \f {\bbz} { nm\mathbb{Z}}  $ et $ \f {\bbz} { n\mathbb{Z}} \times \f {\bbz} { m\mathbb{Z}} $, il suit que ces groupes sont isomorphes.\\

%\hr\\

%Supposons maintenant que $( n,m) \neq 1$, et supposons qu'il existe un isomorphisme
%\[ 
%\phi: \f { \mathbb{Z}} { nm \mathbb{Z}} \mapsto \f { \mathbb{Z}} { n \mathbb{Z}} \times \f { \mathbb{Z}}  { m \mathbb{Z}} 
%\]
%Or, car $\phi$ est bijectif, $\phi$ préserve l'ordre des éléments d'un groupe.\\
%Supposons, sans perte de généralité que $n< m$ ( il suffit de changer les éléments sinon). Alors, car $n$ et $m$ sont premiers entre eux, il existe $c$ tel que
%\[ 
%n^{c} = m \text{ avec $0<c<m$. } 
%\]



		








\end{document}
