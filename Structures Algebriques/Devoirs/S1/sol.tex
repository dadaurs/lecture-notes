\documentclass[11pt, a4paper, twoside]{article}
\usepackage[utf8]{inputenc}
\usepackage[T1]{fontenc}
\usepackage[french]{babel}
\usepackage{lmodern}
\usepackage{amsmath}
\usepackage{amssymb}
\usepackage{amsthm}
\usepackage{mathtools}
\newtheorem{lemma}{Lemme}
\newtheorem{thm}{Theoreme}
\begin{document}
\title{Série 1}
\author{David Wiedemann}
\maketitle
\begin{lemma}\label{degres}
On montre d'abord que:
\[ 
	\forall a(t) , b(t) \in \mathbb{R}[t], \deg (ab) = \deg(a) + \deg(b)
\]

\end{lemma}
\begin{proof}
Pour alléger la notation, on pose que: $\deg(a(t)) = A$ et $\deg(b(t)) = B$.\\
\begin{align*}
	a(t) \cdot b(t) &= \left( \sum_{i=0}^{A}a_i t^{i}\right) \cdot \left( \sum_{j=0}^{B}b_j t^{j}\right)\\
			&= a_A b_B t^{A} t^{B}\\
			&+ a_A t^{A} \sum_{j=0}^{B-1} b_j t^{j}\\
			&+ b_B t^{B} \sum_{i=0}^{A-1}a_i t^{i}\\
			&+ \left( \sum_{i=0}^{A-1} a_i t^{i}\right) \left( \sum_{j=0}^{B-1}b_j t^{j}\right)
\end{align*}
Ici, on peut clairement voir que le terme $a_A b_B t^{A} t^{B} =a_A b_B t^{A+B} $ est du plus haut degré, et donc le lemme est prouvé.
\end{proof}
\begin{lemma}\label{somme}
	Soit $a(t), b(t) \in \mathbb{R}[t]$ et $\deg(a(t))\geq\deg(b(t))$ alors $\deg(a(t) + b(t)) \leq \deg (a(t))$.
\end{lemma}
\begin{proof}
Il suffit à nouveau de développer la somme du polynôme.
On posera de plus que si $j>\deg(b(t))$, alors $b_j=0$.
\begin{align*}
	a(t) + b(t) &= \sum_{i=0}^{\deg(a(t))} a_i t^{i} + \sum_{j=0}^{\deg(b(t))} b_j t^{j}\\
		    &= \sum_{k=0}^{\deg(a(t))} (a_k + b_k) t^{k}
\end{align*}
On voit clairement que le terme de plus haut degre est le terme $t^{\deg(a(t))}$.\\
Le cas $a_k = -b_k$ impliquerait que le degré de $a(t)+ b(t) < \deg(a(t))$.
\end{proof}
\begin{lemma}\label{produit}
	Soit $a(t), b(t) \in \mathbb{R}[t]$, alors 
	\[ 
		a(t) \cdot b(t) = 0
	\]
	implique $a(t)=0$ ou $b(t)=0$.
\end{lemma}
\begin{proof}
	On utilisera le lemme \ref{degres}.\\
	Par l'absurde, assumons que $a(t) \neq 0$ et $b(t)\neq 0$, alors
	\begin{align*}
		\deg(a(t)\cdot b(t)) &= \deg(a(t)) + \deg(b(t))\\
				 &\geq 0 
	\end{align*}
	Contradiction, car $\deg(0) = - \infty $.\\
	Donc $a(t)=0$ ou $b(t)=0$
	
\end{proof}


On peut donc finalement montrer que la division Euclidienne dans $\mathbb{R}[t]$ existe et est unique.
\begin{proof}
$ $\newline
\textbf{Unicité}\\
Supposons par l'absurde que $\exists b,r,b',r' \in \mathbb{R}[t]$ et $\deg(r'),\deg(r) < \deg(q)$, tel que:
\begin{align*}
	a(t) &= q(t) \cdot b(t) + r(t)\\
	     &= q(t) b'(t) + r'(t)\\
	0 &= q(t)(b(t) - b'(t)) + (\underbrace{  r(t) - r'(t) }_{\coloneqq r''}) 
\end{align*}
Sans perte de généralité, on peut également assumer que $\deg(r) \geq \deg(r')$.\\
On utilise maintenant le lemme \ref{somme} pour remarquer que
\[ 
	\deg(r''(t)) \leq \deg(r(t)) < \deg(q(t))
\]
Par le lemme \ref{somme} et le lemme \ref{degres}, on a que 
\[ 
	q(t)(b(t) - b'(t)) = 0
\]
Finalement par le lemme \ref{produit}, $q(t)=0$ ou $b(t)-b'(t)=0$.\\
Par hypothèse, $q(t)\neq 0$, et donc $b(t) = b'(t) $, donc $b(t)$ est unique.\\
Pour prouver l'unicité de $r(t)$, on utilise l'unicité de $b(t)$.
\begin{align*}
	q(t)(b(t) - b'(t)) + ( r'(t) - r(t)) &=0\\
	r'(t) - r(t) &= 0\\
	r'(t) = r(t)
\end{align*}
Donc $b(t)$ et $r(t)$ sont uniques.\\

\textbf{Existence}\\
On procède par induction sur le degré de $a(t)$.\\
On vérifie d'abord le cas $\deg(a(t))=0$.\\
On peut distinguer deux cas:
\begin{itemize}
	\item Si $\deg(q(t))=0$, alors $a(t) = k_1 \in \mathbb{R}$ et $q(t) = k_2 \in \mathbb{R}$, alors
\[ 
	b(t) = \frac{k_1}{k_2} \text{ et } r(t)=0 \Rightarrow a(t) = b(t) \cdot q(t) + r(t)
.\]
\item Si $\deg(q(t))> 0$, alors
\[ 
	b(t) = 0  \text{ et } r(t) = a(t) \Rightarrow  a(t) = q(t) \cdot b(t) + r(t)
.\]

\end{itemize}
Par recurrence, supposons que le cas $\deg(a(t))=n$ est vrai et montrons pour $\deg(a(t))=n+1$.\\
Posons que $a(t)=a'(t) + a_{n+1} t^{n+1}$, avec $\deg(a'(t))=n$, et que $q(t) = q'(t) + q_m t^{m}$, avec $\deg(q(t)) = m$ et $\deg(q'(t))= m-1$.\\
On distingue à nouveau 2 cas:
\begin{itemize}
	\item Si, $\deg(q(t)) > \deg(a(t))$, dans ce cas, il suffit de poser que $b(t) = 0$ et que $r(t)= a(t)$, alors
		\[ 
			a(t) = q(t) \cdot b(t) + r(t)
		\]
	\item Supposons donc que $\deg(a(t)) \geq \deg(q(t))$, on peut ecrire:
		\begin{align*}
			a(t) &= a_{n+1} t^{n+1} + a'(t)\\
			     &= a_{n+1} t^{n+1} \left( \frac{q_m t^{m}}{q_{m} t^{m}}\right) + a'(t)\\
			     &= a_{n+1} t^{n+1} \left( \frac{q(t) - q'(t)}{q_{m} t^{m}}\right) + a'(t)\\
			     &= a_{n+1} t^{n+1} \frac{q(t)}{q_m t^{m}}\\
			     &- a_{n+1} t^{n+1} \frac{q'(t)}{q_m t^{m}}\\
			     &+ a'(t)\\
		\end{align*}
		
Dans cette expression, on remarque que
\[ 
	\deg\left(-\frac{a_n}{q_m} t^{n-m+1} q'(t)\right) < n+1 \text{, ceci découle du lemme \ref{produit}.}
\]
Donc le terme ci-dessus, $-\frac{a_n}{q_m} t^{n-m+1} q'(t)$, admet, par hypothèse de récurrence, $c(t)$ et $r_c(t)$ tel que
\[ 
	-\frac{a_n}{q_m} t^{n-m+1} q'(t) = c(t) \cdot q(t) + r_c(t)
\]
Le degré de $k(t)$ est aussi inférieur à $n+1$, et donc, par le même raisonnement, $\exists d(t) , r_d(t)$ tel que $k(t)= q(t)\cdot d(t) + r_d(t)$.\\
On a donc
\begin{align*}
	a(t) &= \frac{a_{n+1}}{q_m} t^{n+1-m} q(t) - \frac{a_n}{q_m} t^{n-m+1} ( q'(t)) + k(t)\\
	     &= \frac{a_{n+1}}{q_m} t^{n+1-m} q(t) - c(t) \cdot q(t) - r_c(t) + q(t) \cdot d(t) + r_d(t)\\
	     &= q(t) \left( \frac{a_{n+1}}{q_m} t^{n+1-m} - c(t) + d(t) \right) - r_c(t) + r_d(t) \\
\end{align*}

\end{itemize}
\end{proof}






\end{document}
