\documentclass[11pt, a4paper, twoside]{article}
\usepackage[utf8]{inputenc}
\usepackage[T1]{fontenc}
\usepackage[french]{babel}
\usepackage{lmodern}
\usepackage{amsmath}
\usepackage{amssymb}
\usepackage{amsthm}
\usepackage{mathtools}
\newtheorem{lemma}{Lemme}
\newtheorem{thm}{Theoreme}
\begin{document}
\title{Série 1}
\author{David Wiedemann}
\maketitle
\begin{lemma}\label{degres}
On montre d'abord que:
\[ 
	\forall a(t) , b(t) \in \mathbb{R}[t], \deg (ab) = \deg(a) + \deg(b)
\]

\end{lemma}
\begin{proof}
Pour alléger la notation, on pose que:$deg(a(t)) = A$ et $deg(b(t)) = B$.\\
\begin{align*}
	a(t) \cdot b(t) &= \left( \sum_{i=0}^{A}a_i t^{i}\right) \cdot \left( \sum_{j=0}^{B}b_j t^{j}\right)\\
			&= a_A b_B t^{A} t^{B}\\
			&+ a_A t^{A} \sum_{j=0}^{B-1} b_j t^{j}\\
			&+ b_B t^{B} \sum_{i=0}^{A-1}a_i t^{i}\\
			&+ \left( \sum_{i=0}^{A-1} a_i t^{i}\right) \left( \sum_{j=0}^{B-1}b_j t^{j}\right)
\end{align*}
Ici, on peut clairement voir que le terme $a_A b_B t^{A} t^{B} =a_A b_B t^{A+B} $ est du plus haut degré, et donc le lemme est prouvé.
\end{proof}
\begin{lemma}\label{somme}
	Soit $a(t), b(t) \in \mathbb{R}[t]$ et $\deg(a(t))>\deg(b(t))$ alors $deg(a(t) + b(t)) \leq \deg (a(t))$.
\end{lemma}
On peut donc finalement montrer que la division Euclidienne dans $\mathbb{R}[t]$ existe et est unique.
\begin{proof}
$ $\newline
\textbf{Unicité}\\
Supposons par l'absurde que $\exists b,r,b',r' \in \mathbb{R}[t]$ et $\deg(r'),\deg(r) < \deg(q)$, tel que:
\begin{align*}
	a(t) &= q(t) \cdot b(t) + r(t)\\
	     &= q(t) b'(t) + r'(t)\\
	0 &= q(t)(b(t) - b'(t)) + (\underbrace{  r(t) - r'(t) }_{\coloneqq r''}) 
\end{align*}
Sans perte de généralité, on peut également assumer que $\deg(r) \geq \deg(r')$.\\
On utilise maintenant le lemme \ref{somme} pour remarquer que
\[ 
	\deg(r''(t)) \leq \deg(r(t)) < \deg(q(t))
\]

\[ 
	q(t)(b(t) - b'(t)) = 0
\]
et donc que $b(t) - b'(t) = 0$, donc $b(t)$ est unique.\\
Donc 
\begin{align*}
	q(t)(b(t) - b'(t)) + ( r'(t) - r(t)) &=0\\
	r'(t) - r(t) &= 0\\
	r'(t) = r(t)
\end{align*}
Donc $b(t)$ et $r(t)$ sont uniques.\\

\textbf{Existence}\\
On procede par induction sur le degre de $a(t)$.\\
Clairement si $\deg(a(t))=0$ et $\deg(q(t))=0$, alors $a(t) = k_1$ et $q(t) = k_2$, alors
\[ 
	b(t) = \frac{k_1}{k_2} \Rightarrow a(t) = b(t) \cdot q(t) + 0
.\]
Donc $r(t)=0$.\\
Si $\deg(q(t))> 0$, alors
\[ 
	b(t) = 0  \text{ et } r(t) = a(t) \Rightarrow  a(t) = q(t) \cdot 0 + r(t)
.\]

Par recurrence, supposons que le cas $\deg(a(t))=n$ est vrai et montrons pour $\deg(a(t))=n+1$.\\
Posons que $a(t)=k(t) + a_{n+1} t^{n+1}$, avec $\deg(k(t))=n$ et que $q(t) = q'(t) + q_m t^{m}$, alors
Soit, $\deg(q(t)) > \deg(a(t))$, dans ce cas, il suffit de poser que $b(t) = 0$ et que $r(t)= a(t)$.\\

Supposons donc que $\deg(a(t)) \geq \deg(q(t))$, on peut ecrire:
\[ 
a(t) = \frac{a_{n+1}}{q_m} t^{n+1-m} q(t) - \frac{a_n}{q_m} t^{n-m+1} ( q'(t)) + k(t)\\
\]
Dans cette expression, on peut voir que
\[ 
	\deg\left(\frac{a_n}{q_m} t^{n-m+1} q'(t)\right) < n+1
\]
En effet, le degre de $q'(t)$ est au pire $m-1$, donc par le lemme 1, le terme de plus haut degre sera de degre $n-m+1+m-1=n$.\\
Donc le terme ci-dessus $\frac{a_n}{q_m} t^{n-m+1} q'(t)$ admet, par hypothese de recurrence, $c(t)$ et $r_c(t)$ tel que
\[ 
	\frac{a_n}{q_m} t^{n-m+1} q'(t) = c(t) \cdot q(t) + r_c(t)
\]
Le degre de $k(t)$ est aussi inferieur a $n+1$, et donc, par le meme raisonnement, $\exists d(t) , r_d(t)$ tel que $k(t)= q(t)\cdot d(t) + r_d(t)$.\\
On a donc
\begin{align*}
	a(t) &= \frac{a_{n+1}}{q_m} t^{n+1-m} q(t) - \frac{a_n}{q_m} t^{n-m+1} ( q'(t)) + k(t)\\
	     &= \frac{a_{n+1}}{q_m} t^{n+1-m} q(t) - c(t) \cdot q(t) - r_c(t) + q(t) \cdot d(t) + r_d(t)\\
	     &= q(t) \left( \frac{a_{n+1}}{q_m} t^{n+1-m} - c(t) + d(t) \right) - r_c(t) + r_d(t) \\
\end{align*}

\end{proof}






\end{document}
