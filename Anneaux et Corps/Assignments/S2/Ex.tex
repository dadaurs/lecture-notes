\documentclass[11pt, a4paper]{article}
\usepackage[utf8]{inputenc}
\usepackage[T1]{fontenc}
\usepackage[francais]{babel}
\usepackage{lmodern}
\usepackage{amsmath}
\usepackage{faktor}
\usepackage{mathtools}
\usepackage{tikz}
\usepackage{tikz-cd}
\usepackage{amssymb}
\usepackage{amsthm}
\renewcommand{\vec}[1]{\overrightarrow{#1}}
\newcommand{\del}{\partial}
\DeclareMathOperator*{\sgn}{sgn}
\DeclareMathOperator*{\id}{Id}
\DeclareMathOperator*{\im}{Im}
\DeclareMathOperator*{\re}{Re}
\DeclareMathOperator*{\vol}{Vol}
\newcommand\norm[1]{\left\vert#1\right\vert}
\newcommand\ns[1]{\left\vert\left\vert\left\vert#1\right\vert\right\vert\right\vert}
\newcommand\Norm[1]{\left\lVert#1\right\rVert}
\newcommand\N[1]{\left\lVert#1\right\rVert}
\newcommand\abs[1]{\left\vert#1\right\vert}
\newcommand\inj{\hookrightarrow}
\newcommand\surj{\twoheadrightarrow}
\newcommand\ded[1]{\overset{\circ}{#1}}
\newcommand\sidenote[1]{\footnote{#1}}
\newcommand\eng[1]{\left\langle#1\right\rangle}
\newcommand\hr{
    \noindent\rule[0.5ex]{\linewidth}{0.5pt}
}

\newcommand{\incfig}[1]{%
    \def\svgwidth{\columnwidth}
    \import{./figures}{#1.pdf_tex}
}
\newcommand{\filler}[1][10]%
{   \foreach \x in {1,...,#1}
    {   test 
    }
}

\newcommand\contra{\scalebox{1.5}{$\lightning$}}
\makeatother
\def\@lecture{}%
\newcommand{\lecture}[3]{
    \ifthenelse{\isempty{#3}}{%
        \def\@lecture{Lecture #1}%
    }{%
        \def\@lecture{Lecture #1: #3}%
    }%
    \subsection*{\@lecture}
    \marginpar{\small\textsf{\mbox{#2}}}
}

\begin{document}
\title{Série 3 Exercice 8}
\author{David Wiedemann}
\maketitle
\section*{1}
Indeed, let $\frac{a}{b}\in \mathbb{Q}$ in reduced form such that $\nu_p(\frac{a}{b} )=0 $. By the definition of $p$-adic, this means that we may suppose both $a$ and $b$ share no common factors with $p$, then $\frac{b}{a}$ also shares no common factor with $p$ and hence $\nu_p( \frac{b}{a}) =0$, implying $ \frac{b}{a}\in R_\nu$.\\
Finally, $\frac{a}{b}\cdot \frac{b}{a}= \frac{1}{1}$ which finally implies that $\frac{a}{b}$ is invertible in $R_\nu$.
\section*{2}
First we show that all $( p^{n}) $ are distinct ideals of $ R$, indeed suppose there exists $a,b \in \mathbb{N}$ such that $ ( p^{a}) = ( p^{b} )$, without loss of generality suppose $a<b$.\\
Hence, there exists an element $ \frac{a}{b} \in \mathbb{Q}$ with $ \nu_p( \frac{a}{b} ) \geq 0$ such that $ \frac{a}{b}p^{b} = p^{a} $.\\
As $ \mathbb{Q}$ is a field, this implies that $ \frac{a}{b}= p^{a-b}$ which means $ \frac{a}{b}$ has a negative valuation which contradicts our hypothesis.\\

Now we show that the ideals mentionned in the exercise are indeed all the ideals of $ R$.\\
Let $I$ be an non-zero ideal of $ R$.\\
Define $ a = \inf_{x \in I\setminus \left\{ 0 \right\} 	} \left\{ \nu( x)  \right\} $. Since $\nu\vert_{I \setminus \left\{ 0 \right\} }  $ has codomain $ \mathbb{N}$, this infimum exists and is attained by some element $y \in I$.\\
Since we may write $ y = p^{a} \frac{d}{c}$ where $d$ and $c$ are coprime to $p$.\\
By part $1$, we know that $\frac{d}{c}$ is invertible, hence implying that ( since $I$ is an ideal) $p^{a}\in I$.\\

We pretend that $ I = ( p^{a}) $, to do this, we show the double inclusion.\\
First, note that, since by definition $ p^{a}\in I$, we immediatly get that $ ( p^{a}) \subset I$ since $( p^{a}) $ is the smallest ideal containing $ p^{a}$.\\
Furthermore, let $x \in I$, then by definition of $a$, $\nu( x) \geq a$.\\
Since we may then write $x = p^{\nu( x) }\frac{d}{c} = p^{a} p^{\nu( x) -a}\frac{d}{c}$ where $d$ and $c$ are coprime to $p$, this implies that $x\in ( p^{a}) $.\\
Hence, if $I$ is a  non-zero ideal, $I$ is of the form $p^{n}$ for some $n$ and since these ideals are disjoint, we have characterised all of them.
\section*{3}
Using the exercise of week 2, we know that $ \mathbb{Z} \subset R$.\\
Hence consider the composition $ \mathbb{Z} \xhookrightarrow{\iota} R \xmapsto{q_R} \faktor { R} { ( p^{n}) } $ where $\iota$ is the inclusion morphism and $q_R$ is the canonical projection morphism.\\
Furthermore define $ q: \mathbb{Z}\to \faktor { \mathbb{Z}} { ( p^{n}) } $ to be the canonical projection.\\
We now pretend that $\ker ( q_R\circ \iota) = \ker q = ( p^{n}) $, indeed if $ a \in \ker q = ( p^{n}) $, then $p^{n}| a$ hence $p^{n}| \iota( a) \implies q_R( a) = 0$.\\
Similarly, if $ r\in \ker( q_R\circ \iota) $, then $p^{n}| r$, ie. there exists $ \frac{a}{b}\in R$ ( in reduced form)  such that
$p^{n}\frac{a}{b} = r$ since $\nu( \frac{a}{b}) \geq 0$, in particular we may suppose $b$ is coprime to $p$.\\
Hence, since $p^{n} \frac{a}{b}$ is an integer, $b| a$ implying $b=1$.\\
Finally, this means that there exists an integer $a$ such that $p^{n} a = r$ which means that $a\in ( p^{n}) = \ker q$.\\
Hence applying the universal property of the quotient ring, we get an induced morphism as such:
\[\begin{tikzcd}
	{\mathbb{Z}} & {\faktor{R}{(p^n)}} \\
	{\faktor{\mathbb{Z}}{(p^n)}}
	\arrow["{q_R\circ\iota}", from=1-1, to=1-2]
	\arrow["q"', from=1-1, to=2-1]
	\arrow["{\exists !\varphi}"', dotted, from=2-1, to=1-2]
\end{tikzcd}\]
We now show that $q_R \circ iota$ is surjective.
Let $ \left[ p^{i}\frac{a}{b} \right]\in \faktor { R} { ( p^{n}) }  $, where, as always, we have assumed $\frac{a}{b}$ is in reduced form and shares no factors with $p$.\\
Now we pretend that $q_R\circ \iota ( p^{i} a) = [ p^{i} \frac{a}{b}] $, indeed, notice that 
\[ 
p^{i}\frac{a}{b} - p^{i}a = \frac{p^{i}( b-1) a}{b}
\]



\end{document}
