\documentclass[11pt, a4paper]{article}
\usepackage[utf8]{inputenc}
\usepackage[T1]{fontenc}
\usepackage[francais]{babel}
\usepackage{lmodern}
\usepackage{amsmath}
\usepackage{amssymb}
\usepackage{amsthm}
\renewcommand{\vec}[1]{\overrightarrow{#1}}
\newcommand{\del}{\partial}
\DeclareMathOperator*{\sgn}{sgn}
\DeclareMathOperator*{\id}{Id}
\DeclareMathOperator*{\im}{Im}
\DeclareMathOperator*{\re}{Re}
\DeclareMathOperator*{\vol}{Vol}
\newcommand\norm[1]{\left\vert#1\right\vert}
\newcommand\ns[1]{\left\vert\left\vert\left\vert#1\right\vert\right\vert\right\vert}
\newcommand\Norm[1]{\left\lVert#1\right\rVert}
\newcommand\N[1]{\left\lVert#1\right\rVert}
\newcommand\abs[1]{\left\vert#1\right\vert}
\newcommand\inj{\hookrightarrow}
\newcommand\surj{\twoheadrightarrow}
\newcommand\ded[1]{\overset{\circ}{#1}}
\newcommand\sidenote[1]{\footnote{#1}}
\newcommand\eng[1]{\left\langle#1\right\rangle}
\newcommand\hr{
    \noindent\rule[0.5ex]{\linewidth}{0.5pt}
}

\newcommand{\incfig}[1]{%
    \def\svgwidth{\columnwidth}
    \import{./figures}{#1.pdf_tex}
}
\newcommand{\filler}[1][10]%
{   \foreach \x in {1,...,#1}
    {   test 
    }
}

\newcommand\contra{\scalebox{1.5}{$\lightning$}}
\makeatother
\def\@lecture{}%
\newcommand{\lecture}[3]{
    \ifthenelse{\isempty{#3}}{%
        \def\@lecture{Lecture #1}%
    }{%
        \def\@lecture{Lecture #1: #3}%
    }%
    \subsection*{\@lecture}
    \marginpar{\small\textsf{\mbox{#2}}}
}

\begin{document}
\title{Assignment 3}
\author{David Wiedemann}
\maketitle
\section*{1}
We first show that $ \mathbb{Z}_{ ( p) } $ is a ring, to show this, we show that it is in fact a subring of $ \mathbb{Q}$.\\
Clearly $ 1 = \frac{1}{1}\in \mathbb{Z}_{ ( p) } $.\\
Furthermore, let $ \frac{a}{b}, \frac{c}{d}\in \mathbb{Z}_{ ( p) } $, then
\[ 
\frac{ac}{bd} \in \mathbb{Z}_{ ( p) } 
\]
since $ p\not|b , b\not| d \implies p \not | bd$, where we used that $p$ is prime.\\
Similarly, 
\[ 
\frac{a}{b} + \frac{c}{d} = \frac{ ad + bc}{bd} \in \mathbb{Z}_{ ( p) } 
\]
By the same argument as above.\\

Now, suppose $\mathbb{Z}_{ ( p) } $ is a finitely generated ring, then there exist $ c_1,\ldots,c_n\in \mathbb{Z}_{ ( p) } $ which generate $ \mathbb{Z}_{ ( p) } $.\\
Write $ \forall i \quad c_i = \frac{a_i}{b_i}$ where $a_i, b_i\in \mathbb{Z}$.\\
Now, since there exist an infinite number of prime, choose a prime $q$ different from $p$ and such that $ ( q, b_i) = 1\forall 1 \leq i \leq n$.\\
We now pretend that $ \frac{1}{q}\in \mathbb{Z}_{ ( p) } $ is not contained in the subring generated by $ c_1,\ldots, c_n$.\\
Indeed, suppose there exists a polynomial in $ p\in \mathbb{Z}[x_1,\ldots,x_n]$ such that 
\[ 
ev_c( p) = p( c_1,\ldots,c_n) = \frac{1}{q}
\]
We note that for $ \nu_q$ the $q$-adic valuation on $ \mathbb{Q}$, we get that
\[ 
\nu_q(p ( c_1,\ldots,c_n))    \geq 0
\]
This follows from the fact $\nu_q$ is indeed a valuation on $\mathbb{Q}$ (as shown in previous problem sheets) . Indeed, $p( c_1, \ldots,c_n) $ simply is the multiplication and addition of elements of positive valuation.  \\
But $\nu_q( \frac{1}{q}) = -1$, implying $ p( c_1,\ldots,c_n) \neq \frac{1}{q}$, yielding a contradiction.
\section*{2}
First, we show again that $ \mathbb{Z}_p$ is a ring by showing that it is a subring of $ \mathbb{Q}$ ( it clearly is included in $ \mathbb{Q}$ ).\\
Again, note that $ 1= \frac{1}{p^{0}}\in \mathbb{Z}_p$, furthermore, for $ \frac{a}{p^{j}}, \frac{b}{p^{l}}\in \mathbb{Z}_p$, we get that
\[ 
\frac{a}{p^{j}}\cdot \frac{b}{p^{l}} = \frac{ab}{p^{j+l}} \in \mathbb{Z}_p
\]
Furthermore,
\[ 
\frac{a}{p^{j}} + \frac{b}{p^{l}} = \frac{ap^{l}+ bp^{j}}{p^{j+l}}\in \mathbb{Z}_p
\]
Hence $ \mathbb{Z}_p$ is a ring.\\

We now show that it is indeed generated by $ \frac{1}{p}$ by showing that the evaluation map
\[ 
	ev_{\frac{1}{p}} : \mathbb{Z}[x ] \to \mathbb{Z}_p
\]
is surjective.\\
Indeed, let $ \frac{a}{p^{i}}\in \mathbb{Z}_{p}$, then the polynomial $ a x^{i}$ clearly is a preimage for $ \frac{a}{p^{i}}$ implying that $ \mathbb{Z}_p$ is finitely generated by $ \frac{1}{p}$.
\section*{3}
Let $ A \subset \mathbb{Z}_p$ be a subring.\\
Suppose $ A \neq \mathbb{Z}$, then there exists an element $ \frac{b}{p^{i}}\in A, i \neq 0, ( b,p) =1$.\\
Now, since $A$ is a subring, it is closed under addition, hence adding $ \frac{b}{p^{i}}$ $ p^{i-1}$ times to itself implies that $ \frac{b}{p}\in A$.\\
We may suppose that $b$ is coprime to $p$, if it wasn't, the fraction could be reduced further.\\
This implies in particular that $ p-b$ is coprime to $b$.\\
Indeed, suppose they aren't coprime, then there exists $a\in \mathbb{N}$ such that $a| p-b$ and $a|b\implies a|p-b+b = p$ so $a$ would also be a divisor of $p$ and $b$ hence $a=1$.\\
Let $c,d\in \mathbb{Z}$ be such that $ cb + d ( p-b) = 1$, then note that
\[ 
c \frac{b}{p} + ( 1- \frac{b}{p}) d = c \frac{b}{p} + \frac{p-b}{p}d = \frac{1}{p}
\]
Hence $ A$ contains $ \frac{1}{p}$ and since $ \mathbb{Z}_p$ is generated by $ \frac{1}{p}$, this implies that $ A = \mathbb{Z}_p$.
\section*{4}
Indeed, $ \mathbb{Z}\left[\frac{1}{p}, \frac{1}{q}\right]$ is, by definition, finitely generated as it is the subring of $ \mathbb{Q}$ generated by those two elements.\\
Hence, suppose that $ \phi:\mathbb{Z}_{ ( p) }\to \mathbb{Z} \left[ \frac{1}{p}, \frac{1}{q}\right] $ is an isomorphism, then we claim $ \phi^{-1}( \frac{1}{p}), \phi^{-1}( \frac{1}{q})  $ would generate all of $ \mathbb{Z}_{ ( p) } $ which contradicts part 1.\\
We prove the claim, indeed, if $ a \in \mathbb{Z}_{( p) } $, then we may write $ \phi( a) = k_a \frac{1}{p} + l_a \frac{1}{q}\implies a = k_a \phi^{-1}( \frac{1}{p}) + l_a \phi^{-1}( \frac{1}{q}) $ which proves the claim.

\section*{5}
We pretend that in fact $ \mathbb{Z}\left[ \frac{1}{p}, \frac{1}{q}\right]$ is generated by exactly one element.\\
First, we show that $ \mathbb{Z}[ \frac{1}{pq}]= \mathbb{Z}\left[ \frac{1}{p}, \frac{1}{q}\right]$.\\
Indeed, it is clear that $ \mathbb{Z}[ \frac{1}{pq}] \subset \mathbb{Z} \left[  \frac{1}{p}, \frac{1}{q}\right] $.\\
Furthermore, note that $ \frac{1}{p}\in \mathbb{Z}\left[ \frac{1}{pq}\right] $ since $ \frac{1}{pq}\cdot q = \frac{1}{p}$ and similarly $ \frac{1}{q}\in \mathbb{Z}\left[\frac{1}{q}\right] $, which implies that $ \mathbb{Z}\left[\frac{1}{p}, \frac{1}{q}\right] \subset \mathbb{Z}\left[\frac{1}{pq}\right]$.\\

Furthermore, we show that $ \mathbb{Z}\left[ \frac{1}{p},\frac{1}{q}\right]$ cannot be generated by 0 elements, ie. is not isomorphic to $\mathbb{Z}$.\\
Indeed, note that $ \mathbb{Z}_p = \mathbb{Z}\left[\frac{1}{p}\right] \subset \mathbb{Z}\left[ \frac{1}{pq}\right]$ implying in particular that $ \mathbb{Z}\left[ \frac{1}{p}, \frac{1}{q}\right] $ has at least one non-trivial subring.\\
But $ \mathbb{Z}$ has no non-trivial subring, hence $ \mathbb{Z}$ cannot be isomorphic to $ \mathbb{Z} \left[  \frac{1}{p}\right] $.










\end{document}
