\documentclass[11pt, a4paper]{article}
\usepackage[utf8]{inputenc}
\usepackage[T1]{fontenc}
\usepackage[francais]{babel}
\usepackage{lmodern}
\usepackage{amsmath}
\usepackage{amssymb}
\usepackage{amsthm}
\usepackage{mathtools}
\renewcommand{\vec}[1]{\overrightarrow{#1}}
\newcommand{\del}{\partial}
\DeclareMathOperator*{\sgn}{sgn}
\DeclareMathOperator*{\id}{Id}
\DeclareMathOperator*{\im}{Im}
\DeclareMathOperator*{\re}{Re}
\DeclareMathOperator*{\vol}{Vol}
\newcommand\norm[1]{\left\vert#1\right\vert}
\newcommand\ns[1]{\left\vert\left\vert\left\vert#1\right\vert\right\vert\right\vert}
\newcommand\Norm[1]{\left\lVert#1\right\rVert}
\newcommand\N[1]{\left\lVert#1\right\rVert}
\newcommand\abs[1]{\left\vert#1\right\vert}
\newcommand\inj{\hookrightarrow}
\newcommand\surj{\twoheadrightarrow}
\newcommand\ded[1]{\overset{\circ}{#1}}
\newcommand\sidenote[1]{\footnote{#1}}
\newcommand\eng[1]{\left\langle#1\right\rangle}
\newcommand\hr{
    \noindent\rule[0.5ex]{\linewidth}{0.5pt}
}

\newcommand{\incfig}[1]{%
    \def\svgwidth{\columnwidth}
    \import{./figures}{#1.pdf_tex}
}
\newcommand{\filler}[1][10]%
{   \foreach \x in {1,...,#1}
    {   test 
    }
}

\newcommand\contra{\scalebox{1.5}{$\lightning$}}
\makeatother
\def\@lecture{}%
\newcommand{\lecture}[3]{
    \ifthenelse{\isempty{#3}}{%
        \def\@lecture{Lecture #1}%
    }{%
        \def\@lecture{Lecture #1: #3}%
    }%
    \subsection*{\@lecture}
    \marginpar{\small\textsf{\mbox{#2}}}
}

\begin{document}
\title{Série 9}
\author{David Wiedemann}
\maketitle
\section*{1}
First, notice that $\phi$ will fix $\mathbb{Z}$, indeed, since $ \phi( 1) =1$ and $\phi( -1)+ \phi( 1) = 0 \implies \phi(-1 )  = -1$ , we have that $\forall a \in \mathbb{Z}$, if $a$ is positive 
\[ 
\phi( a) = \phi( 1 + 1 \ldots + 1) = 1 + \ldots + 1 = a
\]
and if $a$ is negative
\[ 
\phi( a) = \phi( -1 - \ldots -1 ) = -1 - \ldots - 1 = a
\]

Now, pick an element $ \frac{p}{q}$ of $ \mathbb{Q}$, then $ q \cdot \frac{p}{q}= p$ and thus
\[ 
p = \phi( p) = \phi\left( q \cdot \frac{p}{q}\right) = \phi( q) \cdot \phi\left( \frac{p}{q}\right) = q \cdot\phi\left( \frac{p}{q}\right) 
\]
Hence $\phi( \frac{p}{q}) = \frac{p}{q}$ which shows that $\phi$ restricted to $\mathbb{Q}$ is the identity.
\section*{2}
Let us denote $ S \coloneqq \left\{ s \in F | \exists a,b \in \mathbb{Z} : s^{2} + as + b =0 \right\} $.\\
We show the double inclusion.
\subsubsection*{ $A \subset S$ }
Let $x + y \sqrt{ -d} \in A$ (ie. $x,y\in \mathbb{Z}$).\\
Let $a= -2x$ and $ b = y^{2}d+ x^{2}$, then note that
\begin{align*}
	&( x+ y \sqrt{-d} )^{2} + a ( x+y \sqrt{-d} ) + b \\
	&= x^{2} + 2xy \sqrt{-d} - y^{2} d + ax + ay \sqrt{-d} +b \\
	&= ( x^{2} - y^{2}d^{2} + ax +b  ) + ( 2xy + ay )  \sqrt{-d} \\
	&= ( x^{2} -y ^{2}d^{2} - 2x^{2} + y^{2} d + x^{2}) + ( 2xy - 2xy) \sqrt{-d} = 0
\end{align*}
Which show the inclusion $ A \subset S$ 
\subsection*{ $ A \supset S$ }
Given $s \in F$, since $F$ is a quadratic (finite) extension of $\mathbb{Q}$ and thus $ \mathbb{Q}\left( \sqrt{-d} \right) = \mathbb{Q} \left[  \sqrt{-d} \right] $, we may write $s $ as $ s= \frac{p}{q} + \frac{x}{y}	 \sqrt{-d} $, where we suppose $ p,q,x,y \in \mathbb{Z}$ and $ ( p,q) = ( x,y) = 1$.\\
Let us now first suppose that $ \frac{p}{q}, \frac{x}{y}\neq 0$, we'll treat these special cases later.\\
Let $a,b \in \mathbb{Z}$ such that $ s^{2}+ as + b = 0$.\\
Plugging $s= \frac{p}{q}+ \frac{x}{y} \sqrt{-d} $ into the polynomial yields
\begin{align*}
\frac{p^{2}}{q^{2}} + 2 \frac{p}{q} \frac{x}{y} \sqrt{-d}  - \frac{x^{2}}{y^{2}}d + a \frac{p}{q} + a \frac{x}{y} \sqrt{-d} + b = 0
\end{align*}
as $ \sqrt{-d} $ and $1$ are linearly independent over $\mathbb{Q}$, we get the system of equations
\begin{align}
\frac{p^{2}}{q^{2}} - \frac{x^{2}}{y^{2}} d + a \frac{p}{q} + b =0\\
 2 \frac{p}{q} \frac{x}{y} + a \frac{p}{q} = 0
\end{align}
From equation 2 above we deduce (and by our assumption that $\frac{p}{q} \neq 0$) that
\[ 
2 \frac{x}{y} + a =0
\]
By coprimality of $x$ and $y$, this implies that $ y =1$ or $y=2$, we treat these cases separately:
\subsubsection*{If $y=1$ }
So suppose $ \frac{p}{q}\in \mathbb{Q}\setminus \mathbb{Z}$, ie. $q \neq 1$ .\\
Rewriting equation $1$ above then yields
\begin{align*}
	\frac{p^{2}}{q^{2}} - x^{2} d + a \frac{p}{q} + b &=0\\
	p^{2} - x^{2} d q^{2} + a p q + b q ^{2} &=0\\
	x^{2}d q^{2} &= p^{2} + a pq + bq^{2}\\
\end{align*}
Since $q| x^{2}dq^{2}$, $ q | p^{2} + apq + bq^{2}\implies q | p^{2} $ which is a contradicition, since we supposed that $p$ and $q$ share no common factors and $q\neq 1$.\\
Thus $q=1$ and $ s = p + x \sqrt{ -d} $.
\subsubsection*{ If $y=2$ }
By equation 2 above, we know that then $ x=-a$.
This implies in particular that $a$ is not a multiple of $2$ as $( x,2) =1$. 
Plugging this into equation $1$ yields:
\begin{align*}
	\frac{p^{2}}{q ^{2}} - \frac{a^{2}}{4} d + a \frac{p}{q}+b &=0\\
	4p^{2} - a^{2}d q^{2} + 4 a pq + 4bq^{2} &=0\\
	a^{2}dq^{2} &= 4p^{2} + 4 a pq + 4 bq^{2}
\end{align*}
Now, since $d$ is square free and $a$ is not a multiple of 2, $4 | q^{2}$ and in particular $2 | q$, thus rewrite $q = 2 e $ for some integer $e \in \mathbb{Z}$.\\
\begin{align*}
	a^{2} d 4 e^{2} &= 4 p^{2} + 4 a p 2 e + 4b 4 e^{2}\\
	a^{2} d e^{2} &= p^{2} + 2 ape + 4be^{2}
\end{align*}
Reducing the above modulo 4 yields the congruence relation
\begin{align}
	a^{2} d e^{2} &\equiv p^{2} + 2 a p e \mod 4
\end{align}
Now, if $e$ is a multiple of $2$, then the above relation simply reduces to
\begin{align*}
	0 &\equiv p^{2}\mod 4
\end{align*}
Which implies $4 | p^{2} \implies 2 | p$ which would mean $\frac{p}{q}$ is not irreducible, a contradiction.\\
Thus, $e$ is not a multiple of 2.\\
Notice that an uneven number squared is always congruent to $1$ modulo 4, since $( 2k+1)^{2} \equiv 4k^{2} + 4k + 1 \equiv 1 \mod 4$.\\
Thus,
\[
	a^{2} d e^{2} \mod 4 \equiv ( a\mod 4)^{2} ( d\mod 4 ) ( e\mod 4)^{2} \equiv d\mod 4
\]
And similarly
\[ 
	( p^{2} + 2ape) \mod 4 \equiv ( p\mod 4)^{2} + 2 ( ape \mod 4) \equiv 1 + 2 \mod 4 \equiv 3 \mod 4
\]
Thus, equation 3 implies that $d \equiv 3\mod 4 $ which contradicts our hypothesis.



%It now suffices to check that this is impossible for the different congruence classes of $d$ modulo 4.
%\subsubsection*{If $d \equiv 0 \mod 4$ }
%Then $d$ is divisible by 4, but $d$ is square free, which is impossible.
%\subsubsection*{If $d\equiv 1 \mod 4$ }
%Notice that an uneven number squared is always congruent to $1$ modulo 4, since $( 2k+1)^{2} \equiv 4k^{2} + 4k + 1 \equiv 1 \mod 4$.\\
%Using this, in equation 3 above yields with $d\equiv 1 \mod 4$ .\\
%\begin{align*}
	%a^{2} e ^{2} &\equiv p^{2} + 2 ape \mod 4\\
	%1 \cdot 1 &\equiv 1 + 2 ( 1\cdot 1 \cdot 1) \mod 4	
%\end{align*}
%Where we used that $a,e $ and $p$ are all uneven and that reduction modulo some integer is a morphism of rings.\\
%But $1 \not\equiv 3 \mod 4$ and we get a contradiction.
%\subsubsection*{If $d\equiv 2\mod 4$ }
%Further reducing the congruence relation in 3 yields
%\[ 
%0 \equiv p^{2} \mod 2
%\]
%Whence $2| p^{2} \implies 2 |p $ which is a contradiction.\\

%Hence, we have eliminated all possibilities for the values of $d$.\\
It remains to show the special cases mentionned at the start.
\subsubsection*{If $\frac{x}{y}=0$ }
We may suppose $ \frac{p}{q}\neq 0 $ and that $( p,q) =1$  then plugging this into our quadratic equation yields
\begin{align*}
	\frac{p^{2}}{q^{2}}+ a \frac{p}{q}+b &=0\\
	p^{2} + a p q +b q^{2} &= 0\\
 	p^{2} &= - q ( ap +bq) 
\end{align*}
But then $q | p^{2}$ and thus $q=1$ by our hypothesis
\subsubsection*{If $ \frac{p}{q}= 0$ }
Again, recall $( p,q) =1$.\\
Then plugging $ \frac{p}{q} \sqrt{-d} $ into the equation yields
\begin{align*}
-d\frac{p^{2}}{q^{2}} + a \frac{p}{q} \sqrt{-d}  + b =0
\end{align*}
Now, if $a\neq 0$ this equation can not hold since $ \sqrt{-d} $ and $1$ are linearly independent over $ \mathbb{Q}$.\\
If $a =0$, then rewriting the above gives
\[ 
 d p^{2} = bq^{2}
\]

Whence, since $( p,q) =1$, $q^{2} |d$ which contradicts our hypothesis.\\

And that's it for part 2 (whew).

\section*{3}
First, notice that we may apply the results from section 2 to $ \mathbb{Q}( i) $ and $ \mathbb{Q}( \sqrt{-5} ) $ as both $1$ and $5$ are square free and congruent to $1\mod 4$.\\
Suppose $\phi: \mathbb{Q}( i) \to \mathbb{Q}( \sqrt{-5} ) $ is an isomorphism.\\
Notice that then $\phi$ induces an isomorphism of rings between $ \mathbb{Z}[i]$ and $ \mathbb{Z}[\sqrt{-5} ]$.\\
Indeed, let $ s\in \mathbb{Q}( i) $ be integral over $ \mathbb{Z}$, ie. an element such that $ s^{2} + as +b =0$ for some integers $a$ and $b$, then applying $\phi$ to this gives an element $\phi( s) $ satisfying
\[ 
\phi( s) ^{2} + a \phi( s) +b =0 \quad\text{(since $\phi$ fixes $\mathbb{Q}$)} 
\]
Which will thus be integral in $ \mathbb{Q}(  \sqrt{-5} ) $ over $\mathbb{Z}$, ie. is an element of $ \mathbb{Z}[\sqrt{-5} ]$, it is clear that this mapping is an isomorphism with inverse $\phi^{-1}$ .\\
Now we prove that $ \mathbb{Z}[i] \not\simeq \mathbb{Z}[ \sqrt{-5} ]$ which will contradict the fact that $\phi$ is an isomorphism.\\

Indeed, let $\psi $ be such an isomorphism, by the same argument as in 1, $\psi$ fixes $\mathbb{Z}$.\\
Thus $\psi^{-1}(  \sqrt{-5} )^{2}= -5$.\\
Let $a+bi= \psi^{-1}(  \sqrt{-5} ) $ be such an element, then $a^{2}+2abi -b^{2} = -5$, implying either $ a$ or $b$ equal to $0$, since $a^{2}-b^{2} = -5$ and $ a^{2}, b^{2}>0$, this implies $a=0$ and thus $-b^{2} = -5$.\\
But $5$ is prime and thus such a $b$ cannot exist, contradicting the fact that $\psi$ is an isomorphism.





















		


\end{document}
