\documentclass[11pt, a4paper]{article}
\usepackage[utf8]{inputenc}
\usepackage[T1]{fontenc}
\usepackage[francais]{babel}
\usepackage{lmodern}
\usepackage{amsmath}
\usepackage{amssymb}
\usepackage{amsthm}
\usepackage{faktor}
\renewcommand{\vec}[1]{\overrightarrow{#1}}
\newcommand{\del}{\partial}
\DeclareMathOperator*{\sgn}{sgn}
\DeclareMathOperator*{\id}{Id}
\DeclareMathOperator*{\im}{Im}
\DeclareMathOperator*{\re}{Re}
\DeclareMathOperator*{\vol}{Vol}
\newcommand\norm[1]{\left\vert#1\right\vert}
\newcommand\ns[1]{\left\vert\left\vert\left\vert#1\right\vert\right\vert\right\vert}
\newcommand\Norm[1]{\left\lVert#1\right\rVert}
\newcommand\N[1]{\left\lVert#1\right\rVert}
\newcommand\abs[1]{\left\vert#1\right\vert}
\newcommand\inj{\hookrightarrow}
\newcommand\surj{\twoheadrightarrow}
\newcommand\ded[1]{\overset{\circ}{#1}}
\newcommand\sidenote[1]{\footnote{#1}}
\newcommand\eng[1]{\left\langle#1\right\rangle}
\newcommand\hr{
    \noindent\rule[0.5ex]{\linewidth}{0.5pt}
}

\newcommand{\incfig}[1]{%
    \def\svgwidth{\columnwidth}
    \import{./figures}{#1.pdf_tex}
}
\newcommand{\filler}[1][10]%
{   \foreach \x in {1,...,#1}
    {   test 
    }
}

\newcommand\contra{\scalebox{1.5}{$\lightning$}}
\makeatother
\def\@lecture{}%
\newcommand{\lecture}[3]{
    \ifthenelse{\isempty{#3}}{%
        \def\@lecture{Lecture #1}%
    }{%
        \def\@lecture{Lecture #1: #3}%
    }%
    \subsection*{\@lecture}
    \marginpar{\small\textsf{\mbox{#2}}}
}

\begin{document}
\title{Series 12, Bonus Exercise}
\author{David Wiedemann}
\maketitle
\section*{1}
Recall from a theorem that, to show that $f= x^{3}+ax+1, a \in \mathbb{Z}_+$ is irreducible over $ \mathbb{Q}$ , it suffices to show that it has no roots over $\mathbb{Q}$.
Hence, suppose that $ \frac{w}{z}\in \mathbb{Q} $ is a zero of $ f $, further, without loss of generality suppose that $ w $ and $ z $ share no common factors.
Plugging this into $f$ yields
\begin{align*}
	\frac{w^{3}}{z^{3}} + a \frac{w}{z} + 1 &= 0\\
	w^{3} + aw z^{2} + z^{3} &= 0\\
	w^{3} = z^{2} ( -aw - z)
\end{align*}
Thus $w$ and $z$ share at least one common factor, a contradiction.\\
As such, we conclude that $f$ has no roots over $ \mathbb{Q}$ and is thus irreducible.
\section*{2}
First notice that by elementary analysis results, $f$ always has at least one real root ($f$ is continuous over $\mathbb{R}$ and $ \lim_{x \to  + \infty} f = + \infty $ , $ \lim_{x \to - \infty } f = - \infty $).\\
Hence, let $l$ be this real root.\\
In this case, note that over $\mathbb{R}$, $f$ splits as 
\[ 
	f( x) =( x-l) ( x^{2}+bx+c) = x^{3} + ( b-l) x^{2} + ( c-bl) x - lc = x^{3}+ ax + 1
\]
We pretend that in this case, $x^{2}+bx+c $ has no real roots, to show this, it is sufficient to show that the discrimant $ b^{2}-4c < 0$
As the family $ \left\{ x^{i} \right\}_{i =0}^{ \infty }$ is a basis for the vector space $ \mathbb{R}[x]$, we conclude that $l,b$ and $c$ must satisfy the following three equations
\begin{align*}
\begin{cases}
b-l =0\\
c-bl = a\\
-lc = 1 
\end{cases}
\end{align*}
In particular, $l=b$ and thus $c= a+b^{2}$.\\
Then, the discrimant becomes
\[ 
\Delta = b^{2}-4c = b^{2}- 4 b^{2} - 4a = - 3b^{2} - 4a
\]
Now as $a>0$ and $b^{2}>0$ (as $b\in \mathbb{R}$) ,we conclude that $ \Delta <0$ and thus $f$ does not have three real roots.
\section*{3}
First, we show that $K$ is indeed of degree $3$, ie. that  $ [ K: \mathbb{Q}] = 3$, this follows immediatly from lemma 4.2.25 in the course, indeed, $f$ is irreducible of degree 3 over $ \mathbb{Q}$ by part $1$ and thus the hypothesis of the proposition applies.\\

Now we show that $K$ is not Galois over $ \mathbb{Q}$, to do this, we construct an embedding of $ K \hookrightarrow \mathbb{R}$.\\
Let $l\in \mathbb{R}\setminus \mathbb{Q}$ be the (unique) real root of $f$, using a theorem from the course we know that $ K \simeq \mathbb{Q}(l)$.\\
Let $\sigma \in \gal(  K / \mathbb{Q}) $, as $\sigma( l) $ must still be a root of $f$.\\
As $l$ is the unique root of $f$ in $K$ (as  $ K \subset \mathbb{R}$ and $f$ has a unique real root) , $\sigma( l) =l$.\\
As $l$ generates $K$ over $ \mathbb{Q}$, this implies that $\sigma= \id$ and thus $| \gal( L / \mathbb{Q}) | = 1$.\\
Since an extension $ K / \mathbb{Q}$ is Galois iff $ [  K : \mathbb{Q}] = | \gal(  L / \mathbb{Q}) |$, we deduce that $K$ is not Galois over $ \mathbb{Q}$.
%As such, the minimal polynomial of $l$ does not split into linear factors in $K$ and we deduce that the extension $ K  / \mathbb{Q}$ is not Galois from proposition 4.6.15.\\
\section*{4}
Let $l, c_1,c_2$ be the three distinct roots of $f$ (where $l$ is the real root and $c_1,c_2$ are the two complex ones).\\
By definition, $L$ being the splitting field, it is generated by $ l, c_1,c_2$, ie. $L = \mathbb{Q}(l, c_1,c_2) $.\\
Using proposition 4.6.3.2 from the course notes, we thus conclude that there exists an injective group morphism $\gal( L / \mathbb{Q}) \hookrightarrow S_3$.\\
Notice that $|S_3|=3! = 6$ and thus it suffices to show $ |\gal( L / \mathbb{Q}) | = 6$, as an injective map between finite sets of same cardinality is bijective.\\

To show this, first notice that as we are working in characteristic 0, any extension of $\mathbb{Q}$ is separable, this follows from our characterisation of perfect fields.\\
As $L$ is a splitting field and is generated by separable elements, proposition 4.6.3.4 in fact implies that $|\gal( L /\mathbb{Q}) | = [ L:\mathbb{Q}] $, so we reduce to showing that $ [ L: \mathbb{Q}] = 6$.\\
We claim that $ [ L:K] =2$.\\
To show this, notice that as $K = \mathbb{Q}(l) $, $L = K( c_1,c_2)$.\\
We pretend that $K( c_1) = L$.\\
Indeed, notice that over $K$, $f$ splits as $( x-l) ( x^{2}+bx+c), b,c\in K$, as $x^{2}+bx+c$ is a degree 2 polynomial over $K$ which does not have roots over $K$, it is irreducible and thus is a minimal polynomial for $c_1$ (as $c_1$ obviously is not a root of $x-l$).\\
From our general quadratic formula, we know that (up to switching the signs in front of the square root)  $c_1= \frac{1}{2} (-b + \sqrt{b^{2}-4c} ) $ and $c_2= \frac{1}{2}( -b- \sqrt{b^{2}-4c})$, in particular $c_2 = -b-c_1$.\\
Thus, $K( c_1) \supset K( c_1,c_2) $, as the inclusion $ K( c_1) \subset K( c_1,c_2)$ is trivial, we deduce that $K( c_1) = L$.\\
As, $K(c_1) = \faktor{K[x]}{( x^{2}+bx+c) }$, $[ K(c_1):K ]=2 $.\\
As such, we may compute  $ [ L:\mathbb{Q}]= [ L:K][K:\mathbb{Q}] = 2\cdot 3 = 6 $.\\
Thus, the injective homomorphism $  \gal( L /\mathbb{Q}) \hookrightarrow S_3$ is in fact a bijection and thus $ \gal( L /\mathbb{Q}) \simeq S_3$.







\end{document}
