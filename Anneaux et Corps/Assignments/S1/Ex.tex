\documentclass[11pt, a4paper]{article}
\usepackage[utf8]{inputenc}
\usepackage[T1]{fontenc}
\usepackage[francais]{babel}
\usepackage{lmodern}
\usepackage{amsmath}
\usepackage{amssymb}
\usepackage{amsthm}
\renewcommand{\vec}[1]{\overrightarrow{#1}}
\newcommand{\del}{\partial}
\DeclareMathOperator*{\sgn}{sgn}
\DeclareMathOperator*{\id}{Id}
\DeclareMathOperator*{\im}{Im}
\DeclareMathOperator*{\re}{Re}
\DeclareMathOperator*{\vol}{Vol}
\newcommand\norm[1]{\left\vert#1\right\vert}
\newcommand\ns[1]{\left\vert\left\vert\left\vert#1\right\vert\right\vert\right\vert}
\newcommand\Norm[1]{\left\lVert#1\right\rVert}
\newcommand\N[1]{\left\lVert#1\right\rVert}
\newcommand\abs[1]{\left\vert#1\right\vert}
\newcommand\inj{\hookrightarrow}
\newcommand\surj{\twoheadrightarrow}
\newcommand\ded[1]{\overset{\circ}{#1}}
\newcommand\sidenote[1]{\footnote{#1}}
\newcommand\eng[1]{\left\langle#1\right\rangle}
\newcommand\hr{
    \noindent\rule[0.5ex]{\linewidth}{0.5pt}
}

\newcommand{\incfig}[1]{%
    \def\svgwidth{\columnwidth}
    \import{./figures}{#1.pdf_tex}
}
\newcommand{\filler}[1][10]%
{   \foreach \x in {1,...,#1}
    {   test 
    }
}

\newcommand\contra{\scalebox{1.5}{$\lightning$}}
\makeatother
\def\@lecture{}%
\newcommand{\lecture}[3]{
    \ifthenelse{\isempty{#3}}{%
        \def\@lecture{Lecture #1}%
    }{%
        \def\@lecture{Lecture #1: #3}%
    }%
    \subsection*{\@lecture}
    \marginpar{\small\textsf{\mbox{#2}}}
}

\begin{document}
\title{Series 2 Exercise 7}
\author{David Wiedemann}
\maketitle
\section*{$\nu( 1) = 0$ and $\nu( -1) =0$  }
Indeed, note that 
\[ 
\nu( 1\cdot 1) = \nu( x) +\nu( 1) = \nu(1) \iff 2\nu( 1) =\nu( 1) \iff \nu( 1) =0
\]
Now for the second part, notice that since $-1\cdot -1 = 1$ we get
\[ 
\nu( -1\cdot -1) = \nu( -1) +\nu( -1) = \nu( 1) = 0\iff 2\nu( -1) =0 \iff \nu( -1) =0
\]
\section*{$R_\nu$ is a subring of $K$ }
To show $R_\nu$ is a subring, we have to show that $  1,0\in R_{\nu} $ and that $R_\nu$  is closed under addition and multiplication.\\

Using the first part of the exercise, we immediatly get that $1\in R_\nu$ since $\nu( 1) \geq 0$  and by definition $0\in R_\nu$.\\
\subsubsection*{ $R_\nu$ is closed under multiplication}
Let $x,y \in R_\nu\setminus \left\{ 0 \right\} 	$, we get that $\nu( x\cdot y) = \nu( x) +\nu( y) \geq 0$ since $\nu( x) ,\nu( y) \geq 0$ by hypothesis.\\
If either $x$ or $y$ is equal to 0, then clearly $x\cdot y = 0\in R_\nu.$\\
Hence $R_\nu$ is closed under multiplication.
\subsubsection*{ $R_\nu$ is closed under addition}
Indeed, let $x,y\in R_\nu$, now $ \nu( x+y) \geq \min ( \nu( x) ,\nu( y) ) \geq 0$ hence $x+y\in R_\nu$.\\

This show that $R_\nu$ is a subring of $K$.
\section*{$K$ is the fraction field of $R_\nu$ }
Before proving the result, we notice two things.
\begin{itemize}
\item Given $x\in K$, we have that $\nu( x^{-1}) = - \nu( x) $, this follows immediatly using part 1.
\item If $\nu$ is a non-trivial valuation, we have that $\forall N \in \mathbb{N} \exists x\in K$ such that $\nu( x) \geq N$.\\
	Indeed, by the above point we can find $x\in K, \nu( x) > 0$( which exists since the valuation is non-trivial), simply note that there exists $n\in \mathbb{N}$ such that $\nu( x^{n}) =n\cdot \nu( x) \geq 0$ 
\end{itemize}
First, suppose the valuation is non-trivial and fix $x\in K$ such that $\nu( x) \geq 0$ 
We now show that $K$ satisfies the universal property of the fraction field.\\
Let $L$ be a field and $j: R_\nu\to L$ a ring homomorphism.\\
Given $a\in K$, let $n \in \mathbb{N}$ such that $\nu( x^{n}\cdot a) \geq 0$, then we define
\[ 
	f( a) = f( \frac{a\cdot x^{n}}{x^{n}}) = \frac{j( a\cdot x^{n}) }{ j( x^{n}) }
\]
We now show that $f$ is well defined ( ie. doesn't depend on $x$ and the choice of $n$ ), that $f$ is a ring homomorphism and that it is the unique function $f: K \to L $ with the universal property of the fraction field.









\end{document}
