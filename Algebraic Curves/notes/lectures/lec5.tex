\documentclass[../main.tex]{subfiles}
\begin{document}
\lecture{5}{Fri 25 Mar}{Projective varieties}
While the $i$-th coordinate $x_i$ of a point $ [ x_1: \ldots:x_{n+1} ] \in \mathbb{P}^{n}$ is not well defined, the equation $x_i=0$ or $x_i\neq 0$  is well defined.\\
Hence we can write
\[ 
U_i = \left\{ [ x_1:\ldots:x_n] |x_i\neq 0 \right\} 
\]
Clearly $ \mathbb{P}^{n}=\cup_i U_i$.\\
Furthermore for all $i$, we have a bijection
\begin{align*}
\phi_i: \mathbb{A}^{n}\to U_i\\
( x_1,\ldots,x_n) \mapsto [ x_1:\ldots:x_{i-1}:1 : x_{i+1} :\ldots: x_{n+1}  ] 
\end{align*}
And this is clearly a bijection.\\
We'll see in a bit, that the $\phi_i$'s provide an open cover of $ \mathbb{P}^{n}$ by $ \mathbb{A}^{n}$ 
\begin{defn}
	The set 
	\[ 
	H_{ \infty } \coloneqq  \mathbb{P}^{n}\setminus U_{n+1} = \left\{ x\in \mathbb{P}^{n}| x_{n+1} =0 \right\} 
	\]
	is called the hyperplane at infinity.
\end{defn}
One can identify $ H_{ \infty } = \mathbb{P}^{n-1}$\\
Thus
\[ 
\mathbb{P}^{n}= U_{n+1} \coprod H_{ \infty } = \mathbb{A}^{n}\coprod \mathbb{P}^{n-1}
\]
\begin{exemple}
$ \mathbb{P}^{0}= \text{ point } $ \\
$ \mathbb{P}^{1}= \mathbb{A}^{1}\coprod \text{ point } $ is called the projective line.\\
Similarly $ \mathbb{P}^{2}$ is called the projective plane.
\end{exemple}
\subsection{Projective algebraic sets}
For a general $ F\in k[x_1,\ldots,x_n]$, the equation $F( x) =0, x\in \mathbb{P}^{n}$ doesn't make sense.\\
But it does if $F$ is homogeneous, say of degree $d$, since then
\[ 
F( \lambda x ) = \lambda^{d} F( x) =0 \forall x \in \mathbb{A}^{n+1},\lambda\in k^{*}
\]
\begin{defn}[Projective set]
	For any set $ S \subset k[x_1,\ldots,x_n]$ of homogeneous polynomials we set
	\[ 
	V( S) = \left\{ [ x_1:\ldots:x_n] \in \mathbb{P}^{n}| F( x_1,\ldots,x_n) =0\forall F\in S \right\} 
	\]
	A subset of $ \mathbb{P}^{n}$ is algebraic if it is of the form $V( S) $ as above.
\end{defn}
\begin{exemple}
Take $V( X^{2}-YZ) \subset \mathbb{P}^{2} $, how to draw it?\\
We draw the intersections $V\cap U_i$ 	
\end{exemple}
\begin{defn}[Homogeneous ideal]
	An ideal $I \subset k[x_1,\ldots,x_n]$ is homogeneous if it is generated by homogeneous elements.\\
	The for $I$ a homogeneous ideal we set
	\[ 
	V( I) = V( T) \subset \mathbb{P}^{n}
	\]
	where $T$ is the set of forms in $I$.
\end{defn}
\begin{rmq}
Since the ring is noetherian, we can always find a finite number of homogeneous generators.\\
For $I = ( x_1,\ldots,x_{n+1} ) $ we have $V( I) = \emptyset$, we denote this ideal by $I_+$, it's called the irrelevant ideal.
\end{rmq}
\begin{exemple}
$( x,y^{2}) $ is homogeneous, $ ( x+y^{2},y^{2}) $ is also homogeneous but $( x+y^{2}) $ is not.
\end{exemple}
\begin{lemma}
	$I$ is a homogeneous ideal if and only if for every $F \in I$, if we write $F = \sum_{i \geq 0} F_i$ with $F_i$ homogeneous of degree $i$.
\end{lemma}
\begin{proof}
Let $ G^{( 1) },\ldots,G^{( k) }$ be a set of homogeneous generators of $I$ with degrees $d_1,\ldots,d_k$.\\
Any $ F= \sum F_i$ can be written as $ F = \sum A^{( i) }G^{( i) }$ for some $A^{( i) }$.\\
Since the degree is additive we get $F_j = \sum A^{( i) }_{j-d_i} G^{ ( i) }$ \\
For the other direction, let $G^{( 1) ,\ldots,G^{( k) }}$ any set of generators, then $G_j^{( i) }\in I$ and then the set of $G_j^{( i) }$ is a set of generators.
\end{proof}
Furthermore, the sum, the product, the intersection and the radical of homogeneous ideals are homogeneous.\\
A homogeneous ideal is prime if for any homogeneous $f,g\in k[x_1,\ldots,x_n]$ 
\[ 
fg\in I \implies f\in I \text{ or } g\in I
\]
\begin{defn}[Zariski topology]
We define the Zariski topology on $ \mathbb{P}^{n}$ by taking the open sets to be the complements of algebraic sets.	
\end{defn}
This defines a topology using the properties above.
\begin{defn}
	An algebraic set $V \subset \mathbb{P}^{n}$ is irreduciable if it is irreducible as a topological space.
\end{defn}
As in the affine case, there is a correspondence
\[ 
	\left\{ \text{ Algebraic subsets in} \mathbb{P}^{n} \right\} \leftrightarrow \left\{ \text{ Homogeneous ideals in  } k[x_1,\ldots,x_{n+1} ]  \right\} 
\]
Where $I( V) $ is the ideal generated by $ \left\{ F \in k[x_1,\ldots,x_n]| F \text{ homogeneous } ,F( v) =0 \forall v\in V   \right\} $ 

\begin{rmq}
If we need to distinguish between the affine and projective correspondence we'll write $V_a,I_a$ and $V_p,I_p$ respectively.
\end{rmq}
\begin{defn}[Cone]
For $V \subset  \mathbb{P}^{n}$ algebraic, we define the conve over $V$ as
\[ 
C( V) = \left\{ ( x_1,\ldots,x_{n+1} ) \in \mathbb{A}^{n+1}| [ x_1,\ldots,x_{n+1} ] \in V \right\} \cup \left\{ ( 0,\ldots,0)  \right\} 
\]
	
\end{defn}
\begin{lemma}
\begin{enumerate}
\item For $V \neq \emptyset$, then
	\[ 
	I_p( V) = I_a( C( V) ) 
\]
\item If $ I \subsetneq k[x_1,\ldots,x_n]$ homogeneous, then
	\[ 
	C( V_p( I) ) = V_a( I) 
	\]
	
\end{enumerate}
\end{lemma}
\begin{proof}
\begin{enumerate}
\item $G\in I_p( V) $ homogeneous and $ ( x_1,\ldots,x_{n+1} ) \in C( V) $, then $G( x_1,\ldots,x_{n+1} ) =0$ \\
	Conversely, if $G\in I_a( C( V) ) $ write
	\[ 
	G= \sum_i G_i,\quad G_i \text{ homogeneous } 
	\]
	Then, for every $x\in C( V) $ and $\lambda\in k^{*}$ we have $ \lambda x \in C( V) $ hence
	\[ 
	0 = G( \lambda x ) = \sum_i \lambda^{i}G_i( x) 
	\]
	Let $\tilde G ( y) = \sum_{i}^{ } y^{i}G_i( x) \in K[Y]$, this has infinitely many $0$'s.\\
	Which in turn implies $G_i \in I_p( V) $ 
	

\item Notice for $G$ homogeneous non-constant, then
	\[ 
	C( V_p( G) ) = V_a( G) 
	\]
	Since $I$ is generated by homogeneous polynoials, the satement holds.
\end{enumerate}

\end{proof}
\begin{propo}[Projective nullstellensatz]
Let $I$ be a homogeneous ideal, then
\begin{itemize}
	\item If $ V_p( I) = \emptyset$, then $ \sqrt{I} = k[x_1,\ldots,x_{n+1} ]$ or $ \sqrt{I} = I_+$ 
	\item If $V_p( I) = \emptyset$ then $I_p( V_p( I) ) = \sqrt{I} $ 
\end{itemize}

\end{propo}
\begin{proof}
\begin{itemize}
\item If $V_p( I) = \emptyset\iff V_a( I) \subset \left\{ ( 0,\ldots,0)  \right\} $ which implies $\sqrt{I} \supset ( x_1,\ldots,x_{n+1} ) $.\\
\item $I_p( V_p( I) ) = I_a( C( V_p( I) ) ) = I_a( V_a( I) ) = \sqrt{I} $ 
\end{itemize}

\end{proof}
\begin{crly}
There is a one-to-one correspondence between radical homogeneous ideals and projective algebraic sets.\\
Furthermore $V_p( I) $ is irreducible $\iff$ $I$ is prime.
\end{crly}
\begin{rmq}
Points in $\mathbb{P}^{n}$ do not correspond to maximal ideals.
\end{rmq}
We can also relate affine and projective algebraic sets through the charts
\[ 
\phi_i : \mathbb{A}^{n}\to U_i
\]
We'll focus on $\phi \coloneqq \phi_{n+1} : \mathbb{A}^{n}\to U \coloneqq U_{n+1} 	$ \\
For $F\in k[x_1,\ldots,x_n]$ homogeneous, we define
\[ 
F_*( x_1,\ldots,x_n) = F( x_1,\ldots,x_n,1) 
\]
Conversely, for $ G\in k[x_1,\ldots,x_n]$, we write
\[ 
G= \sum_{i=0}^{ d} G_i \text{ and define } G^{*}( x_1,\ldots,x_{n+1} ) = x_{n+1}^{ d}G_0+ \ldots + G_d= X_{n+1}^{d}G( \frac{x_1}{x_{n+1} },\ldots, \frac{x_n}{x_{n+1} }) 	
\]
\begin{defn}[Homogenization]
	$( \cdot)_{*} $ and $ ( \cdot)^{*}$ are called dehomogenisation and homogenization.
\end{defn}
For $I$ an ideal, we denote by $I^{*}$ be the homogeneous ideal generated by $\left\{ F^{*}|F\in I \right\} $.\\
Conversely, if $V= V_a( I) $, we write
\[ 
V^{*}= V_p( I^{*}) 
\]
$V^{*}$ is called the projective closudre of $V$ in $ \mathbb{P}^{n}$.\\
Similarly if $I$ is homogeneous, then
\[ 
I_{*} = \left\{ F_*| F\in I \right\} 
\]
and if $V= V_p( I) $, we set $V_* = V_a( I_*) $ 	

\begin{exemple}
Let $F= X_1^{2}-X_2$, then
\[ 
F^{*}= X_1^{2}-X_2X_3
\]

\end{exemple}
\begin{lemma}
If $V \subset \mathbb{A}^{n}$ is closed, then $\phi( V) = V^{*}\cap U $  \\
Conversely, if $V \subset \mathbb{P}^{n}$ is closed then $ \phi^{-1}( V\cap U) = V_*$ \\
In particular $\phi$ is a homeomorphism		
\end{lemma}
\begin{proof}
Recall that $\phi( x_1,\ldots,x_n) = [ x_1:\ldots:x_n:1] $.\\
For $V \subset \mathbb{A}^{n}$ write $V = V_a( F_1,\ldots,F_k) $ then
\[ 
V^{*}= V_p( F_1^{*},\ldots,F_k^{*}) 
\]
But $ F_i^{*}= X_{n+1}^{d}F_i(  \frac{x_1}{x_{n+1} ,\ldots, \frac{x_n}{x_{n+1} }}) $ $ F_i( v) =0 \iff F_i^{*}( \phi( v) ) =0 \implies \phi( V) = V^{*}\cap U$ 
\end{proof}






	


		
\end{document}	
