\documentclass[../main.tex]{subfiles}
\begin{document}
\lecture{4}{Fri 18 Mar}{Morphisms of Affine Varieties}
In general any morphism $\phi: V\to W$ induces a morphism of rings ( of $k$-algebras) $\tilde\phi:\Gamma( W) \to\Gamma( V) $ by composition, ie.
\[ 
\tilde\phi( f) =f\circ\phi
\]
\begin{propo}
This defines a one to one correspondence 
\[ 
\left\{ \text{ Morphisms $\phi:V\to W$  }  \right\} \leftrightarrow \left\{ \text{ $k$-algebra homomorphisms $\tilde\phi:\Gamma( W) \to \Gamma( V) $  }  \right\}
\]

\end{propo}
In particular $\phi$ is an isomorphism iff $\tilde\phi$ is an isomorphism.
\begin{proof}
Need to construct for any $\alpha:\Gamma( W) \to \Gamma( V) $ a morphism $ \overline\alpha:V\to W$ s.t.
\[ 
	\tilde{\overline{\alpha}} = \alpha
\]
Suppose $V \subset \mathbb{A}^{n}, W \subset \mathbb{A}^{m}$ and write
\[ 
	\Gamma( V) = \faktor { k[x_1,\ldots,x_n]} { I( V) } \text{ and } \Gamma( W) = \faktor { k[y_1,\ldots,y_m]} { I( W) }
\]
Choose lifts $T_i$ of $\alpha( [ Y_i] ) $ in $k[x_1,\ldots,x_n]$ .\\
In particular $\forall f \in\Gamma( W) $ and $F$ a lift,then
\[ 
\alpha( f) =F( T_1,\ldots,T_m) \mod I( V) 
\]
Then define $ T:\mathbb{A}^{n}\to \mathbb{A}^{m}:x\mapsto ( T_1( x) \ldots T_m( x) ) $ .\\
We claim that $T( V) \subset W$ .\\
From the diagram, we see that for any $G\in I( W) $, $G( T_1,\ldots,T_m) \in I( V) $, hence for any $v\in V$ , $0= G( T_1,\ldots,T_m) ( v) = G( T( v) ) $ which means that $T( v) \in W$.\\
Now 
\[ 
	\tilde{\overline{\alpha}} :\Gamma( W) \to\Gamma( V) 
\]
satisfies $\forall v \in V\forall f \in\Gamma( W) $ 
\[ 
	\tilde{\overline{\alpha}}( v) =f( \overline{\alpha}( v) ) = f( T( v) ) = \alpha( f( v) ) \implies \tilde{\overline{\alpha}} = \alpha
\]

\end{proof}
\begin{defn}
	The quotient field $K( V) $ of $\Gamma( V) $ is called the field of rational function on $V$.
\end{defn}
Let $f\in K( V) $ is defined at a point $p\in V$ if we can write $f$ as the quotient $f= \frac{a}{b}$ and $b( p) \neq 0$.\\
The pole set of $f\in K( V) $ is the set of points where $f$ is not defined.
\begin{rmq}
$\Gamma( V) $ is not a UFD in general, and so the presentation $ f= \frac{a}{b}$ is not unique.
\end{rmq}
\begin{exemple}
$V= ( xy-zw) \subset \mathbb{A}^{4}$ and let $ \overline{x}, \overline{y}, \overline{z},\overline{w}\in\Gamma( V) $ be the respective images.\\
Then $f= \frac{\overline{x}}{\overline{y}}= \frac{\overline{z}}{\overline{w}}$.\\
Hence $f$ is defined whenever $Y\neq 0$ or $w\neq 0$ \\
Hence the pole set of $f$ is $ \left\{ Y=0 \right\} \cap \left\{ W=0 \right\} $ 
\end{exemple}
\begin{defn}[Local Ring]
	The local ring of $V$ at a point $p\in V$ is a subring $K( V) $ defined by
	\[ 
	\O_p( V) = \left\{ f\in K( V) | f \text{ defined at $p$  }  \right\} 
	\]
	
\end{defn}
We have natural inclusions $\Gamma( V) \subset \O_p( V) \subset K( V) $ 
\begin{rmq}
$\Gamma( V) , \O_p( V) $ and $K( V)  $ are intrinsic to $V$, ie. if $V\simeq W$ then $\Gamma( V) \simeq \Gamma( W) $ and $\O_p( V) \simeq \O_{p'} ( W) $ 
\end{rmq}
\begin{propo}
Let $p\in V$ and $m_p \subset \Gamma( V) $ be the corresponding maximal ideal, then
\[ 
\O_p( V) \simeq \Gamma( V)_{m_p}  
\]
In particular $\O_p( V) $ is a noetherian local domain and we have that
\[ 
\Gamma( V) = \bigcap_{p\in V}  \O_p( V) \subset K( V) 
\]

\end{propo}
\begin{proof}
Recall that $m_p= \left\{ f\in\Gamma( V) | f( p) =0 \right\} $, then
\begin{align*}
	\Gamma( V) _{m_p} &= \left\{ f\in K( V) | f = \frac{a}{b}, b \notin m_p \right\} \\
	&= \O_p( V) 
\end{align*}
The rest follows from standard properties of localization.\\
In particular for any domain $R$ we have that
\[ 
R= \bigcap_{m \in R, m \text{ maximal } } R_m
\]
\end{proof}
Notice that the notions of regular functions is sufficient to define morphisms of local rings etc.\\
How can we extend this to quasi-affine varieties?
\begin{exemple}
Consider $ V( XY-1) \subset \mathbb{A}^{2}$ .\\
There is a natural projection $\phi:V( XY-1) \to x\in \mathbb{A}^{1}$.\\
The image of $\phi$ is $ \mathbb{A}^{n}\setminus \left\{ 0 \right\} $ quasi-affine and we'd like $\phi$ to be an isomorphism, ie.
\[ 
\phi^{-1}( x) = ( x, \frac{1}{x}) 
\]
Ie. the map $x\to \frac{1}{x}$ should be a regular function on $ \mathbb{A}^{1}\setminus \left\{ 0 \right\} $.
\end{exemple}
\begin{defn}
	Let $V \subset \mathbb{A}^{n}$ be quasi-affine.\\
	A map $f:V\to \mathbb{A}^{1}=k$ is called regular if $\forall v \in V$ there exists an open neighbourhood $v\in U \subset V$ and $g,h\in k[x_1,\ldots,x_n]$ s.t. $h( V) \neq 0 \forall x \in U$ and $f( x) = \frac{g( x) }{h( x) }$ 
\end{defn}
Why do we need the $U$ ?
\begin{exemple}
Consider again $V= V( XY-ZW) \setminus V( Y,W) $ and consider $f= \frac{x}{w}= \frac{z}{y}$ on $V$.\\
None of the two presentations works on $V$ 
\end{exemple}
\begin{defn}
	Let $\O( V) $ be the ring of regular functions on $V$ 
\end{defn}
\begin{rmq}
$f: V \mathbb{A}^{1}\setminus \left\{ 0 \right\} \to \mathbb{A}^{1}: x\mapsto \frac{1}{x}$ is regular.\\
Then we may take $U=V$, it is not hard to see that
\[ 
	\O( V) = k[x] [ \frac{1}{x},\frac{1}{x^{2}},\ldots] 
\]
In particular $\O( V) \supsetneq \Gamma( \mathbb{A}^{1}) $ 
\end{rmq}
If $V \subset \mathbb{A}^{n}$ is affine, then we have $ k[x_1,\ldots,x_n]\to \O( V): F \mapsto \left( v\mapsto F( v) \right)  $ .\\
\begin{propo}
For $V$ affine, we have that $\Gamma( V) \simeq \O( V) $.
\end{propo}
\begin{proof}
We have $O( V) \subset O_p(V ) $ $\forall p\in V$ hence $\Gamma( V) \hookrightarrow O( V) \hookrightarrow \bigcap_{p\in V} O_p( V) = \Gamma( V) $ 
\end{proof}
\begin{lemma}
Let $V$ be a quasi-affine subset and $f:V\to \mathbb{A}^{1}$ regular, then $f$ is continuous ( with respect to the Zariski topology) 
\end{lemma}
\begin{proof}
It is enough to show that $f^{-1}( X) $ is closed for any closed $X$ .\\
Without loss of generality $X= \left\{ x \right\} $.\\
Let $V = \cup_i U_ii$ a cover such that $f|_{U_i} = \frac{g_i}{h_i}$ and $h_i\neq 0$ on $U_i$.\\
Then $f^{-1}( X) \cap U_i = \left\{ v\in U_i | f( v) = \frac{g_i( v) }{h_i( v) } \right\}= \left\{ v\in U_i| x\cdot h_i( v) - g_i( v) =0 \right\}  $ which is an algebraic set.\\
Hence $f^{-1}( X) \cap U_i$ is closed which implies $f^{-1}( X) $ is closed.
\end{proof}
\begin{crly}
Let $f,g\in O( V) $ and $U \subset V$ non empty and open s.t. $f|_U= g|_U$ then $f=g$ 
\end{crly}
\begin{proof}
Using an exercise, open subsets are dense, since $f,g$ are continuous
\[ 
f|_U = g|_U \implies f|_{ \cl U} = f|_{ \cl V} 
\]

\end{proof}
\begin{rmq}
Let $U \subset V$ open, then the restriction of functions induces $\O( V) \to \O( U) $.\\
i.e. $\O( -) $ defines a sheaf of $k$-algebras on $V$.\\
Using this one can define a general algebraic as a topological space $X$ with some sheaf $\O_X$ which locally looks like a quasi-affine variety $V$ with $\O( -) $.\\
We'll define $\O_p( V) $ and $K( V) $ for $V$ quasi-affine, but these depend only on "local structure".\\
We can guess $\O_p( V) = \O_p( \cl V) $ and similarly for the quotient field.
\end{rmq}
\section{( Quasi-)Projective and general algebraic varieties}
Affine varieties usually "go to infinity" when we draw them.\\
This leads to complications in the theory
\begin{exemple}
Two distinct lines in $ \mathbb{A}^{2}$ they will intersect in $1$ point unless they're parallel
\end{exemple}
\subsection{Projective space}
\begin{defn}[Projective n-space]
$\mathbb{P}^{n}$ is the set 
\[ 
	\mathbb{P}^{n} = \faktor { K^{n+1}\setminus \left\{ 0 \right\} } { \sim} 
\]
Where we identify 
\[ 
	( x_1,\ldots, x_{n+1} ) \sim ( y_1,\ldots, y_{n+1} ) \text{ if  } \exists \lambda \in K^{\ast} \text{ s.t. } x_i = \lambda y_i
\]
Elements in $ \mathbb{P}^n$ 	are called points.
\end{defn}
If $p\in \mathbb{P}^{n}$ is the equivalence classe of $ ( x_1,\ldots, x_{n+1} )\in \mathbb{A}^{n+1} $ we write
\[ 
p = [ x_1:\ldots:x_n] 
\]
$x_1,\ldots, x_n$ are the homogenuous coordinates of $p$.
\begin{rmq}
Any point in $ \mathbb{A}^{n}\setminus \left\{ 0 \right\} $ defines a line through the origin and $x,y\in \mathbb{A}^{n}\setminus \left\{ 0 \right\} $ define the same line iff $x=\lambda y$ 
\end{rmq}

	

	
\end{document}	
