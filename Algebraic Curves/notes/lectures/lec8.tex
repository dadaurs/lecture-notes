\documentclass[../main.tex]{subfiles}
\begin{document}
\lecture{8}{Fri 29 Apr}{Dimension}
We'll use the following without proof:
\begin{thm}[Krull's Hauptidealsatz]
Let $A$ be a noetherian ring, $f\in A$ neither a 0-divisor nor a unit.\\
Then any minimal ideal containing $f$ has height $1$.
\end{thm}
We can now prove the above proposition.
\begin{proof}
$V \not \subset H = V( f) $ which means $f\neq 0$ in $\O( V) $.\\
Hence, $f$ is not a zero-divisor.\\
If $f$ is a unit in $\O( V)$, then $V\cap H= \emptyset$ and there is nothing to prove.\\
Otherwise every irreducible component of $V\cap H \subset V$ corresponds to a prime ideal $p$ in $\O( V) $ and $p$ is minimal .\\
Hence $ht( p) =1$ and thus $\dim V( p) = \dim ( \faktor { \O( V) } { p} ) = \dim( \O( V) ) - ht( p) = d-1$ \\

To prove the second statement we argue by induction.\\
For $r=0$, this is trivially true.\\
Let $I= ( f_1,\ldots,f_r) $ with $r \geq 1$ and $W$ an irreducible component of $V( I) $.\\
By induction any irreducible component $W'$ of $V( f_1,\ldots,f_{r-1} ) $ has dimension $n- ( r-1) $.\\
By $a$ every irreducible component of $W' \cap V( f_r) $ has dimension $ \geq n-r$.\\
$W$ is a union of irreducible components of $W'\cap V( f_r) $.
\end{proof}
\section{Local Properties of plane curves }
\begin{defn}[Plane curve]
	Let $F,G\in K[x,y]$ are equivalent if $\exists \lambda\in K^{\times} $ such that $F= \lambda  G$.
\end{defn}
An affine plane curve is an equivalence class of non constant polynomials in $K[x,y]$.\\
If $F= \prod F_i e^{i}$ with $F_i$ irreducible.\\
We call $F_i$ the components of $F$ ad $e_i$ the multiplicities.
Note that we can recover the $F_i$ from $V( F) $ but not the $e_i$'s.\\
The degree of an affine plane curve is the degree of $F$ as a polynomial.\\
A line is a curve of degree 1.\\
If $F$ is irreducible, we write $\Gamma( F) , \O_p( F) $ for $\Gamma( V( F) ) $ etc.
\begin{defn}[Singular point]
	Let $F$ be a plane curve and $P= ( a,b) \in F$.\\
	Then $P$ is called a simple point if either
	\[ 
	\frac{\del F}{\del X}( p)  = F_X( p) \neq 0 \text{ or } \frac{\del F}{\del Y}( p) \neq 0
	\]
	Ie., if the jacobian of $F$ has full rank.
In this case, the line $L( x,y) = F_x( p) ( X-a) + F_y( p) ( Y-b) $ is the tangent line to $F$ at $p$ 	.\\
A point which is not simple is called singular or multiple.\\
A curve with only simple points is called non-singular or smooth.
\end{defn}
We'll usually arrange things such that $p= ( 0,0) $ is a singular point of $F$.\\
Writing $F= F_m+\ldots + F_n$ for $F_i$ forms of degree $i$, $( 0,0) \in F\iff m \geq 1$ 	
\begin{defn}
The integer $m= m_{p} ( F) $ is called the multiplicity of $F$ at $p= ( 0,0) $ 	
\end{defn}
We have that $p=( 0,0) $ is simple iff $m=1$.\\
And in this case $F_1$ is the tangent line to $F$ at $( 0,0) $.\\
If $m=2$ then $p=( 0,0) $ is called a double point.\\
Since $F_m$ is homogeneous and in two variables we can factor it as
\[ 
F= \prod L_i^{r_i}
\]
where $L_i$ are distinct lines through the origin.\\
To see this, notice that the dehomonigenization $( F_m)_{\ast} = F_m( X,1) $ factors into linear terms.
\begin{defn}
	The $L_i$ are called the tangent lines to $F$ at $( 0,0) $.
\end{defn}
$L_i$ is a simple ( double, triple,...) tangent if $r_i=1 ( 2,3) $.\\
A point $P$ is an ordinary multiple point of $F$ , if $F$ has $m$ distinct tangents at $p$.\\
An ordinary double point is called a node. It is in many ways the simplest example of a singular point.
\begin{exemple}
Let $F= Y^{2}-X^{3}-X^{2}= F_2 + F_3$.\\
We can write $F_2= ( Y-X) ( Y+X) $ 
\end{exemple}
If $F= \prod F_i^{e_i}$ then $m_p( F) = \sum_{i}^{ }e_i m_p( F_i) $.\\
And if $L$ is a tangent to $F_i$ with multiplicity $r_i$, then $L$ is a tangent to $F$ with multiplicity $\sum e_i r_i$ .\\
For $p=( a,b) $ let $T( x,y) = ( x+a,y+b) $ and set $F^{T}( x,y) = F( T( x,y) ) $ .\\
We then define $m_p ( F) = m_{( 0,0) } ( F^{T}) $ and similarly for tangent lines, multiple points etc.
\begin{thm}
	Let $P$ be a point on an irreducible plane curve $F$ and $ \mathfrak{m}_p( F) \subset \O_p(F ) $ the corresponding maximal ideal.\\
	Then for sufficiently large $n$, we have that
	\[ 
		m_p( F) = \dim_K \left( \faktor { \mathfrak{m}_p( F) ^{n}}{\mathfrak{m}_p( F) ^{n+1}} \right) 
	\]
\end{thm}
\begin{exemple}
Let $F= X$ and $p=( 0,0) $,
then
\[ 
	\mathfrak{m}_P( F) = ( Y,X)  \subset \left( \faktor { k[x,y]} { ( x )} \right) _{( x,y) }  
\]
and
\[ 
	\mathfrak{m}_p( F) ^{n}= ( Y^{n}) 
\]
and thus $ \faktor { ( Y^{n}) } { ( Y^{n+1}) } $ is generated by $Y^{n}$ 

\end{exemple}
We'll need the following lemma
\begin{lemma}
	Let $I \subset K[X_1,\ldots,X_n]$ be an ideal such that $V( I)= \left\{ P_1,\ldots,P_n \right\}  $ is finite.\\
	Let $\O_i= \O_{P_i} \left( \mathbb{A}^{n}\right) $.\\
	Then there is an isomorphism
	\[ 
		\Phi: \faktor { K[x_1,\ldots,x_n]} { I} \to \prod \faktor { \O_i} { I\cdot \O_i} 
	\]
	
\end{lemma}
We'll prove the theorem assuming this.\\
Let's write $\O, \mathfrak{m}$ and $m$ for $\O_p( F) ,\mathfrak { m} _p( F) $ and $m_p( F) $.\\
We have the exact sequence 
\[ 
	0\rightarrow \faktor { \mathfrak{m}^{n}} { \mathfrak{m}^{n+1}} \rightarrow \faktor { \O} { \mathfrak{m}^{n+1}} \rightarrow \faktor { \O} { \mathfrak{m}^{n}} \to 0
\]
The theorem follows if $\dim ( \faktor { \O} { \mathfrak{m}^{n}} ) = n\cdot m +s$ and all $n \geq m$ 	
Without loss of generality $P= ( 0,0) $ and hence $\mathfrak{m}^{n}= I^{n}\O$.\\
Then
\[ 
	\faktor { \O} { \mathfrak{m}^{n}} = \faktor { \O} { I^{n}\O} = \faktor{ \O_p( \mathbb{A}^{2}) } { ( F, I^{n} )} = \faktor { K[x,y]} { ( F,I^{n}) } 	
\]
Notice $\forall G \in I^{n-m}$ $GF\in I^{n}$.\\
We get a short exact sequence
\[ 
	\faktor { K[x,y]} { I^{n-m}} \xrightarrow{F} \faktor { k[x,y]} { I^{n}} \to \faktor { K[x,y]} { ( F,I^{n}) } \to 0
\]
Since $\dim_k \faktor { K[x,y]} { I^{n}} = \frac{n( n+1) }{2}$ we have 
\[ 
	\dim \faktor { K[x,y]} { ( F,I^{n}) } = \frac{n( n+1) }{2}- \frac{( n-m) ( n-m+1) }{2}= mn - \frac{m( m-1) }{2}
\]



\end{document}	
