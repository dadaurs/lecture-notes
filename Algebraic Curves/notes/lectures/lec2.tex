\documentclass[../main.tex]{subfiles}
\begin{document}
\lecture{2}{Wed 02 Mar}{Affine space and algebraic sets}
\subsection{Affine spaces and algebraic sets}
Let $k$ be a field.
\begin{defn}
	For every $n \geq 0$ the affine $n$ -space $ \mathbb{A}_k^{n}$ is the set $k^{n}$ .	
\end{defn}
In particular $ \mathbb{A}^{0}$ is a point, $ \mathbb{A}^{1}$ is a line, $ \mathbb{A}^{2}$ the affine plane.\\
Given a subset $ S \subset k [ X_1,\ldots, X_n] $ of polynomials, we set
\[ 
V( S) = \left\{ x= ( x_1, \ldots, x_n) \in \mathbb{A}^{n}| f( x_1,\ldots, x_n)=0 \forall f \in S  \right\} 
\]
If $S$ is finite, we write $V(  f_1,\ldots, f_k) $ for $V( S) $.\\
If the set $S$ is a singleton, then we call $V( S) $ a hyperplane.\\
Any subset of $ \mathbb{A}^{n}$  $V$ algebraic if $V=V( S) $ for some subset of polynomials.
\begin{lemma}
	\begin{itemize}
	\item 
	Let $S \subset k[X_1,\ldots, X_n]$ and $I$ the ideal generated by $S$, then $V( S) = V( I) $.

\item Let $ \left\{ I_\alpha \right\} $ be a collection of ideals, then 
	\[ 
	V( \bigcup_\alpha I_\alpha) = \bigcap_{\alpha} V( I_\alpha) 
	\]
	
\item If $I \subset J$ then $V( J) \subset V( I) $ 
\item For polynomials $f,g \in k[x_1,\ldots, x_n]$, then $V( f) \cup V( g) = V( f\cdot g) $ \\
For ideals $I,J$ ideals, then $V( I) \cup V( J) = V( I\cdot J) $ where $IJ = \left\{ fg | f\in I , g\in J \right\} $ 
\item For $a= ( a_1,\ldots, a_n) \in \mathbb{A}^{n}, v( \left\{ x_1-a_1,\ldots \right\} ) = \left\{ a \right\} 	$
	\end{itemize}
\end{lemma}
\begin{proof}
\begin{enumerate}
\item Let $h \in \sum_i f_i g_i \subset I $ with $f_i \in S$ and $x\in V( S) $, then $f_i( x) =0\forall i$ hence $h(x)=0 \implies x\in V( I) 	\implies V( S) \subset V( I) 	 $.\\
	Furthermore, if $x\in V( I) $, then in particular $f( x) = 0 \forall f \in S \subset I$, hence $x \in V( S) $ and $V( S) \supset V( I) $ 
\item Let $x\in V( \cup I_\alpha) $, then for any $\alpha$ and $f\in I_\alpha$, we must have $f( x) =0$, hence $x\in V( I_\alpha) \implies x\in \bigcap_{\alpha} V( I_\alpha) $.\\
	Conversely, if $x\in \bigcap_{\alpha} V( I_\alpha) $ and $f\in \bigcup_{\alpha} I_\alpha$, then $f\in I_\alpha$ for some $\alpha$, then $f( x) =0$ hence $x\in V( \bigcup_\alpha I_\alpha) $ 
\end{enumerate}

\end{proof}
By Hilbert's basis theorem $k [ x_1,\ldots, x_n] $ is Noetherian hence every ideal is finitely generated.\\
\begin{crly}
Every algebraic set $V \subset  \mathbb{A}^{n}$ is of the form 
\[ 
V= V( f_1,\ldots, f_k) = V( f_1) \cap \ldots \cap V( f_k) 
\]

\end{crly}
\subsection{Ideals of a set of points and the nullstellensatz}
Using the previous section, we have a map
\[ 
	V:\left\{ \text{ Ideals in } k[X_1,\ldots, X_N] \right\} \mapsto \left\{ \text{ algebraic sets in } \mathbb{A}^{n} \right\} 
\]
Conversely, for any subset $X \subset \mathbb{A}^{n}$ we define
\[ 
	I( X) \coloneqq \left\{ f\in k[X_1,\ldots,X_N]| f( x) =0\forall x \in X \right\} \subset  k[X_1,\ldots, X_N]
\]

\begin{lemma}
\begin{enumerate}
\item If $X \subset Y$ then $I( X) \supset I( Y) $ 
\item For $J \subset k[X_1,\ldots, X_N]$ an ideal $I( V( J) ) \supset J$ 
\item For $W \subset \mathbb{A}^{n}$ algebraic, $V( I( W) ) =W$ 
\end{enumerate}

\end{lemma}
\begin{proof}
\begin{enumerate}
\item Let $f\in I( Y) $, then $f$ vanishes on $X$ and hence $f$ in $I(X) $ 
\item $ I( V( J) ) = \left\{ f\in k[x_1,\ldots, x_n]| f( x) =0 \forall x \in V( J)  \right\} \supset J$ 
\item By definition $V( I( X) ) \supset X$ for any $X$.\\
	If in addition, if $X= V( J) $ algebraic, then $V( I( X) ) = V( ( I( V( J) ) ) ) \subset V( J)= X $ 
\end{enumerate}
\end{proof}
There are essentially two reasons why $ I( V( J) ) \supsetneq J$ in general
\begin{enumerate}
	\item $J= ( x^{n}) \subset k[x]\implies V( x^{n}) = \left\{ 0 \right\} $ and $I( \left\{ 0 \right\} ) = ( x) $ 
	\item $ ( x^{2}+1) \subset \mathbb{R}[x]$ and $I( \emptyset) = \mathbb{R}[X]$ 
\end{enumerate}
\begin{lemma}
For any $X \subset \mathbb{A}^{n}$, $I( X) $ is a radical ideal
\end{lemma}
\begin{proof}
If $f^{n}\in I( X) $ for some $n$, then $f( x) ^{n}= 0$ and hence $f( x) =0$ 
\end{proof}
So the first phenomenon is related to the fact that $J$ is not radical, the second is related to the fact that $\mathbb{R}$ is not algebraically closed.
\begin{thm}[Hilbert's Nullstellensatz]
	Let $K$ be algebraically closed, $J \subset k[X_1,\ldots, X_n]$, then 
	\[ 
	I( V( J) ) = \sqrt{J} 
	\]
	
\end{thm}
Using this, there is a one to one correspondence
\[ 
	\left\{ \text{ radical ideals in $k[X_1,\ldots, X_n]$  }  \right\} \leftrightarrow \left\{ \text{ algebraic subsets of }  \mathbb{A}^{n} \right\} 
\]
\begin{thm}[Weak Nullstellensatz]
	Let $K$ be algebraically closed, every maximal ideal $I \subset K[X_1,\ldots, X_n]$ is of the form $ I= \left\{ x_1-a_1,\ldots, x_n-a_n \right\} $ with $a= ( a_i) \in \mathbb{A}^{n}$ 
\end{thm}
\begin{crly}
	Let $I \subset K[X_1,\ldots, X_n]$ be any ideal, then $V( I) $ is a finite set $\iff \faktor { k[X_1,\ldots, X_n]} { I} $ is a finite dimensional $K$- vector space.\\
	In this case
	\[ 
	|V( I) | \leq \dim_k \faktor { k[X_1,\ldots, X_n]} { I}
	\]
\end{crly}
\begin{proof}
	First, we show that if $ \faktor{k[x_1,\ldots,x_n]}{I}$ is finite dimensional, then $V( I) $ is finite.\\ 
	Let $I \subset k[X_1,\ldots, X_n]$ be any ideal and $P_1, \ldots, P_r \subset V( I) $ be distinct points, we want to show that $ r \leq  \dim_k \faktor{k[x_1,\ldots,x_n]}{I}$  .\\
	We can choose ( Exercise) $F_1,\ldots, F_r \in K[X_1,\ldots, X_n]$ s.t. $F_i( P_j) = \delta_{ij} $, then we write $f_1,\ldots, f_r$ for the residues of $F_1,\ldots, F_r$ in $\faktor{  K[X_1,\ldots, X_n] } { I} $.\\
	We claim $f_1,\ldots, f_r$ are linearly independent.\\
	Indeed suppose $\sum_i \lambda_i f_i = 0$, this implies $ \sum_i \lambda_i F_i \in I$ hence $0 = \sum \lambda_i F_i( P_j) $ which implies $\lambda_j = 0 $, hence the $f_i$ are linearly independent.\\
	It follows that $\dim_k \faktor{  K[X_1,\ldots, X_n] } { I} < \infty \implies |V( I) | < \infty $ and in this case $\dim_k \faktor{  K[X_1,\ldots, X_n] } { I} \geq |V( I) |$.\\
	Now assume $V( I) $ is a finite set $ \left\{ P_1,\ldots, P_r \right\} \subset \mathbb{A}^{n}$ and write $P_i = ( a_{i1}, \ldots, a_{in}  ) $ and define $F_j= \prod_{i =1}^{r}( X_j- a_{ij} ) $.\\
	By construction $F_j\in I( V( I) )= \sqrt{I}  $\\
	$\exists N>0$ such that $F_j^{N}\in I$.\\
	Hence $f_j^{N}=0$ in $\faktor{  K[X_1,\ldots, X_n] } { I} $, but $f_j^{N}= ( x_j^{Nr}) +  \text{ lower order terms } 	$.\\
	This means that $X_j^{Nr}$ is a $K$-linear combination of $ \left\{ 1,\ldots, X_j^{Nr-1} \right\} $.\\
	This means that $X_j^{s}$ is a linear combination for any $s>0$.\\
	Hence taking products for different $j's$, we see that the set $ \left\{ x_1^{m_1},\ldots, x_n^{m_n} \right\} $ generates $ \faktor{  K[X_1,\ldots, X_n] } { I}$ 
\end{proof}
Due to these theorems, we'll always suppose $K$ is algebraically closed.






		
\end{document}	
