\documentclass[../main.tex]{subfiles}
\begin{document}
\lecture{10}{Fri 13 May}{Intersection numbers}
\subsection{Intersection Numbers}
\begin{defn}[Intersection Number]
	Let $p\in \mathbb{A}^{2}$, $F,G$ two plane curves, then the intersection number of $F$ and $G$ at $p$ $I( p,F\cap G) = \dim_k \O_p( \mathbb{A}^{2}) /( F,G) \in \mathbb{N}\cup \infty 	$ 
\end{defn}
\begin{defn}[Transversal intersection]
	Two plane curves $F,G$ intersect transversally at $p\in F\cap G$ if $p$ is a  simple point of $F$ and $G$ and the tangent of $F$ at $p$ is different from the tangent of $G$ at $p$.
\end{defn}
In fact
\begin{propo}
Two plane curves intersect transversally iff $I( p,F\cap G) =1$ 
\end{propo}
\begin{proof}
	Suppose $F$ irreducible.
\[ 
	\faktor{ \O_p( \mathbb{A}^{2}) }{( F,G) }= ( \faktor{ k[x,y]}{( F,G) })_{m_p} = ( \faktor{\Gamma( F) }{G})_{m_p} = \faktor{\O_p( F) }{( g) }  
\]
where $g$ is the image of $G$ at $\O_p$ .\\
Note that since $G( p) =0, g\in m_p$.\\
Further, recall that the tangent of $G$ at $p$ is a uniformizer of $\O_p( F) $.\\
Thus 
\[ 
\faktor{\O_p( F) }{( g) }= \faktor{\O_p( F) }{( L) }\simeq k
\]

\end{proof}

\begin{exemple}
Let $G=x^{2}-y$ and $F=y$ \\
Then
\begin{align*}
	\faktor{\O_p( \mathbb{A}^{2}) }{( F,G) }\simeq ( \faktor{ k[x,y]}{( x^{2}-y,y) })_{m_p} = \faktor{k[x]}{( x^{2}) } 
\end{align*}

\end{exemple}
We now move on to axiomatic properties of intersection numbers	
\begin{thm}[Axiomatic properties]
	\begin{enumerate}
	\item $ I( p,F\cap G) = \infty $ iff $p$ lies on a common component of $F$ and $G$ 
	\item $I( p,F\cap G) =0$ iff $p\notin F\cap G$ 
	\item ( locality) $I( p,F\cap G) $ only depends on the components of $F$ and $G$ passing through $p$.
	\item $I(p, F\cap G )= I( p, G\cap F)  $ 
	\item ( lower bound) $I( p, F\cap G) \geq m_p( f) m_p( G)  $ and we have equality iff $F$ and $G$ have no common tangent line at $P$ .
	\item (compatibility with products) If $F= \prod F_i ^{r_i}$ and $G= \prod G_i ^{s_j}$, then
		\[ 
		I( p,F\cap G) = \sum_{i,j} r_i s_j
		\]
	
	\item $I( p,F\cap G) = I( p, F\cap ( G+ FA) ) \forall a \in k[x,y]$ 
	\item If $p$ is a simple point of $F$ 
		\[ 
		I( p,F\cap G) = ord_p^{F}( G) 
		\]
		
	\item (local-global) if $F,G$ have no common component, then  $\sum_{p \in F\cap G} I( p,F\cap G) = \dim_k( \faktor{k[x,y]}{( F,G) }) $ 
	\end{enumerate}
\end{thm}
\begin{proof}
Lets first prove $1$ and $9$ .\\
If $F,G$ have no common component, then $|V( F,G) | < \infty $ and by some lemma
\[ 
	\faktor{k[x,y]}{ ( F,G) }= \prod_{p\in F\cap G} \faktor{ \O_p( \mathbb{A}^{2}) }{( F,G) }
\]
Thus, $\forall p\in F\cap G, I( p,F\cap G) < \infty $, taking dimensions yields
\[ 
	\dim_k \faktor{ k[x,y]}{( F,G) } = \sum_{p\in F\cap G}^{ } I( p,F\cap G) 
\]
If $H$ is a common irreducible component of $F,G$.\\
The slogan is that $ \faktor{\O_p( \mathbb{A}^{2}) }{( F,G) }$ is ``bigger than'' $\Gamma( H) $.\\
Indeed, we have a surjective map ( just a quotient) 
\[ 
\faktor{\O_p( \mathbb{A}^{2}) }{( F,G) } \to \faktor{\O_p( \mathbb{A}^{2}) }{( H) }
\]
Since $\dim \Gamma( H) = \infty $, we have that $I( p,F\cap G) = \infty $.\\
Now we prove $2$.\\
Note that $p\in F\cap G \iff  m_p \supset ( F,G) \iff ( F,G) \subsetneq \O_p( \mathbb{A}^{2}) \iff I( p, F\cap G) \neq 0$.\\
To show $3$, write $F= F_1F_2$, if $F_2( P) \neq 0$, thus $F_2( P) $ is a unit in $\O_p( \mathbb{A}^{2}) $ thus $( F,G) = ( F_1,G) $, hence we get equality.
To show $6$, note that it is sufficient to show that $\forall F,G,H\in k[x,y]$
\[ 
I( p,F\cap GH) = I( P,F\cap G) + I( P,F\cap H) 
\]
We may also assume that $F,GH$ have no common irreducible components, otherwise the equality is trivial.\\
We'll decompose the ring $ \faktor{ \O_p( \mathbb{A}^{2}) }{ ( F,GH) }$.\\
We'll write $\O \coloneqq \O_p( \mathbb{A}^{2}) $.\\
Then we have a short exact sequence
\[ 
	0\to \faktor{\O}{( F,H) }\xrightarrow{\xi} \faktor{\O}{( F,GH) }\xrightarrow{\phi} \faktor{\O}{( F,G) }\to 0
\]
Where $\phi$ is the quotient map $( F,G) \supset ( F,GH) $ and $\xi$ is the multiplication by $G$.\\
Showing that this is exact will imply the claim, it is in fact enough to show that
\[ 
	0 \to \faktor{k[x,y]}{( F,H) }\xrightarrow{\xi} \faktor{k[x,y]}{( F,G) }\to \faktor{k[x,y]}{( F,G) }\to 0
\]
Note that $\phi$ is surjective since it is a quotient.\\
$\psi$ is injective:\\
Let $z\in k[x,y]$ , we want to prove that if $Gz \in ( F,GH) $ then $z\in ( F,H) $.\\
Assume that $Gz= AF + BGH \iff AF= G( z-BH) $, since $F,G$ have no common components $F| z-BH$ thus there exists $C$ such that $z= BH+CF\in( F,H) $ \\
To show $7$, simply notice that $( F,G) = ( F,G+AF) $.\\
Finally, wlog, suppose $F$ is irreducible (if it isn't split into irreducible components), then
\[ 
\faktor{\O( \mathbb{A}^{2}) }{( F,G) }= \faktor{\O_p( F) }{( g) }
\]
Finally, taking dimensions we get the desired equality.\\
We now go on to prove part 5.\\
Write $m= m_p( F) , n=m_p( G) $ and assume $p= ( 0,0) $, then $m_p = I = ( X,Y) $.\\
Consider the resolution
\[ 
	\faktor{ k[x,y]}{I^{n}}\times \faktor{ k[x,y]}{I^{m}}\to \faktor{ k[x,y]}{I^{n+m}}\to \faktor{k[x,y]}{( I^{n+m},F,G) }
\]
where the first map is given by $\psi( A,B) = FA+BG$.\\
Since $ \faktor{k[x,y]}{( I^{n+m},F,G) }\simeq \faktor{\O_p( \mathbb{A}^{2}) }{( I^{n+m},F,G) }$ .\\
Since $\O_p( \mathbb{A}^{2}) /( F,G) $ surjects onto this ring, $( I( p,F\cap G) ) \geq \dim_k \faktor{k[x,y]}{( I^{n+m},F,G) } \geq  \dim \faktor{k[x,y]}{I^{m+n}} - \dim_{k}  \faktor{k[x,y]}{ I^{n}}- \dim_k \faktor{k[x,y]}{I^{m}}= mn	$.\\
Thus $I( p,F\cap G) \geq mn$ with equality iff $\alpha: \faktor{ \O_p( \mathbb{A}^{2}) }{( F,G) }	\to \faktor{\O_p( \mathbb{A}^{2}) }{( I^{m+n},F,G) }$ is an isomorphism and $\psi$ is injective.\\
Thus the proposition follows if can show  that if $F,G$ have no common tangents at $P$, then $I^{t}\subset ( F,G) \cdot \O_p( \mathbb{A}^{2}) $ for $t \geq m+n-1$.\\
Further we have to show that $ \psi$ is injective iff $F,G$ have no common tangents.\\
First we show that $I^{t}\subset ( F,G) \cdot \O_p( \mathbb{A}^{2}) $ for larget $t$:\\
Let $V( F,G) = \left\{ P, Q_1,\ldots,Q_s \right\} $ and choose $H$ such that $H( p) \neq 0$ and $H( Q_i) =0$.\\
Thus $\exists N$ such that $( HX)^{N}$ and $( HY) ^{N}\in ( F,G) $.\\
But $H^{N}$ is a unit, thus $X^{N},Y^{N}\in( F,G) \cdot \O_p( \mathbb{A}^{2}) $.\\
Thus $I^{2N}\subset ( F,G) \cdot \O_p( \mathbb{A}^{2}) $.\\
Now let $L_1,\ldots,L_m$ be the tangents of $F$ at $P$ and $M_1,\ldots,M_n$ the tangents of $G$ at $P$.\\
Set $A_{i,j} = L_1\ldots L_iM_1\ldots M_j$ where we set $L_i=L_m$ if $i \geq m+1$ and $M_j=M_n\forall j>n$.\\
Using an exercise, the set $ \left\{ A_{i,j} | i+j=t \right\} $ is a basis for $ \faktor{I^{t}}{I^{t-1}}$.\\
Hence we need to show that $A_{i,j} \subset ( F,G) \cdot \O_p( \mathbb{A}^{2}) $ if $i+j \geq m+n-1$.\\
Wlog $i \geq m$ then $A_{i,j} =  A_{m,0} B$ with $B\in I^{i+j-m}$ and we can write $F= A_{m,0} + F'$ with $F'\in I^{m+1}$ and we can write $A_{i,j} -BF = BF' \in I^{i+j+1}$.\\
We can now repeat this with elements in $I^{i+j+1}, I^{i+j+2},\ldots$.\\

Let's show the second point.\\
Let $A,B\in k[x,y]$ with $ \psi( A,B) = AF+BG=0$.\\
Write $A= A_r+\ldots+A_d$ and $B=B_s+ \ldots +B_{d'} $.\\
We need to show $ r \geq n$ and $s \geq m$, assume not, then write 
\begin{align*}
	A_r F_m +B_s G_n + \text{ higher order } \in I^{m+n}
\end{align*}
Thus $A_r F_m = -B_s G_n$, since by assymption $F_m, G_n$ have no common factors, thus $F_m |B_s, G_n|A_r$ thus $m \leq s$ and $n \leq r$.\\
For the other direction, assume $L$ is a common tangent, then $F_m= L F_{m-1} '$ and $G_n = L G_{n-1}'$, then $\psi( G_{n-1}', -F_{n-1}') = G_{n-1'} F- F_{m-1}' G= 0$ 

		

\end{proof}
\subsection{Algorithm for $I( p,F\cap G) $ }
The main idea is to notice that it is easy to compute $\forall G$ 
\[ 
I( p,Y\cap G) 
\]
So let's suppose $p=( 0,0) $ ( wlog) and $F,G\ni p$.\\
Let $r= \deg ( F( x,0) ) , j = \deg ( G( x,0) ) $ and wlog $r \leq s$.\\
Now, if $r=0$, then $Y|F$ hence by $6$ 
\[ 
I( p,F\cap G) = \underbrace{I( p,Y\cap G)}_{ \text{ easy } }  + \underbrace{I( p,H\cap G)}_{ \text{ compute recursively } } 
\]
So write $G( X,0) = X^{m}( a_0 + a_1 X \ldots ) $ where $m>0$ because $p\in G$.\\
Then $I( p, Y\cap G) =m$\\
Now for the second case, assume $F( X,0) $ are monic, then let $H= G= X^{s-r}F$ then $I( p,F\cap G) = I( p, F\cap G) = I( p,F\cap H) $ and $\deg ( H( x,0) ) <r$ and we repeat until $r$ or $s$ is $=0$.\\



\end{document}	
