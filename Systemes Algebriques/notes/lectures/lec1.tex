\documentclass[../main.tex]{subfiles}
\begin{document}
\lecture{1}{Tue 15 Sep}{Introduction}
\begin{center}
	\textbf{Parties}
\end{center}

\begin{itemize}
	\item preuves et ensembles\\
	\item Theorie des nombres\\
	\item Theorie des groupes	
\end{itemize}
\chapter{Preuves}

Une grande partie du bachelor est de faire des preuves, il est donc important de comprendre quand une preuve est correcte.\\

Il y a deux types de preuves:
\begin{itemize}
	\item Preuves formelles\\
		Tres precise, mais difficile a lire.\\
	\item Preuves d'habitude\\
		Approximation des preuves formelles, en remplacant qqes parties par du texte ``humain''.
		Il faut s'assurer qu'on peut traduire cette preuve en preuve formelle.
\end{itemize}
\section{Proprietes de preuves formelles}
\begin{itemize}
	\item Elles utilisent seulement des signes/symboles mathematiques.
		\begin{itemize}
			\item $\exists$ ( existe)\\
			\item $\forall$ ( pour tout)\\
			\item $\exists !$ ( existe unique)\\
			\item  $\land$  ( et) \\
			\item $\lor$ ( ou) \\
			\item $\neg$ (non) \\
			\item  $\Rightarrow$ ( implique)\\
			\item etc
		\end{itemize}
	\item Elle consiste de lignes, et il y a des regles strictes que ces lignes doivent suivre.\\
	\item Regles
		\begin{itemize}
			\item Axiomes\\
			\item Propositions qu'on a deja montrees.\\
			\item Tautologies\\
				Exemples
				\[ 
					\neg ( A \lor B) \iff ( ( \neg A) \lor ( \neg B))
				\]
		\end{itemize}
	\item Modus Ponens: Si on a que 
		\begin{align*}
		\begin{cases}
		A \Rightarrow B\\
		A
		\end{cases}\\
		\end{align*}
		Alors $B$ est vrai\sidenote{ Pour lire plus, regarder ``Calcul des predicats'' sur wikipedia}
\end{itemize}

Dans ce cours $0$ n'est ni positif, ni negatif.

\begin{defn}[division d'entiers]\label{def:divison_dentiers}
	q divise $a$ ( $q \vert a$)
	si il existe un entier $r$ tel que $a=q\cdot r$.
\end{defn}
\begin{propo}[Division avec reste]\index{division}\index{reste}\label{propo:division_avec_reste}
	$a, q \neq 0$ entiers non-negatifs, 
	\[ 
		\Rightarrow \exists \text{ entiers non-negatifs }
	\]
	$b$ et $r$ t.q.
	\[ 
	a=b\cdot q + r
	\]
	et 
	\[ 
	r < q
	\]
\end{propo}
\begin{proof}
	\textbf{Unicite}
	Supposons que $\exists b,r,b',r'$ entiers non-negatifs et $r<q$ et $r'<q$.
	\begin{align*}
	a &= b q + r\\
	a &= b' q + r'
	\end{align*}
	Alors
	\[ 
		\underbrace{ ( b-b')  }_{-q,0,q}q = \underbrace{ r'-r }_{-q < r'-r < q}
	\]
	$\Rightarrow$ $r' -r = 0$ \\
	$(b-b') q = 0 \Rightarrow b=b'$
	
	\textbf{Existence}\\
	Par induction sur $a$.\\
	$\bullet$ $a=0 \Rightarrow b=0$ et $r=0$\\
	$0$ supposons que on connait l'existence pour a remplace par $a-1$.
	Alors, $\exists c,s$ tq
	\begin{align*}
	a-1 = c q + s\\
	s<q
	\end{align*}
	Alors, soit $s<q-1$ 
	\begin{align*}
		a= ( a-1) + 1 \\
		= cq +s +1
	\end{align*}
	Alors on peut dire que $s+1=r$.
	Sinon $s=q-1$ \\
	\begin{align*}
		a= ( a-1) +1\\
		= cq +\underbrace{ s +1 }_{=q}\\
		= (c+1) \cdot q + 0
	\end{align*}
\end{proof}
\section{Ensembles}
\underline{Premiere approche:}\\
ensemble $=$ \{ collection de choses \}\\
Exemple:\\
\[ 
	\underbrace{ \{\{ \{\emptyset\}, \emptyset \}\emptyset\} }_{A}
\]
$\Rightarrow A \in A$ 
\begin{propo}[Paradoxe de Russel]\index{Paradoxe}\index{Russel}\label{propo:paradoxe_de_russel}
\[ 
	B = \{ A \text{est un ensemble} \vert A \in A\}
\]
peut pas etre un ensemble.
\end{propo}
\begin{proof}
Supposons que $B$ est un ensemble et $B \subset B \iff B \not\subset B \iff B \subset B \ldots$ 
\end{proof}
\underlin{Question:}\\
Alors, qui sont les ensembles?
\underlin{Reponse:}\\
Axiome de Zermelo-Fraenkel\\
\hr\\
Quelques exemples de Zermelo-Fraenkel\\
1) et 2) impliquent que $\emptyset$ est un ensemble.\\
2)$A$ ensemble, $E(x)$ expression
$\rightarrow \{a \in A \vert E(a) \text{vrai}\}$
3) $A_i$ ensembles ( $i \in I$)\\
\[ 
	\rightarrow \Cup_{i\in I}  A_i
\]
est un ens.
4)...\\
5) axiome de l'ensemble puissance\\
$A$ ensemble
\[ 
	\rightarrow 2^{A} = \{ B \subseteq A \vert B \text{sous-ens.} de A\}
\]
Exemple: $\{0,1\}=A$ \\

\[ 
	2^{A} = \{ \emptyset , \{0\}, \{1\}, \{0,1\}\}
\]
6) $A_i$ ensembles ( $i \in I$)
$\rightarrow$ on peut choisir $a_i \in A_i$ a la meme fois\\
7) etc...\\
\underline{Consequences}
1) Les ensembles finis existent.\\
( i)  $\emptyset$ \\
 (ii)  $\{\emptyset\}$ \\
 ...\\

 2) $\mathbb{N} = \{0,1,2,\ldots\}$ est un ensemble
 3) $\mathbb{Z}= \{\ldots,-2,-1,0,1,2,\ldots\}$\\
 4) $2 \cdot \mathbb{N} = \{x \in \mathbb{N} \vert 2 \vert x\}$
 5) $A \subseteq B$ \\
 Alors on peut definir la difference
 \[ 
	 B \setminus A = \{x \in B \vert x \not\in A\}
 \]
 6) $A,B \subseteq C$ 
 \[ 
	 A \cap B = \{x \in C \vert x \in A, x \in B\} 
 \]
 
 













\end{document}	
