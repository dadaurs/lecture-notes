\documentclass[../main.tex]{subfiles}
\begin{document}
\lecture{22}{Mon 17 May}{Changements de variable}
\subsection{Formule de changement de variables}
En dimension $n=1$:\\
Soit $F = [ \alpha,\beta] , - \infty  < \alpha, \beta< \infty $ et $f: F \to \mathbb{R}$ continue.\\
On souhaite calculer $\int_F f( x) dx= \int_{ \alpha }^{ \beta }f( x) dx$\\
Soit alors $E = [ a,b] $ et $\psi: E \to \mathbb{R}$ de classe $C^{1}( E) $ et telle que $\psi( a) =\alpha, \psi( b) = \beta$. Alors on a
\[ 
	\int_{ \alpha }^{ \beta }f( x) dx = \int_{ a }^{ b } f( \psi( u) ) \psi'( u) du
\]
Pour generaliser a $n>1$, on se restreint a des bijections de classe $C^{1}$, voir a des diffeomorphismes.\\
Soit $\psi$ un diffeomorphisme entre $E = [ a,b] $ et $F= [ \alpha,\beta] $.\\
Puisque $\psi$ est un diffeomorphisme, on a que $\psi' \neq 0$ sur $E$.\\
Soit $\psi'>0$ sur $E$, alors $\psi$ est strictement croissante sur $E$, et donc
\[ 
	\int_{ E }^{  }f = \int_{ a }^{ b } f( \psi( u) ) |\psi'( u) | du
\]
Si $\psi'< 0$ sur $E$, alors $\psi$ est strictement decroissante, alors
\begin{align*}
	\int_{ F }^{  } f( x)  dx &= \int_{ \alpha }^{ \beta } f( x) dx\\
			 &= \int_{ \psi( b)  }^{ \psi( b)  }\\
			 &= \int_{ a }^{ b } f( \psi( u) ) |\psi'( u) | du\\
\end{align*}
Donc peu importante le signe de $\psi'$, on a 
\[ 
	\int_{ F }^{  } f = \int_{E} \tilde f ( u)  | \psi'( u) |du
\]
Considerons maintenant $n>1$ :\\
On considere une transformation affine $\psi $ de $ \mathbb{R}^n$.\\
Donc on peut ecrire
\[ 
	x= \psi( u) = Au
\]
Alors si on considere un petit carre centre en $( u_1,u_2) $, il sera transforme en un parallelepipede centre en $A( u_1,u_2) $.
\begin{figure}[H]
    \centering
    \incfig{transformation-affine}
    \caption{transformation affine}
    \label{fig:transformation-affine}
\end{figure}
Donc on peut calculer 
\[ 
	\vol F = v_1]\times v_2 = ( r, Ae_1) \times ( r_2A e_2) = r_1 r_2 |a_1\times a_2| = r_1 r_2 | \det A| = \vol E | \det A|
\]
Pour une transformation quelconque, on peut toujours l'ecrire localement comme
une transformation lineaire., ie
\[ 
	x = \psi( u_0)  + D \psi ( u-u_0) + R( u) = D \psi( u_0) u + \psi( u_0 ) - D \psi( u_0)  + R( u) 
\]
Donc on a ``en gros'' que 
\[ 
	``|dx| = |\det D \psi( u_0)|| du|'' 
\]
\begin{thm}
Soit $U,V \subset \mathbb{R}^n$ ouvertes et $\psi: U \to V$ tel que
\begin{itemize}
\item $\psi$ est un diffeomorphisme entre $U$ et $V$
\item $U$ et $V$ soient mesurables.
\item Pour tout $r>0$, $U \cap B( 0,r) $ et $V \cap B( 0,r) $ soient mesurables.
\item Toutes les composantes de $D \psi$ sont bornees sur tout sous-ensemble borne.
\end{itemize}
Soit encore $E \subset U$ borne non-vide et $F = \psi( E)  \subset V$ (  aussi un sous-ensemble borne) .\\
Alors 
\begin{enumerate}
	\item $E$ est mesurable ( au sens de Jordan)  si et seulement si $F = \psi( E) $ est mesurable.
	\item Si $E$ est mesurable et $f: F = \psi( E) \to \mathbb{R}$ est continue et bornee sur $F$, alors $f \in \mathcal{R}( F) $ et 
\[ 
	\int_{ F }^{  }f( x) dx = \int_{ E }^{  } f( \psi( u) ) | \det D \psi( u) | du
\]
\end{enumerate}
\end{thm}
\subsection{Quelques applications }
\begin{exemple}[1]
	Soit $F = \left\{ ( x,y) : 1 \leq  x^{2} + y^{2} \leq  4, x \geq 0, y \geq 0 \right\} $.\\
	\[ 
		f( x,y) = \frac{1}{1+ x^{2}+ y^{2}}
	\]
	On veut calculer $ \int_{ F }^{  }f( x,y) dx dy$.\\
	On a 
	\[ 
\begin{pmatrix}
x\\ y
\end{pmatrix} = \begin{pmatrix}
\rho \cos \theta\\ \rho \cos \theta
\end{pmatrix} : U = ]0, = \infty [  \times ] - \pi, \pi[ \to V = \mathbb{R}^{2} \setminus \left\{ y=0, x \leq 0 \right\} 
	\]
	On a bien que $E = [ 1,2]  \times [ 0, \frac{\pi}{2}]  = \psi^{-1}( F) $, mesurable et borne.\\
	Toute composante de $\psi $ et $D \psi$ est bornee sur $E$.\\
	On a 
	\[ 
	| \det D \psi| = \left| \det \begin{pmatrix}
		\cos \theta & - \rho \sin \theta \\ \sin \theta & + \rho \cos \theta
	\end{pmatrix}\right| = | \rho \cos^{2}\theta + \rho \sin ^{2} \theta | = \rho
	\]
Donc 
	
	\begin{align*}
		\int_{ F }^{  }f( x,y) dx dy &= \int_{ F }^{  } \frac{1}{1+ x^{2} + y^{2}}dx dy\\
					     &= \int_{ E }^{  }\frac{1}{1+ \rho^{2}} | \det D \psi( \rho ,\theta) d \rho d \theta\\
					     &= \int_{ [ 1,2] \times [ 0, \frac{\pi}{2}]  }^{  }\frac{1}{1+ \rho^{2}} \rho d\rho d \theta\\
					     &= \int_{ 1 }^{ 2 } \frac{\pi}{2} \frac{\rho }{1 + \rho ^{2}} d \rho
\end{align*}

	
\end{exemple}
\begin{exemple}[2]
	$F = \left\{ ( x,y) : x^{2} + y^{2} \leq 1 \right\} $ et $f( x,y) = \frac{1}{1+ x^{2} + y^{2}}$.\\
	Changement en coordonnees polaires.\\
	$f$ est continue sur $F$ donc $f \in \mathcal{R}( F) $.\\
	Soit $\tilde F = F \setminus \left\{ y=0, x \leq 0 \right\} $.\\
	Puisque $F \cap \left\{ y=0, x \leq 0 \right\} $ est un ensemble de mesure nulle, alor $\tilde F$ est mesurable et
\[ 
	\int_{ \tilde F }^{  } f( x,y) dx dy = \int_{ F }^{  }f( x,y) dx dy
\]
Donc
\[ 
	\int_{ F }^{  }f( x,y) dx dy = \int_{ \tilde F }^{  }f( x,y)  dx dy = \int_{ \tilde E }^{  } \frac{1}{1+ \rho ^{2}} \rho d \rho d \theta = \int_{ 0 }^{ 1 } ( \int_{ - \pi }^{ \pi } \frac{\rho }{1 + \rho ^{2}}) d\theta d \rho
\]

\end{exemple}
\subsection{Integrales generalisees}
\begin{defn}[Fonction absolument integrable]
	Soit $E \subset \mathbb{R}^n$ ouvert non vide ( pas forcement borne) et $f: E \to \mathbb{R}$ (  pas forcement borne).\\
	Soit $ \left\{ K_j, j \in \mathbb{N} \right\} $ une suite de sous-ensembles de $E$ tel que
	\begin{itemize}
	\item $K_j$ est borne, compact et mesurable
	\item $K_j \subset \ded { K_{j+1} } $ 
	\item $ \bigcup_{j \in \mathbb{N}} K_j= E$
	\end{itemize}
	Soit $f$ borne et integrable sur chaque $K_j$.\\
	On dit que $f$ est absolument integrable sur $E$ si 
	\[ 
		\lim_{j \to  + \infty} \int_{ K_{j}  }^{  }|f( x) | dx 
	\]
	existe ( finie) .
\end{defn}
Dans ce cas, on pose que $ \int_{ E }^{  }f( x) dx = \lim_{j \to  + \infty} \int_{ K_j  }^{  }f( x) dx$ 
On peut montrer que si $f$ est absolument integrable sur $E$, alors $\int_{ E }^{  }f( x) dx$ ne depend pas du choix de la suite $ \left\{ K_j \right\}_{j \in \mathbb{N}}  $.
	





\end{document}	
