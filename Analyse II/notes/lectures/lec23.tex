\documentclass[../main.tex]{subfiles}
\begin{document}
\lecture{23}{Wed 19 May}{Equations differentielles Ordinaires}
\section{Equations differentielles ordinaires}
\subsection*{Cas scalaire}
Soit $f: I \times E \to \mathbb{R},\quad I \subset \mathbb{R}$ un intervalle ouvert et $E \subset \mathbb{R}$ connexe et ouvert.\\
Soit $u : I \to E \subset \mathbb{R}$ de classe $C^{1}( I) $.\\
On dit que $u$ est solution de l'equation differentielle ordinaire definie par $f$ si 
\[ 
	u'( t) = f( t, u( t) ) \quad \forall t \in I
\]
Cette equation est appellee equation differentielle ordinaire scalaire du premier ordre.
\begin{itemize}
\item equation differentielle: equation qui fait intervenir la fonction inconnue ainsi que ses derivees
\item ordinaire: l'inconnue de $u$ depend d'une seule variable.
\item scalaire: $u$ est une fonction scalaire.
\item premier ordre: l'equation fait uniquement intervenir la derivee premiere.
\end{itemize}
\subsection*{Cas vectoriel}
La solution $\vec{u}: I \to E \subset \mathbb{R}^n$, avec $E \subset \mathbb{R}^n$ un ouvert.
\[ 
	\vec{u}( t) = ( u_1( t) , \ldots, u_n( t) ) 
\]
Et $\vec{f}: I \times E \to \mathbb{R}^n$.\\
Et l'EDO vectorielle aura la forme
\[ 
	\vec{u}'( t) = \vec{f}( t,\vec{u}( t) ) \quad \forall t \in I
\]
\begin{align*}
	u_1'( t) &= f_1( t, u_1( t) ,\ldots, u_n( t) ) \\
	\vdots &\\
	u_n'( t) &= f_n( t, u_1( t) ,\ldots, u_n( t) ) \\
\end{align*}
C'est donc un systeme de $n$ EDO scalaires couplees.
\subsection*{EDO d'ordre $n$ ( scalaire) }
On a 
\[ 
u: I \to \mathbb{R}
\]
de classe $C^{n}( I) $.\\
\[ 
f:I \times E \to \mathbb{R}\quad E \subset \mathbb{R}^n \text{ ouvert } 
\]
Et on peut alors ecrire l'equation differentielle ordinaire d'ordre $n$ 
\[ 
	u^{( n) } = f( t, u( t) , u'( t) , \ldots, u^{( n-1) }( t) ) \quad \forall t \in I
\]
La relation entre EDO d'ordre  $n$ scalaire et EDO vectorielle de dimension $n$.\\
On peut toujours reecrire une EDO d'ordre $n$ comme systeme d'EDO coupplees:\\
Soit $u$ une solution du systeme.\\
On introduit les $n$ fonctions 
\begin{align*}
	u_1( t) &= u( t) \\
	u_2( t) &= u'( t) \\
	\vdots &\\
	u_n ( t) &= u^{( n-1) }( t) 
\end{align*}
On peut donc ecrire
\[ 
	\vec{u}'( t) = \begin{pmatrix}
		u_1'( t) \\ \vdots \\ u_n'( t) 
	\end{pmatrix} = \begin{pmatrix}
	u'( t) \\
	u''( t) \\
	\vdots \\ 
	u^{(n)}( t) 
	\end{pmatrix} =
	\begin{pmatrix}
		u_2( t) \\
		u_3( t) \\
		\vdots\\
		u_n( t) \\
		f( t,u_1( t) , u_2( t) , \ldots, u_n( t) ) 
	\end{pmatrix} 
\]
Donc en definissant
\[ 
\vec{f}: I \times E \to \mathbb{R}^n
\]
avec
\[ 
	\vec{f}( t,\vec{u}) = 
	\begin{pmatrix}
	u_2\\
	u_3\\
	\vdots\\ 
	u_{n-1} \\
	f( t,u_1,u_2, \ldots, u_n) 
	\end{pmatrix} 
\]
Alors, on peut ecrire
\[ 
	\vec{u}'( t) = \vec{f}( t,\vec{u}( t) ) 
\]
\subsection{Probleme de Cauchy}
Soit $\vec{f}: I \times E \to \mathbb{R}^n ,\quad E \subset \mathbb{R}^n$ ouvert.\\
Prenons $\vec{f}$ continue sur $I \times E$, $( t_0,\vec{u}_0) \in I \times E$.\\
On cherche $\vec{u}:I \to \mathbb{R}^n$ telle que 
\[ 
	\vec{u}'( t) = \vec{f}( t,\vec{u}( t) ) \quad \forall t \in I
\]

Un probleme de cauchy est une solution $\vec{u}( t) $ telle que
\[ 
\begin{cases}
	\vec{u}'( t) = \vec{f}( t,\vec{u}( t) ) \\
	\vec{u}( t_0) = \vec{u}_0
\end{cases}
\]
Soit $I_+ = [ t_0,T[ $,avec $- \infty < t_0 < T \leq + \infty $, $\vec{f}: I_+ \times E \to \mathbb{R}^n$, $E \subset \mathbb{R}^n$ ouvert, $\vec{f}$ continue. On cherche $\vec{u}: I_+ \to \mathbb{R}^n$ continue sur $I_+$ et continument differentiable sur $\ded I_+ = ]t_0,T[$ tel que
\[ 
\begin{cases}
	\vec{u}'( t) = \vec{f}( t,\vec{u}( t) ) \\
	\vec{u}( t_0) = \vec{u}_0
\end{cases}
\]
On definit de la meme maniere le probleme a valeurs finales.
\begin{defn}[Solution locale ]
	On appelle solution locale du probleme de Cauchy un couple $( J, \vec{u}) $ ou $J$ est un intervalle contenu dans $I$ , qui contient $t_0$ et $\vec{u} \in C^{1}( J) $ qui satisfait le probleme de Cauchy.
	\begin{itemize}
		\item On dit qu'une solution locale $( K,w) $ du PC prolonge strictement $( J,u) $ si $J \subset K $, $J \neq K$ et $w( t) = u( t) \forall t \in J$
		\item On dit que un couple $( J,u )$ est une solution maximale si il n'existe pas d'autres solutions locales qui la prolongent.
		\item On dit que $( J,u) $ est globale si $J=I$.
	\end{itemize}
	
\end{defn}
\subsection{Methode de separation de variables}
Pour une EDO scalaire du premier ordre, si
\[ 
f: I \times E \to \mathbb{R}, I \subset \mathbb{R}, E \subset \mathbb{R}
\]
Si $f$ se laisse ecrire comme
\[ 
	f( t,x) = g( t) k( x) \quad g: I \to \mathbb{R}, k : E \to \mathbb{R} \text{ continue } 
\]
On peut considerer le probleme de cauchy
\[ 
\begin{cases}
	u'( t) = g( t) k( u( t) ) \\
	u( t_0) = u_0
\end{cases}
\]
\subsection*{Methode ``informelle'' }
On cherche 
\[ 
	\frac{du}{dt}= g( t) k( u) 
\]
\begin{align*}
	\Rightarrow \frac{du}{k( u) } &= g( t) dt
	\int_{ u_0 }^{ u }\frac{du}{k( u) } &= \int_{ t_0 }^{ t }g( t) dt
\end{align*}
Si on definit
\[ 
	G( t) = \int_{ t_0 }^{ t }g( s) ds \quad F( u) = \int_{ u_0 }^{ u }\frac{1}{k( y) }dy
\]

Si $F$ est inversible, alors on a
\[ 
	u= F^{-1}( G( t) - G( t_0) + F( u_0) ) 
\]
\begin{thm}
	Soit $I, \tilde E \subset \mathbb{R}$ des intervalles ouverts, $g: I \to \mathbb{R}$ continue, $K: \tilde E \to \mathbb{R}$ continue et tel que $k( u) \neq 0\forall u \in \tilde E$.\\
	Soit $( t_0,u_0) \in I \times \tilde E$ et
	\[ 
		G( t) = \int_{ t_0 }^{ t }g( s) ds, F( u) = \int_{ u_0 }^{ u }\frac{1}{k( y) }dy
	\]
	Alors il existe un intervalle $J \subset I$ contenant $t_0$ tel que $G( J) \subset \im F$ et une fonction $u: J \to \mathbb{R}$ de classe $C^{1}( J) $ definie par 
	\[ 
		u( t) = F^{-1}( G( t) ) 
	\]
	et le couple $( J,u) $ est solution du probleme de Cauchy.\\
	De plus, cette solution est unique au sens que tout autre solution locale $( K,w) $ du Probleme de Cauchy satisfait
	\[ 
		w( t) = u( t) \forall t \in J\cap K
	\]
	tant que $k( u( t) ) \neq 0$ alors $F^{-1}( G( t) ) $ donne une solution de l'equation.
	
\end{thm}







\end{document}	
