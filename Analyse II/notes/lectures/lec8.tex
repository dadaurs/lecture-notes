\documentclass[../main.tex]{subfiles}
\begin{document}
\lecture{8}{Wed 17 Mar}{Derivee partielles et directionnelle}
\section{Derivees de fonctions a plusieurs variables}
\subsection{Derivees Directionelles}
\begin{defn}[Derivees directionnelle]
	Soit $f:E\subset \mathbb{R}^n \to \mathbb{R}$ et $\vec{x_0} \in \overset{\circ}{E}$ et $\vec{v} \in \mathbb{R}^n$ un vecteur arbitraire.\\
	On dit que $f$ est derivable dans la direction $ \vec{v}$, au point $x_0$, si 
	 \[ 
		 \lim_{t \to 0} \frac{f( x_0 + tv) - f( x_0) }{t}
	\]
	existe et on note $ D_v f( x_0) $.
	
\end{defn}
Si on prend $\N { \vec{v}}$ ( norme euclidienne) , alors on appelle $D_vf( x_0) $ la derivee directionnelle de $f$ dans la direction $\vec{v}$ au point $x_0$.\\
en particulier, on peut prendre $\vec{v} = e_i$, dans ce cas on utilise la notation
\[ 
	D_{e_i} f( x_0) = \frac{\del f}{\del x_i}( x_0) = \lim_{t \to 0} \frac{f( x_0 + t e_i) - f( x_0)}{t} 
\]
et on appelle $\frac{\del f}{\del x_i}( x_0)$ la i-eme derivee partielle de $f$ au point $x_0$.
\begin{defn}[Gradient]
	Soit $f: E \subset \mathbb{R}^n\to \mathbb{R},  x_0 \in \ded E$.\\
	Si toutes les derivees partielles de $f$ en $x_0$ existent, alors on appelle le vecteur gradient
	\[ 
		\nabla f( x_0) \in \mathbb{R}^n, \nabla f( x_0) = 
		\begin{pmatrix}
			\frac{\del f}{\del x_i}( x_0) \\
			\vdots\\
			\frac{\del f}{\del x_n}( x_0) 
		\end{pmatrix}
	\]
	
\end{defn}
\begin{defn}[Matrice Jacobienne]
	On appelle matrice Jacobienne $D f( x_0) \in \mathbb{R}^{1\times n}$
	\[ 
		D f( x_0) = \left( \frac{\del f}{\del x_1}( x_0) , \ldots, \frac{\del f}{\del x_n}( x_0)  \right) 
	\]
	

\end{defn}

\subsection{Fonctions Differentiables}


\begin{defn}[Differentiabilite]
	Soit $f:E \subset \mathbb{R}^n \to \mathbb{R}$ et $x_0 \in \ded E$. On dit que $f$ est differentiable ( ou derivable)  en $x_0$ si il existe une application lineaire $L_{x_0} : \mathbb{R}^n \to \mathbb{R}$ et une fonction $g: E \to \mathbb{R}$ tel que 
	\[ 
		f( x) = f( x_0) + L_{x_0}( x-x_0) + g( x) \forall x \in E
	\]
	et $\lim_{x \to x_0} \frac{g( x) }{\N{x-x_0}}=0$.
	
\end{defn}
\begin{thm}
Soit $f: E \subset \mathbb{R}^n \to \mathbb{R} $ differentiable en $x_0 \in \ded E$, alors
\begin{itemize}
\item Toutes les derivees partielles de $f$ en $x_0$ existent.
\item On a 
\[ 
	L_{x_0} ( x-x_0) = \sum \frac{\del f}{\del x_i}( x_i - x_{0} ) = Df( x_0) ( x-x_0) 
\]
\item Toutes les derivees directioneelles existent et
	\[ 
		D_{v} f( x_0) = \sum \frac{\del f}{\del x_i}( x_0) v_i = \nabla f( x_0)^{T}\vec{v} = D f( x_0) \vec{v}
	\]

\item $f$ est continue en $x_0$.
	
\end{itemize}
\end{thm}
\begin{proof}
\begin{itemize}

\item On a
	\begin{align*}
		\frac{\del f}{\del x_i}( x_0) &= \lim_{t \to 0} \frac{f( x_0 + t e_i) -f( x_0) }{t}\\
					      &= \lim_{t \to 0} \frac{f( x_0) + L_{x_0} ( x_0+ t e_i - x_0) + g( x_0+ t e_i) }{t}\\
					      &= L_{x_0}( e_i)  + \lim_{t \to 0} \frac{g( x_0 + t e_i) }{t}\\
					      = L_{x_0} a_i
	\end{align*}
	
\item On a 
	\[ 
		f( x) = f( x_0) + L_{x_0} ( x-x_0) + g( x) 
	\]
	Donc
	\[ 
		\lim_{x \to x_0} f( x) = f( x_0)  + \lim_{x \to x_0} L_{x_0} ( x-x_0) + \lim_{x \to x_0} g( x) = f( x_0) 
	\]


\item 
	\begin{align*}
		D_v f( x_0) = \lim_{t \to 0} \frac{f( x_0 + tv) - f( x_0) }{t}\\
		= \lim_{t \to 0}  \frac{D f( x_0) t v + g ( x_0 + tv) }{t}\\
		= Df( x_0)  \vec{v}
	\end{align*}
	
	
	
\end{itemize}

\end{proof}



\end{document}	
