\documentclass[../main.tex]{subfiles}
\begin{document}
\lecture{10}{Mon 29 Mar}{Derivees d'ordre superieur}
\begin{proof}

\begin{figure}[H]
    \centering
    \incfig{thmschwarz}
    \caption{thmschwarz}
    \label{fig:thmschwarz}
\end{figure}
Soit $ s,t>0$ suffisamment petit tel que
\[ 
x+ se_i , x + te_j, x + se_i + te_j \in E
\]
Posons
\begin{align*}
	\Delta ( s,t)  &= f( x + s e_i + t e_j)  - f( x+ se_i)  - f( x+ t e_j ) + f( x) \\
		       &= \frac{\del f}{\del x_j}( x + s e_i ) t - \frac{\del f}{\del x_j} ( x) t \\
		       &= \frac{\del }{\del x_i} \frac{\del ^{2} f}{\del x_j} st\\
		       &= \frac{\del }{\del x_i} f( x+ te_j) s - \frac{\del }{\del x_i }f( x) s
\end{align*}
Plus formellement, on peut ecrire
\[ 
	\Delta ( s,t)  = ( f( x+ se_i + te_j)  - f( x+ se_i) ) - (  f( x+ te_j) - f( x) ) 
\]
On definit
\[ 
	g( \xi)  = f(  x+ \xi e_i + t e_j) - f( x+ \xi e_i) 
\]
et donc
\begin{align*}
	\Delta ( s,t) = g( s) - g( 0) 
\end{align*}
et $g$ est derivable car $f$ est derivable
\[ 
	g'( \xi) = \frac{\del f}{\del x_i} (  x+ \xi e_i + t e_j)  - \frac{\del f}{\del x_i} ( x+ \xi e_i) 
\]
par le TAF, on a
\begin{align*}
	g( s) - g( 0) &= g'( \tilde s)  s\\
		      &= ( \frac{\del f}{\del x_i}( x+ \tilde s e_i + t e_j)  - \frac{\del f}{\del x_i}( x+ \tilde s e_i) ) s\\
\end{align*}
On definit maintenant
\[ 
	\phi( y) = \frac{\del f}{\del x_i}( x+\tilde s e_i + y e_j) 
\]
Alors on a
\begin{align*}
	\Delta ( s,t) = ( \phi( t) - \phi( 0) ) s
\end{align*}
A nouveau, $\phi$ est derivabe, et donc on a 
\begin{align*}
	\Delta ( s,t) &= \phi'( \tilde t)  ts\\
		      &= \frac{\del ^{2}}{\del x_j \del x_i} (  x+ \tilde s e_i + \tilde t e_j)  ts
\end{align*}
Si on prend $t=s$, on a
\begin{align*}
	\lim_{s \to 0} \frac{1}{s^{2}}\Delta ( s,s) = \lim_{s \to 0} \frac{1}{s^{2}} \left( \frac{\del ^{2}f}{\del x_j \del x_i}\left(  x+ \tilde s + \tilde t e_j\right) s^{2}\right)
\end{align*}
On peut appliquer exactement le meme raisonnement dans l'autre sens, et on obtient le resultat desire.
\end{proof}
\subsection{Derivees d'ordre superieur}
Soit $E\subset \mathbb{R}^n$ ouvert non vide, $f: E \to \mathbb{R}$ et on fixe $i_1, \ldots, i_p \in \left\{ 1, \ldots, n \right\} $.\\
On definit la derivee partielle par rapport aux variables $x_{i_1} , \ldots, x_{i_p} $, on note alors
\begin{align*}
	\frac{\del ^{p}f}{\del x_{i_p} \ldots \del x_{i_1}  } = \frac{\del }{\del x_{i_p} }(  \ldots ( \frac{\del }{\del x_{i_2} }( \frac{\del f}{ \del x_{i_1} }) ) ) (x) 
\end{align*}
\begin{crly}
	Soit $i_1, \ldots, i_p$ fixe et $\sigma $ une permutation des nombres $ \left\{ 1, \ldots, p \right\} $.\\
	Si $ \frac{\del ^{p}}{ \del x_{i_p} \ldots \del x_{i_1} }$ et $ \frac{\del ^{p}f}{ \del x_{i_{\sigma( p) } } \ldots \del x_{i_{\sigma( 1) } } 	}$ existent et sont continues en $x$ pour toute permutation alors ils sont egaux
\end{crly}

\subsection{Developpement limite et formule de Taylor}
On veut generaliser la definition pour la dimension 1, on veut un polynome de degre $p$ dans les variables $( x_1, \ldots, x_n) $, en utilisant la notation multi-entiers, on note
\begin{align*}
	p( x) &= \sum_{\alpha =  ( \alpha_1, \ldots,\alpha_n),|\alpha|\leq 2 }  c_{\alpha} x^{\alpha}\\
\end{align*}
De maniere generale, on peut donc ecrire
\begin{align*}
	q( x) = \sum_{\alpha \in \mathbb{N}^{n},|\alpha| \leq p} c_{\alpha} x^{\alpha}
\end{align*}
Le developpement limite d'ordre $p$ d'une fonction
$f: E \to \mathbb{R}$ autour d'un point $x \in \ded E$, aura donc la forme
\begin{align*}
	f( y) = \sum_{\alpha \in \mathbb{N}^{n},|\alpha| \leq p} c_{\alpha} ( y-x) ^{\alpha} + R_p( y)
\end{align*}
Ou $R_p$ satisfait
\[ 
	\lim_{y \to x} \frac{R_p( y) }{\N { y -x}^{p}}=0
\]
Soit $f: E \to \mathbb{R}$, $f \in C^{p+1}( E)$, $E$ un ouvert non vide et soient $x,y \in E$ tel que $[x.y]\in E$, soit $g( t) = f( x+ t( y-x) ) $, pour $t \in [ 0,1] $, on voit que $g \in C^{p+1}( [ 0,1] ) $.\\
On peut donc ecrire
\begin{align*}
	g(t )  &= g( 0)  + g'( 0) t + \ldots + \frac{g^{p}( 0) }{p!}t^{p} + R_p( y) 
\end{align*}
On a donc
\begin{align*}
	g'( t) = \sum_{i_1=1}^{ n} \frac{\del f}{\del x_{i_1} }( x_t) \frac{d( x_t)i_1 }{dt} = \sum_{i_1=1}^{ n} \frac{\del f}{\del x_{i_1} }( xt) ( y_{i_1} - x_{i_1} ) = \sum_{|\alpha|=1} \frac{\del ^{|\alpha|}f}{\del x^{\alpha}}( y-x) ^{\alpha} \\
\end{align*}
De meme, on trouve
\begin{align*}
	g''( t)  &= \frac{d}{dt}(  \frac{d}{dt}f( x_t) ) \\
		 &= \sum_{|\alpha|=2} \frac{2!}{\alpha!} \frac{\del^{2}f}{\del x^{\alpha}}( x_t) ( y-x) ^{\alpha}
\end{align*}
La formule de Taylor s'ecrit donc
\begin{align*}
	f( y) = g( 1) &= \sum_{k=0}^{ p} \frac{g^{k}( 0) }{k!}t + R_p( y) \\
		      &= \sum_{k=0}^{ p} \sum_{|\alpha| =k} \frac{1}{k!} \frac{k!}{\alpha!} \frac{\del ^{\alpha}}{\del x^{\alpha}} ( x) ( y-x) ^{\alpha} + R_p( 1) 
\end{align*}
La formule de lagrange donne
\begin{align*}
	R_p ( 1) = \sum_{|\alpha| = p+1} \frac{1}{\alpha!} \frac{\del ^{|\alpha| }f}{\del x^{\alpha}} ( x + \theta ( y-x) ) ( y-x) ^{\alpha}
\end{align*}

 







    


\end{document}	
