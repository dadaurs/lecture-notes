\documentclass[../main.tex]{subfiles}
\begin{document}
\lecture{25}{Mon 31 May}{theoremes d'existence et edo du second ordre}
Soit $I \subset \mathbb{R}$ un intervalle ouvert, $E \subset \mathbb{R}^n$ ouvert, $\vec{f}: I \times E \to \mathbb{R}^{n}$. On considere le probleme de Cauchy:\\
Trouver $\vec{u} : I \to \mathbb{R}^{n}$ de classe $C^{1}$ tel que
\[ 
	\begin{cases}
	\vec{u}'( t) = \vec{f}( t,u( t) ) \forall t \in I\\
	u( t_0) = u_0
	\end{cases}
\]
\begin{thm}[Cauchy-Peano]
	Si $\vec{f}$ est continue, alors $\forall ( t_0, \vec{u}_0) \in I \times E$ , il existe au moins une solution locale $( J,u) $ du probleme de Cauchy avec $J \subset I$.
\end{thm}
\begin{thm}[Cauchy-Lipschitz/version locale]
	Si $\vec{f}$ est continue et localement lipschitzienne par rapport au deuxieme argument, alors pour tout $ ( t_0, u_0) \in I \times E$, il existe une unique solution locale $ ( J,u) $ du probleme de Cauchy avec $J \subset I$.	
\end{thm}
\begin{defn}
Soit $I \subset \mathbb{R}$ ouvert et $\vec{f} : I \times \mathbb{R}^{n}\to \mathbb{R}^n$.\\\
On dit que $\vec{f}$ est globalement Lipschitzienne par rapport au deuxieme argument s'il existe une fonction continue non negative $l : I \to \mathbb{R}_+$ telle que
\[ 
	\N { f( t,\vec{u}) - f( t,v) } \leq  l( t) \N { u -v}  \forall u,v \in \mathbb{R}^n
\]
\end{defn}
\begin{thm}[Cauchy-Lipschitz/ version globale]
	Si $\vec{f}$ est globalement lipschitzienne par rapport au deuxieme argument alors $\forall ( t_0, \vec{u_0}) \in I \times \mathbb{R}^n$ il existe une unique solution globale $u \in C^{1}( I, \mathbb{R}^n) $ du probleme de Cauchy.
\end{thm}
\subsection{EDO du second ordre}
On cherche des solutions a des expressions du type
\[ 
	u''( t) = f( t,u( t) , u'( t) ) 
\]
\begin{itemize}
\item $I \subset \mathbb{R}$ ouvert
\item $E \subset \mathbb{R}^{2}$ ouvert
\item $f: I \times E \to \mathbb{R}$ continue
\end{itemize}
On cherche $u: I \to \mathbb{R}$ de classe $C^{2}$.
On peut reecrire ce systeme comme systeme d'EDO du premier ordre.
\begin{align*}
	u_1( t) = u( t) \\
	u_2(t ) = u'(t ) \\
	\begin{cases}
		u_1'( t) = u_2( t) \\
		u_2'( t) = f( t,u( t) ,u_2( t) ) 
	\end{cases}
\end{align*}
Ainsi, si on considere le probleme de Cauchy
\begin{align*}
	u'( t) = f( t,u( t) ) \\
	u( t_0 ) = u_0 = \begin{pmatrix}
	x_0\\y_0
	\end{pmatrix} 
\end{align*}
\subsubsection{EDO lineaires du second ordre}
\[ 
	u'' ( t) = g( t) - a( t) u'( t) - b( t) u( t) , \quad t \in I \subset \mathbb{R}
\]
avec $g,a,b\in C^{0}( I) $ .\\
On peut prendre le probleme de Cauchy 
\[ 
\begin{cases}
	u''( t) + a( t) u'( t) + b( t) u( t) = g( t) \\
	u( t_0) = u_0, u'( t_0) =v_0
\end{cases}
\]
Pour trouver la solution on peut d'abord prendre le probleme homogene associe, 
\[ 
	u''( t) + a( t) u'( t) + b( t) u( t) =0
\]
\begin{defn}
On dit que deux solutions $z_1, z_2: I\to \mathbb{R}$ de classe $C^{2}$ de l'equation homogene
\[ 
	u''( t) + a( t) u'( t) + b( t) u( t) =0
\]
Sont lineairement independantes si $\forall \alpha, \beta\in \mathbb{R}$ 
\[ 
	\alpha z_1( t) + \beta z_2( t) = 0 \forall t \in I \Rightarrow \alpha = \beta =0
\]
Inversement  $z_1, z_2$ sont lineairements independants si il existe deux constantes $\alpha, \beta$ non simultanement nulles telles que
\[ 
\alpha z_1 + \beta z_2 = 0
\]
\end{defn}
Soit $S = \left\{ z \in C^{2}( I) : z'' + az' + bz = 0 \right\} $ l'ensemble des solutions de l'equation homogene.
\begin{itemize}
\item $S$ est un espace vectoriel
\item $S$ est un espace vectoriel de dimension 2 et
	\[ 
	S = \left\{ C_1 z_1 + C_2 z_2: C_1, C_2 \in \mathbb{R} \right\} 
	\]
	
\end{itemize}	






\end{document}	
