\documentclass[../main.tex]{subfiles}
\begin{document}
\lecture{21}{Wed 12 May}{caracterisation des fonctions integrables}
\begin{thm}
Un ensemble borne $E \subset \mathbb{R}^n$ est mesurable au sens de Jordan si et seulement si $\del E$ est negligeable.
\end{thm}
\begin{proof}
``$\Rightarrow $''   $E$ mesurable implique $\del E$ negligeable.\\
Soit $R \subset \mathbb{R}^n$ un pave contenant $E$, alors $ \mathbb{I}_E \subset \mathcal{R}( R) $, donc $\forall \epsilon >0$, il existe une partition $P_\epsilon$ du pave $R$ telle que
\[ 
	\overline S ( \mathbb{I}_E, P_\epsilon)  - \underline { S}  ( \mathbb{I}_E, P_\epsilon) = \sum_{Q\in P_\epsilon}^{ } \vol Q < \epsilon
\]
Soit $\mathcal{E} = \left\{ Q \in P_{\epsilon} : Q \cap E \neq \emptyset, Q\cap R \setminus E \neq \emptyset \right\} $.\\
Par hypothese,
\[ 
\sum_{Q \in \mathcal{E}}^{ } \vol Q < \epsilon
\]
De plus, si $ \forall x \in \del E$, il existe au moins un $Q \in P_\epsilon$ qui le contient. Alors soit $x \in \ded Q$, soit $x \in \del Q $.\\
Si $x \in \ded Q \Rightarrow  Q \in \mathcal{E}$. \\
\[ 
\del E \subset \underbrace{\bigcup_{Q \in \mathcal{E}} Q \cup}_{A} \underbrace{\bigcup_{Q \in P_{\epsilon} } \del Q}_{B}
\]
Notons que $\forall Q \in P_\epsilon, \del Q = \bigcup_{i=1} ^{2n}R_i, \vol R_i = 0$.\\
Donc on a recouvert $\del E$ par un nombre fini de paves et 
\[ 
	\vol ( A \cup B) = \vol A + \vol B \leq \epsilon
\]
Donc $\del E$ est negligeable.\\
$\Leftarrow$  $\del E$ $ \Rightarrow $ $E$ mesurable.\\
$\del E$ negligeable $ \Rightarrow $ $ \mathbb{I}_{\del E} \in \mathcal{R}( R) $ et $\int_{ R }^{  }\mathbb{I}_{\del E} $.\\
$\forall \epsilon>0$, $\exists P_{\epsilon} $ une partition de $R$ telle que
\[ 
	\overline{S}( \mathbb{I}_{\del E} , P_\epsilon) = \sum_{ \substack { Q \in P_\epsilon\\ Q \cap \del E \neq \emptyset} }^{ } \vol Q 
\]
\begin{rmq}
Si $Q \in \mathcal{E}$, alors $Q \cap \del E \neq \emptyset$
\end{rmq}
Ainsi





\end{proof}
\[ 
\overline{S}( \mathbb{I}_{\del E} , P_\epsilon)  - \underline { \mathbb{I}_E, P_\epsilon } = \sum_{Q \in \mathcal{E}}^{ }\vol Q \leq  \sum_{ \substack { Q \in P_\epsilon \\ Q \cap \del E \neq \emptyset } }^{ }\vol Q < \epsilon	
\]
\subsection{Caracterisation des fonctions integrables}
\begin{thm}
	Soit $R \subset \mathbb{R}^n$ un pave et $f : R \to \mathbb{R}$ bornee tel que l'ensemble des points de discontinuite de $f$ dans $R$ soit negligeable. Alors $f \in \mathcal{R}( R) $.
\end{thm}
\begin{proof}
	Soit $N$ l'ensemble des points de discontinuite de $f$ dans $R$ et $M = \sup_{x \in R} | f( x) |$.\\
	$N$ negligeable implique que $\forall \epsilon>0$, il existe une partition $P_\epsilon$ de $R$ telle que $ \mathbb{I}_N \in \mathcal{R}( R) $et $ \int_R \mathbb{I}_N = 0$.\\
	Donc $ \overline{S}(  \mathbb{I}_N, P_\epsilon) < \frac{\epsilon}{1 + 2M} $.\\
	Soit $ K = \bigcup_{\substack{ Q \in P_\epsilon\\Q \cap N = \emptyset }} Q  $, une union finie de paves compacts, donc $K$ est compact.\\
	De plus, $f$ est uniformement continue sur $K$.\\
	Donc $\forall \epsilon>0 \exists \delta_\epsilon: \forall x,y \in K , \N {  x-y} < \delta_\epsilon$, on a que $|f( x) -f( y) | < \frac{\epsilon}{1+ 2 \vol ( R) }$.\\
	On peut toujours raffiner la partition 
	\[ 
	P_K = \left\{  Q \in P_\epsilon, Q \subset K \right\} 
	\]
	de telle sorte que $\forall Q \in P_K, \forall x,y \in Q \N { x-y} < \delta_{\epsilon} $.\\
	Ainsi , soit $P_\epsilon' = \left\{  Q \in P_\epsilon, Q \cap N \neq \emptyset \right\} \cup P_K$
	\begin{align*}
		\overline{S}( f,P_\epsilon) - \underline { S} ( f,P_\epsilon) &= \sum_{Q \in P_K}^{ }( \sup f - \inf f  )\vol Q + \sum_{Q \in P_\epsilon, Q \cap N \neq \emptyset}^{ }(  \sup f - \inf f) \vol Q\\
			     &\leq \frac{\epsilon}{1 + 2 \vol R} \sum_{Q \in P_K}^{ } + 2M 	\sum_{Q \in P_\epsilon, Q \cap N \neq \emptyset} \vol Q < \epsilon
	\end{align*}
	
\end{proof}
\begin{crly}
Soit $E \subset \mathbb{R}^n$ borne et mesurable et $f: E \to \mathbb{R} $ continue sur $\ded E$ et bornee.\\
Alors $f \in \mathcal{R}( E) $.
\end{crly}
\begin{proof}
Soit $R \subset \mathbb{R}^n$ un pave contenant $E$, etudions 
\[ 
\tilde f : R \to \mathbb{R}^n
\]
le prolongement par $0$ de $f$ sur $R$.\\
L'ensemble $\tilde N$ des points de discontinuite de $\tilde f$ est surement contenu dans $\del E$ qui est negligeable, donc $\tilde f \in \mathcal{R}( R) \Rightarrow f \in \mathcal{R}( E) $.
\end{proof}
\begin{crly}
	Soit $E \subset \mathbb{R}^n$ est borne, mesurable et ferme ( compact ) et $f : E \to \mathbb{R}$ continue. Alors $f \in \mathcal{R}( E) $.
\end{crly}
\subsection{Proprietes de l'integrale de Riemann}
Soit $E \subset \mathbb{R}^n$ borne mesurable et $f : E \to \mathbb{R}$ integrable.
Alors
\[ 
	\inf_{x \in E} f( x) \vol E \leq \int_E f( x) dx \leq \sup_{x \in E}  f( x)\vol E 
\]
\begin{proof}
	$f( x) \leq \sup_{x\in E} f( x) \forall x \in E$, et la fonction constante $= \sup_{y\in E} f( y) $ est integrable, donc
	\[ 
		\int_E f( x) dx \leq  \int_E ( \sup_{y \in E  } f(y)) dy
	\]
\end{proof}
Si $ E \subset \mathbb{R}^n$ est borne, mesurable, compact et connexe par arcs et $f \in C^{0}( E) $, alors 
\[ 
	\exists x_0 \in E: \int_E f( x) dx = f( x_0) \vol E
\]

\begin{proof}
	Par le resultat precedent. on a que
	\[ 
		\min_ f \vol E \leq  \int_E f( x) dx \leq  \max_E f\vol E
	\]
	Puisque $f$ prend toutes les valeurs entre $\min_E f$ et $\max_E f $.\\
	Donc $\exists x_0\in E: f( x_0) \vol E = \int_E f$.
\end{proof}
Soit $E_1, E_2 \subset \mathbb{R}^n$ bornes tel que $E_1 \cap E_2$ est negligeable.\\
Soit $f: E_1 \cup E_2 \to \mathbb{R}$ borne.\\
Si $f\vert_{E_1} \in \mathcal{R}( E_1) $ et $f\vert_{E_2} \in \mathcal{R}( E_2) $, alors $f \in \mathcal{R}( E_1\cup E_2) $, alors
\[ 
	\int_{E_1\cup E_2} f( x) dx = \int_{ E_1 }^{  }f( x) dx + \int_{ E_2 }^{  }f( x) dx
\]
De plus, si $E_1, E_2, E$ sont mesurables, alors $f\vert_{E_1} \in \mathcal{R}( E_1) $, $f \vert _{E_2} \in \mathcal{R}( E_2) $ et on a a nouveau
\[ 
	\int_{E_1\cup E_2} f( x) dx = \int_{ E_1 }^{  }f( x) dx + \int_{ E_2 }^{  }f( x) dx
\]
\subsection{Formule des integrales iterees}
\begin{defn}[Domaine simple]
	Soit $E \subset \mathbb{R}^{n+1}$, $z = ( x,y) , x \in \mathbb{R}^n, y \in \mathbb{R}$.\\
	On dit que $E$ est un domaine simple par rapport a $y$ s'il existe un compact mesurable  $K \subset \mathbb{R}^n$ (  $K \in \mathcal{J}( \mathbb{R}^n) $) et deux fonctions $g,h \in K \to \mathbb{R}$ tellles que
	\[ 
		g( x) \leq h( x) \forall x \in K
	\]
	et $E$ a la forme 
\[ 
	E = \left\{ ( x,y) \in \mathbb{R}^{n+1}: g( x) \leq  y \leq  h( x) , x \in K \right\} 
\]

\end{defn}
\begin{thm}
	Soit $E \subset \mathbb{R}^{n+1}$ un domaine simple de la forme $E = \left\{ ( x,y) \in \mathbb{R}^{n+1}, g( x) \leq  y \leq  h( x)  \right\} , x \in K$, ou $K \in \mathcal{J}( \mathbb{R}^n) $ et $g,h \in C^{0}( K) $.\\
	Soit $f: E \to \mathbb{R}$ continue.\\
	Alors $f\in \mathcal{R}( E) $, et 
	\[ 
		\int_E f( x,y) dx dy = \int_K \left( \int_{ g( x)  }^{ h( x)  }f( x,y) dy \right) dx
	\]
	En particulier, $E$ est mesurable.
\end{thm}




\end{document}	
