\documentclass[../main.tex]{subfiles}
\begin{document}
\lecture{1}{Mon 22 Feb}{Introduction}
\section{Intégrales généralisées}
Peut-on définir une intégrale sur un intervalle ouvert plutot que sur un intervalle fermé? ie.
\[ 
f: [ a,b[ \to \mathbb{R} \text{ c.p.m. } 
\]
\begin{defn}[Intégrales généralisées ( sur un intervalle borné non fermé) ]
	Soit $f:[a,b[ \to \mathbb{R}$ continue par morceaux ( $a <b$).\\
	En particulier, $f$ est c.p.m.  sur tout intervalle $[a,x]$, $a<x<b$
	Soit $F( x)= \int_{ a }^{ x } f( t) dt $.\\
	On dit que l'integrale generalisee $ \int_{ a }^{ b }f( x) dx$ existe ( ou converge) si $ \lim_{x \to b} F( X) $ existe, dans ce cas, on note
	\[ 
		\int_{ a }^{ b }f( t)  dt = \lim_{x \to b} F( x) - F( a) 
	\]
	Si $\lim_{x \to b-} F( x) $ n'existe pas, alors on dit que 
	\[ 
		\int_{ a }^{ b }f( t) dt
	\]
	diverge.
	Definition analogue pour le cas $]a,b]$.
	
\end{defn}
On souhaite definir $ \int_{ -\frac{\pi}{2} }^{ \frac{\pi}{2} } tan( x) dx =0$.\\
Dans certains cas cette integrale vaut 0. Mais si on calcule
\[ 
	\lim_{\epsilon \to  0} \int_{ - \frac{\pi}{2} + \epsilon^{2} }{\frac{\pi}{2}- \epsilon } tan( t) dt = \lim_{\epsilon  \to 0+} ( - \ln ( \cos ( \frac{\pi}{2}-\epsilon) ) + \ln( \cos( - \frac{\pi}{2}+ \epsilon^{2}) ) ) = - \infty 
\]
Il faut donc une definition qui est coherente.
\begin{defn}[Integrale sur un intervalle borne ouvert]
	Soit $f:]a,b[\to \mathbb{R}$ c.p.m et $c \in ]a,b[$.\\
	Si les integrales generalisees $ \int_{ a }^{ c }f( t) dt$ et $ \int_{ c }^{ b }f( t) dt$ existent, alors on definit l'integrale 
	\[ 
		\int_{ a }^{ b }f( t) dt = \int_{ a }^{ c }f( t) dt + \int_{ c }^{ b }f( t) dt
	\]
	Si une des deux integrales diverge, alors le tout diverge.
\end{defn}



\end{document}
\begin{lemma}
	La valeur de  l'integrale $ \int_{ a }^{ b }f( t) dt$ ne depend pas de $c$, si elle converge.
\end{lemma}
\begin{proof}
	Soit $d \in ]a,b[$, different de $c$, alors on a
	\begin{align*}
	\int_{ a }^{ d }f( t) dt = \lim_{x \to a+}  \int_{ x }^{ d }f( t) dt = \lim_{x \to a+} \int_{ x }^{ c }f( t) dt + \int_{ c }^{ d }f( t) dt\\
	= \int_{ c }^{ d }f( t) dt + \lim_{x \to a+} \int_{ x }^{ c }f( t) dt
	\end{align*}
	Donc l'integrale existe.\\
	Si elle existe, on trouve
	\begin{align*}
		&\int_{ a }^{ b } f( t) dt = \int_{ a }^{ d }f( t) dt + \int_{ d }^{ b }f( t) dt\\
		&= \lim_{x \to a+} \int_{ x }^{ d }f( t) dt + \lim_{y \to b- } \int_{ d }^{ y }f( t) dt\\
		&= \int_{ c }^{ d }f( t) dt + \lim_{x \to a+} \int_{ x }^{ c }f( t) dt + \lim_{y \to b-} \int_{ d }^{ c }f( t) dt + \int_{ c }^{ y }f( t) dt\\
		&= \int_{ a }^{ b } f( t) dt
	\end{align*}
	
	
	
\end{proof}
\begin{rmq}
	Soit  $f:]a,b[\to \mathbb{R}$ continue.\\
	Si $f$ admet une extension par continuite sur $[a,b]$, alors on verifie facilement que 
	\[ 
		\int_{ a }^{ b } f( t) dt
	\]
	existe et coincide avec 
	\[ 
		\int_{ a }^{ b } \tilde f ( t) dt
	\]
	ou $\tilde f$ est l'extension par continuite de $f$ sur $[a,b]$.
\end{rmq}
\begin{lemma}[Critere de Comparaison]
Soit $f,g:[a,b[\to \mathbb{R}$ continues par morceaux et supposons qu'il existe $c\in [ a,b[ $ tel que 
\[ 
	0\leq f( x) \leq g( x)  \forall x \in [ c,b[ 
\]
et si $\int_{ c }^{ b }g( x) dx$ existe, alors $ \int_{ a }^{ b }f( x) dx$ existe aussi.
De meme si $ \int_{ c }^{ b }f( x) dx$ diverge, alors $ \int_{ a }^{ b }f( x) dx$

\end{lemma}

