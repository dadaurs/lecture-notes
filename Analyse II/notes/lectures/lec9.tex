\documentclass[../main.tex]{subfiles}
\begin{document}
\lecture{9}{Wed 24 Mar}{Derivees secondes}
	Cas scalaire:\\
	Soit $E \subset \mathbb{R}^n$ ouvert, $x,y \in E$ et $f : E \to \mathbb{R}$ derivable sur $E$.\\
\begin{figure}[H]
    \centering
    \incfig{derivee_taf}
\end{figure}
On denote par $[x,y]$ le segment ( ferme)  entre $x$ et $y$ et $]x,y[$ le segment ouvert entre $x$ et $y$.
\begin{thm}[Theoreme des accroissements finis dans $\mathbb{R}^n$]
	Soit $x,y\in E\subset \mathbb{R}^n$ et $f: E\to \mathbb{R}$, alors il existe $z \in [ x,y] $ tel que
	\[ 
		f( y) -f( x) = \nabla f( z) ^{T} ( y-x)  = Df( z) ( y-x) 
	\]
\end{thm}
\begin{proof}
	Soit $g( t) = f( x + t ( y-x) ) $ pour $t \in [ 0,1] $.\\
	On a alors
	\[ 
		g'( t) = \frac{d}{dt}g( t) = \frac{d}{dt} f( \phi( t) ) 
	\]
	ou $phi( t) = x + t( y-x) $.\\
	Puisque $f$ et $\phi$ sont derivables, on conclut que $g$ est aussi derivable.\\
	Donc
	\begin{align*}
		g'( t) = Df( \phi( t) ) \cdot D \phi( t)\\
		= \sum \frac{\del f}{\del x_i}( x + t( y-x) ) ( y_i -x_i)\\
		=\nabla f( x + t( y-x) )^{T}( y-x)  = Df( x + t( y-x) ) ( y-x) 
	\end{align*}
	Le taf applique a $g$ donne $\exists s \in  ] 0,1[ $	tel que
	\[ 
		g( 1) - g( 0) = g'( s) 
	\]
Donc
\[ 
	f( y) - f( x)  = Df ( x + s( y-x) ) ( y-x) 
\]
On conclut en posant $z= x+ s( y-x) $.

\end{proof}
Le cas vectoriel:
\begin{thm}[Taf dans le cas vectoriel]
	Soit $f: E \subset \mathbb{R}^n \to \mathbb{R}^m$ \\
	On essaie de representer $f( y) - f( x) $ a l'aide des derivees de $f$.\\
	On peut ecrire TAF pour chaque composante, mais les $z_k$ ne sont en general pas les memes.\\
	Cependant, on peut toujours ecrire pour $f \in C^{1}( E)  $
	\[ 
		f( y) - f( x) = \int_{ 0 }^{ 1 } Df( x+ s( y-x) )( y-x) ds
	\]
	
\end{thm}
\subsection{Derivees d'ordre superieur}
\begin{defn}[Derivees partielles secondes ( cas scalaire) ]
	Soit $f: E \to \mathbb{R}$, $E$ ouvert.\\
	Supposons que pour un indice $i = \left\{ 1, \ldots n \right\} $ fixe, la derivee partielle $\frac{\del f}{\del x_i}( x) $ existe $\forall x \in E$.\\
	Si $ \frac{\del f}{\del x_i}$ admet la derivee partielle selon $x_j$, alors on dit que $f$ a une derivee partielle seconde en $x$ et on note 
	\[ 
		\frac{\del ^{2}f}{\del x_j \del x_i} ( x) = \frac{\del }{\del x_j} ( \frac{\del f}{\del x_i}) ( x) 
	\]
	
\end{defn}
\begin{defn}[Matrice hessienne]
	Soit $f: \mathbb{R}^n \to \mathbb{R}$ tel que toutes les derivees partielles existent que toutes les derivees secondes existent
	
	\begin{align*}
		H_f ( y) =
	\begin{pmatrix}
		\frac{\del }{\del x_1}( \frac{\del f}{\del x_1})( y)  &\frac{\del }{\del x_1}( \frac{\del f}{\del x_2})( y)  & \ldots\\
		\frac{\del }{\del x_2}( \frac{\del f}{\del x_1})( y) &\frac{\del }{\del x_2}( \frac{\del f}{\del x_2})( y) & \ldots\\
		\frac{\del }{\del x_3}( \frac{\del f}{\del x_1})( y) &\frac{\del }{\del x_3}( \frac{\del f}{\del x_2}) ( y)& \ldots\\
	\end{pmatrix}
	\end{align*}
	
\end{defn}
\begin{defn}[Espace $C^2( E) $]
	On dit que $f: E \to \mathbb{R}$ est de classe $C^{2}$ si toutes les derivees partielles secondes sont continues.
\end{defn}
\begin{defn}[Derivees directionnelles secondes]
Soit $v \in \mathbb{R}^n, \N v = 1$. Alors, etant donne $D_v f: E \to \mathbb{R}$, on peut essayer de calculer la derivee directionnelle de $D_v f$ dans la direction $w \in \mathbb{R}^n.$\\
Si une telle derivee exise, on dit que $f$ admet une derivee directionnelle seconde dans les directions $v$ et $w$ au point $x$ et on note
\[ 
	D_{wv} f( x) = D_w ( D_v f) ( x) 
\]

\end{defn}

\begin{lemma}
	Soit $f\in C^2( E) $, $E$ ouvert et $v ,w \in \mathbb{R}^n$ tel que $\N v = \N w =1$.\\ Alors $D_{wv} f $ existe en tout $x \in E$ et
	\begin{align*}
		D_{wv} f( x)  &= w^{T} H_f( x) v\\
			      &= \sum_{i=1}^{ n}w_i ( \sum_{j=1}^{ n}H_f( x)_{ij} v_j)\\
			      &= \sum_{i,j=1}^{ n} \frac{\del ^{2} f}{\del x_i \del x_j} ( x)  w_i v_j
	\end{align*}
	
\end{lemma}
\begin{proof}
	Si $f \in C^{2}$ alors $f \in C^{1}$, alors $D_v f( x) = \nabla f ( x) ^{T} v = \sum \frac{\del f}{\del x_i}( x) v_i $.\\
	Mais puisque $f \in C^{2}$, $\frac{\del f}{\del x_i}\in C^{1}\forall i$, donc
	\begin{align*}
	D_w ( D_v f) ( x)  = \nabla ( D_v f) ^{T} w = \sum_{i=1}^{ n} \frac{\del }{\del x_i }( D_v f)  w_i\\
	= \sum_{i=1}^{ n} \sum_{j=1}^{ n} \frac{\del }{\del x_i} ( \frac{\del f}{\del x_j} ( x) ) v_j w_i
	\end{align*}
	Ce qui donne le resultat desire.
	
\end{proof}
\begin{thm}[Theoreme de Schwarz]
	Soit $f: E \to \mathbb{R}$. Pour $i,j \in \left\{ 1, \ldots, n \right\}  $ fixes.\\
Supposons que $\frac{\del f}{\del x_i}, \frac{\del f}{\del x_j}, \frac{\del ^{2} f}{\del x_i \del x_j}, \frac{\del ^{2} f}{\del x_j \del x_i}$ existent sur $E$ et sont continues en $x \in E$. Alors
\[ 
	\frac{\del ^{2}f}{\del x_i \del x_j }( x)  = \frac{\del ^{2} f}{\del x_j \del x_i}( x) 
\]
\end{thm}


	
\end{document}	
