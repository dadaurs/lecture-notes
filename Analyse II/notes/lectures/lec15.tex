\documentclass[../main.tex]{subfiles}
\begin{document}
\lecture{15}{Wed 21 Apr}{fonctions implicites-cas vectoriel}
\subsection{Cas Vectoriel}
Soit $f: E \subset \mathbb{R}^{n+m}\to \mathbb{R}^m$, avec 
\[ 
	f( z) = \begin{pmatrix}
		f_1( z) \\ \vdots \\ f_m( z) 
	\end{pmatrix}
\]
et soit $\Sigma = \left\{ z \in E: f( z) = 0 \right\} $, on peut reecrire ceci comme
\begin{align*}
	\Sigma &= \left\{ z \in E: f_i( z) =0 \right\} \\
	&= \bigcap_{i=1}^{m} \Sigma_i
\end{align*}
ou $\Sigma_i= \left\{ z \in E: f_i( z) =0 \right\}  $.\\
\begin{exemple}
Soit $f: E \subset \mathbb{R}^{3} \to \mathbb{R}^{2}$, defini par
\[ 
	f( z_1,z_2,z_3)  = \begin{pmatrix}
		f_1( z_1,z_2,z_3) \\ f_2( z_1,z_2,z_3) 
	\end{pmatrix}
\]
Alors 
\[ 
	\Sigma = \Sigma_1\cap \Sigma_2 = \left\{ ( z_1,z_2,z_3 ) \in Eh: f_1( z_1,z_2,z_3)=0  \right\} \cap \left\{ ( z_1,z_2,z_3 ) \in E: f_2( z_1,z_2,z_3)=0  \right\}
\]
\begin{figure}[H]
    \centering
    \incfig{surfaces-intersection}
    \caption{surfaces intersection}
    \label{fig:surfaces-intersection}
\end{figure}
Peut on representer $\Sigma$ comme le graphe d'une fonction de $n$ variables?\\
Pour $z= ( x,y) , x \in \mathbb{R}^n, y \in \mathbb{R}^m$, on veut ecrire 
\[ 
	y= \phi( x) : G( \phi) = \Sigma \cap V
\]

\end{exemple}
On etudie d'abord le cas d'une fonction affine, soit
\[ 
f_a: \mathbb{R}^{n+m} \to \mathbb{R}^m
\]
une fonction affine, on peut donc ecrire
\[ 
	f_a( z) = Az +b, \qquad A \in \mathbb{R}^{m \times n+m}, b \in \mathbb{R}^{m}
\]
\[ 
Az + b = \left[ A_1 | A_2 \right] \begin{pmatrix}
x \\ y
\end{pmatrix}
 =  A_1x + A_2y + b
\]

On considere maintenant 
\begin{align*}
	\Sigma &= \left\{ z \in \mathbb{R}^{n+m}: f_a( z) =0 \right\} \\
	       &= \left\{ ( x,y)  \in \mathbb{R}^{n}\times \mathbb{R}^{m}: A_1x + A_2 y + b =0 \right\} \\
	       &= \left\{ ( x,y)  \in \mathbb{R}^{n} \times \mathbb{R}^{m}: A_2y = - ( b+ A_1x)  \right\} 
	       \intertext{ Si $A_2$ est inversible, on peut ecrire $y$ comme fonction unique de $x$ }
	       &= \left\{ ( x,y)  \in \mathbb{R}^{n} \times \mathbb{R}^{m}: y = - ( A_2) ^{-1}( b + A_1x)  \right\} 
\end{align*}
Dans le cas general, pour $f : E \subset \mathbb{R}^{n+m} \to \mathbb{R}^{m}$ de classe $C1$.\\
\[ 
	\Sigma = \left\{ z \in E: f( z) =0 \right\} ,
\]
On ecrit
\begin{align*}
	f( z) &= \underbrace{f( z_0)  + Df( z_0) \cdot ( z-z_0)}_{ \coloneqq f_a( z) } + R_f( z) \\
	      &= f( z_0) + \left[ D_xf( z_0)  | D_yf( z_0)  \right] \begin{pmatrix}
	      x-x_0\\ y - y_0
	      \end{pmatrix}
	      + R_f( z) 
\end{align*}
La matrice $D_xf( z_0) $ est de taille $m \times n$ et $D_yf( z_0) $ est de taille $m\times m$, on peut donc ecrire
\[ 
	f( z) \approx f_a( z)  = f( z_0) + D_xf( z_0) ( x-x_0)  + D_yf( z_0) ( y-y_0) 
\]
En posant $f_a( z) =0$, on s'attend a ce que $f( z) =0$ definit une fonction implicite $y= \phi( x) $ si $\det( D_y) f( z_0) $	
\begin{thm}[Fonctions Implicites - Cas vectoriel]
	Soit $E \subset \mathbb{R}^{n+m}$ ouvert non vide, $f: E \to \mathbb{R}^{m}$ de classe $C^{1}$, $\Sigma= \left\{ z \in E : f( z) =0 \right\} , z_0 \in \Sigma$.\\
	Si $\det ( D_yf( z_0)  ) \neq 0$, alors il existe un ouvert $V \subset E$ contenant $z_0$, un ouvert $U \subset \mathbb{R}^n$, et une fonction $\phi: U \to \mathbb{R}^m$ de classe $C^{1}$ tel que
	\begin{itemize}
		\item $\phi( x_0) = y_0$ 
		\item $\forall x \in U, ( x,\phi( x) ) \subset V, f( x,\phi( x) ) =0  $
		\item $G( \phi) = \Sigma \cap V$ 
		\item $ \det ( D_y f( x, \phi( x) ) ) \neq 0 \forall x \in U$
			\[ 
				D \phi ( x) = - D_yf( x,\phi( x) ) ^{-1} D_x f( x,\phi( x) ) 
			\]
		
		\item Si $f$ est de classe $C^{k}$, alors $\phi$ est aussi de classe $C^{k}$.
	\end{itemize}
\end{thm}
\begin{proof}
On construit la fonction $F: E \to \mathbb{R}^{n+m}$, avec
\[ 
	F( x,y) = \begin{pmatrix}
		x\\f( x,y) 
	\end{pmatrix}
	=
	\begin{pmatrix}
	u\\w
	\end{pmatrix}
\]
On veut montrer que $f$ est un diffeomorphisme local autour de $z_0= ( x_0,y_0) $:
\begin{align*}
	DF( x_0,y_0) = D \begin{pmatrix}
		x\\ f( x,y) 
	\end{pmatrix}
	=
	\begin{pmatrix}
		I & 0 \\ D)xf( z_0)  & D_yf( z_0) 
	\end{pmatrix} 
\end{align*}
On a 
\[ 
	\det DF( z_0) = \det D_yf( z_0) \neq 0
\]
par hypothese.\\
Donc $F$ est un diffeomorphisme local.\\
Il existe donc $V' \subset E$ contenant $z_0= ( x_0,y_0) $ et un ouvert $U'\subset \mathbb{R}^{n+m}$ contenant $( x_0,0) $ tel que $F: V' \to U'$ est un diffeomorphisme
\begin{figure}[H]
    \centering
    \incfig{v'-vers-u'}
    \caption{V' vers U'}
    \label{fig:v'-vers-u'}
\end{figure}
Il existe $\delta, \tilde \delta >0$ tel que $\hat{U} = \left\{ ( x,y)  : x \in B( x_0,\delta) , y \in B( 0,\tilde \delta)  \right\} \subset U' $.\\
Soit $V= F^{-1}(  \hat{U}) $ et on considere la restriction $f: V \to \hat{U}$, d'inverse $G: \hat{U} \to V$
\[ 
	F( x,y) = \begin{pmatrix}
		x\\ f( x,y) 
	\end{pmatrix} 
=
\begin{pmatrix}
u\\ w
\end{pmatrix} 
\]
On peut donc reecrire ceci comme
\[ 
\begin{cases}
x=u\\
y= \psi( u,w) = \psi( x,w) 
\end{cases}
\]
L'existence de $\psi: \hat{U}\to \mathbb{R}^m$ est donnee par hypothese.
Donc la fonction implicitge cherchee est $\phi( x) \coloneqq \psi( x,0) $. $\phi$ est definie sur le voisinage de $x$ :
\[ 
	\phi: U = B( x_0,\delta) \to \mathbb{R}^m
\]
En effet, on veut verifier que $f( x,\phi( x) ) =0 \forall x \in U$, donc
\[ 
	( x,0) = F\circ G( x,0)  = F( G( x,0) ) = F( x,\psi( x,0) ) = ( x,f( x,\phi( x) ) ) 
\]
Et donc $f( x,\phi( x) )=0\forall x \in U $.\\
On verifie encore que $\Sigma \cap V \subset G( \phi) $.\\
En effet
\begin{align*}
	\forall ( x,y)  \in \Sigma \cap V \\
	( x,y) = ( G\circ F) ( x,y)  = G( F( x,yt) ) \\
	= G( x,f( x,y) ) = G(x,0) = ( x,\psi( x,0) ) = ( x,\phi( x) ) 
\end{align*}
Donc $( x,y) \in G( \phi) $.\\
$\phi$ est de classe $C^{1} $ par la composition de fonctions de classe $C^{1}$.\\
Puisque $F$ est un diffeomorphisme sur $V$, il s'ensuit que $\det DF( x,y) \neq 0 \forall ( x,y) \in V$, donc en particulier, on a que
\[ 
	\det D_y f( x,\phi( x) ) \neq 0 \forall x \in U
\]
On pose
\[ 
	\tilde f( x) = f( x,\phi( x) ) =0
\]

Etant donne que 
\begin{align*}
	f( x,\phi( x) ) &= 0 \forall x \in U, \text{ avec } f,\phi \in C^{1}\\
	0 &= Df( x,\phi( x) ) \\
	0 &= D_x f( x,\phi( x) ) + D_y f( x,\phi( x) ) D \phi( x) 
\end{align*}
On en deduit l'egalite



\end{proof}






\end{document}	
