\documentclass[../main.tex]{subfiles}
\begin{document}
\lecture{12}{Mon 12 Apr}{Fonctions Bijectives et diffeomorphismes}
\section{Fonctions Bijectives}
Soit $\vec{f}: E \subset \mathbb{R}^n \to F\subset\mathbb{R}^n$.\\
Si $\vec{f}$ est une bijection entre $E$ et $F$ alors $\forall \vec{y} \in F, \exists ! \vec{x} \in E: f( \vec{x}) = \vec{y}$\\
On peut donc definir une application inverse $\vec{g}: F \to E$ tel que
$\forall \vec{y} \in F, f( g( \vec{y}) )= \vec{y}$ et de maniere equivalente, $\forall x \in E, g( f( x) ) =x$
\subsection*{Pourquoi etudier les bijections?}
\begin{itemize}
	\item \textbf{Exemple 1}\\
		On souhaite resoudre le probleme $f( x) =y$ pour un $y \in F$ donne. Si $f$ est une bijection, on sait qu'il existe une solution.\\
	\item \textbf{ Changement de variable }\\
		Soit $f: E \to F$ bijective et $\phi: F \to \mathbb{R}$.\\
		On  peut reecrire $\phi$ en fonction de variables $x \in E$, $\tilde \phi = \phi \circ f$, donc $x \mapsto \tilde \phi ( x) = \phi( f( x) ), \forall x \in E $ .\\
		Vice versa etant donne $\tilde \phi : E \to \mathbb{R}, \vec{x} \mapsto \tilde \phi( x) $.\\
		On peut la reecrire en fonction de $y \in F$.\\
		On aura donc
		\[ 
			\phi( y) = \tilde \phi( g( y )) 
		\]
		On utilise ceci, en partie pour les coordonnees polaires.
\end{itemize}
\begin{defn}[Homeomorphisme]
	Soit $E, F\subset \mathbb{R}^n$ ouverts non-vides. On dit que $f: E \to F$ est un homeomorphisme si elle est bijective et $f$ et son inverse $g: F \to E$ sont continues.
\end{defn}
\begin{defn}[Diffeomorphisme]
	Soit $E, F\subset \mathbb{R}^n$ ouverts non-vides. On dit que $f: E \to F$ est un diffeomorphisme global si elle est bijective et $f$ et son inverse $g: F \to E$ sont $C^{1}$.\\
De maniere plus generale, on dit que $f$ est un $k-$diffeomorphisme si $f$ et son inverse sont de classe $C^{k}$.
\end{defn}
\begin{defn}[Diffeomorphisme local]
	Soit $E \subset \mathbb{R}^n$ ouvert non vide et $x_0 \in E$ et $f: E \to \mathbb{R}^n$ de classe $C^{1}$.\\
	On dit que $f$ est un diffeomorphisme local en $x_0$, si il existe un ouvert $U \subset E$ contenant $x_0$ et un ouvert $V\subset \mathbb{R}^n$  contenant $y_0 \in f( x_0) $ tel que $\vec{f}: U \to V$ est un diffeomorphisme.
\end{defn}
Clairement, si $f$ est un diffeomorphisme global, c'est en particulier un diffeomorphisme local en tout point $x\in E$, mais la reciproque n'est pas vraie en general.
On a toutefois le resultat suivant
\begin{thm}
	Soit $E,F \subset \mathbb{R}^n$ ouverts non vides et $f: E \to F$ bijective et un diffeomorphisme local en tout point $x \in E$. Alors, $f$ est un diffeomorphisme global.
\end{thm}
\subsubsection*{Question}
Sous quelles conditions, $\vec{f}$ est elle un diffeomorphisme local en $x \in E$.\\
Dans le cas $n=1 $, $f$ est un diffeomorphisme local en $x_0\in \mathbb{R}$ si et seulement si $f'( x_0) \neq 0$.\\
Soit $f: \mathbb{R}^n\to \mathbb{R}^n$ une fonction affine
\[ 
	\vec{f}( x) = Ax + b
\]
Quand est-ce que $f$ est inversible, ou, etant donne $y \in \mathbb{R}^n$,  $f( x) = y$ a une solution unique, si et seulement si $\det A \neq 0$\\

De maniere plus generale, vu que $f$ est $C^{1}$, on a 
\[ 
	f( x) = f( x_0)  + Df( x_0) ( x-x_0)  + R_f( \vec{x}) 
\]
Autour de $x_0$, on a donc
\[ 
	f( x) \approx f( x_0) + Df( x_0) ( x-x_0) 
\]
et donc $f$ est un diffeormorphisme local si et seulement si $\det ( Df( x_0) ) \neq 0$
\begin{thm}[Condition necessaire d'inversion locale ]
	Soit $f: E subset \mathbb{R}^n\to \mathbb{R}^n$, avec $E$ ouvert non vide, un diffeomorphisme local en $x_0$. Alors, $\det ( D f( x_0) ) \neq 0$.	
\end{thm}
\begin{proof}
	Par definition de diffeomorphisme local, il existe un ouvert $U \subset E$ contenant $x_0$ et un ouvert $V \subset \mathbb{R}^n $ contenant $y_0 = f( x_0) $ tel que $f: U \to V$ est une bijection et soit $g: V \to U$ la fonction inverse de classe $C^1$ par hypothese.\\
	Puisque $g( f( x) )=x \forall x \in U $, on a 
	\[ 
		D( g( f( x) ) ) = Dg( f( x) ) Df( x) = \id
	\]
	Et donc $Df( x_0) $ est inversible.\\
\end{proof}
\begin{thm}
Soit $K \subset \mathbb{R}^n$ ferme et $\phi: K \to \mathbb{R}^n$ telle que
\begin{itemize}
	\item $\phi( K) \subset K $
	\item Il existe $\rho \in ]0,1[$ tel que $\forall x,y \in K$ 
		\[ 
	\N{ \phi( x) - \phi( y) }  \leq \rho \N { x-y} 
		\]
		( dans ce cas, on dit que l'application est contractante) 
		
\end{itemize}
Alors $\phi$ possede un unique point fixe $\exists ! v \in K$ tel que $v= \phi( v) $


\end{thm}
\begin{defn}[Norme spectrale]
	On definit
	\begin{align*}
		\ns { A} = \sup_{x \in \mathbb{R}^n, \N{x}=1} \N { A x} 
	\end{align*}
	
\end{defn}
\begin{defn}[Norme de frobenius]
	On note
\begin{align*}
\N { A}_F = \sqrt{ \sum_{i=1}^{ n} \sum_{j=1}^{ n}A_{ij} ^{2}} 
\end{align*}

\end{defn}
\begin{lemma}
	Soit $a,b$ finis et $f: [ a,b] \to \mathbb{R}^n$ continue. Alors
	\begin{align*}
		\N {  \int_{ a }^{ b }f( t) dt} \leq \int_{ a }^{ b }\N { f( t) } dt
	\end{align*}
	
\end{lemma}

					
\begin{thm}[Condition suffisante d'inversion locale ]
	Soit $E \subset \mathbb{R}^n $ un ouvert non vide, $f: E \to \mathbb{R}^n$ de classe $C^1$ et $x_0\in E$. Si $\det Df( x_0) \neq 0$, alors $f$ est un diffeomorphisme local en $x_0$. De plus si $g: V \to U $ est un inverse local, avec $U \subset E$ un ouvert contenant $x_0$ et $V$ un ouvert contenant $y_0 = f( x_0) $, on a 
	\[ 
		D g( f( x) ) = Df( x) ^{-1} \forall x \in U	
	\]
	On va utiliser le theoreme du point fixe de Banach.
	
\end{thm}

\begin{proof}
On montre l'existence d'un inverse local.\\
Par hypothese $x \to Df( x) $ est continue. Donc
\[ 
	\exists r_1, \det ( Df( x_0) ) \neq 0 \forall x \in B( x_0,r) \cap E
\]
Considerons 
\[ 
	x \to \id - Df( x_0) ^{-1} Df( x) =: A( x) 
\]
On a a nouveau que $A( x) $ est continue et $A( x_0)=0 $.\\
Donc, il existe $r_2>0$ tel que $\forall x\in B( x_0,r_2)  \cap E \frac{-1}{2n} \leq A_{ij} ( x) \leq \frac{1}{2n}$
Donc $\forall x \in B( x_0,r_2) \cap E \ns { A( x) } \leq \N { A( x) } _F = \sqrt{ \sum_{ij} A_{ij} ( x) ^{2}}\leq  \sqrt{ \sum_{i,j}  \frac{1}{4 n^{2}}} = \frac{1}{2} $.\\
Donc il existe $r \leq \min \left\{ r_1,r_2 \right\} $ tel que
\begin{itemize}
	\item $B( x_0,r) \subset E $
	\item $\det Df( x) \neq 0\forall x \in B( x_0,r) $ 
	\item $\ns { A( x) } \leq \frac{1}{2}\forall x \in B( x0,r) $
\end{itemize}
On veut montrer que $f$ est localement inversible, donc $\forall y \in V \exists ! x \in U: f(x ) = y $.\\
On a
\begin{align*}
	f( x) = y &\iff 0 = y - f( x) \\
		  &\iff 0 = Df(x_0 ) ^{-1} ( y- f( x) ) \\
		  &\iff x= x + D f( x_0) ^{-1} ( y -f( x) ) 
\end{align*}
On a 
\[ 
	D \phi^{y}( x)  = D^{y}( x - Df( x_0) ^{-1}( f( x) -y) ) = A( x) 
\]

On montre donc que $\phi^{y}$ est contractante, donc
\begin{align*}
	\forall x_1,x_2 \in \overline{B}( x_0,r) 
\end{align*}
On veut calculer
\begin{align*}
	\N { \phi^{y}( x_1)  - \phi^{y}( x_2) } &= \N {  \int_{ 0 }^{ 1 } D\phi^{y}( x_1 + t( x_2-x_1) ) ( x_2-x_1) dt} \\
						&\leq \int_{ 0 }^{ 1 } \N { D \phi^{y}( \ldots) ( x_2-x_1) dt} \\
						& \leq \int_{ 0 }^{ 1 }\ns { D \phi^{y}( \ldots) } \N { x_2-x_1} dt\\
					&\leq \frac{1}{2}\N { x_2-x_1} 		
\end{align*}
Donc $\phi^{y}$ est contractante sur $B( x_0,r) $ pour tout $y \in \mathbb{R}^n$.\\
Il nous faut encore montrer que $ \phi^{y}( \overline{B}( x_0,r) ) \subset \overline{B}( x_0,r) $




				
\end{proof}




	


\end{document}
