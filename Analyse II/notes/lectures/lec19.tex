\documentclass[../main.tex]{subfiles}
\begin{document}
\lecture{19}{Wed 05 May}{integrales multiples}
	Soit $R \subset \mathbb{R}^n$ un pave  ( $R = [ a_1,b_1] \times [ a_2,b_2] \times \ldots \times [ a_n,b_n] $) et 
	\[ 
	f: R \to \mathbb{R}
	\]
	borne.\\
\begin{defn}[Somme de Darboux]
	Soit $P$ une partition du pave $R$, alors on definit les sommes inferieures $ \underline { S} ( f,P) = \sum_{Q \in P}  \inf_{x \in Q} f( x) Vol( Q)   $.\\
	On peut definir les sommes superieures $\overline{S} ( f,P) = \sum_{Q \in P} \sup_{x \in Q} f( x) Vol( Q) $	
\end{defn}
\begin{lemma}
Soit $P$ une partition de $R$ et $P''$ un raffinement de $P$, alors
\[ 
	\underline { S}  ( f,P)  \leq \underline{S} ( f,P'') \leq \overline{S}( f,P'') \leq  \overline{S}( f,P) 
\]
Pour toute partition $P,P'$ de $R$ on a toujours 
\[ 
	\underline{S} ( f,P)  \leq \overline{S} ( f,P') 
\]
\end{lemma}
\begin{proof}
Puisque $P''$ est un raffinement de $P$,
\[ 
	\forall Q \in P \quad Q = \bigcup_{\substack{ Q''\in P''\\Q'' \subset Q	 }} Q'' \quad Vol( Q) = \sum_{\substack{ Q'' \in P''\\ Q'' \subset Q }} Vol( Q'') 
\]
On a 
\begin{align*}
	\underline { S} ( f,P)  = \sum_{Q \in P}^{ }\inf_{x \in Q} f( x) Vol( Q) \\
	= \sum_{Q\in P} \inf_{x \in Q} f( x) \sum_{ \substack { Q '' \in P'' \\ Q'' \subset Q} }^{ } Vol( Q'') \\
	\leq \sum_{Q \in P} \sum_{ \substack { Q'' \in P''\\Q'' \subset Q} } \inf_{x \in Q''} f( x) Vol( Q'') \\
	= \sum_{Q'' \in P''} \inf_{x \in Q''} f( x) Vol( Q'') = \underline{S} ( f,P'') 	
\end{align*}
L'inegalite intermediaire est evidennte, et la troisieme se demontre de la meme maniere.\\
Il suffit de remarquer que tout couple $P,P'$ de partitions de $R$ admet un raffinement commun, nommons le $P''$, alors on a
\[ 
	\underline { S} ( f,P)  \leq  \underline { S} ( f,P)  \leq  \overline{S}( f,P'')  \leq  \overline{S}( f,P') 
\]
\end{proof}
\begin{defn}[Fonction integrable au sens de Riemann]
	Soit $R \subset \mathbb{R}^n$ un pave et $f : R \to \mathbb{R}$ bornee. On appelle integrale de Riemann inferieure $\underline{ \int }_R f( x) dx= \sup \left\{ \underline { S} ( f,P) :  P \text{ partition de  } R \right\} $.\\
	On definit de maniere analogue l'integrale de Riemannn superieure
\[ 
\overline{ \int }_R f( x) dx= \inf \left\{ \overline { S} ( f,P) :  P \text{ partition de  } R \right\} 
\]
Une fonction $f$ est dite integrable au sens de Riemann si 
\[ 
	\overline{ \int }_R f( x) dx = \underline{ \int }_R f( x) dx
\]
dans ce cas, on note 
\[ 
	\int_R f( x) dx 
\]
On note $ \mathcal{R}( R) $ l'ensemble des fonctions $f: R \to \mathbb{R}$ bornees et Riemann integrables..
\end{defn}
\begin{rmq}
Pour toute fonction $f: R \to \mathbb{R}$ bornee,
\[ 
	\underline { \int } _R f( x) dx \text{ et } \overline{\int }_R f( x)  dx
\]
sont bien definies.\\
Enn effet, $\forall P,P'$ des partitions de $R$, on a 
\[ 
	- \infty  <\underline { S} ( f,P) \leq  \underline {\int}_R f( x) dx \leq  \overline{\int}_R f( x) dx \leq \overline{S}( f,P')  < + \infty 
\]

\end{rmq}
\subsection{Caracterisation equivalente de fonctions integrables}
\begin{lemma}
Soit $R \subset \mathbb{R}^n$ et $f: R \to \mathbb{R}$ bornee.\\
$f$ est Riemann-inntegrable si et seulement si $\forall \epsilon >0$, il existe une partition $P_\epsilon$ 
\[ 
	\overline{S}( f,P_\epsilon)  - \underline { S} ( f,P_\epsilon) < \epsilon
\]

\end{lemma}
\begin{proof}
$f$ integrable implique qu'il existe une partition $P_\epsilon$ tel que \ldots\\
\[ 
	\int_R f( x) = \underline { \int } _R f( x) dx = \sup_{P \text{ partition de } R} \underline{S} ( f,P) 
\]
Alors
\[ 
	\exists P_\epsilon ' : \underline { S} ( f,P_\epsilon' ) \geq \underline { \int } _R f( x) dx - \frac{\epsilon}{2}= \int_R f( x) dx - \frac{\epsilon}{2}
\]
et 
\[ 
	\int_R f( x) dx = \overline { \int } _R f( x) dx = \inf_{ P \text{ partition de  } R} \overline { S} ( f,P) 
\]
et donc 
\[ 
	\exists P_\epsilon'' : \overline { S} ( f,P_\epsilon'') \leq  \int_R f( x) dx + \frac{\epsilon}{2}
\]
Soit maintenant $P_\epsilon$ un raffinement commun de $P_\epsilon'$ et $P_\epsilon''$, alors
\[ 
	\overline { S} ( f,P_\epsilon) \leq  \overline { S} ( f,P_\epsilon'') \leq \int_R f( x) dx + \frac{\epsilon}{2}		
\]
et
\[ 
	\underline { S} ( f,P_\epsilon) \geq \underline { S} ( f,P_\epsilon)  \geq  \int_R f( x) dx - \frac{\epsilon}{2}
\]
On montre la direction inverse\\
\begin{align*}
	\overline { \int } _R f( x)  dx \leq  \overline{S}( f,P_\epsilon) \\
	\underline { \int } _Rf( x) dx \geq  \underline { S} ( f,P_\epsilon) \\
	\overline { \int } _R f( x) dx - \underline { \int } _R f( x) dx \leq  \overline { S} ( f,P_\epsilon) - \underline { S} ( f,P_\epsilon) \leq  \epsilon
\end{align*}
Puise que $\epsilon$ est arbitraire, on a que $\overline { \int } _R f( x) dx = \underline { \int } _R f( x) dx$ donc $f$ est Riemann integrable.
\end{proof}
\begin{thm}
Soit $R \subset \mathbb{R}^n$ un pave et $f: R \to \mathbb{R}$ une fonction continue.\\
Alors $f$ est Riemann-integrable.
\end{thm}
\begin{proof}
$R$ est compact, donc $f$ est uniformement continue
\[ 
	\forall \epsilon>0, \exists \delta _\epsilon >0: \forall x,y \in R, \N { x-y} <\delta_\epsilon \Rightarrow |f( x) -f( y) | <\epsilon
\]
On peut toujours construire une partition $P$ de $R$ tel que 
\[ 
\forall Q \in P, \forall x,y \in Q: \N { x-y} < \delta_\epsilon
\]
On pose $P$ une partition tensorielle $\forall Q $ a des cotes de longueur $h$, alors
\[ 
	\forall x,y \in Q: \N { x-y} = \sqrt{ \sum_{}^{ }( x_i-y_i) ^{2}} \leq  h \sqrt{n} < \delta_\epsilon
\]
Il suffit donc de prendre $h < \frac{\delta_\epsilon}{ \sqrt{n} }$ 
\begin{align*}
\overline { S} ( f,P)  - \underline { S} ( f,P)  = \sum_{Q \in P}^{ }( \sup_{x \in Q} f( x) - \inf_{x \in Q} f( x)  ) Vol( Q) = \sum_{Q \in P} (  \max_{x \in Q} f( x) - \min_{x \in Q} f( x) ) Vol( Q) \\
= \sum_{Q \in P}  ( f( \overline{x}_Q ) - f( \underline { x} _Q) ) Vol( Q) \\
< \sum_{Q \in P} \epsilon Vol( Q) = \epsilon Vol( Q) 
\end{align*}
Donc $\forall \epsilon >0$, on a trouve une partition $P_\epsilon$ tel que
\[ 
	\overline { S} ( f,P_\epsilon) - \underline { S} ( f,P_\epsilon) <\epsilon
\]
Donc $f$ est Riemann-integrable.

		
\end{proof}
Soit $R \subset \mathbb{R}^n$ un pave et $\hat{R} \subset \mathbb{R}^n$ un autre apve tel que $R \subset \ded { \hat{R}} $.\\
Soit $f: R \to \mathbb{R}$ bornee et considerons le prolongement par $0$ de $f$ sur $\hat{R}$: $\hat{f}: \hat{R} \to \mathbb{R}$.\\
Alors, $f$ est Riemann integrable sur $R$ si et seulement si $\hat{f}$ est Riemann-integrable sur $\hat{R}$.
\subsection{Formule d'integrales iterees}
Soit $R \subset \mathbb{R}^{n+m}$, $\forall z \in  R: z = ( \underbrace{ z_1, \ldots, z_n}_{x \in \mathbb{R}^{n}}, \underbrace{z_{n+1} , \ldots, z_{n+m}}_{y \in \mathbb{R}^{m}	} ) $.\\
Notons  $R = \underbrace{[ a_1, b_1] \times \ldots \times [ a_n,b_n] }_{R^{( 1) }}\times \underbrace{ [ a_{n+1} ,b_{n+1} ] \times \ldots \times [ a_{n+m} , b_{n+m}  ]  }_{R^{( 2) }}$
\begin{thm}[de Fubini]
	Soit $f : R \to \mathbb{R}$ une fonction bornee et Riemann-integrable, 
	\[ 
		f( x,y) , x \in R^{( 1) }, y \in R^{( 2) }.	
	\]
	Si $\forall y \in R^{( 2) }$ la fonction $f( \cdot, y) : R^{( 1) }\to \mathbb{R}$ est Riemann-integrable, alors la fonction
	\[ 
		y \mapsto G( y) = \int_{R^{( 1) }} f( x,y) dx: R^{( 2) }\to \mathbb{R}
	\]
	est aussi Riemann-integrable sur $R^{( 2) }$ et 
	\[ 
		\int_{R} f( z) dz = \int_{R^{( 2) }} G( y) dy = \int_{R^{( 2) }} ( \int_{R^{( 1) }} f( x,y) dx) dy	
	\]
\end{thm}
\begin{crly}
Si $f: \mathbb{R}\to \mathbb{R}$ est continue, alors
\[ 
	\int_R f( z) dz = \int_{R^{( 1) }} ( \int_{R^{( 2) }} f( x,y) dy) dx = \int_{R^{( 2) }} ( \int_{R^{( 1) }} f( x,y) dx) dy
\]

\end{crly}



	


		
\end{document}	
