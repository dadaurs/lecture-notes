\documentclass[../main.tex]{subfiles}
\begin{document}
\lecture{13}{Wed 14 Apr}{theoreme d'inversion locale}
\begin{proof}
	On a montre l'existence d'une fonction inverse en trouvant un point fixe de la fonction
	\[ 
		\phi^{y}( x) = x - ( Df( x_0) ) ^{-1}( f( x) -y) 
	\]
	Il existe $r>0$ tel que
	\begin{itemize}
		\item $\overline{B}( x_0)\subset E$ 
		\item $\ns { D \phi ^{y}( x) }\leq \frac{1}{2}$ 	
		\item $\det ( Df( x) )\neq 0 \forall x \in \overline{B}( x_0,r) $
	\end{itemize}
	On a montre que pour tout point $y \in \mathbb{R}^n, \phi^{y}$ est contractante sur $\overline{B}( x_0,r) $ et que $\phi^{y}( \overline{B}( x_{0},r) ) \subset B( x_0,r) $ pour un $y \in B( y_0,\tilde r) $.\\
	On a donc l'existence d'un unique point $x \in B( x_0,r) : x = \phi^{y}( x) \iff f( x) =y$, ou encore
		\[ 
			\forall y \in B( y_0,r) =: V \exists ! x \in B( x_0,r) \cap f^{-1}( V) : f( x) =y
		\]
		Or $B(x_0,r ) \cap f^{-1}( V)=:U  $ est un ouvert, et donc $f:U\to V$ est inversible et on peut donc definir une fonction inverse $g$.\\
		Montrons maintenant que $g$ est continue en montrant qu'elle est Lipschitz sur $V$. 
		On veut montrer qu'il existe $L>0$ tel que $\forall y_1,y_2\in V$ 
		\[ 
			\N { g( y_1) - g( y_2) } \leq L \N { y_1-y_2} 
		\]
		En notant $\N { x_1-x_2} $ les preimages, on peut reecrire
		\begin{align*}
			\N { x_1-x_2} &= \N { \phi^{y_1}( x_1) - \phi^{y_2}( x_2) } \\
				      &\leq \N { \phi^{y_1}( x_1) - \phi^{y_1}( x_2) } + \N { \phi^{y_1}( x_2) - \phi^{y_2}( x_2) } \\
				      &\leq \frac{1}{2} \N { x_1-x_2} + \N { D f( x_0) ^{-1}(  y_2-y_1) } \\
				      &\leq \frac{1}{2}\N { x_1-x_2}  + \ns { Df( x_0)^{-1}} \N { y_2-y_1} 
		\end{align*}
		Et donc on a
		\[ 
			\N { x_1-x_2} \leq 2 \ns { Df( x_0) ^{-1}} \N { y_1-y_2} 
		\]
		On montre que $g$ est de classe $C^{1}$ ( en utilisant le fait que $f$ est de classe $C^{1}$).\\
		Soit $y,y_1 \in V $ et $x = g( y) , x_1= g( y_1) $.\\
		On veut montrer que $g$ est differentiable en $y$. On essaie d'ecrire un developpement limite de $g$ en $y$.\\
		On a 
		\[ 
			\underbrace{f( x_1)}_{y_1} = \underbrace{f( x)}_{y}  + Df( x) ( x_1-x)  + R_f( x_1) 
		\]
		\begin{align*}
			Df( x) ( x_1-x) &= y_1 - y - R_f( x_1) \\
			x_1-x	 &= Df( x) ^{-1}( y_1-y) - Df( x)^{-1} Rf( x_1)\\
			g( y_1) - g( y) = Df( x) ^{-1}( y_1-y) - \underbrace{Df( x) ^{-1}R_f( x_1)}_{R_g( y_1) 	} 
		\end{align*}
		On veut montrer que $\lim_{y_1 \to y} \frac{R_g( y_1) }{\N { y_1-y} }=0$	
		\begin{align*}
			\lim_{y_1 \to y} \frac{R_g( y_1) }{\N { y_1-y} }&\leq \lim_{y_1 \to y} \frac{\ns { Df( x) ^{-1}} \N { R_f( x_1) } }{\N { y_1-y} }\\
									&= \lim_{y_1 \to y} \ns { Df( x) ^{-1}}  \frac{\N { x_1-x} }{\N { y_1-y} } \frac{\N { R_f( x_1) } }{\N { x_1-x} }\\
									&= \lim_{y_1 \to y} \ns { Df( x) ^{-1}} 2 \ns { Df( x_0) ^{-1}}  \frac{\N { R_f( x_1) } }{\N { x_1-x} }	
		\end{align*}
		Donc $g$ est differentiable en $y \in V$ et 
		\[ 
			Dg( y) = Df( x) ^{-1} \text{ ou } x = g( y) 
		\]
		
		

	
\end{proof}
\subsection{Fonctions Implicites et Hypersurfaces de $\mathbb{R}^{n}$}
Considerons une fonction $\phi: U\subset \mathbb{R}^{2}\to \mathbb{R}$
\begin{figure}[H]
    \centering
    \incfig{hypersurfaces}
    \caption{hypersurfaces}
    \label{fig:hypersurfaces}
\end{figure}
En particulier si $\phi$ est differentiable en  $x_0\in U$, alors
\[ 
	\phi( x) = \underbrace{\phi( x_0) + D\phi( x_0 ) ( x-x_0) }_{T_\phi^{1}( x) \text{ fonction affine en  } x}+ R_{\phi} ( x) 
\]
Donc le graphe  $G( T_{\phi,x_0} ^{1}) = \left\{ z \in \mathbb{R}^{n+1}, z=( x,y) ,x \in \mathbb{R}^n, y \in T_{\phi,x_0} ^{1} \right\} $ se reecrit comme l'ensemble
\[ 
	\left\{ z\in \mathbb{R}^{n+1}: y-y_0 - \sum_{i=1}^{ n} \frac{\del \phi( x_0) }{\del x_i}( x-x_0) =0 \right\} 
\]
En definissant
\[ 
	v = ( - \frac{\del \phi}{\del x_1}( x_0) , \ldots,  1) , z= ( x_{1} , x_2 , \ldots, x_n, y) 
\]
On peut ecrire
\[ 
	G( T_{\phi,x_0} ^{1}) = \left\{ z \in \mathbb{R}^{n+1}: v\cdot ( z-z_0) = 0 \right\} 
\]
Ce graphe definit un hyperplan de $\mathbb{R}^{n+1}$ appele l'hyperplan tangent au graphe de $\phi$ en $z_0= ( x_0,y_0) $
On essaie de resoudre le probleme inverse, ie. decrire le plan d'une surface comme le plan d'une fonction.\\
\begin{defn}[Hypersurfaces de classe $C^k$]
On dit que $\Sigma\subset \mathbb{R}^{n+1}$ est une hypersurface de classe $C^{k}$ autour de $z_0$ si elle est le graphe d'une fonction  de classe $C^{k}$ autour de $z_0$, cad, qu'il existe un ouvert $V\subset \mathbb{R}^{n+1}$ contenant $z_0$, un indice $i \in \left\{ 1, \ldots, n+1 \right\} $, un ouvert $U \in \mathbb{R}^{n}$  et une fonction $\phi: U \to \mathbb{R}$ tel que
\[ 
	\Sigma \cap V = \left\{ x \in \mathbb{R}^{n+1}: x_i = \phi( x_1, \ldots, x_{i-1} , x_{i+1} , \ldots, x_{n+1} )  \right\} 
\]

\end{defn}
En particulier, on considere des surfaces definies par
\[ 
	\Sigma = \left\{ x \in \mathbb{R}^{n+1}: f( x) =0 \right\} 
\]
On se demande quand est-ce que $\Sigma$ est une hypersurface ( au moins localement autour d'un point $z_0$).\\
Si $\Sigma$ est une hypersurface autour d'un point $z_0\in \Sigma$, il existe $V \subset \mathbb{R}^{n+1}$ contenant $z_0$ et $\phi: U \to \mathbb{R} $ tel que
\[ 
	\Sigma \cap V = \left\{  x : x_i = \phi( x_{ni} )  \right\} 
\]
Alors on dit que la fonction $\phi$ est definie implicitement par la relation $f( x) =0$.


\end{document}	
