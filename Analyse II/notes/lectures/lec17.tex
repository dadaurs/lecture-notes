\documentclass[../main.tex]{subfiles}
\begin{document}
\lecture{17}{Wed 28 Apr}{Extremas lies}
\subsection{Extremas lies}
\begin{exemple}
Parmi tous les cylindres d'un volume donne $ \overline{V}$, on cherche celui qui a une surface minimale.
On a $S (R,H) = 2 \pi R H + 2 \pi R^{2} $ et $V( R,H) = \pi R^{2}H$, on cherche
\[ 
	\min_{R>0,H>0} S( R,H) 
\]
sous la contrainte $V( R,H) = \overline{V}$
\end{exemple}
\subsection*{Formulation du probleme}
Soit $E \subset \mathbb{R}^n$, $E$ ouvert non vide, $f: E \to \mathbb{R}$ de classe $C^{1}$.\\
$g:E \to \mathbb{R}$ de classe $C^{1}$, on cherche
\[ 
	\min_{x \in E} f( x)  \text{ sous la contrainte } g( x) =0
\]
De facon equivalent, si on pose $\Sigma_g = \left\{ x \in E : g( x) =0 \right\} $,
\[ 
	\min_{x \in \Sigma_g} f( x) 
\]
\begin{exemple}
On cherche
\[ 
	\min_{( x,y) \in \mathbb{R}^{2}} x^{2}+y^{2}
\]
tel que $x+y-1=0$.\\
Les courbes de niveau de $f$ sont $f( x,y) =c \iff x^{2}+ y^{2}=c$, sous la contrainte $g( x,y) =0 \iff y=1-x$.\\
Si $( x^{*},y^{*} )$ est le point de minimum sous-contrainte, alors on a
\[ 
	\nabla f ( x^{*},y^{*}) \parallel \nabla g( x^{*},y^{*}) 
\]
Donc $\nabla g ( x^{*},y^{*}) $ est un vecteur orthogonal a $\Sigma_g$ en $( x^{*},y^{*} )$ ( vecteur orthogonal a l'hyperplan tangent a $\Sigma_g$ en $( x^{*},y^{*}) $) .\\
$\nabla f( x^{*},y^{*}) $ est un vecteur normal a la courbe de niveau $f( x,y) = f( x^{*},y^{*}) $.\\
Cette affirmation est equivalent a 
\[ 
	\exists \lambda \in  \mathbb{R}: \nabla f( x^{*},y^{*}) = \lambda \nabla g( x^{*},y^{*}) 
\]


\end{exemple}
\begin{thm}[Condition necessaire pour extremas lies]
	Soit $E \subset \mathbb{R}^n$ ouvert non vide ( $n \geq 2$), $f,g : E \to \mathbb{R}$ de classe $C^{1}$, $z^{*} \in \Sigma_g$ un point d'extremum local lie. Si $\nabla g( z^{*}) \neq 0$, alors $\exists \lambda \in \mathbb{R}$ tel que 
	\[ 
		\nabla f( z^{*}) = \lambda \nabla g( z^{*}) .
	\]
\end{thm}
\begin{proof}
	Puisque le gradient $\nabla g( z^{*}) \neq 0$, on peut utiliser le theoreme des fonctions implicites.
	\[ 
		\nabla g ( z^{*}) \neq 0 \Rightarrow \exists i \in  [ n] : \frac{\del g }{\del z_i}( z^{*}) \neq 0
	\]
	Supposons $i=n $, on note $y=z_n,$ $x= ( z_1,\ldots,z_{n-1} ) $.\\
	Grace au theoreme des fonctions implicites, on sait qu'il existe un ouvert $V \subset E$ contenant $z^{*}$, un $\delta >0$ et $\phi: B( x^{*},\delta) \to \mathbb{R}$ tel que
	\begin{itemize}
		\item $\phi( x^{*}) =y^{*}$ 
		\item $\forall x \in B( x^{*},\delta) , ( x,\phi( x) ) \in V$, et $g( x,\phi( x) ) =0$.
		\item $G( \phi) = \Sigma_g \cap V$
	\end{itemize}
On a 
\[ 
	g( x,y) =0 \iff y = \phi( x) \forall x \in B( x^{*},\delta) 
\]
On definit $\tilde f( x) = f( x,\phi( x) ) $.\\
Si $z^{*}$ est un point d'extremum local lie de $f$ sur $\Sigma_g$, alors $x^{*}$ est un point d'extremum libre libre de $\tilde f$ sur $B( x^{*},\delta) $.\\
Donc, $\nabla_x \tilde f( x^{*}) =0$, donc
\begin{align*}
	\iff 0= \frac{\del \tilde f}{\del x_i}( x^{*}) = \frac{\del f}{\del x_i}( x^{*},y^{*}) + \frac{\del f}{\del y}( x^{*},y^{*}) \frac{\del \phi}{\del x_i  }( x^{*}) 
\end{align*}
Mais 
\[ 
	\frac{\del \phi}{\del x_i} ( x^{*}) = - \frac{ \frac{\del g}{\del x_i}( x^{*},y^{*}) }{\frac{\del g}{\del y}*x^{*},y^{*}}
\]
Donc
\[ 
	\frac{\del f}{\del x_i}( z^{*}) - \frac{\del f}{\del g}( z^{*})  \frac{ \frac{\del g}{\del x_i }}{\frac{\del g}{\del y }( z^{*}) }=0
\]

On pose
\[ 
	\lambda = \frac{ \del_y f}{\del_y g}( z^{*})  \Rightarrow \frac{\del f}{\del x_i}( z^{*}) = \lambda \frac{\del g}{\del x_i}( z^{*}) 
\]
Donc toutes les composantes sont proportionnelles et on a
\[ 
	\nabla f ( z^{*}) = \lambda \nabla g( z^{*}) 
\]
\begin{rmq}
	Si $z^{*}$ est un extremum local lie de $f$ sur $\Sigma_g$ et $\nabla g ( z^{*}) \neq 0$, alors $\exists \lambda \in \mathbb{R}$ 
	\[ 
		\begin{cases}
		\nabla f( z^{*}) =\lambda \nabla g( z^{*}) \\
g( z^{*}) =0
		\end{cases}
	\]

	Ensemble, ceci forme un systeme de $n+1$ equations.\\
	$\lambda$ est appele multiplicateur de Lagrange
\end{rmq}
On definit la fonction de Lagrange ( ou le lagrangien) 
\begin{align*}
	\mathcal{L} : &E \times \mathbb{R} &\to \mathbb{R}
		      & \mathcal{L}( z,\lambda) = f( z) - \lambda g( z) 
\end{align*}
On a que 
\[ 
	\nabla_{( z,\lambda) } \mathcal{L}( z,\lambda) = 
	\begin{pmatrix}
	\nabla_z \mathcal{L}\\
	\ldots\\
	\frac{\del \mathcal{L}}{\del \lambda}
	\end{pmatrix} 
\]
Donc, si $z^{*}$ est un extremum local de $f$ sur $\Sigma_g$, alors il existe $\lambda^{*}\in \mathbb{R}: \nabla \mathcal{L}( z^{*},\lambda^{*}) $.




\end{proof}





	
\end{document}	
