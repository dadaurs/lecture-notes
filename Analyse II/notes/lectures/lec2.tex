\documentclass[../main.tex]{subfiles}
\begin{document}
\lecture{2}{Wed 24 Feb}{Integrales Generalisees}
\begin{thm}[Critere de Comparaison]\label{thm:Critere de Comparaisoncritere_de_comparaison}
	Soit $f,g: [ a,b[ \to \mathbb{R} $ c.p.m. et supposons $\exists c \in [ a,b[ $ tel que
	\[ 
		0 \leq f( x) \leq g( x) \forall x \in [ c,b[ 
	\]
	Si $ \int_{ a }^{ b }g( x) dx$ existe alors $ \int_{ a }^{ b }f( x) dx$ existe aussi\\
	Si $ \int_{ a }^{ b }f( x) dx$ diverge alors $\int_{ a }^{ b }g( x) dx$ diverge aussi.
\end{thm}
\begin{proof}
	Si $\int_{ a }^{ b }g( x) dx$ existe, alors $\int_{ c }^{ b }g( x) dx$ existe.\\
	Donc
	\begin{align*}
		\int_{ a }^{ b }f( x) dx &= \lim_{x \to b-} \int_{ a }^{ x }f( t) dt\\
		&= \lim_{x \to b-} ( \int_{ a }^{ c }f( t) dt + \int_{ c }^{ x }f( t) dt)\\
		&= \int_{ a }^{ c }f( t) dt + \lim_{x \to b-} \int_{ c }^{ x }f( t) dt\\
		&\leq \int_{ a }^{ c }f( t) dt	+ \lim_{x \to b-} \int_{ c }^{ x }g( t) dt < + \infty 
	\end{align*}
	En notant $F( x) = \int_{ a }^{ x }f( t) dt$ , $F$ est non decroissante, et bornee superieurement sur l'intervalle $[a,b[ \Rightarrow $ $\lim_{x \to b-} F( x) $ existe.	
\end{proof}
\begin{exemple}
	$f( x) = \abs{\sin( \frac{1}{x})}$ sur $]0,1]$, on a
	\[ 
		0 \leq f( x) \leq 1
	\]
	1 est integrable, et donc l'integrale de $f( x) $ existe.
\end{exemple}
\subsection{Integrales absoluments convergentes}
\begin{defn}[Integrale absolument convergente]\label{defn:Integrale absolument convergenteintegrale_absolument_convergente}
	Soit $I$ un intervalle du type $[a,b[$,$]a,b]$ ou $]a,b[$ et $f:I \to \mathbb{R}$ c.p.m.\\
	On dit que l'integrale generalisee de $f$ sur $I$ est absolument convergente si
	\[ 
		\int_I |f( x) | dx
	\]
	existe.
\end{defn}
\begin{thm}[absolument convergente implique convergente]\label{thm:absolument convergente implique convergenteabsolument_convergente_implique_convergente}
	Si l'integrale $\int_{ a }^{ b }f( x) dx$ converge absolument, alors il converge.
\end{thm}
\begin{proof}
	Notons $f_+( x) = \max \left\{ f( x) , 0 \right\} $ et $f_-( x) = - \min \left\{ f( x) ,0 \right\} $ et on a
	$|f( x)| = f_+( x) + f_- $.\\
	Donc
	\[ 
		0 \leq f_+( x) \leq |f( x) | \text{ et  } 0 \leq f_-( x)  \leq |f( x) | \forall x \in I	
	\]
	Par critere de comparaison, si
	\[ 
		\int_{ a }^{ b }|f( x) | dx \text{ existe  } \Rightarrow \text{ alors } \int_{ a }^{ b }f_+( x)  dx , \int_{ a }^{ b }f_-( x) \text{ existent } 
	\]
	
	et donc $\int_{ a }^{ b }f( x) dx$	
\end{proof}
	
\begin{rmq}
Soit $f:I \to \mathbb{R}$ c.p.m Si $f$ est bornee sur $I$, alors  
\[ 
	\int_I f( x) dx
\]
existe.
\end{rmq}
\begin{thm}[Critere de comparaison ( II) ]\label{thm:Critere de comparaison ( II) critere_de_comparaison_ii_}
	Soit $f: [ a,b[ \to \mathbb{R} $ c.p.m.\\
	S'il existe $\alpha \in ]- \infty ,1[$ tel que
	\[ 
		\lim_{x \to b-} f( x) ( b-x) ^{\alpha}= l \in \mathbb{R}
	\]
	Alors
	\[ 
		\int_{ a }^{ b }f( x) dx 
	\]
	existe.\\
	S'il existe $\alpha \geq 1$ tel que 
	\[ 
		\lim_{x \to b-} f( x) ( b-x) ^{\alpha}= l \neq 0 
	\]
	alors
	\[ 
		\int_{ a }^{ b }f( x) dx
	\]
	diverge.
	
\end{thm}
\begin{proof}
Par definition de la limite $\forall \epsilon >0, \exists b-a>\delta_\epsilon>0$ tel que
\[ 
	|f( x) ( b-x) ^{\alpha}-l| < \epsilon \forall x
\]
\[ 
	\Rightarrow l-\epsilon \leq f( x) ( b-x) ^{\alpha}\leq l+\epsilon
\]
et donc
\[ 
	0\leq |f( x) | \leq \frac{|l|+\epsilon}{( b-x)^{\alpha}}
\]
Puisque le terme de droite est integrable, on conclut par le critere de comparaison.
Pour la deuxieme partie, soit $\alpha\geq 1$ et $l\neq 0$.\\
Supposons $l>0$, on a
\[ 
	l-\epsilon \leq f( x) ( b-x) ^{\alpha}		
\]
Le meme raisonnement que ci-dessus donne que l'integrale de $f$ diverge.
\end{proof}
\subsection{Integrale generalisee sur un intervalle non borne}
\begin{defn}[Integrale sur un intervalle non borne]\label{defn:Integrale sur un intervalle non borneintegrale_sur_un_intervalle_non_borne}
	Soit $f: [ a, + \infty [  \to \mathbb{R} $ c.p.m.\\
	On dit que $ \int_{ a }^{ + \infty  } f( x) dx$ existe si
	\[ 
		\lim_{x \to  + \infty} \int_{ a }^{ x }f( x) dx
	\]
	existe et dans ce cas, on note
	\[ 
		\int_{ a }^{ + \infty  }f( x) dx = \lim_{x \to  + \infty} \int_{ a }^{ x }f( t) dt
	\]
	idem si $f: ]- \infty , a[ \to \mathbb{R}$.\
	Soit $f:]a,+ \infty [ \to \mathbb{R} $ c.p.m. on dit que $\int_{ a }^{ \infty  }f( x) dx$ existe s'il existe $c\in ]a, \infty [ $ tel que 
	\[ 
		\lim_{x \to a+}  \int_{ x }^{ c }f( t) dt \text{ et   } \lim_{y \to  + \infty} \int_{ c }^{ y }f( t) dt
	\]
	existent.
\end{defn}






\end{document}	
