\documentclass[11pt, a4paper, twoside]{article}
\usepackage[utf8]{inputenc}
\usepackage[T1]{fontenc}
\usepackage[francais]{babel}
\usepackage{lmodern}
\newcommand\ns[1]{\left\vert\left\vert\left\vert#1\right\vert\right\vert\right\vert}
\newcommand\Norm[1]{\left\lVert#1\right\rVert}
\newcommand\N[1]{\left\lVert#1\right\rVert}

\usepackage{amsmath}
\usepackage{amssymb}
\usepackage{amsthm}
\begin{document}
\title{Série 7}
\author{David Wiedemann}
\maketitle
\section*{1}
Pour montrer que $f$ est un difféomorphisme global, il faut montrer que $f^{-1} \in C^{1}$ ( l'existence de $f^{-1}$ est donnée par hypothèse).\\
Par hypothèse, en tout point $x\in E$, $f\in C^{1}$ et $f^{-1}$ est $C^{1}$ dans un voisinage de $f( x) $ et donc en particulier au point $f( x) $.\\
Ainsi, $\forall y \in F$, $f^{-1}$ est continument différentiable en $y$ et donc $f^{-1}\in C^{1}$.\\
Ainsi, $f$ est une bijection de $E\to F$ qui est $C^{1}$ et l'inverse de $f$ est également $C^{1}$, donc $f$ est un difféomorphisme.

\section*{2}
Montrons d'abord que $f_{\epsilon} $ est bijective.\\
Soit $y \in \mathbb{R}^{n}$, montrons que $y$ possède un unique antécédent par $f_\epsilon$.\\
Soit 
\begin{align*}
	g_y\colon &\mathbb{R}^{n} \mapsto \mathbb{R}^n\\
	     &x \mapsto y - \epsilon \cdot h( x) 
\end{align*}
Nous allons montrer que $g_y$ possède un unique point fixe en montrant que $g_y$ satisfait les hypothèses du théorème du point fixe de Banach.\\
Etant donné que $ \mathbb{R}^n$ est fermé, il est clair que $g_y ( \mathbb{R}^n) \subset \mathbb{R}^n$, il suffit donc de montrer que $g_y$ est contractante.\\
Soit $x_1, x_2\in \mathbb{R}^n$, on a 
\begin{align*}
	\N { g_y( x_2) - g( x_1) } &= \N { \epsilon h( x_1) - \epsilon h( x_2) } \\
				   &= \N { \epsilon\left(  Dh( z) ( x_1-x_2) \right)} \\
				   &\leq \ns { \epsilon Dh( z) }  \N { x_1-x_2} \\
\end{align*}
Où $z$ est donné par le théorème des accroissements finis.\\
Et car $\epsilon < M^{-1}$, on a $\epsilon M <1$ et donc $\ns{\epsilon Dh( z)} <1$ pour tout $z \in \mathbb{R}^n$.\\
Ainsi $g_y$ est contractante et possède un point fixe, ie. il existe $x$ tel que
\[ 
	x = y - \epsilon \cdot h( x) 
\]
Ou encore
\[ 
	y= x + \epsilon \cdot h( x) 
\]
Et donc $x$ est l'inverse unique de $y$ par $f_\epsilon$.\\
Montrons maintenant que $f_\epsilon$ est un difféomorphisme local en tout point $x_0\in \mathbb{R}^n$.\\
Pour montrer ceci, on va montrer que la matrice $Df_\epsilon( x_0) $ est inversible.\\
Supposons par l'absurd que $Df_\epsilon( x_0) $ n'est pas inversible, alors il existe $v,w \in \mathbb{R}^n$ deux vecteurs linéairement indépendants satisfaisant
\begin{align*}
	Df_\epsilon( x_0) \cdot v &= Df_\epsilon( x_0) \cdot w\\
	v + \epsilon Dh( x_0) \cdot v &= w + \epsilon Dh( x_0) \cdot w\\
	v-w &= - \epsilon Dh( x_0)\cdot ( v-w) 
	\N { v-w} &= \N { \epsilon Dh( x_0) ( v-w) } 
\end{align*}
Or
\[ 
\N { \epsilon Dh( x_0) ( v-w) } \leq \ns { \epsilon Dh( x_0) } \N { v-w} < \N { v-w} 
\]
Et donc
\[ 
\N { v-w} < \N { v-w} 
\]
Ce qui est une contradiction.\\
Ainsi, $Df_\epsilon( x_0) $ est inversible et donc la jacobienne  de $f_\epsilon$ est inversible en tout point $x\in \mathbb{R}^n$, et donc $f_\epsilon$ est un difféomorphisme local en tout point de $ \mathbb{R}^n$.\\
On conclut par la partie 1 et donc $f_\epsilon$ est un difféomorphisme global.






\end{document}
