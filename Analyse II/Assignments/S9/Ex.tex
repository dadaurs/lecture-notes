\documentclass[11pt, a4paper]{article}
\usepackage[utf8]{inputenc}
\usepackage[T1]{fontenc}
\usepackage[francais]{babel}
\usepackage{lmodern}
\usepackage{amsmath}
\usepackage{amssymb}
\usepackage{amsthm}
\renewcommand{\vec}[1]{\overrightarrow{#1}}
\newcommand{\del}{\partial}
\DeclareMathOperator*{\sgn}{sgn}
\DeclareMathOperator*{\id}{Id}
\DeclareMathOperator*{\im}{Im}
\DeclareMathOperator*{\re}{Re}
\DeclareMathOperator*{\vol}{Vol}
\newcommand\norm[1]{\left\vert#1\right\vert}
\newcommand\ns[1]{\left\vert\left\vert\left\vert#1\right\vert\right\vert\right\vert}
\newcommand\Norm[1]{\left\lVert#1\right\rVert}
\newcommand\N[1]{\left\lVert#1\right\rVert}
\newcommand\abs[1]{\left\vert#1\right\vert}
\newcommand\inj{\hookrightarrow}
\newcommand\surj{\twoheadrightarrow}
\newcommand\ded[1]{\overset{\circ}{#1}}
\newcommand\sidenote[1]{\footnote{#1}}
\newcommand\eng[1]{\left\langle#1\right\rangle}
\newcommand\hr{
    \noindent\rule[0.5ex]{\linewidth}{0.5pt}
}

\newcommand{\incfig}[1]{%
    \def\svgwidth{\columnwidth}
    \import{./figures}{#1.pdf_tex}
}
\newcommand{\filler}[1][10]%
{   \foreach \x in {1,...,#1}
    {   test 
    }
}

\newcommand\contra{\scalebox{1.5}{$\lightning$}}
\makeatother
\def\@lecture{}%
\newcommand{\lecture}[3]{
    \ifthenelse{\isempty{#3}}{%
        \def\@lecture{Lecture #1}%
    }{%
        \def\@lecture{Lecture #1: #3}%
    }%
    \subsection*{\@lecture}
    \marginpar{\small\textsf{\mbox{#2}}}
}

\begin{document}
\title{Série 9}
\author{David Wiedemann}
\maketitle
On va d'abord montrer l'existence du minimum, on montrera ensuite qu'il est unique.\\
Soit $x \in E$, par hypothèse, il existe un $M \in [ 0, \infty [  $, tel que $\forall y \in E: \N { y} \geq M \Rightarrow f( y) \geq f( x) $.\\
Donc en particulier, pour $\N { y} > M$, on a $f( y) \geq f( x) $.\\
Considérons donc l'ensemble des points $ \overline{B}(0,M ) \cap E $.\\
Cet ensemble est trivialement  compact, car l'intersection de deux ensembles fermés est fermée et car tous les éléments de l'ensemble ont une norme finie.\\
On sait qu'une fonction continue atteint son minimum sur un ensemble compact, donc il existe $x_m \in \overline{B}(0,M ) \cap E $ tel que, $f( x_m) \leq  f( x) \forall x \in x_m \in \overline{B}(0,M ) \cap E$.\\
Or, par définition, $\forall y \in E \setminus \overline{B}( 0,M) $, on a $ f( y) \geq f( x) \forall x \in  \overline{B}(0,M ) \cap E$, et donc $x_m$ est un minimum global.\\
Montrons maintenant l'unicité du minimum.\\
Par l'absurde, supposons qu'il y ait deux minimums globaux $a$ et $b \in E$, alors, par convexité de l'ensemble $E$, le segment $[a,b]= \left\{ \lambda a + ( 1-\lambda) b | \lambda \in [ 0,1]  \right\} $	est contenu dans $E$, et donc en particulier le point $\frac{1}{2}a + \frac{1}{2}b$.\\
Par stricte convexité, et car $f( a) =f( b) $, on a 
\[ 
	\frac{1}{2}f( a) + \frac{1}{2}f( b) = f( a) > f( \frac{1}{2}a + \frac{1}{2}b) 
\]
Ce qui contredit la minimalité du point $a$ et du point $b$.\\
Ainsi, le minimum est unique.


\end{document}
