\documentclass[11pt, a4paper]{article}
\usepackage[utf8]{inputenc}
\usepackage[T1]{fontenc}
\usepackage[francais]{babel}
\usepackage{lmodern}
\usepackage{amsmath}
\usepackage{amssymb}
\usepackage{amsthm}
\renewcommand{\vec}[1]{\overrightarrow{#1}}
\newcommand{\del}{\partial}
\DeclareMathOperator*{\sgn}{sgn}
\DeclareMathOperator*{\id}{Id}
\DeclareMathOperator*{\im}{Im}
\DeclareMathOperator*{\re}{Re}
\DeclareMathOperator*{\vol}{Vol}
\newcommand\norm[1]{\left\vert#1\right\vert}
\newcommand\ns[1]{\left\vert\left\vert\left\vert#1\right\vert\right\vert\right\vert}
\newcommand\Norm[1]{\left\lVert#1\right\rVert}
\newcommand\N[1]{\left\lVert#1\right\rVert}
\newcommand\abs[1]{\left\vert#1\right\vert}
\newcommand\inj{\hookrightarrow}
\newcommand\surj{\twoheadrightarrow}
\newcommand\ded[1]{\overset{\circ}{#1}}
\newcommand\sidenote[1]{\footnote{#1}}
\newcommand\eng[1]{\left\langle#1\right\rangle}
\newcommand\hr{
    \noindent\rule[0.5ex]{\linewidth}{0.5pt}
}

\newcommand{\incfig}[1]{%
    \def\svgwidth{\columnwidth}
    \import{./figures}{#1.pdf_tex}
}
\newcommand{\filler}[1][10]%
{   \foreach \x in {1,...,#1}
    {   test 
    }
}

\newcommand\contra{\scalebox{1.5}{$\lightning$}}
\makeatother
\def\@lecture{}%
\newcommand{\lecture}[3]{
    \ifthenelse{\isempty{#3}}{%
        \def\@lecture{Lecture #1}%
    }{%
        \def\@lecture{Lecture #1: #3}%
    }%
    \subsection*{\@lecture}
    \marginpar{\small\textsf{\mbox{#2}}}
}

\begin{document}
\title{Série 13}
\author{David Wiedemann}
\maketitle
Soit $V( t) $ une primitive de $v( t) $ , on peut réecrire l'équation différentielle par le théorème fondamental de l'analyse par:
\begin{align*}
	\int_{ 1 }^{ x } v( s)ds &= \frac{v( x) ^{2}}{x}\\
	V( x) - V( 1	) &= \frac{ v( x) ^{2}}{x}
	\intertext{ En dérivant par rapport à $x$, on trouve l'équation différentielle }
	v( x) &= \frac{2 v( x) v'( x) x - v( x) ^{2}}{x^{2}}\\
	v( x)  &= 2 v'( x) x - v\\
	v'( x)  &= \frac{v( x) }{2x}+ \frac{1}{2}x
\end{align*}
On voit facilement qu'une solution à l'équation différentielle homogène associée est de la forme
\begin{align*}
	y( t) = C  \sqrt { t} 
\end{align*}
On peut chercher une solution particulière à l'équation différentielle en considérant un polynôme quelconque de degré 2.\\
Soit $ w( t) = a t^{2} + bt + c$, en substituant dans l'équation différentielle, on trouve
\begin{align*}
	2at + b &= \frac{at^{2}+ bt + c}{2t} + \frac{1}{2}t\\
	2at + b &= \frac{1}{2}at +\frac{1}{2}b + \frac{1}{2}t + \frac{c}{2t}
\end{align*}
Ainsi, en comparant les exposants, on obtient la condition
\[ 
2a = \frac{1}{2}a + \frac{1}{2} \Rightarrow  a = \frac{1}{3}
\]
De plus, il est clair que $c=b=0$ .\\
Ainsi, on trouve que $\frac{ t^{2}}{3}$ est une solution particulière de l'équation différentielle.\\
On en déduit que la forme de la solution générale est 
\[ 
	v( t) = \frac{t^{2}}{3}+ C e^{\frac{1}{4}t^{2}} 
\]
On utilisant la condition initiale que $v( 1) = 0$ , on déduit que
\[ 
\frac{1}{3}+ C \sqrt { 1}  = 0 \Rightarrow  C = -\frac{1}{3}
\]
Ainsi la solution à l'équation différentielle est 
\[ 
	v( t) = \frac{t^{2} - \sqrt { t} }{3} 
\]






\end{document}
