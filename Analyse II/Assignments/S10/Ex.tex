\documentclass[11pt, a4paper]{article}
\usepackage[utf8]{inputenc}
\usepackage[T1]{fontenc}
\usepackage[francais]{babel}
\usepackage{lmodern}
\usepackage{amsmath}
\usepackage{amssymb}
\usepackage{amsthm}
\renewcommand{\vec}[1]{\overrightarrow{#1}}
\newcommand{\del}{\partial}
\DeclareMathOperator*{\sgn}{sgn}
\DeclareMathOperator*{\id}{Id}
\DeclareMathOperator*{\im}{Im}
\DeclareMathOperator*{\re}{Re}
\DeclareMathOperator*{\vol}{Vol}
\newcommand\norm[1]{\left\vert#1\right\vert}
\newcommand\ns[1]{\left\vert\left\vert\left\vert#1\right\vert\right\vert\right\vert}
\newcommand\Norm[1]{\left\lVert#1\right\rVert}
\newcommand\N[1]{\left\lVert#1\right\rVert}
\newcommand\abs[1]{\left\vert#1\right\vert}
\newcommand\inj{\hookrightarrow}
\newcommand\surj{\twoheadrightarrow}
\newcommand\ded[1]{\overset{\circ}{#1}}
\newcommand\sidenote[1]{\footnote{#1}}
\newcommand\eng[1]{\left\langle#1\right\rangle}
\newcommand\hr{
    \noindent\rule[0.5ex]{\linewidth}{0.5pt}
}

\newcommand{\incfig}[1]{%
    \def\svgwidth{\columnwidth}
    \import{./figures}{#1.pdf_tex}
}
\newcommand{\filler}[1][10]%
{   \foreach \x in {1,...,#1}
    {   test 
    }
}

\newcommand\contra{\scalebox{1.5}{$\lightning$}}
\makeatother
\def\@lecture{}%
\newcommand{\lecture}[3]{
    \ifthenelse{\isempty{#3}}{%
        \def\@lecture{Lecture #1}%
    }{%
        \def\@lecture{Lecture #1: #3}%
    }%
    \subsection*{\@lecture}
    \marginpar{\small\textsf{\mbox{#2}}}
}

\begin{document}
\title{Série 10}
\author{David Wiedemann}
\maketitle
\section*{1}
Montrons d'abord que la fonction $f( x,y) $ est intégrable au sens de Riemann sur $[0,1 ]^{2}$.\\
Nous allons montrer que pour tout $\epsilon>0$, il existe une partition $P_\epsilon$ de $[0,1]^{2}$ satisfaisant
\[ 
	\overline{S}( f,P_\epsilon) -\underline { S} ( f,P_\epsilon) < \epsilon.
\]
Faisons d'abord l'observation que $\underline { S} ( f,P) =0$ pour toute partition $P$ de $ [ 0,1] ^{2}$, en effet on peut supposer que tous les éléments $Q \in P$ sont des pavés non dégénérés ( si ils l'étaient, ils ne contribueraient pas à la somme) 	, et alors
\[ 
	\underline { S} ( f,P) = \sum_{Q \in P} \inf_{z \in Q} f( z) \vol( Q) 
\]
et $\inf_{z \in Q} f( z) = 0 $ pour tout $Q \in P$ car par densité des irrationnels, il existe $ ( a,b)  \in Q$ tel que $a,b \notin \mathbb{Q}$, on en déduit que $\underline { S} ( f,P)$\\
\hr
Soit donc $\epsilon>0$, et soit $n \in \mathbb{N}^{*}$ satisfaisant $ \frac{1}{n}< \frac{\epsilon}{2}$.\\
Soit $k = \frac{\epsilon}{4 ( n-1) } $.\\
Considérons la partition 
\[ 
P_\epsilon = \left\{ [ 0, \frac{1}{n}], [ \frac{1}{n}, \frac{1}{n-1}-k] , [ \frac{1}{n-1}-k, \frac{1}{n-1}+k] , [ \frac{1}{n-1}+k , \frac{1}{n-2}-k] , \ldots ,[ 1-k, 1] 	\right\} \times [ 0,1] 
\]
La somme de Darboux supérieure est
\begin{align*}
	\overline{S}( f, P_\epsilon) - \underline{S}( f, P_\epsilon) &= \overline{S}( f, P_\epsilon)\\
								     &= \sum_{Q \in P_\epsilon} \sup_{z \in Q} f( z)  \vol  ( Q) \\
				     &= \frac{1}{n} + \sum_{i=1}^{ n-1} \sup _{ z \in \left[ \frac{1}{n-i}-k, \frac{1}{n-i}+k\right] \times [ 0,1] } f( z) \vol ( \left[ \frac{1}{n-i}-k, \frac{1}{n-i}+k\right] \times [ 0,1] ) \\
				     &= \frac{1}{n} + ( n-1) 2k\\
				     & < \frac{\epsilon}{2} + \frac{\epsilon}{2}  = \epsilon
\end{align*}
\section*{2}
Etant donné que l'intégrale existe, et que pour toute partition $P$ de $[0,1]^{2}$, la somme de Darboux inférieure est nulle, on en déduit que 
\[ 
	\int_{[0,1]^{2}} f = 0.
\]





\end{document}
