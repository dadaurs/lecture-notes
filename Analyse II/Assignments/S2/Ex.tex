\documentclass[11pt, a4paper]{article}
\usepackage[utf8]{inputenc}
\usepackage[T1]{fontenc}
\usepackage[francais]{babel}
\usepackage{lmodern}

\usepackage{amsmath}
\usepackage{amssymb}
\usepackage{amsthm}
\newcommand{\del}{\partial}
\begin{document}
\title{Série 2}
\author{David Wiedemann}
\maketitle
\section*{1}
On montre la double implication.\\
$ \Rightarrow $ :\\
Supposons d'abord $E$ ouvert, alors $\forall x \in E \exists \delta >0$ tel que $B( x,\delta) \subset E$, et ainsi $x \in \overset{\circ}{E}$. On en deduit que $E \subset \overset{\circ}{E}$.\\
Car $\overset{\circ}{E} \subset E$, on en déduit que $\overset{\circ}{E} =E$.\\
$\Leftarrow $ :\\
Supposons donc $E= \overset{\circ}{E}$, ainsi $\forall x \in E \exists \delta >0  $ tel que $  B( x, \delta) \subset E $ et donc $E$ est ouvert.
\section*{2}
On montre la double inclusion.\\
$\overset{\circ}{( E^{c} )} \subset ( \overline{E})^{c} $ \\
Soit $x \in \overset{\circ}{( E^{c} )}$. Supposons, par l'absurde, que $x \in \overline{E}$, alors $x \in E $ ou $x \in \del E$.\\
Si $x \in \overset{\circ}{( E^{c} )}$, en particulier $x \in E^{c}$ et donc $x \in E$ est absurde.\\
Si $x \in \del E$, alors $\forall \delta >0,B(x, \delta ) \not\subset E$ et $\forall \delta >0,B(x, \delta ) \not\subset E^{c}$, ainsi $x \notin \overset{\circ}{( E^{c} )}$ et on a une contradiction.\\
$\overset{\circ}{( E^{c} )} \supset ( \overline{E})^{c} $ \\
Soit $x \in ( \overline{E} )^{c}$, alors, $x \notin \del E$ et $x \notin E$.\\
Par définition, $\del E = \del E^{c}$ et donc $x \notin \del E^{c}$, ainsi $\forall x \in ( \overline{E})^{c}$, $\exists  \delta >0 $ tel que $B( x, \delta) \subset E^{c}$, on en deduit que $ x\in \overset{\circ}{E^{c}}$	.


\section*{3}
Comme résultat préliminaire, on montre que $ \overset{\circ}{ E } = E \setminus \del E$.\\
Pour démontrer ceci, on procède par double inclusion.\\
Soit $x \in \overset{\circ}{ E } $, alors $\exists \delta>0$ tel que $B( x,\delta)\subset E$, et ainsi $x \notin \del E$, mais $x \in E$, par définition, on en déduit $x \in E \setminus \del E$.\\
Soit  $x \in E \setminus \del E$, ainsi, par définition du bord et parce que $x \in E$, $\exists \delta>0$ tel que $B( x,\delta) \subset E $, et donc $x \in \overset{\circ}{ E }$.\\
Avec ce résultat, on démontre maintenant que $\overset{\circ}{ ( E )^{c} } = \overline{E^{c}}	$, on procède à nouveau par double inclusion.\\
$\subset:$\\
Soit $x \in\overset{\circ}{ ( E )^{c} }$, car $ \overset{\circ}{ E } = E \setminus \del E$, $x \in \del E \cup E^{c}$.\\
Or, par définition du bord, $\del E = \del E^{c} $, et donc $x \in \del E^{c}\cup E^{c}$, et par une formule du cours $\del E^{c}\cup E^{c} = \overline{E^{c}}$, on en déduit finalement que $x\in \overline{E^{c}}$.\\
$\supset:$\\
Soit $x \in \overline{E^{c}}$, montrons que $x \notin \overset{\circ}{ E }$.\\
On utilise à nouveau que $\overline{E^{c}} = \del E \cup E^{c}$.\\
Si $x \in E^{c}$, alors il est clair que $x \notin\overset{\circ}{ E }$.\\
Si $x \in \del E$, alors $\forall \delta>0, B( x,\delta) \not\subset E$, et donc $x \notin \overset{\circ}{ E }$.\\
Ainsi, $x \in \overset{\circ}{ ( E )^{c} }$.
\section*{4}
On montre la double implication.\\
$ \Rightarrow :$\\
Supposons $E$ fermé, ainsi $E^{c}$ est ouvert.\\
Si $E^{c}$ est ouvert, on a par la partie 1 que $E^{c}= \overset{\circ}{ ( E )^{c} }$.\\
Ensuite, par la partie 2, on a que $\overset{\circ}{ ( E )^{c} } = ( \overline{E})^{c}$.\\
Ainsi $E^{c}= ( \overline{E})^{c}$ et on en déduit que $E = \overline{E}$.\\
$\Leftarrow :$\\
On suppose maintenant que $E = \overline{E}$, montrons que le complementaire de $E$ est ouvert.\\
Par la partie 1,  $E^{c}$ est ouvert si et seulement si $E^{c}= \overset{\circ}{ E ^{c} }$, grâce à la partie 2, la partie 1 et l'hypothèse que $E = \overline{E}$, la suite d'égalités suivantes prouve que $E$ est fermé:
\[ 
	E^{c}= ( \overline{E}) ^{c} = \overset{\circ}{ E ^{c} }
\]









\end{document}
