\documentclass[11pt, a4paper]{article}
\usepackage[utf8]{inputenc}
\usepackage[T1]{fontenc}
\usepackage[francais]{babel}
\usepackage{lmodern}

\usepackage{amsmath}
\usepackage{amssymb}
\usepackage{amsthm}

\newcommand{\del}{\partial}
\newcommand\N[1]{\left\lVert#1\right\rVert}


\begin{document}
\title{Série 6}
\author{David Wiedemann}
\maketitle
\section*{1}
Montrons d'abord que le produit de convolution existe pour tout $x$ dans $\mathbb{R}$.\\
Par hypothèse, $f$ est à support compact, notons $[a,b]\subset \mathbb{R}$ l'intervalle compact sur lequel elle ne s'annulle pas.\\
Alors on peut écrire
\begin{align*}
	f \ast g ( x) &= \int_{-\infty}^{+\infty} f( t) g( x-t) dt\\
		      &= \int_{ a }^{ b } f( t) g( x-t) dt
\end{align*}
$f$ et $g$ étant continue sur $\mathbb{R}$, elles le sont en particulier sur $[a,b]$, respectivement $[x-b,x-a]$, et donc $f \ast g( x) $ existe et est bien défini pour tout $x$.\\
Montrons maintenant que $f\ast g\in C^{k}( \mathbb{R}) $.\\
Soit $h( x,t) \in C^{k}( \mathbb{R}^{2}) $, un théorème du cours donne que  si $ \frac{\del h}{\del x}$ existe, alors
\[ 
	\frac{\del }{\del x} \int_{ c }^{ d } h( x,t) dt = \int_{ c }^{ d } \frac{\del}{ \del x}h( x,t) dt
\]
où $[c,d]$ est à nouveau un intervalle fermé quelconque.\\
En appliquant ce théorème au produit de convolution  en un point $x_0 \in \mathbb{R}$ et en utilisant que $g \in C^{k}( \mathbb{R}) $, on trouve que pour tout $n \in \mathbb{N}, n \leq k$
\begin{align*}
	\int_{ a }^{ b } f( t)  \frac{\del^{n} }{\del x^{n}} g( x_0-t) dt &= \int_{ a }^{ b } \frac{\del^{n} }{\del x^{n}} f( t) g( x_0-t) dt\\
									&= \frac{\del^{n} }{\del x^{n}} \int_{ a }^{ b } f( t) g( x_0-t) dt\\
									&= \frac{\del^{n} }{\del x^{n}} \int_{-\infty}^{+\infty} f( t) g( x_0-t) \\
									&= \frac{\del^{n} }{\del x^{n}} f \ast g ( x_0) 
\end{align*}
On en déduit que $f \ast g \in C^{k}( \mathbb{R})  $.\\
\section*{2}
Montrons d'abord que $f_n$ est $C^{ \infty }_c( \mathbb{R} ) $.\\
Grâce à la partie 1, on sait déjà que $f_n \in C^{ \infty }( \mathbb{R}) $, il suffit donc de montrer qu'il existe un compact $I = [ a',b'] $ en dehors duquel $f_n$ s'annulle.\\
Faisons l'observation que  $\forall n \in \mathbb{N}$, $g_n$ s'annulle ( en particulier)  en tout point $x \in \mathbb{R} \setminus [ -1,1] $.
Ainsi, pour $x>b+1$ 
\begin{align*}
	f\ast g_n (x ) &= \int_{-\infty}^{+\infty} f( t) g_n( x-t) dt\\
		     &= \int_{ a }^{ b } f( t) g_n( x-t) dt\\
		     &= \int_{ a }^{ b } f( t) \cdot 0 dt =0
\end{align*}
Et le même raisonnement montre que pour $x<a-1$, on a également
\[ 
	f \ast g_n ( x) = 0
\]
Ainsi, il existe un intervalle fermé en dehors duquel $f\ast g$ s'annulle et on en déduit que $f\ast g$ est également à support compact.
Montrons maintenant que $f_n$ converge uniformément vers $f$.

Soit $\epsilon >0$, on va montrer qu'il existe un $n^{*}$ tel que $\forall n> n^{*}$, $\N { f_n( x) - f( x) } \leq \epsilon, \forall x \in \mathbb{R}$.\\
On utilisera que $f$ est continue, et donc uniformément continue sur tout intervalle fermé de  $\mathbb{R}$ .\\
Ainsi, il existe $n \in \mathbb{N}$ tel que $\forall x,y \in [ a,b] , |x-y| < \frac{1}{n}  $ implique $ | f( x) - f( y) | < \epsilon$.\\
Pour garder la notation compacte, on notera $ c = \frac{1}{ \int_{ \mathbb{R}} g}$.\\
On a donc
\begin{align*}
	f_n ( x)  &= \int_{-\infty}^{+\infty} f( t) g_n ( x-t) dt\\
			      &= c \int_{x- \frac{1}{n}}^{x+ \frac{1}{n}} f( t) n g( n( x-t) ) dt\\
			      &= c \int_{x-\frac{1}{n}}^{x+ \frac{1}{n}}(  f( x) + ( f( t) -f( x) )  )n g( n( x-t) ) dt\\
			      &= c \int_{ x- \frac{1}{n} }^{ x+ \frac{1}{n} } f( x) n g( n( x-t) ) dt + c \int_{ x-\frac{1}{n} }^{ x+ \frac{1}{n}	 }	( f( t) -f( x) ) n g( n( x-t) ) dt\\
			      &= c \int_{ -1 }^{ 1 } f( x)  g( t ) dt + c \int_{ x-\frac{1}{n} }^{ x+ \frac{1}{n}	 }	( f( t) -f( x) ) n g( n( x-t) ) dt\\
			      &=  f( x) + c \int_{ x-\frac{1}{n} }^{ x+ \frac{1}{n}	 }	( f( t) -f( x) ) n g( n( x-t) ) dt\\
\end{align*}
Faisons maintenant l'observation que par hypothèse, $ |f( t) - f( x) | < \epsilon$, et ainsi 
\begin{align*}
	\left|c \int_{ x-\frac{1}{n} }^{ x+ \frac{1}{n} } ( f( t) - f( x) ) n g( n( x-t) )  \right| & \leq  c \int_{ x- \frac{1}{n} }^{ x+ \frac{1}{n} } \epsilon n \left|g( n( x-t) ) \right| dt\\
										       &= \epsilon
\end{align*}
Ou la dernière égalité tient parce que $g( x) $ est une fonction positive.\\
On en déduit que 
\[ 
	f_n( x) - f( x)  \leq  f( x)  + | \epsilon| - f( x)  = \epsilon
\]
Et de même que
\[ 
	f_n ( x)  - f( x)  \geq f( x)  -  | \epsilon| -f ( x)  = - | \epsilon|
\]
Et donc $f_n$ converge uniformément vers $f$.
\section*{3}
Tout d'abord, notons qu'il est clair que $g$ est non négative, en effet, la fonction $ e^{x} $ étant strictement positive, on voit que l'image de $g$ est positive.\\
Montrons maintenant que $g$ est à support compact, en effet, si $x \in \mathbb{R} \setminus [ -1,1] $, alors en particulier $x \notin ]-1,1[$ et donc $g( x) =0$ par définition.\\
Montrons maintenant que $g$ est infiniment différentiable.\\
Montrons d'abord par récurrence que $ \forall x \in ]-1,1[,\forall k \in \mathbb{N}, \frac{d^{k}g}{dx^{k}} = \frac{P( x) }{Q( x) }	 \cdot e^{\frac{1}{x^2-1}}  $, où $P$ et $Q$ sont deux polynômes. \\
Le cas $n=1$ est clair, en effet,
\[ 
	\frac{dg}{dx}= \left( \frac{-2x}{( x^2-1) }\right) e^{\frac{1}{x^2-1}} 
\]
Supposons le résultat démontré pour $n$ et montrons le résultat pour $n+1$, ainsi on a
\begin{align*}
	\frac{d^{n+1}g}{dx^{n+1}} &= \frac{d}{dx} \left( \frac{d^{n}g}{dx^{n}}\right) \\
				  &= \frac{d}{dx} \left( \frac{P( x) }{Q( x) } e^{\frac{1}{x^2-1}}  \right) \\
				  &= \frac{P'( x) Q( x) - P( x) Q'( x) }{Q( x) ^{2}} e^{\frac{1}{x^{2}-1}} + \frac{P( x) }{Q( x) } \frac{-2x}{x^{2}-1} e^{\frac{1}{x^{2}-1}} \\
				  &= \frac{( x^{2}-1 )( P'(x ) Q( x) - P( x) Q'( x) ) -2x P( x)Q( x) }{Q( x) ^{2}( x^{2}-1) } e^{\frac{1}{x^{2}-1}} 
\end{align*}
Ainsi, on a montré que $g \in C^{ \infty }_c ( ]-1,1[) $.\\
Montrons maintenant que $ \forall k \in \mathbb{N}, \lim_{x \to 1 -} \frac{d^{k}g}{dx^{k}}= \lim_{x \to -1 +} \frac{d^{k}g}{dx^{k}} = 0 $ ( ie. que les limites à gauche et à droite existent et sont nulles).\\
La fonction $g$ étant paire, il suffit de montrer une des deux limites.\\
Considérons la $k$-ième dérivée de $g$, 
\[ 
	\lim_{x \to 1-} \frac{d^{k}g}{dx^{k}} = \lim_{x \to 1-} \frac{P( x) }{Q( x) } e^{ \frac{1}{x^{2}-1	}} 
\]
Posons $u= \frac{1}{x^{2}-1}$, alors on a
\begin{align*}
	\lim_{x \to 1-} \frac{P( x) }{Q( x) } e^{ \frac{1}{x^{2}-1}} &= \lim_{u \to - \infty } \frac{P( \sqrt{1 + \frac{1}{u}} ) }{Q( \sqrt{1+ \frac{1}{u}} ) } e^{u}  = 0
\end{align*}
car l'exponentielle décroit plus rapidement que tout polynôme.\\
Ainsi
\[ 
	\lim_{x \to 1-} \frac{d^{k}g}{dx^{k}} = \lim_{x \to 1+} \frac{d^{k}g}{dx^{k}}= 0
\]
et
\[ 
	\lim_{x \to -1-} \frac{d^{k}g}{dx^{k}} = \lim_{x \to -1+} \frac{d^{k}g}{dx^{k}}= 0
\]
Et on en déduit que $g \in C^{ \infty }_c ( \mathbb{R}) $.





 












\end{document}
