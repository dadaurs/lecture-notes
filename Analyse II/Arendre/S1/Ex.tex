\documentclass[11pt, a4paper]{article}
\usepackage[utf8]{inputenc}
\usepackage[T1]{fontenc}
\usepackage[francais]{babel}
\usepackage{lmodern}

\usepackage{amsmath}
\usepackage{amssymb}
\usepackage{amsthm}
\begin{document}
\title{Série 1}
\author{David Wiedemann}
\maketitle
\section*{1}
Faisons d'abord l'observation que pour tout $\gamma>0$, on a
\[ 
	\lim_{x \to  + \infty} \frac{\ln( x) }{x^{\gamma}} = \lim_{x \to  + \infty} \frac{\frac{1}{x}}{\gamma x^{\gamma-1}} = \lim_{x \to  + \infty} \frac{1}{\gamma x^{\gamma}} = 0
\]
Où on a utilisé la règle de Bernoulli-L'Hospital.\\
Notons donc maintenant que pour tout $\alpha>1$, il existe, par densité des réels un $\beta \in \mathbb{R}$ qui satisfait $\alpha>\beta>1$.
En Choisissant un  $\beta$ satisfaisant cette propriété, on trouve que 
 \[ 
	 \lim_{x \to  + \infty} x^{-\alpha} \ln( x) x^{\beta} =0
\]
Ainsi, par un théorème du cours, l'intégrale doit exister.


On procède par intégration par parties pour trouver le résultat, on a ainsi
\begin{align*}
	\int_{ 1 }^{ \infty  } x^{-\alpha}\ln( x) dx &= \left[ \frac{1}{-\alpha+1}x^{-\alpha+1}\ln( x) \right]_{1}^{ \infty } - \int_{ 1 }^{ \infty  } \frac{1}{-\alpha+1}x^{-\alpha+1} \frac{1}{x} dx
\end{align*}
Le premier terme vaut 0, il vient
\begin{align*}
	\int_{ 1 }^{ \infty  } x^{-\alpha}\ln( x) dx &=  - \int_{ 1 }^{ \infty  } \frac{1}{-\alpha+1}x^{-\alpha}  dx\\
						     &=  \frac{1}{( \alpha-1) ^{2}} 
\end{align*}
\section*{2}
Pour la suite posons $f: [ 1, + \infty[ \to \mathbb{R},f(x) = \frac{\ln( x) }{x^{\alpha}}$ et $g:[1, \infty[ \to \mathbb{R}$ définie par
\[ 
	g( x) = k^{-\alpha}\ln( k)  \text{ si } x \in [ k,k+1[ 
\]
On fait d'abord l'observation que 
\[ 
	f'( x) = x^{-\alpha-1}\left( 1- \alpha \ln( x) \right) 
\]
Ainsi on a, par un théorème d'analyse I, que l'extremum local se situe en 
\[ 
	f'( x) =0 \Rightarrow 	1 - \alpha \ln( x) = 0 \Rightarrow x = e^{\frac{1}{\alpha}} 
\]
Ou l'on ne considére pas la solution $x=0$ car elle n'appartient pas à l'ensemble de définition.\\
De plus, on note que $1-\alpha \ln( x) <0 \quad \forall x> e^{\frac{1}{\alpha}}$ et $1-\alpha \ln( x) >0 \quad \forall x \in [ 1, e^{\frac{1}{\alpha}}[ $ et ainsi $x= e^{\frac{1}{\alpha}}$ est un maximum global\footnote{Ces deux propriétés suivent de $\ln( x) $ étant une fonction strictement croissante}.\\

Par hypothèse, $\alpha>1$, et ainsi la position du maximum est constamment plus petite que $e^{\frac{1}{1}}=e$.\\
Si $f( x) $ est décroissante sur $[e^{\frac{1}{\alpha}}, + \infty [ $, $f( x-1) $ est décroissante sur $[e^{\frac{1}{\alpha}}+1, + \infty [ $ et donc en particulier sur $[4, + \infty[ $ pour toute valeur de $\alpha \in ]1, + \infty[$, il en suit que
\[ 
	g( x) \leq f( x-1) \quad \forall x > 4 
\]


Notons maintenant que
\[ 
	\sum_{k=1}^{ N } k^{-\alpha}\ln( k) = \int_{ 1 }^{ N+1  } g( x) dx
\]
pour $N \in \mathbb{N}\setminus \left\{ 0 \right\} $.\\
Ainsi, pour tout $X \in [4, + \infty[$ 
\[ 
0 \leq	\int_{ 4 }^{ X }g( x) dx \leq \int_{ 4 }^{ X }x^{-\alpha}\ln( x) dx 
\]
où on a utilisé que $g( x) $ est positive sur l'intervalle.\\
Par le critère de comparaison, on en déduit que
\[ 
	\int_{ 4 }^{ \infty  } g( x) dx 
\]

converge, car majorée par une intégrale convergente et minoree par 0.\\
On finit la preuve en notant que
\[ 
	\sum_{k=1}^{ + \infty } k^{-\alpha}\ln( k) = 2^{-\alpha}\ln( 2) + 3^{-\alpha}\ln( 3) +\int_{ 4 }^{ \infty  }f( x) dx
\]
Et ainsi la série converge.
\section*{3}
Etant donne que $f( x-1) $ atteindra son maximum en $e^{\frac{1}{\alpha}}+1$ et sera donc décroissante pour toute valeur plus grande, on peut considèrer un encadrement à partir de $x=4$.\\
Avant de l'expliciter, on constate que
\[ 
	\forall x>4\quad f( x-1) \geq g( x) \geq f(x) 
\]
Ces inégalités suivent directement de $f( x)  $ étant décroissante sur l'intervalle considéré.\\

Etant donné que, par la section 1, $ \int_{ 4 }^{ + \infty  }f( x) dx$ converge, on a, par un théorème du cours
\[ 
	\int_{ 4 }^{ + \infty  } f( x-1) dx \geq \int_{ 4 }^{ + \infty  }g( x) dx = \sum_{k=4}^{ \infty } k^{-\alpha} \ln( \alpha) \geq \int_{ 4 }^{ \infty  }f( x) dx
\]
En évaluant les integrales, on trouve ainsi que
\[ 
	\frac{3^{1-\alpha}( \alpha-1) \log( 3) +1}{( \alpha-1) ^{2}} \geq\sum_{k=4}^{ \infty } k^{-\alpha} \ln( k)\geq \frac{4^{1-\alpha}( \alpha-1) \log( 4) +1}{( \alpha-1) ^{2}} 
\]
Finalement, en ajoutant les trois premiers termes de la somme on trouve

\begin{align*}
	&2^{-\alpha}\ln( 2) + 3^{-\alpha}\ln( 3)  + \frac{3^{1-\alpha}( \alpha-1) \log( 3) +1}{( \alpha-1) ^{2}}\\
	&\geq\sum_{k=1}^{ \infty } k^{-\alpha} \ln( k)\\
	&\geq 2^{-\alpha}\ln( 2)+ 3^{-\alpha}\ln( 3) +\frac{4^{1-\alpha}( \alpha-1) \log( 4) +1}{( \alpha-1) ^{2}} 
\end{align*}
Et ainsi, on a majoré et minoré la série par des termes dépendant de $\alpha$.














\end{document}
