\documentclass[11pt, a4paper, twoside]{article}
\usepackage[utf8]{inputenc}
\usepackage[T1]{fontenc}
\usepackage[francais]{babel}
\usepackage{lmodern}

\usepackage{amsmath}
\usepackage{amssymb}
\usepackage{amsthm}
\begin{document}
\title{Série 1}
\author{David Wiedemann}
\maketitle
\section*{1}
Faisons d'abord l'observation que pour tout $\gamma>0$, on a
\[ 
	\lim_{x \to  + \infty} \frac{\ln( x) }{x^{\gamma}} = \lim_{x \to  + \infty} \frac{\frac{1}{x}}{\gamma x^{\gamma-1}} = \lim_{x \to  + \infty} \frac{1}{\gamma x^{\gamma}} = 0
\]
Où on a utilisé la règle de Bernoulli-L'Hospital.\\
Notons donc maintenant que pour tout $\alpha>1$, il existe, par densité des réels un $\beta \in \mathbb{R}$ qui satisfait $\alpha>\beta>1$.
En Choisissant un  $\beta$ satisfaisant cette propriété, on trouve que 
 \[ 
	 \lim_{x \to  + \infty} x^{-\alpha} \ln( x) x^{\beta} =0
\]
Ainsi, par un théorème du cours, l'intégrale doit exister.


On procède par intégration par parties pour trouver le résultat, on a ainsi
\begin{align*}
	\int_{ 1 }^{ \infty  } x^{-\alpha}\ln( x) dx &= \left[ \frac{1}{-\alpha+1}x^{-\alpha+1}\ln( x) \right]_{1}^{ \infty } - \int_{ 1 }^{ \infty  } \frac{1}{-\alpha+1}x^{-\alpha+1} \frac{1}{x} dx
\end{align*}
Le premier terme vaut 0, on a ainsi que
\begin{align*}
	\int_{ 1 }^{ \infty  } x^{-\alpha}\ln( x) dx &=  - \int_{ 1 }^{ \infty  } \frac{1}{-\alpha+1}x^{-\alpha}  dx\\
						     &=  \frac{1}{( \alpha-1) ^{2}} 
\end{align*}
\section*{2}
Considérons la fonction $f:]-1, \infty[ \to \mathbb{R}$ définie par
\[ 
	g( x) = k^{-\alpha}\ln( k)  \text{ si } x \in [ k,k+1[ 
\]
Notons maintenant que
\[ 
	\sum_{k=1}^{ N } k^{-\alpha}\ln( k) = \int_{ 1 }^{ N  } g( x) dx
\]
pour $N \in \mathbb{N}\setminus \left\{ 0 \right\} $.\\
Notons que la fonction $x^{-\alpha}\ln( x) $ satisfait
\[ 
	x^{-\alpha}\ln( x) \geq g( x) \forall x \in ]1, \infty[
\]
Ainsi, on a, pour tout $X \in ]1, + \infty[$ 
\[ 
	\int_{ 1 }^{ X }g( x) dx \leq \int_{ 1 }^{ X }x^{-\alpha}\ln( x) dx 
\]
Par le critère de comparaison, on en déduit que
\[ 
	\int_{ 1 }^{ \infty  } g( x) dx 
\]

converge, car majorée par une fonction dont l'intégrale converge et minorée par 0( la fonction est strictement positive sur l'intervalle considéré). \\
On finit la preuve en notant que
\[ 
	\sum_{k=1}^{ + \infty } k^{-\alpha}\ln( k) = \int_{ 1 }^{ \infty  }f( x) dx
\]
Et ainsi la série converge.
\section*{3}
Pour la suite posons $f: [ 1, + \infty[ \to \mathbb{R},f(x) = \frac{\ln( x) }{x^{\alpha}}$.\\
On fait d'abord l'observation que 
\[ 
	f'( x) = x^{-\alpha-1}\left( 1- \alpha \ln( x) \right) 
\]
Ainsi on a, par un théorème d'analyse I, que l'extremum local se situe en 
\[ 
	f'( x) =0 \Rightarrow 	1 - \alpha \ln( x) = 0 \Rightarrow x = e^{\frac{1}{\alpha}} 
\]
Ou l'on ne considére pas la solution $x=0$ car elle n'appartient pas à l'ensemble de définition.\\
De plus, on note que $1-\alpha \ln( x) <0 \quad \forall x> e^{\frac{1}{\alpha}}$ et $1-\alpha \ln( x) >0 \quad \forall x \in [ 1, e^{\frac{1}{\alpha}}[ $ et ainsi $x= e^{\frac{1}{\alpha}}$ est un extremum global\footnote{Ces deux propriétés suivent de $\ln( x) $ étant une fonction strictement croissante}.\\
Par hypothèse, $\alpha>1$, et ainsi la position de l'extremum global est constamment plus petit que $e^{\frac{1}{1}}=1$.\\
On peut donc considèrer un encadrement à partir de $x=3$.\\
Avant de l'expliciter, on constate que
\[ 
	f( x-1) \geq g( x) \geq f(x) 
\]
L'inégalité $f( x-1) \geq f( x) $ suit du théorème des accroissements finis, en effet, pour tout $m>3$, on a l'existence d'un $c \in [m,m+1]$ satisfaisant
\[ 
	0 > f'( c)  = f( m+1) -f( m) \text{ impliquant } f( m+1) < f( m) 
\]
Les égalités $f( x-1) \geq g( x)$ et $g( x) \geq f( x) $ suivent immédiatement de $f$ étant décroissante $\forall x> 3$.\\

Etant donné que, par la section 2, $ \int_{ 3 }^{ + \infty  }$ converge, on a, par un théorème du cours
\[ 
	\int_{ 3 }^{ + \infty  } f( x-1) dx \geq \int_{ 3 }^{ + \infty  }g( x) dx = \sum_{k=3}^{ \infty } k^{-\alpha} \ln( \alpha) \geq \int_{ 3 }^{ \infty  }f( x) dx
\]
En évaluant les integrales, on trouve ainsi que
\[ 
	\frac{4^{1-\alpha}( \alpha-1) \log( 4) +1}{( \alpha-1) ^{2}} \geq\sum_{k=3}^{ \infty } k^{-\alpha} \ln( k)\geq \frac{3^{1-\alpha}( \alpha-1) \log( 3) +1}{( \alpha-1) ^{2}} 
\]
Finalement, en ajoutant les deux premiers termes de la somme on trouve

\[ 
	2^{-\alpha}\ln( 2) + \frac{4^{1-\alpha}( \alpha-1) \log( 4) +1}{( \alpha-1) ^{2}} \geq\sum_{k=1}^{ \infty } k^{-\alpha} \ln( k)\geq 2^{-\alpha}\ln( 2) +\frac{3^{1-\alpha}( \alpha-1) \log( 3) +1}{( \alpha-1) ^{2}} 
\]














\end{document}
