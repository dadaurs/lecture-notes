\documentclass[11pt, a4paper]{article}
\usepackage[utf8]{inputenc}
\usepackage[T1]{fontenc}
\usepackage[francais]{babel}
\usepackage{lmodern}
\newcommand\N[1]{\left\lVert#1\right\rVert}
\usepackage{amsmath}
\usepackage{amssymb}
\usepackage{amsthm}

\begin{document}
\title{Série 6 Exercice à rendre}
\author{David Wiedemann}
\maketitle
\section*{1}
On considère une suite $ \left\{ x_i \right\} \subset E$\footnote{On n'exclut pas les suites constantes} convergeant vers $x$.\\
Par un théorème du cours, on sait que $x \in \overline{E}$.\\
On considère maintenant les deux suites $( f( x_i) )_{i \in \mathbb{N}}  $ et $( g( x_{i} ) )_{i \in \mathbb{N}}$.\\
Car $f$ et $g$ sont continues, 
\[ 
	\lim_{i \to \infty } f( x_i) = f( x) \text{ et } \lim_{i \to  + \infty} g( x_i) = g( x) 
\]
Ainsi, on a que, pour tout $\epsilon>0$, il existe $N \in \mathbb{N}$ satisfaisant que pour tout $i>N$, on a
\[ 
	|f( x_i) - f( x)| < \epsilon
\]
Or $x_i \in E$ pour tout $i$ et donc
\[ 
	|f( x_i) - f( x)| =|g( x_i) - f( x)| <\epsilon
\]
Ainsi $g( x_i) $ converge vers $f( x) $, et donc, $g( x) = f( x) $.\\
Car ceci est vrai pour toute suite de $E$ et tout élément $x \in \overline{E}$, on en déduit que 
\[ 
	f( x) =g( x)  \forall x \in \overline{E}
\]

\section*{2}
On considère à nouveau une suite $ \left\{ x_i \right\} \subset E$ convergeant vers $x \in \overline{E}$.\\
Par continuité, on sait à nouveau que 
\[ 
	\lim_{i \to +\infty } f( x_i) = f( x) \text{ et } \lim_{i \to  + \infty} g( x_i) = g( x) 
\]

Notons que, par hypothèse, on a
\[ 
	f( x_i) \leq g( x_i) \forall i \in \mathbb{N}
\]
Ainsi, par une propriété du cours, on a bien que
\[ 
	\lim_{i \to  + \infty} f( x_i) \leq \lim_{i \to  + \infty} g( x_i) 
\]
et donc
\[ 
	f( x) \leq g( x) 
\]
Etant donne que ceci est valable pour toute suite  $ ( x_i)_{i \in \mathbb{N}} \subset E$ et tout $x \in \overline{E}$, on a montré que
\[ 
	f( x) \leq g( x) \forall x \in \overline{E}
\]

\section*{3}
Montrons d'abord que $ E:= \left\{ x \in \mathbb{R}^n: h( x) >0 \right\} $ est un ensemble ouvert.\\
Soit $ y= ( y_1, \ldots, y_n)  \in E$.\\
Par l'absurde, supposons que $E$ n'est pas ouvert, alors $\exists y \in E $ tel que $\forall \delta>0 \exists x \in B( y,\delta)$ satisfaisant $h( x) \leq 0$.\\
Soit $\epsilon= \frac{1}{2}h( y) $, alors, par la continuité de $h$, il existe $\delta>0$, satisfaisant
\[ 
	\N{y-x} < \delta \Rightarrow |h( y) -h( x)| < \epsilon
\]
Or, par hypothèse, $h( x) \leq 0$ et donc
\[ 
	|h( y) - h( x) | = |h( y) | + |h( x) | > \epsilon
\]
ce qui constitue une contradiction à l'hypothèse. On en déduit que $E$ est un ensemble ouvert.\\

Montrons maintenant que $ F:= \left\{ x \in \mathbb{R}^n: h( x) =0 \right\} $.\\
Notons d'abord que, par symmetrie, l'ensemble $ E':= \left\{ x \in \mathbb{R}^n: -h( x) >0 \right\} = \left\{ x \in \mathbb{R}^n: h( x) <0 \right\} $ est également ouvert.\\
Car l'union de deux ensembles ouverts est ouverte, $E \cup E'$ est ouvert.\\
Ainsi, le complémentaire $( E \cup E')^{c}$ est fermé.\\
Or, il est clair que $( E \cup E')^{c}= F $ et donc $F$ est fermé.






\end{document}
