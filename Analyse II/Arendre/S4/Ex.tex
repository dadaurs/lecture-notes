\documentclass[11pt, a4paper, twoside]{article}
\usepackage[utf8]{inputenc}
\usepackage[T1]{fontenc}
\usepackage[francais]{babel}
\usepackage{lmodern}
\usepackage{amsmath}
\usepackage{amssymb}
\usepackage{amsthm}
\newcommand\N[1]{\left\lVert#1\right\rVert_2}

\begin{document}
\title{Série 8, Exercice à rendre}
\author{David Wiedemann}
\maketitle
\section*{1}
Par un théoreme du cours, il faut montrer que pour toute suite $( x^{( k) }) $ convergeant vers 0, on a
\[ 
\lim_{k \to  + \infty} 	N( x^{( k) }) =0
\]

Soit donc $x^{( k) }$ une suite convergeant vers 0. On peut écrire $  x^{( k) }= \lambda_k\cdot e_k$, ici $\lambda_k \in \mathbb{R}$ et $e_k \in \mathbb{R}^{n}$.\\
On peut egalement supposer que $e_k$ satisfait $N( e_k ) = N( e_j ) \neq 0 \quad \forall k,j \in \mathbb{N}^{*} $. L'existence de ces $e_j$ est garantie car $N( \lambda \cdot x ) = |\lambda| \cdot N( x) $.\\
La norme étant bornée, on déduit que Ainsi, $\lim_{k \to  + \infty} \lambda_k = 0$ et donc
\[ 
	\lim_{k \to  + \infty} N( x^{( k) }) = \lim_{k \to  + \infty} | \lambda_k| \cdot N( e_k) = 0
\]
\section*{2}
Soit $x_0 \in \mathbb{R}^n$, soit $x^{( k) }$ une suite convergeant vers $x_0$, alors faisons l'observation que \footnote{On utilise ici l'inégalité triangulaire inversée, la preuve de cette dernière suit de l'inégalité triangulaire et ne dépend donc pas du choix de la norme.}
\[ 
	|N( x^{( n) }- x_0)  - N( x_0)|  \leq N(x^{( n) } ) \leq N(  x^{( n) }-x_0 )  + N( x_0) \forall n \in  \mathbb{N}^{*}
\]
Or, la suite $x^{( n) }- x_0$ tend vers 0.\\
En conclut, par le théorème des deux gendarmes et donc
\[ 
	\lim_{k \to  + \infty} N( x^{( k) }) = N( x_0) 
\]
pour toute suite $x^{( k) }$ et donc $N( x) $ est une fonction continue.

\section*{3}
Etant donné que la sphère unitaire centrée en $0$ est le bord de l'ensemble $ B( 0,1) $, il est immédiat que cet ensemble est borné. On dénotera la sphère unitaire par $S$.\\
Pour montrer que $S$ est fermé, on remarque que $S = \partial B( 0,1) $ et donc $\forall x \in S\quad \exists \delta >0 $ tel que $B( x,\delta) \cap B( 0,1)  \neq \emptyset$ et $B( x,\delta) \cap B( 0,1) ^{c} \neq \emptyset$.\\
Ainsi, il est immédiat que le complémentaire de $S$ est ouvert et donc $S$ est fermé.\\
$S$ étant fermé et borné, $S$ est compact.\\
Montrons maintenant que toute norme est équivalenter à la norme euclidienne.\\
$N$ étant une fonction continue, $N $ atteint ses extremas sur $S$:
\[ 
	\exists x_m, x_M \text{ tel que } N( x_m) = \inf_{i \in S} N( i) \text{ et } N( x_M) = \sup_{i \in S} N(i)
\]
Si $N$ atteint son maximum ( respectivement minimum)  plusieurs fois, il suffit de choisir un des $x_M$.\\
Posons maintenant
\[ 
	C_1= N( x_m)  \text{ et } C_2= N( x_M) 
\]
On a alors \footnote { Ceci suit du fait que $\frac{x}{\N{x}}$ est dans $S$ pour tout $x \in \mathbb{R}^n$	  }  que $\forall x \in \mathbb{R}^n\setminus \left\{ 0 \right\} $
\[ 
	N ( x)  = N( \N { x} \frac{x}{\N { x} }) = \N { x}  N( \frac{x}{\N{x}}) \leq \N{x} C_2
\]
et de la même manière, on trouve que
\[ 
	N( x) \geq \N { x} C_1
\]
et on en déduit 
\[ 
	\N  { x} C_2\geq N( x) \geq \N { x} C_1
\]
Cette inégalité est bien sur également respectée dans le cas $x=0$, et donc les normes sont équivalentes.

\section*{4}
Soient $A, B: \mathbb{R}^n\to \mathbb{R}$ deux normes, alors par la partie 3, $\exists a_1, a_2, b_1,b_2 \in \mathbb{R}^{+}$ tel que $\forall x \in \mathbb{R}^n$
\[ 
	\N { x} a_1 \leq A( x) \leq \N { x} a_2 \text{ et } \N { x} b_1 \leq B( x) \leq \N { x} b_2
\]
On en déduit que 
\[ 
	\frac{ A(x)}{a_2} \leq \frac {  B( x)} { b_1} \Rightarrow A( x) \leq \frac{a_2}{b_1} B( x) 
\]
et de même
\[ 
	A( x) \frac{1}{a_1}\geq B( x) \frac{1}{b_2} \Rightarrow A( x)  \geq \frac{a_1}{b_2}	B( x) 
\]
Ainsi, les normes $A( x) $ et $B(x) $ sont semblables.\\
On en déduit que toutes les norme sont équivalentes sur $ \mathbb{R}^n$.







\end{document}
