\documentclass[../main.tex]{subfiles}
\begin{document}
\lecture{9}{Mon 28 Mar}{Theorie combinatoire des groupes}
\subsection{Bouteille de Klein}
Soit $K$ la bouteille de Klein, quotient de $ I^{2}$.\\
On comprend que $K$ s'ecrit comme
\[ 
K = ( S^{1}_a \vee S^{1}_b) \cup_{abab^{-1}} e^{2}
\]
On decoupe $I\times I$ le long de deux segments verticaux le long de $ ( \frac{1}{3},t) $ et $ ( \frac{2}{3},t )$\\
Ainsi, $K$ est un quotient de trois bandes verticales, et aussi de deux bandes $ A_2$ et $ \faktor { A_1\coprod A_3} { b'\sim b''} $.\\
\section{Theorie Combinatoire des Groupes}
But: Decrire et manipuler des groupes de maniere agreable pour pouvoir construire des pushouts.
\subsection{Groupes libres}
\begin{exemple}
Le groupe ayant un seul generateur $a$, sujet a aucune relation autre que les axiomes de groupe est le groupe $ \left\{ a^{n}| n \in \mathbb{Z} \right\} / a^{0}=1\simeq \mathbb{Z}$.\\
On observe que pour tout groupe $G$  $ \hom_{ \gr } ( F( a) , G) \simeq \hom_{\set} ( \left\{ a \right\} , UG)  $ .
\end{exemple}
\begin{defn}[Groupe libre]
	Soit $I$ un ensemble, le groupe libre $F( I) $ a $I$ generateur est obtenu en associant a chaque indice $\alpha \in I$ un generateur $x_\alpha\in F( I) $.\\
	Tous les mots sont obtenus par concatenation de $x_\alpha ^{n}$ pour $n\in \mathbb{Z}$ avec les identifications $x_\alpha ^{0}= 1, 1\cdot x_\alpha = x_\alpha = x_\alpha \cdot 1$ et $x_\alpha x_\alpha ^{-1} = 1$ 
\end{defn}
De la construction de $F( I) $, on comprend qu'un homomorphisme $\phi:F( I) \to G$ est determine et meme equivalent a la donnee des images $g_\alpha = \phi( x_\alpha) $.\\
Ces isomorphismes  etant naturels, on a que
\begin{propo}
Le foncteur $F( -) $ est adjoint a gauche de $U$.
\end{propo}
Le groupe libre abelien est un quotient de $F( a,b) $ via $\phi: F( a,b) \to \mathbb{Z}\times \mathbb{Z}$ determine par $\phi( a) = ( 1,0) $ et $\phi( b) = ( 0,1) $.\\
$\ker\phi$ contient $aba^{-1}b^{-1}= [ a,b] $, ainsi, il contient  aussi $ [ a,b]^{n}$ pour $ n \in \mathbb{Z}$ qui forment un sous-groupe cyclique infini dans $F( a,b) $.\\
Cependant, ce sous-groupe n'est pas normal et en fait $\ker\phi$ est le sous-groupe normal engendre par $ [ a,b] $.\\
\subsection{Presentations}
\begin{lemma}
Tout groupe est quotient d'un groupe libre.
\end{lemma}
\begin{proof}
Soit $G$ un groupeet $ \left\{ g_\alpha  | \alpha\in I\right\} $ un ensemble de generateurs ( par exemple tous les $g\in G$ ).\\
On definit $\phi: F( I) \to G: x_\alpha\mapsto g_\alpha$, alors $\phi$ est surjective.
\end{proof}
\begin{defn}
	Soit $\phi: F( I) \to G$ un homomorphisme surjectif. Un element de $\ker\phi$ represente par un mot $r_\beta$ en les generateurs $x_\alpha$ est un relateur.\\
	Pour un choix de generateurs, $\beta\in J$ de $\ker\phi$ comme sous-groupe normal de $F( I) $ on appelle $ \langle x_\alpha, \alpha\in I | r_\beta, \beta \in J\rangle$ une presentation de $G$.\\
	Chaque relateur correspond a une relation $r_\beta=1$ dans le quotient de $F( I) $ 
\end{defn}
Autrement dit $G$ est isomorphe a ce quotient.
\subsection{Graphes de Cayley}
On cherche a representer geometriquement un groupe donne par une presentation.
\begin{defn}[Graphe de Cayley]
	Soit $ G = \langle x_\alpha| r_\beta\rangle$, le graphe de Cayley $\Gamma ( G, \left\{ x_\alpha \right\} ) $ est le graphe oriente et colore dont les sommets sont $g\in G$ et les aretes relient $g$ et $gx_\alpha$, oriente de $g$ vers $gx_\alpha$, de couleur $\alpha$.
\end{defn}
\begin{rmq}
Comme espace topologique, ce graphe est un quotient d'intervalles, un pour chaque arete, et on identifie les sommets a l'element du groupe voulu.
\end{rmq}




\end{document}	
