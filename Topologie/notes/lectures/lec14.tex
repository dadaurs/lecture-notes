\documentclass[../main.tex]{subfiles}
\begin{document}
\lecture{14}{Mon 02 May}{Revetements}
On veut calculer $\pi_1\left( T^{2}\# \mathbb{R}P^{2}\right) $.\\
Comme $ \mathbb{R}P^{2} \# \mathbb{R}P^{2} \# \mathbb{R}P^{2}\simeq T^{2} \# \mathbb{R}P^{2}$.\\
On calcule le groupe fondamental grace a la presentation de $Y \coloneqq T^{2}\# \mathbb{R}P^{2}$ .\\
Si $P$ est le pushout de $ \vee_1^{3} \leftarrow \delH \simeq S^{1}\hookrightarrow H \simeq D^{2}$, on a une application induite $P\to Y$, c'est une bijection d'un compact vers un hausdorff, c'est un homeomorphisme.\\
Donc $\pi_1 Y = \pi_1 P = \pi_1\left( ( S^{1}_d \vee S^{1}_e\vee S^{1}_f)\cup_p e^{2} \right) \simeq \langle \delta, \epsilon, \phi | r\rangle$ .\\
Ici, $r$ est le mot $[p] \in \left[ S^{1}, S^{1}\vee S^{1}\vee S^{1}\right] $ 	qui est l'image de $\id_{\del H} $ par $p_\ast$.\\
Comme $\id_{\del H} = d' \ast e' \ast \overline{d''} \ast \overline{e''} \ast f' \ast f''$.\\
Son image est le lace $d\ast e \ast \overline{d} \ast \overline{e} \ast f \ast f $.\\
Sa classe d'homotopie est donc donnee par $\delta \epsilon \delta^{-1}\epsilon^{-1} \phi^{2}$.\\
De la boite noire, il suit que toute surface est soit la sphere, soit une somme connexe de tores et plans projectifs.\\
S'il n'y pas de $ \mathbb{R}P^{2}$, on a une somme connexe de $g$ tores, un tore a $g$ trous.\\
Sinon, il y a au moins une copie de $ \mathbb{R}P^{2}$ et la surface $S$ sera de la forme $ T^{2} \# \ldots \# T^{2} \# \mathbb{R}P^{2} \ldots \# \mathbb{R}P^{2}= ( T^{2})^{\# a} \# ( \mathbb{R}P^{2})^{\# b} = ( \mathbb{R}P^{2})^{\#2a+b}$.
\begin{thm}[Theoreme de classification]
	Une surface est homeomorphe a , $ ( T^{2} )^{\# g}$  ou $ ( \mathbb{R}P^{2})^{k}$ pour $g \geq 1, k \geq 1$.
\end{thm}
On pourrait montrer pour conclure que les $\pi_1$ de ces surfaces sont tous distincts, on va plutot etudier  $H_1 S = ( \pi_1S)_{ab} $.\\
On sait que $\pi_1 S^{2} =1 , H_1 S^{2}=0$.\\
On calcule $H_1( ( T^{2}) ^{\# g}) = \mathbb{Z}^{2g}$.
\begin{lemma}
$H_1(  ( \mathbb{R}P^{2})^{\# k}) = \mathbb{Z}^{k-1}\times \faktor { \mathbb{Z}} { 2\mathbb{Z}} $ 
\end{lemma}
\begin{proof}
\begin{align*}
\pi_1( ( \mathbb{R}P^{2})^{\# k}) =\langle a_1,\ldots, a_k | a_1^{2} \ldots a_k ^{2}\rangle
\end{align*}
On construit $\phi: \pi_1(  ( \mathbb{R}P^{2} )^{\# k}) \to A= \mathbb{Z}^{k-1}\times \faktor { \mathbb{Z}} { 2\mathbb{Z}}; a_i \to e_i, a_k \to f- ( e_1 +\ldots + e_{k-1} ) $ ou $e_i $ est le ieme generateur de $ \mathbb{Z}^{k-1}$.\\
On verifie que $\phi( a_1^{2} \ldots a_k^{2}) =0 $ pour etre sur que $\phi$ est un homomorphisme.\\
Calculons $\phi( a_1^{2} \ldots a_k ^{2}) = 2 \phi ( a_1) +\ldots +  2\phi( a_k) ) + 2 ( f- e_1 -\ldots - e_{k-1} ) = 2f =0 $.\\
Comme $A$ est abelien, $\phi$ induit $ \overline{\phi}: \pi_{ab} \to A$.\\
On construit l'inverse a $\psi$ en envoyant $e_i \to \overline{a_i}$ et $f\to \overline{a_1 a_2 \ldots a_k}$.\\
Comme $A$ est un groupe libre abelien produit avec $ \faktor { \mathbb{Z}} { 2 \mathbb{Z}} $, il suffit de verifier que $\psi( 2f) =0$ ce qui est immediat.\\
On termine en verifiant que $\psi$ est l'inverse de $\phi$.\\
Comme les groupes $0, \mathbb{Z}^{2g}, \mathbb{Z}^{k-1}\times \faktor { \mathbb{Z}} { 2 \mathbb{Z}} $.\\
\end{proof}
\begin{rmq}
Pour reconnaitre $S$, il n'est pas necessaire de calculer $H_1 S$, il suffit de savoir si $S$ est orientable (sinon elle contient un ruban de moebius) et de calculer $\xi( S) $, la caracteristique d'euler.\\
En general $\xi( S) $ est $\#$ sommets $- \#$ arretes + $\#$ faces pour un polyedre.\\
Pour nous, c'est donc $ 1-  l + 1= 2-l$ 
\end{rmq}
\section{Les Revetements}
\subsection{Definitions}
Tous les espaces sont Hausdorrf, connexes par arcs et localement connexes par arcs.\\
Ie.,  tout voisinage de tout point contient un voisinage de ce point qui est connexe par arcs.\\
Pour $x\in X$, on appellera $p^{-1}( x) = Fx$ la fibre au-dessus de $X$ 
\begin{defn}
	Une application $p: E \to X$ est un revetement si tout point $x\in X$ admet un voisinage ouvert trivialisant $U \ni x$, tel que $p^{-1}( U) $ est une reunion disjoint de $U_i, i \in I$, appeles feuillets, avec $p|_{U_i} : U_i \to U$ un homeomorphisme.
\end{defn}
\begin{exemples}
\begin{enumerate}
\item $\id_X$ est un revetement
\item $\exp : \mathbb{R}\to S^{1}$ est un revetement
\item $q: S^{2} \to \mathbb{R}P^{2}$ est un revetement
\item Pour $F$ discret, la projection $p_2 : X\times F\to X$ est le revetement trivial.
\end{enumerate}

\end{exemples}
\begin{lemma}
Les fibres sont des espaces discrets
\end{lemma}
\begin{proof}
Soit $e\in Fx$ et $U\ni x$ un ouvert trivialisant.\\
Alors $e$ appartient a un seul feuillet $U_i$ et donc dans $Fx$ 
\end{proof}
\begin{propo}
Un revetement est une surjection ouverte, en particulier un quotient. 
\end{propo}


	

\end{document}	
