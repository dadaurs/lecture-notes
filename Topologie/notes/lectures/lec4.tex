\documentclass[../main.tex]{subfiles}
\begin{document}
\lecture{4}{Mon 07 Mar}{Attachements de Cellules}
\subsection{Recollements}
On construit de nouveaux espaces a l'aide de pieces plus simple.\\
On se donne $f:A\to X,g:A\to Y$ deux applications. On recolle $X$ et $Y$ le long de $A$ 
\begin{defn}[Recollement]
	Le recollement de $X$ et $Y$ le long de $A$ est le quotient de $X \coprod Y$ par la relation d'equivalence engendree par $f( a) \sim g( a) \forall a\in A$ 
\end{defn}
\begin{rmq}
Il ne suffit pas d'identifier $f( a) \sim g( a) $ pour que la relation soit une relation d'equivalence.\\
Pour garantir la transitivite, on a des zigszags d'equivalence $f( a) \sim g( a) = g( b) \sim f( b) =f( c) \sim g( c) \ldots$ 
\end{rmq}
\begin{exemple}
Si $A= \ast$, $f( \ast) =x_0\in X, g( \ast) = y_0 \in Y$, alors le recollement $ X\cup_\ast Y$ est le wedge $X\vee Y$ 
\end{exemple}
On notera le recollement $X\cup_A Y$.\\
Si $q:X\coprod Y \to X\cup_A Y$ est le quotient, alors l'inclusion $ i_1:X \to X\coprod Y$ induit $i= q\circ i_i:X \to X\cup_A Y$ et de meme pour l'inclusion de $Y$.
\begin{propo}
Le recollement $X\cup_A Y$ est le pushout de $ Y \leftarrow A \rightarrow X$.
\end{propo}
\begin{proof}
On doit montrer l'existence et l'unicite de $\theta$.\\
Puisque chaque element de $X\cup_A Y$ admet un representant dans $X$ ou $Y$, on doit poser $\theta(  [ x] ) = \alpha( x) \forall x \in X$ et $\theta( [ y] ) =\beta( y) \forall y\in Y $.\\
On montre l'existence.\\
Posons $\Theta:X\coprod Y\to Z$ l'application determinee par $\alpha$ et $\beta$.\\
On verifie que $\Theta$ est compatible avec $\sim$. Soit $a\in A$, alors $\Theta( f( a) ) =\alpha( f( a) ) = \beta( g( a) )=\Theta( g( a) )  $.\\
Ainsi $\Theta$ passe au quotient et induit $\theta$, qui est donc bien continue.
\end{proof}
Des maintenant, on suppose que $g: A \subset Y $ est l'inclusion d'un sous-espace ferme.
\begin{lemma}
Soit $C \subset Y$, alors la saturation de $C$ est
\[ 
f( C\cap A) \coprod ( C\cup f^{-1}\circ f( C\cap A) ) 
\]


\end{lemma}
\begin{proof}
On va regarder ce qui se passe pour tout $c\in C$.\\
Si $c\notin A$, alors $q^{-1}( q( c) ) = \left\{ c \right\} $, sinon $q^{-1}( q( c) ) $ contient $f( c) \in X$ et $f^{-1}( f( c) ) \subset Y$ 
\end{proof}
\begin{lemma}
Si $C \subset X$ , $q^{-1}( q( C) ) = C \coprod f^{-1}( C) \subset X\coprod Y$ 
\end{lemma}
\begin{proof}
Comme ci-dessus, si $c\in C$ n'est pas dans l'image de $f$ , on a $q^{-1}( q( c) ) = \left\{ c \right\} $, sinon on a $c\in X$ et $f^{-1}( c) \subset A \subset Y$ 
\end{proof}
\begin{propo}
Soient $X$ et $Y$ deux espaces separes, $g : A \subset Y$ l'inclusion d'un compact, alors $X\cup_A Y$ est separe.
\end{propo}
\begin{proof}
On observe que $X\coprod Y$ est separe. Avant d'appliquer le critere de separabilite, on montre que l'application quotient est fermee. Comme un ferme de $X\coprod Y$ est la reunion disjointe de deux fermes on a deux cas.\\
Si $C \subset X$ ferme, alors $q( C) $ est ferme $\iff$ $q^{-1}( q( C) ) $ est fermee. Par le lemme ci-dessus,
\[ 
q^{-1}( q( C) ) = C \coprod f^{-1}( C) 
\]
qui sont fermes.\\
Si $C \subset Y$, alors $q^{-1}( q( C) ) = f( C\cap A)  \coprod ( C \cup f^{-1}( f( C\cap A) ) ) $ 
\end{proof}
On a $f( C\cap A) $ compact et donc ferme puisque $Y$ est separe.\\
Pour conclure, on verifie les deux conditions du critere.\\
Pour conclure, on verifie les deux conditions du critere, la saturation d'un ferme est fermee grace aux preparatifs.\\
Soit $z\in X\coprod Y$, on doit montrer que $q^{-1}( q( z) ) $ est compact, les lemmes ci-dessus permettent de conclure parce que si $z=a\in A, f^{-1}( f( a) ) $ est un ferme d'un compact et est donc compacte.	
\subsection{Attachement de cellules}
Ici $g: A \subset CA = \faktor { A\times I} { A\times 1}  $.\\
Soit $f:A\to X$, le recollement $X\cup_A CA$ aussi note $ X\cup_f CA$ est appele attachement d'une $A$-cellule sur $X$ le long de $f$.\\
Si $A= S^{n-1}$ alors cet attachement est celui d'une $n$-cellule
\begin{rmq}
$CS^{n-1}\simeq D^{n}$, on note $X\cup_f CS^{n-1}= X\cup_f e^{n}$ ou $X\cup_fD^{n}$ et on appelle $e^{n}\simeq D^{n}$ une $n$-cellule ( fermee.) 
\end{rmq}
\begin{propo}
Si $X$ est separe et $A$ est compact et separe, alors $X\cup_fCA$ est separe.\\
Si en plus $X$ est compact
\end{propo}
\begin{proof}
Le premier point suit de la proposition precedente car $CA$ est separe, le 2eme point suit du critere de compacite car $X\coprod Ca$ est compact.
\end{proof}
\begin{defn}[Suspension]
	La suspension de $A$ est le quotient $ \faktor { A\times I} { ( a,0) \sim ( a',0) \text{ et } ( a,1) \sim ( a',1) } $ 
\end{defn}



\end{document}	
