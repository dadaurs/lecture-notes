\documentclass[../main.tex]{subfiles}
\begin{document}
\lecture{10}{Mon 04 Apr}{Amalgamations}
\subsection{Produits libres}
Soit $G= \langle x_\alpha| r_\beta\rangle$ et $H = \langle y_\gamma | s_\delta\rangle$, on forme le produit libre
\[ 
G\ast H = \langle x_\alpha,y_\gamma | r_\beta, s_\delta\rangle
\]
\begin{propo}
Le produit libre $G\ast H$ est le coproduit de $G$ et $H$ dans la categorie des groupes, ie. $\hom( G\ast H, M) \simeq \hom( G,M) \times \hom( H,M) $ pour tout groupe $M$.
\end{propo}
\begin{proof}
	Soit $G\hookrightarrow G\ast H \hookleftarrow H$.\\
	Soit $\omega: G\ast H \to M$, alors on peut lui associer $\omega\circ\iota$ et $\omega\circ j$.\\
	Conversement, etant donne $\phi: G\to M$ et $\psi: H\to M$, montrons que $\exists! \omega: G\ast H \to M$ tel que $\omega\circ\iota= \phi$ et $\omega\circ j = \psi$.\\
	Pour l'existence, on definit $\tilde\omega: F( x_\alpha, y_\gamma) \to M$.\\
	Comme $\phi( r_\beta) = 1 =\psi( s_\delta) $, $\tilde\omega$ passe au quotient et induit une application  $\omega: G\ast H\to M$.\\
	L'unicite de $\omega$ suit du fait que ce soit une colimite.
\end{proof}
\begin{rmq}
On definit de la meme facon un produit libre d'un nombre arbitraire de groupes.
\end{rmq}
\subsection{Pushouts de groupes}
Soient $\alpha:K\to G$ et $\beta:K\to H$ deux homomorphismes, on veut construire le pushout de $G\leftarrow K \rightarrow G $ 
\begin{defn}[Pushout]
	Le pushout ou amalgame $G \ast_K H$ est le quotient de $G\ast H$ par le sous-groupe normal genere par les elements de la forme $\iota( \alpha( x) ) j( \beta( x) ^{-1} )$ 
\end{defn}
On appelle aussi $i$ et $j$ les compositions $G\to G\ast H \to G\ast_K H$ et $H \to G\ast H \to G\ast_K H$.\\
On a ainsi un carre commutatif
\[\begin{tikzcd}
	K & G \\
	H & {G\ast_K H}
	\arrow["\beta"', from=1-1, to=2-1]
	\arrow["\alpha", from=1-1, to=1-2]
	\arrow["\iota", from=1-2, to=2-2]
	\arrow["j"', from=2-1, to=2-2]
\end{tikzcd}\]
\begin{propo}
Le pushout est un pushout.
\end{propo}
\begin{proof}
\[\begin{tikzcd}
	K & G \\
	H & {G\ast_K H} \\
	&& M
	\arrow["\beta"', from=1-1, to=2-1]
	\arrow["\alpha", from=1-1, to=1-2]
	\arrow["\iota", from=1-2, to=2-2]
	\arrow["j"', from=2-1, to=2-2]
	\arrow["\psi", bend right, from=2-1, to=3-3]
	\arrow["\phi", bend left, from=1-2, to=3-3]
	\arrow["{\exists! \omega}", from=2-2, to=3-3]
\end{tikzcd}\]
On construit $\tilde\omega: G\ast H \to M$ par la propriete universelle du produit libre ( a l'aide de $\phi$ et $\psi$ ).\\
Cet homomorphisme  $\tilde \omega$ passe au quotient parce que  $\tilde\omega( \iota( \alpha( x) ) j( \beta( x) ) ^{-1}) = \psi( \alpha( x) ) \psi( \beta( x) ) ^{-1}= 1$ et on a une application induite $\omega: G\ast_K H \to M$.\\
L'unicite est immediate.
\end{proof}
\section{Seifert-van Kampen}
On souhaite calculer le groupe fondamental d'un pushout d'espaces et l'identifier.
\subsection{Groupe fondamental d'un recollement}
Soient $A,B \subset X$, $X= A \cup B$, deux sous-espaces ouverts tels que $C= A\cap B$ est connexe par arcs.\\
On choisit $x_0\in C$ comme point de base pour $C,A,B$ et $X$.\\
On appelle $\iota: A \subset X, j: B\subset X, \alpha:C \subset A, \beta: C \subset B$.\\
On obtient alors un pushout:
\[\begin{tikzcd}
	C & A \\
	B & X
	\arrow["\alpha", from=1-1, to=1-2]
	\arrow["\beta"', from=1-1, to=2-1]
	\arrow["j"', from=2-1, to=2-2]
	\arrow["\iota", from=1-2, to=2-2]
\end{tikzcd}\]
On va montrer que
\[\begin{tikzcd}
	{\pi_1(C,x_0)} & {\pi_1(A,x_0)} \\
	{\pi_1(B,x_0)} & {\pi_1(X,x_0)}
	\arrow["{\alpha_*}", from=1-1, to=1-2]
	\arrow["{\beta_*}"', from=1-1, to=2-1]
	\arrow["{j_*}"', from=2-1, to=2-2]
	\arrow["{\iota_*}", from=1-2, to=2-2]
\end{tikzcd}\]
est un pushout.\\
Par la propriete universelle du pushout, ce carre nous fournit $\phi:\pi_1( A,x_0) \ast_{ \pi_1( A,x_0) } \pi_1( B,x_0) \to \pi_1 ( X,x_0) $ 
\begin{lemma}
$\phi$ est surjectif.
\end{lemma}
\begin{proof}
Soit $\gamma: I \to X$ un lacet base en $x_0$.\\
Le recouvrement de $X$ par $A$ et $B$ donne un recouvrement ouvert $\gamma^{-1}( A) ,\gamma^{-1}( B) $ de l'intervalle $I$, un espace metrique compact.\\
donc il existe un nombre de lebesgue $\delta>0$ tel que tout sous-ensemble de $I$ de diametre $< \delta$ est contenu dans $\gamma^{-1}( A) $ ou $\gamma^{-1}( B) $ ( ou les deux.)\\
On choisit donc $n > \frac{1}{\delta}, n \in \mathbb{N}$ de sorte que $\gamma_{[\frac{k}{n}, \frac{k+1}{n}]} $ est un chemin dans $A$  ou dans $B$.\\
Pour alternet les images dans $A$ et dans $B$, on concatene les chemins qui se suivent dans le meme ouvert pour choisir $s_0=0 < s_1= \frac{k_1}{n} < \ldots < \frac{k_r}{n} = 1 $ de telle sorte que $\gamma$ envoie $ [ s_0,s_1] $ dans ( disons) $A$ , $ [ s_1,s_2] $ dans $B$ etc.\\
On definit $\gamma_i = \gamma|_{ [ s_{i-1} , s_i] } $.\\
Comme $C$ est connexe par arcs, il existe des chemins $\gamma^{i}$ dans $C$, allant de $x_0$ a $\gamma( s_i) $.\\
On decompose a homotopie pres, le chemin $\gamma$ en concatenation de lacets ( d'abord des chemins) .\\
\begin{align*}
	\gamma &\simeq \gamma_1 \star \gamma_2 \star\ldots \star \gamma_r \\
	&\simeq \gamma_1 \star \overline{ \gamma^{1} } \star \gamma^{1} \star\ldots\\
	&\simeq \underbrace{( \gamma_1 \star \overline { \gamma^{1}} ) }_{ \omega_1	}\star \underbrace{( \gamma^{1}\star \gamma_2 \star\overline{  \gamma^{2} })}_{ \omega_2} \star\ldots
\end{align*}
ou $\omega_1$ est un lacet base en $x_0$ et entierement contenu dans $A$ ou dans $B$.\\
Alors $ [ \gamma] = [ \omega_1] . \ldots . [ \omega_r] = i_* ( \omega_1) j_*( \omega_2) \ldots$ 

\end{proof}
Pour montrer que $\phi$ est un isomorphisme, on cherche a identifier le noyau de $\pi_1A \ast \pi_1B \to \pi_1 A \ast_{\pi_1} C \pi_1B \to \pi_1X$.\\
Soit $\gamma$ la concatenation de lacets $\gamma_1\ast \ldots\ast \gamma_r$ , avec $\gamma_{2i+1} $  dans $A$, et $ \gamma_{2i}  $ dans $B$.\\
On suppose que $[ \gamma ]=1$ dans $\pi_1X$, ie. que $\gamma\simeq c_{x_0} \iff \exists H : I^{2} \to X $ entre $\gamma$ et $c_{x_0} $.\\
Par le meme argument de nombre de lebesgue, on decoupe le carre en rectangles que $H$ envoie entierement dans $A$ ou dans $B$.





\end{document}	
