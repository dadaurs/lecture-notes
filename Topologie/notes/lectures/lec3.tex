\documentclass[../main.tex]{subfiles}
\begin{document}
\lecture{3}{Mon 28 Feb}{Groupes topologiques}
\begin{crly}
Soit $A \subset X$ un sous-espace compact d'un espace $X$ separe. Alors le collapse $ \frak X A$ est separe.
\end{crly}
\begin{proof}
Il suffit de verifier les proprietes du theoreme.\\
Soit $\overline{x} \in \frak X A$.\\
Si $x\in A, q^{-1}( x) =A$ est compact. Si $x\notin A, q^{-1}( \overline{x})= \left\{ x \right\}  $ qui est compact.\\
Soit $F$ un ferme de $X$, alors si $F\cap A= \emptyset$, on a que $q^{-1}( q( F) ) = F$ ferme, sinon $F\cap A \neq \emptyset$ et alors
\[ 
q^{-1}( q( F) ) = F\cup A
\]
Comme $A$ est compact et $X$ separe, alors $A$ ferme.
\end{proof}
\begin{exemple}
Soit $\sim$ une relation d'equivalence sur $ \mathbb{R}^{2}$ defini par $ ( x,y) \sim ( x',y') \iff ( x-x',y-y') \in \mathbb{Z}^{2}$.\\
Alors 
\[ 
\frak { \mathbb{R}^{2}} { \sim} 
\]
est un tore, separe, or la proposition ne s'applique pas car $ q^{-1}( 0,0) = \mathbb{Z}^{2}$.
\end{exemple}
\begin{defn}[Espaces projectifs]
	L'espace projectif reel $ \mathbb{R}P^{n}$ est le quotient de $ S^{n}$ par la relation antipodale $ x\sim y \iff x= \pm y$ pour $x,y \in S^{n} \subset  \mathbb{R}^{n+1}$ 
\end{defn}
\begin{exemple}
	\begin{itemize}
	\item 
	
$ \mathbb{R}P^{0}= \frak { S^{0} } { \sim} = \ast$, $ \mathbb{R}P^{1}= \frak { S^{1} } { \sim} \simeq S^{1}$.\\
\item De plus $ \mathbb{R}P^{2}= S^{2} /\sim$ est le plan projectif 
	\end{itemize}
\end{exemple}
\begin{propo}
$ \mathbb{R}P^{n}$ est compact et separe
\end{propo}
Suit immediatement des propositions.\\
L'analogue complexe donne
\begin{defn}[Espace projectif complexe]
	L'espace projectif complexe $ \mathbb{C}P^{n}$ est le quotient de $S^{2n+1}\subset \mathbb{C}^{n+1}$ par la relation $x\sim y \iff \exists \alpha \in S^{1}$ tel que $x= \alpha y$.
\end{defn}
De meme, pour les quaternions $ \mathbb{H}$, on peut definir $ \mathbb{H}P^{n}$, pour les octonions on peut construire $ \mathbb{O}P^{0}, \mathbb{O}P^{1}\simeq S^{8}, \mathbb{C}P^{2}$ 
\subsection{Quotients par des actions de groupe}
\begin{defn}[Groupe topologique]
	Un groupe topologique est un groupe $G$ tel que les applications de multiplication $\mu: G\times G \to G$ et l'inverse $\iota: G\to G$ sont continues.
\end{defn}
Tout groupe peut etre vu comme un groupe topologique discret.
\begin{exemple}
Le cercle unite $S^{1} \subset  \mathbb{C}$ muni de la multiplication complexe est un groupe topologique
\end{exemple}
\begin{rmq}
Les seules spheres qui sont des groupes topologiques sont $S^0, S^1, S^3 $ 
\end{rmq}
\begin{lemma}
Si $H <G$ est un sous-groupe d'un groupe topologique $G$, la topologie induite en fait un groupe topologique.
\end{lemma}
\begin{defn}
	Une action d'un groupe topologique $G$ sur un espace $X$ est une application $\mu: X\times G \to X$ telle que 
	\[ 
	\mu( x,1_G) = x\forall x \in X \text{ et } \mu( x,gg') =\mu(  \mu( x,g), g'  ) 
	\]

	
\end{defn}
\begin{defn}
	Soit $\mu$ une action de $G$ sur $X$, l'espace des orbites $ \frak X G$ et l'espace quotient de $X$ par la relation $x\sim y\iff \exists g\in G $ tel que $y= \mu( x,g) $ 	
\end{defn}
\begin{rmq}
Si $H < G$ est un groupe topologique, alors $H$ agit sur $G$ par multiplication a droite et $ \frak G H$ est l'espace des orbites $gH$. Si $H$ est un sous-groupe normal, ce quotient est un groupe.
\end{rmq}
\begin{propo}
Soit $\mu$ une action d'un groupe topologique $G$ sur un espace $X$, alors
\begin{enumerate}
\item $q: X\to \frak X G$ est ouverte
\item Si $X$ est compact, le quotient est compact
\item Si $X$ et $G$ sont compact et separe, alors $ \frak X G$ aussi.
\end{enumerate}
\end{propo}
\begin{proof}
Soit $U \subset X$ ouvert, $q( U) $ est ouvert car $q^{-1}( q( U) )= \bigcup_{g\in G} U\cdot g $ et $U\cdot g$ est ouvert car la translation est continue et est meme un homeomorphisme.\\
La propriete 2 est immediate.\\

On considere $X\times X\times G \to X\times X$ en envoyant $( x,y,g) \mapsto ( x,yg) $, cette application est continue.\\
Le graphe $\Gamma$ de la relation definie par $\mu$ est l'image de $\Delta \times G$.\\
Comme $X$ est separe, $\Delta$ est ferme donc compact et $G$ est compact.\\
Ainsi $\Gamma$ est compact dans $X\times X$ separe donc $\Gamma$ est ferme.\\
Soient $xG$ et $yG$ deux orbites differentes, ie. $( x,y) \notin \Gamma$.\\
Il existe donc des ouverts $x\in U, y\in V$ tel que $U\times V\cap \Gamma = \emptyset$.\\
Comme $q$ est ouverte, $q( U), q( V)  $ sont des voisinages ouverts des orbites $xG$ et $yG$ respectivement. On conclut en remarquant que ces images sont disjointes.\\
Sinon on aurait $zG$ commun, ie. $zg\in U, zg'\in V$ pour $g,g' \in G$ et alors $ ( zg, zgg^{-1}g') \in \Gamma \cap ( U\times V) $ 
\end{proof}
\subsection{ $SO( n) $ }
\begin{propo}
Soit $G$ compact et $X$ separe.
Soit $\mu$ une action transitive de $G$ sur $X$.\\
Alors, si $G_x  $, alors
\[ 
	\frak G G_x = X
\]
pour tout $x\in X$.
\end{propo}
\begin{proof}
On definit $\mu_x:G\to X$ envoyant $g\mapsto xg$, continue.\\
On observe que $\mu_x$ envoie $G_x$ sur $x$ et par transitivite, $\mu_x$ est surjective.\\
Par la propriete universelle du quotient, $\mu_x$ passe au quotient.\\
$\bar\mu_x$ est une bijection continue. C'est un homeo car $ \frak G G_x$ est compact, $X$ separe. 
\end{proof}



	




\end{document}	
