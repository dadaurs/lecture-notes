\documentclass[../main.tex]{subfiles}
\begin{document}
\lecture{13}{Mon 25 Apr}{Consequence de Seifert}
\begin{propo}
Soit $f: A\to X$ une application pointee avec $A$ connexe par arcs.\\
Soit $Y= X\cup_f CA$, alors $\pi_1 Y \simeq \pi_1X \ast_{\pi_1 A} 1$ 
\end{propo}
\begin{proof}
$Y$ est recouvert par $q( X) $ et $q( CA) $, mais ce ne sont pas des ouverts de $Y$, il faut les epaissir.\\
On pose $X' = q\left(  X \coprod Col( A) \right) $ et $C'A = A \times ] \frac{1}{4},1 ] / A\times 1$ un "petit" cone ouvert.\\
On voit que $C' A \simeq \ast$ ( comme $CA$ ) et $ C'A$ et $X'$ sont  des ouverts de $Y$ car ce sont des images par $q$ d'ouverts satures.\\
De plus, $X'$ admet $X$ comme retracte de deformation fort.\\
Enfin, $X'\cap C' A = q\left( Col( A) \cap ( A \times ] \frac{1}{4},1])/ A\times 1\right) = q\left( A\times ] \frac{1}{4}, \frac{3}{4}\right) \simeq A $.\\
On peut donc appliquer le theoreme de Seifert van Kampen car $A$ est connexe par arcs.\\
Pour conclure, on affirme que $ j: C'A \cap X' \hookrightarrow X'$ induit $f_\ast: \pi_1 A \to \pi_1X$.\\
On considere  $ X' \hookleftarrow A' \times ]\frac{1}{4}, \frac{3}{4}[ \to A'$ 
\end{proof}
\subsection{Attachement de cellules standard}
Si $ Y = X\cup_f e^{n}$ Comme $\pi_1 S^{n-1}= 1 $, pour $n \geq 3$ on a 
\begin{crly}
Si $n \geq 3$, $\pi_1 Y \simeq \pi_1 X$.\\
Si $ n=2$, $\pi_1 Y\simeq \faktor{ \pi_1 X}{N_f}$ ou $N_f$ est le sous-groupe normale engendre par $f_\ast ( 1) $ ou $f_\ast: \mathbb{Z} \simeq \pi_1 S^{1}\to \pi_1 X	$ qui envoie $1 \mapsto f_\ast( 1) $ 
\end{crly}
\begin{proof}
Par la proposition on a un pushout de groupes $ \pi_1 X \leftarrow \pi_1 S^{1} \rightarrow \pi_1( CS^{1}) $ qui est $\pi_1Y$, donc $\pi_1  Y \simeq \pi_1 X \ast_{\pi_1 S^{1}} 1\simeq \faktor { \pi_1 X} { N_f} $ 
\end{proof}
Il reste a etudier les attachements de $1$ cellule.\\
Il y a deux cas pour $f: S^{0}\to ( X,x_0 )$.\\
SI $f( -1) $ et $x_0$ ne sont pas dans la meme composante connexe, alors $\pi_1 Y= \pi_1( X,x_0) \ast \pi_1 ( X,f( -1) ) $ .\\
Si $x_1$ et $x_0$ sont dans la meme composante connexe, alors $f$ est homotope a l'application $g = c_{x_0} $ via $\gamma$ un chemin entre $x_0$ et $x_1$.\\
Alors $X\cup_f e^{1}\simeq X\cup_g e^{1}$ ( puisque des applications homotopes donnent des attachements homotopes) or puisque $g$ est constante, ceci est homotope a $X\vee e^{1}$.\\
Ainsi, si $X$ est bien pointe, alors $\pi_1Y \simeq \pi_1X \ast \mathbb{Z}$.
\subsection{Classification des surfaces}
Rappel: Une surface est un espace compact, hausdorff et localement homeomorphe a $ \mathbb{R}^{2}$.\\
On va supposer connu que toute surface est triangularisable, ie. $S$ est un quotient d'une reunion disjointe finie de triangles en identifiant uniquemenet des paires de cotes, a homeomorphisme pres.\\
On va supposer que $S$ est le quotient d'un polygone $P$ a $2k$ cotes $ a_1',\ldots,a_k', a_1'', \ldots, a_k''$ ou $a_i'$ et $a_i''$ sont identifies deux a deux, et tous les sommets sont identifies a un point.\\
On reconnait dans cette description la somme connexe de $i$ copies de $ \mathbb{R}P^{2}$ et de tores.
\begin{propo}
Si $S$ est le quotient de $P$ par une relation donnee par un  mot $w,$ $P$ un $2k$-gone et $S'$ est le quotient de $P$ par une relation donee par un mot $ w'$ , $ P'$ un $2l$ gone, alors $S\# S'$ est le quotient d'un $( 2k+2l) $ gone par la relation donnee par $ww'$
\end{propo}
\begin{proof}
On appelle $A_0, \ldots, A_{2k-1} $ les sommets de $P$ et on choisit un voisinage $U$  d'un point interieur dont le bord ne recontre $\del P$ qu'en $A_0$.\\
De memme pour $P'$ avec $U'$ et $\del U \cap P' = \left\{ A_0' \right\} $.\\
Comme $ P\setminus  U$  est le quotient d'un $( 2k+1) $-gone dont les cotes sont $a_1', a_1''\ldots, a_k', a_k''$ et $b$ entre $ B_0$ et $A_0$.\\
Ici, $B_0$ est la deuxieme extremite de $a_k''$.\\
La somme connexe est le quotient d'un $( 2k+1) $-gone et un $ ( 2l+1) $-gone, vu que le quotient d'un quotient est un quotient, $ S\# S'$ est construite en identifiant d'abord $b$ et $b'$ de sorte a obtenir un $( 2k+2l) $-gone puis en identifiant les cotes 2 a 2 selon les instructions donnees par le mot parcouru de $A_0$ sur le bord dans le sens trigonometrique.\\
C'est bien la concatenation $ww'$.
\end{proof}
 


\end{document}	
