\documentclass[11pt, a4paper]{article}
\usepackage[utf8]{inputenc}
\usepackage[T1]{fontenc}
\usepackage[francais]{babel}
\usepackage{lmodern}
\usepackage{amsmath}
\usepackage{amssymb}
\usepackage{amsthm}
\usepackage{faktor}
\usepackage{tikz}
\usepackage{tikz-cd}
\usepackage{mathtools}
\DeclareMathOperator{\cl}{cl}
\DeclareMathOperator{\id}{Id}

%\renewcommand{\vec}[1]{\overrightarrow{#1}}
\newcommand{\del}{\partial}
\DeclareMathOperator*{\sgn}{sgn}
\DeclareMathOperator*{\id}{Id}
\DeclareMathOperator*{\im}{Im}
\DeclareMathOperator*{\re}{Re}
\DeclareMathOperator*{\vol}{Vol}
\newcommand\norm[1]{\left\vert#1\right\vert}
\newcommand\ns[1]{\left\vert\left\vert\left\vert#1\right\vert\right\vert\right\vert}
\newcommand\Norm[1]{\left\lVert#1\right\rVert}
\newcommand\N[1]{\left\lVert#1\right\rVert}
\newcommand\abs[1]{\left\vert#1\right\vert}
\newcommand\inj{\hookrightarrow}
\newcommand\surj{\twoheadrightarrow}
\newcommand\ded[1]{\overset{\circ}{#1}}
\newcommand\sidenote[1]{\footnote{#1}}
\newcommand\eng[1]{\left\langle#1\right\rangle}
\newcommand\hr{
    \noindent\rule[0.5ex]{\linewidth}{0.5pt}
}

\newcommand{\incfig}[1]{%
    \def\svgwidth{\columnwidth}
    \import{./figures}{#1.pdf_tex}
}
\newcommand{\filler}[1][10]%
{   \foreach \x in {1,...,#1}
    {   test 
    }
}

\newcommand\contra{\scalebox{1.5}{$\lightning$}}
\makeatother
\def\@lecture{}%
\newcommand{\lecture}[3]{
    \ifthenelse{\isempty{#3}}{%
        \def\@lecture{Lecture #1}%
    }{%
        \def\@lecture{Lecture #1: #3}%
    }%
    \subsection*{\@lecture}
    \marginpar{\small\textsf{\mbox{#2}}}
}


\begin{document}
\title{Série 1}
\author{David Wiedemann, Nino Courtecuisse, Matteo Mohammedi}
\maketitle
\section*{1}
On montre la double implication.\\

$\Longleftarrow$\\
Pour montrer que $p$ est une application quotient, il suffit de montrer que $F \subset B$ est fermé si et seulement si $p^{-1}( F) $ est fermé .\\
Puisque $p$ est continue (c'est la composition de $q$ avec l'inclusion $ A \hookrightarrow X$), si $F$ est fermé alors $p^{-1}( F) $ est fermé.

De plus, si $p^{-1}( F) $ est fermé, alors c'est un ensemble fermé saturé et par hypothèse il existe un fermé saturé $ E \subset X$ tel que $E\cap A = p^{-1}( F )$ d'ou $q( E) \cap B = F$. Dès que $E$ est un fermé saturé, $q( E) $ est un fermé et ainsi $F$ est fermé .\\

$\Longrightarrow$\\
Supposons maintenant que $p$ est un quotient, soit $ F \subset A$ un fermé $p$-saturé.\\
Lorsque qu'on restreint \(p=q_{|A} : A\to B\) on a les topologies de sous-espace sur \(A\) et \(B\). Donc si on suppose que \(p\) est un quotient, alors la topologie quotient de \(p\) sur \(B\) coincide avec la topologie de sous-espace \(B\subset Y\). \\
Ainsi si \(F\subset A\) est un fermé \(p\)-saturé, on a \(F=p^{-1}(p(F))\) qui est fermé et donc par définition de la topologie quotient \(p(F)\subset B\) est fermé. Comme la topologie de sous-espace \(B\subset Y\) coincide, il existe un fermé \(E\subset Y\) tel que \(p(F) = B\cap E\). \\
Ainsi \(F = p^{-1}(B\cap E) = A\cap q^{-1}(E)\) et \(q^{-1}(E) \subset X\) est fermé car \(q\) est continue et \(q\)-saturé car \(q\) est surjective. \\


\section*{2}
Comme indiqué sur piazza, on supposera que l'application $p$ est un quotient, sinon l'énoncé est faux en prenant le contre exemple $ X = \mathbb{R}, A = [ 0,1) $ et $x\sim y \iff x-y \in \mathbb{Z}$.\\

Soit $A \subset X$ comme dans l'énoncé.\\
Soit $\sim$ la relation d'équivalence sur $X$, on notera $\sim'$ la relation d'équivalence induite sur $A$.\\
On notera $\iota: A \hookrightarrow X$ l'inclusion et  $q_A: A \to \faktor A { \sim' }$, $q_X: X \to \faktor X \sim$ les applications canoniques.\\
On montre le resultat en deux temps, on montrera que
\begin{itemize}
\item $q_X\circ \iota$ passe au quotient de $q_A$ et induit une application $g: \faktor A { \sim' }\to \faktor { X} { \sim} $
\item L'application $q_A$ passe au quotient de $q_X\circ \iota$ 
\end{itemize}
et on conclura.
\subsection*{ $q_X \circ\iota$ passe au quotient de $q_A$ }
En effet, si $ a\sim' b\in A$, on a que $ q_X\circ \iota( a) = q_X( a) = q_X( b)  $ car $\sim'$ est la restriction de $\sim$, ainsi on a une application induite
\[\begin{tikzcd}
	A & {\faktor {X} {\sim}} \\
	{\faktor {A} {\sim'}}
	\arrow["{q_A}"', from=1-1, to=2-1]
	\arrow["{q_X\circ\iota}", from=1-1, to=1-2]
	\arrow["{\exists ! f}"', dashed, from=2-1, to=1-2]
\end{tikzcd}\]
\subsection*{ $q_A$ passe au quotient de $q_X\circ \iota$ }
Remarquons que $q_X \circ \iota= p$ et est donc par hypothese une application quotient.\\

On a bien que si $p ( a) = p( b) $, alors $a\sim b \iff a\sim' b \iff q_A( a) = q_A( b) $ et on a une deuxieme application induite 
\[\begin{tikzcd}
	A & { \faktor {A} {\sim'}} \\
	{\faktor {X} {\sim}}
	\arrow["{q_X\circ\iota}"', from=1-1, to=2-1]
	\arrow["{q_A}", from=1-1, to=1-2]
	\arrow["{\exists ! g}"', dashed, from=2-1, to=1-2]
\end{tikzcd}\]

Finalement, en composant les diagrammes, on obtient

\[\begin{tikzcd}
	A & { \faktor {A} {\sim'}} \\
	{\faktor {A} {\sim'}}
	\arrow["{q_A}"', from=1-1, to=2-1]
	\arrow["{q_A}", from=1-1, to=1-2]
	\arrow["{g\circ f}"', dashed, from=2-1, to=1-2]
\end{tikzcd}\]
\[\begin{tikzcd}
	A & {\faktor {X} {\sim}} \\
	{\faktor {X} {\sim}}
	\arrow["{q_X\circ\iota}"', from=1-1, to=2-1]
	\arrow["{q_X\circ\iota}", from=1-1, to=1-2]
	\arrow["{f\circ g}"', dashed, from=2-1, to=1-2]
\end{tikzcd}\]


Dès que \(Id_{\faktor A { \sim '}}\) et \(Id_{\faktor A { \sim '}}\) font aussi respectivement commuter les diagrammes, on a par l'unicité de la propriété universelle, que \(g\circ f = Id_{\faktor A { \sim '}}\) et \(f\circ g = \id_{ \faktor X \sim}\). 

\section*{3}
Soit \(q:\mathbb{R}\to\mathbb{R}/\mathbb{Z}\) le quotient considéré où ici \(\mathbb{Z}\) agit par translation sur \(\mathbb{R}\), ie \(\mathbb{R}/\mathbb{Z} = \mathbb{R}/\sim\) avec \(\sim\) la relation définie par \(x\sim y \iff x-y\in\mathbb{Z}\) pour \(x,y\in\mathbb{R}\). \\
On note $\sim '$ la relation restreinte a $I$ et \(p=q_{|I}:I\to \mathbb{R}/\sim\) la restriction de \(q\). On a clairement que $\sim '$ identifie les points $0$ et $1$ et donc $\sim '$ est la même relation d'équivalence que décrite dans l'énoncé. \\
On vérifie donc les deux hypotheses de la partie 2 de l'exercice :
\begin{itemize}
\item D'abord $q_{\vert_I}$ est bien surjectif. En effet, soit $x\in \mathbb{R}$, alors $x- \lfloor x\rfloor\in I$ et $ x- \lfloor x \rfloor \sim x$ .
\item Montrons que \(p\) est une application quotient en appliquant le critère de la partie 1. \\
Soit $ F\subset I$ un fermé \(p\)-saturé, on prétend que la \(q\)-saturation de $F$ dans $ \mathbb{R}$ reste un fermé. On aura alors \(F = q^{-1}(q(F))\cap I\) avec \(q^{-1}(q(F))\subset \mathbb{R}\) un fermé \(q\)-saturé, ce qui impliquera par la partie 1. que \(p\) est un quotient. \\
On distingue deux cas : 

\subsection*{Si $ 0 \in F$ }
Alors \(1\in F\) car \(F\) est \(p\)-saturé et \(1\sim'0\). \\
Soit \(x\in q^{-1}(q(F))^c\). Alors par surjectivité de \(p\), il existe \(a\in I\) tel que \(a\sim x\). Alors en particulier, \(a\not\in F\) et, dès que \(F\subset I\) est fermé, il existe un ouvert \(U\subset I\) tel que \(a\in U \subset F^c\). \\
Dès que \(0,1\notin F^c\) on a \(U\subset (0,1)\) et donc \(U\) est aussi ouvert dans \(\mathbb{R}\). Comme \(q\) est un quotient par action de groupe, \(q\) est ouverte et donc \(q(U)\) est ouvert et \(q^{-1}(q(U))\subset \mathbb{R}\) aussi par continuité. Or \(a\sim x\) implique \(x\in q^{-1}(q(U))\) et par construction \(q^{-1}(q(U)) \subset q^{-1}(q(F))^c\). Donc \(q^{-1}(q(F))^c\) est ouvert et \(q^{-1}(q(F)) \subset \mathbb{R}\) est fermé.  \\
\subsection*{ Si $ 0\notin F$ }
%Alors \(1\notin F\) car \(F\) est \(p\)-saturé et \(1\sim'0\). Donc \(F\) est aussi fermé dans \(\mathbb{R}\). Alors la \(q\)-saturation \(q^{-1}(q(F)) = (F+\mathbb{Z})\) est aussi fermée dans \(\mathbb{R}\). 
	Soit $ b \in q^{-1}( q( F) )^{c} $, si $ b \notin \mathbb{Z}$, on considère le même ouvert $U$ que ci-dessus et on le choisit disjoint de $ 0$ et de $1$ (ce qui est toujours possible puisque $ F \cup \left\{ 0,1 \right\} $  reste un fermé).\\
	Si $ b \in \mathbb{Z}$, alors on choisit deux ouverts $U$ et $V$ de $I$ disjoints de $F$ tel que $ 0\in U$ et $ 1\in V$.\\
	Il est alors clair que la saturation de $U\cup V$ dans $ \mathbb{R}$ reste un ouvert qui sera disjoint de $F$.
\end{itemize}

\noindent Ainsi par la partie 1, on deduit que $ p$ est un quotient et ainsi on peut appliquer le critère établi en 2 pour conclure que $\faktor  { \mathbb{R} } { \sim} = \faktor { I} { \sim '} = \faktor { I} { \left\{ 0,1 \right\}  } = S^{1}$. 




\end{document}
