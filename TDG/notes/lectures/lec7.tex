\documentclass[../main.tex]{subfiles}
\begin{document}
\lecture{7}{Sat 16 Oct}{Produits et Coproduits}
\subsection{Produits et Coproduits}
Dans $\ens$, on a les constructions suivantes:\\
$\forall f: X \to Y, g: X \to Z\in \mor \ens$ 
\[ 
\exists ! h : X \to Y\times Z
\]
tel que $\pr_Y\circ h =f, \pr_z\circ h =g$.\\
De meme
$\forall f: X \to Y, g: X \to Z\in \mor \ens$ 
\[ 
\exists! h: X\cop Y \to Z 
\]
tel que
\[ 
h\circ i_x = f, h \circ i_y = g
\]
\hr\\
Formellement, dans une categorie quelconque
\begin{defn}
Soit $C$ une categorie, et soient $b,c \in \ob C$. Un produit de $b$ et $c$ consiste en un objet $a$ de $C$ et de deux morphismes $p: a\to b$ et $q: a\to c$ tel que pour tout couple de morphisme $f: d\to b$ et $g: d\to c$ il existe un unique morphisme $h: d\to a$ tel que $p\circ h = f$ et $q\circ h =g$.	
\end{defn}
\begin{rmq}
En general, le produit de deux objets n'existe pas, mais s'il existe, il est unique a isomorphisme pres.
\end{rmq}
\begin{lemma}
Soit $C$ une categorie, et soient $b,c \in \ob C$ . Si $ b \oal { p} a \oar q c$ et $b \oal p' a \oar q' c$ sont des produits de $b$ et $c$, alors il existe un isomorphisme $h: a\to a'$ qui respecte les morphismes de projection.
\end{lemma}
\begin{proof}
Puisque $b \oal p a \oar q c$ est un produit de $b$ et $c$, la propriete universelle du produit nous dit qu'il existe un unique morphisme $h : a' \to a $.\\
Puisque $b \oal p' a' \oar q' c$ est un produit de $b$ et $c$, $\exists ! k : a\to a'$ tel que
\[ 
p = p' \circ k \text{ et } q= q'\circ k
\]
Montrons que $h$ et $k$ sont des isomorphismes mutuellement inverses.\\
On a que
\[ 
p \circ h \circ k = p' \circ k = p
\]
de meme, on a
\[ 
q\circ h \circ k = q
\]
L'unicite de la propriete universelle implique que $h\circ k = \id_A$ et $k\circ h = \id_{a'} $
\end{proof}
On introduit la notation pour ``le'' produit de $b,c \in \ob C$ ( s'il existe) est note
\[ 
b \oal p_1 b\times c \oar p_2 c
\]
ou parfois simplement $b \times c$.
\begin{defn}[Coproduit]
	Soit $C$ une categorie. Un coproduit de $b$ et $c$ est un objet $a$ et deux morphismes $i: b \to a$ et $j: c\to a$ tel que pour tout couple de morphismes $f: b \to d$ et $g:c \to d$ il existe un unqiue morphisme $h: a \to d$ tel que $h \circ i = f$ et $h \circ j = g$ ce que nous resumons par le diagramme suivant.
\end{defn}
\begin{lemma}
Soit $c$ une categorie, et soient $b,c \in \ob C$ Si $a$ et $a'$ sont des coproduits de $b$ et $c$, alors il existe un isomorphismes $h : a \to a'$ tel que $h \circ j = j', h\circ i = i'$ 
\end{lemma}
\begin{rmq}
	Soit $C$ une categorie, et soient $b,c \in \ob C$. Si $a$ est un produit de $b$ et $c$ dans $C$, alors $a$ est un coproduit dans $C^{op}$.
\end{rmq}
\subsection{Preservation des produits/coproduits}
\begin{propo}
	Soit $C: L \dashv R: D$ 
\begin{enumerate}
	\item Soient $b,c \in \ob C$, si $b\cop c$ existe, alors $L( b\cop c) $ est un coproduit de $L( b) $ et $L( c) $ dans $D$ .
	\item Soient $d,e \in \ob D$. Si le produit $d\times e $ existe, alors son image sous le foncteur $R$ est un produit de $R( d) $ et $R( e ) $ 
\end{enumerate}
\end{propo}
\begin{proof}
	Supposons que $b \oar i_1 b \cop c \oal i_2 c$ est un coproduit de $b$ et $c$ dans $C$, considerons son image sous $L: L( b) \oar { L( i_1) } L( b\cop c) \oal { L( i_2) } L( c) $.\\
	Pour montrer que ceci est un coproduit de $L( b) $ et $L( c) $, il faut verifier la propriete universelle.\\
	Soit alors un couple de morphismes $L( b) \oar f d \oal g L( c) $ dans $D$.\\
	A voir: $\exists ! h : L( b\cop c) \to d$ qui satisfait la propriete universelle.\\
	Observons que $f: L( b) \to d, g: L( c) \to d$ donnent $f^{\#}: b \to R( d) $ et $g^{\#}: c \to R( d) $.\\
	Par la propriete universelle du produit, $\exists! k : b \cop c \to R( d) $ tel que $k\circ i_2 = g^{\#}$ et $k\circ i_1= f^{\#}$.\\
	Le morphisme $k: b \cop c \to R( d) $ correspond a $k^{\flat}: L( b \cop c ) \to d$.\\
	On va montrer que $ k^{\flat}\circ L( i_1) = f$ et que $k^{\flat}\circ L( i_2) = g$.\\
\hr\\
On a $k^{\flat}= \epsilon_d \circ L( k) $.\\
De meme, on a $L( f^{\sharp}) = L( k) \circ L( i_1) $.\\
Il suffit donc de montrer que $f= \epsilon_d \circ L( f^{\sharp}) $.\\
Cependant, $f^{\sharp}= R( f) \circ \eta_b$ , donc $L( f^{\sharp}) = LR( f) \circ L( \eta_b) $.\\
Reste a voir que $k^{\flat}$ est l'unique morphisme faisant commuter ces triangles.\\
Supposons qu'il existe $l: L( b\cop c) \to d$ faisant commuter le diagramme, montrons que $l= k^{\flat}$.\\
De maniere analogue, on trouve que $l^{\sharp}\circ i_1= f^{\sharp}$ et $l^{\sharp}\circ i_2= g$.\\
Par l'unicite de la propriete universelle du coproduit, on en deduit que $l^{\sharp}= k^{\sharp} \Rightarrow l = k$.
\end{proof}



			

\end{document}	
