\documentclass[../main.tex]{subfiles}
\begin{document}
\lecture{1}{Fri 10 Sep}{Introduction}
\section{Une Introduction à la Théorie des Catégories}
\subsection*{Notion Fondamentale: la composition}
\begin{itemize}
\item Composition d'applications
\item l'exemple fondamental d'un groupe est donné par $\aut ( X) $, où la multiplication du groupe est donnée par la composition d'automorphismes.
\end{itemize}
\subsection{Catégories}
\begin{defn}[Graphe dirigé]
	Un graphe dirigé $G$  consiste en un couple de classes $G_0$ et $G_1$, muni de deux applications
	\[ 
	\dom : G_1 \to G_0 \text{ et } \cod : G_1 \to G_0
	\]
	appelées domaine et codomaine.
	On pense à $G_0$ comme l'ensemble des sommests et $G_1$ l'ensemble des arêtes de $G$.
\end{defn}
Par exemple, si $x,y \in G_0, f \in G_1$ , alors 
\begin{align*}
	\dom ( f) =x,\quad \cod ( f) =y
\end{align*}
\[\begin{tikzcd}
	X && Y
	\arrow["f", from=1-1, to=1-3]
\end{tikzcd}\]
On introduit la notation
\[ 
	G( x,y) = \left\{ f \in G_1 | \dom ( f) = x , \cod ( f) =y \right\} 
\]
\begin{exemple}
	Soit $X$ un ensemble, et soit $R \subset X \times X$ une relation sur $X$. Alors $G_r=( X,R) $ est un graphe dirigé, où
	\[ 
		\dom: R \to X : ( x_1, x_2) \to x_1 \text{ et } \cod: R \to X: ( x_1, x_2) \to x_2
	\]
Observer que $\forall x_1, x_2 \in X$ 
\[ 
	G_R ( x_1, x_2) =
	\begin{cases}
		\left\{  ( x_1,x_2)  \right\}: ( x_1,x_2) \in R  \\
		\emptyset \text{ sinon } 
	\end{cases}
\]
\end{exemple}
\begin{defn}[Catégories]
	Une catégorie $C$ est un graphe dirigé $( C_0, C_1) $ muni d'applications de composition
	\[ 
		\gamma_{a,b,c} : C( a,b) \times C( b,c) \to C( a,c) : ( f,g) \to g\circ f
	\]
\begin{tikzcd}
A \arrow[r, "f"] \arrow[rr, "f\circ g", bend right=49] & B \arrow[r, "g"] & C
\end{tikzcd}
\begin{itemize}
	\item ( Existence d'identités ) Il existe une application $\id: C_0 \to C_1: c \to \id_c$ tel que
		\[ 
			f\circ \id_a = f = \id_b \circ f \forall f \in C_1( a,b) , \forall a,b \in C_0
		\]
	
	\item ( Associativité) Quelque soient $a,b,c,d \in C_0$ et $f \in C( a,b) , g \in C( b,c) $ et $h \in C( c,d) $ 
		\[ 
			( h \circ g) \circ f = h \circ ( g \circ f)  \in C( a,d) 
		\]

\end{itemize}
\end{defn}
\subsection*{Notation}
On note
\begin{align*}
C_0 = \ob C - \text{ les objets de $C$  } \\
C_1 = \mor C - \text{ les morphismes } 
\end{align*}
\begin{itemize}
\item Si $\ob C, \mor C	$  sont des ensembles, alors $C$ est petite.
\item Si $C( a,b) $ est un ensemble $\forall a, b \in \ob C$ , alors $C$ est localement petite.	
\end{itemize}










\end{document}	
