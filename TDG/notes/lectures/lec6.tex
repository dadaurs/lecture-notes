\documentclass[../main.tex]{subfiles}
\begin{document}
\lecture{6}{Sun 10 Oct}{Caracterisation des Adjonctions}
\subsection{Exemple concret d'adjonction}
On considere $L: \ens \to \vect_\mathbb{K}$ et $ U : \vect_{\mathbb{K}}\to \ens $.\\
Ces adjonctions verifient les identites triangulaires et on a une adjonction $ L \dashv U$.\\
Verifions les identites triangulaires.\\
Soit $ V\in \ob \vect_\mathbb{K}$ Considerer
\[ 
	U( V) \oar{\eta_{U( V) } } UL( UV) 
\]
et
\[ 
	U( LU( V) ) \oar { U( \epsilon( V) ) } U( V) 
\]
On veut voir que $U( \epsilon_V) \circ \eta_{U( V) }= \id_{U( V) }  $ .\\
Par definition de $\eta$ ,
\begin{align*}
	\eta_{U( V) } &\to UL( U( V) ) \\
	v &\mapsto ( \eta_{U( V) }( v) : U( V) \to \mathbb{K} ) 
\end{align*}
ou 
\[ 
	\eta_{U( V) } ( v) : V \to \mathbb{K}: v' \mapsto \delta_{v,v'} 
\]
Par ailleurs
\begin{align*}
	U( \epsilon_V ) : U( LU( V)) &\to U( V) \\
	\omega &\mapsto \sum_{v\in V} \omega( v) \cdot v
\end{align*}
Donc, $\forall v \in U( V) $,
\begin{align*}
	U( \epsilon_V) \circ \eta_{U( V) } ( v) &= U( \epsilon_V) ( \eta_{U( V) } ( v) ) \\
						&= \sum_{v'\in V} \eta_{U( V) } ( v) ( v') \cdot v'\\
						&= v
\end{align*}
Donc $U( \epsilon_V) \circ \eta_{U( V) } = \id_{U( V) } $.\\
\hr
Montrons l'autre egalite triangulaire.\\
Soit $X \in \ob \ens$. Considerons
\begin{align*}
	L( \eta_X) : L( X) &\to L( UL( X) )  \\
	\omega &\mapsto L( \eta_X) : UL( X) \to \mathbb{K}\\
	L( \eta_X) :\psi &\mapsto \sum_{x \in \eta_X^{-1}( \psi) } 
\end{align*}
Pour $\psi\in UL( X) $ ( donc $\psi:X \to \mathbb{K}$ ),
\[ 
	\eta_X^{-1}( \psi) =
	\begin{cases}
		\left\{ x' \right\} : \text{ si }  \psi= \eta_X( x') \\
		\emptyset \text{ sinon } 	
	\end{cases}
\]
donc $L( \eta_X)( \omega) : UL( X) \to \mathbb{K} $ 
\[ 
	\psi \mapsto \sum_{x \in \eta_X^{-1}( \psi) } = 
	\begin{cases}
		\omega( x') : \psi = \eta_X( x') \\
		0 : \psi \neq \eta_X( x') \forall x' \in X
	\end{cases}
\]
De plus
\begin{align*}
	\epsilon_{L( X) } : LU( L( X) ) &\to L( X) \\
	UL( X) \oar( \xi) \mathbb{K} &\mapsto \sum_{\psi \in UL( X) } \xi( \psi) \cdot \psi
\end{align*}
Faisons donc le calcul.\\
Soit $\omega\in L( X) $ 
\begin{align*}
	\epsilon_{L( X) } \circ L( \eta_X) ( \omega) &= \epsilon_{L( X) } ( L( \eta_X) ( \omega) ) \\
						     &= \sum_{\psi\in UL( X) } L( \eta_X) ( \omega) ( \psi) \cdot \psi\\
						     &= \sum_{x \in  X} \omega( x) \cdot \eta_X( x) \\
\end{align*}
Donc $\forall x' \in X$ 
\begin{align*}
	\epsilon_{L( X) } \circ L( \eta_X) ( \omega) ( x') &= \left( \sum_{x \in X}^{ }\omega( x) \eta_X( x)  \right)( x') \\
							   &= \sum_{x\in X} \omega( x) ( \eta_X( x) ( x') ) = \omega( x') 
\end{align*}
\subsection{Caracterisation des adjonctions}
\begin{propo}
Un couple de foncteurs $L:C\to D$ et $R : D \to C$ entre categories localement petites est une adjonction si et seulement si il existe un isomorphisme naturel entre les foncteurs
\[ 
	D( L( -) ,-) : C^{op}\times D	\to \ens : ( c,d) \to D( L( c) ,d) 		
\]
et
\[ 
	C( -,R( -) ) :C^{op}\times D\to \ens: ( c,d) \to C( c,R( d) ) 	
\]

\end{propo}
Nous demontrerons qu'il existe des transformations naturelles $\alpha: D( L( -) ,-) \to C( -,R( -) ) $ et $\beta:C(-,R( -)   ) \to D( L( -) ,-) $ qui sont mutuellement inverses.\\
On a donc besoin de deux applications 
\[ 
	\alpha: \ob ( C^{op}\times D) \to \mor \ens
\]
et 
\[ 
	\beta: \ob ( C^{op}\times D) \to \mor \ens
\]
tel que $\forall ( c,d) \in \ob ( C^{op}\times D) $ 
\[ 
	\alpha_{( c,d) } : D( L( c) ,d) \to C( c,R( d) ) 
\]
et 
\[ 
	\beta_{( c,d) } : C( c,R( d) ) \to D( L( c) ,d) 
\]
De plus, on veut que
\[ 
	\forall ( f^{op},g) \in C^{op}\times D ( ( c,d) ,( c',d') ) 
\]
\begin{align*}
	&D( L( c) ,d) \oar { \alpha_{c,d} } C( c,R( d) ) \oar { C( f^{op},R( g) ) } C( c',R( d') )  \\
	=& D( L( c) ,d) \oar { D( L( f^{op}) ,g) } D( L( c') ,d') \oar { \alpha_{( c',d') } } C( c',R( d') ) 
\end{align*}
et de meme pour l'application naturelle inverse

\begin{align*}
	&C(  c ,R( d )) \oar { \beta{c,d} } D( L( c) ,d)  \oar { D( L( f^{op} ), g ) 	 } D( L( c') ,d')   \\
	=& C( c,R( d) )	\oar { C( f^{op},R( g) )  } C( c',R( d') )  \oar { \beta_{( c',d') } } D( L( c') ,d') 
\end{align*}
Finalement, on soujaite egalement que $\alpha_{( c,d) } $ et $\beta_{ ( c,d) } $ sont des applications ensemblistes mutuellement inverses.\\
On va construire $\alpha$ et $\beta$ a partir des transformations naturelles $ \eta: \id_C \to RL$ et $\epsilon: LR\to \id_D$.\\
\begin{proof}
Supposer que $C\dashv D$ soit une adjonction avec transformation naturelle associees $\eta,\epsilon$.
\subsection*{Premier pas: Construction de $\alpha$ et $\beta$ }
Soit $( c,d) \in \ob ( C^{op}\times D) $ 
\[ 
	\alpha_{( c,d) } : D( L( c) ,d) \to C( c,R( d) ) 
\]
Soit $h: L( c) \to d \in D( L( c) ,d) $, notons qu'on a
\[ 
	c \oar { \eta_C} RL( c) \oar { R( h) } R( d) 
\]
Definissons donc $\alpha_{( c,d) } ( h) = R( h) \circ \eta_C$ \\
\hr
Soit $( c,d) \in \ob ( C^{op}\times D) $, on a alors $\forall k :c \to R( d) \in C( c,R( d) ) $ 
\[ 
	L( c) \oar { L( k) } LR( d) \oar { \epsilon_d} 
\]
Posons donc
\[ 
	\beta_{c,d} ( k) = \epsilon_d \circ L( k) 
\]

\subsection*{Naturalite}
Soit $( f^{op},g) : ( c,d) \to ( c',d') \in C^{op}\times D$.\\
Soit $h \in D( L( c) ,d) $, on a alors
\begin{align*}
	C( f^{op},R( g) ) \circ \alpha_{ ( c,d) } ( h) &= C( f^{op},R( g) ) ( R( h) \circ \eta_c) \\
						       &= R( g) \circ  ( R( h) \circ \eta_c) \circ f\\
						       &= R( g\circ h ) \circ \eta_c \circ f
\end{align*}
Dans l'autre sens, on a
\begin{align*}
	\alpha_{ ( c',d') } \circ D( Lf^{op},g) ( h) &= \alpha_{c',d'} ( g\circ h \circ L( f) ) \\
						     &= R( g \circ h \circ L( f) ) \circ \eta_c\\
						     &= R( g\circ h) \circ RL( f) \circ \eta_{c'} 
\end{align*}
Il faut donc montrer que $\eta_c\circ f = RL( f) \circ \eta_{c'} $ \\
Or $f:c'\to c$ donne 
\[ 
	RL( f) \circ \eta_{c'} = \eta_c \circ f
\]
commute car $\eta$ est une transformation naturelle.
De meme, le fait que $\epsilon$ soit une transformation naturelle implique que $\beta$ en est aussi une.
\subsection*{$\alpha$ et $\beta$ sont mutuellement inverses}
Considerer pour $ ( c,d) \in \ob ( C^{op}\times D) $.\\
On a 
\begin{align*}
	\beta_{( c,d) } \cdot \alpha_{ ( c,d) } ( h) &= \beta_{ ( c,d) } ( R( h) \circ \eta_c)  \\
						     &= \epsilon_d\circ L( R( h) \circ \eta_c) \\
						     &= \epsilon_d\circ LR( h) \circ L( \eta_c )
\end{align*}
On est en train de calculer
\begin{align*}
L( c) \oar { L( \eta_c) } LRL( c) \oar { LR( h) } LR( d) \oar { \epsilon_d} R\\
= L( c) \oar { L( \eta_c) } LRL( c) \oar { \epsilon_{L( c) }  } LR( d) \oar {h} R( d) \\
= L( c) \oar { \id_{L( c) } } \oar { h} R( d) =h
\end{align*}
Donc $h= \epsilon_d\circ LR( h) \circ L( \eta_c) $.\\
De meme l'autre identite triangulaire implique que $\alpha_{ ( c,d) } \circ \beta_{( c,d) } = \id_{C( c,L( d) ) } $.\\
Ainsi $\alpha$ et $\beta$ sont bien des isomorphismes naturels, mutuellements inverses.\\
\hr
Pour completer la caracterisation, il faudrait aussi montrer l'implication inverse.\\
Pour definir $\eta,\epsilon$ a partir de $\alpha,\beta$ 
\begin{itemize}
\item $\eta$ : Considere $\forall c \in \ob C$ ,
	\begin{align*}
	\alpha_{ ( c,L( c) )  } : D( L( c) ,L( c) ) \to C( c,RL( c) ) \\
	\id_{L( c) } \mapsto \alpha_{( c,L( c) ) } ( \id_{L( c) } ) 
	\end{align*}
	Et on definit alors $\eta_c: c \to RL( c) $ par $\eta_c = \alpha_{ ( c,L( c) ) } ( \id_{L( c) } ) $ 				
	
\item $\epsilon$ :Considerer $\forall d \in \ob D$ ,
	\begin{align*}
		\beta_{ R( d) ,d} : C( R( d) ,R( d) ) \to D( LR( d) ,d) \\
		\id_{R( d) } \mapsto \beta( R( d) ,d) ( \id_{R( d) } ) 	
	\end{align*}
		
\end{itemize}
			



				
\end{proof}

			



			



\end{document}	
