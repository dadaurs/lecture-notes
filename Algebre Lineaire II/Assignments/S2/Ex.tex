\documentclass[11pt, a4paper]{article}
\usepackage[utf8]{inputenc}
\usepackage[T1]{fontenc}
\usepackage[francais]{babel}
\usepackage{lmodern}
\usepackage{amsmath}
\usepackage{amssymb}
\usepackage{amsthm}
\renewcommand{\vec}[1]{\overrightarrow{#1}}
\newcommand{\del}{\partial}
\DeclareMathOperator*{\sgn}{sgn}
\DeclareMathOperator*{\id}{Id}
\DeclareMathOperator*{\im}{Im}
\DeclareMathOperator*{\re}{Re}
\DeclareMathOperator*{\vol}{Vol}
\newcommand\norm[1]{\left\vert#1\right\vert}
\newcommand\ns[1]{\left\vert\left\vert\left\vert#1\right\vert\right\vert\right\vert}
\newcommand\Norm[1]{\left\lVert#1\right\rVert}
\newcommand\N[1]{\left\lVert#1\right\rVert}
\newcommand\abs[1]{\left\vert#1\right\vert}
\newcommand\inj{\hookrightarrow}
\newcommand\surj{\twoheadrightarrow}
\newcommand\ded[1]{\overset{\circ}{#1}}
\newcommand\sidenote[1]{\footnote{#1}}
\newcommand\eng[1]{\left\langle#1\right\rangle}
\newcommand\hr{
    \noindent\rule[0.5ex]{\linewidth}{0.5pt}
}

\newcommand{\incfig}[1]{%
    \def\svgwidth{\columnwidth}
    \import{./figures}{#1.pdf_tex}
}
\newcommand{\filler}[1][10]%
{   \foreach \x in {1,...,#1}
    {   test 
    }
}

\newcommand\contra{\scalebox{1.5}{$\lightning$}}
\makeatother
\def\@lecture{}%
\newcommand{\lecture}[3]{
    \ifthenelse{\isempty{#3}}{%
        \def\@lecture{Lecture #1}%
    }{%
        \def\@lecture{Lecture #1: #3}%
    }%
    \subsection*{\@lecture}
    \marginpar{\small\textsf{\mbox{#2}}}
}

\begin{document}
\title{Question Ouverte 2}
\author{David Wiedemann}
\maketitle
Notons d'abord que, parce que $U^{T}U = \id$, la base  $u_1, \ldots, u_n$ forme une base orthonormale de $ \mathbb{R}^n$.\\
Pour simplifier, on notera
\[ 
D = \begin{pmatrix}
	\lambda_1 & & \\
		  & \ddots & \\
		  & & \lambda_n
\end{pmatrix} 
\]
Notons $V= \eng { u_{l+2}, \ldots, u_{n}  } $, alors on a que
\[ 
	\min _{ \substack { x \in S^{n-1}\\ x \perp u_{l +2}, \ldots, x \perp u_{n} } } f( x) = \min _{x \in S^{n-1}\cap V^{\perp}} f( x) .
\]
Notons d'abord que la valeur $\lambda_{l+1} $ est bien atteinte par $f$ sur $S^{n-1} $, en effet
\[ 
	f( u_{l+1} ) = u_{l+1}^{T} A u_{l+1}  = u_{l+1} ^{T} U D U^{T} u_{l+1} = \lambda_{l+1} 
\]

Soit donc $x \in V^{\perp} \cap S^{n-1}$, et soit $x = ( x_1, \ldots, x_{l+1} , 0 ,\ldots, 0) $ l'expression de $x$ dans la base $ \left\{ u_1, \ldots, u_n \right\} $.\\
Alors on a 
\begin{align*}
	f( x) &= \left( \sum_{i=1}^{ n} x_i u_i^{T}\right) \cdot A \cdot\left( \sum_{i=1}^{ n} x_i u_i\right) \\
	&= \left( \sum_{i=1}^{ n} x_i u_i^{T}\right)\cdot U\cdot D\cdot U ^{T}\cdot\left( \sum_{i=1}^{ n} x_i u_i\right)\\
	&= \left( \sum_{i=1}^{ n} x_i ( U ^{T}u_i)^{T} \right)\cdot D  \cdot\left( \sum_{i=1}^{ n} x_i ( U^{T}u_i )\right)\\
	&= \left( \sum_{i=1}^{ n} x_i ( U ^{T}u_i)^{T} \right) \cdot  \left( \sum_{i=1}^{ n}\lambda_i x_i ( U^{T}u_i )\right)\\
	&=  \sum_{i=1}^{ n}\sum_{j=1}^{ n} x_i x_j  \lambda_i  ( Uu_i^{T} )^{T}\cdot( U ^{T}u_j)\\
	\end{align*}
	
	\begin{align*}
	&=  \sum_{i=1}^{ n}\sum_{j=1}^{ n} x_i x_j  \lambda_i  \delta_{i,j} \\
	&= \sum_{i=1}^{ n}\lambda_i x_{i} \\
	&= \sum_{i=1}^{ l+1}\lambda_i x_i\\
	& \geq \sum_{i=1}^{ l+1}\lambda_{l+1} x_i\quad  \text {On utilise que $\lambda_{l+1} \leq  \lambda_i \forall 1 \leq  i \leq l+1 $ } \\
	& \geq  \lambda_{l+1} \sum_{i=1}^{ l+1}x_i\\
	& \geq \lambda_{l+1} .
\end{align*}
Ainsi, pour tout élément dans $x \in V^{\perp},$ on a montré que
\[ 
x^{T} A x \geq \lambda_{l+1} 
\]
Et que $\lambda_{l+1} $ est atteint, ce qui montre que
\[ 	\min _{ \substack { x \in S^{n-1}\\ x \perp u_{l +2}, \ldots, x \perp u_{n} } } f( x) = \lambda_{l+1} 	
\]





\end{document}
