\documentclass[11pt, a4paper]{article}
\usepackage[utf8]{inputenc}
\usepackage[T1]{fontenc}
\usepackage[francais]{babel}
\usepackage{lmodern}
\usepackage{amsmath}
\usepackage{amssymb}
\usepackage{amsthm}
\renewcommand{\vec}[1]{\overrightarrow{#1}}
\newcommand{\del}{\partial}
\DeclareMathOperator*{\sgn}{sgn}
\DeclareMathOperator*{\id}{Id}
\DeclareMathOperator*{\im}{Im}
\DeclareMathOperator*{\re}{Re}
\DeclareMathOperator*{\vol}{Vol}
\newcommand\norm[1]{\left\vert#1\right\vert}
\newcommand\ns[1]{\left\vert\left\vert\left\vert#1\right\vert\right\vert\right\vert}
\newcommand\Norm[1]{\left\lVert#1\right\rVert}
\newcommand\N[1]{\left\lVert#1\right\rVert}
\newcommand\abs[1]{\left\vert#1\right\vert}
\newcommand\inj{\hookrightarrow}
\newcommand\surj{\twoheadrightarrow}
\newcommand\ded[1]{\overset{\circ}{#1}}
\newcommand\sidenote[1]{\footnote{#1}}
\newcommand\eng[1]{\left\langle#1\right\rangle}
\newcommand\hr{
    \noindent\rule[0.5ex]{\linewidth}{0.5pt}
}

\newcommand{\incfig}[1]{%
    \def\svgwidth{\columnwidth}
    \import{./figures}{#1.pdf_tex}
}
\newcommand{\filler}[1][10]%
{   \foreach \x in {1,...,#1}
    {   test 
    }
}

\newcommand\contra{\scalebox{1.5}{$\lightning$}}
\makeatother
\def\@lecture{}%
\newcommand{\lecture}[3]{
    \ifthenelse{\isempty{#3}}{%
        \def\@lecture{Lecture #1}%
    }{%
        \def\@lecture{Lecture #1: #3}%
    }%
    \subsection*{\@lecture}
    \marginpar{\small\textsf{\mbox{#2}}}
}

\begin{document}
\title{Mini-Examen 3}
\author{David Wiedemann}
\maketitle
On notera d'abord que pour $\lambda \in \mathbb{C}$,$ \lim_{n \to  + \infty} \lambda^{n} = 0$ si et seulement si $|\lambda|< 1$ .\\
En effet, écrivons $\lambda= |\lambda| e^{i \theta},  $ avec $\theta \in [ 0, 2\pi[ $, alors on a
\[ 
\lim_{n \to  + \infty} \lambda^{n} = \lim_{n \to  + \infty} |\lambda|^{n} e^{in \theta} 
\]
Car la norme de $ e^{i n \theta} $ est égale à 1 pour tout valeur de $n$  et de $\theta$, il est immédiat que si $|\lambda| < 1$, alors
\[ 
\lim_{n \to  + \infty} \lambda^{n}= 0
\]
De même, si $ \lim_{n \to  + \infty} \lambda^{n}= 0$ , alors il faut que $ \lim_{n \to  + \infty} |\lambda|^{n}=0$ et un résultat d'analyse I implique alors que $ |\lambda| <1$.\\

On dénotera par $ \underline 0$ la matrice nulle.\\
On montre maintenant la double implication.\\
$ \Rightarrow $ \\
Montrons que si $ \lim_{j \to  + \infty} A^{j}= \underline 0,$ alors $|\lambda_i|<1, \forall i \in \left\{ 1, \ldots, n \right\} $.\\
$A$ etant diagonalisable, on va considérer une base de vecteurs propres $v_1, \ldots, v_n$.\\
Notons que, car $ \lim_{j \to  + \infty} A^{j} = \underline{0}$, en particulier, on a que $ \lim_{j \to  + \infty} A^{j} v = 0$.\\
Ainsi, on a que 
\[ 
\lim_{j \to  + \infty} A^{j} v_i = \lim_{j \to  + \infty} \lambda_i^{j} v_i = 0
\]
Car $v_i $ est non nul, on en déduit que $|\lambda_i|< 1$.\\
Ainsi, pour tout $i \in \left\{ 1, \ldots, n \right\} $ , on a $ |\lambda_i| < 1$.\\
$ \Leftarrow$ \\
Supposons que $ \forall i \in \left\{ 1, \ldots, n \right\} $ on a $ |\lambda_i|< 1$ , on va montrer que $ \lim_{j \to  + \infty} A^{j}= \underline 0$ .\\
Soit $V$ une matrice inversible telle que
\[ 
A= V DV^{-1}, \quad D = \begin{pmatrix}
	\lambda_1 &  &\\
		  & \ddots &\\
		  & & \lambda_n
\end{pmatrix} 	
\]
Soit à nouveau $ v_1, \ldots, v_n $ une base diagonalisant $A$, on a que
\[ 
\lim_{j \to  + \infty} A^{j} v_i = \lim_{j  \to  + \infty} \lambda_i^{j} v_i =0
\]
Ainsi, soit $w \in \mathbb{C}^{n}$, on sait qu'on peut exprimer $w$ comme combinaison linéaire des $v_i$:
\[ 
w = \sum_{i=1}^{ n} \alpha_i v_i
\]
Ainsi, on a 
\begin{align*}
\lim_{j \to  + \infty} A^{j} w &= \lim_{j \to  + \infty} A^{j} ( \sum_{i=1}^{ n}\alpha_i v_i) \\
&= \lim_{j \to  + \infty} \sum_{i=1}^{ n} \alpha_i \lambda_i^{j} v_i
\intertext{car chaque terme de la somme converge, on a }
&= \sum_{i=1}^{ n}\alpha_i \lim_{j \to  + \infty} \lambda_i^{j}v_i \\
&= \sum_{i=1}^{ n} 0 = 0
\end{align*}
Car pour tout $w \in \mathbb{C}^{n}, \quad \lim_{j \to  + \infty} A^{j} w = 0 $, ceci implique que
\[ 
\lim_{j \to  + \infty} A^{j} =0.
\]



	

	





\end{document}
