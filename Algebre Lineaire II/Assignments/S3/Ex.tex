\documentclass[11pt, a4paper]{article}
\usepackage[utf8]{inputenc}
\usepackage[T1]{fontenc}
\usepackage[francais]{babel}
\usepackage{lmodern}
\usepackage{amsmath}
\usepackage{amssymb}
\usepackage{amsthm}
\renewcommand{\vec}[1]{\overrightarrow{#1}}
\newcommand{\del}{\partial}
\DeclareMathOperator*{\sgn}{sgn}
\DeclareMathOperator*{\id}{Id}
\DeclareMathOperator*{\im}{Im}
\DeclareMathOperator*{\re}{Re}
\DeclareMathOperator*{\vol}{Vol}
\newcommand\norm[1]{\left\vert#1\right\vert}
\newcommand\ns[1]{\left\vert\left\vert\left\vert#1\right\vert\right\vert\right\vert}
\newcommand\Norm[1]{\left\lVert#1\right\rVert}
\newcommand\N[1]{\left\lVert#1\right\rVert}
\newcommand\abs[1]{\left\vert#1\right\vert}
\newcommand\inj{\hookrightarrow}
\newcommand\surj{\twoheadrightarrow}
\newcommand\ded[1]{\overset{\circ}{#1}}
\newcommand\sidenote[1]{\footnote{#1}}
\newcommand\eng[1]{\left\langle#1\right\rangle}
\newcommand\hr{
    \noindent\rule[0.5ex]{\linewidth}{0.5pt}
}

\newcommand{\incfig}[1]{%
    \def\svgwidth{\columnwidth}
    \import{./figures}{#1.pdf_tex}
}
\newcommand{\filler}[1][10]%
{   \foreach \x in {1,...,#1}
    {   test 
    }
}

\newcommand\contra{\scalebox{1.5}{$\lightning$}}
\makeatother
\def\@lecture{}%
\newcommand{\lecture}[3]{
    \ifthenelse{\isempty{#3}}{%
        \def\@lecture{Lecture #1}%
    }{%
        \def\@lecture{Lecture #1: #3}%
    }%
    \subsection*{\@lecture}
    \marginpar{\small\textsf{\mbox{#2}}}
}

\begin{document}
\title{Mini-Examen 3}
\author{David Wiedemann}
\maketitle
On notera d'abord que pour $\lambda \in \mathbb{C}$,$ \lim_{n \to  + \infty} \lambda^{n} = 0$ si et seulement si $|\lambda|< 1$ .\\
En effet, écrivons $\lambda= |\lambda| e^{i \theta},  $ avec $\theta \in [ 0, 2\pi[ $, alors on a
\[ 
\lim_{n \to  + \infty} \lambda^{n} = \lim_{n \to  + \infty} |\lambda|^{n} e^{in \theta} 
\]
Car la norme de $ e^{i n \theta} $ est égale à 1 pour tout valeur de $n$  et de $\theta$, il est immédiat que si $|\lambda| < 1$, alors
\[ 
\lim_{n \to  + \infty} \lambda^{n}= 0
\]
De même, si $ \lim_{n \to  + \infty} \lambda^{n}= 0$ , alors il faut que $ \lim_{n \to  + \infty} |\lambda|^{n}=0$ et un résultat d'analyse I implique alors que $ |\lambda| <1$.\\

On montre maintenant la double implication.\\
$ \Rightarrow $ \\
Etant donné que $A$ est diagonalisable, il existe deux matrices $D, U \in \mathbb{C}^{n\times n}$ tel que $D$ est diagonale, $U$ inversible et telle que l'égalité suivante soit satisfaite
\[ 
A = U D U^{-1}.
\]
Ecrivons 
\[ 
D = 
\begin{pmatrix}
	\lambda_1 & &\\
		  & \ddots &\\
		  & & \lambda_n
\end{pmatrix} 
.\]
On obtient alors
\begin{align*}
\lim_{k \to  + \infty} A^{k} &= \lim_{k \to  + \infty} U D^{k} U^{-1}\\
			     &=U (  \lim_{k \to  + \infty}D^{k} ) U^{-1}\\
			     &= U ( \lim_{k \to  + \infty} 
			     \begin{pmatrix}
				     \lambda_1 ^{k} &  & \\
						    & \ddots &\\
						    &  & \lambda_n^{k}
			     \end{pmatrix} 
			     ) U^{-1} = 0_n
			     \intertext{En multipliant par $U^{-1}$ à gauche et par $U$ à droite, on trouve}
			     \lim_{k \to  + \infty} 
			     \begin{pmatrix}
				     \lambda_1^{k} & & \\
						   & \ddots &\\
						   & & \lambda_n^{k}
			     \end{pmatrix} &= 0
\end{align*}
Et on en déduit que pour tout $i = 1, \ldots, n$, on a $|\lambda_i|< 1	.$ \\
$\Leftarrow$ \\
Si $|\lambda_i| < 1$ pour tout $i = 1, \ldots, n$ , on a
\begin{align*}
	\lim_{k \to  + \infty} A ^{k} &= U ( \lim_{k \to  + \infty} D^{k}) U^{-1}\\
				      &= U (  \lim_{k \to  + \infty} 
				      \begin{pmatrix}
					      \lambda_1^{k} & & \\
							& \ddots & \\
							& & \lambda_n^{k}
				      \end{pmatrix} 
				      ) U^{-1}
				      \intertext{Etant donné que pour $i = 1, \ldots, n$ , on a $ | \lambda_i | < 1$, en passant à la limite, on trouve}
				      &= U ( 
				      \begin{pmatrix}
					      0 & & \\
						& \ddots & \\
						& & 0
				      \end{pmatrix} ) U^{-1}\\
				      &= 0_n
\end{align*}
Ce qui prouve l'assertion.





\end{document}
