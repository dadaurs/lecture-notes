\documentclass[../main.tex]{subfiles}
\begin{document}
\lecture{7}{Tue 16 Mar}{Cayley-Hamilton}
\subsection{Theoreme de Cayley-Hamilton}

\begin{thm}[Evaluation d'une matrice dans un polynome]
	Soit $p( x) = a_0 + \ldots + a_n x^{n} \in K[x]$
	Pour $A \in K^{n\times n}$, on definit
	\[ 
		p( A) = a_0 \id + \ldots  + a_n A^{n}
	\]
\end{thm}
\begin{thm}[Cayley-Hamilton]
	Soit $A \in K^{n\times n}$ et $p(\lambda ) \in K[\lambda]$
	le polynome caracteristique de $A$, alors $p( A) =0 \in K^{n\times n}$
\end{thm}
\begin{proof}
Supposons d'abord que $A \in K^{n\times n}$ est diagonalisable.\\
Alors $\exists  \left\{ v_1,\ldots \right\} $ une base composee de vecteurs propres de $A$.\\
Considerons
\begin{align*}
p( A) \cdot v_i = a_0 v_i + a_1 A  v_i + \ldots\\
= a_0 v_i + a_1 \lambda_i v_i + \ldots \\
= p( \lambda_i) v_i
= 0
\end{align*}
Supposons donc que $A$ n'est pas diagonalisable.\\
Notons que
\[ 
	\id = \frac{cof( A - \lambda \id)^{T}}{\det( A-\lambda \id) } \cdot ( A - \lambda \id) 
\]
Alors
\[ 
	a_0 + a_1 \lambda \id + \ldots = cof( A-\lambda \id) ^{T}\cdot ( A- \lambda \id) 
\]
\begin{align*}
	cof( A-\lambda \id) ^{T}\cdot ( A- \lambda \id) = B_0 A + \sum_{i=1}^{ n-1}\lambda^{i}( B_iA - B_{i-1} ) -\lambda_n B_{n-1} 
\end{align*}
Ce qui implique
\begin{align*}
a_0 \id = B_0 A\\
a_i \id = B_i A - B_{i-1} \text{ pour  } i \in \left\{ 1, \ldots, n-1 \right\} \\
a_n \id = -B_{n-1} 
\end{align*}
On multiplie chacune de ces equations par $A^{i}$ et on les additionne.
On trouve alors
\[ 
	p( A) = 0
\]

\end{proof}
\begin{defn}[Polynome minimal]
	Le polynome unitaire de degre minimal parmi ceux, qui annullent la matrice  $A \in K^{n\times n}$ est appele le polynome minimal de $A$.
\end{defn}
\begin{proof}
Ce polynome est unique.\\
Supposons qu'il existe $q, p$ des polynomes qui annullent $A$.
Alors
\[ 
p \not | q \text{ et } q \not | p
\]
Donc
\[ 
p = q q' + r
\]
ou $r \neq 0, \deg r < \deg p$, donc
\[ 
	0 = p( A) = r( A)  + q'( A) q( A) = r( A) 
\]
Donc $p$ n'est pas de degre minimal $\contra.$
\end{proof}
\begin{crly}
Soit  $A \in K^{n\times n}$ 
\begin{itemize}
\item $A^{k}$ est combinaison lineaire de $\id, A, \ldots, A^{n-1}$ pour tout $k \in \mathbb{N}$

\item $A$ inversible, alors $A^{-1}$ s'ecrit comme combinaison lineaire de\\
	$\id, A, \ldots, A^{n-1}$
\end{itemize}

\end{crly}
\begin{proof}
\begin{itemize}
\item Pour $k \in 0, \ldots, n-1$ clair.\\
	Soit $k \geq n: x^{k}= q( x) p_A( x)  + r( x) $, on evalue
	\[ 
		A^{k} = q( A) p_A( A)  + r(A) = r( A) 	
	\]
	et $r$ est de degre $n-1$.

\item 
	\begin{align*}
		\det A \neq 0	
	\end{align*}
	Donc il suffit de reformuler $p( A) =0$.
	
	
\end{itemize}

\end{proof}






\end{document}	
