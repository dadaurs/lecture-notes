\documentclass[../main.tex]{subfiles}
\begin{document}
\lecture{9}{Tue 23 Mar}{Formes bilineaires}
\begin{defn}[Matrices Congruentes]
	Deux matrices $A,B \in K^{n\times n}$ sont congruentes s'il existe une matrice inversible $P \in K^{n\times n}$ inversible tel que 
	\[ 
	P^{T}\cdot A \cdot P = B
	\]
	
\end{defn}
\subsection{Orthogonalite}
On supposera que $\eng .$ est une forme bilineaire symmetrique.
\begin{defn}[Base orthogonale]
	Soit $ \left\{ v_1, \ldots, v_n \right\} $ une base de $V$. $B$ est une base orthogonale si $\eng { v_i,v_j} =0$ $\forall i \neq j$.
\end{defn}
\begin{lemma}
Soit $V$ de $\dim V = n$ et $B = \left\{ v_1, \ldots, v_n \right\} $ une base de $V$. $B$ est orthogonale si et seulement si la matrice $A_B^{\eng .}$ est une matrice diagonale.
\end{lemma}
\begin{thm}
	Soit $char( K) \neq 2$ et $\dim V = n < \infty $.\\
	Alors $V$ possede une base orthogonale.
\end{thm}
\begin{proof}
Dans le cas $n=1$, le theoreme est trivial.\\
Si $n>1$, alors on distingue deux cas.\\
Si $\eng { u,u} = 0$, la base est trivialement orthogonale.\\
Sinon, soit $u \in V$ tel que $\eng { u,u} \neq 0$.\\
On complete avec $v_2, \ldots, v_n \in V$ tel que $ \left\{ u, v_2, \ldots \right\} $ est une base de $V$.\\
\begin{figure}[H]
    \centering
    \incfig{gramschmidt}
    \caption{gramschmidt}
    \label{fig:gramschmidt}
\end{figure}
\end{proof}
On construit une nouvelle base definie par
\[ 
\left\{ u, v_2- \beta_2 u , \ldots, v_n - \beta_n u \right\}  := \left\{ u, v_2', \ldots \right\} 
\]
Avec $\beta_i = \frac{\eng { \vec{v_i},u} }{\eng { u,u} }$

On remarque que $u \perp v_i'$ et donc $u \perp span \left\{ v_2', \ldots \right\} $.\\
Par hypothese de recurrence, on voit que qu'on peut repeter ce procede pour $ \left\{ v_2'
, \ldots, v_n'\right\} $
\subsection{Matrices congruentes}
On dit que $A \simeq B$ s'il existe  $P \in K^{n\times n}$ inversible tel que
\[ 
P^{T}A P = B
\]
Etre congruent est une relation d'equivalence.
\begin{lemma}
	Soit $B = \left\{ v_1, \ldots, v_n \right\} $ une base de $V$. $V$ possede une base orthogonale si et seulement si $\exists D$ une matrice diagonale $\in K^{n\times n}$ tel que $A_B^{\eng . }\simeq D$
\end{lemma}
\subsubsection*{Algorithme pour trouver une matrice diagonale congruente a $A \in K^{n\times n}$ symmetrique}
L'algorithme prend $n$ iterations.\\
Apres la $i-1$ ieme iteration $A$ est transformee en
\[ 
\begin{pmatrix}
	c_1 & . & .  \\
	. & c_1 &. \\
	. & . & M 
\end{pmatrix}
\]
Ou $M$ est une matrice quelconque.\\
S'il existe un index $j\geq i$ tel que $b_{j j} \neq 0 $, on echange la colonne $i$ et la colonne $j$ et la ligne $i$ et la ligne $j$.\\
Si $b_{ij} =0$ $\forall j \geq i$, on procede a la $i+1$-ieme iteration.\\
Pour chaque $j \in \left\{ i+1, \ldots, n \right\} $ on additionne $.\frac{- b_{ij} }{b_{ii} }$

	


\end{document}	
