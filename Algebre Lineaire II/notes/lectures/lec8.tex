\documentclass[../main.tex]{subfiles}
\begin{document}
\lecture{8}{Wed 17 Mar}{Formes bilineaires}
\section{Formes Bilineaires}
\begin{defn}[Forme Bilineaire]
	\begin{itemize}
	\item BL1 $\forall u \in V,$ 
		\begin{align*}
			f_u: V &\to K\\
			v &\to \eng{u,v}
		\end{align*}
		est lineaire
	
	\item BL2 $\forall u \in V,$ 
		\begin{align*}
			f_u: V &\to K\\
			v &\to \eng{v,u}
		\end{align*}
		est lineaire
	
	\end{itemize}
	
	
\end{defn}
La forme $\eng .$ est dite symmetrique si pour tout $u,v \in V: \eng{u,v}= \eng{v,u}$.\\
La forme $\eng .$ est dite non degeneree a gauche ( resp. a droite) si
$\forall v \in V$ $\eng{v,w}=0 \Rightarrow w=0$.\\
Soit $V$ un espace vect de dimension $n$ et $ \left\{ v_1, \ldots, v_n \right\} $ une base.\\
$ x,y \in V$ sont representes comme combinaison lineaire de $ \left\{ v_1, \ldots \right\} $, soit $x = \sum x_i v_i$, et $ y = \sum y_i v_i$, alors
\begin{align*}
	\eng { \sum x_i v_i, y} &= \sum \eng { x_i v_i, y} \\
				&= \sum x_i \eng { v_i, y} \\
				&= \sum x_i \eng { v_i, \sum y_j v_j}\\
				&= \sum x_i \sum y_j \eng{v_i, v_j}\\
				&= ( x_1, \ldots, x_n) 
				\begin{pmatrix}
					\eng { v_1, v_1} & \ldots & \eng {v_1, v_n } \\
\vdots & \ddots & \vdots \\
\eng { v_n, v_1}  & \ldots & \eng { v_n,v_n} 
				\end{pmatrix}
				( y_1, \ldots, y_n)^{T}
\end{align*}
\begin{propo}
	Soit $V$ un espace vectoriel sur $K$ de dimension finie et 
	$B= \left\{ b_1,\ldots, b_n \right\} $ une base de $V$. \\
	Soit $f: V \times V \to K$ une forme bilineaire. \\
	Les conditions suivantes sont equivalentes
	\begin{itemize}
		\item $\rg ( A_B ^{f}) = n$ 
		\item $f$ est non degeneree a gauche
		\item $f$ est non degeneree a droite
	\end{itemize}
	
\end{propo}
\begin{proof}
On demontre que 1 est equivalent a 2.\\
Il faut montrer que $\exists u \in V $ tel que $f( v,u) \neq 0$, or
\[ 
	f( v,u) = [ v] _B ^{T} \cdot A_B^{f} \cdot [ u] _B
\]
mais $\rg A_B^{f}=n \Rightarrow  [ v]_B^{T}\cdot A_B^{f}\neq 0^{T}$.\\
Soit $i \in  \left\{ 1, \ldots, n \right\} $ tel que la $i$-eme composante de $( [v]_B^{T}\cdot A_B^{f} )_i \neq 0$, alors pour $u=b_i$ on a fini.\\
Supposons maintenant que $\rg A_{B} ^{f}<n$, alors
$\exists x \in K^{n}\setminus \left\{ 0 \right\} $ tel que $x^{T}\cdot A_B^{f}=0$ \\
donc les lignes de $A$ sont lineairements independantes.
\end{proof}
\subsection{Orthogonalite}
Soit $\eng .$ une forme bilineaire symetrique.
\begin{defn}[Orthogonalite]
	Deux elements $u,v$ sont orthogonaux si
	\[ 
	\eng { u,v} =0
	\]
	
	
\end{defn}
\begin{defn}[Complement orthogonal]
	Soit $E \subseteq V$, alors
	\[ 
	E^{\perp}= \left\{ u \in V: u \perp e \forall e  \in E \right\} 
	\]
	
	
\end{defn}
\begin{propo}
Soit $E \subseteq V$, alors $E^{\perp}$ est un sous-espace de $V$.
\end{propo}
\begin{lemma}
Soit $K$ un corps de characteristique differente de 2.\\
Si $ \eng { u,u } =0$ pour tout $u \in V$, alors $\eng { u,v} =0 \forall u,v \in V$
\end{lemma}
\begin{proof}
Soient $u,v \in V$ :
\[ 
2 \eng { u,v} = \eng { u+v, u+v} - \eng { u,u} - \eng { v,v} 
\]
et donc $\eng { u,v } =0$.

\end{proof}



\end{document}	
