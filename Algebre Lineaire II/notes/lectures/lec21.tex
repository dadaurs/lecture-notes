\documentclass[../main.tex]{subfiles}
\begin{document}
\lecture{21}{Tue 11 May}{forme normale de Jordan}
\begin{lemma}
Pour $A,B \in \mathbb{C}^{n\times n}$, si $AB=BA$, alors 
\[ 
e^{A+B}  = e^{A} e^{B} 
\]
\end{lemma}
On va montrer que pour chaque matrice $A \in \mathbb{C}^{n\times n}$ se laisse factoriser comme
\[ 
A = P \cdot 
( \begin{pmatrix}
	\lambda_1 & & \\
		  & \ddots & \\
		  && \lambda_n
\end{pmatrix} +N  )\cdot P^{-1}
\]
Ou $P$ est inversible et $N \in \mathbb{C}^{n\times n}$ est nilpotente.\\
En effet, avec ce theoreme, on peut voir que
\[ 
	e^{tA}  = P e^{t ( D+N) } P^{-1}
\]
ou $D$ est la matrice diagonale.\\
Un theoreme demontre en exercice donne alors
\begin{align*}
	e^{tA}  &= P e^{ t( D+N) } P^{-1}\\
	&= P e^{t D}  e^{tN} P^{-1}\\
	&= P
	\begin{pmatrix}
		e^{t \lambda_1}  & & \\
				 & \ddots & \\
				 & & e^{t \lambda_n} 
	\end{pmatrix} \left[ \sum_{i=0}^{ j} \frac{t^{i}}{i!}N^{i}\right] P^{-1}
\end{align*}
\begin{defn}[Bloc de Jordan]
Un bloc de jorand est done par
\[ 
\begin{pmatrix}
	\lambda_1 & 1 & & \\
		  & \ddots & 1 & \\
		  & & & \lambda_n
\end{pmatrix} 
\]
Une matrice $A \in \mathbb{C}^{n\times n}$ est en forme normale de Jordan si 
\[ 
A = \begin{pmatrix}
	B_1 & & \\
	    & \ddots & \\
	    & & B_k
\end{pmatrix} 
\]
ou les $B_i$ sont des blocs de Jordan.
\end{defn}
\begin{lemma}
Si $J \in K^{n\times n}$ est en forme noramle de Jordan et $J = D +N$, alors $DN = ND$.
\end{lemma}
\begin{thm}
	Soit $A \in K^{n\times n}$ et $m( x) \in K[x]$ un polynome avec coefficient dominant egal a 1.\\
	De plus soit $m( A) =0$.\\
	Soit $\deg m $ minimal parmis tous les polynomes qui satisfont cette condition, alors $m( x) $ est unique et est appele polynome minimal de $A$.
\end{thm}
\begin{proof}
	Soit $m'( x) \in K[x]$ aussi un tel polynome, cad $m'( A) =0$, $\deg m' = \deg m$ et de coefficient dominant $1$.\\
	On va montrer que $m'|m$.\\
	Par division euclidienne, on a 
	\[ 
		m( x) = q( x)  m'( x)  + r( x) 
	\]
Or 
\[ 
	m( A) =0 = q( A) m'( A)  + r( A)  \Rightarrow  r( A) =0 \Rightarrow r =0
\]
Et donc $m = m' q \Rightarrow \deg q =0$
\end{proof}
\begin{defn}
	Soit $A \in K^{n\times n}$ une matrice.\\
	Soit $W \subset K^{n}$, $W$ est invariant sur $A$, si $\forall w \in W, A \cdot w \in W$
\end{defn}
\begin{lemma}
	Soit $f( x) \in K[x]$ et $A \in K^{n\times n}$, soit $v \in \ker f( A) $, alors $Av \in \ker f( A)  $.\\
	cad que $\ker f( A) $ est invariant sur $A$.
\end{lemma}
\begin{proof}
	A montrer: pour $v \in K^{n},$ si $f( A) v =0 $, alors
	\[ 
		f( A) A v =0
	\]
	Mais $f( A) A v = A f( A) v =0$	
\end{proof}
 
	

	


\end{document}	
