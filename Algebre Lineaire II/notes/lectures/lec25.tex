\documentclass[../main.tex]{subfiles}
\begin{document}
\lecture{25}{Tue 25 May}{systemes d equations diophantiennes}
Soit $Ax = b, x \in \mathbb{Z}^{n}, A \in \mathbb{Z}^{m\times n},b \in \mathbb{Z}^{m}$.\\
On va supposer que $rang A = m$ .\\
\begin{lemma}
	Soit $U \in \mathbb{Z}^{n \times n}$ inversible ( sur $\mathbb{Q}$ ) :
	\[ 
		U^{-1}\in \mathbb{Z}^{n\times n}\iff \det U = \pm 1
	\]
	$U$ est unimodulaire.
\end{lemma}
On veut donc trouver un $U \in \mathbb{Z}^{n\times n}$ unimodulaire tel que 
\[ 
AU = [ H|0] , \text{ ou }  H \in \mathbb{Z}^{m\times m}  \text{ inversible  } 
\]
On a donc
\begin{align*}
Ax = b, x \in \mathbb{Z}^{n} \text{ solution } \\
\iff AUy = b, y \in \mathbb{Z}^{n} \text{ solution } 
\end{align*}
Ou $y = U^{-1}x$.\\
On dit que $AU$ est en forme normale d'Hermite si $h_{ij} < h_{ii} $ pour tout $i =1, \ldots, m, j = 1, \ldots, i-1$ .\\
En consequence, 
\[ 
AU = [ H|0] \text{ en FNH } 
\]
tel que 
\[ 
Ax = b, x \in \mathbb{Z}^{n} \text{ solution  } \iff Hy = b, y \in \mathbb{Z}^{m}
\]
\begin{lemma}
	Soit $ ( a_1, \ldots, a_n)  \in \mathbb{Z}^{1\times n} \setminus \left\{ 0 \right\} $ , alors $\exists U \in \mathbb{Z}^{n\times n}$  tel que
	\[ 
		( a_1, \ldots, a_n) \cdot U = ( \gcd ( a_1, \ldots, a_n) , 0, \ldots,0) 
	\]

\end{lemma}
\begin{proof}
Recurrence sur le nombre de composantes $\neq 0$ 
$n=1$ \\
On multiplie eventuellement avec $-1$ pour obtenir le gcd et on echange les lignes avec les collonnes.
$n>1$ \\
Soient $a_i, a_j \neq 0$, $j \neq i$, on multiplie 
\[ 
	( \ldots,a_i,\ldots, a_j, \ldots) 
	\begin{pmatrix}
		1 & &\\
		  & \ddots &\\
		  & & 1
	\end{pmatrix} 
\]
avec $x $ dans la $i,i$ -eme composante, $y$ dans la $i,j$ -eme composante, $- \frac{a_i}{d}$ dans la $j,j$ composante et $ \frac{a_j}{d}$ dans la $j,i$ -eme composante.\\
Et car
\[ 
	\gcd ( a,b,c) = \gcd ( \gcd( a,b) ,c) 
\]
On conclut par recurrence.

\end{proof}
\begin{thm}
Soit $A \in \mathbb{Z}^{m\times n}, rang A = m$ , alors $\exists U \in \mathbb{Z}^{m\times n}$  tel que $AU$ est en forme normale de Hermite.
\end{thm}
\begin{proof}
On montre a nouveau par recurrence sur $m$ .\\
$m=1$ \\
Vrai par le lemme ci-dessus.\\

$m>1$ \\
Par le lemme, il existe $U$ tel que
\[ 
A U = \begin{pmatrix}
	\gcd ( a_{11} , \ldots)  & 0 & &\\
	a_{21} & A' & &
	a_{21} &  & &
	a_{21} & & &
\end{pmatrix} 
\]


Pour $m>1$ , par recurrence, $\exists U' \in \mathbb{Z}^{k-1\times n-1}$ unimodulaire tel que 
\[ 
A' U' = [ H' |0] 
\]

\end{proof}
\begin{lemma}
Soit $A \in \mathbb{Z}^{m\times n}, rang A = m$.\\
La FNH de $A$ est unique.
\end{lemma}
\begin{proof}
	Soient $U_1, U_2 \in \mathbb{Z}^{n\times n}$ unimodulaire et $H_1 \neq H_2\in \mathbb{Z}^{m\times m}$ en FNH tel que
	\[ 
	AU_1 =[ H_1|0 ] , AU_2 = [ H_2|0] 
	\]
	Alors 
	\[ 
	A \mathbb{Z}^{n}= \left\{ Ax : x \in \mathbb{Z}^{n} \right\} = H_1 \mathbb{Z}^{m} = H_2 \mathbb{Z}^{m}
	\]
	Soit $i \in \left\{ 1, \ldots,  m \right\} $ minimal tel qu'il existe $j \in \left\{ 1, \ldots,i \right\} $ tel que $h_{ij} \neq h'_{ij} $, sans perte de géneralité $h_{ij} >h'_{ij} $.\\
	Soient $h \in \mathbb{Z}^{m}, h' \in \mathbb{Z}^{m}$ les j-emes colonnes de $H_1$ et $H_2$ respectivement.\\
	Alors on a
	\[ 
h - h' = 
\begin{pmatrix}
0\\
\vdots \\
0\\
h_{ij} - h'_{ij} \\
\vdots
\end{pmatrix} 
	\]
$\exists y_1, y_2 \in \mathbb{Z}^{m}: H_1 y_1 = H_2y_2 = h-h' $.\\
Alors 
\[ 
	h_{ij} -h'_{ij} = z_1 h_{ii} = z_2 h'_{ii} , z_1, z_2 \in \mathbb{Z}\setminus \left\{ 0 \right\} 
\]
Donc
\begin{align*}
\Rightarrow h_{ij}  - h'_{ij } \geq  h_{ii} \\
\Rightarrow h_{ij } -h'_{ij} \geq h'_ii
\end{align*}
Si $i=j$ , alors $h'_{ij} \neq 0$ , alors
\[ 
h_{ij} - h'_{ij} < h_{ij } \contra
\]
Si $i>j$ 
\[ 
h_{ij} - h'_{ij} < h_{ii}  \contra
\]
\end{proof}
\section{La forme normale de Smith}
\begin{thm}
Soit $A \in \mathbb{Z}^{m\times n}$.\\
Il existe $U \in \mathbb{Z}^{m\times m}, V \in \mathbb{Z}^{n\times n}$ unimodulaires tel que
\[ 
UAV  = D
\]
est diagonale, ou $d_i | d_{i+1} $ et $d_i \in \mathbb{N} \forall i$ 
\end{thm}
\begin{proof}
On veut transformer $A$ en une matrice de la forme
\[ 
\begin{pmatrix}
	d & 0 \\
	0 & A'
\end{pmatrix} 
\]
tel que $d$ divise chaque composante de $A'$.\\
Si $A' =0$ et on a fini.\\
Autrement $U' \in \mathbb{Z}^{m-1\times m-1}$ et $v' \in \mathbb{Z}^{n-1 \times n-1}$ unimodulaire tel que
\[ 
U' A' V' = D'
\]
On va identifier l'element non nul de la premiere ligne ou colonne de $A\neq_0$ de valeur absolue minimale.\\
Si l'element pivot ne divise pas tous les autres elements de $A$, on transforme $A$ tel que le nouvel element pivot est plus petit.\\
Si $ a_{11}\not | a_{1j} \Rightarrow a_{1j} = q a_{11} +r$ , avec $0<r < a_{11}$.\\
On soustrait $q$ fois la colonne $1$ de $A$  de la colonne $j.$\\
ainsi, $0<a_{1j}< a_{11} $\\
Ensuite en echange la colonne $1$ et $j$ et le nouvel element pivot est plus petit.\\
Sinon, $a_{11} | a_{i_1} , i >1$ \\
Si $a_{11}| a_{1i} \forall i =2 , \ldots, a_{11} |a_{1j} \forall j = 2, \ldots i$ \\
Ainsi, on peut eliminer tous les elements sur la premiere ligne et sur la premiere colonne.\\
Supposons donc que $a_{11} \not | a_{ij} $ , alors on ajoute la $i$ -eme ligne a la premiere ligne et on effectue la division euclidienne.
\end{proof}





\end{document}	
