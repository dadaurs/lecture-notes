\documentclass[../main.tex]{subfiles}
\begin{document}
\lecture{12}{Wed 31 Mar}{...}
\begin{defn}
	Soit $V$ un espace Euclidien, et $\eng .$ un produit scalaire.\\
	Une base $ \left\{ u_1,\ldots, u_n \right\} $ orthogonale  est appelee orthonormale si  $\N u_i= 1 \forall i$.\\


\end{defn}
\begin{crly}
Soit $V\in \mathbb{R}^{m\times n}$ une matrice de plein rang colonne, alors on peut factoriser $V= U^{*}\cdot R$ ou $U^{*} \in \mathbb{R}^{m\times n}$dont les colonnes sont deux-a-deux orthogonales et de norme $=1$, et ou $R$ est une matrice triangulaire superieur
\end{crly}
\subsection{La methode des moindres carres}
Soit $A\cdot x = b$ un systeme lineaire en $m$ variables sans solution.\\
On cherche un $x$ tel que $\N { A\cdot x - b} $ est minimale.
On resout donc
\[ 
\min_{x \in \mathbb{R}} \N { A\cdot -b} 
\]

\begin{thm}
	Soit $V$ un espace euclidien et soient $v_{1}, \ldots,v_n$ des vecteurs deux-a-deux orthogonaux non-nuls. Soit $v \in V$ et $\alpha_i = \frac{\eng { v,v_i} }{\eng{v_i,v_i}}$, alors
	\[ 
\N { v - \sum_{i=1}^{ n}\alpha_i v_i}  \leq \N { v - \sum_{i=1}^{ n}\beta_i v_i} 
	\]
	pour tout $\beta_1, \ldots,\beta_n \in \mathbb{R}$
\end{thm}
\begin{proof}
on a
\begin{align*}
	\N { v- \sum_{i=1}^{ n}\beta_i v_i} ^{2} = \N { \underbrace{v - \sum_{i=1}^{ n}\alpha_i v_i }_{ \text{ perpendiculaire a tous les $v_i$ }}-  \sum_{i=1}^{ n}( \beta_i - \alpha_i) v_i } ^{2}\\
	= \N { v - \sum_{i=1}^{ n}\alpha_i v_i} ^{2} + \N { \sum_{i=1}^{ n}( \beta_i - \alpha_i) v_i} ^{2} \geq \N { v- \sum_{i=1}^{ n}\alpha_i v_i} 
\end{align*}

\end{proof}
Donc, pour resourdre $\min_{x\in \mathbb{R}^n} \N { Ax -b} $, on calcule d'abord une base orthogonale de l'espace engendre par les vecteurs-collone de $A$.\\
Ensuite, on calcule la projection de $b$, cad $ \sum_{i=1}^{ n} \frac{\eng { b, a_i^{*}} }{\eng {a_i^{*}, a_i^{*}} }$.\\
Ensuite, on resout $Ax = proj( b) $ et on trouve un $x$ proche.
\begin{thm}
Les solutions du systeme 
\[ 
A^{T} \cdot A x = A^{T}b
\]
sont les solutions optimales de $\min_{x\in \mathbb{R}^n} \N { Ax -b} $
\end{thm}
\begin{proof}
	$x$ est une solution optimale $\iff$ $A\cdot x = proj( b) $, de plus $proj( b) $ est le vecteur $v$ unique dans $ \left\{ A \cdot x : x \in \mathbb{R}^n \right\} $ tel que $b-v \perp span \left\{ A \right\} = \left\{ A\cdot x : x \in \mathbb{R}^n \right\} $ \\
		Donc
		\[ 
			A^{T}A x = A^{T} b \iff A ^{T} ( Ax-b) = 0 \iff Ax - b \perp  \left\{ A\cdot x : x \in \mathbb{R}^n \right\} 
		\]
		
\end{proof}
\subsection{Formes sesquilineaires et produits hermitiens}
Soit $v = \begin{pmatrix}
a_1 + ib_1\\ \vdots\\ a_n + i b_n
\end{pmatrix}
\in \mathbb{C}^{n}
$, avec $a_i,b_i \in \mathbb{R}$.\\
On definit
\[ 
\sum_{i=1}^{ } a_i^{2} + b_i^{2} = \sum_{i=1}^{ n} v_i \overline{v_i}
\]
\begin{defn}[Produit Hermitien]
Soit $V$ un espace vectoriel sur $\mathbb{C}$, $\eng .$ une application, alors on a
\begin{itemize}
\item PH1:$\eng { v,w} = \overline{\eng { w,v} }\forall v,w\in V$ 
\item PH2
	\[ 
	\eng { u,v+w} = \eng { u,v} + \eng { u,w} , \eng { w+u,v} = \eng { v,w} + \eng { u,w} 
	\]
	
\item PH3
\[ 
\forall x \in \mathbb{C}, u,v \in V, \eng { xu ,v} = x\eng { u,v} , \eng { u,xv} = \overline{x} \eng { u,v} 
\]

\end{itemize}
\begin{enumerate}
\item Une forme sesquilineaire satisfait PH2, PH3
\item Forme hermitienne satisfait PH1,PH2, PH3
\item Un produit hermitien satisfait PH1,PH2,PH3 et de plus
	 \[ 
	\eng { v,v} >0 \forall v \in V \setminus \left\{ 0 \right\} 
	\]
	
\end{enumerate}
Le produit hermition est l'analogue d'un produit scalaire.
\end{defn}
\begin{defn}[Matrice hermitienne]
	$A \in \mathbb{C}^{n \times n} $ est appellee hermitienne si $A^{T}= \overline{A}$
\end{defn}
\begin{propo}
	Soit $V$ un espace vectoriel sur $\mathbb{C}$ de dimension finie et soit $B $ une base de $V$. Une forme sesquilineaire est une forme hermitienne si et seulement si $A_B^{f}$ est une matrice hermitienne.
\end{propo}
Si $B,B'$ sont deux bases differentes, alors $f( v,w) = [ v] _B^{T} A_B^{f}\overline{ [ w] }_B$.\\
Si $B'$ est une autre base, et $P_{BB'} , P_{B'B} $les matrices de changement de base correspondentes.
Alors on a
\[ 
	[ v_{B'} ] ^{T}( P_{B'B} ) ^{T}A_B^{f} \overline{P_{B'B} } \overline{[w]}_B' = f( v,w) 
\]
On en deduit que
\[ 
	A_{B'} ^{f}= ( P_{B'B} ) ^{T} A_B^{f} \overline{P_{B'B} }
\]
\begin{defn}[Matrices Complexes congruentes]
	Deux matrices complexes $A,B$ sont congruentes complexes, si il existe $P$ une matrice inversible satisfaisant
	\[ 
	A= P^{T}B \overline{P}
	\]
	
\end{defn}

Comme avant, une base $ B=\left\{ b_1, \ldots \right\} $ est une base orthogonale si et seulement si $A_B^{\eng .}$ est diagonale.\\
\begin{thm}
Soit $V$ un espace vectoriel complexe et $\eng .$ une forme hermitienne, alors $V$ possede une base orthogonale.
\end{thm}
On utilise le procede analogue aux espaces hermitiens.



\end{document}
