\documentclass[../main.tex]{subfiles}
\begin{document}
\lecture{16}{Wed 21 Apr}{Valeurs Singulieres}
\subsection{Decomposition en valeurs singulieres}
\begin{thm}[Decomposition en valeurs singulieres]
	Soit $A \in \mathbb{C}^{m\times n}$, il existe des matrices unitaires $P \in \mathbb{C}^{m\times n}, Q \in \mathbb{C}^{n \times n}$ tel que $A= P D Q$ avec $D \in \mathbb{R}_{ \geq 0} ^{m\times n}$ une matrice diagonale.
\end{thm}
\begin{proof}
On veut $A = P 
\begin{pmatrix}
	\sigma_1 & & \\
		 & \ddots & \\
		 & & \sigma_r
\end{pmatrix} Q$ avec $\sigma_1 \geq \ldots \geq \sigma_r >0$ les valeurs singulieres.\\
Soit $u_1,\ldots, u_n$ une base orthogonale par rapport au produit hermitien standard compose de valeurs propres associees a $\sigma_{1}^{2} \geq \sigma_2^{2} \ldots \geq  \sigma_r^{2} \geq \sigma_{r+1} ^{2} = \ldots =0  $. \\
On definit 
\[ 
Q \coloneqq 
\begin{pmatrix}
u_1^{*}\\ \vdots \\ u_n^{*}
\end{pmatrix} 
\]
Et soit
\[ 
v_i \coloneqq  \frac{A u_i}{\sigma_i}, \quad  i=1,\ldots,r
\]
et on complete $v_1, \ldots, v_r$ en une base orthogonale de $\mathbb{C}^{m}$, on va montrer que 
\[ 
	P \coloneqq ( v_1, \ldots, v_r, v_{r+1} , \ldots,v_m) \in \mathbb{C}^{m\times m}
\]
est unitaire.\\
Il est clair que $v_j^{*}v_j = 1\forall j \geq r+1$, sinon, pour $1 \leq i,j \leq  r$, on a
\begin{align*}
	v_i^{* }v_j &= \frac{u_i^{*}A^{*}}{\sigma_i} \cdot \frac{A \cdot u_j}{\sigma_j} \\
	&= \frac{u_i^{*}\sigma_j^{2} u_j}{\sigma_i\sigma_j}\\
	&=
	\begin{cases}
	0 \text{ si } i \neq j\\
	1 \text{ si } i=j
	\end{cases}
\end{align*}
Il reste a verifier que
\[ 
( P^{*} A Q^{*} )_{ij} = 
\begin{cases}
	0 \text{ si }  i \geq r \text{ ou } j > 0 \text{ ou  } i\neq j\\
	\sigma_i \text{ autrement } 
\end{cases}
\]
Pour $i>r$ et $j \leq r$, on a donc
\[ 
u_i^{*}A u_j = v_i^{*} \sigma_j v_j = 0
\]
Et finalement, pour $i \leq j \leq r$, on a 
\[ 
= \frac{u_i^{*}A^{*}}{\sigma_i}A u_j 
\]
\end{proof}
\subsection{Pseudo-inverse d'une matrice}
\begin{defn}[Pseudo inerse]
	Pour une matrice $A\in \mathbb{C}^{m \times n}$, on note
	\[ 
	D^{+} =
	\begin{pmatrix}
		\frac{1}{\sigma_1} & & &\\
				   & \ddots & & &\\
				   & & \frac{1}{\sigma_r} & &
	\end{pmatrix} 
	\in \mathbb{R}^{n\times m}
	\]
	ou  $\sigma_i$ sont les valeurs singulieres de $A$.
\end{defn}
\begin{rmq}
La factorisation en valeurs singulieres n'est pas unique.
\end{rmq}
On va montrer que le pseudo-inverse d'une matrice est unique.\\
\begin{thm}
Soit $A \in \mathbb{C}^{m\times n}$, il existe au plus une seule matrice $X \in \mathbb{C}^{n\times m}$ qui satisfait les conditions de penrose
\begin{itemize}
\item $A X A = A$ 
\item $( A \cdot X) ^{*}= A X $ 
\item $XAX=X$ 
\item $( X \cdot A) ^{*}= XA$
\end{itemize}

\end{thm}
\begin{proof}
Supposons que $X,Y \in \mathbb{C}^{n\times m}$ satisfait les conditions de penrose
\begin{align*}
X &= XAX\\
&= XAYAX\\
&=XAYAYAYAX\\
&= ( XA) ^{*}( YA) ^{*} Y (AY ) ^{*}( AX) ^{*}\\
&= A^{*}X^{*}A^{*} Y^{*}Y Y^{*}A^{*}X^{*}A^{*}\\
&= ( AXA) ^{*}Y^{*} Y Y^{*} ( AXA) ^{*}\\
&= A^{*}Y^{*}Y Y^{*} A^{*}\\
&= ( YA) ^{*}Y ( AY) ^{*}= YAYAY = YAY
\end{align*}

\end{proof}
\begin{thm}
Soit $A \in \mathbb{C}^{m\times n}$, alors $A^{+}$ verifie les regles de penrose.
\end{thm}
\begin{proof}
On verifie facilement pour $D$ diagonale
\begin{itemize}
\item 
	\[ 
	A A^{*}A = PDQQ^{*} D^{*}P^{*}P DQ = PDQ =A
	\]

\item idem pour le reste.
	
\end{itemize}

\end{proof}







\end{document}	
