\documentclass[../main.tex]{subfiles}
\begin{document}
\lecture{3}{Tue 02 Mar}{Factorisation des polynomes sur un corps}
\subsection{Factorisation des polynomes sur un corps}
Soit $K$ un corps.
\begin{defn}[Polynome irreductible]
	Un polynome $p( x)  \in K[x] \setminus \left\{ 0 \right\} $ est irreductible si
	\begin{itemize}
	\item $\deg p \geq 1$ 
	\item si $p( x) = f( x) \cdot g( x) $, alors $\deg f =0$ ou $\deg g=0$.
	\end{itemize}
	
\end{defn}
\begin{thm}
	Un polynome de degre 2 sur $K[x]$ est irreductible si et seulement si le polynome ne possede pas de racines.
\end{thm}
\subsection{Diviseurs Communs le plus grand}
\begin{thm}
Soient $f( x) ,g( x) \in K[x]$ pas tous les deux nuls.\\
On considere l'ensemble $I= \left\{ u \cdot f + v \cdot g: u,v \in K[x] \right\} $.\\
Il existe un polynome $d( x) \in K[x]$ satisfaisant
\[ 
	I = \left\{ h \cdot d : h \in K[x] \right\} 
\]


\end{thm}
\begin{proof}
Soit $a \in I \setminus \left\{ 0 \right\} $ de degre minimal.\\
L'ensemble $ \left\{ h \cdot d: h \in K[x] \right\} $ est clairement un sous-ensemble de $I$.\\
Il reste a montre l'inclusion inverse.\\
Si $d$ ne divise pas $uf + vg$, la division avec reste donne
\[ 
	uf + vg = qd + r \iff r = uf +vg -qd = ( u -q u') f + ( v-qv') g
\]
Or le reste est non nul, mais le reste est de degre inferieur a $
\deg d$. $\contra$
\end{proof}
\begin{defn}[Polynome Unitraire]
	Un polynome $f( x) \in K[x]$ dont le coeff. dominant $=1$ est un polynome unitaire.
\end{defn}
\begin{defn}[Diviseur Commun]\label{defn:Diviseur Commundiviseur_commun}
	Soient $f,g \in K[x]$ non-nuls.\\
	Un diviseur commun de $f$ et $g$ est un polynome qui divise $f$ et $g$.
\end{defn}
\begin{thm}
	Soient $f,g \in K[x]$ non-nuls.\\
	Soit $d \in K[x]$ comme dans le theoreme precedent.\\
	\begin{itemize}
	\item $d$ est un diviseur commun de $f$ et $g$.
	\item Chaque diviseur commun de $f$ et $g$ est un diviseur de $d$.
	\item Si $d$ est unitaire, alors $d$ est unique.
	\end{itemize}
	
\end{thm}
\begin{proof}
\begin{itemize}
\item $f \in I \Rightarrow \exists h $ tel que $hd =f \iff d |f$ et  $g \in I \Rightarrow  d| g$ 
\item Soit $d' \in K[x]$ tq $d' | f, d'|g$, on veut montrer que $d' |d$.
	\[ 
	f = f' d', g=g'd'
	\]
	des que $d \in I$, il existe $u,v \in K[x]$ tel que
	\[ 
	d = uf + vg = uf'd' + vg'd' = ( uf'+vg' ) d' \Rightarrow d' | d
	\]
	

\item Soit $d' \in I$ tel que $I = \left\{ hd' | h \in K[x] \right\} $.\\
	Soient $d,d'$ unitaires.\\
	$d| d'$ et $d'| d$, donc ils sont les memes a un facteur pres.
	
\end{itemize}

\end{proof}
\begin{defn}[PGCD]
	L'unique polynome unitaire $d \in K[x]$ qui satisfait les conditions ci-dessus est appele le plus grand commun diviseur de $f$ et $g$.
\end{defn}
\begin{thm}[Algorithme d'Euclide]
	Soient $f_0, f_1$ non nuls et 
	\[ 
	\deg f_0 \geq \deg f_1
	\]
	On cherche $\gcd( f_0,f_1) $ 
	Si $f_1=0$, alors $\gcd = f_0$.\\
	Si $f_1 \neq 0$ 
	On pose 
	\[ 
		f_{0} = q_1 f_1 + f_2 
	\]
	Soit $h \in K[x]: h | f_0 $ et $h| f_1 \Rightarrow  h | f_2$ 
	Et donc on pose $\gcd( f_0,f_1) = \gcd( f_1,f_{2}) $
	On repete jusqu'a trouver un $f_k$ nul.
	
\end{thm}
Grace a l'algorithme d'Euclide, on peut aussi trouver $u,v \in K[x]$ tel que $uf_0 + vf_1 = \gcd( f_0, f_1) $.\\
En effet, on a
\[ 
\begin{pmatrix}
f_i\\ f_{i+1} 
\end{pmatrix}
=
\begin{pmatrix}
	0 & 1\\
	1 & - q_i
\end{pmatrix}
\begin{pmatrix}
f_{i-1} \\
f_i
\end{pmatrix}
\]
et donc en appliquant cette matrice plusieurs fois, on trouve une dependance lineaire entre $f_{k-1} $ et $f_k$ \\
Et donc le $\gcd( f_0, f_1)  = \frac{1}{ \text{ coeff dominant de } f_{k-1} } ( uf_0 + v f_1) $







\end{document}	
