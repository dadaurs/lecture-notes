\documentclass[../main.tex]{subfiles}
\begin{document}
\lecture{19}{Tue 04 May}{Systemes differentiels lineaires}
\section{Systemes differentiels lineaires}
Etant donne $a_{ij} \in \mathbb{R}, 1 \leq i \leq n, 1 \leq j \leq n$, on cherche une solution au systeme
\[ 
\begin{cases}
	x'_1( t) = a_{11}x( t)  + \ldots + a_{1n} x_n( t) \\u
	\vdots\\
	x'_n( t) = a_{n1}x( t)  + \ldots + a_{nn} x_n( t) 
\end{cases}
\]
On cherche $x_i: \mathbb{R}\to \mathbb{R}$ derivable qui resolvent le systeme d'equations lineaires.
\begin{exemple}
\[ 
	x'( t) = x( t) 
\]
Une solution est: $x( t) = e^{t} $.\\
Une autre est: $x( t) = 2 e^{t} $. \\
Si on exige les \underline{conditions initiales} $x( 0) =5$, on aura la solution
\[ 
	x( t) = 5 e^{t} 
\]
\end{exemple}
\begin{thm}
	Etant donne les conditions initiales $x( 0) $, il existe une \underline{solution unique} qui respecte les conditions intiales.
\end{thm}
On peut reecrire notre systeme comme
\[ 
A \cdot x = x', A \in \mathbb{R}^{n\times n}
\]
Supposons que $x( t) = v e^{\lambda t} $ est une solution du systeme( $v \in \mathbb{R}^n$).\\
Alors,
\[ 
	x'( t) = A \cdot x( t) = \lambda v e^{\lambda t} 
\]
Donc $v$ est un vecteur propre de $A$.
\begin{lemma}
Soit $\mathcal{X} = \left\{ x : x \text{ solution du systeme differentiel }  \right\} $, alors $\mathcal{X}$ est un espace vectoriel sur $\mathbb{R}$.
\end{lemma}
\begin{thm}
	Soit $ \left\{ v_1, \ldots, v_n \right\} $ une base de vecteurs propres de $A$ associee aux valeurs propres $\lambda_1, \ldots, \lambda_n$.\\
	Alors
	\[ 
	x_i = e^{\lambda_i t}  v_i, \quad i = 1, \ldots, n
	\]
est une base de $ \mathcal{X}$.	
	
\end{thm}
\begin{proof}
On a deja vu que $x_i$ est une solution du systeme, car
\[ 
A \cdot x_i = A v_i e^{\lambda_i t} 
\]
Soient $x( 0) \in \mathbb{R}^n$ des conditions initiales, on veut trouver 
\[ 
\beta_1 , \ldots, \beta_n \in \mathbb{R}^n \text{ tel que } \sum \beta_i x_i 
\]
est une solution qui respecte $x( 0) $.\\
Soit $x( 0) = \sum \beta_i x_i( 0) = \sum \beta_i v_i$.\\
Cette combinaison lineaire existe car les $v_i$ forment une base.\\
Supposons $\gamma_1,\ldots, \gamma_n \in \mathbb{R}$ tel que
\[ 
	\sum \gamma_i x_i ( t) =0
\]

\end{proof}
Considerons maintenant $A = P \begin{pmatrix}
	\lambda_1 & & \\
		  & \ddots & \\
		  & & \lambda_n
\end{pmatrix} P^{-1}$, ou $P \in \mathbb{C}^{n\times n}, \lambda_1, \ldots, \lambda_n \in \mathbb{C}$.\\
Toute fonction $f: \mathbb{R}\to \mathbb{C}$ s'ecrit comme
\[ 
	f( t) = f_R( t)  + i f_I( t) 
\]
avec $f_R, f_I : \mathbb{R} \to \mathbb{R}$.\\
$f$ est derivable si $f_R$ et $f_I$ sont derivables.\\
\begin{rmq}
Si $x_1, \ldots, x_n: \mathbb{R}\to \mathbb{C}$ sont derivables, alors $x= \begin{pmatrix}
x_1\\\vdots\\x_n
\end{pmatrix} $ est une solution du systeme si
\[ 
x' = A \cdot x
\]

\end{rmq}
\begin{lemma}
Si $\lambda \in \mathbb{C}$ est une valeur propre de $A$ et si $v \in \mathbb{C}^{n}\setminus \left\{ 0 \right\} $ un vecteur propre correspondant, alors
\[ 
	x( t) = e^{\lambda t}  v
\]
est une solution complexe du systeme

\end{lemma}
\begin{proof}
\[ 
x' = \lambda e^{\lambda t}  v = e^{\lambda t}  A v= Ax
\]
\end{proof}
\begin{lemma}
Etant donne une solution complexe $x= x_R + i x_I $ du systeme, alors $x_R$ et $x_I$ sont des solutions reelles du systeme.
\end{lemma}
\begin{proof}
\[ 
	x'_R + i x_I' = x' = A x = A x_R + i A x_I
\]
\end{proof}
\subsection*{Marche a suivre pour la resolution d'un systeme lineaire, avec valeurs propres complexes}
\begin{itemize}
\item Soient $v_i = u_i + i w_i \in \mathbb{C}^{n}$ une base de vecteurs propres, alors on peut ecrire
	\[ 
	v_{2j -1} = \overline{v}_{ 2j } \text{ et } \lambda_{2j -1} = \overline{\lambda}_{2j} \quad i \leq  j \leq  k \leq \frac{n}{2}
	\]

\item $ \left\{ u_1, \ldots, u_k , w_1, \ldots, w_k, v_{2k+1} ,\ldots, v_n \right\} $ une base de $ \mathbb{R}^nu$ 
\item Soit $v= u + iw$ une solution avec $\lambda = a + ib$ 
	\begin{align*}
	v = e^{\lambda t}  v = e^{at } ( \cos ( bt)  + i \sin(  bt) ) \cdot ( u+iw) \\
	= e^{at } \left( ( \cos ( bt) u - \sin ( bt) w  ) +( \sin( bt) u + \cos ( bt) w)   \right) 
	\end{align*}
	
	
\end{itemize}








\end{document}	
