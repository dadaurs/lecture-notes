\documentclass[../main.tex]{subfiles}
\begin{document}
\lecture{15}{Tue 20 Apr}{Theoreme Spectral}
\begin{thm}[Theoreme spectral reel]
	Soit $A \in \mathbb{R}^{n\times n}$ symmetrique cad $A^{T}=A$, alors il existe $P \in \mathbb{R}^{n\times n}$ orthogonale tel que
	\[ 
	A = P D P^{T}
	\]
	Avec $D$ une matrice diagonale.\\
	Donc $A$ est semblable et congruente a une matrice diagonale.
	Pour $P=(P_1,P_2, \ldots ) $ les vecteurs colonne de $P$, $P_1, \ldots$ forment une base orthonormale de vecteurs propres de $A$, cad
	\begin{align*}
		A \cdot p_i &= P D P^{T}P_i \\
			    &=P \lambda_i e_i = \lambda_i P e_i
	\end{align*}
	
\end{thm}

\begin{defn}
Soit $K \subseteq \left\{ 1,\ldots,n \right\} $ et $A \in \mathbb{R}^{n\times n}$, ecrivons
\[ 
K = \left\{ l_1,\ldots, l_k \right\}  \text{ ou }  l_1 < l_2 < \ldots <l_k
\]
Alors $A_k \in \mathbb{R}^{k \times k}$, avec $a_{k,ij} = a_{l_i,l_j}  $.
\end{defn}
\begin{thm}
Soit $A \in \mathbb{R}^{n\times n}$ symmetrique.\\
$A$ est semi definie positive si 
\end{thm}
\begin{proof}
Soit $A \in \mathbb{R}^{n \times n} $ symmetrique et semi definie positive pour $K \subseteq \left\{ l_1,\ldots \right\} $.\\
$A_k$ est semi-definie positive, donc 
\[ 
A_K = P^{K^{T}} D'_k P_K
\]
Et donc
\[ 
	\det ( A_k) >0
\]
L'autre implication est identique au theoreme spectral.

\end{proof}
\begin{thm}
Soit $A \in \mathbb{R}^{n\times n}$ symmetrique et soit $f: \mathbb{R}^{n}\to \mathbb{R}$ definie par
\[ 
	f( x) = x^{T} A x
\]
, alors 
\[ 
	\max_{x \in S^{n-1}}  f(x)  = \lambda_1
\]
et
\[ 
	\min_{x \in S^{n-1}}  f(x)  = \lambda_n
\]
sont des valeurs propres qui satisfont
\[ 
\lambda_1 > \lambda_2 > \ldots > \lambda_n
\]


\end{thm}
\begin{proof}
	Si $P= ( p_1, \ldots, p_n) $ alors $ \left\{ p_1, \ldots \right\} $ est une base orthonormale de vecteurs propres de $A$. \\
	Soit $x \in \mathbb{R}^{n}$ et $ \N { x} _2^{2}=x^{T}x$ On peut donc reecrire
	\[ 
		x^{T}x = \sum_{i=1}^{ n} ( \alpha_i) ^{2}
	\]
	Donc, pour $x \in S^{n-1}$, on a
	\begin{align*}
		f( x) &= x^{T} \sum \beta_i \lambda_i p_i\\
		&= \sum_{i=1}^{ n}\beta_i ^{2} \lambda_i
	\end{align*}
	
	
\end{proof}
\begin{thm}[Theoreme Min-Max]
	Soit $A \in \mathbb{R}^{n\times n}$ symmetrique et $\lambda_1 \geq \lambda_2 \geq \ldots \geq \lambda_n$ les valeurs propres de $A$. Alors
	\begin{align*}
		\lambda_k &= \max_{U \leq \mathbb{R}^{n}, \dim( U) =k}\min_{x \in S^{n-1}\cap U}  x^{T}A x\\
			  &= \min_{U \leq  \mathbb{R}^{n}, \dim( U) = n-k }  \max_{x \in S^{n-1}\cap U}  x^{T} A x
	\end{align*}

\end{thm}
\begin{proof}
$\lambda_k$ est atteint par l'espace $span \left\{ p_1, \ldots, p_k \right\} $ et $p_k^{T} A p_k= \lambda_k$.
Pour 
\[ 
x= \sum_{i=1}^{ k} \alpha_i p_i \in span \left\{ p_1, \ldots,p_k \right\} \cap S^{n-1}
\]
Alors
\[ 
x^{T}A x = \sum_{i=1}^{ k} \alpha_i^{2}\lambda_i \geq \lambda_k	
\]
 
Donc
\[ 
\min_{x\in S^{n-1}, x \in span \left\{ p_1, \ldots, p_k \right\} } 
\]
Il reste a montrer que pour tout $U \subseteq \mathbb{R}^n$, on a 
\[ 
	\dim( U) = k \Rightarrow \min_{x\in S^{n-1}, x\in U} x^{T}A x \leq \lambda_k
\]

\end{proof}





\end{document}	
