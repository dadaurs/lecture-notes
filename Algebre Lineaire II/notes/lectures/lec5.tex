\documentclass[../main.tex]{subfiles}
\begin{document}
\lecture{5}{Tue 09 Mar}{Vecteurs/Valeurs Propres}
\begin{crly}
	Soit $f:V \to V$ un endomorphisme et $ \left\{ v_1, \ldots, v_n \right\} $ une base de $V$.\\
	Alors $f$ est diagonalisable si et seulement si il existe une matrice inversible $P \in K^{n\times n}$ tel que $P^{-1}A_BP $ est diagonale.
\end{crly}
\begin{proof}
$f$ est diagonalisable $\iff$ $\exists B'= \left\{ w_1, \ldots \right\} $ tel que $A_{B'} $ est diagonale.\\
Mais $A_{B'} = P^{-1} A_B P$
\end{proof}
\begin{defn}[Matrices semblables]
	$A,B \in K^{n\times n}$ sont semblables s'il existe $P \in K^{n\times n}$ inversible tel que
	\[ 
	P^{-1}A P = B
	\]
	
	
\end{defn}
Donc si $f$ est diagonalisable, la matrice de $f$ est semblable a une matrice diagonale.

\begin{defn}[Sous-espace propre]
	Soit $f:V \to V$ un endomorphisme et $\lambda$ une valeur propre de $f$, alors
	\[ 
		E_{\lambda} = \ker ( f- \lambda \cdot \id) 
	\]
	est l'espace propre de $f$ associe a $\lambda$.\\
	$\dim E_\lambda$ est la multiplicite geometrique de $\lambda$.
	
\end{defn}
\begin{lemma}
	Soit $f:V \to V$ un endomorphisme et $v_1, \ldots, v_r$ des vecteurs propres associes aux valeurs propres $\lambda_1, \ldots, \lambda_r$ distinctes.\\
	Alors $ \left\{ v_1, \ldots,v_r \right\} $ est un ensemble libre.
\end{lemma}
\begin{proof}
$r=1$ est evident.\\
Pour $r=2$ :\\
Supposons que $v_1,v_2$ sont lineairement dependants, alors il existe $\exists \alpha_1, \alpha_2 \in K\setminus \left\{ 0 \right\} $ tel que
\[ 
\alpha_1 v_1 + \alpha_2 v_2 = 0
\]
Spg $\lambda_2 \neq 0$, en appliquant f, on trouve
\begin{align*}
0 = \alpha_1 f( v_1) + \alpha_2 f( v_2) \\
0 = \alpha_1 \frac{\lambda_1}{\lambda_2}v_1 + \alpha_2 v_2\\
0 = \alpha_1 (  1- \frac{\lambda_1}{\lambda_2}) v_2
\end{align*}
Pour $r>2$ \\
Supposons l'assertion est fausse et soit $r>2$ minimal tel que $v_1, \ldots,v_r$ sont lin. dependants..
Soit
\[ 
\alpha_1 v_1 + \ldots = 0 
\]
avec $\alpha_i \neq 0$ $\forall i$, alors
\[ 
0 = \alpha_1 \frac{\lambda_1}{\lambda_r}v_1 + \ldots + \alpha_r v_r
\]
En soustrayant les deux egalites, on trouve
\[ 
	0 = \alpha_{1} ( 1-\frac{\lambda_1}{\lambda_r}) v_1 + \ldots 
\]

Ce qui contredit la minimalite.


\end{proof}
\begin{crly}
Soit $f:V \to V$ un endomorphisme de $V$ sur $K$ et $\dim V =n$.\\
Soient $\lambda_1, \ldots,$ les valeurs propres differentes de $f$.\\
Soit $n_1 \ldots$ les multiplicites geometriques respectives.\\
Soient $B_i = \left\{ v_1^{( i) }, \ldots, v_{n_i}^{( i ) } \right\} $ des bases de $E_{\lambda_i} $, alors
\[ 
	\bigcup_i B_i
\]
est un ensemble libre.\\
$f$ est diagonalisable $\iff$ $n_1 + \ldots + n_r=n$

\end{crly}
\begin{proof}
Soit
\[ 
	\sum_{i=1}^{ r} \sum_{j=1}^{ n_i} \alpha_{ij} v_{j} ^{( i) }=0
\]
Montrons que $\alpha_{ij} = 0 \forall i,j$
"Immediat" par lemme d'avant.\\
On remarque immediatement que si $\sum n_i = n$, les vecteurs propres forment une base.\\
A l'inverse, soit $f$ diagonalisable, cad il existe une base $B$ de $V$ composee de vecteurs propres. Soit  $m_i = |B \cap E_{\lambda_{i} }| $, donc $m_i$ est le nombre de vecteurs dans $B$ associe a $\lambda_i$.\\
Clairement $\sum m_i = n$, mais $m_i \leq n_i \leq \dim E_{\lambda_i} $, donc $\sum n_i  = n$.

\end{proof}

	




\end{document}	
