\documentclass[../main.tex]{subfiles}
\begin{document}
\lecture{17}{Tue 27 Apr}{Valeurs singulieres}
$A \in \mathbb{C}^{m\times n}$, avec $A = P D Q^{*}$ avec $D$ diagonale, $P$ unitaire.\\
On a defini 
\[ 
A^{+} = Q D^{+} P^{*}
\]
avec $D^{+}= \begin{pmatrix}
	\frac{1}{\sigma_1} & & \\
			   & &\ddots &\\
			   & & &\frac{1}{\sigma_n}
\end{pmatrix} $
\subsection{Encore des systemes d'equation}
On essaie a nouveau de resoudre
\[ 
A x = b, \quad A \in \mathbb{C}^{m\times n}, b \in \mathbb{C}^{m}
\]
On veut trouver
\[ 
\min_{x \in \mathbb{C}^{n}} \N { A x -b} ^{2}
\]
On a, entre autre resolu $A ^{T}A x = A^{T}b$.\\
On va utiliser la pseudo-inverse de $A$ pour trouver la solution.\\ 
On veut trouver $x \in \mathbb{C}^{n}$ la solution optimale tel que $\N { x} $ est optimale.\\
\begin{thm}
Soit $A \in \mathbb{C}^{m\times n}, b \in \mathbb{C}^{m}$, alors $x= A^{+}b$	est une solution optimale de norme minimale parmi les solutions du systeme $Ax = b$.
\end{thm}	
\begin{proof}
Soit $x \in \mathbb{C}^{n}$ et $Q \in \mathbb{C}^{n\times n}$ unitaire, alors
\[ 
\N { x} ^{2}= x^{*}x = x^{*}Q^{*}Q x = \N { Qx} ^{2}
\]
On a donc
\begin{align*}
	\min_{x \in \mathbb{C}^{n}} \N { Ax -b} &= \min _{x \in \mathbb{C}} \N { PD \underbrace{Qx}_{ \coloneqq y} - b} \\
	&= \min_{y \in \mathbb{C}^{n}} \N { Dy - P^{*}b} \\
	&= \min_{y \in \mathbb{C}^{n}} \N { D y -c} 
\end{align*}
De plus $y$ est une solution optimale $\iff$ $y_{r+1} = \ldots = y_n =0$\\
Et alors, $x  = Q^{*}y = Q^{*}D^{+}P^{*}b$ est la solution optimale de norme minimale unique du probleme.

\end{proof}
\subsection{Le meilleur sous-espace approximatif}
Etant donne $a_1, \ldots, a_m\in \mathbb{R}^{n}, 1 \leq  k \leq n$.\\
On veut trouver un sous-espace $H \subseteq \mathbb{R}^n$, $\dim  H \leq k$ tel que
\[ 
	\sum_{i=1}^{ m}d ( H,\alpha_i) ^{2}
\]
est minimale.\\
On choisit une base orthonormale de $H$: $ \left\{ u_1, \ldots, u_k \right\} $, on peut facilement trouver la projection sur $U$, avec
\[ 
	proj( a_i) = \sum_{j=1}^{ k} \eng { \alpha_i,u_j}  u_j
\]
Grace au theoreme de pythagore, on a
\[ 
	\N { a_i} ^{2} = \N { proj( a_i) } ^{2}+ d ( a_i,H) ^{2}
\]
Donc
\begin{align*}
	\sum_{i=1}^{ m}\N { a_i} ^{2} &= \sum_{i=1}^{ m}\N { proj( a_i) } ^{2} + \sum_{i=1}^{ m} d ( a_i,H) ^{2}\\
				      &= \underbrace{\sum_{i=1}^{ m} \sum_{j=1}^{ k} ( u_j^{T}a_i) ^{2}}_{ \text{ A maximiser } } + \sum_{i=1}^{ m}d ( a_i,H) ^{2}
\end{align*}
On veut trouver un $H \subseteq \mathbb{R}^n$ tel que
\[ 
\sum_{j=1}^{ k} u_j^{T}A^{T}A u_j
\]
avec $A = \begin{pmatrix}
a_1^{T}\\ \vdots \\ a_m^{T}
\end{pmatrix} $.\\
On veut maintenant trouver $H \subseteq \mathbb{R}^n, \dim H = k$ et avec n'importe quelle base orthogonale tel que
\[ 
\sum_{j=1}^{ k} u_j^{T} A^{T} A u_j
\]
est maximale.\\
\subsubsection{$k=1$}
On veut trouver 
\[ 
\max_{u \in S^{n-1} } u^{T}A^{T}A u
\]
Avec le theoreme spectrale, on trouve la valeur propre maximale, et alors le sous-espace propre associe est solution.
Par recurrence, on a
\begin{align*}
\sum_{j=1}^{ k} w_j^{T} A^{T}A w_j &= \sum_{j=1}^{ k-1} w_j ^{T}A^{T}Aw_j + w_k ^{T}A^{T}A w_k\\
				   &\leq  \sum_{j=1}^{ k-1}u_j^{T}A^{T} A u_j + u_k ^{T} A^{T} A u_k
\end{align*}



\end{document}	
