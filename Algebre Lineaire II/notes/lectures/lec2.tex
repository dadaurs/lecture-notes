\documentclass[../main.tex]{subfiles}
\begin{document}
\lecture{2}{Wed 24 Feb}{Polynomes}
\begin{thm}
	Soit $K$ un corps fini de characteristique $q$, alors $K \supseteq \mathbb{Z}_q$.\\
	De plus $K$ est un espace vectoriel de $\mathbb{Z}_q$ de dimension finie.\\
\end{thm}
\begin{crly}
Soit $K$ un corps infini. Deux polynomes sont egaux si et seulement si leurs evaluations sont les memes.
\end{crly}
\begin{proof}
Une direction est triviale.\\
L'autre suit immediatement du theoreme 1.6
\end{proof}
\subsection{Division avec reste}

\begin{thm}
	Soit $R$ un anneau, $f,g \in R[x], g\neq0$ et soit le coefficient de $g \in R^{*}$ \\
	Il existe $q,r \in R[x]$ uniques tel que
	\begin{enumerate}
		\item $f( x) =q( x) g( x) + r( x) $ 
		\item $\deg r < \deg g$
	\end{enumerate}
	
\end{thm}
\begin{proof}
	Si $\deg f < \deg g$, on a fini.\\
	Soit donc $\deg f \geq g$, donc
	\[ 
		f( x) = a_0 + \ldots + a_n x^{n}
	\]
	et 
	\[ 
		g( x) = b_0 + \ldots b_m x^{m}
	\]
	et $b_m^{-1}$ existe.\\
	On procede par induction sur $n$.\\
	Si $n=m$:\\
	On note que
	\[ 
		f( x) - \frac{a_n}{b_m} g( x) 
	\]
	est un polynome de degre $<n$ 
	Si $n>m$:\\
	On note que
	\[ 
		f( x) - \frac{a_n}{b_m}x^{n-m}g( x) 
	\]
	est un polynome de degre $<n$.\\
	Par hypothese d'induction il existe $q( x) ,r( x) $ tel que
	\begin{itemize}
		\item $f( x) - \frac{a_n}{b_m}x^{n-m}g( x) + r( x)  $
		\item $\deg r <\deg g$
	\end{itemize}
	et donc on a fini de montrer l'existence.\\
	Supposons maintenant qu'il existe $r'$ et $q'$ satisfaisant les memes proprietes que $q$ et $g$, alors on a
	\[ 
		q( x) g( x)  + r( x)  = q'( x) g( x) + r'( x) 
	\]
	Donc
	\[ 
	r' \neq r \text{ et  } q' \neq q
	\]
	en comparant les degre, on a une contradiction.
		
\end{proof}
\subsection{Factorisation des polynomes sur un corps}
\begin{defn}[Diviseurs de polynomes]
	Soit $q( x) \in K[x]$.\\
	$q$ divise $f$ si il existe $g( x) $ tel que
	\[ 
		q( x) g( x) = f( x) 
	\]
	On dit que $q$ est un diviseur de $f$, on ecrit $q( x) |f( x) $
	
\end{defn}
\begin{defn}[Racine]\label{defn:Racineracine}
	Soit $p( x) \in K[x]$, et soit $\alpha \in K$ tel que $p( \alpha ) =0$
\end{defn}
\begin{thm}
	Soit $f( x) \in K[x] \setminus  \left\{ 0 \right\}  $, alors $\alpha \in K$ est une racine de $f$ si et seulement si $( x-a) |f( x) $
\end{thm}
\begin{proof}
	Si $( x-\alpha) q( x) = f( x) $, alors on a fini.\\
	sinon, la division de $f( x)$ par $x-\alpha$ avec reste donne
	\[ 
		f( x) = q( x) ( x-\alpha) +r \text{ ou } r \in K
	\]
	Si $r\neq 0$, alors $f( \alpha) = q( \alpha) ( \alpha-\alpha) +r =r=0$ et donc $( x-a) | f( x) $	
\end{proof}
\begin{defn}[Multiplicite d'une racine]
	La multiplicite d'une racine $\alpha$ de $p( x) \in K[x]$ est le plus grand $i\geq 1$ tel que 
	\[ 
		( x-\alpha) ^{i} | p( x) 
	\]
	
	
\end{defn}
\begin{thm}[Theoreme fondamental de l'algebre]
	Tout polynome $p( x) \in \mathbb{C}[x]\setminus \left\{ 0 \right\} $ de degre $\geq 1$ possede une racine complexe.
	
\end{thm}





\end{document}
