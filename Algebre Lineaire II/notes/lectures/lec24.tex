\documentclass[../main.tex]{subfiles}
\begin{document}
\lecture{24}{Wed 19 May}{algebre lineaire sur les entiers}
\section{Algebre lineaire sur les entiers}

But:\\
Etant donne $A \in \mathbb{Z}^{m\times n}, b \in \mathbb{Z}^{m}$, on veut trouver $x \in \mathbb{Z}^{n}$ tel que
\[ 
Ax = b
\]
Si $m=n$, $\det A \neq 0$, alors on a une solution rationelle de la forme
\[ 
x= A^{-1}b
\]
Dans ce cas, le systeme $Ax = b$ est resoluble si et seulement si $A^{-1}b \in \mathbb{Z}^{m}$.\\
Mais que fait-on si le systeme est sous-determine?\\
Un probleme plus specifique, discute pour la premiere fois par Gauss est
\[ 
a,b,c \in \mathbb{Z} \setminus \left\{ 0 \right\} 
\]
\[ 
ax + by = c, x,y \in \mathbb{Z}.
\]
Le systeme est soluble si et seulement si $\gcd ( a,b) | c$
\begin{proof}
$\Rightarrow$ Supposons $d|a, d|b$.\\
Soit $x,y \in \mathbb{Z}$ une solution du systeme
\begin{align*}
	xa + yb &= c\\
	d( xa' + yb') &= c \Rightarrow d|c \Rightarrow \gcd( a,b) |c
\end{align*}
$\Leftarrow$ $\exists x', y' \in \mathbb{Z}$ tel que $\gcd( a,b) = d = x' a + y' b$
\end{proof}
Un algorithme pour trouver la solution est l'algorithme d'Euclide.
\subsection{Forme normale d'Hermite}
On veut resoudre des systemes de la forme $Ax=b, x \in \mathbb{Z}^{n}$.\\
On veut trouver $Q \in \mathbb{Q}^{n \times n}$ inversible tel que
\[ 
Aq = T
\]
Ou $T$ est une matrice triangulaire inferieure triangulaire.
\begin{defn}[Matrice unimodulaire]
	Une matrice $Q \in \mathbb{Z}^{n\times n}$ est unimodulaire si $\det Q = \pm 1$.
	
\end{defn}
\begin{lemma}
Soit $Q \in \mathbb{Z}^{n\times n}, \det Q \neq 0$, alors $Q^{-1} \in \mathbb{Z}^{n\times n}\iff Q$ est unimodulaire.
\end{lemma}
\begin{proof}
$\Leftarrow$ \\
On a 
\[ 
Q^{-1} = \frac{1}{\det Q} \tilde Q^{T}
\]
Or la matrice des cofacteurs possede comme composantes des determinants de matrices $( n-1) \times ( n-1) $ sous-matrices de $Q $.\\
Mais $\det A \in \mathbb{Z}$ pour $A \in \mathbb{Z}, \det A = \sum_{\sigma \in S_n}^{ }sign( \sigma) \prod_{i=1} ^{n}a_{i\sigma( i) } $.\\
Donc l'inverse de la matrice est une matrice integrale.\\
$\Rightarrow$ \\
\[ 
\id = \det \id = \det Q Q^{-1} = \det Q \det Q^{-1}
\]
Il en suit le theoreme.
\end{proof}
Donc, pour revenir au probleme $Ax = b, x \in \mathbb{Z}^{n}$, on a 
\[ 
A U U^{-1}x = b
\]
Donc $x \in \mathbb{Z}^{n}$ est solution du systeme si et seulement si
\[ 
U^{-1}x \in \mathbb{Z}
\]
est solution de $AU = b$.
Ajouter un multiple entier d'une colonne a une autre colonne.




\end{document}	
