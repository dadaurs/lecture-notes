\documentclass[../main.tex]{subfiles}
\begin{document}
\lecture{7}{Wed 10 Mar}{Polynome caracteristique}
\section{Le polynome caracteristique}
Soit $A$ une matrice $n \times n$, $\lambda \in K$ est une valeur propre de l'endomorphisme defini par $A$ si et seulement si $\ker( A - \lambda\id) \supsetneq \left\{ 0 \right\}  $.
On note
\[ 
	\det( A- \lambda I) = \sum_{\pi \in S_n}  \sgn( \pi ) \prod_{i=1}^{n}( A-\lambda \id)_{i\pi( i) } 
\]
On observe que $\lambda$ est une valeur propre de $f$ si et seulement si $\lambda$ est une racine de $p_A$.\\
Soit $f:V \to V$ un endomorphisme, $B= \left\{ v_1, \ldots \right\} $ une base de $V$. Le polynome caracteristique de f est donne par
\[ 
	\det ( A_B - \lambda \id) 
\]

Cette definition fait du sens, car le changement de base n'influence pas la valeur du determinant.
\begin{defn}[Multiplicite algebrique]
	La multiplicite algebrique d'une valeur propre est la multiplicite comme racine du polynome caracteristique.
\end{defn}
\begin{propo}
Soit $f$ un endomorphisme de $V \to V$.\\
Soit $\lambda \in K$ une valeur propre.\\
La multiplicite geometrique de $\lambda$ est au plus la mutliplicite algebrique.
\end{propo}
\begin{proof}
Soit $ \left\{ v_1, \ldots, v_r \right\} $ une base de $E_\lambda$, on complete cette base en une base de $V$ avec $ \left\{ w_1, \ldots, w_{n-r}   \right\} $.
Dans cette base, la representation de la matrice de $A- \lambda \id$ implique que
\[ 
	\det (  A- x \id) = ( \lambda -x) ^{r} \det C
\]
et donc $r$ est au plus la multiplicite algebrique.

\end{proof}
\begin{thm}[Theoreme de diagonalisation]
	Soit $V$ un espace vectoriel sur $K$ de dimension $n$, $f:V \to V$ un endomorphisme  $\lambda_1, \ldots \in K$ les valeurs propres distinctes, alors $f$ est diagonalisable si et seulement si
	\begin{itemize}
		\item $p_f( x)= ( -1) ^{n}\prod_{i=1}^{r}( x-\lambda_i)^{g_i}$ 
		\item $\dim E_{\lambda_i}= g_i$ pour tout $i$
	\end{itemize}
	
	
\end{thm}
\begin{proof}
	Soit $f$ diagonalisable et soit $B= \left\{ v_1, \ldots \right\} $ une base composee de vecteurs propres. $A_B$ est une matrice diagonale, alors $p_f( x) = \det( A_B- x \id) = ( -1)^{n} \prod ( \lambda_i -x)^{g_i} $.\\
	De plus $\dim( \ker( A_B- \lambda_i \id) ) = g_i$\\

	Soient $m_i$ les multiplicites geometriques des valeurs propres.\\
	car 
	\[ 
		\deg ( p_f ) =n
	\]
	on a fini.
\end{proof}





\end{document}	
