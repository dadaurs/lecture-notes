\documentclass[../main.tex]{subfiles}
\begin{document}
\lecture{1}{di 29 jul 16:00}{Title of the lecture}



\begin{center}
	\textbf{Exercice 1.4\\}
\end{center}
$\forall n \in \mathbbm{N} $
$ \exists a_1 ,..., a_k $  such that 
\begin{align*}
\sum_{k=0}^{N}a_k 10^{k} = n
\end{align*}
Thus
\begin{align*}
	n &= \sum_{k=1}^{N} a_k (10^{k}-1) + \sum_{k=0}^{N} a_k\\
\end{align*}
If 9 | n $\Rightarrow$ $ n = 9u$ 
\begin{align*}
	9u &= \sum_{k=1}^{N} 9 a_k 10^{k-1} + \sum_{k=0}^{N}a_k \Rightarrow 9 | \sum_{k=0}^{N} a_k	
\end{align*}
\begin{center}
\textbf{Exercice 1.6\\
$ (333)^{\frac{1}{3}}$ irrationel}
\end{center}
Par l'absurde, supposer que
\[ 
\exists a,b in \mathbbm{N} | \frac{a}{b} = 333^{\frac{1}{3}}
\]
$\frac{a}{b}$ sous forme simplifiee, $\Rightarrow$ \\
\begin{align*}
	\frac{a^3}{b^3} &= 333\\
	a^3 &= 333 b^3\\
	3 | a \Rightarrow 27 | a^3 \Rightarrow 3 | b
\end{align*}
\contra
Contradiction car alors $\frac{a}{b}$ pas simplifie
\begin{center}
\textbf{Exercice 4.1\\
Theoreme d'Euclide et de la hauteur}
\end{center}
\textit{Prouver les deux theoremes}\\
On sait que
\begin{align}
a^{2} + b^{2} = c^{2}\\
a \cdot b  = c \cdot h\\
b^{2} = b'^{2} + h^{2}\\
a^{2} = a'^{2} + h^{2}\\
\frac{b'}{b} = \frac{h}{a}\\
\frac{a}{a'} = \frac{h}{a}
\end{align}
Par 3 et 4, on sait que
\begin{align*}
b^{2} = b'^{2} + a^{2} - a'^{2}\\
b^{2} - a^{2}= (b'-a')(b'+a')\\
b^{2} -a^{2} = b'\cdot c - a' \cdot c\\
b^{2} = b' \cdot c - a' \cdot c\\
b^{2} = b' c - \frac{ahc}{b} + a^{2}\\
\Rightarrow ahc = a^{2}b\\
\Rightarrow b^{2} = b'c = a^{2}\frac{b}{b} + a^{2} = b'c
\end{align*}

\textbf{Theoreme de la hauteur}
\begin{align*}
	ab &= ch
	5 \Rightarrow  b' = \frac{hb}{a} 6 \Rightarrow a' = \frac{ah}{b}\\
	   &\Rightarrow  a' \cdot b' = \frac{bhah}{ba} = h^{2}
\end{align*}

\begin{center}
\textbf{Exercice 4.4}
\end{center}

\begin{equation}
\begin{cases}
	3x + y -22 &= 0\\
	x - 4y + 10 &= 0\\
\end{cases}
\end{equation}
\begin{align*}
13y - 52 = 0\\
y = 4\\
\Rightarrow x-6 = 0\\
\Rightarrow \vec{AI} = 
\begin{pmatrix}
4\\
-1
\end{pmatrix}\\
\Rightarrow 4x - y = 7 \mathnote{Substitute value for P}
\end{align*}
\hr





\begin{align*}
\begin{pmatrix}
2 \\
3
\end{pmatrix}
\begin{pmatrix}
1\\
3\\
4 \\
6
\end{pmatrix}
\end{align*}


\begin{propo}[exemple]\label{propo:exemple}
	a l aise ou quoi
\end{propo}



\end{document}
