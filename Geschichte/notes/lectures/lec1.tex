\documentclass[../main.tex]{subfiles}
\begin{document}
\lecture{1}{Wed 12 Oct}{Babylon}
Doe <atje,atolgescjocjte hat ihre eigene Geschichte.\\
Montucla beginnt mit einer Reflexion darueber, was Mathematik ist, in der er sich auf das klassiche Griechenland bezieht.\\
Diese Vorlesung wird ohne vorgegeben Definition dessen beginnen, was Mathematik ist.
\subsection*{Was sind die primaeren Objekte der Mathematikgeschichte}
Die Objekte der Mathematikgeschichte sind mathematische Texte.\\
Ob ein vorgelectes Objekt ein mathematischer Text is oder nicht, muss ggbf diskutiert werden.\\
Sobald man sich nicht um mit schriftlichen Objekten kuemmert, wierd es kompliziert.
\section{Mathematik in Mesopotamien}
ca. -3200 bis ca. -600.\\
\begin{itemize}
\item Sumerer: seit -4000, Keilschrift beginnt ca. -3200.\\
	Ab ende des -3. Jahrhunderts wird sumerisch nicht mehr gesprochen aber weiter als amtssprache benutzt wird ( wie Latein), bis ca. -1750.
\item Adier: Das Adische ist eine semitische Sprache; aehnlich zu Hebraeisch und Arabisch.
\end{itemize}
Die mathematischen Taefelchen beinhalten
\begin{itemize}
\item Schultexte ( fuer schreiber) 
\item Offizielle Texte
\item Praktische texte ( von Kauflaeuten) 
\end{itemize}

Diese Texte haben zwei Zeichen, ein "Keil" $\Gamma$ und ein "Winkelhaken" "C"\\
Die Schreibung von Zahlen, ist eine Mischung von basis 10 und 60.\\
Dieses system hat in unserer zeit ueberlebt (60s sind eine Minute, etc) \\
In dieser Kultur ist Mathematik einer von drei Wissensgebiete: Wahrsagung, Medizin und Mathemati



\end{document}	
