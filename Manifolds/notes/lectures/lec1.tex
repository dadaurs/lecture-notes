\documentclass[../main.tex]{subfiles}
\begin{document}
\lecture{1}{Wed 12 Oct}{Introduction}
\section{Recap}
Recall theorems about differentiable maps
\begin{itemize}
\item Implicit function theorem\\
	For $U \subset \mathbb{R}^{p},V\subset \mathbb{R}^{q}$, $f\in C^{k}( U\times V, \mathbb{R}^{q}) , 1 \leq k \leq \infty $ and $ ( a,b) \in U\times V$ st.
	\[ 
	D_2 f( a,b) = D( f( a, -) ) ( b) 
	\]
	is invertible. Then there exists $a\in U_1 \subset U,b \in V_1 \subset V$ and $\phi\in C^{k}( U_1,V_1) $ such that
	\[ 
	f( x,x') = y_0
	\]
	iff $x'= \phi( x) $ 

\item Inverse function theorem\\
	If $U \subset \mathbb{R}^{p}$ is open and $f\in C^{k}( U, \mathbb{R}^{q}) , 1 \leq k \leq \infty, a \in U$ such that
	\[ 
	Df( a) 
	\]
	is invertible, then there are $a \in U_1 \subset U$ and $f( a) \in V_1 \subset \mathbb{R}^{q}$  open such that 
	\[ 
	f|_{U_1} : U_1\to V_1
	\]
	is a diffeomorphism and
	\[ 
	Df^{-1}|_U ( x) = ( Df( f^{-1}|_U ( x) ) ) ^{-1}
	\]
	for all $x\in U$ in particukar $f^{-1}$ is $C^{k}$ 
\item Rank theorem\\
	$U \subset \mathbb{R}^{p}$ open and $f\in C^{k}( U, \mathbb{R}^{q}) , 1 \leq k \leq \infty $, $a\in U, b \coloneqq f( a) , r= rank( Df( a) ) $ 
	then there are diffeomorphisms
	\[ 
	\psi: U_\psi\to V_\psi \text{ and } \phi:U_\phi\to V_\psi
	\]
	with $U_\psi,V_\psi\subset \mathbb{R}^{p}$ and $U_\phi,V_\phi \subset \mathbb{R}^{q}$ such that
	\[ 
		\phi\circ f\circ \psi( x_1,\ldots,x_p) = ( x_1,\ldots,x_r, \tilde{f}( x_1,\ldots,x_p) ) 
	\]
	If $rk( D( f) ) $ is contant around $r$, then we can obtain $\tilde f =0$ 
	
\end{itemize}
\section{Manifolds}
\begin{defn}[Basis]
	A basis for a topology on $X$ is a collection $B$ of open sets such that every open set in $X$ is the union of sets in $B$.\\
\end{defn}
$X$ is called second countable if it has a countable topological basis.
\begin{defn}[Chart]
	Let $X$ be a topological space
	\begin{enumerate}
	\item A chart on $X$ is a  pair  $( U,\phi) $ where $U \subset X$ open and $\phi: U \to \mathbb{R}^{n}$ for some $n$ which is a homeomorphism onto an open subset.
	\item An atlas is a collection of charts $ A= \left\{ ( U_i,\phi_i) |i \in I \right\} $ such that $X= \bigcup_{i \in I} U_i$ 
	\item $A$ is called smooth (  $C^{k}$ ,continuous, holomorphic, algebraic,\ldots) if and only if for any
		\[ 
			( U_i,\phi_i)_{i \in \left\{ 1,2 \right\} } \in A
		\]
		we have $\phi_1\circ\phi_2^{-1}$ is smooth ( $C^{k}$ ,\ldots) wherever it is defined.
	\item A chart $( U,\phi) $  is compatible with an atlas $A$ if and only if 
		\[ 
		A \cup \left\{ ( u,\phi)  \right\} 
		\]
		is smooth
	\item An atlas $A$  is maximal if it contains all charts compatible with $A$.
	For any atlas $A$ ( not necessarily maximal), denote $A_{max} $ the maximal atlas containing it.\\
	This maximal atlas is necessarily unique
	\end{enumerate}
\end{defn}
\begin{defn}[Manifold]
	A smooth manifold of dimension $n$   is a second countable Hausdorrf space with a maximal smooth atlas of dimension $n$.
\end{defn}
\subsection*{Why Hausdorff?}
Consider $ \mathbb{R} /\sim$, $x\sim y \iff |x| = |y| >1$, this space is locally homeomorphic to $\mathbb{R}$ but the points $x$ and $y$ cannot be separated.
\subsection*{Why second countable?}
Take a disjoint union of infinitely many manifolds.\\
For a connected example, take $\aleph_1 \times [ 0,1)$ with the order topology.\\
\subsection{Smooth maps}
A function $f:M\to N$ between smooth manifolds is called smooth if for each $p \in M$, there are charts $( U,\phi), ( V,\psi) $  $p\in U \subset M, f( p) \in V\subset N$ such that 
\[ 
\psi\circ f \circ\phi^{-1}
\]
is smooth.\\

$f$ smooth implies $\tilde\psi\circ f \circ \tilde\phi^{-1}$ is smooth for any charts $( \tilde U, \tilde\phi) ,( \tilde V,\tilde\psi) $ where this is defined.



\end{document}	
