\documentclass[11pt, a4paper]{article}
\usepackage[utf8]{inputenc}
\usepackage[T1]{fontenc}
\usepackage{babel}
\usepackage{geometry}
\usepackage{lmodern}
\usepackage{tikz}
\usepackage{tikz-cd}
\usepackage{quiver}
\usepackage{amsmath}
\usepackage{mathtools}
\usepackage[style=alphabetic,sorting=nyt]{biblatex}
\renewcommand{\vec}[1]{\overrightarrow{#1}}
\newcommand{\del}{\partial}
\DeclareMathOperator*{\id}{Id}
\newcommand{\ev}{\mathrm{ev}}
\DeclareMathOperator*{\cmod}{mod}
\newcommand{\sq}{\mathrm{Sq}}
\usepackage{hyperref}
\usepackage{color}
\RequirePackage{xcolor}
\usepackage[amsmath,hyperref,thmmarks]{ntheorem}


\newtheorem{thm}{Theorem}
\newtheorem{defn}{Definition}

\newtheorem{propo}[thm]{Proposition}
\newtheorem{notn}[thm]{Notation}
\newtheorem{rmk}[thm]{Remark}

\newtheorem{crly}[thm]{Corollary}

\newtheorem{lemma}[thm]{Lemma}

\renewcommand{\qedsymbol}{{$\Box$}}
\theoremstyle{plain}	
\theorembodyfont{\upshape}
\theoremsymbol{\ensuremath{ \Box }}
\newtheorem*{proof}{Proof}
\theoremsymbol{}

\usepackage{mathpazo}
\usepackage{euler}
\addbibresource{../sources.bib}

	
\newcommand\norm[1]{\left\vert#1\right\vert}
\newcommand\ns[1]{\left\vert\left\vert\left\vert#1\right\vert\right\vert\right\vert}
\newcommand\Norm[1]{\left\lVert#1\right\rVert}
\newcommand\N[1]{\left\lVert#1\right\rVert}
\newcommand\abs[1]{\left\vert#1\right\vert}
\newcommand\inj{\hookrightarrow}
\newcommand\surj{\twoheadrightarrow}
\newcommand\ded[1]{\overset{\circ}{#1}}

\hypersetup{
bookmarks,
bookmarksdepth=3,
bookmarksopen,
bookmarksnumbered,
pdfstartview=FitH,
colorlinks,backref,hyperindex,
%linkcolor=Sepia,
%anchorcolor=BurntOrange,
%citecolor=MidnightBlue,
%citecolor=OliveGreen,
%filecolor=BlueViolet,
%menucolor=Yellow,
%urlcolor=OliveGreen
linkcolor={red!80!black},
citecolor={green!70!black},
menucolor=Yellow,
urlcolor={blue!70!black}
}


\usepackage{abstract}
\renewcommand{\abstractname}{}    % clear the title
\renewcommand{\absnamepos}{empty} % originally center

%\usepackage{amssymb}
%\usepackage{amsthm}
\begin{document}
\title{The Steenrod Algebra and Its Dual }
\author{David Wiedemann}
\maketitle
\begin{abstract}
	These are notes for the seminar "Advanced Topics in Homotopy Theory" given by Prof. Stefan Schwede and Dr. Jack Davies in Bonn during the WS2023/24. Our goal is to present the main results of Milnor's paper ``The Steenrod Algebra and its Dual'' \cite{steenrod_algebra_and_its_dual_milnor}.
\end{abstract}
\tableofcontents

\section{Hopf Algebras}
\subsection{Bi-Algebras}
We start by studying Hopf algebras independently. 
Throughout, let $k$ be a field.
\begin{defn}[Algebra]
	An \textbf{Algebra} is a triple $\left( \mathcal{A},\mu, \eta\right) $ with $\mathcal{A}$ a $k$-vector space together with two maps   $\mu\colon A\otimes A \to A$ (multiplication) , $\eta\colon k \to A$ (unit)  making the following diagrams commute
	% https://q.uiver.app/#q=WzAsNCxbMCwwLCJcXG1hdGhjYWx7QX0gXFxvdGltZXMgXFxtYXRoY2Fse0F9XFxvdGltZXMgXFxtYXRoY2Fse0F9Il0sWzEsMCwiQVxcb3RpbWVzIFxcbWF0aGNhbHtBfSJdLFswLDEsIlxcbWF0aGNhbHtBfVxcb3RpbWVzIFxcbWF0aGNhbHtBfSJdLFsxLDEsIlxcbWF0aGNhbHtBfSJdLFswLDEsIlxcaWRcXG90aW1lc1xcbXUiXSxbMSwzLCJcXG11Il0sWzIsMywiXFxtdSIsMl0sWzAsMiwiXFxtdVxcb3RpbWVzIFxcaWQiLDJdXQ==
\[\begin{tikzcd}
	{\mathcal{A} \otimes \mathcal{A}\otimes \mathcal{A}} & {A\otimes \mathcal{A}} \\
	{\mathcal{A}\otimes \mathcal{A}} & {\mathcal{A}}
	\arrow["\id\otimes\mu", from=1-1, to=1-2]
	\arrow["\mu", from=1-2, to=2-2]
	\arrow["\mu"', from=2-1, to=2-2]
	\arrow["{\mu\otimes \id}"', from=1-1, to=2-1]
\end{tikzcd}\]
% https://q.uiver.app/#q=WzAsNCxbMCwwLCJrXFxvdGltZXMgXFxtYXRoY2Fse0F9Il0sWzEsMCwiXFxtYXRoY2Fse0F9XFxvdGltZXNcXG1hdGhjYWx7QX0iXSxbMiwwLCJcXG1hdGhjYWx7QX1cXG90aW1lcyBrIl0sWzEsMSwiXFxtYXRoY2Fse0F9Il0sWzIsMSwiXFxldGFcXG90aW1lcyBpIiwyXSxbMCwxLCJpXFxvdGltZXNcXGV0YSJdLFswLDNdLFsyLDNdLFsxLDMsIlxcbXUiLDFdXQ==
\[\begin{tikzcd}
	{k\otimes \mathcal{A}} & {\mathcal{A}\otimes\mathcal{A}} & {\mathcal{A}\otimes k} \\
	& {\mathcal{A}}
	\arrow["{\eta\otimes i}"', from=1-3, to=1-2]
	\arrow["i\otimes\eta", from=1-1, to=1-2]
	\arrow[from=1-1, to=2-2]
	\arrow[from=1-3, to=2-2]
	\arrow["\mu"{description}, from=1-2, to=2-2]
\end{tikzcd}\]
\end{defn}
Dualizing these definitions, we unsurprisingly obtain
\begin{defn}[Coalgebra]
	A \textbf{coalgebra} is a triple $( C,\Delta,\epsilon) $ where $C$ is a $k$-vector space togethere with two maps $\Delta\colon C \to C \otimes C$ (comultiplication) and $\epsilon\colon C \to k$ (augmentation) making the following diagrams commute
% https://q.uiver.app/#q=WzAsNCxbMCwwLCJDIl0sWzEsMCwiQ1xcb3RpbWVzIEMiXSxbMCwxLCJDXFxvdGltZXMgQyJdLFsxLDEsIkNcXG90aW1lcyBDIFxcb3RpbWVzIEMiXSxbMCwxLCJcXERlbHRhIl0sWzEsMywiXFxpZCBcXG90aW1lcyBcXERlbHRhIl0sWzIsMywiXFxEZWx0YSBcXG90aW1lcyBcXGlkIiwyXSxbMCwyLCJcXERlbHRhIiwyXV0=
\[\begin{tikzcd}
	C & {C\otimes C} \\
	{C\otimes C} & {C\otimes C \otimes C}
	\arrow["\Delta", from=1-1, to=1-2]
	\arrow["{\id \otimes \Delta}", from=1-2, to=2-2]
	\arrow["{\Delta \otimes \id}"', from=2-1, to=2-2]
	\arrow["\Delta"', from=1-1, to=2-1]
\end{tikzcd}\]
% https://q.uiver.app/#q=WzAsNCxbMSwwLCJDIl0sWzAsMSwia1xcb3RpbWVzIEMiXSxbMSwxLCJDXFxvdGltZXMgQyJdLFsyLDEsIkNcXG90aW1lcyBrIl0sWzAsMiwiXFxEZWx0YSJdLFswLDFdLFswLDNdLFsyLDMsIlxcaWQgXFxvdGltZXMgXFxlcHNpbG9uIiwyXSxbMiwxLCJcXGVwc2lsb24gXFxvdGltZXMgXFxpZCJdXQ==
\[\begin{tikzcd}
	& C \\
	{k\otimes C} & {C\otimes C} & {C\otimes k}
	\arrow["\Delta", from=1-2, to=2-2]
	\arrow[from=1-2, to=2-1]
	\arrow[from=1-2, to=2-3]
	\arrow["{\id \otimes \epsilon}"', from=2-2, to=2-3]
	\arrow["{\epsilon \otimes \id}", from=2-2, to=2-1]
\end{tikzcd}\]
\end{defn}
Since taking duals commutes with tensor products, notice that the dual  $C^{\vee}$ naturally gets an algebra structure.\\
We define (co-)algebra morphisms in the obvious way.
\begin{defn}[Bialgebra]
	A \textbf{bialgebra} is a tuple $( \mathcal{A}, \mu, \eta, \Delta, \epsilon) $ such that $( \mathcal{A}, \mu,\epsilon)$ is an algebra, $( \mathcal{A}, \Delta, \epsilon) $ is a coalgebra and such that $\Delta$ and $\epsilon$ are algebra morphisms
\end{defn}
Equivalently, one can also require $\mu$ and $\epsilon$ to be coalgebra morphisms.\\
If $\mathcal{A}= \bigoplus_{n \in \mathbb{N}}  \mathcal{A}_n$ is a graded algebra, we define the \textbf{dual algebra} by
\[ 
\mathcal{A}^{\ast} \coloneq A_n^{\ast}, \text{ with } A_n^{\ast}=\hom( A_{-n} , k) 
\]
We call a graded algebra $\mathcal{A}$ \textbf{graded commutative} if for all homogeneous elements $\alpha,\beta \in \mathcal{A}$, we have $\alpha \beta= (-1)^{\dim \alpha\dim\beta}\beta \alpha$. (omitting $\mu$ for sanity reasons)
The graded algebra $\mathcal{A}$ is \textbf{connected} if $\mathcal{A}_0$ is generated by $1$, equivalently $\eta\colon k \to \mathcal{A}_0$ is an isomorphism.\\
We can similarly define the notion of a  graded coalgebra and of a connected coalgebra.


\subsection{Antipode maps}
Let $C$ be a bi-algebra as above and let $f,g\colon C\to C$ be linear maps, we define the convolution $f\ast g$ of $f$ with $g$ as the composition
\[ 
	C \xrightarrow{\Delta} C\otimes C \xrightarrow{f\otimes g} C\otimes C \xrightarrow{\mu} C.
\]
\begin{defn}[Antipode]
	An antipode $S\colon C\to C$ is an endomorphism such that
	\[ 
	S\ast \id = \id\ast S = \eta\circ \epsilon.
	\]
\end{defn}
\begin{defn}[Hopf Algebra]
A Hopf Algebra is a bi-algebra with an antipode
\end{defn}
For specific classes of bialgebras, there is a way of constructing an antipode map.
\begin{thm}
Let $\mathcal{A}$ be a connected graded bialgebra such that $\Delta( x) = x\otimes 1 + 1 \otimes x + \sum_i a_i \otimes b_i $ with $\dim a_i,\dim b_i >0$, then $\mathcal{A}$ admits an antipode map.
\end{thm}
\begin{proof}
Let $x \in \mathcal{A}$, to define $S$, we proceed inductively on the degree of $x$. If $\dim x = 0$, we define $S( x) = x$.\\
Inductively, suppose we've defined $S$ for all $x$ of degree $<n$ and write $\Delta( x) = x\otimes 1 + 1 \otimes x + \sum_i a_i \otimes b_i$ as above. Since $\Delta$ respects the grading, we may suppose that $\dim b_i <n$, we let
\[ 
S( x) \coloneq -x - \sum_i a_i S( b_i) 
\]
One now easily checks that $S$ is an antipode.
\end{proof}



%It can be shown that $\ast$ makes the set of endomorphisms of $C$ into a group whose identity is $\eta\circ \epsilon$.
\section{The Steenrod Algebra}
Let $p$ be a prime.
\begin{defn}[Stable Cohomology operation]
	A \textbf{stable $\cmod p$ cohomology operation} $\theta$ of type $r\in \mathbb{Z}$ is a family of natural transformations $( \theta_n)_{n \in \mathbb{N}} $
\[ 
\theta_n\colon H^{n}( -, \mathbb{F}_p) \to H^{n+r}( -, \mathbb{F}_p) 
\]
such that the following diagram commutes for every space $X$ 
% https://q.uiver.app/#q=WzAsNCxbMCwwLCJIXm4oWCxcXG1hdGhiYntGfV9wKSJdLFsxLDAsIkhee24rcn0oWCxcXG1hdGhiYntGfV9wKSJdLFswLDEsIkhee24rMX0oXFxTaWdtYSBYLFxcbWF0aGJie0Z9X3ApIl0sWzEsMSwiSF57bityKzF9KFxcU2lnbWEgWCxcXG1hdGhiYntGfV9wKSJdLFswLDEsIlxcdGhldGFfbiJdLFswLDJdLFsxLDNdLFsyLDMsIlxcdGhldGFfe24rMX0iLDJdXQ==
\[\begin{tikzcd}
	{H^n(X,\mathbb{F}_p)} & {H^{n+r}(X,\mathbb{F}_p)} \\
	{H^{n+1}(\Sigma X,\mathbb{F}_p)} & {H^{n+r+1}(\Sigma X,\mathbb{F}_p)}
	\arrow["{\theta_n}", from=1-1, to=1-2]
	\arrow[from=1-1, to=2-1]
	\arrow[from=1-2, to=2-2]
	\arrow["{\theta_{n+1}}"', from=2-1, to=2-2]
\end{tikzcd}\]
\end{defn}
We can trivially compose two cohomology operations $\theta, \theta'$ of type $r$ (resp. $r'$ ) to obtain a cohomology operation of type $r+r'$, this motivates the following definition.
\begin{defn}[Steenrod Algebra]
	The $\cmod p$ \textbf{Steenrod Algebra} $\mathcal{A}_p$  is the ring freely generated by the stable cohomology operations.
	This ring comes with a natural grading coming from the type of the cohomology operation.
\end{defn}
For those familiar with (maps of) spectra, the most natural way to define the Steenrod algebra is by the formula $\mathcal{A}_p = H\mathbb{F}_p^{\ast}( H\mathbb{F}) = \bigoplus_n H\mathbb{F}_p^{n}( H\mathbb{F}_p) $.
\begin{rmk}
	Notice that if $\theta$ and $\theta'$ are two cohomology operations of different types, their sum $\theta+\theta'$ in $\mathcal{A}_p$ does \textbf{not} define a cohomology operation in any natural way.\\
	Despite this, $\mathcal{A}_p$ still naturally acts on the \textbf{full} cohomology $H^{\ast}( X) $ of a space, when viewed as an abelian group.
\end{rmk}
As we will establish in the next section, $\mathcal{A}_p$ carries a Hopf algebra structure which makes $H^{\ast}( X) $ into a (Hopf-)module.
Before showing this, we present structural results about the Steenrod algebra.
\subsection{Steenrod Powers}
From now on, $H^{\ast}( -) $ will always denote $\cmod p$ cohomology for a fixed prime $p$.
\begin{defn}[Steenrod Powers]
	Suppose $p>2$,
	the \textbf{Steenrod powers} are the stable cohomology operations 
	\[ 
		P^{i}\colon H^{q}( -, \mathbb{F}_p) \to H^{q+ 2i ( p-1) }( -, \mathbb{F}_p) 
	\]
	uniquely determined by the following properties
	\begin{enumerate}
	\item $P^{0}= \id$ 
	\item if $x \in H^{2n}( X,A, \mathbb{F}_p) $, then $P^{n}x = x^{p}$ 
	\item if $x \in H^{n}( X,A) $, then $P^{i}x =0$ for all $2i >n$ 
	\item $\delta P^{i} = P^{i}\delta$ where $\delta$ is the boundary homomorphism
	\item $P^{i}( xy) = \sum_{j+k=i} P^{j}x P^{k}y$
	\end{enumerate}
\end{defn}
\begin{defn}[Steenrod Squares]
	The \textbf{Steenrod squares} are the unique stable $\cmod 2$ cohomology operations
	$$
\sq ^{i} \colon H^{q}( -, \mathbb{F}_2) \to H^{q+i}( -, \mathbb{F}_2) 
$$
	uniquely determined by
	\begin{enumerate}
	\item $P^{0}= \id$ 
	\item if $x\in H^{n}( X, A, \mathbb{F}_2) $, then $\sq^{n}( x) = x^{2}$ 
	\item if $x \in H^{n}( X,A, \mathbb{F}_2) $, then $\sq^{i}x =0$ for all $i> n$ 
	\item $\sq^{n}( xy) = \sum_{i+j=n}\sq^{i}x \sq^{j}y$ 
	\item $\delta \sq^{i}= \sq^{i}\delta$ 
	\end{enumerate}
\end{defn}

The natural transformation $\beta\colon H^{n}( -) \to H^{n+1}( -) $ induced by the short exact sequence $0 \to \mathbb{Z}_p \to \mathbb{Z}_{p^{2}} \to \mathbb{Z}_p \to 0$ is also stable, we call it the \textbf{Bockstein morphism}.\\
For $p=2$, the Bockstein coincides with $\sq^{1}$.
It is a famed result of Steenrod that these operations generate the Steenrod algebra.
\begin{thm}[Structure of the Steenrod Algebra]\cite[Ch. VI, Sec. 2]{cohomology_operations_Steenrod}
	Let $p$ be an odd prime. Call a sequence $I = ( \epsilon_0, s_1, \epsilon_1, s_2, \ldots) $ \textbf{admissible} if it is finite, $s_i \geq 1, \epsilon = 0,1$ and $s_i \geq p s_{i+1} + \epsilon_i$. 
	The set
	\[ 
	P^{I} \coloneq \beta^{\epsilon_0}P^{s_1}\beta^{\epsilon_1}P^{s_2}, \quad I \text{ admissible } 
	\]
	is a basis for the Steenrod algebra.
\end{thm}
There is a similar result for $p=2$, which we do not make explicit.
\section{The Diagonal Morphism}
From now on, $p$ is a prime different from 2 and $\mathcal{A}\coloneq \mathcal{A}_p$.\\

The main goal of this talk is to present a proof that $\mathcal{A}_p$ has the structure of a Hopf algebra and to make its structure more explicit.\\
Throughout, let $X$ be a space.
We start by constructing the diagonal morphism $\psi^{\ast}\colon \mathcal{A}^{\ast} \to \mathcal{A}^{\ast} \otimes \mathcal{A}^{\ast}$.
\begin{propo}
	There is a unique diagonal morphism $\psi^{\ast}\colon \mathcal{A}^{\ast} \to  \mathcal{A}^{\ast} \otimes \mathcal{A}^{\ast}$ such that
	\begin{enumerate}
		\item For all $\theta \in  \mathcal{A}^{\ast}$, $\psi^{\ast}( \theta) = \sum_i \theta_i' \otimes \theta_i"$ and $\alpha,\beta \in H^{\ast}( X) $  we have
			\[ 
			\theta( \alpha\smile \beta) = \sum ( -1) ^{\dim \theta_i''\dim \alpha}\theta_i'( \alpha) \smile \theta_i''( \beta) 
			\]
			
	\item The morphism $\psi^{\ast}$ is a ring morphism.
	\end{enumerate}
\end{propo}
\begin{proof}
Let $\mathcal{A}^{\ast}\otimes \mathcal{A}^{\ast}$ act on $H^{\ast}( X) \otimes H^{\ast}( X) $ by
\[ 
( \theta'\otimes\theta'' )( \alpha\otimes \beta) = ( -1)^{\dim \theta''\dim \alpha}\theta'( \alpha) \otimes \theta''( \beta) 
\]
and we let $c: H^{\ast}( X) \otimes H^{\ast}( X) \to H^{\ast}( X) $ denote the cup product.\\
\textbf{$\psi^{\ast}$ exists}\\
Let $R \subset \mathcal{A}^{\ast}$ be the set of all $\theta$ such that 
\[ 
\theta( \alpha\smile \beta) = c \rho ( \alpha\otimes \beta) 
\]
for some $\rho \in \mathcal{A}^{\ast}\otimes \mathcal{A}^{\ast}$. We want to show that $R = \mathcal{A}^{\ast}$.\\
Notice that $R$ is closed under multiplication and addition. If $\theta_1, \theta_2\in R$, then 
\[ 
	\theta_1\theta_2( \alpha\smile \beta) = c\rho_1\rho_2( \alpha\otimes \beta) \text{ and } 	( \theta_1 + \theta_2) ( \alpha\smile\beta) = c (  ( \rho_1 + \rho_2 )( \alpha \otimes \beta ) )
\]
Hence, it suffices to show that $R$ contains the Bockstein and the Steenrod powers which follows from the formulas
\begin{align*}
	\delta( \alpha\smile \beta) &= \delta \alpha\smile \beta + ( -1)^{\dim \alpha}\alpha \smile \delta( \beta) \\
	P^{n}( \alpha\smile \beta) &= \sum_{i+j =n}  P^{i}( \alpha) \smile P^{j}( \beta) 
\end{align*}
\textbf{$\psi^{\ast}$ is unique}\\
Let $K \coloneq K( \mathbb{F}_p, n+1) $ and $\gamma \in H^{n+1}( K) $ correspond to the identity map, the map
\begin{align*}
	\ev_{\gamma} \colon \mathcal{A}^{\ast}_i &\to H^{n+1+i}( K) \\
	\theta &\mapsto \theta\gamma
\end{align*}
is an isomorphism for all $i \leq n$, it follows that
\begin{align*}
	j\colon \left( \mathcal{A}^{\ast} \otimes \mathcal{A}^{\ast} \right)_i &\to H^{2n+2+i}( K\times K) \\
	\theta\otimes \theta'&\mapsto ( -1)^{\dim\theta'\dim\gamma} \theta( \gamma) \otimes \theta'( \gamma) 
\end{align*}
is too.\\
Let $\theta\in \mathcal{A}^{\ast}_i$, suppose $\rho, \rho'$ both satisfy the required equality, then
\[ 
j( \rho) =c\rho\left( ( \gamma\otimes 1) \otimes ( 1\otimes \gamma) \right) = c\rho'\left( ( \gamma\otimes 1)\otimes ( 1\otimes \gamma)  \right) =j(\rho' ) 
\]
The unicity of $\psi^{\ast}$ implies that it is a ring morphism.
\end{proof}
\begin{rmk}\label{rmk_psi_dual}
From this proof, we can in particular single out the action of $\psi^{\ast}$ on generators, namely, it follows that 
\begin{align*}
	\psi^{\ast}( \delta) &= \delta\otimes 1 + 1 \otimes \delta\\
	\psi^{\ast}( P^{n}) &= \sum_{i+j=n} P^{i}\otimes P^{j}.
\end{align*}
\end{rmk}

\begin{thm}[The Steenrod Algebra is a Hopf Algebra]
The maps 
\[ 
	\mathcal{A} \xrightarrow{\psi^{\ast}} \mathcal{A}\otimes \mathcal{A} \xrightarrow{\phi^{\ast}} \mathcal{A}
\]
Give $\mathcal{A}$ the structure of a Hopf algebra. Furthermore $\phi^{\ast}$ is associative and $\psi^{\ast}$ is associative and commutative.
\end{thm}
\begin{proof}
It suffices to show that $\psi^{\ast}$ is associative and commutative.
\subsubsection*{Associativity}
It suffices to check the identity
\[ 
	( \psi^{\ast}\otimes 1) \psi^{\ast}= ( 1\otimes\psi^{\ast}) \psi^{\ast}
\]
This identity clearly holds on generators, namely
\begin{align*}
	\left( \psi^{\ast}\otimes 1\right) \left( \delta\otimes 1 + 1 \otimes \delta\right) &= \delta \otimes 1 \otimes 1 + 1\otimes \delta \otimes 1 + 1\otimes 1 \otimes \delta \\
								      &= \left( 1\otimes \psi^{\ast}\right) \left( \delta\otimes 1 + 1 \otimes \delta\right) 
	\intertext{and}
	\left( \psi^{\ast}\otimes 1\right) \left( \sum_{i+j=n} P^{i}\otimes P^{j}\right) &=  \sum_{i+j=n} \left( \sum_{i'+j'=i} P^{i'}\otimes P^{j'}\right) \otimes P^{j}  \\
	&= \sum_{i+j+k =n} P^{i}\otimes P^{j}\otimes P^{k}\\
	&= ( 1\otimes \psi^{\ast}) \left( \sum_{i+j=n} P^{i}\otimes P^{j}\right).
\end{align*}
\subsubsection*{(Graded) Commutativity}
Let
\begin{align*}
	T\colon \mathcal{A}\otimes \mathcal{A}&\to \mathcal{A}\otimes \mathcal{A}\\
	\theta\otimes\theta' &\mapsto ( -1)^{\dim\theta\dim\theta'} \theta'\otimes\theta.
\end{align*}
We have to check that $\psi^{\ast}= T\psi^{\ast}$, which one can check again on generators:
\begin{align*}
T( 1\otimes \delta + \delta \otimes 1) &= 1\otimes \delta + \delta\otimes 1
\intertext{and}
T( \sum_{i+j=n} P^{i}\otimes P^{j} ) &= \sum_{i+j=n} ( -1)^{4ij ( p-1)^{2}  } P^{j}\otimes P^{i}
\end{align*}

\end{proof}
\section{The dual Steenrod Algebra}
For the rest of this talk, we focus on the dual Steenrod algebra $\mathcal{A}_\ast \coloneq \mathcal{A}^{\vee}$, whose multiplication is induced by $\psi^{\ast}$. Our goal is to fully determine the structure of $\mathcal{A}_\ast$.\\
To single out an appropriate set of generators for $\mathcal{A}_\ast$, we analyze how $\mathcal{A}_\ast$ (co-)acts on the cohomology ring of a specific space. We start by describing this co-action formally and then introduce the relevant space.
\subsection{The coaction of $\mathcal{A}_\ast$ }
Given that we are working over a vector space, cohomology and homology are dual. Hence, given $\theta \in \mathcal{A}$ and $\mu \in H_\ast$, the rule
\[ 
\theta\cdot \mu ( \alpha) \coloneq \mu ( \theta( \alpha) ) \text{ for all } \alpha\in H^{\ast}
\]
gives a well defined action
\[ 
\lambda_\ast\colon \mathcal{A}\otimes H_\ast \to H_\ast
\]

We denote the dual of this action by $\lambda^{\ast}\colon H^{\ast}\to \mathcal{A}_\ast \otimes H^{\ast}$. The restriction of $\lambda_\ast$ 
\[ 
\lambda_i \colon \mathcal{A}\otimes H^{n+i} \to H^{n}
\]
also gives rise to dual morphisms $\lambda^{i}\colon H^{n}\to \mathcal{A}_\ast \otimes H^{n+i}$ which satisfy
\[ 
	\lambda^{\ast}= \lambda^{1}+ \lambda^{2}+ \ldots. \footnote{Elements in $H^{\ast}$ are always finite sums, so this sum should be understood as $\bigoplus_{i} \lambda^{i}$}
\]
We can also understand the action of $\mathcal{A}$ better in terms of $\lambda^{\ast}$.
\begin{lemma}
Let $\lambda^{\ast}( \alpha) = \sum_i \alpha_i \otimes \omega_i$ and $\theta \in \mathcal{A}$, then
\[ 
\theta\alpha = \sum_i ( -1)^{\dim \alpha_i \dim \omega_i}\langle \theta, \omega_i\rangle \alpha_i
\]
\end{lemma}
\begin{proof}
By definition of the action, we have
\begin{align*}
\langle \mu, \theta \alpha\rangle &= \langle \mu\theta, \alpha\rangle\\
&= \langle \mu\otimes \theta, \lambda^{\ast}\alpha\rangle\\
&= \sum_i ( -1)^{\dim \alpha_i\dim \omega_i} \langle \mu,\alpha_i\rangle \langle \theta, \omega_i\rangle
\end{align*}
And the general equality follows.

\end{proof}


\subsection{Generators for $\mathcal{A}_\ast$}
Fix some large integer $N$ and let $X = S^{2N+1} / \mathbb{Z}_p = sk_{2N+1} K( \mathbb{F}_p,1) $. The ($\cmod p$)  cohomology ring of $X$ has the following properties
\[ 
H^{1}( X) = \langle \alpha \rangle, H^{2}( X) = \langle \beta \rangle, H^{2i}( X) = \langle \beta^{i}\rangle, H^{2i+1}( X) = \langle \alpha\beta^{i}\rangle,
\]
where $\beta= \delta \alpha$ and $i \leq N$ \\
\begin{notn}
We define
\[ 
M^{k} \coloneq P^{p^{k-1}}\cdots P^{p}P^{1}
\]
\end{notn}
\begin{lemma}
	For all $\theta \in \mathcal{A}$
\[ 
\theta \beta =
\begin{cases}
	\beta^{p^{k}} &\text{ if } \theta = M_k\\
	0 & \text{ else. } 
\end{cases}
\]
\end{lemma}
\begin{proof}
Let $\mathcal{P}= 1+ P^{1}+ P^{2}+ \ldots$, from the properties of the Steenrod powers, we notice that
\[ 
\mathcal{P}\beta = \beta+ \beta^{p} \text{ thus } \mathcal{P}\left( \beta^{p^{r}}\right) = \beta^{p^{r}}+ \beta^{p^{r+1}}.
\]
Hence $P^{p^{r}}( \beta^{p^{r}}) = \beta^{p^{r+1}}$ and $P^{j}( \beta^{p^{r}}) $ for $j\neq p^{r}$ and $j>0$.
From this, we deduce the statement.
\end{proof}
We will now explicitly determine a basis for $\mathcal{A}_\ast$.
\begin{lemma}
There exist elements $\tau_i,\in \mathcal{A}_\ast^{2p^{k}-1}$ such that
\[ 
\lambda^{\ast}\alpha = \alpha\otimes 1 + \beta \otimes \tau_0 + \ldots + \beta^{p^{r}}\otimes\tau_r.
\]
Similarly, there exist elements $\xi_i \in  \mathcal{A}_\ast^{2p^{i}-2}$ with $\xi_0=1$ such that
\[ 
\lambda^{\ast}\beta = \beta\otimes \xi_0 + \beta^{p}\otimes \xi_1+ \ldots  +\beta^{p^{r}}\otimes \xi_r
\]

\end{lemma}
\begin{proof}
From the above, it follows that
\[ 
\lambda^{\ast}\beta = \lambda^{0}\beta + \lambda^{2p-2}\beta+ \ldots + \lambda^{2p^{k}-2}\beta.
\]
As the cohomology of $X$ is one-dimensional in all degrees, we deduce that $\lambda^{2p^{k}-2}( \beta) = \beta^{p^{k}}\otimes \xi^{k}$. The exact same argument works for $\lambda^{\ast}\alpha$.
\end{proof}
We now study the evaluation pairing $\mathcal{A}_\ast\times \mathcal{A} \to \mathbb{F}_p$. We easily establish the following lemma
\begin{lemma}
We have $\xi_k ( M_k) =1$ but $\xi_k( \theta)=0 $ for any other monomial.\\
Furthermore
\[ 
\langle M_k \delta, \tau_k \rangle = 1
\]
and $\langle \theta, \tau_k\rangle$ for any other monomial.
\end{lemma}
\begin{proof}
We know that
\[ 
M_k \beta = \beta^{p^{k}}= \sum_i ( -1)^{2 p^{i}\dim \xi^{i}} \langle M_k, \xi_i\rangle \beta^{p^{i}}
\]
Proving the equality. The second equality follows from the same argument applied to $\alpha$ and $M_k\delta$.
\end{proof}
We are ready to prove the main structure theorem for the dual Hopf algebra.
\begin{thm}
	There is a graded isomorphism
	\[ 
		\mathcal{A}_\ast \simeq \Lambda [ \tau_0,\tau_1,\ldots] \otimes \mathbb{F}_p[\xi_1,\xi_2,\ldots], \quad \text{ where } \dim \tau_i = 2p^{i}-1 , \dim \xi_i = 2p^{i}-2.
	\]
	Here $\Lambda[\tau_0,\ldots]$ denotes the exterior algebra and $\mathbb{F}_p[\xi_1,\xi_2,\ldots]$ is the polynomial algebra. This isomorphism is graded 

\end{thm}
\begin{proof}
	Let $\mathcal{I}$ be the set of finite sequences $( \epsilon_0, r_1,\epsilon_1,\ldots) $ with $\epsilon_i = 0,1$ and $r_i \in \mathbb{N}$. 
	Given $I \in \mathcal{I}$, we define
	\[ 
	\omega( I) \coloneqq \tau_0^{\epsilon_0}\xi_1^{r_1}\tau_1^{\epsilon_1}\xi_2^{r_2}\cdots.
	\]
We claim it is sufficient to show that the set of $\omega( I) $ form a basis for $\mathcal{A}_\ast$. Indeed, the $\tau_i, \xi_j$ then don't observe any additional identities and the graded commutativity gives the desired isomorphism.\\
We may order the set $\mathcal{I}$ colexicographically, ie. $( a_1, \epsilon_1,a_2,\cdots) < ( b_1, \epsilon_1', b_2, \cdots) $ if $a_i<b_i$ for the largest $i$ such that $a_i$ and $b_i$ differ (remember that the sequences are finite).\\

We also associated to a $J = ( \epsilon_0, r_1, \epsilon_1,\ldots)  \in \mathcal{I} $ an element of $\mathcal{A}$.
\[ 
\theta( J) = \delta^{\epsilon_0}P^{s_1}\delta^{\epsilon_1}P^{s_2}\cdots,
\]
where $s_j= \sum_{i=k}^{ \infty }( \epsilon_i+ r_i) p^{i-k}$.\\

One can check that the $\theta( J) $ are the basic monomials of the Cartan basis for $\mathcal{A}$.\\

To show the isomorphism, we show that the basic monomials in $\mathcal{A}$ form an ``almost dual'' basis to the set of $\omega( I) $.\\
For this, we use the following lemma.
\begin{center}
Let $I< J \in \mathcal{I}$, then $\left\langle \theta( J), \omega( I)\right\rangle = 0 $ if $I < J$, furthermore $\left\langle \theta( I) , \omega( I) \right\rangle = \pm 1$. $\qquad (\star)$ 
\end{center}
The proof of $(\star)$ is the main technical step in the proof and we skip it.
Let $\mathcal{I}_n \subset \mathcal{I}$ be the set of sequences such that $\dim \omega( I) = \dim \theta( I) =n$.
The matrix $\left( \langle \theta( J) ,\omega( I)  \right)_{I,J \in \mathcal{I}_n}$ is upper-triangular with $\pm 1$ on the diagonal, hence, the pairing is non-degenerate and the $\omega( I) $ generate the $n$-th graded part of $\mathcal{A}_\ast$.
%\subsection*{Proof of $(\star)$}
%We start by proving that $\langle \theta( I) ,\omega( I) \rangle = \pm 1$.
%We proceed by induction on the dimension. 
%The case of dimension 0 is trivially true.
%For the induction step, we distinguish different cases for $I$.
%\begin{itemize}
	%\item \textbf{The last non-zero element of $I$ is $r_k$ }\\
%Let $I' = ( \epsilon_0,r_1,\cdots, \epsilon_{k-1}, r_k-1, 0, \ldots) $, then $\omega( I) = \omega( I') \xi_k$.
%We find
%\begin{align*}
%\langle \theta( I) , \omega( I) \rangle &= \left\langle \theta( I) , \psi_\ast \left( \omega( I') \otimes \xi_k\right) \right\rangle\\
%&=\left\langle \psi^{\ast}\theta( I), \omega( I') \otimes \xi_k\right\rangle
%\intertext{ Using our description of $\psi^{\ast}$ in remark \ref{rmk_psi_dual}, we get }
%&= \sum_i \sum_{\epsilon_i'+ \epsilon_i'' = \epsilon_i}\sum_{}   \pm \langle \delta^{\epsilon_0'}\cdots P^{s_k'}, \omega( I') 
%\end{align*}
%\end{itemize}
\end{proof}
We also state the case for $p=2$ without proof, the proof can be found in the original paper too and proceeds in very similar steps.
\begin{thm}[The mod 2 dual Steenrod Algebra]
	Let $\mathcal{A}_2$ be the mod 2 Steenrod algebra and $\mathcal{A}_{2\ast} $ its dual
Let $\xi_i\in \mathcal{A}_{2\ast} $ be the dual basis of the basis $\sq^{2^{i-1}}\cdots \sq^{2}\sq^{1}\in \mathcal{A}_2$, then there is a graded isomorphism
\[ 
	\mathcal{A}_{2\ast} \simeq \mathbb{F}_2[\xi_1,\xi_2,\ldots].
\]

\end{thm}
\subsection{The comultiplication in $\mathcal{A}_\ast$}

If we want to fully describe $\mathcal{A}_\ast$ as a Hopf algebra, we also have to describe the comultiplication $\phi_\ast\coloneq  ( \phi^{\ast})^{\vee}$

\begin{propo}
We have
\begin{align*}
	\phi_\ast( \xi_k) &= \sum_{i=0}^{k}\xi_{k-i}^{i}\otimes \xi_i\\
\phi_\ast( \tau_k) &= \sum_{i=0}^{k}\xi_{k-i} ^{p^{i}}\otimes \tau_i + \tau_k \otimes 1
\end{align*}
\end{propo}
\begin{proof}
	We first notice that the commutativity of
	% https://q.uiver.app/#q=WzAsNCxbMCwwLCJIX1xcYXN0XFxvdGltZXMgXFxtYXRoY2Fse0F9XFxvdGltZXMgXFxtYXRoY2Fse0F9Il0sWzEsMCwiSF9cXGFzdFxcb3RpbWVzIFxcbWF0aGNhbHtBfSJdLFswLDEsIkhfXFxhc3RcXG90aW1lcyBcXG1hdGhjYWx7QX0iXSxbMSwxLCJIX1xcYXN0Il0sWzAsMSwiMVxcb3RpbWVzIFxccGhpXlxcYXN0Il0sWzAsMiwiXFxsYW1iZGFfXFxhc3RcXG90aW1lczEiLDJdLFsyLDMsIlxcbGFtYmRhX1xcYXN0IiwyXSxbMSwzLCJcXGxhbWJkYV9cXGFzdCJdXQ==
\[\begin{tikzcd}
	{H_\ast\otimes \mathcal{A}\otimes \mathcal{A}} & {H_\ast\otimes \mathcal{A}} \\
	{H_\ast\otimes \mathcal{A}} & {H_\ast}
	\arrow["{1\otimes \phi^\ast}", from=1-1, to=1-2]
	\arrow["{\lambda_\ast\otimes1}"', from=1-1, to=2-1]
	\arrow["{\lambda_\ast}"', from=2-1, to=2-2]
	\arrow["{\lambda_\ast}", from=1-2, to=2-2]
\end{tikzcd}\]
implies the identity
\[ 
	( \lambda^{\ast}\otimes 1) \lambda^{\ast}= ( 1\otimes\phi_\ast) \lambda^{\ast}.
\]
	Let $\alpha,\beta \in H^{\ast}( X) $ with $X$ as before, then
\begin{align*}
	\lambda^{\ast}( \beta) &= \sum \beta^{p^{j}}\otimes \xi_j\\
	\lambda^{\ast}( \beta^{p^{i}}) = \sum \beta^{p^{i+j}}\otimes \xi_j^{p^{i}}
\end{align*}
Hence, from the identity above, we get
\begin{align*}
( \lambda^{\ast}\otimes 1) \lambda^{\ast}( \beta) &= \sum_{i,j} \beta^{p^{i+j}}\otimes \xi_j^{p^{i}}\otimes \xi_i\\
&= ( 1\otimes \phi_\ast) \lambda^{\ast}( \beta) \\
&= \sum \beta^{p^{k}}\otimes \phi_\ast( \xi_k) 
\end{align*}
And hence we deduce the identity for $\phi_\ast( \xi_k) $, the identity for $\phi_\ast( \tau_k) $ is deduce in the same way.

\end{proof}




\printbibliography
\end{document}
