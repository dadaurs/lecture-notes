\documentclass[../main.tex]{subfiles}
\begin{document}
\lecture{11}{Wed 18 Nov}{Rotations Du Solide}
\section{Rotations Du Solide}
\subsection{Corps Solide indeformable}
\begin{defn}[Solide indeformable]
	Systeme de points materiels fixes les uns par rapport aux autres.
\end{defn}
Le nombre de points materiels peut etre tres grand, on remplace alors les sommes sur ces points par des integrales.
\[ 
	\vec{r}_G = \frac{1}{M} \int \vec{r} dm ( \vec{r}) = \frac{1}{M}\int \vec{r} \rho( \vec{r}) d^{3} \vec{r}
\]
\subsection{Vitesse et acceleration d'un point solide}
On fixe un repere sur le solide, soit $y$ un vecteur dans ce vecteur.
\[ 
	\frac{d \vec{y}}{dt}= \frac{d}{dt} ( \sum y_i \hat{y}_i) = \sum y_i \frac{d \vec{y}_i}{dt}= \vec{\omega} \land \sum y_i \hat{y}_{i} = \vec{\omega} \land \vec{y}
\]
Pour tout point $P$ du solide
\[ 
	\vec{v}_p = \frac{d}{dt}( \vec{r}_A + \vec{AP}) = \vec{v}_A + \frac{d}{dt} \vec{AP} = \vec{v}_A + \vec{\omega}\land \vec{AP}
\]
De meme
\[ 
	\vec{a}_p = \vec{a}_A + \dot{  \vec{\omega} } \land \vec{AP} + \vec{\omega} \land ( \vec{\omega} \land \vec{AP}) 
\]
\subsection{Mouvement plan-sur-plan}
\begin{defn}
	mouvement tel qu'un plan du solide $S$ reste constamment dans un plan fixe $\Pi$ du referentiel\\
	$\iff$ \\
	a tout instant les vitesse de tous les points du solide sont paralleles a un plan fixe $\Pi$ du referentiel
\end{defn}

\subsection{Moment cinetique par rapport a un point quelconque}
\begin{align*}
	\vec{L}_Q = \sum_{\alpha} Q \vec{P}_\alpha \land m_\alpha \vec{v}_\alpha = \sum_\alpha ( \vec{QO} + \vec{OP}_\alpha) \land m \vec{v}_\alpha\\
	= \vec{QO} \land \sum_\alpha m \vec{v}_\alpha + \sum_\alpha \vec{OP}_\alpha \land m \vec{v}_\alpha\\
	= \vec{L}_O + \vec{QO} \land M \vec{v}_G
\end{align*}
\begin{thm}[Theoreme du Transfert]
\[ 
\vec{L}_Q = \vec{L}_O + \vec{QO} \land M \vec{v}_G
\]

\end{thm}
\begin{align*}
\vec{L}_G = \sum_\alpha \vec{r}_\alpha ^{*} \land m_\alpha \vec{v}_\alpha\\
= \sum_{\alpha} \vec{r}_\alpha ^{*} \land m_\alpha \vec{v}_\alpha ^{*} + (  \sum_\alpha m_\alpha \vec{r}_\alpha ^{*}) \land \vec{v}_G = \vec{L}_G ^{*}
\end{align*}
Avec ca, on a donc que
\[ 
	\vec{L}_Q = \vec{L}_G ^{ ( *) } + \vec{QG} \land M \vec{v}_G
\]

\subsection{Theoreme du Moment Cinetique Generalise}
\begin{align*}
	\frac{d L_Q}{dt} = \frac{d \vec{L}_O}{dt} + \frac{d}{dt} ( \vec{QO} \land M \vec{v}_G)  = \vec{M}_O ^{ \text{ ext } } + \vec{QO} \land M \vec{a}_G - \vec{V}_Q \land M \vec{v}_G
\end{align*}
On a 
\[ 
	\vec{M}_Q ^{ext} = \sum_\alpha M_{Q,\alpha} ^{ \text{ ext } } = \sum_\alpha \vec{QP}_\alpha \land \vec{F}_\alpha ^{ \text{ ext } } = \sum_\alpha ( \vec{QO} + \vec{OP}_\alpha) \land \vec{F}_\alpha ^{ext} = \vec{QO} \land \vec{F}^{ext} + \sum_\alpha \vec{OP}_\alpha \land \vec{F}_\alpha ^{ext}
\]
Donc
\[ 
\frac{d \vec{L}_Q}{dt} = \vec{M}_Q ^{ext} - \vec{v}_Q \land M \vec{v}_G
\]
On remarque que si $Q=G$, ou $\vec{v}_Q // \vec{v}_G$, on a le theoreme du moment cinetique








\end{document}	
