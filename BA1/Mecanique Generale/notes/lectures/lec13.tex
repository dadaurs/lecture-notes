\documentclass[../main.tex]{subfiles}
\begin{document}
\lecture{13}{Wed 02 Dec}{Mouvement de Solides}
Pour tout point $C$ d'un solide, on peut toujours choisir un repere orthonorme tel que la matrice representant le tenseur d'inertie soit diagonale.

\begin{defn}
Le repere d'inertie est le repere dans lequel le tenseur est diagonal.\\
Les axes principaux d'inertie sont les axes du repere d'inertie.\\
Les moments d'inertie principaux sont les moments d'inerties par rapport aux axes principaux d'inertie.
\end{defn}
Si le solide est symetrique, les axes suivants sont des axes principaux d'inertie au point $C$ 
\begin{itemize}
\item Tout axe de symetrie
\item L'axe passant par $C$ et perpendiculaire a un plan de symetrie.
\item Tout axe passant par $C$ et perpendiculaire a un axe de symetrique d'ordre $n\geq 3$.
\end{itemize}
\begin{thm}[Theoreme de Steiner]
	\begin{align*}
		( \tilde{I}_A )_{ij} = ( \tilde{I}_G)_{ij} + M \left[ \vec{AG}^{2} \delta_{ij}  - ( AG) _i( AG)_j  \right] 
	\end{align*}
	

\end{thm}


\end{document}	
