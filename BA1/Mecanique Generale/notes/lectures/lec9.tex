\documentclass[../main.tex]{subfiles}
\begin{document}
\lecture{9}{Wed 04 Nov}{Systemes de points Materiels}
\subsection{2eme loi et theoreme du moment cinétique}
La résultante des forces appliquées au point matériel $P$ :
\[ 
\vec{F}= \sum \vec{F}_i
\]
Quantité de mouvement du point matériel de masse $m$ 
\[ 
\vec{p}= m \vec{v}
\]
En dérivant, et en appliquant la deuxieme loi de Newton, on trouve
\[ 
\frac{d \vec{p}}{dt}= \vec{F}
\]
Le moment de la force résultante $\vec{F}$ par rapport à un point $O$ du référentiel
\[ 
\vec{M}_O = \vec{r} \land \vec{F} = \sum \vec{r} \land \vec{F}_i
\]
Moment cinétique du point matériel par rapport au point $O$ : 
\[ 
\vec{L}_O = \vec{r} \land \vec{p} = \vec{r} \land m \vec{v}
\]
Le théorème du moment cinétique dit
\[ 
\frac{d \vec{L}_O}{dt}= \vec{M}_O
\]

\subsection{Systeme de points matériels}

On suppose ue chaque point matériel $P_a$ du système subit:
\begin{itemize}
\item une force extérieure $F_a^{ext}$ dont l'origine est extérieure au système
\item des forces intérieures $F^{\beta\to \alpha}$ exercées par les autres points $P_\beta$ du système 
\end{itemize}
On peut appliquer la troisième de Newton, à $\alpha$ et $\beta$.
Donc
\[ 
\vec{F}^{\beta\to\alpha} + \vec{F}^{\alpha\to \beta}=0
\]
et
\[ 
	\vec{M}^{\beta\to\alpha} + \vec{M}^{\alpha\to \beta}= ( \vec{r}_\alpha - \vec{r}_\beta) \land \vec{F}^{\beta \to \alpha}=0
\]

\begin{align*}
\sum_{\alpha} \sum_{\beta\neq \alpha}\vec{F}^{\beta\to\alpha} =0\\
\sum_{\alpha} \sum_{\beta\neq \alpha}\vec{M}^{\beta\to\alpha} =0
\end{align*}
On définit la quantité de mouvement totale
\[ 
\vec{p} = \sum_\alpha \vec{p}_\alpha
\]
Si on dérive $\vec{p}$, on obtient
\[ 
\frac{d \vec{p}}{dt} = \sum_\alpha \sum_{\beta\neq\alpha} \vec{F}^{\beta\to \alpha} + \sum_\alpha \vec{F}_\alpha ^{ext} = \vec{F}^{ext}
\]
On trouve donc
\[ 
\frac{d \vec{p}}{dt} = \vec{F}^{ext} 
\]
De la même manière, on trouve,
\[ 
\frac{d \vec{L}_O}{dt}= \sum_{\alpha} \vec{M}_{O,\alpha} ^{ext} = \vec{M}_{O} ^{ext}
\]

On a donc trouvé les \underline{lois générales de la dynamique pour un système de points matériels}

\begin{align*}
\frac{d \vec{p}}{dt} = \vec{F}^{ext} \\
\frac{d \vec{L}_O}{dt} = \vec{M}_{O} ^{ext}
\end{align*}

\subsection{Système à l'équilibre}
Un système est à l'équilibre si
\[ 
\begin{cases}
	\vec{r}_\alpha( t) = \text{ constante } \\
	\vec{v}_\alpha( t) = constante
\end{cases}
\]
Dans ce cas, on a
\[ 
\begin{cases}
\vec{p}= \sum_\alpha m_\alpha \vec{v}_\alpha = 0\\
\vec{L}_O = \sum_\alpha \vec{r}_\alpha \land m_\alpha \vec{v}_\alpha = 0
\end{cases}
\]
et donc
\[ 
\begin{cases}
\frac{d \vec{p}}{dt}=0\\
\frac{d \vec{L}_O}{dt} =0
\end{cases}
\]
et donc
\[ 
\vec{F}^{ext}=0 \text{ et } \vec{M}_O^{ext}=0
\]
\hr\\
On appellera un système ``partiellement isolé'' selon une direction fixe $\hat{u}$ :
\begin{align*}
\vec{F}^{ext}\cdot \hat{u} =0 \Rightarrow \vec{p}\cdot \hat{u}= \text{ constante } \\
\vec{M}^{ext}_O\cdot \hat{u} =0 \Rightarrow \vec{L}_O \cdot \hat{u} = \text{ constante } \\
\end{align*}
\subsection{Centre de masse}
Le centre de masse est un point de l'espace $G$ défini par
\[ 
\vec{r}_G = \frac{1}{M} \sum_\alpha m_\alpha \vec{r}_\alpha
\]
où $M$ est la somme des masses.\\
Si les masses $m_\alpha$ sont constantes, la vitesse du centre de masse est
\[ 
\vec{v}_G = \frac{d \vec{r}_G}{dt}= \frac{1}{M} \sum_\alpha m_\alpha \vec{v}_\alpha = \frac{\vec{p}}{M} \Rightarrow \vec{p}= M \vec{v}_G
\]
Donc
\[ 
\frac{d \vec{p}}{dt}= \vec{F}^{ext} \Rightarrow M\vec{a}_G = \vec{F}^{ext}
\]
C'est le theoreme du centre de masse.









\end{document}	
