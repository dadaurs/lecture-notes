\documentclass[../main.tex]{subfiles}
\begin{document}
\lecture{12}{Wed 25 Nov}{Moment Cinetique}
\subsection{Moment Cinetique d'un solide quelconque}
Soit $A$ un point du solide.
\begin{align}
	\vec{L}_A = \sum_\alpha \vec{AP}_\alpha \land m \vec{v}_\alpha = \sum_\alpha \vec{AP}_\alpha + m_\alpha ( \vec{v}_\alpha + \vec{\omega} \land \vec{AP}_\alpha) \\
	= \sum_\alpha m_\alpha \vec{AP}_\alpha \land \vec{v}_A + \sum_\alpha m_\alpha \left[\vec{AP}_\alpha \land ( \vec{\omega} \land \vec{AP}_\alpha)\right] \\
\end{align}
\subsection{Moment d'inertie par rapport a un axe de rotation fixe}
\[ 
	\vec{L}_C = \sum_\alpha m_\alpha \left[ ( \vec{CP})^{2} \vec{\omega} - ( \vec{CP}_\alpha \cdot \vec{\omega}) \vec{CP}_\alpha\right] 
\]
Si on projette $L_C$ sur l'axe de rotation, on trouve
\[ 
L_\Delta = \vec{L} \cdot \hat{\omega} = \omega \sum_\alpha m_\alpha d_\alpha^{2}
\]
On definit
\[ 
I_\Delta = \sum_\alpha m_\alpha d_\alpha^{2}
\]
C'est le moment d'inertie du solide.

En developpant l'expression, on trouve
\[ 
	\vec{L}_G = \tilde{I}_G \cdot \vec{\omega}
\]
\subsection{Tenseur d'inertie}
Par rapport a un point $A$ appartenant au solide
\[ 
	\vec{L}_A = \vec{AG} \land  M \vec{v}_A + \tilde{I}_A \cdot \vec{\omega}
\]







\end{document}	
