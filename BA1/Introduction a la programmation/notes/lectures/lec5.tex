\documentclass[../main.tex]{subfiles}
\begin{document}
\lecture{5}{Thu 01 Oct}{Programmation}
Comment calculer l'expression suivante sans produire d'erreur 
\[ 
	\frac{ \sqrt{20+7x -x^{2}} \log (  \frac{1}{x+5})}{\frac{x}{10} - \sqrt{\log(x^{3} - 3x + 7} - \frac{x^{2}}{5}}
\]
Il faut decomposer
\begin{lstlisting}
if ( x + 5.0 ==0){
cerr << "expressions invalide pour x=" << x<<endl;
return 1;
}
\end{lstlisting}
etc...
\begin{lstlisting}
if(x + 5.0 < 0.0) { 
cerr << "expressions invalide pour x=" << x<<endl;
return 1;
}
\end{lstlisting}
Attention, ceci est du copier colle, il auriat fallu poser
\begin{lstlisting}
double aux(x+5.0);
if (aux ==0){
cerr << "expressions invalide pour x=" << x<<endl;
return 1;
}
if(aux < 0.0) { 
cerr << "expressions invalide pour x=" << x<<endl;
return 1;
}
double resultat(log(1.0/aux));
...
\end{lstlisting}




\end{document}	
