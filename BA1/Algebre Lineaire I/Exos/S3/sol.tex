\documentclass[11pt, a4paper, twoside]{article}
\usepackage[utf8]{inputenc}
\usepackage[T1]{fontenc}
\usepackage[francais]{babel}
\usepackage{lmodern}

\usepackage{amsmath}
\usepackage{amssymb}
\usepackage{amsthm}
\newcommand\hr{
    \noindent\rule[0.5ex]{\linewidth}{0.5pt}
}

\begin{document}
\title{Série 3, Exercice 5}
\author{David Wiedemann}
\maketitle
\section*{5.1}
Si $\phi^{-1}\left( \left\{ h \right\} \right)$ vide, il n'y pas de solutions.\\
Supposons $\phi^{-1}( \left\{ h \right\} )$ non vide, $\exists g_0$ tq. $\phi(g_0) = h$.\\
Soit $e \in \ker \phi$, alors
\[ 
	\phi ( g_0 \cdot e   ) \underbrace{=}_{\phi \text{ morphisme } } h \cdot e_H = h
\]
ou $e_H$ est l'élément neutre dans $H$.\\
Ceci est valable $\forall e \in \ker \phi$.
\section*{5.2}
Supposons à nouveau que $e \in \ker \phi$ et que $\phi^{-1}( \left\{ h \right\} )$ non vide.\\
On pose, à nouveau, que $ \phi( g_0 ) =h $.
\[ 
	\phi(e. g_0) = e_H . h = h \text{ avec  $e_H$ l'élément neutre de $H$.} 
\]
\hr\\
Ici, on voit que  $ g_0$ est une solution particulière à l'équation $\phi(g) =h$.\\
Dans la résolution de systémes d'équations de la forme
\[ 
Ax =b, \text{ où $A$ est une matrice } 
\]
On peut trouver les solutions du systéme homogène:
\[ 
Ax=0, \text{ les solutions de ce système forment le $\ker$ } 
\]
et les superposer à la solution spécifique $b'$ qui satisfait
\[ 
A b' = b
\]
\end{document}
