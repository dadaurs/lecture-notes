\documentclass[11pt, a4paper, twoside]{article}
\usepackage[utf8]{inputenc}
\usepackage[T1]{fontenc}
\usepackage[francais]{babel}
\usepackage{lmodern}

\usepackage{amsmath}
\usepackage{amssymb}
\usepackage{amsthm}
\newcommand{\Bij}{\mathrm{Bij}}
\newcommand{\Id}{\mathrm{Id}}
\begin{document}
\title{Série 2}
\author{David Wiedemann}
\maketitle
\section*{4.1}
On vérifie le critère du sous-groupe.\\
Soit $\sigma, \gamma \in \Bij(X)_Y $, alors:
\[ 
	\sigma \circ \gamma (Y)=\sigma(\gamma(Y))= \sigma(Y) = Y
\]
donc $\sigma \circ \gamma \in \Bij(X)_Y$\\
On montre que $\Bij(X)_Y$ est clos sous l'inversion:\\
Soit $\sigma \in \Bij(X)_Y: X \to X$, alors $\exists \sigma^{-1}: X \to X$ tel que $\sigma^{-1}\circ \sigma = \sigma \circ \sigma^{-1} = \Id_X$. On a donc:
\begin{align*}
	\sigma^{-1}(\sigma(Y)) &= \Id_X(Y)\\
	\sigma^{-1}(Y) &= Y
\end{align*}
Donc $\sigma^{-1} \in \Bij(X)_Y$.\\
On en conclut que $\Bij(X)_Y$ forme un sous-groupe de $\Bij(X)$
\section*{4.2}
Dans un groupe à trois éléments, il suffit de trouver un contre exemple.
Soit
\begin{align*}
\sigma: 
\begin{cases}
x_1 \to x_1\\
x_2 \to x_3\\
x_3 \to x_2\\
\end{cases}
\text{ et } 
\tau: 
\begin{cases}
x_1 \to x_2\\
x_2 \to x_1\\
x_3 \to x_3\\
\end{cases}
\end{align*}
Alors on a:
\begin{align*}
\sigma \circ \tau:
\begin{cases}
x_1 \to x_2\\
x_2 \to x_3\\
x_3 \to x_1
\end{cases}
\text{ et } 
\tau \circ \sigma:
\begin{cases}
x_1 \to x_3\\
x_2 \to x_1 \\
x_3 \to x_2
\end{cases}
\end{align*}
Considérons maintenant un groupe à plus que trois éléments et notons $Y=\{x_1,x_2,x_3\}, Y \subset X$.\\
On pose:
\begin{align*}
S:
\begin{cases}
x_n \to x_n \text{ si } x_n \notin Y\\
x_{n} \to \sigma(x_n) \text{ si }  x_n \in Y
\end{cases}
\text{ et }
G:
\begin{cases}
x_n \to x_n \text{ si } x_n \notin Y\\
x_n \to \gamma(x_n) \text{ si } x_n \in Y
\end{cases}
\end{align*}
On remarque que $S, G \in \Bij(X)_Y$.\\
Si on compose $G$ avec $S$, on remarque que les applications ne commutent pas:
\begin{align*}
G \circ S:
\begin{cases}
x_n \to x_n \text{ si } x_n \notin Y\\
x_n \to \gamma(\sigma(x_n)) \text{ si } x_n \in Y
\end{cases}
\text{ et } 
S \circ G:
\begin{cases}
x_n \to x_n \text{ si } x_n \notin Y\\
x_n \to \sigma(\gamma(x_n)) \text{ si } x_n \in Y
\end{cases}
\end{align*}
car $\sigma\circ\gamma \neq \gamma\circ\sigma$, on voit que $S \circ G \neq G \circ S$.



\section*{4.3}
Si $X$ possède deux éléments, il y a deux éléments dans $\Bij(X)$ :
\begin{align*}
\Id:
\begin{cases}
x_1 \to x_{1}\\
x_2 \to x_2
\end{cases}
\text{ et } 
C:
\begin{cases}
x_1 \to x_2\\
x_2 \to x_1
\end{cases}
\end{align*}
On vérifie facilement que $C \circ \Id = \Id \circ C$, et donc le groupe est commutatif.\\
Si $X$ possède un élément, il y a un élément dans $\Bij(X)$ :
\begin{align*}
\Id:
\begin{cases}
x_1 \to x_1
\end{cases}
\end{align*}
Clairement l'identité commute avec elle même, et donc le groupe est commutatif.





%[La résolution des exercices ici...] %commentaire



\end{document}
