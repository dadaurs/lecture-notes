\documentclass[../main.tex]{subfiles}
\begin{document}
\lecture{22}{Mon 30 Nov}{Engendrement du groupe lineaire}
\begin{propo}
Si $M$ et $N$ sont lignes equivalents
\[ 
	rg( M) = rg( N) 
\]

\end{propo}
\begin{proof}
Si $M \sim_{ lig } N \Rightarrow M \sim N $.
\end{proof}
\subsection{Engendrement du groupe lineaire}
\begin{propo}
	Soit $M \in M_d( K) $ une matrice carree alors $M$ est inversible si et seulement si $M$ est ligne equivalente a la matrice identite $\id$.
\end{propo}
\begin{proof}
	$M$ est inversible si et seulement si $rg( M) =d$, et donc $M$ inversible si et seulement si $M$ est ligne equivalente a $R$, $R \in M_d( K) $ une matrice a $d$ echelons.\\
	$R= \id$.
\end{proof}
\begin{crly}
	Le groupe lineaire $GL_d( K) $ est engendre par les matrices de transformation elementaires.
\end{crly}
\begin{proof}
	Soit $M \in GL_d( K)$, donc
	\[ 
	M \sim_{lig} \id
	\]
	Donc, il existe $T_1, \ldots ,T_k$ des matrices de transformations elementaires tel que
	\[ 
	\id = T_k \ldots T_1 M
	\]
	Donc
	\[ 
	M = T_1^{-1}\ldots T_k^{-1}
	\]
	
\end{proof}

\subsection{Extraction d'une base}
Soit 
\[ 
G = \left\{ w_1, \ldots , w_l \right\} \subset K^{d}
\]
et $W = \eng{G}$.
\begin{propo}
	Soit $M \in M_{l\times d} ( K) $ la matrice dont les $l$ lignes sont formees des vecteurs lignes $w_i, i \leq l$. Soit $R$ la matrice echelonee reduite associee a $M$ et 
	\[ 
		w'_i = Lig_I( R) 
	\]
	Les lignes de $R$ possedent $r$ echelons on a
	\[ 
	\dim W = r
	\]
	et les $r$ premieres lignes
	\[ 
	\mathcal{B} _W = \left\{ w_i', i \leq r \right\} 
	\]
	forment une base de $W$.
\end{propo}
\begin{proof}
On a vu que les lignes de $R$ sont CL des lignes de $M$.\\
Mais on sait que les $w_i', i \leq r$ forment une famille libre et 
\[ 
rg R = rg M = \dim W
\]

\end{proof}
\subsection{Resolution de systemes lineaires}
Soit $\phi: V \to W$ et $w \in W$, on cherche l'ensemble des $v \in V$ tel que
\[ 
	\phi( v) = w
\]
On cherche l'ensemble des antecedents de $w \in W $ par l'application $\phi$.\\
On cherche 
\[ 
	\phi^{-1}( \left\{ w \right\} ) = \left\{ v \in V \quad \phi( v) = w \right\} 
\]
C'est un cas particulier d'une question sur les groupes
\[ 
	\phi: ( G, \cdot) \to ( H,\cdot) 
\]
\begin{lemma}
Soit $\phi: G \mapsto H$ un morphisme de groupes, alors pour tout $h \in H$, on pose
\[ 
	Sol_\phi( h) = \phi ^{-1}(  \left\{ h \right\} ) = \left\{ g \in G, \phi( g) =h \right\} \subset G
\]
la preimage de $h$ par $\phi$. En particulier, $Sol_\phi( e_H) = \ker \phi$. Alors $Sol_\phi( h) $ est
\begin{itemize}
	\item soit l'ensemble vide ( ssi $h \notin \phi( G) $) 
	\item soit il existe $g_0\in Sol_\phi( h) $ et
		\[ 
			Sol_\phi( h) = g_0Sol_\phi( e_H) = g_0\ker \phi = \left\{ g_0.k, \phi( k) =e_H \right\} 
		\]
		
\end{itemize}


\end{lemma}

\begin{proof}
	Si $h \notin \phi( G) $, il n'existe pas de $g$ tel que
	\[ 
		\phi( g) =h 
	\]
	et 
	\[ 
		Sol_\phi( h) = \emptyset
	\]
	
	Si $h \in \phi( G) $, alors $\exists g_0 \in G $ tel que $\phi( g_0) =h$,
	donc l'ensemble n'est pas vide.\\
	Alors
	\[ 
		\phi( g) =h = \phi( g_0) 
	\]
	et donc
	\[ 
		\phi( g_0) ^{-1}\phi( g) = e_h
	\]
	Donc
	\[ 
		g_0^{-1}g = k \in \ker \phi = Sol_\phi( e_H) 
	\]
	Reciproquement, soit $g= g_0k, k \in \ker \phi$, alors
	\[ 
		\phi( g) = \phi( g_0) \phi( k) = \phi( g_0) = h
	\]
	
	
\end{proof}
\begin{crly}
	$G= ( V,+) , H = ( W,+) , \phi: V \to W$ une application lineaire.\\
	Soit $w \in W$.\\
	Si $w \notin \phi( W) $, l'ensemble des solutions est vide.\\
	Sinon, $w \in \phi( V) $, soit $v_0$, un antecedent, alors
	\[ 
		Sol_\phi( w) = \left\{ v \in V | \phi( v) = w \right\} = v_0 + \ker \phi
	\]
	
\end{crly}
\begin{defn}
Les inconnues $v_{j_i} $ pour $j_i$ etant un echelon sont appellees inconues principales du systeme. Les inconnues $v_j$ pour $j\leq d$ qui n'est pas un echelon sont appellees inconnues libres dy systeme.
\end{defn}


	




\end{document}	
