\documentclass[../main.tex]{subfiles}
\begin{document}
\lecture{20}{Mon 23 Nov}{Operations Elementaires}
\begin{thm}
	Soit $P( X) $ un polynome reel non-constant alors l'equation admet au moins une solution dans $\mathbb{C}$.
\end{thm}
\begin{thm}[Gauss-Wantzel]
On peut exprimer les parties reelles et imaginaires du nombre complexe $\omega_n = e^{i 2 \pi / n} $ par extraction successive de racines carrees si et seulement si
\[ 
n=2k \text{ ou bien  } n = 2^{k} \prod_i p_i
\]
ou $\prod_i p_i$ est un produit ( non-vide) de nombres premiers tous distincts et ``de Fermat'' : on dit qu'un nombre premier $p_i$ est de Fermat si $p_i = F_{f_i} = 2^{2^{f_i}}+1$, avec $f_i \geq 0$ un entier
\end{thm}
\section{Operations Elementaires Sur Les Matrices}
\begin{defn}[Operations Elementaires]
	\begin{enumerate}
	\item $T_{ij} \quad L_i \rightleftarrows L_j\quad i,j \leq d'$ 
	\item $D_{i,\lambda} \lambda\neq 0 L_i \to \lambda L_i$
	\item $CL_{ij,\mu} : L_i \to L_i + \mu L_j$
	\end{enumerate}
\end{defn}
\begin{propo}
Ces operations sont des applications lineaires bijectives
\end{propo}
\begin{propo}
Les trois operations elementaires sont obtenues par multiplication a gauche de $M$ par des matrices convenables: pour $1\leq i,j \leq d'$ 
\begin{itemize}
\item $T_{ij} $ 
\item $D_{i,\lambda} $ 
\item $Cl_{ij,\mu} $
\end{itemize}
ou les matrices carrees $T_{ij} , D_{i\lambda} , Cl_{ij,\mu} $ sont definies par
\begin{align*}
T_{ij}  = \id -E_{ii}  - E_{jj} + E_{ij} + E_{ji} \\
D_{i,\lambda} = \id + ( \lambda -1)  E_{ii} , \lambda\neq 0\\
Cl_{ij,\mu}  = \id + \mu. E_{ij} 
\end{align*}

\end{propo}



\end{document}	
