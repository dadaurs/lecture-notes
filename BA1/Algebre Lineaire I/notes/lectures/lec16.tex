\documentclass[../main.tex]{subfiles}
\begin{document}
\lecture{16}{Mon 09 Nov}{lundi}
\begin{thm}
	L'espace $M_d( K) $ muni de l'addition des matrices et de la multiplication est un anneau ( non-commutatif en general) dont l'element neutre est la matrice carree nulle $0_d = 0_{d\times d} $ et dont l'unite est la matrice identite $\id_d$
\end{thm}
\begin{thm}
	Soit $V$ de dimension finie $d$ et $B$ une base de $V$, l'application 
	\[ 
		Mat_B: End( V) \mapsto M_d( K) 
	\]
	est un isomorphisme d'anneaux ( et donc de $K $-algebres) pour les lois d'addition et de multiplication decrites precedemment.
\end{thm}
De plus, on a que
\[ 
	^{t}\bullet: M_d( K) \mapsto M_{d} ( K) 
\]
est un endomorphisme.
\subsection{Le groupe lineaire}
\[ 
	End( V) ^{\times}= \left\{ \phi \in End( V) \text{ qui sont bijectifs et donc inversible pour la composition  }  \right\} 
\]
On note
\[ 
	M_d( K) ^{\times}= GL_d( K) = \text{ le groupe lineaire de  } K^{d}
\]

Donc
\begin{align*}
mat_B: GL( V) \mapsto GL_d( K) \\
\phi \mapsto mat_B( \phi) 
\end{align*}
\begin{propo}[Critere d'inversibilite]
	Pour qu'une matrice carree $M= ( m_{ij} ) _{i,j\leq d} \in M_d( K) $soit inversible il faut et il suffit que la famille des collonnes $Col( M) $ forme une famille libre.
\end{propo}
\begin{proof}
	Si $M= mat_B( \phi) \phi\in End( V) $.\\
	Si $Col( M) $ est une famille libre, l'ensemble des images des elements de $B$ forme une famille libre, elle est de taille $d$, donc elle est generatrice.\\
	Donc $\phi$ est surjective et injective.
\end{proof}
\begin{rmq}
Dans ce critere, il est equivalent de regarder la famille des lignes.
\end{rmq}
\begin{propo}
	La transposition est une bijection de $GL_d( K) $ sur lui-meme qui verifie
	\[ 
		\forall M,N \in GL_d( K) ^{t}M^{-1}= ^{t}M^{-1}, ^{t}( M.N)= ^{t}N. ^{t}M
	\]
	
\end{propo}
\begin{proof}
Si $M$ est inversible
\[ 
\exists N= M^{-1}
\]
tel que
\[ 
M.M^{-1}= Id\quad M^{-1}M= Id
\]
On utilise la formule de transposition sur $M.M^{-1}$

\end{proof}
\subsection{Changement de Base}

Soient $B\subset V$ et $B'\subset W$ et $\phi: V \to W$.\\
A nouveau, supposons que $B_n\subset V$ et $B'_n\subset W$, avec
\[ 
	\phi \to mat_{B'_nB_n} ( \phi) = M_n
\]
Quelle est la relation entre $M$ et $M_n$.
\begin{propo}[Formule de changement de base]
	Soient $B,B_n\subset V$ et $B',B'_n\subset W$ des bases de $V$ et $W$. On a la relation
	\[ 
		Mat_{B'_nB_n} ( \phi) = Mat_{B'_nB'}(\id_w) Mat_{B'B} ( \phi).Mat_{B,B_n} ( \id_v) 
	\]
	

\end{propo}

\begin{proof}
On a 
\[ 
\phi= \id_w \circ\phi\circ\id_v
\]
On utilise le calcul des matrices associee a des compositions d'applications lineaires dans ces bases convenables.\\
On a
\[ 
\phi: V \to V \to W
\]
avec
\[ 
	mat_{B'B'} ( \phi\circ \id_v) = mat_{B'B} ( \phi) mat_{BB_n } ( \phi) 
\]

\end{proof}
\begin{defn}[Matrice de Passage]
	\[ 
		M_{BB_n} = M_{BB_n} ( \id_v) = \text{ la matrice exprimant les coordonnees de  } \left\{ \id_v( e_{nj} j\leq d)  \right\} = B_n
	\]
	exprimes dans la base $B$.

\end{defn}

\begin{rmq}
Les $M_{BB_n} $ sont inversibles.
\end{rmq}
\begin{propo}
Soit trois bases $B,B_1,B_2\subset V$, on a
\begin{enumerate}
\item Formule d'inversion
	\[ 
	Mat_{BB_1} Mat_{BB_1} = \id_d
	\]
	En particulier une matrice de passage est inversible ( dans $M_d( K) $) et son inverse est la matrice de passage de la base initiale a la nouvelle base

\item Formule de transitivite
	\[ 
	Mat_{BB_2} = Mat_{BB_1} Mat_{B_1B} 
	\]

\end{enumerate}

\end{propo}
\begin{proof}
Consequence directe de la formule de la matrice associee a la composition de 2 applications lineaires appliquees a 
\[ 
\phi = \id_V \quad \psi=\id
\]
Si on applique la formule a $B_2=B$ on trouve le resultat desire.


\end{proof}
Le cas des endomorphismes si $W= V$, $B'=B$ et $B'_n = B_n$.\\
Soit $\phi \in End( V) $ 
\[ 
	mat_{B_nB_n}( \phi) = mat_{B_nB} mat_{BB} ( \phi) mat_{BB_n} 
\]
On a vu que 
\[ 
mat_{B_nB} = mat_{BB_n}^{-1}
\]
Donc la formule de changement de base de $\phi$.
\[ 
	mat_{B_nB_n} (\phi) = mat_{BB_n}^{-1} mat_{BB} ( \phi) mat_{BB_n} 
\]

Si la base de depart egal a la base d'arrivee, on le note
\[ 
	mat_{B_n} ( \phi) = mat_{B_nB} mat_B( \phi) mat_{B_nB} ^{-1}
\]
\begin{defn}
	Deux matrices $M$ et $N \in M_{d'\times d} ( K) $sont dites equivalentes si il existe des matrices inversibles $A \in GL_{d'} ( K), B\in GL_d( K)  $ telles que
	\[ 
	N= A.M.B
	\]
		
\end{defn}
\begin{propo}
Deux matrices sont equivalents si et seulement si il existe $V$ de dimension $d$ et $W$ de dimension $d' $, des bases $B,B_n\subset V$et $B'B'_n\subset W$ et une application lineaire $\phi> V \to W$ telles que
\[ 
	M= mat_{B'B} ( \phi) N= mat_{B'_nB_n} ( \phi) 
\]

\end{propo}
\begin{proof}
	Si $M= mat_{B'B} ( \phi)$ et $N= mat_{B'_nB_n} ( \phi) $, alors
	\[ 
	N= mat_{B'_nB'} M.mat_{BB_n} 
	\]
	
\end{proof}
\begin{propo}
Si $M$ et $N$ sont equivalents alors
\[ 
	rg( M) = rg( N) 
\]
\end{propo}
\begin{proof}
	\[ 
		rg( M) = rg( \phi) = rg( N) 
	\]
	
\end{proof}
\begin{rmq}
La relation ``etre equivalent'' est une relation d'equivalence.
\end{rmq}
\subsection{Conjugaison}
\begin{defn}
	Soit $C\in GL_d( K) $ une matrice inversible. On note $Ad( C)$ l'application dite de conjugaison par $C$ :
	\[ 
		Ad( C) : M \mapsto C.M.C^{-1}
	\]
	
\end{defn}
\begin{propo}
	La conjugaison $Ad( C)$ est un automorphisme de l'algebre $M_d( K) $ 
\end{propo}
\begin{proof}
	\begin{align*}
		C.( \lambda M + N).C^{-1}= \lambda Ad( C) .M + Ad(  C).N
	\end{align*}
	Multiplicativite
	\begin{align*}
	CMNC^{-1}= CM\id NC^{-1}= CMC^{-1}CNC^{-1}
	\end{align*}
	Identite
	\begin{align*}
	C^{-1}CMC^{-1}C = M
	\end{align*}
	
	
	
\end{proof}
\begin{defn}[Application Adjointe]
	\[ 
		Ad: C \in GL_d( K) \to Ad( C)  \in GL( M_d( K) ) 
	\]
	$Ad$ est un morphisme de groupe.\\
	On le verifie...


\end{defn}
On donne un nom a $\Im( Ad) = Ad( GL_d) $, on l'appelle le groupe des automorphismes interieurs.

\begin{propo}
	L'application adjointe $Ad( \bullet) $ est un morphisme de groupes. Son noyau est forme par les matrices scalaires
	\[ 
	\ker Ad = K^{\times}\id
	\]
	
\end{propo}
\begin{proof}
Si $C= \lambda.\id$, avec $\lambda\neq 0$, $\lambda\in K^{\times}$, alors
\[ 
C^{-1} = \lambda^{-1}\id
\]
Donc
\[ 
CMC^{-1}= M
\]
Donc
\[ 
	K^{\times}\id \subset \ker Ad( \bullet) 
\]
Soit Donc $C \in GL_d( K) $ telle que
\[ 
	\forall M \in M_d( K) 
\]
on a
\[ 
CMC^{-1}= M
\]
Il suit
\[ 
	CM = MC \Rightarrow CM-MC = 0
\]
Donc
\[ 
	\forall M [ C,M] = CM-MC = 0
\]
L'application $M\to [ C,M]$ est lineaire et dire que pour tout $M$ 
\[ 
[ C,M] = 0 \iff [ C,E] =0
\]
pour $E$ une base des matrices carrees.\\
On prend la base canonique.\\
\begin{lemma}
\[ 
E_{ij} E_{kl} = \delta_{j k} E_{il} 
\]
\end{lemma}
\begin{proof}[du lemme]
	\[ 
	E_{ij} = mat_{B_0} E_{ij} 
	\]
	Donc
	\[ 
		E_{ij} ( v) = e_j^{*}. e_i
	\]
	Donc 
	\[ 
		E_{ij} E_{kl} ( v) = E_{ij} ( E_l^{*}( v) e_k) = e_l^{*}( v) E_{ij} ( e_k)  = e_l^{*}( v) e_j^{*}( e_k) e_i
	\]
	


	Or $e_{j}^{*}( e_k)  = \delta_{jk} $.\\
	Donc
	\[ 
		E_{ij} E_{kl} ( v) = \delta _{j k} e_{l} ^{*}( v) e_i = \delta _{jk} E_{il} ( v) 
	\]
	

\end{proof}
Soit $C$ tel que pour tout $E_{ij} $ on a 
\[ 
C E_{ij}  = E_{ij} C
\]
Donc
\[ 
C = \sum_{k,l\leq d} c_{kl} E_{kl} 
\]
Donc
\begin{align*}
	E_{ij} . C &= E_{ij} ( \sum_{k,l} c_{kl} E_{kl} ) \\
		   &= \sum_{k,l} c_{kl} E_{ij} E_{kl}\\
		   &= \sum_{k,l} c_{kl} \delta_{jk} E_{il} \\
		   &= \sum_{l\leq d} c_{jl} E_{il} 
\end{align*}
De meme
\begin{align*}
	C. E_{ij} &= \sum_{k,l\leq d} c_{kl} E_{kl} E_{ij}\\
                  &= \sum_{k,l\leq d } c_{kl} \delta_{li} E_{kj}\\
		  &= \sum_{k\leq d} c_{ki} E_{kj} 
\end{align*}
Seule possibilite pour l'egalite est que $c_{ki} = 0$ sauf si $k=i$ de meme $c_{jl} =0$ sauf si $l=j$.\\
Donc 
\[ 
CE_{ij} = E_{ij} C
\]
si et seulement si
\[ 
c_{ii} E_{ij} = C_{jj} E_{ij} 
\]
et les $c_{ki} = 0$ si $k\neq i$ et $c_{jl} =0$ si $l\neq j$.
Donc $c_{ii} = c_{jj} $ valable $\forall i,j \leq d$ et $C_{ii} \neq 0$ car $C$ inversible.
\end{proof}



\begin{defn}
	On dit que deux matrices $M.N$ sont semblables on conjuguees si il existe $C \in GL_d( K) $ tel que
	\[ 
	N= CMC^{-1}
	\]
	
\end{defn}
\begin{propo}
La relation etre semblable est une relation d'equivalence.
\end{propo}
\begin{proof}
\begin{itemize}
\item Reflexive
	\[ 
	M= \id M\id^{-1}
	\]

\item Symmetrique si $N= CMC^{-1}$ 
	\[ 
	M= C^{-1}NC
	\]

\item Transitive
	\[ 
	N= CMC^{-1} \text{ et } O = DND^{-1}
	\]
	Alors
	\[ 
		O= DCMC^{-1}D^{-1} = ( DC) M ( DC) ^{-1}
	\]
	
	
\end{itemize}
La classe d'equivalence de $M$ s'appelle la classe de conjugaison de $M$.

\end{proof}
\begin{rmq}
	$M$ et $N \in M_d( K) $sont semblables si et seulement il existe $V$ de $\dim d$, deux bases $B,B_n$ de $V$ et $\phi \in End( V) $ tel que
	\[ 
	M= mat_B \phi
	\]
	et 
	\[ 
		N= mat_{B_n} \phi
	\]
	
\end{rmq}












	








\end{document}	
