\documentclass[../main.tex]{subfiles}
\begin{document}
\lecture{27}{Tue 15 Dec}{Fin Determinants}
\begin{rmq}
Les resultats precedents valent aussi si $M$ est ``triangulaire superieure par bloc''.
\end{rmq}
\subsubsection{Operations sur les lignes/Colonnes}
\begin{lemma}
Soient $T_{ij} , D_{i,\lambda} , CL_{ij,\mu} $ les matrices des transformations elementaires des lignes d'une matrice, on a
\begin{align*}
\det T_{ij} = -1\\
\det D_{i,\lambda} = \lambda\\
\det CL_{ij,\mu} = 1
\end{align*}
\begin{proof}
$T_{ij} $ est une matrice de transposition
\[ 
	\det T_{ij} = sign( ij) = -1
\]
$D_{i,\lambda} $ est une matrice diagonale avec des 1 sur la diagonale sauf a la $i$ eme ligne ou on a $\lambda$.\\
$CL_{ij,\mu} $ est triangulaire superieure ou inferieure, donc
\[ 
\det CL_{ij,\mu} = 1
\]


\end{proof}

\end{lemma}
\begin{crly}
Supposons que $N$ soit deduite de $M$ par un des trois types de transformations elementaires sur les lignes de $M$, alors on a
\begin{itemize}
\item $\det M = - \det N$ 
\item $\det M = \lambda^{-1}\det N$ 
\item $\det M = \det N$
\end{itemize}

\end{crly}
\begin{proof}
immediat par lemme.
\end{proof}
\subsubsection{Developpement de Lagrange suivant une colonne /ligne}
Soit $M$ une matrice, on definit $M( k|l) $ contient tous les indices de $M=( m_{ij} ) $, pour $i\neq k$ ou $i \neq j$

\begin{defn}
Pour $k,l\leq d$ 
\begin{itemize}
	\item Le determinant de $M( k|l) $ est le $( k,l) $ mineur de $M$.
	\item $( -1)^{k+l}\det( M( k|l) ) $ est le $( k,l) $ cofacteur de $M$.
\end{itemize}

\end{defn}
\begin{thm}
On a pour tout $j\leq d$ 
\[ 
	\det M = \sum_{i=1}^{d} m_{ij} ( -1)^{i+j}\det( M( i|j) ) 
\]

\end{thm}
\subsection{Le polynome caracteristique}
\begin{defn}
Le polynome caracteristique de $M$ est le determinant
\[ 
	P_{car,M} ( X) = \det ( X.\id - M)  = \sum_{\sigma} sign( \sigma) \prod_{i=1}^{d} ( X \delta_{i\sigma( i) } - m_{i\sigma( i) } ) \in K[X]
\]

\end{defn}
\begin{propo}
Le polynome caracteristique est un polynome unitaire de degre $d$ et si on ecrit
\[ 
	\det( X.\id - M) = X^{d} + a_{d-1} X^{d-1} + \ldots + a_d
\]
On a 
\begin{align*}
	a_0 = P( 0)  = ( -1)^{d}\det M\\
	a_{d-1} = -tr( M)  = m_{11} + \ldots 
\end{align*}

\end{propo}
\begin{proof}
\begin{align*}
	P_{car,M}( X) = \sum_\sigma sign( \sigma) \prod_{i=1}^{d}( X\delta_{i\sigma( i) } - m_{i\sigma( 1) } ) 
\end{align*}
On trouve que $P_{car,M} ( X) $ a son terme de plus haut degre provenant de $\sigma=\id$ et donc $P_{car,M} ( X) $ est unitaire de degre $d$.\\
Si $\sigma\neq \id$, $\exists i$ tel que $\sigma( i) =j \neq i$, mais alors $\sigma( j) \neq j$.\\
Donc le produit 
\[ 
	sign( \sigma) \prod_{i=1}^{d}( X\delta_{i\sigma( i) } -m_{i\sigma( i) } ) 
\]
contient au  moins $2$ facteurs de degre $\leq 0$ et le produit est donc de degre $\leq d-2$, et donc 
\[ 
	P_{car,M} ( X)  = X^{d} + a_{d-1} X^{d-1} + \ldots + a_0
\]

\end{proof}
\begin{propo}
Le polynome est invariant de la classe de conjugaison de la matrice $M$
\end{propo}
\begin{proof}
	\begin{align*}
		\det( XP\id P^{-1} - PMP^{-1}) \\
		=\det( P( X\id - M) P^{-1}) \\
		- \det( X\id - M) 
	\end{align*}
	
\end{proof}
\begin{defn}
	Soit $\phi \in End( V) $ une application lineaire, on definit son polynome caracteristique par
	\[ 
		P_{car, \phi} ( X) = P_{car,M} ( X) 
	\]
	ou $M= Mat( \phi)$ est la matrice de $\phi$ dans une base quelconque de $V$.
\end{defn}
\begin{defn}
On definit la trace de $\phi$ comme etant la trace de $M$ 
\[ 
	tr( \phi) = tr( M) = m_{11} + \ldots
\]
et cette definition ne depend pas du choix de la base $B$.
\end{defn}
\begin{thm}
Soit $P_{car,\phi} $ le polynome caracteristique d'une application lineaire $\phi$.\\
Les enonces suivants sont equivalents
\begin{enumerate}
	\item Le scalaire $\lambda\in K$ est racine de $P_{car, \phi} :P_{car,\phi} ( \lambda) = 0$ 
	\item Il existe $v \in V \setminus \left\{ 0 \right\} $ tel que $\phi( v) = \lambda v$	
\end{enumerate}

\end{thm}
\begin{proof}
	Si $\lambda$ est tel que $P_{car,\phi} ( \lambda) =0$ $\iff$ $\det ( \lambda\id - \phi) = 0$.\\
		Ceci implique que le ker est non-nul.\\
		Idem dans l'autre direction.
\end{proof}
\begin{defn}
Soit $\lambda\in K$, le sous-espace 
\[ 
	V_{\phi,\lambda} = \ker( \phi -\lambda\id) = \left\{ v \in V, \phi( v) = \lambda v \right\} 
\]
est appelle sous-espace propre associe a $\lambda$. Si $V_{\phi,\lambda} \neq \left\{ 0 \right\} $, on dit que $\lambda$ est une valeur propre de $\phi$ et tout vecteur non-nul de $V_{\phi,\lambda} $ est appelle vecteur propre de $\phi$ associe a la valeur propre $\lambda$.\\
L'ensemble des valeurs propres de $\phi$ est appelle le spectre de $\phi$ ( dans $K$) et est note
\[ 
	Spec_{\phi} ( K) 
\]

\end{defn}



\end{document}	
