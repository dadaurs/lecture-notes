\documentclass[../main.tex]{subfiles}
\begin{document}
\lecture{19}{Tue 17 Nov}{Nombres Complexes 2}
\begin{defn}
	$pol( z) $ s'appelle la decomposition polaire de $z$. Le premier terme $|z|$ est le module et se note aussi $\rho( z) $ et le second terme $\frac{z}{|z|}$ est appele argument complexe de $z$ et on le note
	\[ 
		\frac{z}{|z|}= e^{i \theta( z) } 
	\]
	Si on decompose l'argument complexe en partie reelle et imaginaire
	\[ 
		\frac{z}{|z|}= e^{i\theta( z) } = c( z) + i s( z) 
	\]
	On a donc
	\[ 
		c( z) \in [ -1,1] \text{ s'appelle le cosinus de $z$ } 
	\]
	\begin{align*}
	s( z) \in [ -1,1] \text{ s'appelle le sinus de $z$ } 
	\end{align*}
\end{defn}
\begin{propo}[Formules de trigonometrie]
	On retrouve les formules habituelles de trigonometrie
	\begin{itemize}
	\item Formules de produit: pour $z,z' \in \mathbb{C}^{\times}$
		\[ 
			c( z.z') = c( z)  c( z' ) - s( z) s( z') , s( z.z') = s( z) .c( z') + s( z') c( z) 
		\]

	\item Formule d'inveresion
		\[ 
			e^{i \theta( \frac{1}{z}) } = c( z) - i s( z) 
		\]
		
	\item Formule de l'angle double
		\[ 
			c( z^{2}) = c( z) ^{2}- s( z) ^{2}, s( z^{2}) = 2 s( z) c( z) 
		\]
		
		et plus generalement

	\item Formules de Moivre: pour $n\geq 0$ 
		\[ 
			c( z^{n}) + i s( z^{n}) = ( c( z) + i s( z) ) ^{n} = \sum_{k=0}^{ n}C_{n} ^{k}c^{n-k}s^{k}
		\]
		
	\end{itemize}
	

\end{propo}
\begin{proof}
Pour les formules de Moivre, on a 
\begin{align*}
	e^{i\theta( z^{n}) } &= ( e^{i\theta( z) } )^{n} = ( c( z) + i s( z) ) ^{n}\\
			     &= c( z) ^{n}+ n c( z) ^{n=1}i s( z)  + \sum_{k=2}^{ n}C_n^{k} c( z) ^{n-k}( i s( z) ) ^{k}\\
			     &= \sum_{k=0}^{ n}C_n^{k}c( z) ^{n-k}( i s( z) ) ^{k}\\
			     &= \sum_{k=0}^{ n} C_n^{k} i ^{k} c( z) ^{n-k} s( z) ^{k}	
\end{align*}
En posant
\[ 
k' = 2 [ \frac{k}{2}] ,\quad k = k' + 
\begin{cases}
0 \text{ si pair } \\
1 \text{ si impair } 
\end{cases}
\]

on obtient que
\begin{align*}
	e^{i\theta( z^{n}) } &= ( e^{i \theta( z) }) ^{n}= ( c( z) + i s( z) ) ^{n}\\
			     &= \sum_{k'=0}^{ \frac{n}{2}} C_n^{2k'} ( -1) ^{k'} c( z) ^{n-2k'} s( z) ^{2k'}
			     + \sum_{k'=0}^{ \frac{n}{2}} C_n^{2k'+1} ( -1) ^{k'} i c( z) ^{n- ( 2k'+1) } s( z) ^{2k'+1}	
\end{align*}


\end{proof}
\begin{thm}
Il existe un unique morphisme de groupe
\[ 
	\theta \in ( \mathbb{R}, +) \mapsto \exp i \theta \in ( \mathbb{C}^{( 1) }, \times) 
\]
qui est derivable et qui verifie
\[ 
	e^{i \bullet'}( 0) = i
\]
Ce morphisme est surjectif et son noyau est de la forme
\[ 
\ker e^{i \bullet} = 2\pi. \mathbb{Z}
\]

\end{thm}
On dit que  $\theta \to e^{i \theta}$ est derivable si les fonctions partie reelle et partie imaginaires sont derivables
\[ 
	( e^{i\theta})' = ( \re e^{i\theta} )' + i( \im e^{i\theta} )'
\]
Theoreme sans preuve.
\begin{defn}
Soit $z$ un nombre complexe de module 1.\\
L'argument de $z$ 
\[ 
	\arg ( z)  = \theta ( \mod 2\pi) 
\]
Plus generalement, pour $z\in \mathbb{C}^{\times}$, on definit son argument par 
\[ 
\arg z = \arg \frac{z}{|z|}
\]


\end{defn}
\begin{defn}
Soit $\theta \in \mathbb{R}$, on a 
\[ 
e^{i\theta } = \cos \theta + i \sin \theta
\]
De ceci, on retrouve les formules d'addition.

\end{defn}




\end{document}	
