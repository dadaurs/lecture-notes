\documentclass[../main.tex]{subfiles}
\begin{document}
\lecture{9}{Tue 13 Oct}{Corps}
\begin{propo}
Soit $A$ un anneau integre, alirs il existe un corps $K$ et un morphisme d'anneau injectif
\[ 
\iota: A \inj K
\]
de sorte qu'on peut considerer $A$ comme un sous-anneau de $K$ en identifiant $A$ à $\iota(A) \subset K$ et tel que $K$ a la propriete de minimalite suivante: pout tout corps $K'$ et tout morphisme injectif
\[ 
\iota' : A \inj K'
\]
de sorte que $A$ peut etre identifie a un sous-crops de $K'$, il existe un morphisme ( necessairement injectif) 
\[ 
\iota' : K \inj K'
\]
prologeant le morphisme $\iota'$ ( ainsi $A$ et $K$ peuvent etre vus comme des sous-anneaux de  $K'$) 
\end{propo}
\begin{defn}
On appelle ce corps $K$ le corps des fractions de $A$.
\end{defn}

\begin{exemple}
\begin{itemize}
\item $Frac(\mathbb{Z}) = \mathbb{Q}$	
\item $Frac( \mathbb{R}[X]) = \mathbb{R}(X)$ ( défini comme avant)
\end{itemize}
\end{exemple}
\begin{proof}
Construison $K$.\\
$A$ est intègre.\\
On considère l'ensemble produit
\[ 
A \times A \setminus \left\{ 0 \right\} = \left\{ ( a,b) | a,b \in A, b \neq 0_A \right\} 
\]
On définit sur cet ensemble une relation.\\
$ ( a,b) \sim ( a',b')$ si et seulement si $a.b' = a'.b$, la relation  $\sim$ est une relation d'équivalence.
\begin{itemize}
\item Symmetrique: Si $a.b'= a'.b \iff a'.b = a.b' \iff ( a',b')\sim ( a,b)$
\item Reflexive: $ ( a,b) \sim ( a,b) \iff a.b = a.b$ 
	\item Transitive: $(a,b) \sim ( a',b')$ et $(a',b') \sim ( a'', b'')$.\\
		On a $a.b' = a'.b$ et $a'.b'' = a''.b'$.
		\begin{align*}
			\implies ab'b'' &= a'b'b''\\
		\implies a.b''.b' &= a.b''.b'\\
		\implies a.b''b' = a'b''b = a''b'b = a''bb'\\
		\implies ( ab'' -a''b ). b' = 0_A
		\end{align*}
		Comme $A$ est intègre,
		\[ 
		ab'' - a''b = 0_A \text{ ou bien } b'= 0_A
		\]
		Donc 
		\[ 
	ab'' - a''b = 0_A	
		\]
		Donc $ ( a,b) \sim ( a'', b'')$
\end{itemize}
Soit $K = A \times A\setminus \left\{ 0 \right\} / \sim$ l'ensemble des classe d'équivalences.\\
On note $\frac{a}{b}$ la classe de l'élément $(a,b)$.\\
On va munir $K$ d'une addition et d'une multiplication d'un $0_K$, d'une $1_K$ ainsi que 
\[ 
\iota: A \inj K
\]

Il faut maintenant vérifier toutes les propriétés d'un corps.
\[ 
+: \frac{a}{b} + \frac{a'}{b'}  = \frac{ab' + a'b}{bb'}
\]
$b.b'\neq 0_A$ vrai car $b,b' \neq 0$ et $A$ integre.

On doit vérifier que cette définition ne dépend que des classes d'équivalence $\frac{a}{b}$ et $\frac{a'}{b'}$.\\
Si $ ( a'',b'')\sim ( a',b')$ on veut voir que $\frac{a}{b} + \frac{a'}{b'} = \frac{a}{b} + \frac{a''}{b''}$.
On doit vérifier que 
\[ 
	\underbrace{( ab' + a'b )}_{abb'b'' + a'b^{2}b''}.bb'' = \underbrace{( ab'' + a''b)}_{abb'b'' + a''b^{2}b'}.bb'
\]
On sait que $a'b'' = a''b'$.
\[ 
\Rightarrow a'b^{2}b'' = a''b^{2}b'
\]
On fait pareil pour définir la multiplication $\times$ 
\[ 
\frac{a}{b} \times \frac{a'}{b'} = \frac{a.a'}{b.b'}
\]
et on doit vérifier que si $\frac{a''}{b''} = \frac{a'}{b'}$alors $\frac{a}{b}\times \frac{a'}{b'} = \frac{a}{b} \times \frac{a''}{b''}$ sachant que $a'b'' = a'' b'$.\\
On vérifie que  $+, \times$ sont commutatives, associatives, distributives.\\
On définit $0_k = \frac{0}{1_A}$ et $1_K = \frac{1_A}{1_A}$ \\
Enfin, dire que $\frac{a}{b} \neq 0_K \iff a \text{ et } b \neq 0_A$ et alors si $\frac{a}{b}\neq 0_K$ $\frac{b}{a}\times \frac{a}{b} = \frac{1_A}{1_A}= 1_K$.\\
On a un morphisme injectif
\[ 
\iota: A \inj K
\]
donné par 
\[ 
	\iota(a) = \frac{a}{1_A}
\]
On vérifie que c'est un morphisme d'anneau et, si $\iota(a) = 0_K = \frac{0_A}{1_A} \iff \frac{a}{1_A} = \frac{0_A}{1_A} \iff a = 0_A$, donc
\[ 
\ker \iota = \left\{ 0_A \right\} 
\]
donc $\iota$ est injectif.
\end{proof}
\subsection{Caractéristique des Corps}
$K$ un corps,
\begin{align*}
Can_K: \mathbb{Z} \to A\\
n \to n.1_K = n_k\\
\ker(Can_K) = p \mathbb{Z}, p \geq 0
\end{align*}

\begin{defn}[Caractéristique]\index{Caractéristique}\label{defn:caracteristique}
	L'entier $p$ s'appelle la caractéristique du corps $K$ et se note
	\[ 
		car(K)
	\]
\end{defn}
Si $p=0$ : $\ker Can_K = \left\{ 0_\mathbb{Z} \right\} $, donc $Can_K$ est injectif et donc $\mathbb{Z}$ peut être vu comme sous-anneau de $K$.
\[ 
n \in \mathbb{Z} \to n_K \in K
\]
Si $n \neq 0, n_K \neq 0$ et $\frac{1}{n_K}$ existe et pour tout $a,b \in \mathbb{Z},b \neq 0$, on définit 
\[ 
	( \frac{a}{b})_K = a_K /b_K \in K
\]
On dispose d'un morphisme injectif
\begin{align*}
Can_K : \mathbb{Q} \inj K\\
\frac{a}{b} \to \frac{a_K}{b_K}
\end{align*}
Si $Car(K) = 0$, le corps $\mathbb{Q}$ est un sous-corps de $K$.
\begin{lemma}
	Si $car(K) > 0 $, alors $car(K)=p$ est un nombre premier.
\end{lemma}
\begin{proof}
Si $p=1$, $\ker Can_K = \mathbb{Z}$ 
\[ 
	\Rightarrow Can_K ( 1) = 1_K =0_K \contra
\]
Donc $p\geq 2$.\\
Soit une factorisation 
\[ 
p=q_1\cdot q_2
\]
non-triviale ( $ q_1,q_2 \geq 2$ )
\[ 
	0_K=Can_K(p) = Can_K(q_1\cdot q_2) = Can_K(q_1) \cdot Can_K(q_2)
\]
Comme $K$ est intègre, $Can_K(q_1) = 0_K$
\begin{align*}
q_1 \in \ker Can_K = p \mathbb{Z}\\
q_1 = pk, k \in \mathbb{Z}\setminus \left\{ 0 \right\} 
\end{align*}
Donc $q_1\geq p$ mais comme $q_2 \geq 2	$
\[ 
q_2 \leq \frac{p}{2} < p
\]
Donc $p$ est premier.
\end{proof}
\begin{defn}
\[ 
	\mathbb{F}_p = Can_K(\mathbb{Z}) = \mathbb{Z}.1_K
\]

\end{defn}
\begin{lemma}
	L'anneau $\mathbb{F}_p$ est un corps fini de cardinal $p$.
\end{lemma}
\begin{proof}
Si $n\in \mathbb{Z}$ et $k \in \mathbb{Z}$ 
\[ 
	( n+pk)_K = n_K + p_k . k_K = n_k
\]
Donc, si $r \in \left\{ 0, \ldots, p \right\} $ le reste de la division euclidienne de $n$ par $p$
\[ 
	\mathbb{Z}.1_K =  \left\{ 0_K, 1_K, \ldots, ( p-1)_K \right\} 
\]
$\mathcal{F}_p$ est de cardinal $p$.\\
Il faut montrer que si $0<i \neq j \leq p-1$ 
\[ 
i_K \neq j_K
\]
mais 
\[ 
	i_K - j_K = ( i-j)_K
\]
et comme $0 \leq i,j \leq p-1$, $0 \neq |i-j| < p$
Donc $i-j$ ne peut pas etre un multiple de $p$, donc $i-j \notin \ker Can_K$
Donc 
\[ 
	( i-j)_K = i_K - j_K \neq 0_K
\]

\end{proof}
\begin{lemma}
	Un anneau commutatif integre et fini est un corps
\end{lemma}
\begin{proof}
exercice
\end{proof}
$\mathbb{F}$ est integre car c'est un sous-anneau du corps $K$ et il est fini de cardinal  $p$.

\begin{defn}
	Le corps $ \mathbb{Q} \subset K$ si $car(K) = 0$ ou bien $\mathbb{F}_p \subset K$ ( si $car(K)=p > 0$) s'appelle le sous-corps premier de $K$.
\end{defn}
\begin{rmq}
Le corps
\[ 
	\mathbb{F}_p \simeq ( \faktor{\mathbb{Z}}{p \mathbb{Z}}, + ,\times)
\]
l'anneau des classes de congruences module $p$
\end{rmq}
\subsection{Arithmétique des corps de caractéristique $p>0$}
\begin{propo}
Soit $K$ un corps de caractéristique $p>0$, alors l'application
\begin{align*}
\bullet^{p}: K \to K\\
x \to x^{p}
\end{align*}
est un morphisme d'anneaux non-nul ( donc nécessairement injectif).
\end{propo}
\begin{defn}
	Soit $K$ un corps de caractersitique $p$, le morphisme d'anneau precedent s'appelle le morphisme de Frobenius ( ou simplement le Frobenius)de $K$ se note
	\[ 
	frob_p: x \to x^{p}
	\]
\end{defn}
\begin{proof}
$\forall x, y \in K$
\begin{align*}
( x.y)^{p} = x.y.x.y.x.y.x.y \ldots\\
= x^{p} y^{p}
\end{align*}
$\forall x,y \in K$ 
\[ 
	( x+y)^{p} = x^{p} + y^{p}
\]
Comme $K$ est commutatif, on a la formule du binome de Newton
\begin{align*}
	( x+y)^{p} &= \sum_{k=0}^{ p} \binom{k}{p} x^{k} y ^{p-k}\\
&= x^{p} + y^{p} + \sum_{k=1}^{ p-1}\binom{k}{p} x^{k} y^{p-k}
\end{align*}
\begin{lemma}
Si $1 \leq k \leq p-1$, alors
\[ 
	p | \binom{p}{k}
\]

\end{lemma}
Or
\[ 
	\binom{p}{k} x^{k} y^{p-k} = \binom{p}{k}x^{k} y^{p-k} = 0_K \cdot x^{k} y^{p-k}
\]

\end{proof}














\end{document}	
