\documentclass[../main.tex]{subfiles}
\begin{document}
\lecture{24}{Mon 07 Dec}{Determinants}
\subsection{Formes Symmetriques/Alternees}
\begin{defn}
Une forme multilineaire
\[ 
\Lambda: V^{n}\mapsto K
\]
est dite 
\begin{itemize}
\item Symetrique si $\forall i \neq j \leq n$ 
	\[ 
		\Lambda( v_1, \ldots, v_i, \ldots v_j) = \Lambda( v_1, \ldots, v_j,\ldots, v_{i} ,\ldots) 
	\]
	Autrement dit si sa valeur ne change pas quand on echange deux composantes.
\item Alternee si $\forall i \neq j \leq n$ 
	\[ 
		\Lambda(  v_1, \ldots, v_i, \ldots v_j) = -\Lambda(  v_1, \ldots, v_j, \ldots v_i) 
	\]
	Autrement dit si sa valeur est changee en son opposee si on echange deux composantes distinctes.\\
\end{itemize}

\end{defn}
\begin{thm}
	On suppose que $car( K) \neq 2$. Soit $d= \dim V$. On a 
	\[ 
		\dim Alt^{n}( V;K) = 
		\begin{cases}
		0 \text{ si $n>d$ } \\
		1 \text{ si $n=d$ } \\
		 C^{n}_d \text{ si } n \leq d
		\end{cases}
	\]
	
\end{thm}
\begin{proof}
Soit $\Lambda$ qui est alternee, alors $\forall i\neq j$ 
\[ 
	\Lambda( v_1, \ldots, v, \ldots, v, \ldots v_n) 
\]
Donc
\[ 
	2\Lambda( v_1,\ldots, v \ldots, v, \ldots ) = 0
\]
Donc
\[ 
	\Lambda( v_1, \ldots, v, \ldots, v, \ldots, v_n) = 0
\]
Plus generalement, si la famille
\[ 
\left\{ v_1, \ldots, v_n \right\} 
\]
est liee alors
\[ 
	\Lambda( v_1, \ldots, v_n) =0
\]
Si la famille est liee, un des vecteurs s'exprime en fonction des autres.\\
Supposons que c'est $v_n$.\\
Alors
\[ 
v_n = \sum \lambda_i v_i
\]
Donc
\[ 
	\Lambda( v_1, \ldots, v_n) = \lambda_1 \Lambda( v_1, \ldots, v_1) + \ldots + \lambda_{n-1}  \Lambda( v_1, \ldots, v_{n-1} )= 0
\]
Si $n>d$, toute famille $ \left\{ v_1, \ldots, v_n \right\} $de $n$ vecteurs dans un espace de $\dim d< n$ est liee et donc 
\[ 
	\Lambda( v_1, \ldots, v_n) =0
\]
Cas $d=n$.\\
On va montrer que $\dim Alt^{( d )}( V,K) \leq 1$.\\
Comme $\Lambda$ est multilineaire,
\[ 
	\Lambda = \sum_{j_1=1} \ldots \sum_{j_d=1} \Lambda( e_{j_1} , \ldots, e_{j_d} ) e_{j_1} ^{*}\otimes \ldots \otimes e_{j_d} ^{*}
\]
Si pour  $l, l'\leq d$, on a 
\[ 
j_{l} = j_{l'} \Rightarrow e_{j_l} = e_{j_l'} 
\]
et comme la forme est alternee
\[ 
	\Lambda( \ldots, e_{j_l} , \ldots, e_{j_{ l' }} , \ldots)=0
\]
Donc les seuls termes non nuls de la decomposition precedente sont ceux tels que $j_1\neq j_2\neq \ldots$ et donc les coefficients sont des $\lambda( e_1, \ldots, e_d) $ et des permutations de ces vecteurs.\\
La forme $\Lambda$ est determinee des qu'on connait la valeur de 
\[ 
	\Lambda( e_1, \ldots, e_d) \in K
\]
Donc 
\[ 
	\dim Alt^{( d) }( V,K) \leq 1
\]
Pour montrer que $\dim Alt^{d}( V;K) =1$, il suffit de construire une forme alternee en $d$ variables qui est $\neq 0$.\\

\end{proof}

Soit $\sigma$ une permutation, on note alors $\Lambda_{|\sigma} $est la forme lineaire associee a la permutation des index.

\begin{propo}
	Pour tou $\sigma \in S_{n} $l'application $\bullet_{|\sigma} $ definit un endomorphisme du $K-ev$ $Mult^{n}( V;K) $.\\
	L'application
	\[ 
	\sigma \in S_n \mapsto \bullet_{|\sigma} 
	\]
	verifie
	\[ 
	\forall \Lambda, \Lambda_{|Id_n} = \Lambda
	\]
	$\forall \Lambda, \forall \sigma\tau \in S_n$, on a 
	\[ 
		\Lambda_{|\sigma\circ\tau} = ( \Lambda_{|\tau} )_{\sigma} 
	\]
	
\end{propo}
\begin{proof}
	\begin{align*}
	&\Lambda_{|\sigma\circ\tau} ( v_1, \ldots, v_n) \\
	&=\Lambda( v_{\sigma( \tau( 1) ) }, \ldots ) 
	\end{align*}
	De meme
	\begin{align*}
	\bullet_{|\sigma} ( \bullet_{|\tau} ( \Lambda) ) ( v_1, \ldots) \\
	= \Lambda_\sigma ( v_{\tau( 1) } ,\ldots,) \\
	= \Lambda( v_{\sigma( \tau( 1) ) }, \ldots ) 
	\end{align*}
	

\end{proof}
\begin{thm}
	Les formes multilineaires alternees $Alt^{n}( V;K) $ sont exactement les formes multilineaires verifiant $\forall \sigma \in S_n \Lambda_{|\sigma} = sgn( \sigma) \Lambda$
\end{thm}
\begin{proof}
	Pour les formes alternees, si $\Lambda$ verifie $\forall \sigma \Lambda_{|\sigma} = sgn( \sigma)\Lambda$, en particulier si $\sigma= \tau_{ij} 	$est la transposition qui permute $i$ et $j$ sa signature vaut $-1$.\\
Reciproquement si $\forall i \neq j$ 
\[ 
\Lambda_{|_{\tau_{ij} } } = - \Lambda
\]
Alors $\forall \sigma \in S_n$ 
\[ 
\sigma = \tau_1 \circ \ldots \circ \tau_t
\]
Alors
\[ 
	\Lambda_{|\sigma} = \Lambda_{\tau_{t-1} \ldots|\tau 1} = ( -1) ^{t}\Lambda = sgn( \sigma) ( \Lambda) 
\]


\end{proof}
\begin{thm}
	Soit $K$ un corps, $( G,\cdot) $ un groupe fini, $V$ un $K$-ev de dimension finie et 
	\[ 
		\iota: G \mapsto GL( V) 
	\]
	un morphisme de groupe de $G$ vers le groupe des automorphismes de $V$. Soit
	\[ 
		\Xi: G \mapsto ( K^{\times}, \times) 
	\]
	un morphisme de $G$ vers le groupe multiplicatif de $K$. Soit $v \in V$, alors le vecteur
	\[ 
		v_{\Xi} = \sum_{h \in G} \Xi^{-1}( h) \iota( h) ( v) 
	\]
	verifie pour tout $g \in G$ 
	\[ 
		\iota( g) ( v_{\Xi} ) = \Xi( g) v_{\Xi} 
	\]
	
\end{thm}
\begin{proof}
	Abus de notation: on notera $g \in G, v \in V, \quad g.v \text{ pour } \iota( g) ( v) $.\\
	\[ 
		v_{\Xi} = \sum_{h \in G} \Xi( h) ^{-1}. h( v) 
	\]
	Soit $g \in G$ 
	\begin{align*}
		g( v_{\Xi} ) &= g\left( \sum_{h\in G} \Xi( h) ^{-1}h( v)   \right) \\
			     &= \sum_{h\in G} \Xi( h) ^{-1}g( h( v) ) \\
			     &= \sum_{h \in G} \Xi( h) ^{-1}( g.h) ( v) 
	\end{align*}
	On pose $h'= g.h$ 
	\begin{align*}
		g( v_{\Xi} ) &= \sum_{h'\in G} ( \Xi( g)^{-1}\Xi( h') ) h'( v) \\
			     &= \Xi( g) \sum_{h'\in G} \Xi( h') ^{-1}h'( v) 
	\end{align*}
	Donc
	\[ 
		g( v_{\Xi} )= \Xi( g) v_{\Xi} 
	\]
	
	
\end{proof}


	



\end{document}	
