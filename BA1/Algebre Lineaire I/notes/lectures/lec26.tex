\documentclass[../main.tex]{subfiles}
\begin{document}
\lecture{26}{Mon 14 Dec}{Calcul de Determinants}
\begin{defn}
Le noyau du morphisme $\det$ est appele groupe speciale lineaire de $V$ et on le note
\[ 
	SL( V) = \ker \det
\]
C'est un sous-groupe normal.
\end{defn}
\begin{rmq}
	Soit $\phi, \psi \in GL( V) $, alors
	\[ 
		\det ( Ad( \psi) ( \phi) )  = \det \phi
	\]
	
\end{rmq}
\begin{defn}
	Soit $M \in M_d( K) $ une matrice carree. Le determinant $\det M$ de $M$ est
	\begin{enumerate}
	\item Le scalaire
		\[ 
		\det M = \det \phi
		\]
		ou $\phi$ est est l'application lineaire associe a $M$.
	\item Le determinant relativement a la base canonique des vecteurs d colonnes de $M$ dans l'espace des vecteurs colonnes.
	\item Le determinant relativement a la base canonique dans l'espace des vecteurs lligne
	\item La somme 
		\[ 
			\det M = \sum_{\sigma \in S_d} sgn( \sigma) m_{\sigma( 1) 1} \ldots
		\]

	\item La somme
		\[ 
			\det M = \sum_{\sigma \in S_d} sgn( \sigma) m_{ 1\sigma( 1)} \ldots
		\]

		
	\end{enumerate}
	
\end{defn}
\begin{proof}
	2)C'est tautologique, les colonnes de $M$ $C_i$ sont les coordonnees de $\phi( e_i) $ mises en colonnes
	\[ 
		\det M = \det \phi = \det ( C_1, \ldots) 	
	\]
4)

\[ 
	v_j = \phi( e_j) 
\]
On applique la formule generale a $m_{ij} = x_{ji} $, on a
\[ 
	\det_B( v_1, \ldots) = \sum_{\sigma} sign( \sigma) m_{\sigma( 1) 1} \ldots
\]
5)
On fait un changement de variable, on pose $i=1, \ldots, d$, on pose $j = \sigma( i) $, alors $i = \sigma^{-1}( j) $, et quand $i$ parcourt $ \left\{ 1, \ldots, d \right\} $, $j$ parcourt egalement cet ensemble. Donc
\[ 
	\det M = \sum_\sigma sgn( \sigma) \prod_{i=1}^{d}m_{\sigma( i) i} = \sum sgn( \sigma) \prod_{j=1}^{d}m_{j\sigma^{-1}( j) } 
\]
Cet expression est precisement le determinant dans la base canonique de $K^{d}$


\end{proof}
\begin{crly}
soit $\phi: V \to V$ et $\phi^{*}: V^{*}\to V^{*}$, alors
\[ 
\det \phi^{*} = \det \phi
\]

\end{crly}
\begin{proof}
On a $\det \phi = \det M = \det ^{t}M$
\end{proof}
\begin{crly}
Soit $M$ et $N$ deux matrices semblables, alors
\[ 
\det M = \det N
\]

\end{crly}
\subsection{Calculs de Determinants}
\subsubsection{Blocs de Matrices}
\begin{thm}
	Supposons que la matrice $M \in M_d( K) $ s'ecrit sous forme triangulaire superieure par blocs
	\[ 
	M = 
	\begin{pmatrix}
		M_1 & *\\
		0 & M_2
	\end{pmatrix}
	\]
	
Alors
\[ 
\det M = \det M_1 \det M_2
\]

\end{thm}
\begin{proof}
	Soit $M= ( m_{ij} ) $, alors
	\[ 
		\det M = \sum_{\sigma \in S_d} sign( \sigma) m_{\sigma( 1) 1} \ldots 
	\]
	Dans la somme ci-dessus, on a alors
	\[ 
		j \leq d_1, \sigma( j) > d_1
	\]
	La contribution de ces termes sera donc nulle.\\
	Il ne reste donc dans la somme que les termes correspondant aux permutations $\sigma$ telles que $\forall j \leq d_1, \sigma( j) \leq d_1$.\\
	Donc $\sigma$ laisse $ \left\{ 1, \ldots, d_1 \right\} $ et definit donc une permutation
	\[ 
	\sigma_1: \left\{ 1, \ldots, d_1 \right\} \to \left\{ 1, \ldots , d_1 \right\} 
	\]
Comme $\sigma$ est une permutation de $ \left\{ 1, \ldots, d \right\} , $ donc  $\sigma$ laisse stable $ \left\{ d_1, \ldots, d \right\} $, et induit donc une permutation de cet ensemble.
\end{proof}




\end{document}	
