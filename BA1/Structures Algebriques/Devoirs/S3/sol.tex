\documentclass[11pt, a4paper]{article}
\usepackage[utf8]{inputenc}
\usepackage[T1]{fontenc}
\usepackage[francais]{babel}
\usepackage{lmodern}
\usepackage{amsmath}
\usepackage{amssymb}
\usepackage{amsthm}
\newcommand\hr{
    \noindent\rule[0.5ex]{\linewidth}{0.5pt}\newline
}
\newtheorem{theorem}{Théorème}
\newtheorem{lemma}{Lemme}
\begin{document}
\title{Série 3}
\author{David Wiedemann}
\maketitle
On utilisera sans preuve que la composition de deux injections et une injection, que la composition d'une bijection avec une injection est une injection, que la composition d'une injection avec une bijection est une injection et que la composition d'une bijection avec une bijection est une bijection.\\
\begin{lemma}
	$ $\\
	\begin{enumerate}
		\item $|A| \leq |B|$ et $|B| \leq |C|$ implique $|A| \leq |C|$.
		\item $|A| \leq |B|$ et $|B| = |C|$ implique $|A| \leq |C|$.
		\item $|A| = |B|$ et $|B| = |C|$ implique $|A| = |C|$.
	\end{enumerate}
\end{lemma}
\begin{proof}
	\begin{enumerate}
		\item Si $|A| \leq |B|$ alors il existe une injection  $\phi: A \to B$.\\
		Si $|B| \leq |C|$, alors il existe une injection $\psi:B \to C$.\\
		Donc 
		\[ 
			\psi \circ \phi : A \to C
		\]
		est une injection, et donc
		$|A| \leq |C|$.
	\item Si $|A| \leq |B|$ alors il existe une injection  $\phi: A \to B$.\\
	Si $|B| = |C|$, alors il existe une bijection $\psi:B \to C$.\\
	Donc
		\[ 
			\psi \circ \phi : A \to C
		\]
		est une injection, et donc $|A| \leq |C|$.
	\item Si $|A| = |B|$, alors il existe une bijection  $\phi: A \to B$.\\
	 Si $|B| = |C|$, alors il existe une bijection  $\psi: B \to C$.\\
	 Alors
	 \[ 
	 \psi \circ \phi : A \to C
	 \]
	 est une bijection et donc $|A| = |C|$
	 
	 
		
	\end{enumerate}
\end{proof}


\begin{theorem}
	Il existe une infinité de nombres premiers.
\end{theorem}

\begin{proof}
	Supposons par l'absurde qu'il existe un nombre fini de nombres premiers. Notons le nombre de nombres premiers $K$. On dénote par
	\[ 
	p_1,\ldots,p_K
	\]
	les $K$ nombres premiers.\\
	Alors le nombre
	\[ 
	N=\prod_{i=1} ^{K} p_i + 1
	\]
	est premier.\\
	En effet, par l'unicité de la décomposition en nombres premiers, $\exists p_k$ tel que $p_k | N+1$, or ceci implique que $p_k$ divise $1$ car $p_k | N$,\footnote{$p_k$ est un des facteurs de $N$ par définition} donc $p_k=1$ ce donc $p_k$ n'est pas premier. \\
	Il existe donc un $K+1$-ième nombre premier.
	
\end{proof}
\section*{1}
On construit une bijection de $\mathbb{Z}$ vers $\mathbb{N}$.\\
\begin{align*}
	\phi \colon \mathbb{Z} &\to \mathbb{N}\\
	m & \to 
	\begin{cases}
	2m \text{ si } m \geq 0\\
	-2m +1 \text{ si } m < 0
	\end{cases}
\end{align*}
On considère que le nombre 0 est pair.\\
Pour vérifier que cette application définit une injection, on montre la surjectivité et l'injectivité.\\
\subsection*{Surjectivité}
Soit $n \in \mathbb{N}$, si $n$ pair, $\exists k \in \mathbb{N} \text{ tel que } n = 2k$. Alors $k$ est l'antécédent de $n$ par $\phi$.\\
Si $n$ impair, $\exists j \in \mathbb{N}$ tel que $2j+1=n$, on pose $k=-j$, alors $-2k+1 =n$ et $k$ est l'antécédent de $n$.
\subsection*{Injectivité}
Supposons $\exists k,j \in \mathbb{Z}$ tel que $\phi(k)=\phi(j)$. Si $k$ et $j$ sont de signe différent, alors soit $\phi(k)$ ou $\phi(j)$ est impair et donc l'égalité ne peut pas tenir.\\
Supposons donc $k, j  > 0$, alors $\phi(k) = 2k $ et $\phi(j) = 2j$ donc $2k=2j$ et $j=k$.\\
Si $k, j <0 $, alors $\phi(k) = -2k + 1$ et $\phi(j) = -2j +1$ donc $-2k+1 = -2j + 1 \Rightarrow k =j$.	\\

On en déduit que l'application $\phi$ est bijective et que $|\mathbb{Z}| = |\mathbb{N}|$.
\section*{2}
Par Cantor-Schroeder-Bernstein, il suffit de trouver une injection de $ \mathbb{N}^{n} \to \mathbb{N}$
et de $\mathbb{N} \to \mathbb{N}^{n}$.\\
\subsection*{ Injection de $\mathbb{N} \to \mathbb{N}^{n}$}
Soit 
 \begin{align*}
	 \phi: \mathbb{N} &\to \mathbb{N}^{n}\\ 
	 k &\to (k, \underbrace{0, \ldots, 0}_{n-1 \text{ fois } })
\end{align*}
Cette application est clairement injective car $ ( m, 0, \ldots, 0) = (j, 0, \ldots, 0)$ implique $m=j$.
\subsection*{Injection de $ \mathbb{N}^{n} \to \mathbb{N}$}
Soit
\begin{align*}
	\psi: \mathbb{N}^{n} &\to \mathbb{N}\\
	( a_1, \ldots, a_n) & \to \prod_{i=1} ^{n} p_{i} ^{a_i}
\end{align*}
où $ p_1, \ldots, p_n$ sont les $n$ premiers nombres premiers. Ces nombres premiers existent car, par le théorème 1, il y a une infinité de nombres premiers.\\
L'injectivité de cette application suit directement de l'unicité de la décomposition en nombres premiers.\\
En effet, si $ (a_1, \ldots, a_n) \neq ( b_1, \ldots ,b_n) \in \mathbb{N}^{n}$, alors l'unicité implique que
\[ 
\prod_{i=1} ^{n} p_i^{a_i} \neq \prod_{i=1} ^{n}p_i^{b_i}
\]
et donc l'application $\phi$ est injective.\\
\hr
On conclut avec Cantor-Schroeder-Bernstein et on a que $ |\mathbb{N}^{n}| = | \mathbb{N}|$
\section*{3}
On utilise à nouveau Cantor-Schroeder-Bernstein.\\
\subsection*{Injection de $\mathbb{N} \to \mathbb{Q}$}
L'application
\begin{align*}
	K: \mathbb{N} &\to \mathbb{Q}\\
	 n &\to n
\end{align*}
est évidemment une injection.\\



\subsection*{Injection de $\mathbb{Q} \to \mathbb{N}$}
On montre un résultat préliminaire.\\
\begin{theorem}
Si $ A_1, \ldots, A_n$ $n$ ensembles infini dénombrables, alors
\[ 
K=A_1 \times \ldots \times A_n \text{ est infini dénombrable. } 
\]
\end{theorem}
\begin{proof}
	Soit $(a_1, \ldots, a_n) \in K $.\\
	Par hypothèse, $\exists \phi_1,\ldots, \phi_n$ des bijections $\phi_i: A_i \to \mathbb{N}, 0<i\leq n$.
	% Demander si on peut Supposer ca une bijection.
	L'application
	\begin{align*}
		\Phi: K &\to \mathbb{N}^{n}\\
		( a_1,\ldots, a_n) &\to ( \phi_1(a_1), \ldots, \phi_n(a_n))
	\end{align*}
	est une bijection.\\
	En effet, soit $b=(b_1,\ldots,b_n) \in \mathbb{N}^{n}$, alors $(\phi^{-1}(b_1), \ldots, \phi^{-1}(b_n)) \in K$ est un antécédent, unique, de $(b_1,\ldots, b_n)$ et il existe pour tout $ b \in \mathbb{N}^{n} $.\\
	Par la partie 2, on sait qu'il existe une bijection de $\Psi: \mathbb{N}^{n} \to \mathbb{N}$ et donc
	\[ 
	\Psi \circ \Phi
	\]
	est une bijection de $K \to \mathbb{N}$.
\end{proof}
\hr
On est pret à montrer l'injection de $\mathbb{Q} \to \mathbb{N}$.\\
On construit d'abord une bijection de $ \mathbb{Z} \times \mathbb{N} \setminus \left\{ 0 \right\} \to \mathbb{N} $.\\
Soit $\phi: \mathbb{Z} \to \mathbb{N}$ la bijection définie dans la partie 1 et soit $t_1: \mathbb{N} \setminus \left\{ 0 \right\} \to \mathbb{N}$ la bijection\footnote{L'injectivité et la surjectivité de $t_1$ sont évidentes.}:
\[ 
t_1: n \to n-1
\]
On peut donc, par le théorème 2, construire une bijection de $G:\mathbb{Z} \times \mathbb{N} \setminus \left\{ 0 \right\} \to \mathbb{N} $.

On définit la surjection\footnote{La surjectivité suit du fait qu'à chaque fraction, on puisse assimiler plusieurs 2-uplet.}
 \begin{align*}
	 Q: \mathbb{Z} \times \mathbb{N} \setminus \left\{ 0 \right\} &\to \mathbb{Q}\\
	 ( a,b) & \to \frac{a}{b}
\end{align*}
Par l'exercice 5, de la série 2, on peut construire une injection $F$
\begin{align*}
	F: \mathbb{Q} &\to \mathbb{Z} \times \mathbb{N} \setminus \left\{ 0 \right\} 
\end{align*}
Et donc, par le lemme 1, on obtient que
\begin{align*}
|\mathbb{Q}| \leq |\mathbb{Z} \times \mathbb{N} \setminus \left\{ 0 \right\}| = |\mathbb{N}|\\
\Rightarrow | \mathbb{Q}| \leq |\mathbb{N}|
\end{align*}
\hr
On conclut avec Cantor-Schroeder-Bernstein.
\section*{4}
\begin{theorem}
	On montre que l'union infinie dénombrable d'ensembles infinis dénombrables est, au plus, infini dénombrable.
\end{theorem}
\begin{proof}
Soit
\[ 
K = \bigcup_{i \in \mathbb{N}} E_i
\]
et soit $|E_i| \leq |\mathbb{N}|$\footnote{Si un des ensembles est fini, celà ne change rien au résultat} .\\
Sans perte de généralité, on peut supposer que tous les ensembles sont non-vides. Si un ensemble est vide, il ne contribue pas à l'union et il suffit donc de le supprimer et de réindexer les ensembles restant.\\
On considère d'abord que tous les ensembles sont disjoints, on montrera par la suite qu'une union d'ensembles non-disjoints peut se ramener à ce cas.\\
Car $|E_i| \leq |\mathbb{N}|$, il existe $\phi_i$ une injection de $E_i$ vers $\mathbb{N}$.\\
Considérons donc l'injection
\begin{align*}
	\phi : K &\to \mathbb{N}^{2}\\
	e &\to (\phi_i(e),i) 
\end{align*}
Le $i$ est bien déterminé car les ensembles $E_k$ sont disjoints.\\
L'injectivité de l'application $\phi$ est immédiate, en effet si $( m,n )\neq (m',n')$, alors on distingue deux cas.\\
Si $m\neq m', n = n'$ les éléments $\phi^{-1}_n(m) \neq \phi^{-1}_n(m')$ par injectivité de $\phi_n$.\\
Si $m=m', n\neq n'$, alors $\phi^{-1}(m,n) \neq \phi^{-1}(m',n')$ car $E_n \cap E_{n'} = \emptyset.$\\
Le cas où $m\neq m'$ et $n\neq n'$ est évident.\\
On a donc que
\[ 
|K| \leq |\mathbb{N}^{2}| \underbrace{=}_{ \text{ par la partie 2 } } |\mathbb{N}|
\]
et donc, par le lemme 1, que
 \[ 
|K| \leq |\mathbb{N}|
\]
Supposons maintenant que les ensembles ne sont pas disjoints, alors, par l'algorithme suivant, on peut se ramener au cas d'ensembles disjoints.
\begin{flushleft}
Pour $i$ dans $\mathbb{N}$ \\
\quad Pour $k$ entre 1 et $|E_k|$ \\
\quad \quad Si $s_{ki}$ apparait pour la deuxième fois\\
\quad \quad \quad Supprimer la valeur $s_{ki} $ de tous les $E_n, n\geq i \in \mathbb{N}$\\
\quad \quad Si $E_i = \emptyset$, alors supprimer $E_i$ et réindexer.
\end{flushleft}
Si on avait supprimé un nombre infini d'ensembles $E_i$ et qu'il en restait un nombre fini, alors on distingue deux cas:
\begin{itemize}
	\item On a une union finie d'ensembles finis, auquel cas le résultat est évident.
	\item On a une union finie d'ensembles, au plus, infini dénombrables, et le résultat est également évident car l'injection  $\Phi$ définie ci-dessus reste 
\end{itemize}

On a donc montré que:
\[ 
|\bigcup_{i \in \mathbb{N}} E_i| \leq |\mathbb{N}|
\]
\end{proof}
On veut montrer que l'ensemble des polynômes à coefficients dans $\mathbb{Q}$, noté $\mathbb{Q}[t]$ est infini dénombrable.\\
Considérons d'abord l'ensemble des polynômes de degré $n$ dans $\mathbb{Q}[t]$, noté $\mathbb{Q}^{n}[t]$.\\
On posera que 
\[ 
	a(t) \in \mathbb{Q}^{n}, a(t) = \sum_{i=0}^{ n} a_i t^{i}
\]

On construit la bijection
\begin{align*}
Q:	\mathbb{Q}^{n}[t] &\mapsto \mathbb{Q}^{n+1}\\
	a(t) &\mapsto (a_0,a_1,\ldots, a_n)
\end{align*}
La bijectivité de cette application est immédiate, car deux polynômes sont égaux si leurs coefficients sont égaux.\\
Or on sait, par la partie 3, que $|\mathbb{Q}| = |\mathbb{N}|$. Il en suit, par le théorème 2 que $|\mathbb{Q}^{n}| = |\mathbb{N}|$, on en conclut que
\[ 
	|\mathbb{Q}^{n}[t] | = |\mathbb{N}^{n+1}| = |\mathbb{N}|
\]
On démontre maintenant que
\[ 
	\bigcup_{i \in \mathbb{N}} \mathbb{Q}^{i}[t]	= \mathbb{Q}[t]	
\]
On montre la double inclusion.\\
L'inclusion de gauche à droite est immédiate.\\
Soit $a(t) \in \mathbb{Q}[t]$, posons que $\deg a(t) = k$, alors $a(t) \in \mathbb{Q}^{k}[t]$ et donc
$a(t) \in \bigcup_{i \in \mathbb{N}} \mathbb{Q}^{i}[t]$
On peut maintenant conclure grâce au théorème 3, en effet
\[ 
\bigcup_{i \in \mathbb{N}} \mathbb{Q}^{i}[t]
\]
est une réunion infinie dénombrable d'ensembles infinis dénombrables, donc
\[ 
	|\mathbb{Q}[t]|=|\bigcup_{i \in \mathbb{N}} \mathbb{Q}^{i}[t]| \leq |\mathbb{N}|
\]
On peut facilement construire une injection de $\mathbb{N}$ à $\mathbb{Q}[t]$, comme par exemple
\[ 
	n \mapsto n+1 \in \mathbb{Q}[t]
\]

Par Cantor-Schroeder-Bernstein, on conclut donc que
\[ 
	|\mathbb{Q}[t]| = |\mathbb{N}|
\]

L'ensemble des polynômes à coefficients dans $\mathbb{Q}$ est donc infini dénombrable.






\hr
On pose
\[ 
A = \left\{ z \in \mathbb{C} | z \text{ algébrique }  \right\}
\]
Soit $a(t) \in \mathbb{Q}[t]$, on dénote par $S_{a(t)}$, l'ensemble des solutions de l'équation $a(t)=0$. Cet ensemble est fini car un polynôme admet toujours un nombre de solutions égal à son degré.\\
On veut montrer que 
\[ 
	A = \bigcup_{a(t) \in \mathbb{Q}[t]} S_{a(t)} 
\]
On montre la double inclusion.\\
Soit $z \in A$, alors $\exists Z(t) \in \mathbb{Q}[t]$ tel que $Z(z)=0$, donc $z \in S_{Z(t)} $, donc 
$$z \in \bigcup_{a(t) \in \mathbb{Q}[t]} S_{a(t)}. $$\\
Soit
$$z \in \bigcup_{a(t) \in \mathbb{Q}[t]} S_{a(t)}  $$
donc $\exists b(t) \in \mathbb{Q}[t]$ tel que $b(z)=0$, donc $z$ algébrique, donc $z \in A$.\\
\hr
Or $ S_{a(t)} $ est fini $\forall a(t) \in \mathbb{Q}[t]$ et donc, par le théorème 3, on a que
\[ 
|A| =	|\bigcup_{a(t) \in \mathbb{Q}[t]} S_{a(t)} | \leq |\mathbb{N}|
\]
Or, les nombres entiers sont clairement des nombres algébriques, et donc il existe une injection
\[ 
\mathbb{N} \mapsto A
\]
Donc, par Cantor-Schroeder-Bernstein,
\[ 
|\mathbb{N}| = |A|
\]

Donc l'ensemble des nombres algébriques est dénombrable.






























\end{document}
