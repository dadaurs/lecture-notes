\documentclass[11pt, a4paper]{article}
\usepackage[utf8]{inputenc}
\usepackage[T1]{fontenc}
\usepackage[francais]{babel}
\usepackage{lmodern}

\usepackage{amsmath}
\usepackage{amssymb}
\usepackage{amsthm}
\newtheorem{lemma}{Lemme}
\newtheorem{thm}{Theoreme}
\newcommand\hr{
    \noindent\rule[0.5ex]{\linewidth}{0.5pt}\newline
}
\begin{document}
\title{Série 2}
\author{David Wiedemann}
\maketitle
On démontre d'abord la propriété énoncée dans l'exercie 2, partie 2:
\begin{lemma}
	$(a,b) \notin R$ si et seulement si $R_a \cap R_b = \emptyset$.
\end{lemma}
\begin{proof}
On montre l'implication dans les deux sens.\\
\framebox[1.1\width]{$  \Longrightarrow $}\\
Supposons, par l'absurde, $(a,b) \notin R$ mais $R_a \cap R_b$ non vide.
Supposons $ c \in R_a \cap R_b$, alors $c \in R_a$ et donc $(a,c) \in R$.
De même, $c \in R_b$ et donc $(c,b) \in R$.\\
Par la transitivité de la relation d'équivalence, on a donc:
\[ 
	(a,b) \in R
\]
ce qui est une contradiction à notre hypothèse.\\
\framebox[1.1\width]{$  \Longleftarrow $}\\
Supposons, par l'absurde, $(a,b) \in R$ mais $R_a \cap R_b = \emptyset$, alors
\[ 
	( a,b) \in R \Rightarrow b \in R_b \text{ et }  b \in R_a
\]
Ce qui est une contradiction à l'hypothèse que $R_a \cap R_b = \emptyset$.
\end{proof}

Ce lemme nous montre que deux classes d'équivalence différentes ne peuvent pas contenir d'éléments commun.\\
Finalement, on sait que les classes d'équivalence forment une partition de $A$, il n'y a pas d'éléments dans $A$ qui n'appartienne pas à une classe d'équivalence. En effet, supposons que $p \in A$, $(p,p) \in R$ et donc $p \in R_p$, donc tout élément est dans une classe d'équivalence.\\
Grâce à ceci, on peut déduire que le nombre d'éléments de $R$ vaut la somme du nombre d'éléments des classes d'équivalence.\\
\hr
On voit que 6 peut s'écrire de trois manières comme somme de trois nombres non-nuls:
\begin{align*}
	6&= 2+2+2\\
	6&= 4+1+1\\
	6&= 3+2+1
\end{align*}

\hr
On montre d'abord que 
\[ 
	| R_i \times R_i | = | R_i| ^{2}
\]
\begin{proof}
On pose que $|R_i|=n$.\\
Par définition $R_i \times R_i = \left\{ ( a,b) | a,b \in R_i \right\} $.\\
La cardinalité de $R_i \times R_i$ est donc simplement la répartition de $n$ éléments sur 2 places, donc $n^{2}= | R_i|^{2}$
\end{proof}
Car les classes d'équivalences forment une partition de $A$, on a que:
\[ 
|R| = |R_1\times R_1| +|R_2\times R_2|+|R_3\times R_3|
\]
Ou $ R_1,R_2,R_3$ sont les trois classes d'équivalence sur $A$, engendrées par $R$.
\begin{proof}
Supposons 
\[ 
|R| < |R_1\times R_1| +|R_2\times R_2|+|R_3\times R_3|
\]
Sans perte de généralité, supposons que $R_1 \times R_1$ contient un élément de plus (il suffit de réindexer les ensembles si nécessaire).\\
Donc, $\exists ( a,b) \in R_1 \times R_1 \Rightarrow  a, b \in R_1$, or ceci implique, par définition que $(a,b) \in  R$.\\
Supposons donc
\[ 
|R| > |R_1\times R_1| +|R_2\times R_2|+|R_3\times R_3|
\]
Donc, $\exists ( a,b) \in R$, tel que $(a,b) \notin R_1, R_2, R_3$.\\
Si $a=b$, alors l'élément $a$ définit une nouvelle classe d'équivalence ce qui est une contradiction à l'hypothèse qu'il y ait 3 classes d'équivalences.\\
Si $a\neq b$, alors par hypothèse, $a \in R_i$ et $b \in R_j ( j \neq i)$, or par le lemme 1, ceci implique que $R_i = R_j$, ce qui contredit l'hypothèse qu'il y ait 3 classes d'équivalence.\\
On en déduit que
\[ 
|R| = |R_1\times R_1| +|R_2\times R_2|+|R_3\times R_3|
\]
\end{proof}
Il suffit maintenant de calculer les 3 cas énumérés plus haut.\\
\hr
Si une classe d'équivalence a 4 éléments,
les deux autres classes d'équivalence ont chacune 1 élément.\\
En tout $R$, contiendra donc $4\times 4 + 1\times 1 + 1 \times 1=18$ éléments.\\
\hr
Si une classe d'équivalence possède 1 élément, une autre 2 éléments et la dernière 3 éléments, $R$ contiendra
$(3\times 3) + (2\times 2) + ( 1 \times 1 ) =14$ éléments.\\
\hr
Si chacune des trois classes d'équivalence possède 2 éléments $R$ contiendra
$ (2 \times 2) + (2 \times 2) +(2 \times 2) = 12$ éléments.\\
\hr
On peut construire des relations d'équivalence qui satisfont la répartition des éléments tel que ci-dessus.\\
Soit $A = \left\{ x_1,x_2,x_3,x_4,x_5,x_6 \right\} $\\
Soit $ A_1, A_2, A_3$	une partition de l'ensemble $A$, avec les trois ensembles non-vides.\\
On remarque qu'on peut choisir les $ A_1, A_2, A_3$ en utilisant la répartition telle que définie ci-dessus ( ie. 2-2-2, 4-1-1 et 3-2-1).\\
On pose que $( x_i,x_j ) \in R$ si $x_i$ et $x_j$ sont dans le même $A_k$.\\
Cette relation satisfait les propriétés d'une relation d'équivalence, en effet:
\begin{enumerate}
	\item Identité:\\
		Si $x_i \in A_k$, alors, clairement $(x_i, x_i) \in R$
	\item Reflexivite:\\
		Si $( x_i,x_j) \in R $, alors $x_i \in A_k$ et $x_j \in A_k$, donc $(x_j, x_i) \in R$.
	\item Transitivité:\\
		Si $(x_i, x_j) \in R$ et $(x_j,x_g) \in R$, alors $x_i,x_j, x_g \in A_k$, donc $(x_i,x_g) \in R$
\end{enumerate}
\hr
On a donc montré qu'il y a 3 valeurs possibles pour la cardinalité de $R: 18,14$ ou $12$, et qu'il existe en effet des relations d'équivalence qui possèdent ce nombre d'éléments.




\end{document}
