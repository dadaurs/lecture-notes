\documentclass[../main.tex]{subfiles}
\begin{document}
\lecture{10}{Tue 17 Nov}{mardi}
\begin{exemple}
\[ 
\mathbb{Z} \to \mathbb{Z} / n \mathbb{Z}
\]
\[ 
n \mapsto [ n] 
\]

 $\left\{\text{  homoms } \mathbb{Z} / n \mathbb{Z} \to G \right\} \leftrightarrow \left\{ \text{ hohoms } \mathbb{Z} \to G | n \mathbb{Z} \leq \ker \phi \right\} $
\end{exemple}
\subsection{Sous-groupes engendres par plusieurs elements}
\begin{lemma}
$H_i \leq G$ $\forall i \in I$, alors $\bigcap_i H_i \leq G$
\end{lemma}
\begin{proof}
\begin{enumerate}
\item $\bigcap_i \neq \emptyset$ parce que $e \in H_i$ 
\item $g,h \in \bigcap_i H_i$ $\Rightarrow$ $\forall i g.h, g^{-1} \in H_i$.
\end{enumerate}
\end{proof}
\begin{defn}
$S \subseteq G$ des sous-ensembles, alors
\[ 
	\eng{S} = \text{ plus petit sous-groupe contenant $S$ } 
\]
Il existe parce que
\[ 
	\eng{S} = \bigcap_{S \subseteq H \leq G} H
\]

\end{defn}




\end{document}	
