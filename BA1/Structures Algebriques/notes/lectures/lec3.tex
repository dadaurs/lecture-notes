\documentclass[../main.tex]{subfiles}
\begin{document}
\lecture{3}{Tue 29 Sep}{mardi}
\begin{proof}


C'est suffisant de montrer que
\[ 
	H(W) = W
\]
On montre la double inclusion
$\subset:$ \\
$W \subseteq \bigcap_{x \subseteq A, H(X) \subseteq X} X$, alors
\begin{align*}
	H(W) &\subseteq \bigcap_{x\subseteq A, H(X) \subseteq X} H(X)\\
	     &\subseteq \bigcap_{x \subseteq A, H(X) \subseteq X} X =W 
\end{align*}
$\supseteq:$ \\
$H(W)$ est un $X$ comme dans la definition de $W$.
\[ 
	\Rightarrow W \subseteq H(W)
\]
\end{proof}
Question:\\
$\abs{ \mathbb{R} } = \omega_1$?\\
Hypothese du continu\\
On peut montrer qu'on ne peut pas demontrer ca.
\section{Theorie des nombres}
\subsection{Algorithme d'Euclide}

\begin{defn}
	$a \in \mathbb{Z}, b \in \mathbb{Z}, b \neq 0$, alors
	\[ 
		\underbrace{( a,b)}_{ \text{ plus grand commun diviseur } }= \left\{ c \in \mathbb{Z}^{>0} \vert \text{    }  c \vert a, c \vert b \right\} 
	\]
	Cette valeur existe car il y a une borne superieure donnee par $\abs{b}$.\\
	
\end{defn}
\begin{lemma}
$ a_1, b \in \mathbb{Z}, a \neq 0, r \in \mathbb{Z}$
\[ 
	( a,b) = ( a,b+ra)
\]

\end{lemma}
\begin{proof}
Si qqchose divise a et b, il divse aussi a. Il divise aussi b + a
\[ 
	( b+ra) -ra = b
\]
Detail dans les notes moodle
\end{proof}
\begin{defn}[Algorithme d'Euclide]\index{Algorithme d'Euclide}\label{defn:algorithme_d_euclide}
	$a,b \in \mathbb{Z}^{0}$, soit
\begin{align*}
a_1:= \max \left\{ a,b \right\} \\
a_2 := \min \left\{ a,b \right\} 
i:=2
\end{align*}
\textbf{ Pas recursif: }\\
Si $qi \vert q_{i-1} \rightarrow$ on arrete et on pose $t:=i$.\\
Sinon $q_{i-1} = s_i q_i + q_{i+1} $
\begin{align*}
q_i \not\vert q_{i-1} \Rightarrow q_{i+1 \neq 0}\\
\text{ et } q_{i+1} < q_i
\end{align*}
\[ 
q_1>q_2>q_3> \ldots q_t >0, \text{ avec } q_i \text{ entier }
\]
\end{defn}
\begin{lemma}
$\exists m,n \in \mathbb{Z}$ tel que
\[ 
am + bn = q_t
\]

\end{lemma}
\begin{proof}
On demontre que $q_i$ 
\[ 
m_i q_i + n_i q_{i+1} = q_t
\]
On utilise l'induction descendante sur $i$.
$\exists m_i, n_i \in \mathbb{Z}$ \\
$i=t-1$
\[ 
1 q_t + 0 q_{t-1} = q_t
\]
Pas d'induction\\
\[ 
q_i = s_o q_{i+1}  + q_{i+2} 
\]
Par hypothese d'induction
\begin{align*}
\underbrace{m_{i+1} q_{i+1 } + n_{i+1} q_{i+2}}_{= m_{i+1} q_{i+1} + n_{i+1} ( q_i - s_{i+1} q_{i+1} )} = q_t\\
= \underbrace{n_{i+1}}_{m_i} q_i + \underbrace{( m_{i+1} - n_{i+1} s_{i+1})}_{n_i} q_{i+1} \\
\end{align*}
\end{proof}
\begin{lemma}
$q_t | q_i $ pour chaque $i$.
\end{lemma}
\begin{proof}
On demontre de la meme facon que le lemme d'avant avec induction descendante.
\end{proof}
On peut combiner les deux lemmes:
donc 
\begin{align*}
	(a,b) | q_t\\
	q_t | ( a,b)
\end{align*}
Donc l'algorithme d'Euclide donne le pgcd.
\subsection{Theoreme fondamental de l'arithmetique}
\begin{defn}[Entier]\index{Entier}\label{defn:entier}
	Soit $p \geq 2$ un entier
	\begin{enumerate}
	\item $p$ irreductible si pour chaque $a | p  \Rightarrow a=1$ ou $a=p$,  $a \in \mathbb{N}$
	\item $p$ premier: $\forall a,b \in \mathbb{Z}^{>0}$ 
		\[ 
		p | a.b \Rightarrow p | a \text{ ou } p | b
		\]
	\end{enumerate}
\end{defn}
\begin{lemma}
$q, a,b \in \mathbb{Z}^{>0}$ 
\begin{align*}
	q | a.b &\text{ et } ( q,a) =1\\
		& \Rightarrow q | b
\end{align*}
\end{lemma}
\begin{proof}
	$(q,a) = 1$ 
	\begin{align*}
	1 = mq + na \text{ ,avec } m,n \in \mathbb{Z}\\
	b= mqb + nab\\
	\Rightarrow q | b
	\end{align*}
\end{proof}
\begin{propo}
Soit $p\geq 2$ entier\\
p irreductible $\iff p$ premier
\end{propo}
\begin{proof}
$\Leftarrow$ \\
On veut montrer que $a.b = p, \Rightarrow a=1 \text{ ou } b=1$
On sait que $p$ premier
\[ 
p | a.b \Rightarrow  p | a \text{ ou } p | b
\]
\[ 
\underbrace{\Rightarrow}_{a,b \geq p} p | a \text{ ou }  p |b
\]
$\Rightarrow$ :\\
$p$ irreductible
\[ 
p | ab
\]
Deux possibilites:\\
\begin{enumerate}
	\item $p|a$ on a fini 
	\item $p \not | a \Rightarrow ( p,a) \neq p$\\
		$p$ irreductible ( $(p,a) | p$), donc
		\[ 
			( p,a) =1
		\]
		Donc $\Rightarrow p | b$ 
\end{enumerate}
\end{proof}
\begin{thm}
$n \in \mathbb{Z}^{>0},$ \\
\[ 
n = \prod_{i=1} ^{r} p_i, \text{ avec } p \text{ premiers } 
\]
et c'est unique modulo l'ordre des premiers
\end{thm}
\begin{proof}
Existence\\
Induction sur $n$ \\
$n=2$ premier donc verifie.\\
Pas d'induction:
2 possibilites:
\begin{itemize}
\item $n$ premier $\Rightarrow p_1 = n, r=1$ 
\item $n$ n'est pas premier
	$\Rightarrow$ pas irreductible\\
	$\Rightarrow a.b =n$ \\
	tel que $a,b < n$\\
	$\Rightarrow a= \prod p_i$   et $b = \prod p_i$, donne la decomposition pour  $n$.
\end{itemize}
Unicite\\
 \[ 
n = \prod_{i=1} ^{r} p_i = \prod _{j=1} ^{s} q_j, \text{ avec  } r \leq s
\]
\begin{itemize}
\item $s=1 \Rightarrow  r=1$ verifie
\item $s>1$, alors
	\[ 
	q_1 | \prod_{i=1} ^{r} p_i
	\]
	$\Rightarrow q_1$ premier\\
	\[ 
	\Rightarrow  \forall l: q_1 | p_l
	\]
	donc
	\begin{align*}
	\frac{n}{q_1}= \prod_{i=1, i \neq l}^{r} p_i = \prod_{j=2} ^{s} a_j	
	\end{align*}
\end{itemize}




\end{proof}













\end{document}	
