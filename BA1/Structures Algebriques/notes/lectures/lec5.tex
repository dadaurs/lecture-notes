\documentclass[../main.tex]{subfiles}
\begin{document}
\lecture{5}{Tue 13 Oct}{mardi}
\subsection{Construction de Groupes avec des quotients}
\begin{exemple}
\[ 
	G = ( \mathbb{Z},+)
\]
On dénote
\[ 
\faktor G R = \left\{ R_x | x\in G \right\} = \left\{ [ x] | x \in G \right\} 
\]

Dans ce cas
\[ 
\faktor {\mathbb{Z} } { m\mathbb{Z}} = \left\{ m \mathbb{Z}, m\mathbb{Z}+1, \ldots , m\mathbb{Z}+ m-1 \right\} 
\]
\end{exemple}
\subsubsection{Recette générale}
Construction de la structure de groupes sur $ \faktor G R$ 
\begin{itemize}
\item Représentant\\
	\[ 
	x \in \faktor G R
	\]
	$g$ est un représentant de $x$ si $x= [ g]$.
\item $ [ g] \cdot [ f] = [ g \cdot f]$
\item $[g]^{-1} = [ g^{-1}]$
\item $e_{\faktor G R} = [ e]$
\end{itemize}
Il faut que ce soit bien défini, donc si
\[ 
	( g,g') \in R, ( f,f') \in R
\]
Alors
\[ 
	( g.f, g'.f') \in R
\]
De même, si $ ( g, g') \in R$ 
\[ 
	\Rightarrow ( g^{-1}, g'^{-1}) \in R
\]

\begin{exemple}
	$ ( \faktor { \mathbb{Z}} { m \mathbb{Z}}, + )$\\
	Il faut vérifier la condition la condition.\\
	\begin{itemize}
		\item  $( g,g' ) \in R, ( f,f') \in R \implies ( g.f, g'.f') \in R$, alors
			\[ 
				m | g-g' \text{ et } m | f-f' \text{ et }  m | g+f - ( g' + f')
			\]

		\item $(g,g') \in R$, alors $ ( g^{-1},g'^{-1}) \in R$, en effet
			\[ 
			m| g-g' \text{ et } m | -g - -g'
			\]
		
	\end{itemize}
	Donc on a vérifié que c'est un  groupe.
	
\end{exemple}
\begin{exemple}
$ \faktor { \mathbb{Z}} { m \mathbb{Z}}^{\times}$\\
pas stable avec la multiplication.
\end{exemple}
Par contre
$ ( \faktor { \mathbb{Z} } { m \mathbb{Z}}, \bullet)$ monoide avec $ [ 1]$ l'élément neutre.
Mais $ [ 0]^{-1}$ n'existe pas
Donc il faut jeter les classes qui n'ont pas d'inverses,i.e. tous les éléments sauf $[p]$, $p$ premiers.\\
Donc  
\[ 
\left\{ [ g] \in \faktor { \mathbb{Z}} { m \mathbb{Z}} | ( g,m)=1 \right\} = ( \faktor { \mathbb{Z}} { m \mathbb{Z}} )^{\times} \subseteq \faktor { \mathbb{Z}} { m\mathbb{Z}}
\]
Pour le moment, il s'agit que d' un sous-ensemble\\
On veut voir que la structure de monoide induit une structure de groupe sur $\faktor { \mathbb{Z}}{ m \mathbb{Z}}$.\\
$ [ g], [ f] \in (  \faktor { \mathbb{Z}}{ m \mathbb{Z}})^{\times} \Rightarrow [ g.f] \in(  \faktor { \mathbb{Z}}{ m \mathbb{Z}})^{\times}$\\
Autrement dit $ ( g,m)=1 ( f,m)=1 \Rightarrow ( g.f, m)=1$\\
Clairement, $ \left\{ 1 \right\} \in(  \faktor { \mathbb{Z}}{ m \mathbb{Z}})^{\times}$\\
De plus, soit $[g] \in (  \faktor { \mathbb{Z}}{ m \mathbb{Z}})^{\times}$, on veut montrer que
\[ 
\Rightarrow [ g^{-1}] \in (  \faktor { \mathbb{Z}}{ m \mathbb{Z}})^{\times}
\]
autrement dit,
\[ 
	( g,m)=1 \implies \exists f \in \mathbb{Z}, \exists x \in \mathbb{Z} \text{ tel que } g.f =1 + mx
\]
Ce qui est immédiat, par Bézout.\\
Donc $(  \faktor { \mathbb{Z}}{ m \mathbb{Z}})^{\times}$ est un groupe!
\begin{defn}[Homomorphismes de groupes]\index{Homomorphismes de groupes}\label{defn:homomorphismes_de_groupes}
	Soient $G,H$ deux groupes.\\
	Une application
	\[ 
	\phi: G \to H
	\]
	est un homomorphisme si 
	\[ 
		\forall g,f \in G: \phi(g.f) = \phi(g) . \phi(f)
	\]
	$\phi$ est un endomorphisme si $\phi$ est un homomorphisme 
	\[ 
	\phi: G \to G
	\]
	$\phi$ est un isomorphisme si 
	\[ 
	\phi: G \to H
	\]
	est un homomorphisme bijectif.\\
$G$ et $H$ sont isomorphes si il existe
\[ 
	\phi: G \to H
\]
un isomorphisme. On note
\[ 
G \simeq H
\]

	
\end{defn}
\begin{lemma}
\[ 
\phi: G \to H
\]
un homomorphisme, alors
\[ 
	\phi ( g^{n}) = \phi(g)^{n}
\]
\end{lemma}
\begin{proof}
pour n=0:\\
\begin{align*}
\text{ à montrer: } \phi(e_G) = e_H\\
e_H \cdot \phi(g) = \phi(g) = \phi(e_G.g) = \phi(e_G) \phi(g)
\end{align*}
Donc $e_H = \phi(e_G)$.\\
Pour $n>0$ :
\[ 
	\phi(g^{n}) = \phi(g. \ldots. g) = \phi(g). \ldots . \phi(g) = \phi(g)^{n}
\]
Pour $n<0$:\\
On a démontré la semaine passée
\begin{align*}
\phi(g)^{n} \cdot \phi(g)^{-n} = phi(g)^{0} = e_H\\
\text{ Il suffit de montrer que } \\
\phi(g^{n)} \text{ est aussi un inverse de } \phi(g)^{-n}\\
\phi(g^{n}) \phi(g^{-n}) = \phi(g^{n} g^{-n}) = \phi(e_G) = e_H
\end{align*}
\end{proof}
\begin{exemple}
\begin{itemize}
	\item  $ ( G,+)$ abélien, $n \in \mathbb{N}$ 
		\[ 
		G \ni x \mapsto n.x
		\]
	C'est un homomorphisme car
	\[ 
		n ( x+y) = nx + ny
	\]

\item 
	\[ 
	\phi: \mathbb{Z} \mapsto G
	\]
	quelconque, alors
	\[ 
		\phi(n\cdot 1) = \phi(1)^{n} \forall n \in \mathbb{Z}
	\]
	
	
		
\end{itemize}
\end{exemple}
Autre direction:\\
Est-ce qu'il existe
\[ 
\phi: \mathbb{Z} \to G
\]
tel que 
\[ 
	\phi(1) = g
\]
Il y a une seule possibilité que ce soit le cas, quand
\[ 
	\phi(n) = g^{n}
\]
C'est un homomorphisme:
\[ 
g^{n} g^{m} = g^{n+m}
\]
Cet homomorphisme existe donc, et il est uniquemenent determinem on l'appelle
\[ 
dexp_g
\]
pour ``exponentielle discrete'' 






	






\end{document}	
