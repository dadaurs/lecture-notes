\documentclass[../main.tex]{subfiles}
\begin{document}
\lecture{4}{Tue 06 Oct}{mardi moitie}

\section{Théorie des Groupes}
\subsection{Groupe symmétrique de n}
Le groupe $Bij(X)$ pour $X= \left\{ 1,\ldots,n \right\} \to S_n$
\[ 
\sigma \in S_n \to
\begin{pmatrix}
	1 & 2 &\ldots n\\
	\sigma(1) & \sigma(2) &\ldots & \sigma(n)
\end{pmatrix}
\]
La multiplication ( loi de composition) est simplement la composition des applications, attention le groupe n'est pas abélien.
\[ 
\begin{pmatrix}
	1 &2 &3\\
	2 &1 &3
\end{pmatrix}
\cdot
\begin{pmatrix}
	1 & 2 &3\\
	2 &3 &1
\end{pmatrix}
= 
\begin{pmatrix}
	1 &2 &3\\
	1 & 3 &2
\end{pmatrix}
\]
Dans l'autre sens:
\[ 
\begin{pmatrix}
	1 & 2 &3\\
	2 &3 &1
\end{pmatrix}
\cdot
\begin{pmatrix}
	1 &2 &3\\
	2 &1 &3
\end{pmatrix}
= 
\begin{pmatrix}
	1 &2 &3\\
	3 &2 &1
\end{pmatrix}
\]
Les autres exemples seront contruit par une relation d'équivalence, on note
\[ 
G / R
\]
Question:\\
Quand est-ce que $G /R$ est-il un groupe?\\
Construction:\\
\[ 
	[ g]= R_g= \left\{ h \in G | ( g,l)\in R \right\} 
\]
la classe de $G$.\\
Multiplication sur $\frac{G}{R}$ \\
Soit $x,y \in G/R$, alors
\[ 
	x = [ g] ,y \in [ f]
\]
On définit
\[ 
x\cdot y := [ g\cdot f]
\]
Problème on peut choisir différents représentatifs.\\
Donc Pour que la définition sooit sensée, iil faut que
\[ 
	[ g\cdot f]= [ g'\cdot f'] ( \forall ( g,g')\in R ( f,f')\in R)
\]
Pour l'inverse\\
\begin{align*}
x\in G /R\\
x =[g]\\
x^{-1} = [ g^{-1}]
\end{align*}
Elément neutre de $G/ R$:\\
$[e] \in G /R$
\begin{propo}
La définition précédente nous donne une structure de groupe sur $G /R$.\\

Les opérations sont bien définies.
\[ 
	( g,g')\in R ( h,h')\in R \Rightarrow ( g.h, g'.h')\in R
\]
\[ 
	( g,g')\in R \Rightarrow ( g^{-1}, ( g')^{-1})\in R
\]

\end{propo}

\begin{proof}
Il faut vérifier les 3 conditions de groupe.\\
\begin{itemize}
	\item ( associativité)
		\[ 
			x\cdot ( y\cdot z)= [ g] \cdot [ f\cdot h]= [ g \cdot f] \cdot [ h] = ( x\cdot y)\cdot z
		\]
		Les deux autres propriétés sont laissées en exercice.
\end{itemize}

\end{proof}
\begin{exemple}[$G = \mathbb{Z}, +$]
Soit
\[ 
	R = \left\{ ( x,y) \in \mathbb{Z} \vert m | x-y \right\} 
\]
\hr\\
\[ 
	G /R = \left\{ m\cdot \mathbb{Z}, m \mathbb{Z}+1, \ldots, m\mathbb{Z}+ ( m-1) \right\} 
\]
Les éléments sont de $G /R$ sont des éléments et des groupes.\\
Il faut vérifier que $+$ et $-$ sont bien définis par rapport à $R$ et ainsi on obtien le groupe
\[ 
	( \mathbb{Z} / m \mathbb{Z} , +) 
\]


\end{exemple}




\end{document}	
