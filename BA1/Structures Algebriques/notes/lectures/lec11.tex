\documentclass[../main.tex]{subfiles}
\begin{document}
\lecture{11}{Tue 24 Nov}{Groupe des Quaternions}
\begin{crly}
Soit $\phi:G \to H$ homomorphisme et $S \subseteq G$, alors
\[ 
	\phi( \eng{S})  = \eng{\phi( S) }
\]

\end{crly}
\begin{proof}
	\begin{align*}
		\eng{\phi( S) } = \left\{ y_1\ldots y_r | y_i \in \phi( S) \text{ ou } y ^{-1}_i \in \phi( S)  \right\} \\
		= \eng{\phi( z_1) \ldots \phi( z_r) \in H | z_i \in S  \text{ ou } z_i^{-1}\in S}\\
		= \eng{\phi( z_1 \ldots z_r)  \in H | z_i \in S  \text{ ou } z_i^{-1}\in S}\\
	\end{align*}
\end{proof}
\begin{center}
\textbf{Prochain But}
\end{center}
Description de $\eng{H,F}$ pour $H,F \leq G$ quand c'est plus facile
\begin{defn}
$H\leq G$, le normalisateur:
\[ 
	N_G( H) = \left\{ g \in G | gHg^{-1} = H \right\} 
\]

\end{defn}
\begin{rmq}
Si $|G| < \infty $ alors 
\[ 
	N_G ( H) = G \iff H \trianglelefteq G
\]

\end{rmq}
\begin{propo}
	$N_G( H) \leq G$.
\end{propo}
\begin{proof}
	$N_G( H) \neq \emptyset$, car l'element neutre est dans le normalisateur.\\
	$g,f \in N_G( H) $ $\Rightarrow$ $gf \in N_G( H) $, car 
	\[ 
		gf H ( gf) ^{-1}= gf H f^{-1}g^{-1}= gHg^{-1}= H
	\]
	$g \in N_G( H) \Rightarrow g^{-1}\in N_G( H) $, donc
	\[ 
	g^{-1}H ( g^{-1} )^{-1}= g^{-1}g H g^{-1}g = H
	\]
	
\end{proof}
\begin{propo}
	$H,F \leq G$, $F\subseteq N_G( H) $ 
	\[ 
		\eng{H,F} = HF =FH
	\]
	ou
	\[ 
	HF = \left\{ hf \in G | h \in H, f \in F \right\} 
	\]
	\[ 
	FH = \left\{ fh \in G | h \in H, f \in F \right\} 
	\]
	
	
\end{propo}
\begin{proof}
\[ 
	HF \subseteq \eng{H,F}
\]
Pour l'autre direction, il suffit de montrer que $HF$ est un sous-groupe.$HF\neq \emptyset$, car l'element neutre est dedans.\\
$hf, h'f' \in HF$, alors
\[ 
hfh'f' = hfh'f^{-1}ff' \in HF
\]


\end{proof}
\subsection{Groupes Lineaires}
\begin{defn}
	$GL( n,K) $ sont les matrices inversibles de la taille $n\times n$ sur un corps $K$.
\end{defn}



























































\end{document}	
