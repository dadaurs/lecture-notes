\documentclass[../main.tex]{subfiles}
\begin{document}
\lecture{6}{Tue 20 Oct}{Th. des Groupes}
\subsection{Produits de Groupes}

\subsection{Produits de Groupes}

\begin{defn}
Soit $G,H$ deux groupes 
\[ 
	G\times H = \left\{ ( g,h) | g\in G, h \in H \right\} 
\]
si $|G|, |H| < \infty $, alors $|G\times H| = |G| \cdot |H|$, on munit $G\times H$ d'une structure de groupe avec la loi
\[ 
	( g,h) \cdot ( g',h') = ( g',h') 
\]

\end{defn}
\begin{lemma}
C'est un groupe avec
\begin{itemize}
	\item $e_{G\times H} = ( e_G, e_H) $ 
	\item $( g,h) ^{-1}= ( g^{-1},h^{-1}) $
\end{itemize}

\end{lemma}
\begin{proof}
En exo
\end{proof}

\subsection{Propriété universelle des Produits}
Si on a $G\times H$, on a deux projections ( homomorphismes)  naturels
\begin{align*}
F\times H \underbrace{\to}_{pr_H} H\\
pr_F( ( f,h) )  = f\\
pr_H( ( f,h) )  = h\\
\end{align*}
Ce sont trivialement des homomorphismes.

\begin{propo}
Soit $G,F,H$ des groupes
\begin{align*}
F\times H \underbrace{\to}_{pr_H} H  \\
\text{ et } \\
F\times H \underbrace{\to}_{pr_F} F\\
\text{ de plus soit } 
\beta: G \to H\\
\alpha: G \to F
\end{align*}
Il existe un homomorphisme unique
\[ 
\gamma : G \to F\times H 
\]
tel que les compositions ci-dessus commutent.
\begin{figure}[H]
    \centering
    \incfig{diagrammes-produits}
    \caption{diagrammes produits}
    \label{fig:diagrammes-produits}
\end{figure}
Donc que
\[ 
\alpha = pr_F \circ \gamma \text{ et } \beta = pr_H \circ \gamma
\]
Donc
\[ 
	\alpha ( g) = pr_F ( \gamma ( g) ) = pr_F( f,h) =f
\]
De plus 
\[ 
	\beta ( g)  = pr_H ( \gamma( g) ) = pr_H ( f,h)  =h
\]
Donc
\[ 
	\gamma ( g) = ( \alpha( g) , \beta( g) ) 
\]
\end{propo}
\begin{proof}
Il faut montrer que $\gamma$ est un homomorphisme.\\
\begin{align*}
	\gamma( g) \gamma( g')  &= ( \alpha( g) ,\beta( g) ) \cdot ( \alpha( g') , \beta( g') ) \\
		     &= ( \alpha( g) \alpha( g') , \beta( g) \beta( g')  )\\
		     &= ( \alpha( gg') ,\beta( gg') ) = \gamma( gg')
\end{align*}
On a utilisé que la composition d'homomorphismes est un homomorphisme.
\end{proof}
\begin{center}
\textbf{Utilisation}
\end{center}
Regardons les homomorphismes de
\[ 
	\mathbb{Z}\to \faktor{\mathbb{Z}}{2\mathbb{Z}} \times \faktor { \mathbb{Z} } {3\mathbb{Z}} 
\]
est la meme chose que considérer les morphismes de 
\begin{align*}
	\mathbb{Z} \to \faktor{\mathbb{Z}}{2\mathbb{Z}}\\
	\text{ et } \\
	\mathbb{Z}\to \faktor{\mathbb{Z}}{3\mathbb{Z}}
\end{align*}
On peut par exemple prendre l'exponentielle discrete de $\mathbb{Z}\to \faktor{\mathbb{Z}}{2\mathbb{Z}}$ et 
$\mathbb{Z}\to \faktor{\mathbb{Z}}{2\mathbb{Z}}$
et les combiner.\\
On ne doit pas vérifier que la ``composition''  est un morphisme car on l'a montré dans la propriété universelle.
\subsection{Sous-groupes}

\begin{defn}[Sous-Groupe]
	Soit $ ( G,\cdot) $ un groupe et $H\subseteq G$ un sous-ensemble.\\
	$H$ est un sous-groupe si
	\begin{enumerate}
	\item $ \left\{ h\cdot h' | h,h' \in H \right\} = H\cdot H \subseteq H $
	\item $\cdot |_H : H\times H \to H$
		nous donne un groupe sur $H$.
	\end{enumerate}
	
\end{defn}
\begin{propo}
	Soit $H\subseteq ( G,\cdot) $ un sous-ensemble.\\
	C'est un sous-groupe si et seulement si
	\begin{enumerate}
	\item $H\neq \emptyset$ 
	\item $h,g \in H \implies h.g \in H$ 
	\item $h \in H \implies h^{-1}\in H $
	\end{enumerate}
	
	De plus, les éléments neutres de $G$ et de $H$ sont les mêmes, inverses aussi
	
\end{propo}
\begin{proof}
$\Rightarrow$ \\
\begin{enumerate}
\item $e_H \in H \Rightarrow H \neq \emptyset$
\item Vrai
\item Il faut démontrer que
	\[ 
	e_H = e_G
	\]
En effet
\[ 
e_H \cdot e_H = e_H = e_G \cdot e_H
\]
Or on peut simplifier, donc
\[ 
e_H = e_G
\]
$\Leftarrow$\\
Il faut démontrer que $e_G \in H$. En effet
\[ 
	H \neq \emptyset \Rightarrow  \exists g \in H \Rightarrow g^{-1}g = e_G \in H
\]

\end{enumerate}

\end{proof}
\begin{exemple}
\begin{enumerate}
\item Sous-groupes triviaux\\
	\begin{align*}
	\left\{ e   \right\} \subseteq G\\
	G \subseteq G
	\end{align*}

\item $ \left\{ 1,-1 \right\} \subseteq \left( \mathbb{Q}\setminus \left\{ 0 \right\} ,\cdot \right) $

\item $m \mathbb{Z} = \left\{ mx | x \in \mathbb{Z} \right\} \subseteq \left( \mathbb{Z}, + \right)  $ 
\item $\faktor{  \mathbb{Z}  }{ n\mathbb{Z} }, m |n$
	être divisible par $m$ est bien définit sur les classes d'équivalences, autrement dit
	\[ 
	x,y \in \mathbb{Z}, n | x-y \text{ alors } m|x \iff m|y
	\]
	Donc
	\[ 
		\left\{ [ x] \in \faktor{\mathbb{Z}}{n\mathbb{Z}} | m | [ x]  \right\} \subseteq \faktor{\mathbb{Z}}{n\mathbb{Z}}	
	\]
	est un sous-groupe.
	
\end{enumerate}
\end{exemple}
\begin{defn}
Soit 
\[ 
\phi: G \to H
\]
un morphisme.
\begin{enumerate}
\item noyau:
	\[ 
	\ker \phi = \left\{ g \in G | \phi g = e_H \right\}  \subset G
	\]
	
\item image:
	\[ 
		Im \phi = \left\{ \phi( g) |g \in G \right\} \subset H
	\]
	
\end{enumerate}

\end{defn}
\begin{propo}
L'image et le noyau sonnt des sous-groupes.
\end{propo}

\begin{proof}
La preuve pour l'image est dans le cours.\\
\[ 
	\ker \phi \neq \emptyset, \text{ car } \phi( e_G) = e_H
\]
On démontre que c'est stable par composition
\[ 
g,f\in \ker \phi
\]
, alors
\begin{align*}
	\phi( g.f) = \phi( g) .\phi(f) = e_H \cdot e_H e_H
\end{align*}
On vérifie que c'est stable par inversion.
\begin{align*}
g \in \ker \phi\\
\phi( g) ^{-1}= \phi( g^{-1}) 
\end{align*}
donc c'est fini.
\end{proof}

	





















\end{document}	
