\documentclass[../main.tex]{subfiles}
\begin{document}
\lecture{14}{Tue 15 Dec}{Fin Produit-Semidirect}
\begin{thm}
	Soit $H,F \leq G$, $F \leq N_G(H ) $ et $H \cap F = \left\{ e  \right\} $, dans ce cas, on peut definir $Ad_{F}^{H}$ et alors
	\[ 
		\eng{H,F} = HF \simeq H \rtimes_{Ad_{F}^{H} F} 
	\]
	De plus, si $H \leq N_G( F) $ 
	\[ 
		\eng{H,F} \simeq H \times F
	\]
	
	
\end{thm}
\begin{proof}

On a deja demontre la partie 1.\\
Il faut demontrer que $Ad_{F}^{H}\simeq \id$, ie que $Ad_{f}^{H}( h) = h$.\\
Donc $\forall f \in F, \forall h \in H$ 
\[ 
fhf^{-1} = h
\]
Or, $\forall f \in F, h \in H$, $fhf^{-1}h^{-1}=e$ car on a deux conjugaisons differentes.
\end{proof}
\begin{rmq}
$H \rtimes F$ est abelien si et seulement si $H,F$sont abeliens et $\phi$ est trivial.
\end{rmq}
\begin{exemple}
	Soit $S_3 = G$ et 
	\[ 
		H = \eng{ ( 123) } \trianglelefteq G \text{ et } F = \eng{ ( 12) }
	\]
	On voit clairement que $H \cap F = \left\{ \id \right\} $. \\
	On trouve donc que $S_3 =H\rtimes F \simeq \mathbb{Z} / 3 \mathbb{Z} \rtimes \mathbb{Z} / 2 \mathbb{Z} $

\end{exemple}
\begin{exemple}
Si on considere $G= D_{2n} $, on trouve
 que 
 \[ 
 G \simeq \mathbb{Z} / n \mathbb{Z} \rtimes_\phi \mathbb{Z} / 2 \mathbb{Z}
 \]
 
\end{exemple}
\begin{propo}
Il existe seulement deux groupes d'ordre 4.
\end{propo}
\begin{proof}
Premier cas:\\
$\exists g \in G: o( g) = 4$. Dans ce cas, on a fini.\\
Sinon, $\forall g \in G \setminus \left\{ e   \right\}: o( g) =2 $.\\
Montrons que ceci implique $G$ abelien.
\[ 
ab = aababb = ba
\]
Donc $G$ abelien.\\
Soit $a \neq b \in G \setminus \left\{ e \right\} $, donc
\[ 
	H = \eng{a} = \left\{ e,a \right\} \text{ et } F = \eng{b} = \left\{ e,b \right\} 
\]
On a alors que
\[ 
G \simeq H \times F \simeq \mathbb{Z} /2\mathbb{Z} \times \mathbb{Z}/2\mathbb{Z}
\]

\end{proof}
\begin{thm}
Soit $|G|=2p$, $p$ premier.\\
Alors
\begin{enumerate}
	\item $\exists h \in G: o( h) = p$
	\item $\exists f \in G: o( f) = 2$ 
	\item $\eng{h} = H, \eng{f} = F \Rightarrow \eng{H,F} = G$ et $G \simeq \mathbb{Z} / p \mathbb{Z} \rtimes \mathbb{Z}/ 2\mathbb{Z}$
\end{enumerate}

\end{thm}
\begin{proof}
si $\exists g \in G$ tel que 
\[ 
	o( g) = 2p
\]
Alors
\[ 
G \simeq \mathbb{Z} / 2p\mathbb{Z}
\]
Sinon, $\forall g \in G\setminus \left\{ e  \right\} , o( g) = 2$ ou $p$.\\
Supposons l'oppose, que $\not\exists h \in G: o( h) =p$, alors
\[ 
	\eng{a,b} = \mathbb{Z}/ 2\mathbb{Z}^{2}
\]
Or $4 \not| 2p$, on a donc une contradiciton.\\
Supposons maintenant que $\forall g \in G \setminus \left\{ e   \right\}: o( g) =p $\\
Prenons $a,b \in G \setminus \left\{ e  \right\} $ tel que $b \notin \eng{a}$
Posons que $H = \eng{a}, F = \eng{b}$, alors on a
\[ 
|H \cap F| \big| |H|, |F|
\]
Parce que $b \notin \eng{a}$. On a donc $|H \cap F| = 1$, on a donc que les sous-groupes engendre par les elements de $G \setminus \left\{ e   \right\} $, alors $G_i$ couvre $G$. On trouve donc que $2p = 1 + r( p-1) $, ce qui est une contradiction.\\
On a un element d'ordre $p$ et un element d'ordre 2 dans le groupe Donc
$H \trianglelefteq G$ est normal car son index est 2, il suit
\[ 
 G \simeq \mathbb{Z} / p \mathbb{Z} \rtimes_\phi \mathbb{Z}/2\mathbb{Z}
\]
\end{proof}
\begin{crly}
	$ ( \mathbb{Z}/ p\mathbb{Z}) ^{\times}$ cyclique pour $p >2$
\end{crly}







\end{document}	
