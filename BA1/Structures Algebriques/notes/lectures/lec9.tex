\documentclass[../main.tex]{subfiles}
\begin{document}
\lecture{9}{Tue 10 Nov}{10 mardi}
\begin{thm}
	Soit $H$ un sous-groupe normal de $G$.\\
Les deux propositions suivantes sont equivalentes:
\begin{enumerate}
\item $G / R_h$ est un groupe
\item $\xi_H$ est un homomorphisme
\end{enumerate}
\end{thm}
\begin{proof}
	Automatique si la multiplication est bien definie par rapport a $R_h$ ( deja fait).\\
\begin{enumerate}
\item Soient $( g,g') \in R_H$ et $( h,h') \in R_H$ 
	Il faut montrer que $( gh,g'h')\in R_H$ 
	\[ 
		( gh) ^{-1}g'h' = h^{-1}g^{-1}g'h' = h^{-1}h' h'^{-1}g^{-1}g'h' \in H
	\]
\item Soient $( g,g') \in R_H$ \\
	Il faut montrer

	\[ 
	( g^{-1} g'^{-1} ) \in R_H
	\]
	On verifie
	\[ 
		( g^{-1} )^{-1}( g') ^{-1}= g g'^{-1}= g g'^{-1}g g^{-1} \in H
	\]
	
	
	 

\end{enumerate}
	
\end{proof}
\begin{rmq}
Notons que
\[ 
|G / H| = |G:H| = \frac{|G|}{|H|}
\]

\end{rmq}
\begin{thm}
\begin{itemize}
\item $H$ sous-groupe normal de $G$ 
\item $\xi_H: G \to G /H$ 
\item $\phi: G \to F$
\end{itemize}
tel que $H \subseteq \ker \phi$.\\
Alors, il existe un unique $\eta: G / H \to F$ tel que
\[ 
\phi = \eta \circ \xi_H
\]
\end{thm}
\begin{proof}
Il y a une possibilite pour $\eta$, si
\[ 
	\eta( gH)  = \eta( \xi_H ( g) ) = \phi( g) 
\]
Il faut encore verifier que $\eta $ est un homomorphisme.\\
Montrons que $\eta$ est bien defini.
Supposons que $gH= g'H$, alors $( g,g') \in R_H$, donc $g^{-1}g'\in H$ \\
Donc 
\[ 
	\eta ( gH) = \phi( g) \text{ et } \eta( g'H) = \phi( g') 
\]
Or $H \subseteq \ker \phi$, donc
\[ 
	\phi( g^{-1}g') = e 
\]
Or car $\phi$ est un homomorphisme, on a 
\[ 
	\phi( g) = \phi( g') 
\]
Donc $\eta$ est bien definit.\\
Montrons que $\eta$ est un homomorphisme.
\[ 
	\eta( gH .g'H) = \eta( gg'H) = \phi( gg') = \phi( g) \phi( g') = \eta( g) \eta( g') 
\]
Montrons que si $H = \ker \phi \Rightarrow \eta$ injectif\\
\[ 
gH \in \ker \eta
\]
, donc
\[ 
	e= \eta( gH) = \phi( g) 
\]
Donc $g \in \ker \phi$, donc $g \in H$, donc $gH = H$, donc 
\[ 
gH = e_{G /H} 
\]
Donc $\eta$ est injectif.\\
On peut donc considerer $G /H$ comme un sous-groupe de $F$.
Donc
\[ 
	G / H \simeq Im( \eta) 
\]
\end{proof}

\begin{crly}
Donc
\[ 
G / \ker \phi \simeq Im \phi
\]
Pour n'importe quel homomorphisme $\phi$.
\end{crly}
\begin{crly}
\[ 
\frac{|G|}{|\ker\phi|}= |Im\phi|
\]
\end{crly}
\begin{crly}
\begin{itemize}
\item $\phi: G \to H$ 
\item $|G|= n$, $|H|= m$ 
\item $( n,m) =1$
\end{itemize}
Alors $\phi = e_H$
\end{crly}
\begin{proof}
$Im \phi \leq H$, donc par Lagrange 
\[ 
|Im\phi| \big\vert m
\]
De meme
\[ 
|\ker \phi| \big \vert n
\]
Donc 
\[ 
| G / \ker \phi | = \frac{n}{|\ker \phi|}\big \vert n
\]
Donc  $|Im \phi| = | G / \ker \phi| = 1$, donc 
\[ 
Im \phi = \left\{ e_H \right\} \text{ et } \ker \phi = G
\]
Et donc $\phi = e_H$.

\end{proof}
\begin{crly}
\[ 
	\eng{g} \simeq \mathbb{Z} / o( g) \mathbb{Z}
\]

\end{crly}
\begin{proof}
\[ 
dexp_g: \mathbb{Z} \to G
\]
On a vu que 
\[ 
	Im dexp_g = \eng{g}
\]
et que 
\[ 
	\ker dexp_g = o( g)  \mathbb{Z}
\]
Donc
\[ 
	\eng{g} = \mathbb{Z} / o( g)  \mathbb{Z}
\]

\end{proof}
\begin{crly}
Soit $|G|= p$ avec $p$ premier, alors
\[ 
G \simeq \mathbb{Z} / p \mathbb{Z}
\]

\end{crly}
\begin{proof}
On a vu que
\[ 
g \in G \setminus \left\{ 0 \right\} 
\]
, alors $\eng{g} = G$.\\
Et $o( g) = p$.
\end{proof}
\subsection{Groupes diedraux}
Graphe simple non oriente
\begin{defn}[Graphe nonoriente]
	C'est un ensemble $V$ de sommets et
	\[ 
	E \subseteq \left\{ S \subseteq V \big\vert |S| = 2 \right\} 
	\]
	( l'ensemble des arretes) 

\end{defn}
\begin{defn}[Isomorphismes des graphes]


	Soit $G= ( V,E) $ et $G'= ( V',E') $ deux graphes.\\
	 \[ 
	\phi: V \to V'
	\]
	est un isomorphisme si
	\begin{itemize}
	\item $\phi$ bijection
	\item $ \left\{ v,w \right\} \in E \iff \left\{ \phi( v) , \phi( w)  \right\} \in E'$
	\end{itemize}
\end{defn}
\begin{defn}[Groupe diedral]
	Le groupe diedral est l'ensemble des automorphismes d'un graphe.

\end{defn}


















































\end{document}
