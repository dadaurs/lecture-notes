\documentclass[../main.tex]{subfiles}
\begin{document}
\lecture{12}{Tue 01 Dec}{Groupe des Quaternions}
\begin{lemma}
	$H \leq G$, et $G$ fini.\\
	Avec $[G:H]=2$, alors $H\trianglelefteq G$.
\end{lemma}
\begin{proof}
L'hhypothese implique
\[ 
|G\setminus H| = |H|
\]
Donc
\[ 
gH \text{ ou } Hg \subseteq G \setminus H
\]
Donc $gH = Hg = G \setminus H$
\end{proof}
\begin{rmq}
$Q_8$ est presque abelien, car il contient un sous-groupe normal abelien.\\
En effet, $\eng{i}=H$ est un sous-groupe est $[Q_8:H] = 2$.\\
\[ 
H \trianglelefteq Q_8 \Rightarrow Q_8 / H \simeq \frac{\mathbb{Z}}{2\mathbb{Z}}
\]
	
\end{rmq}
\subsection{Sous-groupes de $G=GL( n,k) $}
\begin{enumerate}
\item Le centre forme un sous-groupe de $G$, on montre que les matrices sont les matrices diagonales.
\item Le groupe $PGL( n,K) = GL( n,K) / Z( GL( n,K) ) $ 
\item $B( n,K) $ sous-groupe standard de Borel $\leq GL( n,K) $, l'ensemble des matrices triangulaire superieures.
\item $U( n,K) \subseteq B( n,K) $, est le sous-groupe unipotent standard, avec $1$ dans la diagonale.
\item $N_{GL( 2,K) } T( 2,K) = \left\{ 
	\begin{pmatrix}
		a & 0\\
		0 & b
	\end{pmatrix}
	\right\}  $	
\item $Z( GL( n,K) ) \trianglelefteq GL( n,K) $ et considerons
	\[ 
		Z( GL( n,K) ) \cap SL( n,K) \leq SL( n,K) 
	\]	
	\begin{lemma}
	$F,H \leq G$ et $F \trianglelefteq G$, alors
	\[ 
	F\cap H \leq G
	\]
	
	\end{lemma}
	
\end{enumerate}


\end{document}	
