\documentclass[../main.tex]{subfiles}
\begin{document}
\lecture{8}{Tue 03 Nov}{3 Novembre}
\subsection{Theoreme de Lagrange}
\begin{defn}
	Soit $H \leq G$, une classe à gauche ( resp. à droite) est un sous-ensemble 
	\[ 
	g.H = \left\{ g.h | h \in H \right\} 
	\]
	Ce n'est pas forcément un sous-groupe.
\end{defn}
\begin{exemple}
	Soit $G= S_3$,
	\[ 
		H = \left\{ \eng{( 12) } \right\} = \left\{ id, ( 12)  \right\} 
	\]
Regardons les classe à gauche
\begin{enumerate}
\item $g= Id$ 
	\[ 
		Id.H = \left\{ Id, ( 12)  \right\} = ( 12).H
	\]

\item $g= ( 13) $ 
	\[ 
		( 13) . H = \left\{ ( 13) ,( 123)  \right\} 
	\]

\item 
	\[ 
		( 23) .H = \left\{ ( 23) ,( 132)  \right\} 
	\]
	
	
\end{enumerate}
\end{exemple}
\begin{lemma}
\[ 
|gH| = |H| \forall g \in G
\]
\end{lemma}
\begin{proof}
\[ 
	\beta: x \mapsto g^{-1}x
\]
et 
 \[ 
gy \leftarrow y : \alpha
\]
Ces deux applications sont inverses $\Rightarrow$ ils sont en bijection.
\end{proof}
\begin{propo}
Soit $H \leq G$.\\
On définit
\[ 
	R = \left\{ ( g,f) \in G \times G | g^{-1}f \in H \right\}  \subseteq G \times G
\]
Alors
\begin{enumerate}
\item $R$ est une relation d'équivalence
\item les classes d'équivalence de $R$ sont les classes à gauche
\end{enumerate}

\end{propo}
\begin{proof}
\begin{enumerate}
	\item reflexivite $g^{-1}g = e \in H$ donc $( g,g)  \in R$ 
	\item Symétrie $( g,f)  \in R$ donc $g^{-1}f\in H$ 
		\[ 
	f^{-1}g = f^{-1}( g^{-1} )^{-1} = ( g^{-1}f) ^{-1} \in H
		\]
	\item Transitivité $( g,f ) \in R$, $( f,h) \in R$, alors $g^{-1}f, f^{-1}h \in H$,
		\[ 
		g^{-1}h = g^{-1}f f^{-1}h \in H
		\]
		Donc $( g,h) \in R$
	

\end{enumerate}
On veut $R_g = gH$ 
\[ 
	R_g =  \left\{ f \in G | ( g,f) \in R \right\}  = \left\{ f \in G | g^{-1}f \in H \right\} = \left\{ f \in G | \exists x \in H: f = gx \right\} = gH
\]


\end{proof}
En somme:\\
$H \leq G$ \\
Les classes à gauche 
\begin{itemize}
\item Ont les même tailles


\item $H_1,\ldots, H_r$ sont les classes à gauche
	\[ 
	G = \amalg_i H_{i} 
	\]

	La notation $\amalg_i$ signifie que l intersection deux-à-duex est vide.
	

	Donc, si $G$ est fini
	\[ 
	|G| = \sum |H_i| = r|H|
	\]
\end{itemize}
\begin{thm}[Lagrange]
	$G$ est fini et $H\leq G$, alors
	\[ 
	|H| \big\vert |G|
	\]
	De plus $\frac{|G|}{|H|}= \text{ nombre de classes à gauche } = [ G: H] $
	
\end{thm}
\begin{crly}
$g \in G$, alors
\[ 
	\Rightarrow o( g) \big\vert |G|
\]

\end{crly}
\begin{proof}
	$H = \eng{g}$ et ensuite on utilise Lagrange.
\end{proof}
\begin{defn}
$G$ groupe est cyclique si il existe $g \in G$ tel que 
\[ 
	\eng{g} = G
\]

\end{defn}
\begin{crly}
$|G| = p > 0$ avec $p$ premier, alors $G$ cyclique
\end{crly}
\begin{proof}
$g \in G \setminus \left\{ e  \right\} $, donc
\[ 
	1 < o( g)  | p
\]
par Lagrange.\\
Donc $o( g) = p $ et donc $\eng{g} = G$.
\end{proof}
\begin{thm}[Petit theoreme de Fermat]\index{Petit theoreme de Fermat}\label{thm:petit_theoreme_de_fermat}
	Soit $m> 0$ et $a$ entier, avec
	\[ 
		( a,m) =1
	\]
	Alors 
	\[ 
		a^{\phi( m) }\equiv 1 ( m) 
	\]
	
\end{thm}
\begin{proof}
On sait que
\[ 
	[ a]  \in \left( \faktor { \mathbb{Z}} { m \mathbb{Z}} \right) ^{\times}
\]
Donc par lagrange
\[ 
	o( [ a] ) | \phi( m) 
\]
et donc
\[ 
	[ a] ^{\phi( m) } = [ 1] = [ a^{\phi( m) }] 
\]

\end{proof}
Prenons un groupe $G$

et une relation d'équivalence $R$ sur $G$.\\
On a vu que 
\[ 
\faktor G R
\]
est un groupe si la multiplication et l'inverse sont bien définis.\\
Dans ce cas
\[ 
[ g] \cdot [ h] = [ gh] 
\]
Il est équivalent de demander que
\begin{align*}
\xi: G \to G / R\\
g \to Rg
\end{align*}
est un homomorphisme de groupe et donc
\[ 
R_g \cdot R_h = R_{gh} 
\]
On applique çà aux classes à gauche
\[ 
	R= \left\{ ( g,f) | g^{-1}f \in H \right\} 
\]
On essaie de tourner 
\[ 
G / R = \left\{ \text{ classes à gauche }  \right\} 
\]
C'est nécessaire que 
\begin{align*}
\xi_h : G \to G/ H\\
g \to gH
\end{align*}
est un homomorphisme.\\
\[ 
\ker \xi_h = H
\]
Quelles conditions est-ce que ca pose?
\begin{defn}
	$H \leq G$ est \underline{normal} si $\forall g \in G$ 
	\[ 
	\forall h \in H
	\]
	on a 
	\[ 
	g^{-1}h g \in H
	\]
	On appelle ceci le conjugué de $h $ par $g$.
	
\end{defn}
\begin{defn}[Groupe simple]
	Si $H \leq G$ normal $\Rightarrow$ $H$ trivial.

\end{defn}
\begin{propo}
	Soit $\phi: G \to H$ un homomorphisme, alors le noyau de cet homomorphisme est normal $\ker \phi$ est normal.
\end{propo}
\begin{proof}
Soit $g\in G$, $h \in \ker \phi$ 
\[ 
	\phi( g^{-1}h g)  = \phi( g^{-1}) \phi( h) \phi( g) = \phi( g^{-1}) e \phi( g) = e
\]

\end{proof}
\begin{thm}
$H \unlhd G$ et $R$ relation d'équivalence des classes à gauche de $H$ 
\item $G / R$ un groupe et

\item 
	\begin{align*}
	\xi_H : G \to G / R\\
	g \mapsto g H
	\end{align*}
	un homomorphisme.
\end{thm}






























































































\end{document}	
