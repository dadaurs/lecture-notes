\documentclass[../main.tex]{subfiles}
\begin{document}
\lecture{26}{Mon 14 Dec}{Recherche de Primitives}
\begin{exemple}
\[ 
\int_{ a }^{ b } e^{x}  \cos x dx
\]
On a 
\begin{align*}
	\int_{ a }^{ b } e^{x} \cos x dx &= \cos x e^{x} \big\vert_{a}^{b}+ \int_{ a }^{ b } e^{x} \sin x\\
	&= \cos x e^{x} \big\vert_{a}^{b} + \sin x e^{x} \big\vert_{a} ^{b} - \int_{ a }^{ b } e^{x} \cos x dx
\end{align*}
Donc on trouve
\[ 
	\int_{ a }^{ b } e^{x} \cos x dx = \frac{1}{2} \left( e^{x} ( \sin x + \cos x)  \right) 
\]

\end{exemple}
\begin{exemple}
\[ 
	\int_{ a }^{ b } \frac{dx}{( x^{2}+1) ^{n}}
\]
Cas $n=1$ :\\
On a 
$\tan = 1 + \tan^{2}$, donc
\[ 
	\arctan '( x) = \frac{1}{\tan'( \arctan x) } = \frac{1}{1+x^{2}}
\]
Donc
\[ 
\int_{ a }^{ b }\frac{dx}{1+x^{2}}= \arctan \big\vert_{a}^{b}
\]
A present, on peut calculer par recurrence
\[ 
I_n=\int_{ a }^{ b }\frac{dx}{{ 1+x^{2} }^{n}}
\]
Par recurrence, trouvons $I_{n+1} $ a partir de $I_n$.
\begin{align*}
	I_n &= \int_{ a }^{ b }\frac{dx}{( 1+x^{2}) ^{n}}\\
	    &= [ x ( 1+x^{2}) ^{-n}] _{a} ^{b} + \int_{ a }^{ b } n \frac{2 x^{2}}{( 1+x^{2})^{n+1}}dx\\
	    &= [ x ( 1+x^{2}) ^{-n}] _{a} ^{b} + 2n \int_{ a }^{ b } \frac{( x^{2}+1) -1}{( 1+x^{2}) ^{n+1}}dx\\
	    &= [ x ( 1+x^{2}) ^{-n}] _{a} ^{b} + 2n \int_{ a }^{ b }\frac{dx}{( x^{2}+1 )^{n}} - 2n \int_{ a }^{ b }\frac{dx}{( x^{2}+1 )^{n+1}}
\end{align*}
Donc
\[ 
	I_{n+1} = [ \frac{x}{2n ( 1+x^{2}) ^{n}}]_{a}^{b}+\frac{( 2n-1)}{2n} I_n 
\]


	

\end{exemple}
\begin{thm}[Estimation du Reste Dans Taylor]
	Soit $f \in C^{( n+1) }$, alors
	\[ 
		f( x) = \sum \frac{1}{k!}f^{( k) }( a) ( x-a) ^{k} + \frac{1}{n!}\int_{ a }^{ x } f^{( n+1) }( t) ( x-t) ^{n}dt
	\]
	

\end{thm}
\begin{proof}
Cas $n=0$
\[ 
	f( x) = f( a) + \int_{ a }^{ x }f'( t) dt = f( a) + [ f] _{a} ^{x} = f( x) 
\]
Supposons vrai pour $n$.\\
On suppose vrai pour $n$, montrer pour $n+1$.\\
Hypothese de recurrence:
\[ 
	f( x) = \sum_{k=0}^{ n}\frac{1}{k!}f^{(k ) }( a) ( x-a)^{k} + \frac{1}{n!}\int_{ a }^{ x }f^{( n+1) }( t) ( x-t)^{n}dt\\
\]
Soit $h( t) = f^{( n+1) }( t) $, donc $h'( t) = f^{( n+2) }( t) $, et $g'( t) = ( x-t) ^{n}$, donc $g( t) = \frac{-1}{n+1}( x-t) ^{n+1}$.\\
On a
\begin{align*}
	\sum \frac{1}{k!} f^{( k) }( a) ( x-a) ^{k} + \frac{1}{n!} \left[ f^{( n+1) }( t) \frac{-1}{n+1}( x-t)^{n+1}\right]_{t=a}^{x} - \frac{1}{n!}\int_{ a }^{ x } f^{( n+2) ( t) \frac{-1}{n+1}( x-t)^{n+1}}dt\\
\end{align*}
Ici, le dernier terme est bel et bien le reste integral pour $n+1$.



\end{proof}
\subsection{Integration de Fonctions Rationelles}




\end{document}	
