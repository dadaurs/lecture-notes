\documentclass[../main.tex]{subfiles}
\begin{document}
\lecture{22}{Mon 30 Nov}{Integrales}
\subsection{Deux P.S sur $\exp$}
But: definir $x^{y}$ $\forall x y \in \mathbb{R} ( x \geq 0) $ \\
Compatible avec $x^{n}= x x \ldots x$ et compatible avec les regles concernant $x^{n}$ :
\begin{align*}
x^{n+m} = x^{n} x^{m}\\
( x^{n} )^{m} = x^{nm}
\end{align*}
\begin{defn}
Pour $x>0$ et $y \in \mathbb{R}$ quelconque, on pose 
\[ 
	x^{y} = \exp ( y \log x ) 
\]

\end{defn}
\begin{rmq}
	\begin{enumerate}
		\item $\log$ est definit pour $x>0$, car $\exp$ est une bijection de $\mathbb{R} \to \mathbb{R}_+ ^{*}$ 
		\item Cette definition satisfait nos contraintes.\\
			P. ex.: $n \in \mathbb{N}$ 
			\[ 
				n \in \mathbb{N}, \exp ( n \log x) = \exp ( \log x + \ldots + \log x) = \exp( \log x)\ldots \exp( \log x) = x^{n} 
			\]
	
	\end{enumerate}
	
\end{rmq}
On definit les fonctions trigonometriques hyperboliques par
\[ 
	\sinh ( x)  = \frac{e^{x}- e^{-x}}{2}
\]
et
\[ 
	\cosh ( x) = \frac{e^{x}+ e^{-x}}{2}
\]

\begin{exo}
Quelle est la serie entiere qui definit ca?
\end{exo}
\begin{propo}
	Supposons que $f$ ( soit derivable et) satisfasse $f'= f'$.\\
	Alors $f( x) = c \exp x$ ( avec $c=f( 0) $) .
\end{propo}
\begin{proof}
	Posonns $g( x) = f( x) e^{-x} $\\
	Alors $g$ est derivable et
	\[ 
		g'( x) = f'( x) e^{-x} - f( x) e^{-x} =0
	\]
	
	On sait que $g$ est constantem donc $f( x) = c \exp x$.
\end{proof}

\section{Integration}
But: Calculer et definir des aires ( = surfaces) , puis volumes, hypervolumes, etc.\\
Rapport avec la derivation ( thm. fondamental de l'analyse).

\begin{defn}[Subdivisions]
	Considerons l'intervalle $[a,b]$, avec $a< b$ et $f: [ a,b]  \to \mathbb{R}$.\\
	Une subdivision $\sigma$ de $[a,b]$ est une suite finie $x_0, x_1, \ldots, x_n$, tel que
	\[ 
	x_0=a < x_1 < x_2 \ldots < x_n = b
	\]
	
\end{defn}
\begin{defn}[Somme de Darboux inferieure]
La somme de Darboux inferieure associee a $f$ et $\sigma$ est
\[ 
	\underline{S}_{\sigma} ( f)  = \sum_{i=1}^{ n}m_i ( x_i - x_{i-1} ) 
\]
avec 
\[ 
	m_i = \inf \left\{ f( x)  | x \in [ x_{i-1} ,x_i]  \right\} 
\]
Defini pour toute fonction $f$ bornee sur $[ a,b]$.
\end{defn}
$\underline{S}_{\sigma} ( f)$	est une tentative ( a priori trop petite) de definir l'aire determinee par $f$ sur $ [ a,b] $.

\begin{defn}[Somme de Darboux superieure]
La somme de Darboux inferieure associee a $f$ et $\sigma$ est
\[ 
	\overline{S}_{\sigma} ( f)  = \sum_{i=1}^{ n}M_i ( x_i - x_{i-1} ) 
\]
avec 
\[ 
	M_i = \sup \left\{ f( x)  | x \in [ x_{i-1} ,x_i]  \right\} 
\]
Defini pour toute fonction $f$ bornee sur $[ a,b]$.
\end{defn}
\begin{defn}
	 \[ 
		 \underline{S}( f) = sup \left\{ \underline{S}_{\sigma} ( f)  \right\} 
	\]
	\[ 
		 \overline{S}( f) = sup \left\{ \overline{S}_{\sigma} ( f)  \right\} 
	\]
	
	
\end{defn}
\begin{defn}
	La fonction $f$ est dite integrable sur $[a,b]$, si $\underline{S}( f) = \overline{S}( f) $.
\end{defn}
L'integrale de $f$ sur $[a,b]$, note $ \int_{ a }^{ b } f$ est alors ce nombre $\underline{S}( f) = \overline{S}( f)$





\end{document}	
