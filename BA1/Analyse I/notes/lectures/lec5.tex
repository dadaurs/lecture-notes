\documentclass[../main.tex]{subfiles}
\begin{document}
\lecture{5}{Wed 30 Sep}{mercredi 30}
\begin{thm}
	$(x_n)$ converge $\iff \limsup_{n \to  \infty } x_n = \liminf_{n \to \infty } x_n$ 
	Dans ce cas, la limite prend cette meme valeur.
\end{thm}
\begin{proof}
$\Leftarrow$ :\\
Soit $z_n= \sup \left\{ x_p: p \geq n \right\} $,
\[ 
y_n = \inf \left\{ x_p: p \geq n \right\} 
\]
Rappel: $(z_n) \to LS$ et $(y_n) \to LI$ \\
Or, $y_n \leq x_n \leq z_n$. Donc par les 2 gendamrmes
\[ 
	\Rightarrow ( x_n) \to LS = LI
\]
 $\Rightarrow$ :\\
 Hypothese: $\lim_{n \to  + \infty} x_n =l$.\\
 A voir: $LS = LI =l$.\\
 Montrons par exemple que 
 \[ 
 \lim_{n \to  + \infty} z_n = l
 \]
 ( i.e. $LS = l$)\\
 Soit $\epsilon > 0$.
 \begin{align*}
 \exists N \forall n \geq N: \abs{x_n -l } < \frac{\epsilon}{2}\\
 \text{ et } \forall n \geq N: \abs{z_n - LS} < \frac{\epsilon}{4}
 \end{align*}
 Def. de $z_N  \Rightarrow \exists p \geq N: \abs{x_p} > z_N - \frac{\epsilon}{4} $\\
 A present 
 \[ 
	 \abs{LS -l} \leq \underbrace{\abs{LS -z_N}}_{< \frac{\epsilon}{4}} + \underbrace{\abs{z_n -x_p}}_{< \frac{\epsilon}{4}} + \underbrace{\abs{x_p -l}}_{< \frac{\epsilon}{2}}
 \]
 avec $p \geq N$ et $N \geq N$
 Donc $\forall \epsilon > 0:$ 
 \begin{align*}
	 \abs{LS -l } < \epsilon
 \end{align*}
 Donc $LS =l$
\end{proof}
\begin{thm}[Premiere regle de d'Alembert]\index{Premiere regle de d'Alembert}\label{thm:premiere_regle_de_d_alembert}
	Supposons $x_n \neq 0 \forall n$ \\
	Supposons que $\rho = \lim_{n \to  + \infty} \abs{ \frac{x_{n+1} }{x_n} }$ existe\\
	Si $\rho< 1$, alors $\lim_{n \to  + \infty} x_n =0$ \\
	Si $\rho > 1$, alors $(x_n)$ diverge.
\end{thm}
\begin{rmq}
Si $\rho =1$, on ne peut rien concluer
\end{rmq}
\begin{exemple}
\begin{itemize}
\item $x_n = n$ diverge, mais $\lim_{n \to  + \infty} \frac{n+1}{n} = 1$
\item $x_n = \frac{1}{n}$ converge mais $\lim_{n \to  + \infty} \frac{\frac{1}{n+1}}{\frac{1}{n}}= 1$
\end{itemize}
\end{exemple}
\begin{proof}
Supposons $\rho < 1$.\\
A voir: $x_n \to 0$.\\
Soit $\rho < r < 1$.
Convergence pour $\epsilon = r - \rho: \abs{\abs{ \frac{x_{n+1} }{x_n}} - \rho} < r - \rho$
\[ 
	\exists n_0 \forall n \geq n_0: \abs{ \frac{x_{n+1}}{x_n}} < r
\]
i.e. $\abs{x_{n+1} } < r \abs{x_n}$ de meme
$\abs{x_{n+2}} < r \abs{x_{n+1} } < r^{2}\abs{x_n}$\\

Conclusion $\forall m \geq n_0: \abs{x_{m} } < r ^{m-n_0} \abs{x_{n_0}} $\\
Donc
\[ 
	\forall m \geq n_0: \abs{x_m} < r^{m} \abs{x_{n_0}} r^{-n_0} 
\]
Onn sait que $ \lim_{m \to  + \infty} r ^{m} =0$ donc
\[ 
	0 \leq \abs{x_m} \leq r^{m} c	
\]
avec $c$ constante
Cas $\rho >1$.\\
On va montrer que 
$\abs{x_n}$ est non bornee.\\
Soit $1 < r< \rho$.
\[ 
	\exists n_0 \forall n \geq n_0: \abs{x_{n+1}/x_n} > r
\]
Donc
\[ 
	\abs{x_{n+1} } > r \abs{x_n}
\]
comme avant:
\[ 
	x_m > r^{m-n_0} \abs{x_{n_0}} 
\]
\end{proof}
\begin{rmq}
Si $r >1$, alors $ \lim_{n \to  + \infty} r ^{n} = + \infty $
$r^{n}$ est croissante donc il suffit de montrer que la suite est non bornee.\\
Si elle etait bornee, soit $l = \lim_{n \to  + \infty} r^{n} \in \mathbb{R}$\\
Mais $l = \lim_{n \to  + \infty} r^{n+1} = r l$\\
Donc $l\neq 0 \Rightarrow  1 =r $ absurde.
\end{rmq}
\begin{defn}[Sous-suite]\index{Sous-suite}\label{defn:sous_suite}
	Soit $(x_n)_{n=1} ^{\infty }$ une suite.\\
	Une sous-suite de $(x_n)$ est une suite de la forme $(x_{n_k})_{k=1} ^{\infty }$, ou $(n_k)_{k=1} ^{\infty }$ est une suite strictement croissante de $N$.
\end{defn}
\begin{exemple}
	Si $(x_n)$ est une suite, considerer:
	\[ 
	x_2,x_3,x_5,x_7,x_{11},x_{13},\ldots
	\]
	Ici, $n_k = 2,3,5,7,11,\ldots$
\end{exemple}
\begin{propo}
Si $x_n$ converge, alors toute sous-suite converge vers la meme limite.
\end{propo}
\begin{proof}
Soit $l= \lim_{n \to  + \infty} x_n$. Soit $( x_{ n_k } )_{k=1} ^{\infty }$ une sous-suite et $\epsilon > 0$.\\
A voir: $\exists k_0 \forall k > k_0 : \abs{x_{n_k} -l} < \epsilon$\\
Or $\exists n_0 \forall n > n_0: \abs{x_n -l} < \epsilon$.\\
Donc il suffit de choisir $k_0$ tq $n_{k_0} \geq n_0$. \\
( puisque la suite ($n_k$) est croissante.)
\end{proof}


\begin{thm}[Bolzano-Weierstrass]\index{Bolzano-Weierstrass}\label{thm:bolzano_weierstrass}
	Toute suite bornee admet une sous-suite convergente
\end{thm}
\begin{proof}
On va construire une sous-suite qui converge vers $s:= \limsup_{n \to \infty } x_n$ \\
Ici, $(x_n)$ est la suite en question et on pose
\[ 
z_n = \sup \left\{ x_p : p \geq n \right\} 
\]
Par recurrence, $n_1$ quelconque.\\
Supposons $n_{k-1} $ construit et construisons $n_k$:
\[ 
	\exists N \forall n \geq N : \abs{z_n -s} < \frac{1}{k}
\]
Choisissons un $n \geq N$, $n_{k-1} +1 $
\[ 
\exists p \geq n \text{ t.q. } x_p > z_n - \frac{1}{k}
\]

On definit $n_k = p$ ( $n_k > n_{k-1} $)\\
Or, $ \underbrace{\abs{x_{n_k} -s}}_{< \frac{1}{k}} \leq \underbrace{\abs{x_{n_k} -z_n}}_{< \frac{1}{k}} + \underbrace{\abs{z_n -s }}_{<\frac{1}{k}}$\\
Donc $(x_{n_k} \to s ( k \to \infty )$
\end{proof}
\begin{defn}[Point d'accumulation]\index{Point d'accumulation}\label{defn:point_d_accumulation}
	$x$ est un point d'accumulation de la suite $x_n$ s'il existe une sous-suite qui converge vers $x$.
\end{defn}
\begin{exemple}
	\begin{align*}
	x= \limsup_{n \to \infty} x_n\\
	x = \liminf_{n \to \infty } x_n
	\end{align*}
\end{exemple}
\subsection{Suites de Cauchy}
\begin{defn}[Suites de Cauchy]\index{Suites de Cauchy}\label{defn:suites_de_cauchy}
	La suite $(x_n)$ est dire de Cauchy si 
	\[ 
		\forall \epsilon > 0 \exists N \forall n,n' \geq N: \abs{x_n -x_{n'}} < \epsilon
	\]
\end{defn}
Attention:\\
Il ne suffit pas de comparer $x_n$ et $x_{n+k} $ pour $k$ fixe.
\begin{exemple}
$x_n = \frac{1}{1} + \frac{1}{2} + \frac{1}{3} + \ldots + \frac{1}{n}$
\end{exemple}

Cauchy $\iff \forall \epsilon > 0 \exists N \forall n \geq N \forall k \in \mathbb{N}: \abs{x_n - x_{n+k} } < \epsilon$

\begin{lemma}
	Si $(x_n)$ converge, elle est de Cauchy.
\end{lemma}
\begin{proof}
Soit $\epsilon > 0$, soit $l$ la limite.\\
Hypothese:\\
avec $\frac{\epsilon}{2} : \exists N \forall n \geq N: \abs{x_{n} -l} < \frac{\epsilon}{2}$\\
Si $n,n' \geq N$ 
\[ 
	\abs{x_n - x_{n'}} \leq \abs{x_n - l} + \abs{x_{n'} -l} < \epsilon
\]

\end{proof}
\begin{thm}[Convergence des suites de Caucjy]\index{Convergence des suites de Caucjy}\label{thm:convergence_des_suites_de_caucjy}
	Toute suite de Cauchy converge
\end{thm}

\begin{proof}
Soit $( x_n )$ de Cauchy.\\
\begin{lemma}
	$(x_n)$ est bornee.
	\begin{proof}
	Soit $\epsilon = 10$ 
	\[ 
		\forall N \forall n,n' \geq N: \abs{x_n - x_{n'} } < 10
	\]
	Donc $\abs{( x_n )}$ est bornee par
 \[ 
	 \max( \abs{x_{N}} + 10, \abs{x_1} , \abs{x_2}, \ldots, \abs{x_{N-1}} )
\]
	\end{proof}
\end{lemma}
	Appliquer Bolzano-Weierstrass
	\[ 
		\exists \text{ sous-suite } ( x_{n_k})
	\]
	qui converge, soit $l$ sa limite.
	A voir $(x_n)$ converge vers $l$.\\
	soit $\epsilon > 0 \exists k_0 \forall k \geq k_0 \abs{x_{n_k} -l} < \frac{\epsilon}{2}$
	\[ 
		\exists N \forall n,n' \geq N: \abs{x_n - x_{n'} } < \frac{\epsilon}{2}
	\]
	Si $n \geq N, n_{k_0} $ alors
	\[ 
		\abs{x_n -l} \leq \abs{x_n - x_{n_k}} + \abs{x_{n_k} -l} < \epsilon
	\]
	

\end{proof}
	


\end{document}	
