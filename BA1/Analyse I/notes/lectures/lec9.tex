\documentclass[../main.tex]{subfiles}
\begin{document}
\lecture{9}{Wed 14 Oct}{mercredi}
\begin{defn}
	On dit qu'une fonction $f$ est définie au voisinage de $x \in \mathbb{R}$, si
	$\exists \epsilon>0: f$ définie sur
	\[ 
		]x-\epsilon,x[ \text{ et } ]x,x+\epsilon[
	\]
	
\end{defn}
\begin{exemple}
	$f(x_0)=\frac{1}{x_0}$ défini au voisinage de 0.
\end{exemple}
\begin{defn}
Soit $f$ définie au voisinage de $ x_0$.
\[ 
	\lim_{x \to x_0} f(x) =l
\]
signifie
\[ 
\forall \epsilon>0 \exists \delta >0 \forall x
\]
\[ 
	0<|x-x_0| <\delta \Rightarrow |f(x) -l| < \epsilon
\]

\end{defn}
\begin{thm}
Soit $f$ définie au voisinage de $x_0$
\[ 
	\lim_{x \to x_0} f(x) = l \iff \forall \text{ suite } ( a_n)_{n=1} ^{ \infty}
\]
qui converge vers $ x_0$ et $a_n\neq x_0, \forall n$, on  a
\[ 
	\lim_{n \to \infty} f(a_n) =l
\]
\end{thm}
\begin{rmq}
	A priori, $f$ n'est pas définie en $a_n$, mais $\exists n_0, \forall n >n_0: a_n \in$ domaine de définition car $f$ définie au voisinage de $ x_0$
\end{rmq}

\begin{proof}
$\Rightarrow$ \\
Soit $a_n \neq x_0$, une suite convergent vers $x_0$. A voir: Soit  $\epsilon>0,$ cherche $ n_0 \forall n >n_0: |f(a_n) - l| < \epsilon$.\\
Par hypothese, $\exists \delta>0 \forall x$ 
\[ 
	0 < |x-x_0| < \delta \Rightarrow |f(x)-l| <\epsilon \quad ( 1)
\]
Appliquer $\lim a_n = x_0$ à $\delta$ :
\[ 
	\exists n_0, \forall n >n_0: |a_n-x_0| < \delta 
\]
Appliquer à présent 1 à $x= a_n$\\
$\Leftarrow$\\
Soit  $\epsilon >0$, on cherche $\delta>0$ \\
Supposons par l'absurde qu'aucun $\delta$ satisfait la définition.\\
En particulier, $\delta = \frac{1}{n}$ 
\[ 
	\exists x_n: 0<|x_n-x_0| < \frac{1}{n} \text{ et } |f(x_n)-l| \geq \epsilon 
\]
Or
\[ 
	x_n \neq x_0 \text{ et } ( x_n) \to x_0
\]
Par hypothèse
\[ 
	\lim_{n \to \infty} f(x_n) =l
\]
En particulier, pour $\epsilon$,
\[ 
	\exists n_0 \forall n >n_0 : |f(x_n) -l| <\epsilon
\]

\end{proof}
\begin{crly}
	Si $\lim_{x \to x_0} l$ et $ \lim_{x \to x_0} f'(x) = l'$, alors
	\[ 
		\lim_{x \to x_0} f(x) + f'(x) = l+l'
	\]
Idem pour produit.	
\end{crly}
\begin{crly}
	Si $f(x) \geq a, \quad \forall x$ au voisinage de $ x_0$ et
	\[ 
		\lim_{x \to x_0} f(x) = l, \text{ alors } l \geq a
	\]
	
\end{crly}
\begin{crly}
Si
\[ 
	\lim_{x \to x_0} f(x)=l
\]
Alors
\[ 
	\lim_{x \to x_0} |f(x)| = |l|
\]


\end{crly}
\begin{crly}
Pour
\[ 
	\lim \frac{g(x)}{f(x)}
\]
il suffit de traiter $\lim \frac{1}{f(x)}$.

\end{crly}
\begin{lemma}
	Si $ \lim_{x \to x_0} f(x) =l \neq 0$, alors
	 \[ 
		 \exists \epsilon>0 \forall x \in ]x_0-\epsilon,x_0[ \cup ]x_0,x_0+\epsilon[
	\]
	tel que $f(x) \neq 0$
\end{lemma}
\begin{proof}
\[ 
	|f(x) -l | < \frac{|l|}{2}
\]
dans un voisingae de $x_0$, alors
$f(x) \neq 0$
\end{proof}
\begin{crly}
	Si $\lim f(x) =l = \lim g(x)$ et 
	\[ 
		f(x) \leq h(x) \leq g(x) \forall x \text{ au voisinage de $x_0$ } 
	\]
	Alors
	\[ 
		\lim_{x \to x_0} h(x) =l
	\]
	
\end{crly}
\begin{crly}[Cauchy]
	Soit $f$ définie au voisinage de $x_0$, alors
	\[ 
	\lim_{x \to x_0} f(x) \text{ existe  } \iff \forall \epsilon>0 \exists \delta >0 \forall x_1,x_2 \text{ avec } 
	\]
	\[ 
		0< |x_i-x_0| < \delta \quad ( i=1,2)
	\]
	on a
	\[ 
		|f(x_i)- f(x_2)| < \epsilon
	\]
	
	
\end{crly}
\begin{lemma}
	Si $\lim f(a_n)$ existe $\forall$ suite ( $a_n \neq x_0$) convergeant vers $x_0$,alors
	\[ 
		\lim_{x \to x_0} f(x)
	\]
	existe
\end{lemma}
\begin{proof}
	Il suffit de montrer que toutes ces limites $f(a_n)$ ont la même valeur.\\
	En effet, on peut alors appliquer le théorème et $\lim_{x \to x_0} f(x) =l$ \\
	Sinon, $ \lim_{n \to  + \infty} f(a_n) = l \neq l' = \lim_{n \to  + \infty} f(a'_n)$ pour deux telles suites $a_n$ et $a'_n$. A présent
	\[ 
	b_n = 
	\begin{cases}
	a_n \text{ si $a$ pair } \\
	a'_n \text{ si $a$ impair } 
	\end{cases}
	\]

	or $f(b_n)$ converge absurde car elle admet deux sous-suites avec limites distinctes $l,l'$.

\end{proof}
\begin{proof}
Preuve du corollaire ci-dessus.\\
Grace au lemme, il suffit de montrerr que $\forall$ suite $a_n \to x_0$, la suite $f(a_n)$ est de Cauchy.\\
Par hypothèse, $\exists \delta>0 \forall x_1,x_2: 0< |x_i-x_0| < \delta$ implique
\[ 
	|f(x_1) -f(x_2)| < \epsilon
\]
Or, $\exists n_0 \forall n>n_0: |a_n -x_0| < \delta $.\\
Applique $a_n=x_1$ et $a_m = x_2$ donne que $f(a_n)$ est de cauchy.
\end{proof}
\begin{crly}
	Si $\lim_{x \to x_0} f(x)=l$ et $\lim_{x \to x_0} f(x) = l'$, alors $l=l'$.
\end{crly}

\begin{rmq}
On a implicitement utilisé les concept de $+, \cdot, \leq$ sur les fonctions.\\
\hr\\
Ce n'est pourtant pas un corps.\\
Par exemple, $\forall x,y \in $ corps
\[ 
xy=0 \Rightarrow x=0 \text{ ou } y=0
\]

\end{rmq}
Les fonctions ont une opération supplémentaire
\[ 
f \circ g
\]
est définie par 
\[ 
	f\circ g ( x) = f(g(x))
\]
Soit $g: A \to B$ des parties de $\mathbb{R}$, et $f:B \to \mathbb{R}$
avec $g$ défini au voisinage de $x_0$ et $f$ au voisinage de $ g_0$.\\
\begin{propo}
	Supposons $g(x) \neq g_0 \forall x$ au voisinage de $ x_0$\\
	Si $\lim_{x \to x_0} g(x) =y_0$ et $\lim_{y \to y_0} f(y) = l $, alors
	\[ 
		\lim_{x \to x_0} f\circ g ( x) =l 
	\]
\end{propo}
\begin{proof}
Soit $\epsilon>0$, à voir $\exists \delta>0 \forall x:$ 
\[ 
	0< |x-x_0| < \delta \Rightarrow |f(g(x)) -l | < \epsilon
\]
2eme hup nous dit
\[ 
	\exists \eta > 0 \forall y: |y-y_0| <\eta \Rightarrow |f(y)-l| < \epsilon
\]
Idee: appliquer la premiere hypothèse à $\eta$ et poser $y=g(x)$.\\
Ca marche, tant que $y\neq y_0$.
\end{proof}
\begin{exemple}
Exemple délicat:\\
Soit
\[ 
	g(x) =
	\begin{cases}
x\sin \frac{1}{x} \text{ si } x\neq 0\\
0 \text{ si } x=0
	\end{cases}
\]
Clairement $\lim_{x \to 0} =0$.\\
On pose que
\[ 
	f(x)=
	\begin{cases}
	0 \text{ si } x\neq 0\\
	1 \text{ si } x=0
	\end{cases}
\]
On voit que $\lim_{y \to 0} f(y) = 0$.\\
Or 
\[ 
	\lim_{x \to 0} f(g(x))
\]
n'existe pas.


\end{exemple}













\end{document}	
