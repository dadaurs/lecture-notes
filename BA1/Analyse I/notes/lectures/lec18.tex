\documentclass[../main.tex]{subfiles}
\begin{document}
\lecture{18}{Mon 16 Nov}{Developpements Limites}
De maniere generale, si
\[ 
	P( x) = \sum_{k=0}^{ n}c_k ( x-a) ^{k}
\]
Alors
\[ 
	P^{( k) }( a) = k! c_k
\]
\begin{defn}[Developpement limite]
Soit $f$ une fonction definie au voisinage de $a$.\\
Un developpement limit ( DL) d'ordre $n$ pour $f$ en $a$ est la donnee d'un polynome 
\[ 
	\sum_{j=0}^{ n}a_j ( x-a)^{j} 
\]
( la partie principale) et d'une fonction $r$ ( le reste) tel que
\begin{enumerate}
	\item $f( x) = \sum a_j ( x-a)^{j}+ r( x) $ 
	\item $\lim_{x \to a} \frac{r( x) }{( x-a)^{n}}$
\end{enumerate}

\end{defn}
\begin{rmq}
Le cas $n=1$ correspond a notre critere de differentiabilite de $f$ en $a$.\\
Admettre un DL d'ordre 1 en $a$ $\iff$ $f$ derivable en $a$.
\end{rmq}
\begin{rmq}
	Admettre un DL d'ordre 0 en $a$ $\iff$ $ \lim_{x \to a} f( x) $ existe.
\end{rmq}
\begin{rmq}
	Si on note $r( x) = \epsilon( x) ( x-a)^{n}$, on obtient 
	\begin{enumerate}
		\item $f( x) = \sum_{j=0}^{ n}a_j ( x-a) ^{j} + \epsilon( x) ( x-a) ^{n}$
		\item $\lim_{x \to a} \epsilon( x) = 0$
	\end{enumerate}
	Ceci montre que si $f$ admet un DL d'ordre $n$, alors $\forall k \leq n$ 
	\[ 
		f^{( k) }( a) = k! a_k
	\]
	
	
\end{rmq}
\begin{propo}
Si un DL d'ordre $n$ existe, il est unique.
\end{propo}
\begin{proof}
Suffit de montrer l'unicite de la partie principale. Or, les coefficients $a_j$ sont determines par les derivees de $f$,
\[ 
	f^{( k) }( a)  ( k\leq n) 
\]
qui existent bel et bien par la remarque precedente.
\end{proof}
\begin{propo}
Si $f$ admet un DL d'ordre $n$ en $a$, elle adment aussi un DL d'ordre $m \leq n$ en $a$.
\end{propo}
\begin{proof}
On a qu'a tronquer la partie principale de la somme.
\end{proof}
\begin{thm}
	Soit $f\in C^{n+1}( I) $, $I$ intervalle ouvert, $a\in I$.\\
	Alors $f$ admet un DL d'ordre $n$ en $a$, et la partie principale est le polynome de Taylor.
\end{thm}
\begin{proof}
	On va juste utiliser que $f^{( n+1) }$ est bornee sur un voisinage de $a$.\\
	A voir:\\
	\[ 
		\lim_{x \to a} \frac{\frac{1}{n+1!}f^{( n+1) ( t)( x-a)^{n+1} }}{( x-a) ^{n}} = 0
	\]
	
\end{proof}
\begin{exo}
	Si $f$ et $g$ admettent des DL d'ordre $n$ en $a$, alors $f+g$, $f\cdot g$ et $f\circ g$(ici: $f$ admet DL en $g( a) $)  aussi.\\
	Les parties principales seront la somme, respectivement produit, resp. composition des parties principales.
\end{exo}
\begin{exemple}[Un Bel Exemple]
	Soit
	\begin{align*}
		f( x) =
		\begin{cases}
		0 \text{ si  } x=0\\
		\exp( - \frac{1}{x^{2}}) \text{ si } x\neq 0
		\end{cases}
	\end{align*}
	On voit que $f \in C^{ \infty } ( \mathbb{R}) $.\\
	\begin{lemma}
		$f^{( k) }= f . \frac{P}{Q}$, pour deux polynomes $P, Q$.
	\end{lemma}
	Conclusion:\\
	$f$ est une fonction $C^{ \infty }$ avec $f^{( k) }( 0) =0 \forall k$.\\
	Tous les polynomes de Taylor de $f$ en $0$ sont nuls,
	\[ 
		T( x) = \sum_{k=0}^{ n} \frac{1}{k!}0 ( x-a) ^{k}= 0	
	\]
	$f$ admet un DL d'ordre $n$ en zero $\forall n$, partie principale 0.
	
	

\end{exemple}










\end{document}	
