\documentclass[../main.tex]{subfiles}
\begin{document}
\lecture{24}{Mon 07 Dec}{Integrales}
\begin{propo}
	Soit $f \in C^{0}( [ a,b] ) $ et $( \sigma_j) $ une suite de subdivisions.\\
	On suppose que la maille tend vers 0.\\
	Alors
	\[ 
	\lim_{j \to  + \infty} R_{\sigma_j} = \int_{ a }^{ b }f
	\]
	pour toute somme de Riemann $R_{\sigma_j} $ associee a $\sigma_j$.\\
\end{propo}
\begin{exemple}
	Soit $\sigma_j = ( x_0, \ldots, x_j) $ avec
	\[ 
	x_i = a + i \frac{b-a}{j}
	\]
	On a alors
	\[ 
		R= \sum_{i=1}^{ j}f( \xi_i) ( x_i - x_{i-1} ) 
	\]
	
	
\end{exemple}
\begin{proof}
L'uniforme continuite dit
\[ 
\forall \epsilon> 0 \exists \delta>0 \forall x,y:
\]
\[ 
	|x-y| < \delta \Rightarrow |f( x) - f( y) | < \epsilon
\]
Donc, si $h( \sigma) <\delta$, alors
$f( \xi_i) , m_i, M_i$ different au plus de $\epsilon$.

\end{proof}
\begin{lemma}
	Soient $f_1\leq f_2$ integrables sur $[a,b]$.\\
	Alors $\int_{ a }^{ b }f_1\leq \int_{ a }^{ b } f_2$.
\end{lemma}
\begin{proof}
	$\forall \sigma \forall [ x_{i-1} , x_i] : m_i ( f_1) \leq m_i( f_2) $.\\
	Fini.
\end{proof}
\begin{lemma}
	Si $f$ est integrable sur $[a,b]$, alors $|f|$ aussi et $\int_{ a }^{ b }|f|\geq | \int_{ a }^{ b }f|$
\end{lemma}
\begin{proof}
Suit de l'inegalite triangulaire pour les sommes.
\end{proof}
\begin{lemma}
	Si $f$ est integrable sur $[a,b]$ et $a<c<d<b$, alors $f$ integrable sur $[c,d]$.
\end{lemma}
\begin{proof}
	A montrer, $\forall  \epsilon> 0$, on cherche $\sigma$ subdivision de $[c,d]$ tel que $\overline{S}_\sigma( f) < \underline{S}_\sigma( f) + \epsilon$.\\
	Or, par hypothese, $\exists \tau$ sbdivision de $[a,b]$ tel que
	\[ 
		\overline{S}_\tau < \underline{S}_\tau( f) +\epsilon	
	\]
	Posons $\rho= \tau \cup \left\{ c,d \right\} $.\\
	On sait que $\overline{S}_\rho( f) \leq \overline{S}_\tau( f) $.\\
	Posons $\sigma$ la subdivision $\rho$ sans les elements avant c et apres $d$.\\
	\[ 
		\overline{S}_\sigma( f) -\underline{S}_\sigma( f) = \overline{S}_\rho( f) - \underline{S}_\rho( f) - \sum_{i=1}^{ k}( M_i-m_i) ( x_i - x_{i-1} ) - \sum_{i=l+1}^{ n}( M_i-m_{i} ) ( x_i-x_{i-1} ) 
	\]
	
	
\end{proof}
\begin{propo}
	Soit $f$ integrable sur $[a,b]$ et $a<c<b.$ Aors
	\[ 
	\int_{ a }^{ b }f = \int_{ a }^{ c }f + \int_{ c }^{ b }f
	\]
	De plus, si $f$ est integrable sur $[a,c]$ et sur $[c,b]$, alors $f$ est integrable sur $[a,b]$
\end{propo}
\begin{proof}
Soit $\epsilon>0$.\\
Puisque $f$ est integrable sur $[a,c]$ et $[c,b]$, $\exists  \sigma$ et $\tau$.\\
On a alors
\[ 
	\overline{S}_\sigma( f) < \underline{S}_\sigma( f) + \frac{\epsilon}{2}
\]
$\sigma\cup\tau$ est une subdivision de $[a,b]$ et
\[ 
	\overline{S}_{\sigma\cup\tau} = \overline{S}_\sigma + \overline{S}_\tau
\]
Donc, puisque vrai pour tout $\epsilon$, on a fini.\\
On a seulement utilise les hypotheses que $f$ est integrable sur $[a,c]$ et sur $[c,b]$, donc la 2eme partie de la proposition est aussi demontree.	
\end{proof}

\begin{rmq}
Si $a>b$, on definit
\[ 
\int_{ a }^{ b }f = - \int_{ a }^{ b }f
\]
On note aussi
\[ 
\int_{ a }^{ a }f=0
\]

\end{rmq}
\begin{crly}
\[ 
\forall a,b,c : \int_{ a }^{ b }f = \int_{ a }^{ c }f+ \int_{ c }^{ b }f
\]

\end{crly}
\begin{lemma}
	Soit $f$ integrable sur $[a,b] $. Alors 
	\[ 
		\inf ( f) ( b-a) \leq \int_{ a }^{ b }f \leq \sup f ( b-a) 
	\]
	
\end{lemma}
\begin{proof}
	Il suffit de considerer $\underline{S}_{[a,b]} $ et $\overline{S}_{[a,b]} $
\end{proof}
\begin{propo}[Theoreme de la moyenne]
	Soit $f$ une fonction continue sur $[a,b]$.\\
	Alors $\exists c\in ] a,b[ $ tel que 
	\[ 
		\int_{ a }^{ b } = f( c) ( b-a) 
	\]
	
\end{propo}
\begin{proof}
Le lemme ci-dessus implique que
\[ 
\inf f\leq \frac{1}{b-a} \int_{ a }^{ b }f \leq \sup f
\]
Comme $f$ est continue sur $[a,b]$, $\inf f$ et $\sup f$ sont des valeurs de $f$.\\
Donc $\frac{\int_{a}^{b}f}{b-a}$ est une valeur intermediaire, donc par TVI, on a fini.
\end{proof}
\begin{defn}[Primitive]
Soit $f$ une fonction.\\
On dit qu'une fonction $F$ est une primitive de $f$ si $F$ est derivable et $F'=f$.
\end{defn}
\begin{rmq}
\begin{enumerate}
\item Si une primitive  $F$ existe, alors il y en a plusieurs: $F+c$
\item TAF im plique que toute autre primitive est de cette forme.
	\[ 
		F_2' = f = F' \Rightarrow ( F_2-F) ' = 0 \Rightarrow F_2 -F \text{ est constant par TAF } 
	\]
	
\item De nombreuses fonctions sont sans primitives: theoreme de Darboux.
\end{enumerate}

\end{rmq}
\begin{crly}
	Toute fonction continue est la derivee d'une fonction ( derivable) 
\end{crly}

\begin{thm}[Theoreme Fondamental]
	Soit $f:[a,b] \to \mathbb{R}$ continue.\\
	Definissons $F:[a,b] \to \mathbb{R}$ par
	\[ 
		F( x) = \int_{ a }^{ x }f
	\]
	Alors $F$ est une primitive de $f$.

\end{thm}
\begin{proof}
	On sait deja que $F( x) = \int_{ a }^{ x }f$ est bien defini.\\
	Soit $x_0\in]a,b[$, alors pour $x\in [ a,b] $ :
	\[ 
		\underbrace{F( x) -F( x_0) }_{x-x_0} = \frac{\int_{ a }^{ x }f - \int_{ a }^{ x_0 }f}{x-x_0}= \frac{\int_{ x }^{ x_0 }f}{x-x_0}
	\]
	Pour la derniere egalite, il suffit de distinguer les cas.\\
	A demontrer: $\lim_{x \to x_0} \frac{\int_{ x_0 }^{ x }f}{x-x_0}$ existe et vaut $f( x).$\\
	Par le theoreme de la moyenne, $\exists c_x \in [ x_0,x ] $ 
	\[ 
		\int_{ x_0 }^{ x }f = f(x) ( x-x_0) 
	\]
	
	Donc, reste a montrer $\lim_{x \to x_0} f( c_x) = f( x_0) $.\\
	Or $\lim_{x \to x_0} c( x) = x_0$.\\
	Comme $f$ est continue, on conclut par limite de composition.
	
\end{proof}
\begin{thm}[Fondamental, reformulation]	
	Soit $f$ continue sur $[a,b]$ et $G$ une primitive de $f$.\\
	Alors 
	\[ 
		\int_{ a }^{ b }f= G( b) - G( a) 
	\]
	
\end{thm}
\begin{proof}
	Posons $F( x) = \int_{ a }^{ x }f$ TAF $\Rightarrow$ $\exists c \in \mathbb{R}: G = F+c$. Or $F( a) = 0 \Rightarrow  c= G( a) $ \\
	Et $ \int_{ a }^{ b }f= F( b) = G( b) - G( A) $.
\end{proof}















\end{document}	
