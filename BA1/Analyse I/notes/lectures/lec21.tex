\documentclass[../main.tex]{subfiles}
\begin{document}
\lecture{21}{Wed 25 Nov}{Series de Taylor}
\begin{rmq}
Si $f$ est $C^{ \infty }$, on peut definir sa serie de Taylor en $x_0$ 
\[ 
	\sum_{n=0}^{ \infty }\frac{1}{n!} f^{( n) }( x_0) ( x-x_0)^{n}
\]
C'est donc une serie entiere.
\end{rmq}
Attention, en general, cette serie n'est pas $f$!\\
En revanche, si $f$ est definie par une serie entiere $ \sum a_n x^{n}$, alors sa serie de Taylor est elle-meme $\sum a_n x^{n}$.
\begin{crly}
	Soit $f( x) = \sum a_n x^{n}$ et soit $R$ son rayon de convergence.\\
	Alors $f$ est $C^{ \infty }$ sur $]-R,R[$
\end{crly}
\begin{proof}
Par recurrence, puisque la derivee est a nouveau une serie entiere, il suffit de montrer le cas $n=1$.\\
\[ 
	f'( 0) = \sum_{n=1}^{ \infty }n a_n x^{n-1} = 1. a_1
\]

\end{proof}
\begin{propo}
\begin{enumerate}
	\item $\exp( x) > 0 \forall x$ 
	\item $\exp$ est croissante et convexe
	\item $ \lim_{x \to  + \infty} \exp( x) = + \infty $ 
	\item $\exp( x+y) = \exp x \exp y$
	\item $\exp -x = \frac{1}{\exp x}$
\end{enumerate}

\end{propo}
\begin{proof}
Derivons $x \mapsto \exp x \exp -x$, on a 
\[ 
	\exp' x \exp -x + \exp x \exp' -x -1 = 0
\]
De plus 
\[ 
\exp 0 \exp -0 = 1
\]
Donc $\exp x \neq 0 \forall x$ et $\exp -x = \frac{1}{\exp x}$
Fixons $y$. Derivons $x \mapsto \frac{\exp x+y}{\exp x}$, on trouve
\[ 
\frac{\exp x+y- \exp x+y \exp x}{\exp x} =0
\]
Donc la fonction est constante. De plus, en  $x=0$, on trouve $\exp y$, donc $\forall x $ 
\[ 
\exp y \exp x = \exp x+y
\]
On a 
\[ 
\exp x = \exp \frac{x}{2} + \frac{x}{2} = \left( \exp \frac{x}{2}  \right)^{2} > 0
\]
On a finalement
\[ 
\exp x \geq x
\]
pour $x\geq 0$, de meme
\[ 
\exp -x = \frac{1}{\exp x}
\]
il en suit les deux limites.




\end{proof}




\begin{crly}
$\exp$ est une bijection entre les reels et les reels strictements positifs et admet donc un inverse $C^{ \infty }$ notee $\log$
\end{crly}
\begin{thm}[Lemme d'Abel]
	Soit $f( x) = \sum a_n x^{n}$.\\
	On suppose $R \neq 0$.\\
	Si $\sum a_n R^{n}$ converge, alors $f$ est continue ( a gauche) en $R$.\footnote{Meme theoreme en $-R$ }\\
\end{thm}
\begin{proof}
	On suppose que $f( R) $ converge.\\
	Spg $f( R) = 0$.\\
	Notons $b_n = a_n R^{n}$ et $g( w) = \sum b_n w^{n}$ \\
	Donc, il s'agit de prouver:
	\[ 
		\lim_{w \to 1-} g( w) = 0.
	\]
	Soit $s_n= b_0+ b_1 + \ldots + b_n$, donc $s_n \to 0$.\\
	Pour $g( w) = \sum b_n w^{n}$, or 
	\[ 
		b_0 + b_1 w + \ldots b_n w^{n}= s_0 + ( s_1-s_0) w + \ldots + ( s_n - s_{n-1} ) w^{n}
	\]

Or
\[ 
	= s_0 ( 1- w)  + s_1 w ( 1-2)  + \ldots s_{n-1} w^{n-1  }( 1-w) + s_n w^{n}
\]
Donc $g( w) = ( 1-w) \sum s_n w^{n} $\\
Il suffit donc de prouver que 
\[ 
\sum s_n w^{n}
\]
converge et reste bornee independament de $\omega \in [ 0,1] $.\\
Or $s_n$ converge.
\[ 
	| \sum_{n=n_0} ^{ \infty } s_n w^{n}| \leq \epsilon \sum w^{n} = \epsilon w^{n_0} \frac{1}{1-w}
\]
Or 
\[ 
	|g( w) | \leq ( 1-w)  \left\{ | \sum^{n_0-1} s_n w^{n}  | + \epsilon w^{n_0} \frac{1}{1-w}\right\}
\]
Donc $g( w) $ converge vers 0.


\end{proof}






















































\end{document}	
