\documentclass[../main.tex]{subfiles}
\begin{document}
\lecture{27}{Wed 16 Dec}{Decomposition en Elements Simples}
\begin{rmq}
	Liste des Simplifications de $ \frac{P( x) }{Q( x) }$ 
\begin{enumerate}
\item Reduire la fraction
\item Quitte a multiplier par une constante, $Q( x) $ est unitaire.
\item Decomposer $Q( x) $ en facteurs de degre $1$ et $2$
\item sans perte de géneralité, $\frac{P( x) }{Q( x) }$ satisfait $\deg P < \deg Q$ 
\item Decomposition en elements simples.
\end{enumerate}
\end{rmq}
Sous ces hypotheses, $\frac{P( x) }{Q( x) }$ est une somme de termes de la forme
\[ 
	\frac{\alpha x + \beta}{( x^{2}+ 2cx +d)^{p}}
\]
Il faut donc trouver des primitives pour les fractions ci-dessus.\\
Considerons
\[ 
	\frac{\alpha x + \beta}{( x^{2}+1)^{p}}
\]
On separe la fraction en deux.\\
Premiere fraction facile par changement de variable.
Considerons donc
\[ 
\frac{1}{( x^{2}+1 )^{p}}
\]
Facile si $p=1$ ( arctangente) , sinon on a
\[ 
	\frac{\alpha x + \beta}{( x^{2}+2cx +d)^{p}}
\]
L'idee est de pose $x^{2}+ 2cx +d = y^{2}+1$ pour $y$ fonction affine de $x$. On pose
\[ 
	x= \sqrt{d-c^{2}} y - c\iff y = \frac{x+c}{\sqrt{d-c^{2}}}
\]


\end{document}	
