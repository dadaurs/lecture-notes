\documentclass[../main.tex]{subfiles}
\begin{document}
\lecture{15}{Wed 04 Nov}{Derivees}
\begin{exemple}
	Soit $f( x) = \sqrt[q]{x} = x ^{\frac{1}{q}}$.\\
	Donc la dérivée est
	\[ 
		f'( x) = \frac{1}{q ( x^{\frac{1}{q}})^{q-1}}= \frac{1}{q} x^{ \frac{1-q}{q}} = \frac{1}{q} x^{\frac{1}{q}-1}
	\]
On sait donc dériver des puissances rationelles quelconques	
\[ 
	f( x) = x^{\frac{p}{q}} = g( h( x) ) 
\]
donc
\begin{align*}
	f'( x) = p( x^{\frac{1}{q}}) ^{p-1} \frac{1}{q} x^{\frac{1}{q}-1}\\
	= \frac{p}{q} x^{\frac{p-1}{q}+ \frac{1}{q}-1} = \frac{p}{q} x^{\frac{p}{q}-1}
\end{align*}

\end{exemple}
\subsection{Applications de la dérivée}
\subsubsection{Recherche d'extremums}
\begin{defn}[Point Critique]
	$x$ est un point critique de $f$ si $f$ est dérivable en $x$ et $f'( x)=0 $.
\end{defn}

\begin{rmq}
	Bien que $x=0$ soit un point critique de $f( x) =x^{3}$, cette fonction est strictement croissante.
\end{rmq}
\begin{defn}
	$x$ est un maximum local de $f$ si il existe un voisinage de $x$ sur lequel $x$ est un ( vrai) maximum.
\end{defn}
La définition pour les minimas est équivalente.\\
Plus généralement, on parlera d'extremums.
\begin{propo}
	Soit $f$ dérivable en $x_0$ ( donc définie dans son voisinage).\\
	Si $x_0$ est un extremum local de $f$, alors c'est un point critique.
\end{propo}
\begin{rmq}
\begin{enumerate}
\item La réciproque est fausse.
\item Si $f$ n'est pas dérivable, on peut très bien avoir un max / min.
\item Dérivable à droite ( gauche) pas suffisant.
\end{enumerate}
\end{rmq}
\begin{proof}
Sans perte de géneralité, $x_0$ est un max local de $f$.\\
Dérivée à gauche:
\[ 
	\lim_{x \to x_0-}  \frac{f( x) - f( x_0) }{x-x_0} \geq 0
\]
De même, dérivée à droite
\[ 
	\lim_{x \to x_0+}  \frac{f( x) - f( x_0) }{x-x_0} \leq 0
\]
Donc, $f'( x_0) = 0$.
\end{proof}
\begin{propo}[Méthode de recherche d'extremum]
	Pour $f:[a,b] \mapsto \mathbb{R}$.\\
	Les candidats sont
	\begin{enumerate}
		\item Les points critiques dans $] a,b[$ 
		\item Les points non-différentiables
		\item Les bornes: $a,b$
	\end{enumerate}
	
	

\end{propo}
\begin{thm}[theoreme de Rolle]\index{theoreme de Rolle}\label{thm:theoreme_de_rolle}
	Soit $f: [ a,b] \to \mathbb{R}$ continnue et dérivable sur $]a,b[$.
	Si $f( a) = f( b) $ alors $\exists x \in ]a,b[: f'( x) = 0$ 
	
\end{thm}
\begin{proof}
	$\exists x_1 \in [ a,b] : \sup_{x\in [ a,b]} f( x) = f( x_1)   $.\\
	Si $x_1 \in ]a,b[$, alors $f'( x_1) = 0$ par la proposition.\\
	Cas restant: $x_1= a$ ou $b$.\\
	\[ 
		\exists x_2 \in [ a,b] \text{ min } : \inf_{x \in [ a,b] } f( x)  = f( x_2) 
	\]
	Le seul cas restant est donc:
	\begin{center}
	$\max$ et $\min$ atteints en $a$ ou $b$.
	\end{center}
	Or $f( a) = f( b)$, donc $\max f = \min f$, donc $f$ constante.
\end{proof}
\begin{thm}[théorème des accroissements finis TAF]\index{théorème des accroissements finis TAF}\label{thm:theoreme_des_accroissements_finis_taf}
	Soit $f: [ a,b] \to \mathbb{R}$ continue et dérivable sur $]a,b[$.\\
	Alors $\exists x \in ]a,b[: f'( x) = \frac{f( b) -f( a) }{b-a}$.
\end{thm}
\begin{proof}
	Posons $g( x) = f( x) - \frac{f( b) - f( a) }{b-a} ( x-a)  $ 
	$g$ est aussi continue dérivable.\\
	Or 
	\[ 
		g( a) = f( a) \text{ et } g( b) = f( a)  
	\]
	Donc $g$ satisfait l'hypothèse de rolle.\\
	Donc, par Rolle
	\[ 
		\exists x \in ]a,b[ : g'( x)  = 0
	\]
	Or 
	\[ 
		g'( x) = f'( x) - \frac{f( b) -f( a) }{b-a} = 0
	\]
	
\end{proof}
\begin{crly}
Si $f'= 0$, alors $f$ constante.\\
Plus précisément:\\
Soit
\[ 
	f:[a,b] \to \mathbb{R}
\]
continue dérivable sur $]a,b[$ tel que $f'( x) = 0 \forall x \in ]a,b[$, alors $f$ est constante sur $[a,b]$
\end{crly}
\begin{proof}
	Si, par l'absurde, il existe $x,y \in [ a,b] $ avec $f( x) \neq f( y) $.\\
	Par TAF pour  $[x,y]$, il existe $z \in ]x,y[$ :
\[ 
	f'( x) = \frac{f( y) - f( x) }{y-x}\neq 0
\]
Absurde.
\end{proof}
\begin{crly}
	$f'>0$ ( sur un intervalle) implique $f$ strictement croissante.
\end{crly}
\begin{proof}
	Soit $x< y$, à voir $f( x) < f( y) $.\\
	Or, par TAF $\Rightarrow$ $\exists z \in ]x,y[$ 
	\[ 
		f'( z) = \frac{f( y) -f( x) }{y-x}> 0
	\]
\end{proof}
\begin{crly}
	$f'\geq 0$ ( sur un intervalle) $\iff$ $f$ croissante ( pas forcément strictement) 
\end{crly}
\begin{proof}
$\Leftarrow$ :\\
\[ 
	\lim_{x \to x_0} \frac{f( x) -f( x_0) }{x-x_0} \geq 0
\]
$\Rightarrow$ :\\
Par l'absurde:
\[ 
	\exists x < y \text{ tel que } f( x) > f( y) 
\]
Par TAF $\Rightarrow \exists z \in ]x,y[$ tel que
\[ 
	f'( z) = \frac{f( y) - f( x) }{y-x} < 0
\]
Ce qui est absurde.


\end{proof}





























	
















































\end{document}	
