\documentclass[../main.tex]{subfiles}
\begin{document}
\lecture{25}{Wed 09 Dec}{Theoreme Fondamental}
\begin{thm}
	Soit $f_n$ une suite de fonctions integrables sur $[a,b]$ qui converge uniformement vers $f$.\\
	Alors $f$ est integrable et 
	\[ 
	\lim_{n \to  + \infty} \int_{ a }^{ b }f_n = \int_{ a }^{ b }f
	\]
\end{thm}
Si la convergence n'est pas uniforme, alors ce theoreme est faux.\\

\begin{proof}
La convergence uniforme implique
\[ 
\forall \epsilon> 0 \exists n_0 \forall n \geq n_0
\]
\[ 
	\forall x \in [ a,b] : | f_n( x) -f( x) | < \epsilon
\]
Soit $\sigma$ une subdivision de $[a,b]$, alors
\[ 
	|m_i( f_n) - m_i( f) |\leq \epsilon
\]
de meme
\[ 
	|M_i( f_n) - M_i( f) |\leq \epsilon
\]
Donc
\[ 
	|\underline{S}_\sigma( f_n)  - \underline{S}_\sigma( f) | \leq \sum \epsilon ( x_i - x_{i-1} ) =\epsilon ( b-a) 
\]
Pour montrer que $f$ est integrable.\\
Soit $\epsilon>0$\\
On cherche $\sigma$ tel que $\overline{S}_\sigma( f) < \underline{S}_\sigma( f) +\epsilon$.\\
Appliquer l'enonce precedent a $\frac{\epsilon}{3( b-a )}$.\\
Donc $\exists n$ tel que 
\[ 
	\overline{S}_\sigma( f) \leq \overline{S}_\sigma( f_n) +\epsilon< \overline{S}_\sigma( f_n)  + 2\epsilon \leq \overline{S}_\sigma( f) + 3\epsilon
\]
Donc $f$ integrable.\\
\[ 
	\int_{ a }^{ b } f = \overline{S}_f = \lim_{n \to \infty }  \inf_\sigma \overline{S}_\sigma( f) 
\]
\subsection{Recherche de Primitives}
\begin{itemize}
\item Cas simples:
\begin{itemize}
\item si $f= \cos$, alors $F= \sin$ est une primitive.
\item Si $f( x) = x^{n}$, $F( x) = \frac{1}{n+1}x^{n+1}$ est une primitive.
\end{itemize}
\end{itemize}



\end{proof}
\subsubsection{Changement de Variable}
Idee: Utiliser $( F \circ \phi )' = F' \circ \phi \cdot \phi'$.
\begin{propo}
	Soit $f \in C^{0}( [ a,b] ) $ et $\phi: [ \alpha, \beta] \to [ a,b] $ derivable et $\phi( \alpha) = a$ et $\phi( \beta) = b$.\\
	Alors
	\[ 
		\int_{ a }^{ b }f = \int_{ \alpha }^{ \beta }( f\circ \phi) . \phi'
	\]

\end{propo}
\begin{proof}
Par le theoreme fondamental, $f$ admet une primitive $F$.\\
Alors
\[ 
F \circ \phi
\]
est une primitive de $F'\circ \phi \cdot \phi'$.\\
Donc, le theoreme fondamental pour $F' \circ \phi\cdot \phi'$ donne:
\[ 
	\int_{ \alpha }^{ \beta }F'\circ \phi \cdot \phi' = [ F\circ \phi]_{\alpha} ^{\beta} = F( b) - F( a) = \int_{ a }^{ b }f
\]

\end{proof}
\subsubsection{Integration Par Parties}
\begin{propo}
Soient $f,g$ differentiables, alors
\[ 
\int_{ a }^{ b }fg' = [ fg] _{a} ^{b}- \int_{ a }^{ b }f'g
\]

\end{propo}




\end{document}	
