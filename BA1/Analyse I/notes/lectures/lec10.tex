\documentclass[../main.tex]{subfiles}
\begin{document}
\lecture{10}{Mon 19 Oct}{fonctions}
\subsection{Continuité}
\begin{defn}
	Soit $f$ définie au voisinage de $x_0$.\\
	Alors $f$ est dite continue en $ x_0$ si 
	\[ 
		\lim_{x \to x_0} f(x) =f(x_0)
	\]
	
\end{defn}
Donc $f$ continue ( en $x_0$) si on peut ``sortir $f$ de la limite'' ( en $ x_0$ )
\begin{propo}
	$f$ continue en $ x_0 \iff$ toute suite $a_n$ tendant vers $x_0$, on a
		\[ 
			\lim_{n \to  + \infty} f(a_n) = f(x_0)
		\]
		
\end{propo}
\begin{proof}
	Théorème de traduction pour $l=f(x_0)$
\end{proof}
\begin{rmq}
	Pour parler de continuité en $x_0$, il faut que $f$ soit définie en $x_0$ et au voisinage de $ x_0$
\end{rmq}
\begin{crly}
Si $f$ et $g$ sont continues en $ x_0$, alors $f+g$ et $f\cdot g$ aussi.
\end{crly}
\begin{proof}
Idem que avant
\end{proof}

\begin{crly}
	Si de plus $g(x_0) \neq 0$, alors $\frac{f}{g}$ est cont. en $x_0$.
\end{crly}
\begin{rmq}
	On a montré que alors dans ce cas il existe un voisinage de $ x_0$ où $g(x)\neq 0$
\end{rmq}
\begin{propo}
	Soit $g$ continue en $x_0$ et $f$ continue en $g(x_0)$, alors $f\circ g$ est continue en $x_0$.
\end{propo}
\begin{proof}
Ecrivons la définition de $g$ continue en $ x_0$:
\[ 
	\forall \epsilon >0 \exists \delta >0 \forall x: \quad |x-x_0| <\delta \Rightarrow |g(x) -g(x_0)| <\epsilon
\]
Soit $\epsilon >0$. Cherche $\eta >0$ tq $\forall x$:
\[ 
	|x-x_0| < \eta \Rightarrow |f( \underbrace{g(x)}_{=y}) - f(g(x_0))| < \epsilon
\]
Continuité de $f$ en $g(x_0)$ appliquée à $\epsilon$ donne $\theta >0$ tq $\forall y$ 
\[ 
	|y-g(x_0)| \Rightarrow |f(y) - f(g(x_0))| < \epsilon
\]
continuité de $g$ en $x_0$ appliquée à $\theta$ 
\[ 
	\exists \eta >0 \forall x \quad |x-x_0| <\eta \Rightarrow |g(x) - g(x_0)|<\theta
\]
Pour $y=g(x)$ on a montré ce qu'il fallait.

\end{proof}
\begin{defn}[Terminologie Supplémentaire]
	$f$ est définie au voisinage à gauche de $x_0$ si $\exists \epsilon>0$ tq $f$ est définie sur $]x_0-\epsilon,x_0[$.\\
	De même à droite: $]x_0,x_0+\epsilon[$
\end{defn}
\begin{defn}
	Soit $f$ définie au voisinage à droite de $x_0$ 
	\[ 
	\lim_{x \to x_0>} = l 
	\]
	signifie
	\[ 
		\forall \epsilon>0 \exists \delta>0 \forall x >x_0: |x-x_0| <\delta \Rightarrow |f(x)-l|<\epsilon
	\]
	La limite à gauche est définie de la même manière.
\end{defn}



\begin{defn}
$f$ est continue à droite en $ x_0$ si
\[ 
	\lim_{x \to x_0>} f(x) = f(x_0)
\]
Idem à gauche.
\end{defn}
\begin{exo}
\[ 
	\lim_{x \to x_0} f(x) \text{ existe } \iff \text{ les limites à gauche et à droite existent et coincident. } 
\]

\end{exo}
\begin{defn}
	$f$ est continue sur $[a,b]$ si elle est continue sur $]a,b[$ et continue à droite en $a$, à gauche en $b$.
\end{defn}

\begin{defn}[Notation]
	\[ 
		\lim_{x \to x_0} f(x) = + \infty
	\]
	si
	\[ 
		\forall R \exists \delta >0 \forall x : 0 <|x-x_0| < \delta \Rightarrow f(x) >R	
	\]
	Idem pour $- \infty$

\end{defn}

\begin{defn}[Notation]
\[ 
	\lim_{x \to  + \infty} f(x) = l
\]
signifie
\[ 
	\forall \epsilon>0 \exists n_0 \forall x >n_0: |f(x)-l| < \epsilon
\]

\end{defn}
On note $C( [ a,b])$ ou parfois $C^{0}( [ a,b])$ l'ensemble des fonctions continues sur $[a,b]$

\begin{thm}
	Toute fonction continue sur $[a,b]$ est bornée.
\end{thm}
\begin{proof}
	Supposons par l'absurde $f$ non-bornée ( disons sans perte de généralité non majorée). \\
	Donc $\forall n \in \mathbb{N} \exists x_n: f(x_n) >n$.\\
	On a une suite $(x_n)_{n=1} ^{ \infty}$ de $[a,b]$\\
	Par Bolzano-Weierstrass implique qu'on a une sous-suite  $x_{n_k}$ qui converge vers $x\in [ a,b]$ \\
	$f$ continue en $x \iff f(x) = \lim_{k \to  + \infty} f(x_{ n_k })$
\end{proof}
\begin{thm}
Toute fonction $f: [ a,b] \to \mathbb{R}$ continue atteint son $\sup$ donc max.
\end{thm}
\begin{proof}
	On sait déjà que $f$ est bornée, soit donc $s:= \sup \left\{ f(x) | x\in [ a,b] \right\} $ \\
	Si par l'absurde $f(x) \neq s \forall x \in [ a,b]$ posons
	\[ 
		g(x) = \frac{1}{f(x) -s}
	\]
	$g$ est continue et donc $g$ est bornée, disons par $B$.\\
	Absurde car implique $|f(x)-s| > \frac{1}{B}$.
\end{proof}

\begin{propo}
Soient $f,g$ deux fonctions continues sur un intervalle $I$.\\
Soit $A\subseteq I$ une partie dense. Si
\[ 
f|_A = g|_A
\]
Alors $f=g$ sur tout $I$
\end{propo}
\begin{proof}
Soit $x\in I$. Par densité,
\[ 
	\exists ( a_n)
\]
suite de $A$ avec $\lim_{n \to  + \infty} a_n=x$.\\
\[ 
	\text{ Continuité } f(x) = \lim_{n \to  + \infty} f(a_n) = \lim_{n \to  + \infty} g(a_n) = g(x)
\]

\end{proof}
	





	
\end{document}	
