\documentclass[../main.tex]{subfiles}
\begin{document}
\lecture{17}{Wed 11 Nov}{BH}
\subsection{Bernoulli-L'hospital pour infini sur infini}
\begin{thm}
	Soient $f,g: ]a,b[ \to \mathbb{R}$ derivables tel que
	\[ 
		\lim_{x \to a+} f( x) = + \infty = \lim_{x \to a+} g( x) 
	\]
	Si $ \lim_{x \to a+} \frac{f'( x) }{g'( x) }$ existe, alors $ \lim_{x \to a+} \frac{f( x) }{g( x) }$ existe aussi, avec la meme valeur
\end{thm}
\begin{proof}
	Le principe est que pour calculer la limite d'une expression $A$ ( pour $x \to a$), on peut remplacer $A$ par
	\[ 
	ABC
	\]
	pour autant que $\lim_{x \to a+} BC= 1$. \\
	Soit $\epsilon > 0$. Posons $l= \lim_{x \to a+} \frac{f'( x) }{g'( x) } $ 
	A montrer: Pour $x$ suffisamment proche de $a$
	\[ 
		| \frac{f( x) }{g( x) }-l | < \epsilon
	\]

	On sait que $\exists y \forall a <c <y$ 
	\[ 
		|\frac{f'( c) }{g'( c) }-l| < \frac{\epsilon}{2}
	\]
	Par Cauchy sur $[x,y]$, on a 
	\[ 
		\forall a<x<y: | \frac{f( x) -f( y) }{g( x) -g( y) }-l| < \frac{\epsilon}{2}
	\]
	Ecrivons donc
	\[ 
		\frac{f( x) }{g( x) }= \frac{f( x) - f( y) }{g( x) -g( y) } \frac{f( x) }{f( x) - f( y) } \frac{g( x) -g( y) }{g( x) }
	\]
En effet, pour notre $y$ :
\[ 
	\lim_{x \to a+} \frac{f( x) }{f( x) -f( y) }= \lim_{x \to a+} \frac{1}{1-\frac{ g( y )}{f( x) }}=1
\]


	
	
	
\end{proof}
\section{Polynome de Taylor et developpements limites}
But: Approximer des fonctions par polynomes.\\
\begin{defn}[Polynome de Taylor]
	Soit $I$ un intervalle ouvert, $a\in I$ et $f:I \to \mathbb{R}$ $n$ fois derivable.\\
	Le polynome de Taylor de $f$ en $a$ et d'ordre $n$ est
	\[ 
		\sum_{k=0}^{ n} \frac{1}{k!} f^{( k)} ( a)  ( x-a) ^{k}
	\]
	

\end{defn}
\begin{thm}[Formule de Taylor]
Supposons $f$ $n+1$ fois derivable
\[ 
	\forall x \exists t \in ]a,x[ \text{ tel que } 
\]
\[ 
	f( x) = \sum_{k=1}^{ n} \frac{1}{k!}f^{( k) }( a) ( x-a) ^{k} + \frac{1}{( n+1) !} f^{( n+1) }( t)  ( x-a) ^{n+1}
\]


\end{thm}
\begin{rmq}
Excellent si $f$ raisonnable car $\frac{1}{( n+1 )!}$ tres petit.
\end{rmq}
\begin{proof}
	On note $T( y) $ pour le polynome de taylor.\\
	Soit
	\[ 
		g( y) = f( y) - T( y) + \frac{T( x) - f( x) }{( x-a)^{n+1}} ( y-a) ^{n+1}
	\]
	On remarque que
	\[ 
		g^{( k) }( a) = \text{ cst. } ( n+1)( n)\ldots( n+1-k) ( y-a) ^{n+1-k} = 0
	\]
	Or, $g( x) =0$.\\
	On applique donc Rolle a $g$ sur $[a,x]$
	Par Rolle, $\exists x_1 \in ]a,x[$ tel que $g'( x_1) = 0$.\\
	Rolle pour $g'$ sur $[a,x_1]$.\\
	\[ 
		\exists x_2 \in ]a,x_1[ \text{ tel que } g''( x_2) =0
	\]
	etc...\\
	Rolle pour $g^{( n) }$ sur $[a,x_n]$, donc
	\[ 
		\exists t \in ]a,x_n[ \text{ tel que } g^{( n+1) }( t) =0
	\]
Donc
\[ 
	0= g^{( n+1) }( t) = f^{( n+1) }( t) + \frac{T( x) - f( x) }{( x-a) ^{n+1}} ( n+1) !
\]
Donc 
\[ 
	-T( x) + f( x) = f^{( n+1) }( t) ( x-a) ^{n+1} \frac{1}{( n+1) !}
\]

\end{proof}






\end{document}	
