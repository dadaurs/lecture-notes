\documentclass[../main.tex]{subfiles}
\begin{document}

\lecture{1}{Mon 10 Oct}{Intro}
\section{Motivation}
Let $f( x) \in \mathbb{Z}[x]$ be a monic irreducible polynomial and a $p$ a prime.\\
Look at $f_p( x) \in \mathbb{F}_p[x]$, in general, $f_p$ is not irreducible so we can study it's factorizations.
\begin{defn}
	We say $f$ splits completely mod $p$ if $f_p$ factors into distinct linear factors.\\
	We write $Spl( f) = \left\{ p | f_p= \prod ( x-\alpha_i) \alpha_i \neq \alpha_j \forall i\neq j  \right\} $ 
\end{defn}
\underline{Problem}\\
Given $f$, describe the factorisations behaviour of $f_p$ as a function of $p$.\\
Or at least give a rule determining $Spl( f) $.\\

An answer to this illposed problem is a \textbf{Reciprocity Law}.
\begin{exemple}
Let $f( x) = x^{2}-q$ $q>2 $ prime.\\
Observe that
\begin{enumerate}
\item $f_p( x) = ( x-\alpha_p)^{2}$, but this happens iff $p=2,q$ 
\item $f_p( x) = ( x-\alpha_p) ( x+\alpha_p) $ iff $p\in Spl( f) $ iff $( \frac{q}{p}) =1$
\item $f_p( x) $ is irreducible iff $( \frac{q}{p}) =-1$ 
\end{enumerate}
\end{exemple}
To get a rule, we need to compute $\left( \frac{q}{p}\right)$, to do so, we use quadratic reciprocity.
For us, quadratic reciprocity translates to
\begin{crly}
\[ 
	( \frac{q}{p}) =
	\begin{cases}
		( \frac{p}{q})  \text{ if } p \equiv 1 \mod 4\\
-( \frac{p}{q}) \text{ if } p \equiv 3 \mod 4
	\end{cases}
\]
So $Spl( X^{2}-q) $ is determined by congruence conditions modula $4q$.
\end{crly}
\begin{exemple}
Let $\Phi_n$ be the $n$th cyclotomic polynomial, then
\[ 
Spl( \Phi_n) = \left\{ p | p \equiv 1 \mod n \right\} 
\]
\end{exemple}
What about general polynomials?\\
Over $\mathbb{C}$, we can always factor polynomials and so we write $K_f= \mathbb{Q}( \alpha_1,\ldots,\alpha_n) $ for the splitting field of $K_f$ over $ \mathbb{Q}$.\\
$K_f \supset \mathbb{Q}$ is a Galois extension and $\O = \O_{K_f} $ is it's ring of integers.\\
As $\O$ is a dedekind domain, we have
\[ 
p\O = \prod_{i=1} ^{n} \beta_i^{e}, \faktor{\O}{\beta_i} \supset \mathbb{Z}/( p) \text{ a finite extension of $ \mathbb{Z}/p$  } 
\]
We understand finite extensions of $ \mathbb{F}_p$, there Galois group is generated by the Frobenius automorphism.\\
If $p$ does not ramify ( $e_p=1 \iff p\not| D_{K_f} $ ) then we define the Artin-Symbol $\sigma_{\beta_i} \in Galf( K_f | \mathbb{Q}) $ by 
\[ 
\sigma_{\beta_i} ( \alpha) \equiv \alpha ^{p}\mod \beta_i \forall a \in \O
\]
\underline{Fact}:\\
If $\beta_i \neq \beta_j$, then there is $\zeta \in Gal( K_f| \mathbb{Q}) $ such that $\zeta( \beta_i) = \beta_j$, then $\sigma_{\beta_j} = \zeta \sigma_{\beta_i} \zeta^{-1}$.\\
The Artin symbol of $p$ is $\sigma_p = C_{\gal} ( \sigma_{\beta_i} ) $.\\
For now we suppose $Gal( K_f| \mathbb{Q}) $ is an abelian group, in this case, we can turn the Artin Symbols into a map
\[ 
	\mathbb{Q}^{*}\supset \Gamma_{D_{K_f}} = \eng{ p \not| D_{K_f} } \to \Gal( K_f| \mathbb{Q}) 
\]
by sending $p \to \sigma_p$ 

\begin{lemma}
	If $Gal( K_f| \mathbb{Q}) $ is abelian, then, up to finitely many extensions,
	\[ 
	p \in Spl( f) \iff \sigma_p=1
	\]
	
\end{lemma}
\begin{thm}[Artin Reciprocity]
	For $K_f / \mathbb{Q}$ abelian, the Artin map $\sigma: \Gamma_{D_{K_f} } \to \Gal( K_f| \mathbb{Q}) $ is surjective and it's kernel contains the "ray class group".
\end{thm}
Here the ray class group is 
\[ 
\Gamma_a ^{( ray) }= \left\{ r\in \mathbb{Q}^{*}| r= \frac{c}{d} ( ca,d) =1, c\equiv d \mod a \right\} 
\]
For a suitable $a$ tant consists of ramified primes.\\
Define $\tilde{Spl}( f)= Spl( f) \setminus \left\{ p|a \right\}  \cup \left\{ p\equiv 1 \mod a \right\}  $ .
\begin{thm}[Abelian polynomial theorem]
	If $f$ is abelian, then $\tilde { Spl} ( f) $ can be described by congruence conditions wrt a modulus depending only on $f$.\\
	Conversely, if $ \tilde{Spl} ( f) $ is described by congruence conditions, then $\gal( K_f| \mathbb{Q}) $ is abelian.
\end{thm}
\begin{thm}
	Let $f,g$ be polynomials ( monic irreducible), then
	\[ 
	K_f \subset K_g \iff Spl( g) \subset ^{\ast}Spl( f) 
	\]
\end{thm}
This enters in the proof of the converse part of the abelian polynomail theorem.
\section{Interlude: Inverse Limits}
Let $I$ be a directed ordered set  ( $i,j\in I\implies \exists k $ such that $i \leq k, j \leq k$  ) 
\begin{defn}[Inverse System]
	A inverse system consists of data
	\[ 
	\left\{ X_i, f_{i,j} | i,j \in I, i \leq j \right\} 
	\]
	$X_i$ are objects ( topological spaces, groups, etc) and the $f_{i,j} :X_j \to X_i$ such that $f_{i,i} = \id$ and $f_{j,k} \circ f_{k,i} = f_{j,i}  $ 
\end{defn}
\begin{exemple}
Take $X_i = \faktor{\mathbb{Z}}{p^{j}\mathbb{Z}}\to \faktor{\mathbb{Z}}{p^{i} \mathbb{Z}}, i \leq j $.\\
Then, the inverse limit is defined by 
\[ 
X= \invlim_{i \in I} X_i = \left\{ ( x_i) \in \prod X_i | f_{ij} ( x_j) = x_i \forall i \leq  j \right\}  \subset \prod_{i \in I} X_i
\]

\end{exemple}






\end{document}	
