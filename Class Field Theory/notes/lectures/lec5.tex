\documentclass[../main.tex]{subfiles}
\begin{document}
\lecture{5}{Mon 24 Oct}{Cohomology a la Tate}
If $F$ is a non-archimedean local field with normalized valuation $v_f$ and $ E /F$ is a finite extension of degree $n$, then $E$ is again a non-archimedean local field with respect to a unique absolute value extending $|\cdot|_F$.\\
The valuation associated to $|\cdot|_E$ is $w_E$ and
\[ 
|x|_E = |N_{E |F} ( x) |^{\frac{1}{n}}
\]
We have $ \mathbb{Z}= v_F( F^{\times}) \subset w_e( E^{\times}) \subset \frac{1}{n} \mathbb{Z} $.\\
We have an extension of residue fields $k_E /k_F$ 
\begin{defn}
	We define $e= e( E|F) = [ \omega_E( E^{\times}) : v_F( F^{\times}) ] $ and $f= f( E|F) = [ k_E : k_F] $.
\end{defn}
\begin{propo}
Let $E|F$ be a finite extension of non-archimedean local fields.\\
Then $ [ E:F] = n = e\cdot f$ 
\end{propo}
\begin{rmq}
$n \geq e \cdot f$ holds in great generality (we don't need it to be a local field) but equality needs completeness.
\end{rmq}
\begin{proof}
Let $\varpi_E$ be a generator of $p_E \subset \O_E$.\\
Choose $\omega_1,\ldots,\omega_f\in \O_E^{\times}$ such that they reduce to a basis of $k_E$ over $k_F$.\\
We claim that $ \left\{ \omega_j \varpi_E^{i}| j=1 ,\ldots, f \text{ and } i=0,\ldots,e-1 \right\} $ is linearly independent.\\
Take 
\[ 
S=\sum_{i=0}^{ e-1} \underbrace{\sum_{j=1}^{ f} a_{ij}}_{ = S_i} \omega_j \varpi_E^{i}
\]
Suppose $S=0$ with the coefficents not all zero.\\
Let $\alpha_i= \min_{j \in [ f] ,a_{ij} \neq 0} v_F(a_{ij} ) $.\\
Then notice that $\varpi_F^{-\alpha_i}S_i$ has at least one coefficient in $\O_F^{\times}$.\\
Reducing $\mod p_E$ gives a linear in $k_E$ $ \sum_{j=1}^{ f}\tilde{a_{ij} } \omega_j$ and thus at least one of the $\tilde{a_{ij} }\neq 0$.\\
Thus $S_i \neq 0$ and even more $w_E( S_i) \in v_F( F^{\times}) $.\\
Since $S=0$ there must be $ 0 \leq i,j \leq e-1$ with $i\neq j$ such that $w_E( S_i \varpi_E^{i}) = w_E( S_j \varpi_E^{j}) $.\\
Thus $w_E( \varpi_E^{i}) \in w_E( \varpi_E^{j}) + \mathbb{Z}$.\\
But this can only happen if $i=j$.\\
Now, define $M = \sum_{i=0}^{ e-1} \underbrace{ \sum_{j=1}^{ f}\O_F \omega_j }_{=N}	\varpi_E^{i}$ an $\O_F$ module.\\
We claim that $M= \O_E$.\\
We start with some observations
 $\O_E= N + \varpi_E \O_E= N + \varpi_E N + \varpi_E^{2}N + \ldots + \varpi_E^{e}\O_E = M + p_F \O_E$.\\
	Thus $\O_E= M + p_F^{v}\O_E$ for $v \geq 1$ and thus $M$ is dense in $\O_E$.\\
	But $M$ is also closed


\end{proof}
\begin{defn}
	If $E /F$ is a finite extension, then it is called unramified if $ [ k_E:k_F] =n$ (ie. $e=1$ ).\\
	A non-finite extension is called unramified if $k_E /k_F$ is separable and $E$ is a union of finite unramified extensions.
\end{defn}
\begin{propo}
Let $E /F$ and $F'/F$ be two finite extensions of non-archimedean local fields and let $E' = EF'$ . Then
\begin{itemize}
\item $E /F$ unramified, then $E' /F'$ is unramified
\item subextensions of unramified extensions are unramified
\item compositions of unramified extensions are unramified.
\end{itemize}
\end{propo}
\begin{proof}
The second and third property follow from the first one, so we only show the first one.\\
$k_E /k_F$ is generated by $ \overline{\alpha}$.\\
We can lift $\overline{\alpha}$ to $\alpha\in \O_E$ and take the minimal polynomial $f\in \O_F[x]$ for $\alpha$ over $F$.\\
We compute $ [ k_E:k_F] \leq  \deg ( \overline{f}) = \deg f= [ F( \alpha) : F] \leq [ E:F] = [ k_E:k_F]  $.\\
Thus $E= F( \alpha) $ and $\overline{f}$ is the minimal polynomial of $ \overline{\alpha}$ and $E= F'( \alpha) $.\\
Consider the minimal polynomial $g\in \O_{F'} [ x] $ of $\alpha$ over $F'$.\\
Note that $ \overline{g}$ is irreducible by Hensel's lemma (and $\overline{g}$ has no multiple roots as $\overline{g}| \overline{f}$).\\
Now $ [ k_E:k_{F'} ] \leq  [ E':F'] = \deg g = \deg \overline{g} - [ k_{F'} ( \overline{\alpha}) : k_{F'} ] \leq  [ k_{E'} : k_{F'} ] $ 
\end{proof}
\subsection*{Why are unramified extensions so nice?}
\begin{itemize}
\item If $E|F$ is unramified, then there is a morphism of galois groups $Gal( E /F) \to Gal( k_E | k_F) $ sending $\sigma \to \overline{\sigma}$ by defining $\overline{\sigma}( x+ p_E) = \sigma( x) + p_E$ and this is an isomorphism.\\
	As $\gal( k_E| k_F) $ is generated by the frobenius, sending $ \overline{x}\to \overline{x}^{q_F}$ where $\# k_F=q_F$.
\end{itemize}
\begin{defn}
	The automorphism $\phi_{E|F} \in \gal ( E|F) $ that induces the frobenius via the isomorphism above is called the frobenius automorphism.
\end{defn}
\begin{thm}
If $L \supset E \supset F$ are unramified finite extensions of $F$, we have
\[ 
\phi_{E|F} = \phi_{L|F} |_E
\]
and $\phi_{L|F} ^{[E:F]}= \phi_{L |E} $ 
\end{thm}
\subsection*{Interlude: Relevance to the classical situation}
If $L|K$ is a finite extension of algbraic number fields and $p \in \O_K$.\\
Then this prime induces a normalized valuation $v_p$ on $K$.\\
Let $k_p$ be the completion and write $p\O_L= \beta_1^{e_1}\ldots \beta_r ^{e_r}$.\\
Then, we get valuations $\omega_{\beta_i} = \frac{1}{e_i}v_{\beta_i} $ extending $v_p$.\\
Completing with respect to these different valuations, we get $L_i$'s for every $i$.\\
Then $f_j = f( L_j/K_p), e_i = e( L_i |K_p) $ and one has $ \sum_{i=1}^{ r}e_i f_i = n$.\\

If $L |K$ is a Galois extension, we obtain maps $L_\beta\to L_{\sigma\beta} $.\\
In particular, if $\sigma \beta= \sigma$, then $L_\beta\xto{\sigma} L_\beta$ defines an element in $\gal( L_\beta /K_p) $.\\
And we get a morphism $G( \beta) = \left\{ \sigma \in \gal( L/K) | \sigma\beta= \beta \right\} \to \gal( L_\beta|K_p) $.\\
If $p$ is unramified, then this is an isomorphism  and we can pull back the frobenius.

\end{document}	
