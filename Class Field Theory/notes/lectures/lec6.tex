\documentclass[../main.tex]{subfiles}
\begin{document}
\lecture{6}{Thu 27 Oct}{Cohomology of groups}
\section{Cohomology of finite groups}
Let $G$ be a finite group.\\
\begin{defn}
	A $G$-module $A$ is an abelian group on which $G$ acts
	\begin{enumerate}
	\item $1_G\cdot a =a $ 
	\item $\sigma( a+ b) = \sigma a  + \sigma b$ 
	\item $( \sigma\zeta) a = \sigma( \zeta a ) $ 
	\end{enumerate}
\end{defn}
\begin{exemple}
\begin{enumerate}
	\item $ \mathbb{Z}[G]$ 
	\item If $G= \gal( L /K) $, then $G$ acts on $L^{\times}$ and on $L$.
\end{enumerate}
\end{exemple}
The group ring has additional structure, there is a map.
\[ 
	\epsilon: \mathbb{Z}[G] \to \mathbb{Z}
\]
sending $ \sum_{\sigma \in G} n_{\sigma} \sigma\to \sum_{\sigma \in G} n_\sigma$ called augmentation.\\
We call $I_G= \ker \epsilon$ the augmentation ideal.\\
There is the norm element $N_G = \sum_{\sigma\in G} \sigma$.\\
This gives rise to a map $\mu: \mathbb{Z}\to \mathbb{Z}[G]$ sending $n \mapsto n \times N_G$.\\
As any element acts trivially oon the norm, the image of $ \mu$ is an ideal and we can form $J_G= \faktor{\mathbb{Z}[G]}{\mathbb{Z}N_G}$.\\
We get two short exact sequences
\[ 
	0 \to I_G \to \mathbb{Z}[G]\xto\epsilon \mathbb{Z}\to 0
\]

and
\[ 
0 \to \mathbb{Z}\xto\mu \mathbb{Z} \to J_G\to 0
\]
\begin{lemma}
\begin{itemize}
\item As a group, $I_G$ is the free abelian group generated by $\sigma- 1$ as $\sigma$ runs over $G\setminus 1$.
\item As a group, $J_G$ is the free abelian group generated by $\sigma \mod \mathbb{Z}N_G$ as $\sigma$ runs over $G\setminus 1$ 
\item We have $\mathbb{Z}[G] \simeq I_G \oplus \mathbb{Z}\simeq J_G \oplus \mathbb{Z}$.
\end{itemize}
\end{lemma}
\begin{proof}
	\[ 
	x= \sum_{\sigma \in G} n_{\sigma} \sigma = \sum_{1\neq \sigma\in G} n_{\sigma} ( \sigma-1) + ( \sum_{\sigma \in G} n_\sigma) 1_G = \sum_{1\neq \sigma\in G} ( n_\sigma- n_1) \sigma + n_1 N_G
	\]
\end{proof}
\begin{lemma}
$I_G= Ann( \mathbb{Z}N_G) $ and $ \mathbb{Z}N_G= Ann( I_G) $ 
\end{lemma}
\begin{proof}
	$x= \sum_{\sigma \in G} n_\sigma \sigma\in \mathbb{Z}[G]$, if $x$ is in the annihilator,
	\begin{align*}
		xN_G = \sum_{\sigma} n_\sigma \sigma N_G = \sum_{\sigma} n_\sigma N_G \implies ( \sum_{\sigma \in G} n_\sigma) =0
	\end{align*}
\end{proof}
\begin{defn}[Fixed module]
	$A^{G}= \left\{ a\in A|\sigma a = a \forall \sigma \in G \right\}  $.\\
	We also write
	\[ 
	{ } _{N_G} A = \left\{ a\in A |N_G a = 0 \right\} 
	\]
	and 
	\[ 
	I_G A = \left\{ \sum n_\sigma( \sigma a_\sigma - a_\sigma) |a_\sigma \in A, n_\sigma \in \mathbb{Z} \right\} 
	\]
\end{defn}
\begin{defn}
	If $A,B$ are two $G$-modules, then we can turn $\hom_{\mathbb{Z}} ( A,B) ( = \hom( A,B) ) $ into a $G$-module by letting
	\[ 
	\sigma f = \sigma\circ f \circ \sigma^{-1}
	\]
In particular,
\[ 
\hom_G( A,B) = \hom( A,B)^{G}
\]
\end{defn}
\begin{defn}
	If $A,B$ are as before, then $A\otimes B$ is a $G$-module by $\sigma( a\otimes b) = \sigma a \otimes \sigma b$
\end{defn}
\begin{rmq}
In general,
\[ 
	( A\otimes B)^{G}\neq A^{G}\otimes B^{G}
\]
\end{rmq}
\begin{rmq}
Given two $G$-homomorphisms $A\xto h A', B \xto g B'$, we get 
\[ 
	( h,g) : \hom( A',B) \to \hom( A,B') 
\]
by pre/post-composition and
\[ 
h\otimes g : A\otimes B \to A'\otimes B'
\]
\end{rmq}
\begin{defn}[Resolution]
	Let $G$ be a finite group. A complete \underline{free} resolution of the (trivial) $G$-module $ \mathbb{Z}$ is an exact sequence 
	\[ 
		\xleftarrow{d_{-2} } X_{-2}  \xleftarrow{d_{-1} } X_{-1} \xleftarrow{d_0} X_0 \xleftarrow{d_1} X_1\ldots
	\]
	of free $G$-modules $X_q$ such that 
	\[ 
		X_0 \xto { \epsilon } \mathbb{Z} \xto{\mu} X_{-1} 
	\]
	is exact and fits into the above exact sequence.\\
	All the maps are $G$-homomorphisms.
\end{defn}
We define the following standard resolution
\begin{itemize}
	\item $X_0=X_{-1} = \mathbb{Z}[G]$ 
	\item $X_q= \bigoplus_{( \sigma_1,\ldots,\sigma_q) \in G^{q}} \mathbb{Z}[G] \cdot ( \sigma_1,\ldots,\sigma_q) = X_{-q-1} $ 
\end{itemize}
To define our $G$-homomorphisms $d_q$, it suffices to define them on the generators.\\
We define $d_0( 1) = N_G$ and $d_1( \sigma) = \sigma-1$.\\
For $q>1$, define 
\begin{align*}
	d( q) ( \sigma_1,\ldots,\sigma_q) &= \sigma_1 ( \sigma_2,\ldots,\sigma_q)\\
					  &+ \sum_{i=1}^{ q-1}( -1) ^{i} ( \sigma_1,\ldots,\sigma_{i-1} , \sigma_{i} \sigma_{i+1} ,\ldots, \sigma_q)\\
					  &+ ( -1)^{q}( \sigma_1,\ldots,\sigma_{q-1} ) 
\end{align*}

Furthermore $d_{-1} ( 1) = \sigma_{\sigma\in G} [ \sigma^{-1}( \sigma) -( \sigma) ] $ and
\begin{align*}
	d_{-q-1} ( \sigma_1,\ldots,\sigma_q) &= \sum_{\sigma\in G}^{ } \sigma^{-1} ( \sigma,\sigma_1)\\
					     &+ \sum_{\sigma\in G}^{ } \sum_{i=1}^{ q} ( -1)^{i }( \sigma_1,\ldots,\sigma_{i-1} ,\sigma_1\sigma,\sigma^{-1},\sigma_{i+1} ,\ldots,\sigma_q)\\
					     &+ \sum_{\sigma\in G}^{ } ( -1)^{q+1 }( \sigma_1,\ldots,\sigma_q,\sigma) 
\end{align*}
\begin{lemma}
This is a complete free resolution of $ \mathbb{Z}$.
\end{lemma}
\begin{proof}
Nope.
\end{proof}
Now, we are ready to define the (Tate) cohomology groups!\\
Define $A_q= \hom_G( X_q, A) $ (and call an element $x:X_q \to A$ in $A_q$ a $q$-cochain).\\
We get a complex
\[ 
	\ldots \xto{\del_{-2} } A_{-2} \xto{\del_{-1} } A_{-1} \xto{\del_0} A_0 \xto{\del_1} A_1\to \ldots
\]
which is not necessarily exact but $\del_{q+1} \circ\del_q = 0$.\\
Now $Z_q= \ker\del_{q+1} $ are the $q$-cocycles, $R_q= \im\del_q$ are $q$-coboundaries.\\
The $q$th cohomology group of the $G$-module $A$ is the quotient $H^{q}(G,A ) = \faktor{Z_q}{R_q} $ 








\end{document}	
