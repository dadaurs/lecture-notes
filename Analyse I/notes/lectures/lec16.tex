\documentclass[../main.tex]{subfiles}
\begin{document}
\lecture{16}{Mon 09 Nov}{Lundi 09}
\begin{defn}[Fonctions Lipschitzienne]
La fonction $f$ est Lipschitz sur un intervalle $I$ si il existe $l$ tel que $\forall x, y \in I$ 
\[ 
	|f( x) -f( y) | \leq l \cdot |x-y|
\]

\end{defn}
\begin{rmq}
Si $f$ est Lipschitz, $f$ est continue, même uniformément continue sur $I$, poser $\delta = \frac{\epsilon}{l}$.\\
On dit aussi  ``L-lipschitz''.
\end{rmq}
\begin{crly}
Si $f$ est dérivable et $|f'|\leq L$ sur $I$, alors $f$ est L-lipschitz sur $I$
\end{crly}
\begin{proof}
	TAF sur $[x,y]$, donc
	\[ 
		\exists z \in ]x,y[: f( x) - f( y) = f'( z) ( x-y) 
	\]
\end{proof}
\begin{rmq}
	$f( x) = \sqrt x$ sur $[0,1]$ est uniformément continue, mais sa dérivée est non-bornée sur $]0,1[$.
\end{rmq}
\begin{crly}[Théorème de Darboux]
Soit $f$ continu en $x_0$.
Si $f$ est dérivable au voisinage de $x_0$ et si 
\[ 
	\lim_{x \to x_0} f'( x) 
\]
existe, alors $f$ est dérivable en $x_0$.

\end{crly}

\begin{proof}
	TAF sur $[x_0,x]$.\\
	Donc il existe $z \in ]x_0,x[$ tel que 
	\[ 
		f'( z)  = \frac{f( x) - f( x_0) }{x-x_0}
	\]
	Posons $l= \lim_{x \to x_0} f'( x) $, donc
	\[ 
		l = \lim_{x \to x_0} \frac{f( x) - f( x_0) }{x-x_0}
	\]
	Donc $f'( x_0) $ existe et est égal à $l$.
\end{proof}

\begin{rmq}
	Nous avons prouvé que $f'( x_0) $ existe et de plus $f'( x_0) = \lim_{x \to x_0} f'( x) $.\\
Donc la dérivée est continue.
\end{rmq}
\begin{thm}[Theoreme de Cauchy]\index{Theoreme de Cauchy}\label{thm:theoreme_de_cauchy}
	Soient $f,g : [ a,b] \to \mathbb{R}$ continue et dérivable sur $]a,b[$.\\	
	Supposns que $g'$ ne s'annule pas sur $[a,b]$, alors 
	\[ 
		\exists c \in ]a,b[ \text{ tel que } \frac{f( b) - f( a) }{g( b) - g( a) } = \frac{f'( c) }{g'( c) }
	\]
	
\end{thm}
\begin{proof}
Considérons
\[ 
	h( x)  = f( x) - f( a) - ( g(x ) - g( a)  ) \frac{f( b) - f( a) }{g( b) - g( a) }
\]
On a 
\begin{align*}
	h( a) = 0\\
	h( b) = 0
\end{align*}
Rolle ( pour $h$) implique $\exists c \in ]a,b[ $ tel que
\[ 
	h'( c) = 0
\]
Or  
\[ 
	h'( c)  = f'( c) - \frac{f( b) - f( a) }{g( b) - g( a) }g'( c) 
\]
i.e. $\frac{f'( c) }{g'( c) }= \frac{f( b) - f( a) }{g( b) - g( a) } $

\end{proof}
\subsection{Principe de Bernoulli-L'Hospital}
Idée:\\
Calculer des limites du type $\frac{0}{0}$.
\begin{thm}[Bernoulli-L'Hospital]
	Soient $f,g: ]a,b[ \to \mathbb{R}$ dérivables avec
	\[ 
		\lim_{x \to a +} f( x) = 0 = \lim_{x \to a+} g( x) 
	\]
	Supposons 
	\[ 
		\lim_{x \to a+} \frac{f'( x) }{g'( x) }	
	\]
	existe et $g'( x) \neq 0$ au voisinage à droite de $a$.\\
	Alors
	\[ 
		\lim_{x \to a+} \frac{f( x) }{g( x) }= \lim_{x \to a+} \frac{f'( x) }{g'( x) }
	\]
	
	

\end{thm}
\begin{proof}
	On étend $f$ et $g$ par continuité sur $[a,b[$ en posant $f( a) = 0 =g( a) $.\\
	Soit $a<x<b$. Appliquons Cauchy sur  $[a,x]$.\\
	Donc $\exists c \in ]a,x[$ tel que $\frac{f'( c_x) }{g'( c_x) }= \frac{f( x) - f( a) }{g( x) - g( a) } = \frac{f( x) }{g( x) } $.\\
	Par hypothèse, la limite
	\[ 
		\lim_{x \to a+} \frac{f'( x) }{g'( x) }= \lim_{x \to a+} \frac{f'( c_x) }{g'( c_x) }
	\]
	existe.
\end{proof}
\begin{thm}[BH pour l'infini]
	Soient $f,g: ]a,+ \infty [ \to \mathbb{R} $ dérivable.\\
	Supposons $\lim_{x \to  + \infty} f( x) = 0 = \lim_{x \to  + \infty} g( x)  $
	Si $ \lim_{x \to  + \infty} \frac{f'( x) }{g'( x) }$ existe, alors
	\[ 
		\lim_{x \to  + \infty} \frac{f( x) }{g( x) }= \lim_{x \to  + \infty} \frac{f'( x) }{g'( x) }
	\]
	
	
	
\end{thm}

\begin{proof}
Sans perte de géneralité $a>0$. \\
On définit $\phi, \psi: ]0,\frac{1}{a}[ \to \mathbb{R}$ par 
\[ 
	\phi( x) = f( \frac{1}{x}), \psi( x) = g( \frac{1}{x}) 
\]
BH pour $\frac{\phi}{\psi}$ sur $]0,\frac{1}{a}$ 
\[ 
	\lim_{x \to 0+} = \lim_{x \to 0+} \frac{\phi'( x) }{\psi'( x) }
\]



\end{proof}





\end{document}	
