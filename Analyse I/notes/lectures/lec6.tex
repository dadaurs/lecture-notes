\documentclass[../main.tex]{subfiles}
\begin{document}
\lecture{6}{Mon 05 Oct}{lundi}
\begin{rmq}
Ecriture decimale:
$3.1415\ldots$ ou encore $0.333\ldots$ veut dire
\[ 
3 + \frac{1}{10} + \frac{4}{100}+ \frac{1}{1000} + \frac{5}{10000} + \ldots
\]
une somme infinie de fractions.
La difference entre le $n$ ieme terme et le $n'$ ieme terme:
\[ 
\leq 10^{-n} \to 0 \Rightarrow  \text{ Cauchy } 
\]
Cette limite est une "somme infinie".
\end{rmq}
\section{Series}
But: definir les ``sommes infinies'' .\\
\begin{align*}
\to 
\begin{cases}
\text{ Existe? } \\
\text{ Valeur? } 
\end{cases}
\end{align*}
\begin{exemple}
\[ 
e = \frac{1}{0!}+ \frac{1}{1!} + \frac{1}{2!} + \ldots
\]
ou encore
\[ 
	\exp(x) = \frac{1}{0!}x^{0}+ \frac{1}{1!}x^{1} + \frac{1}{2!}x^{2} + \ldots
\]
ou
\[ 
\frac{1}{1}+ \frac{1}{2}+ \frac{1}{3}+\frac{1}{4}+\frac{1}{5}+ \ldots
\]
\end{exemple}
\begin{defn}[Serie]\index{Serie}\label{defn:serie}
	Le symbole $ \sum_{n=0}^{\infty }x_n$ représente
	\[ 
		x_0 + x_1+x_2 + \ldots \text{ et est défini par } 
	\]
	\[ 
	\sum_{n=0}^{\infty } x_n = \lim_{n \to  + \infty} \sum_{k=0}^{\infty } x_k
	\]
	
\end{defn}
\hr\\
On appelle 
\[ 
\sum_{n=0}^{\infty } x_n
\]
une série et on dit qu'elle converge/diverge lorsque la suite $s_n := x_0 + \ldots + x_n$ le fait.

\begin{crly}
	Si $ \sum_{n=0}^{\infty } x_n$ et $ \sum_{n=0}^{\infty } y_n$ existent, alors 
	\[ 
		\sum_{n=0}^{\infty } ( x_n + y_n) =  \sum_{n=0}^{\infty } x_n + \sum_{n=0}^{\infty } y_n
	\]
	
\end{crly}
\begin{proof}
\[ 
\sum_{n=0}^{\infty } x_n = \lim_{n \to  + \infty} s_n, s_n = \sum_{k=0}^{n}x_k
\]
\[ 
\sum_{n=0}^{\infty } y_n = \lim_{n \to  + \infty} t_n, t_n = \sum_{k=0}^{n}y_k
\]
Alors
\[ 
	\sum_{n=0}^{\infty } ( x_n + y_n) = \lim_{n \to  + \infty} u_n, \text{ où } 
\]
\[ 
	u_n = ( x_0+ y_0) + \ldots + ( x_n + y_n) = s_n + t_n
\]
Donc la limite 
\[ 
	\lim_{n \to  + \infty} u_n = \lim_{n \to  + \infty} ( s_n + t_n) = \sum_{n=0}^{\infty } x_n + \sum_{n=0}^{\infty } y_n
\]

\end{proof}
\begin{crly}
Pour $a \in \mathbb{R}, \sum_{n=0}^{\infty }a x_n = a \sum_{n=0}^{\infty } x_n$, si 
\[ 
\sum_{n=0}^{\infty} x_n
\]
existe.

Sans preuve.
\end{crly}
\begin{crly}
\[ 
\sum_{n=n_0}^{ify}x_n \text{ existe si } \sum_{n=0}^{ \infty} x_n
\]

existe et vaut 
\[ 
	\sum_{n=0}^{ \infty} x_n - (x_0+x_1+ \ldots + x_{n_0-1} )
\]
n
\end{crly}
\begin{crly}[Critere de Cauchy pour les séries]
	\[ 
		\sum_{n=0}^{ \infty} x_n converge \iff \forall \epsilon >0 \exists N \forall n > N: \abs{ \sum_{p=N}^{n}x_p} < \epsilon
	\]
	
	( Dans ce cas, $ \abs{ \sum_{n=N}^{ \infty} x_n } \leq \epsilon$)
\end{crly}
\begin{proof}
Appliquer Cauchy à la suite $s_n$:
\[ 
	\exists n_0 \forall n,n'> n_0: \abs{s_n -s_{n'} } < \epsilon
\]
Alors
\[ 
\abs{ \sum_{p=n'+1}^{ n}x_p } < \epsilon
\]
\begin{exemple}
Ecriture decimale,
\end{exemple}

\begin{propo}
Si 
\[ 
\sum_{n=0}^{ \infty} x_n
\]

converge, alors
\[ 
\lim_{n \to  + \infty} x_n =0
\]

\end{propo}
\begin{proof}
	Appliquer Cauchy à $ \abs{ \underbrace{s_n - s_{n-1}}_{= x_n} }$\\
	Attention, la réciproque est FAUSSE.
\end{proof}
2 Exemples\\
\begin{propo}[Serie Geometrique]
Soit $r \in \mathbb{R}$ avec $|r| <1$, alors
\[ 
\sum_{n=0}^{ \infty} r^{n}= \frac{1}{1-r} 
\]

\end{propo}
\begin{proof}
Soit
\[ 
s_n = r^{0}+ r^{1} + \ldots + r^{n} = \frac{1- r^{n+1}}{ 1 - r^{n}}
\]
Or 
 \[ 
\lim_{n \to  + \infty} r^{n+1} = 0
\]
Donc $ s_n \to \frac{1}{1-r}$.

\end{proof}

\[ 
\frac{1}{2}+ \frac{1}{4}+ \ldots  = \sum_{n=1}^{ \infty } \frac{1}{2}^{n} = \frac{1}{1-\frac{1}{2}} -1 = 1
\]
\begin{propo}[Série Harmonique]
\[ 
	\sum_{n=1}^{ \infty} \frac{1}{n} \text{ diverge ( vers } + \infty)
\]

\end{propo}
\begin{proof}
Considérons
\[ 
\frac{1}{1}+ \frac{1}{2} + \frac{1}{3} + \ldots + \frac{1}{2^{n}} + \underbrace{ \frac{1}{2^{n}+1} + \ldots + \frac{1}{2^{n+1}}}_{2^{n+1}-2^{n} = 2^{n} \text{ termes. } } + \ldots
\]
Tous ces termes sont $\geq \frac{1}{2^{n+1}}$ \\
Cette somme est:
\[ 
s_{2^{n+1}} - s_{2^{n}} \geq 2^{n} \frac{1}{2^{n+1}} = \frac{1}{2}
\]
Contredit Cauchy pour $\epsilon = \frac{1}{2}$.\\

\end{proof}
Astuce utile:\\
\[ 
\sum_{n=}^{ \infty} \frac{1}{n-1}-\frac{1}{n}= 1
\]

\begin{proof}
\[ 
s_n = 1- \frac{1}{2} + \frac{1}{2} -\frac{1}{3} + \frac{1}{3} - \frac{1}{4} + \ldots + \frac{1}{n-1} - \frac{1}{n} = 1 - \frac{1}{n}
\]
Donc ca converge.\\
C'est une série téléscopique

\end{proof}
\begin{propo}[Critère de Comparaison]
Supposons  $0 \leq x_n \leq y_n$.\\
Si  
\[ 
\sum_{n=0}^{ \infty} y_n \text{ converge, alors } \sum_{n=0}^{ \infty} x_n \text{ aussi } .
\]
\end{propo}
\begin{proof}
\[ 
s_n = x_0 + \ldots + x_n
\]
est croissante. Donc converge $\iff ( s_n)$ bornée.\\
Mais $y_0 + \ldots + y_n$ converge $\Rightarrow$ bornée et
$s_n \leq y_0 + \ldots + y_n \Rightarrow $ $(s_n)$ bornée
\end{proof}
\begin{rmq}
De plus, 
\[ 
\sum_{n=0}^{ \infty} x_n \leq \sum_{n=0}^{ \infty} y_n
\]
Si, par contre, 
\[ 
\sum_{n=0}^{ \infty} x_n \text{ diverge } \Rightarrow \sum_{n=0}^{ \infty} y_n \text{ diverge } 
\]
\end{rmq}
\begin{crly}
\[ 
\sum_{n=1}^{ \infty} \frac{1}{n^{2}}
\]
converge.

\end{crly}
\begin{proof}
	$\forall n \geq 2: \frac{1}{n^{2}} \leq \frac{1}{n(n-1)} = \frac{1}{n-1} - \frac{1}{n}$\\
	Or 
	\[ 
	\sum_{n=2}^{ \infty} \frac{1}{n-1} - \frac{1}{n} \text{ converge. } 
	\]
	Donc, par comparaison, $ \sum_{n=2}^{ \infty} \frac{1}{n^{2}}$ converge
	\[ 
\Rightarrow \sum_{n=1}^{ \infty} \frac{1}{n^{2}} \text{ converge } .
	\]
	
\end{proof}












\end{proof}









\end{document}	
