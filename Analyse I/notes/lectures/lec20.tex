\documentclass[../main.tex]{subfiles}
\begin{document}
\lecture{20}{Mon 23 Nov}{Series Entieres}
\section{Series Entieres}
But:\\
Etudier des fonctions du type
\[ 
	\sum_{n=0}^{ \infty } a_n ( x-x_0) ^{n}
\]
Motivations:\\
\begin{itemize}
\item Fonctions familiaires,utiles du type $\sin,\cos,\exp,\ldots$ 
\item polynomes de Taylor $\rightarrow$ serie de Taylor?
\end{itemize}
\begin{rmq}
	Nous travaillons surtout avec $x_0=0$, $f( x) = \sum_{n=0}^{ \infty } a_n x^{n}$ \\
	Pour retrouver le cas general:
	se ramener a $f( x-x_0) $.
\end{rmq}
\begin{defn}
On dit qu'un serie de fonctions 
\[ 
	\sum_{n=0}^{ \infty } f_n( x) 
\]
converge uniformement absolument sur  $I$ vers $f$ si
\[ 
	\sum_{n=0}^{ \infty } |f_n|
\]
converge uniformement, i.e.
\[ 
\forall \epsilon > 0 \exists n_0 \forall x \in I:
\]
\[ 
	\sum_{n=n_0}^{ \infty } | f_n( x) | < \epsilon
\]

\end{defn}
\begin{thm}
Supposons que $ \sum_{n=0}^{ \infty } a_n y^{n}$ converge pour $y\neq 0$.\\
Alors
\[ 
\sum_{n=0}^{ \infty } a_n x^{n}
\]
converge $\forall x \in ] - |y|,|y|[$.\\
De plus, pour tout $ 0< r< |y|$, $ \sum_{n=0}^{ \infty }a_n x^{n}$ converge uniformement et absolument sur $[-r,r]$
\end{thm}
\begin{proof}
Il suffit de montrer le deuxieme point.\\
Puisque $ \sum_{n=0}^{ \infty } a_n y^{n}$ converge, on sait que $( a_n y_n) \to 0$ et donc $a_n y^{n}$ est bornee. Donc $\exists B> 0 \forall n: |a_n y_n| \leq B$.\\
Fixons $0< r<|y|$ et considerons $x \in [ -r,r] $.\\
Remarque:\\
$|a_n x^{n}| = |a_n y^{n}| | \frac{x}{y}|^{n} \leq B( \frac{r}{|y|}) ^{n}$.\\
Pour convergence absolue, par comparaison, on etudie
\[ 
	\sum_{n=0}^{ \infty } B ( \frac{r}{|y|}) ^{n}
\]
Converge, car c'est une serie geometrique de raison $\frac{r}{|y|}< 1$	
\end{proof}
\begin{crly}
	$\forall 0< r < |y|$. On obtient une fonction continue $f( x) = \sum_{n=0}^{ \infty }a_n x^{n}$ de $x \in [ -r,r] $\\
\end{crly}
\begin{rmq}
	Bien que $f( y) $ existe par hypothese, l'existence et la continuite en $\pm y$ peut etre problematique.
\end{rmq}
Considerons $ \sum_{n=0}^{ \infty }a_n x^{n}$ 
\begin{defn}
Le rayon de convergence de cette serie entiere est 
\[ 
R := \sup \left\{ |y|: \sum_{n=0}^{ \infty }a_n y^{n} \text{ converge }  \right\} 
\]

\end{defn}
\begin{crly}
\[ 
	f( x)  = \sum_{n=0}^{ \infty }a_n x^{n}
\]
est une fonction continue ( et qui existe) sur $]-R,R[$.\\
De plus $ \sum_{n=0}^{ \infty }a_n x^{n}$ diverge $\forall x$ avec $|x|> R$
\end{crly}
\begin{rmq}
\begin{enumerate}
\item Pour $|x| = R$, tout peut arriver
\item Pour $ \sum a_n ( x-x_0) ^{n}$, on obtient un intervalle de convergence $]x_0-R, x_0+R[$.
\end{enumerate}
\end{rmq}
\begin{crly}
	 \[ 
		 R = \frac{1}{ \limsup_{n\to \infty } \sqrt[n]{|a_n|}}
	\]
, avec $R= + \infty $ si $\limsup =0$
\end{crly}
\begin{proof}
Critere de la racine pour $|x| < R$ :
\[ 
	\limsup \sqrt[n]{|a_n x^{n}|} = |x| \limsup \sqrt[n]{|a_n|}<1 \Rightarrow \text{ converge } 
\]
\end{proof}
\begin{thm}
	Soit $f( x) = \sum a_n x^{n}$ de rayon de convergence $R\neq 0$.\\
	\begin{enumerate}
	\item La serie $ \sum_{n=1}^{ \infty } n a_n x^{n-1}$ a le meme rayon de convergence $R$.
	\item $f$ est derivable sur $]-R,R[$ et sa derivee est $ \sum_{n=1}^{ \infty } n a_n x^{n-1}$
	\end{enumerate}
	
\end{thm}
\begin{proof}
	On fixe $0<r<R$ et on travaille sur $[-r,r]$.\\
	Pour tout $n$, appliquer Taylor a la fonction $x^{n}$ au point $x_0$.\\
	Donc $\exists t$ ( entre $x_0$ et $x$) tel que
	\[ 
		x^{n} = x_0^{n} + n x_0^{n-1} ( x-x_0) + \frac{1}{2!}n ( n-1) t ^{n-2}( x-x_0)^{2} ( x-x_0) 
	\]
On trouve
\[ 
	\frac{x^{n}-x_0^{n}}{x-x_0} = n x_0^{n-1} + \frac{1}{2}n ( n-1) t^{n-2}
\]
On s'interesse a $ \frac{f( x) - f( x_0) }{x-x_0}$
\begin{align*}
=\sum_{n=0}^{ \infty } a_n \frac{x^{n}- x_0^{n}}{x-x_0}\\
= \sum_{n=0}^{ \infty } n a_n x_0^{n-1} + \frac{1}{2}n ( n-1) ( x-x_0) \sum_{n=0}^{ \infty } t^{n-2}_{x,n} 
\end{align*}


\end{proof}
































































\end{document}	
