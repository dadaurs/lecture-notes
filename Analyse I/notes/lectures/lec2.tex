\documentclass[../main.tex]{subfiles}
\begin{document}
\lecture{2}{Wed 16 Sep}{Cours Mercredi}
\section{Exemple d'utilisation}
\begin{defn}[valeur absolue]\label{defn:valeur_absolue}
	\[ 
		\abs{x} = 
		\begin{cases}
			x \text{ si } x \geq 0\\
			-x \text{ si } x < 0\\
		\end{cases}
	\]
\end{defn}
\begin{propo}[Inegalite du triangle]
	Elle dit que
	\[ 
		\forall x,y : \abs{x + y} \leq \abs{x} + \abs{y}
	\]
\end{propo}
\begin{proof}
Cas $x,y \geq 0$ : alors $x+y \geq 0$ 
\[ 
\iff x+y \leq x + y
\]
Ce qui est toujours vrai.\\
Cas $x\geq 0 $ et $y < 0$.\\
Si $x+y \geq 0$, alors
\begin{align*}
	\iff \abs{x+y} &\leq x -y\\
	\iff x+ y &\leq x-y\\
	y &\leq -y
\end{align*}
c'est vrai car $y<0$.\\
Si $x+y < 0$, alors 
\[ 
\iff -x-y \leq x -y
\]
Donc $ -x \leq x$ vrai car $x \geq 0$ .

\end{proof}
\begin{defn}[Bornes]\index{Bornes}\label{defn:bornes}
	Terminologie: Soit $A \subseteq E$ , $E$ corps ordonne.
	\begin{itemize}
		\item Une borne superieure ( majorant) pour $A$ et un nombre $b$ tq
			\[ 
			a \leq b \forall a \in A.
			\]
		\item Une borne inferieure ( minorant) pour $A$ et un nombre $b$ tq
			\[ 
			a \geq b \forall a \in A.		
			\]
	\end{itemize}
	On dira que l'ensemble $A$ est borne si il admet une borne.
\end{defn}
\begin{axiom}[Axiome de completude]\label{axiom:axiome_de_completude}
	\[ 
	\forall A \subseteq \mathbb{R} \neq \emptyset
	\]
	et majoree $\exists s \in \mathbb{R}$ t.q
	\begin{enumerate}
		\item $s$ est un majorant pour $A$.\\
		\item $\forall$ majorant $b$ de $A$, $b \geq s$.
	\end{enumerate}
	Cet axiome finis la partie axiomatique du cours.
\end{axiom}
\begin{rmq}
\begin{enumerate}
\item $\forall s'<s \exists a \in A: a> s'$.\\
\item $s$ est unique.
\end{enumerate}
\end{rmq}
\begin{defn}[Supremum]\label{defn:supremum}
	Ce $s$ s'appelle le supremum de $A$, note $sup(A)$.	
\end{defn}
\begin{rmq}
	$\exists$ ( pour $A$ minore et $\neq \emptyset$) une borne inferieure plus grande que toutes les autres, notee $inf(A)$ ( infimum).
	\[ 
		\inf(A) = - \sup ( -A)
	\]
\end{rmq}
\begin{rmq}
	Si $\sup ( A) \in A$, on l'appelle le maximum.
\end{rmq}
\begin{rmq}
	Si $\inf ( A) \in A$, on l'appelle le minimum.
\end{rmq}
\begin{propo}
$\forall x \in \mathbb{R} \exists n \in \mathbb{N}: n \geq x$.\\
\end{propo}
\begin{proof}
Par l'absurde,\\
Alors 
\[ 
\exists x \in \mathbb{R} \forall n \in \mathbb{N}: n < x
\]
$\Rightarrow \mathbb{N}$ borne et  $\neq \emptyset \Rightarrow \exists s = \sup (\mathbb{N}) $ \\
\[ 
s - \frac{1}{2} < s \Rightarrow  \exists n \in \mathbb{N}: n > s - \frac{1}{2}
\]
$n+1 \in \mathbb{N}$ et $n+1 > s - \frac{1}{2} + 1 = s + \frac{1}{2}$ \\
donc $n+1 > s$ absurde.
\end{proof}
\begin{crly}[Propriete archimedienne]
\begin{enumerate}
	\item $\forall x \forall y > 0 \exists n \in \mathbb{N}: ny > x$.\\
	\item $\forall \epsilon > 0 \exists n \in \mathbb{N}: \frac{1}{n} < \epsilon$
\end{enumerate}
\end{crly}
\begin{proof}
	Pour 2, appliquer la proposition a $x = \frac{1}{\epsilon} \exists n \in \mathbb{N}: n > x = \frac{1}{\epsilon}$\\
	Alors
	\[ 
	\Rightarrow \epsilon > \frac{1}{n}	
	\]
	Pour montrer le 1.\\
	Considerer $\frac{x}{y}$
\end{proof}
On peut maintenant montrer que la racine de deux existe.
\begin{thm}[La racine de deux existe]\index{La racine de deux existe}\label{thm:la_racine_de_deux_existe}
	\[ 
	\exists x \in \mathbb{R}: x^{2}=2
	\]
	
\end{thm}
\begin{proof}
\[ 
	A := \{ y \vert y^{2} < 2 \}
\]
Clairement $A \neq \emptyset$ car $1 \in A$. De plus, $A$ est majore:
$2$ est une borne. (si $y>2, y^{2}>4>2 \Rightarrow y \notin A$).\\
Donc $\exists x = \sup ( A)$\\
Supposons ( par l'absurde) que $x^{2}<2$ \\
Soit $0 < \epsilon < 1, \frac{2-x^{2}}{4x}$.\\
Clairement, par hypothese $2-x^{2} > 0$ et idem pour $4x$ car $x \geq 1$.
Soit $y=x+\epsilon$, alors
\[ 
y^{2} = x^{2} + 2\epsilon + \epsilon^{2} < x^{2} + \frac{2-x^{2}}{2} + \frac{2-x^{2}}{2} = 2
\]
$\Rightarrow y \in A$  Mais $y= x + \epsilon > x$.
Absurde car $x= \sup ( A)$.
Donc $x^{2}\geq 2$.
Deuxiemement, supposons ( absurde) $x^{2}>2$.\\
Soit $0 < \epsilon < \frac{x^{2}-2}{2x}>0$.\\
Posons $b= x - \epsilon$.
\begin{align*}
b < x \Rightarrow \exists y \in A: y>b\\
\Rightarrow y^{2} > b^{2} = x^{2}-2 \epsilon x + \epsilon ^{2} > x^{2} - \underbrace{ 2\epsilon x }_{< x^{2} -2}\\
> x^{2} - ( x^{2}-2) = 2.
\end{align*}
Conclusion: $y^{2}>2$ contredit $y \in a$.\\
Donc $x^{2}=2$.
\end{proof}
\begin{rmq}
Preuve similaire:\\
 \[ 
\forall y > 0 \exists ! x>0: x^{2}=y
\]

\end{rmq}
\begin{propo}[$\mathbb{Q}$ est dense dans $\mathbb{R}$]\index{$\mathbb{Q}$ est dense dans $\mathbb{R}$}\label{propo:_mathbb_q_est_dense_dans_mathbb_r_}
	\[ 
	\forall x < y \in \mathbb{R} \exists z \in \mathbb{Q}: x <z<y
	\]
	
\end{propo}
\begin{lemma}
\[ 
	\forall x \exists n \in \mathbb{Z}: \abs{n-x} \leq \frac{1}{2}
\]
Ou encore:
\[ 
\forall x \exists [ x] \in \mathbb{Z} tq
\]
\[ 
\begin{cases}
[ x] \leq x\\
[ x] +1 > x
\end{cases}
\]
\end{lemma}
\begin{proof}
\[ 
	\exists n \in \mathbb{Z}: n>x ( Archimede).
\]
Soit $ [ x] = \inf \{ n \in \mathbb{Z}: n>x\} -1$
\end{proof}
\begin{proof}[Preuve de la densite]
Archimede: $\exists q \in \mathbb{N}: q > \frac{1}{y-x}$.\\
Donc
\begin{align*}
qy -qx > 1.\\
\Rightarrow  \exists p \in \mathbb{Z} : qx < p < qy
\end{align*}
par exemple:
\[ 
p = [ qy]
\]
si $qy \notin \mathbb{Z}$
ou bien
\[ 
p = qy -1
\]
si $qy \in \mathbb{Z}$
\end{proof}

\end{document}	
