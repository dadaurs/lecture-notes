\documentclass[../main.tex]{subfiles}
\begin{document}
\lecture{14}{Mon 02 Nov}{Derivees}
\begin{crly}
$f$ dérivable en $x_0$ implique $f$ continue en $x_0$.
\end{crly}
\begin{proof}
	$f( x) = \underbrace{f( x_0) + f'( x_0) (  x-x_0)}_{ \text{ continue pour tout $x$ } } + \underbrace{r( x)}_{ \text{ continu en $x_0$ } } $
\end{proof}
\begin{propo}
Soient $f,g$ dérivables en $x_0$ 
\begin{itemize}
	\item $f+g$ est dérivable en $x_0$ et $( f+g) ' = f' + g'$
	\item $fg$ est dérivable en $x_0$ et 
		\[ 
			( fg) ' = f'g + fg' \text{ ( règle de Leibnitz)  } 
		\]
	
	\item Si $g( x_0)  \neq 0$, alors $\frac{f}{g}$ est dérivable en $x_0$ et
		\[ 
			( \frac{f}{g}) ' = \frac{f'g- fg'}{g^{2}}
		\]
		
		
\end{itemize}

\end{propo}
\begin{proof}
\begin{itemize}
\item Somme est déja faite
\item Produit:
	\begin{align*}
		&\frac{f( x) g( x) - f( x_0) g( x_0) }{x-x_0} \\
		&=\frac{f( x) g( x) - f( x_0) g( x) }{x-x_0} + \frac{f(x_0) g( x) - f( x_0) g( x_0)   }{x-x_0}\\
		&= \underbrace{\frac{f( x) - f( x_0) }{x-x_0}}_{\to f'( x_0) 	} \underbrace{g( x)}_{\to g( x_0) }  + f( x_0) \underbrace{\frac{g( x) - g( x_0) }{x-x_0}}_{\to g'( x_0)}
	\end{align*}

\item Quotient:\\
	Il suffit d'appliquer Leibnitz à $f$ et $\frac{1}{g}$ 
	\[ 
		\left( \frac{f}{g}\right) ' = \left( f \frac{1}{g}\right) ' = f' \frac{1}{g} + f \left( \frac{1}{g}\right) ' 
	\]
	Il suffit de montrer que
	\[ 
	\left( \frac{1}{g} \right) = - \frac{g'}{g^{2}}
	\]
	
	Soit donc $g( x_0) \neq 0$.
	\[ 
		\frac{\frac{1}{g( x) - \frac{1}{g( x_0) }}}{x-x_0} = \frac{1}{g( x) g( x_0) } \frac{g( x_0) - g( x) }{x-x_0}
	\]

	

\end{itemize}

\end{proof}
\begin{exemple}
	Soit $n \in \mathbb{Z}$, $n< 0$, $f( x) = x^{n} = \frac{1}{x^{|n|}}$ 
	Donc 
	\[ 
		( x^{|n|}) ' = |n| x^{|n|-1}
	\]
	Donc, par la proposition
	\[ 
		f'( x)  = \frac{- |n| x^{|n|-1}}{x^{2|n|}}
	\]
	Or $|n| = -n$, alors
	\[ 
		f'( x)  = nx^{-n-1+ 2n}= n x^{n-1}
	\]
	
	
\end{exemple}
\begin{propo}
Donc, on trouve
\[ 
	\forall n \in \mathbb{Z}, f( x) = x^{n} \Rightarrow f'( x) = n x^{n-1}
\]
Attention, ne pas écrire $ ( x^{n}) '$
\end{propo}
\begin{thm}[Chain Rule]

\[ 
	( f \circ g) ' = ( f' \circ g ) . g'
\]
Soit $g$ dérivable en $x_0$ et $f$ en $g( x_0) $, alors $f\circ g$ esst dérivable en $x_0$ avec la formule ci-dessus.
\end{thm}
\begin{proof}
Definissons $h$ par
\[ 
	h( y) = 
	\begin{cases}
		\frac{f( y ) - f( g( x_0) ) }{y - g( x_0) } \text{  si } y \neq g( x_0) \\
		f'( g( x_0) ) \text{ si } y = g( x_0) 
	\end{cases}
\]
Alors $h$ est continue en $g( x_0) $ ( par définition de $f$ dérivable en $g( x_0) $).

On a alors
\begin{align*}
&\lim_{x \to x_0} \frac{f( g( x) ) - f( g( x_0) ) }{x-x_0}\\
&= \lim_{x \to x_0} h( g( x) ) \frac{g( x) - g( x_0)  }{x-x_0}\\
&= f'( g( x_0) ) \cdot g'( x_0)
\end{align*}

\end{proof}

\begin{thm}
	Soit $f: ]a,b[ \to ]c,d[$ bijective, continue.\\
	Si $f$ est dérivable en $x_0$ et $f'( x_0) \neq 0$, alors
	$f^{-1}$ est dérivable en $y_0= f( x_0) $ et
	\[ 
		( f^{-1} )'( y_0)  = \frac{1}{f'( x_0) }= \frac{1}{f'( f^{-1}( y_0) ) }
	\]
	
\end{thm}
\begin{proof}
\begin{align*}
	L= \frac{f^{-1}( y) - f^{-1}( y_0) }{y-y_0}\\
	= \left( \frac{f( f^{-1}( y) ) - f( f^{-1}( y_0) )  }{f^{-1}( y)- x_0} \right)^{-1}
\end{align*}
En posant $x= f^{-1}( y) $, on obtient
\begin{align*}
	\left( \frac{f( x) - f( x_0) }{x-x_0} \right)^{-1}
\end{align*}
Or, $f^{-1}$ est continue, donc quand $y \to y_0$, on a que $x\to x_0$.\\
Donc, la limite pour $y\to y_0$ de L est
\[ 
	\lim_{x \to x_0} \left( \frac{f( x) - f( x_0) }{x-x_0}\right) ^{-1} = \frac{1}{f'( x_0) }
\]



\end{proof}



	
	
	

			







































































\end{document}	

