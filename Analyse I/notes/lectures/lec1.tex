\documentclass[../main.tex]{subfiles}
\begin{document}
\lecture{1}{Mon 14 Sep}{Introduction}

\chapter{Introduction}
\section{Buts du Cours}
\textbf{\underline{Officiel:}}\\
Suites, series, fonctions, derivees, integrales , \ldots\\

\textbf{ \underline{Secrets:}}\\
Apprendre le raisonnement rigoureux\\
Creativite\\
Esprit Critique\\
Ne croyez rien tant que c'est pas prouve\\

On construit sur ce qu'on a fait, on recommence pas toujours a $0$, par rapport a d'autres domaines(lettres par exemple)

\begin{thm}[env. -400]\label{thm:env_400}
	Il n'existe aucin nombre ( fraction) x tel que $x^{2} = 2$.
\end{thm}


Ca contredit pythagore nn?

On va demontrer le theoreme.\\\sidenote{On demontre d'abord un lemme}
\begin{lemma}[Lemme]\index{Lemme}\label{lemma:lemme}
	Soit $n \in \mathbb{N}$ 
	Alors $n$ pair $\iff$  $n^{2}$ pair.
\end{lemma}
\sidenote{Il se pourrait qu'il y ait 13 valeurs pour $\sqrt{2}$}
\begin{proof}
$\Rightarrow$ Si n pair $\Rightarrow$ $n^{2}$ pair.

Hyp. $n = 2m ( m \in \mathbb{N})$ \\
Donc $n^{2} = 4m^{2}$, pair. \\

Par l'absurde, $n$ impair. $n=2k + 1 ( k \in \mathbb{N})$.\\
\[ 
	n^{2} = 4k^{2} + 4k + 1 = 2 ( 2k^{2} + 2k) + 1
\]
impair.
Donc si $n$ est impair, alors $n^{2}$ est forcement impair.
Absurde. 
\end{proof}
\begin{proof}
	Supposons par l'absurde $\exists x$ t.q. $x^{2} = 2$ et $x= \frac{a}{b} ( a,b \in \mathbb{Z}, b \neq 0)$.\\
	On peut supposer $a$ et $b$ non tous pairs.(sinon reduire).
	\[ 
	x^{2} = 2 \Rightarrow \frac{a^{2}}{b^{2}} = 2 \Rightarrow a^{2} = 2b^{2} \Rightarrow a^{2}
	\] 
	pair.\\
	Lemme: $a$ pair, i.e. $a=2n ( n \in \mathbb{N})$.
	\[ 
	a^{2}=4n^{2} = 2b^{2} \Rightarrow  2 n^{2} = b^{2}, i.e. b^{2} pair.
	\]
	Lemme: $b$ pair.\\
	Donc $a$ et $b$ sont les deux pairs, on a une contradiction.\\
	$\contra$
\end{proof}

En conclusion, le theoreme est bel et bien vrai, et contredit donc pythagore.
Donc les fractions ( $\mathbb{Q}$ ) ne suffisent pas a decrire/mesurer les longueurs geometriques.
Il faut les nombres reels, on les comprends seulement vraiment depuis 2 siecles.\\
C'est important de chercher ce genre d'erreurs.\\
Prochain but: definir les nombres reels ($\mathbb{R}$).
L'interaction entre les fractions et les nombres reels.

\chapter{Definir $\mathbb{R}$}
On commence avec la definition axiomatique des nombres reels.

\begin{axiom}[Nombres Reels]\index{Nombres Reels}\label{axiom:nombres_reels}
	$\mathbb{R}$ est un corps, en d'autres termes:\\
	Ils sont munis de deux operations: plus et fois.
	\begin{itemize}
		\item Associativite $x + ( y + z)= ( x+ y)+z ( x,y,z \in \mathbb{R})$ \sidenote{L'associativite n'est pas forcement vraie(octonions)}\\
		\item Commutativite $x+y = y+x$.\\
		\item Il existe un element neutre $0$ t.q. $0+x=x, x \in \mathbb{R}$.\sidenote{Il y a aucune difference entre les regles pour l'addition que pour la multiplication.}
		\item Distributivite $x ( yz)= ( xy)z$\\
		\item Il existe un element inverse, unique $-x \in \mathbb{R}$ t.q. $x + (-x)=0$
	\end{itemize}
\end{axiom}

Remarque: Il existe beaucoup d'autres corps que $\mathbb{R}$, par exemple $\mathbb{Q},\mathbb{C}$, $\{0,1,2\} \mod 3$\\

Attention: $ \{0,1,2,3\} \mod 4$ n'est pas un corps!\\
Presque tous marchent, ils satisfont 8 des 9 axiomes.\\
\hr\\
\begin{lemma}[Theorem name]\index{Theorem name}\label{lemma:theorem_name}
$\forall x \exists ! y$ t.q. $x+y = 0.$	
\end{lemma}
\begin{proof}
Supposons $x+y =0 = x+y'$ \\
A voir: $y=y'$.\\
 
\begin{align*}
	y &= y + 0 = y + (x + y') = ( y+x) + y'\\
	  &= ( x+y) + y' = 0 + y' = y'
\end{align*}
CQFD.
\end{proof}

\underline{\texbf{Exercice}}\\
Demontrer que $0$ est unique.

\begin{propo}[Annulation de l'element neutre]\index{Annulation de l'element neutre}\label{propo:annulation_de_l_element_neutre}
	$0\cdot x = 0$
\end{propo}
\begin{proof}
	\[ 
		x = x \cdot 1 = x(1+0) = x\cdot 1 + x \cdot 0 = x + x \cdot 0
	\]
	\[ 
		0 = x + (-x) = x + ( -x) + x \cdot 0
	\]
	$\Rightarrow$ $0=x\cdot 0$
	
\sidenote{ $a-b = a + ( -b)$}

\end{proof}

\begin{crly}[x fois moins 1 egale -x]\label{lemma:x_fois_moins_1_egale_x}
\[ 
	x + x \cdot (-1) = 0
\]
	
\end{crly}
\begin{proof}
	A voir: $x \cdot ( -1)$ satisfait les proprietes de $-x$.\\
	Or
	\[ 
		x + x(-1) = x(1-1) = x \cdot 0 =0.
	\]
	
\end{proof}
\underline{\texbf{Exercice}}\\
Montrer que $\forall x: x-(-x)=x$ et que ceci implique $(-a)(-b) = ab$.\\
\hr\\

Rien de tout ca n'a quelque chose a voir avec $\mathbb{R}$.\\
Il nous faut plus d'axiomes!!
\begin{axiom}[Nombres Reels II]\index{Nombres Reels II}\label{axiom:nombres_reels_ii}
$\mathbb{R}$ est un corps ordonne. Ce qui revient a dire que les assertions suivantes sont verifiees.
\begin{itemize}
	\item $x \leq y$ et $y \leq z$ impliquent $x \leq z$ \\
	\item $(x \leq y \intertext{et} y \leq x)$ $\Rightarrow$ $x=y$
	\item pour tout couple de nombres reels  $x$ et $y$ : ou bien $x \leq y$ ou bien $x \geq y$.
\end{itemize}
\end{axiom}

Exemple de corps ordonnnes:\\
(1) $\mathbb{R}$, (2) $\mathbb{Q}$, (3) $\{0,1,2\} \mod 3$ n'est pas un corps ordonne.

\underline{\texbf{Exercice}}\\
$x \leq y \iff -x \geq -y$ 
\underline{\texbf{Exercice}}\\
$x \leq y$ et $z \geq 0$ $\Rightarrow$ $xz \leq yz$ \\
$x \leq y $ et $z \leq 0$ $\Rightarrow$ $xz \geq yz$.\\
\hr\\
Il nous manque encore un axiome, et c'est le dernier, pour mercredi!


\end{document}	
