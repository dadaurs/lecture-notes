\documentclass[../main.tex]{subfiles}
\begin{document}
\lecture{12}{Mon 26 Oct}{Fonctions}

\begin{crly}
$\forall n \in \mathbb{N}^{*} \forall x \geq 0: \exists y \geq 0: y^{n}=x$\\
Comme ce $y$ est unique ( axiome de $<$) on peut donc définir $\sqrt[n] { x} = y $
\end{crly}
\begin{proof}
	Considérons la fonction $f: \mathbb{R}^{+} \to \mathbb{R}^{+}$, $f( y) = y^{n}$.\\
	$f$ est continu, $f( 0) = 0, \lim_{x \to  + \infty} f( x) = + \infty $.\\
	Rappel: i.e.
	\[ 
		\exists y_0 \forall y \geq y_0: f( y) \geq x
	\]
	TVI pour $ [ 0, y] $ : $\exists z $ tq $f( z) = x$.
\end{proof}
\begin{flushleft}
\textbf{Rappel}
\end{flushleft}
\begin{itemize}
\item $ax+b = 0$ adment une solution ( en $x$) si $a\neq 0$ 
\item $ax^{2} + bx + c$ admet parfois une solution
\item $ax^{3}+ bx^{2} + cx + d= 0$ admet une solution ( $a\neq 0$) 
\item degré 4 admet parfois une solution
\item degré 5: pas de formule avec ``juste'' des racines.
\end{itemize}
\begin{crly}
Tout polynôme de degré impair admet des racines.
\end{crly}
\begin{proof}
\[ 
a_n x^{n} + a_{n-1} x^{n-1} + \ldots + a_0 = 0
\]
$n$ impair, $a_n \neq 0$ \\
Or $ \lim_{x \to  + \infty} f( x) = + \infty $ ( si $a_n > 0$ $- \infty $ si $a_n < 0$) \\
En effet
\[ 
	a_n x^{n} ( 1 + \frac{a_{n-1} }{a_n} x^{-1} + \ldots ) 
\]
Donc $\exists x_1: f( x_1) > 0$ ( resp.$< 0$).\\
De même, $\lim_{x \to - \infty  } f( x) = - \infty $ ( resp $+ \infty $).\\
Donc
\[ 
	\exists x_2: f( x_2) < 0
\]
TVI sur $[x_2, x_1]$ $\Rightarrow$ $\exists x: f( x) = 0$

\end{proof}
\begin{crly}
Soit $f: [ a,b] \to \mathbb{R}$ continue.\\
$f( [ a,b ]) = [ m, M]  $ où $m = \min_{x \in [ a,b] } f( x)  $et $ M = \max_{x\in [ a,b] } f( x) $
\end{crly}

\begin{propo}[1er theoreme de la fonction implicite]
	Soit $f: [ a,b] \to \mathbb{R}$ continue, strictement monotone.\\
	Donc ( corrolaire précédent) , $f$ est bijective 
	\[ 
	f: [ a,b]  \to [ m,M] 
	\]
	i.e. $\exists f^{-1}: [ m,M] \to [ a,b] $ \\
	Alors $f^{-1}$ est continue
\end{propo}
\begin{proof}
Sans perte de géneralité, $f$ strictement croissante.\\
\begin{lemma}
Soit $g: [ m,M] \to [ a,b] $ surjective et strictement croissante.\\
Alors $g$ est continue
\end{lemma}
\begin{proof}
En $x_0$ \\
Soit $\epsilon> 0: \exists x_1: g( x_1) > g( x_0) - \epsilon$ \\
De même, $\exists x_2: g( x_2) < g( x_0)  + \epsilon$.\\
Donc sur $[x_1, x_2]$ $f$ prend des valeurs entre $g( x_0) -\epsilon$ et $g( x_0) + \epsilon$
\end{proof}
Appliquer à $g= f^{-1}$.\\
C'est surjectif, par définition du domaine de $f^{-1}$, i.e. l'image de $f$.

\end{proof}
\begin{crly}
Soit $f: [ a,b] \to [ a,b] $ continue.\\
Alors $\exists x \in [ a,b] : f( x) = x$.
\end{crly}
\begin{proof}
Considérer $g$ 
\[ 
g: [ a,b]  \to \mathbb{R}
\]
avec 
\[ 
	g( x) = x-f( x) 
\]
Donc 
\begin{align*}
g( a)  = a- f( a) \leq 0\\
g( b)  = b - f( b) \geq 0
\end{align*}
TVI $\Rightarrow \quad \exists x: g( x) = 0$ i.e. $f( x) = x$
\end{proof}
\section{Suites de Fonctions}
But: donner un sens à 
\begin{center}
``\textit{$f_n$ converge vers une fonction $f$}''
\end{center}
\begin{defn}
	$( f_n)$ converge ponctuellement vers $f$ si
	\[ 
		\forall x : \quad \lim_{n \to  + \infty} f_n( x) = f( x) 
	\]
\end{defn}
\begin{exemple}
\begin{itemize}
\item 
\begin{figure}[ht]
    \centering
    \incfig{fonction1}
    \caption{fonction1}
    \label{fig:fonction1}
\end{figure}
\item 
\begin{figure}[ht]
    \centering
    \incfig{fonction2}
    \caption{fonction2}
    \label{fig:fonction2}
\end{figure}
Ponctuellement, $f_n \to f$ où
\begin{align*}
	f( x) =
	\begin{cases}
	0 \text{ si } x<1\\
	1 \text{ si } x=1
	\end{cases}
\end{align*}
\begin{rmq}
\[ 
	\lim_{x \to 1} f_n( x) = 1
\]
On pourrait donc prendre
\[ 
	\lim_{n \to  + \infty} \lim_{x \to 1} f_n( x) = 1
\]
Par contre
\[ 
	\lim_{x \underbrace{\to}_{<} 1} \lim_{n \to  + \infty} f_n( x)  = 0
\]

\end{rmq}
Donc, attention à la continuité!
\item 
\begin{figure}[ht]
    \centering
    \incfig{fonction3}
    \caption{fonction3}
    \label{fig:fonction3}
\end{figure}
$f_n$ est continue pour tout $n$, 
\[ 
\max f_n = 1
\]
Or $f_n \to 0$ ponctuellement.
\item 
\begin{figure}[ht]
    \centering
    \incfig{fonction4}
    \caption{fonction4}
    \label{fig:fonction4}
\end{figure}
Or, à nouveau, $f_n \to f=0$

\end{itemize}
\end{exemple}
\begin{defn}[Convergence uniforme de fonctions]
	Une suite $( f_n) _{n=1} ^{ \infty }$ converge uniformément sur $A \subseteq \mathbb{R}$ sur $f$ si:
	\[ 
		\forall \epsilon > 0 \exists n_0 \forall n \geq n_0 \forall x |f_n( x) - f( x) | < \epsilon
	\]
	

\end{defn}











\end{document}	
