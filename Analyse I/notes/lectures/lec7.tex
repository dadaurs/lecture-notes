\documentclass[../main.tex]{subfiles}
\begin{document}
\lecture{7}{Wed 07 Oct}{mercredi}
\begin{defn}[Séries Alternées]
	$(x_n)$ est alternée si $x_n \cdot x_{n+1} \leq 0 \forall n$
\end{defn}
\begin{thm}
	Soit $(x_n)$ alternée, $|x_n|$ décroissante, et
	\[ 
	\lim_{n \to  + \infty} x_n =0
	\]
	Alors
	\[ 
	\sum_{n=0}^{ \infty} x_n \text{ converge. } 
	\]
	
\end{thm}
\begin{exemple}
\[ 
	\sum_{n=1}^{ \infty} \frac{(-1)^{n}}{n}
\]
converge. ( série harmonique alternée)\footnote{En fait la série converge vers $- \log 2$}
\end{exemple}
\begin{proof}
On utilise cauchy.\\
Soit $n,m \in \mathbb{N}$.
\[ 
\underbrace{x_n + x_{n+1}}_{\geq 0} + x_{n+2} + \ldots + x_{n+m-1} + x_{n+m} 
\]
Cas $x_n \geq 0$:\\
Cas où $n$ pair
\[ 
0 \leq \sum_{p=n}^{ n_m}x_p \leq x_n
\]
Si $m$ impair:\\
idem\\
Que $n$ soit pair ou impair
\[ 
	\abs{ \sum_{p=n}^{ n+m}x_p} \leq |x_n|
\]
Or, soit $\epsilon > 0$ 
\[ 
\lim_{n \to  + \infty} x_n = 0 \Rightarrow 
\]
$\exists N \forall n > N| \abs{x_n} \leq \epsilon.$

Donc $\forall n > N, m |$ 
\[ 
	\abs{x_{n} + \ldots + x_{n+m} } < \epsilon
\]
\end{proof}
\subsubsection{Un calcul naif ( avec la série harmonique alternée)}
Soit $S= \sum_{n=1}^{ \infty} \frac{(-1)^{n}}{n}$, existe par le théorème.\\
Note: $S < 0.$\\
 \[ 
	 s_n = \underbrace{-1 + \frac{1}{2}}_{= -\frac{1}{2}} \underbrace{-\frac{1}{3} + \frac{1}{4} }_{< 0}- \ldots + \frac{(-1)^{n}}{n}
\]
$s_n < -\frac{1}{n}, \forall n \text{ pair }  \Rightarrow S \leq -\frac{1}{2}$

\[ 
-1 + \frac{1}{2} - \frac{1}{3} + \frac{1}{4} - \frac{1}{5} + \frac{1}{6} + \ldots
\]
à chaque terme $x_n$, on associe $x_{ 2n }$

\begin{align*}
= -\frac{1}{2} + \frac{1}{4} - \frac{1}{6} + \frac{1}{8} - \frac{1}{10}  + \ldots\\
= \frac{1}{2} ( -1 + \frac{1}{2} - \frac{1}{3} + \frac{1}{4} - \frac{1}{5} + \ldots) = \frac{1}{2} S
\end{align*}
Donc $S = \frac{1}{2}S \Rightarrow S =0$ Faux!\\
Conclusion:\\
On ne peut pas permuter ( en général) les termes d'une série convergente ( somme infinie)
\begin{defn}
On dit que la somme de 
\[ 
\sum_{n=0}^{ \infty} x_n
\]
converge absolument si 
\[ 
\sum_{n=0}^{ \infty} | x_n|
\]
converge.\\
Note: la valeur 
\[ 
\sum_{n=0}^{ \infty} |x_n | 
\]
ne nous intéresse pas
\end{defn}
\begin{rmq}
Si $x_n \geq 0 \forall n$, aucune différence entre ``convergence'' et ``convergence absolue''.
\end{rmq}
\begin{exemple}
\begin{itemize}
\item La série harmonique alternée converge, mais pas absolument.
\end{itemize}
\begin{lemma}
Convergence absolue implique la convergence.
\end{lemma}
\begin{proof}
$\forall n: 0 \leq x_n + |x_n| \leq 2 |x_n|$\\
Donc convergence absolue  $\Rightarrow$ 
\[ 
	\sum ( x_n + |x_n|)
\]
converge.\\
Or $ -\sum_{n=0}^{ \infty} |x_n|$
converge .\\
Somme des deux sommes ci-dessus, implique que
\[ 
\sum_{n=0}^{ \infty} x_n
\]

\end{proof}
\begin{thm}
Si 
\[ 
\sum_{n=0}^{ \infty} x_n 
\]
converge absolument, alors toute permutation converge vers la même somme.
\end{thm}
\begin{exemple}
\[ 
	\sum_{n=0}^{ \infty} \frac{x^{n}}{n!}, \sum_{n=0}^{ \infty} (-1)^{n} \frac{x^{2n+1}}{(2n+1)!}
\]
\end{exemple}
Clarification:\\
Soit $\sigma$ une permutation de $\mathbb{N}$, i.e. bijection.\\
La nouvelle série sera
\[ 
	\sum_{n=0}^{ \infty} y_n \text{ pour } y_n = x_{\sigma(n)} 
\]
Notons $s_n = x_0 + \ldots + x_n$ et 
\[ 
	t_n = y_0 + \ldots + y_n = x_{\sigma(0)} + \ldots + x_{\sigma(n)} 
\]
Le théorème dit: si $ \sum_{n=0}^{ \infty} |x_n|$ existe, alors $\lim s_n = \lim t_n$.
\begin{proof}
1er cas ``facile'' .\\
Supposons $x_n \geq 0 \forall n$.\\
Alors $ \sum_{n=0}^{ \infty} x_n = \sup \left\{ s_n | n \in \mathbb{N} \right\} $\\
On va montrer que $ \underbrace{\sup_n s_n}_{=:s} \geq \underbrace{\sup_n t_n}_{=:t}$ et que $ \sup_n s_n \leq \sup_n t_n$\\
Pour $s \geq t$ : \\
Soit $\epsilon > 0$. Or , par déf, $\exists n t_n > t - \epsilon$\\
ie
\[ 
y_0 + \ldots + y_n > t- \epsilon
\]
ie
\[ 
	x_{\sigma(0)} + \ldots + x_{\sigma(n)} > t - \epsilon
\]
Soit $m = max_{i=0,\ldots, n} \sigma(i)$, alors
\[ 
s_m \geq t - \epsilon
\]
donc 
\[ 
s = \sup s_n > t - \epsilon
\]
vrai $\forall \epsilon> 0 \Rightarrow  s \geq t$\\
En considérant $\sigma ^{-1}$, on obtien de même $t \geq s \Rightarrow s =t$, donc
le théorème vrai SI $x_n \geq 0$.\\
2ème cas: $x_n \leq 0 \forall n$, idem\\
Cas général:\\
Posons $x_n = x'_n _ x''_n$, ou
$x'_n = \max(x_n,0)$ et $x''_n = \min(x_n,0)$, alors
 \[ 
	 x_{\sigma(n)}  = x'_{\sigma(n)}  + x''_{\sigma(n)}
\]
On conclut en appliquant le cas ( 1) a  $x'_n$ et ( 2) ou $x''_n$
\end{proof}
\begin{thm}
Supposons que 
\[ 
\sum_{n=0}^{ \infty} x_n
\]
converge, mais pas absolument.\\
$\forall l \in \mathbb{R} \exists $ permutation $\sigma$ t.q.
\[ 
	\sum_{n=0}^{ \infty} x_{\sigma(n)}  = l.
\]

\end{thm}


\end{exemple}








\end{document}	
