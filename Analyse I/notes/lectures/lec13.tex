\documentclass[../main.tex]{subfiles}
\begin{document}
\lecture{13}{Wed 28 Oct}{Suites de Fonctions 2}
\begin{rmq}
La convergence uniforme implique la convergence ponctuelle
\end{rmq}
\begin{propo}
	Soit $( f_n) $ une suite de fonctions qui converge uniformément.\\
	Supposons que 
	\[ 
		\lim_{n \to  + \infty} \underbrace{\lim_{x \to x_0} f_n( x)}_{=l_n} 
	\]
	existe.\\
	Alors
	\[ 
		\lim_{x \to x_0} \lim_{n \to  + \infty} f_n( x) = \lim_{n \to  + \infty} \lim_{x \to x_0} f_n( x) 
	\]
	
\end{propo}
\begin{proof}
	Soit $f$ la limite de $( f_n) $.\\
	Hyp: $\forall n: l_n = \lim_{x \to x_0} f_n( x) $existe et
	$l = \lim_{n \to  + \infty} l_n$.\\
	But: $\lim_{x \to x_0} f( x) =l$.\\
	Soit donc $\epsilon > 0$, alors
	\[ 
	\exists n_0\forall n\geq n_0: |l_n-l|< \frac{\epsilon}{3}
	\]
	De plus, par convergence uniformee
	\[ 
	\exists n_0\forall n \geq n_0
	\begin{cases}
	|l_n-l|< \frac{\epsilon}{3}\\
	\forall x: |f_n( x) -f( x) |< \frac{\epsilon}{3}
	\end{cases}
	\]
	Donc
	\[ 
		\exists \delta> 0 \forall x: 0< |x-x_0| < \delta \Rightarrow |f_{n_0} ( x) - l_{n_0} | < \frac{\epsilon}{3}
	\]
Soit $0< |x-x_0|< \delta$, on veut
\[ 
	|f( x) -l| <\epsilon
\]
Or
\[ 
	|f( x) -l| \leq |f( x) -f_{n_0} ( x) | + |f_{n_0} ( x) -l_{n_0} | + |l_{n_0} -l|< \epsilon
\]

	
\end{proof}




\begin{thm}
Toute limite uniforme de fonctions continues est continue.
\end{thm}
\begin{proof}
	Soit $f$ la limite uniforme de $( f_n) $, $f_n$ est continue $\forall n$.\\
	Soit $x_0$ avec $f_n$ définie au voisinage de $x_0$.\\
	A voir: $f$ continue en $x_0$, i.e.
	\[ 
		f( x_0) = \lim_{x \to x_0} f( x) = \lim_{x \to x_0} \lim_{n \to  + \infty} f_n( x) = \lim_{n \to  + \infty} \lim_{x \to x_0} f_n( x) = f( x_0)  
	\]
	
\end{proof}
\begin{thm}[Dini]
	Soit $( f_n) $ une suite décroissante de fonctions continues.\\
	Si $( f_n) $ converge ponctuellement vers $f$ continue, alors $f_n$ converge uniformément vers $f$ sur $[a,b]$.

\end{thm}
\begin{exo}
Trouver un contre exemple sans l'hypothèse décroissante.
\end{exo}
\begin{proof}
Par l'absurde, 
\[ 
	\exists \epsilon> 0 \forall n_0\exists n\geq n_0 \exists x_n: |f_n( x_n) - f( x_n) | \geq \epsilon
\]
Par Bolzano-Weierstrass $\Rightarrow$ 
\[ 
	( x_{n_k} ) \text{ qui converge vers  } x
\]
et tel que
\[ 
	|f_{n_k} ( x_{n_k} ) - f( x_{n_k} ) | \geq \epsilon
\]
Convergence de $f_{n} ( x) $ implique
\[ 
	\exists k: |f_{n_k} ( x) -f( x) |< \frac{\epsilon}{3}
\]
Continuité de $f_{n_k} $ et de $f$ en $x$ 
\[ 
\exists \delta> 0 \forall y: |x-y|< \delta \Rightarrow 
\begin{cases}
	|f( y) - f( x) |< \frac{\epsilon}{3}\\
	|f_{n_k} ( y) - f_{n_k} ( x)| < \frac{\epsilon}{3}
\end{cases}
\]
Choisir un $k'$ tel que $|x_{n_k'} -x|< \delta$ \\
Comme 
\begin{itemize}
	\item $|f( x) -f( x_{n_k} )| < \frac{\epsilon}{3}$ 
	\item  $|f_{n_k} ( x) - f_{n_k} ( x_{n_k'} ) |< \frac{\epsilon}{3}$
	\item $f \leq f_{n_k'} \leq f_{n_k} $
\end{itemize}
Donc 
\[ 
	|f_{n_k'} ( x_{n_k'} ) - f( x_{n_k'} ) | < \epsilon 
\]
Absurde.

\end{proof}
\section{Dérivation}
\begin{defn}
	Soit $f$ définie au voisinage de $x_0$.\\
	On dit que $f$ est dérivable en $x_0$ si
	\[ 
		\lim_{x \to x_0} \frac{f( x) -f( x_0) }{x-x_0}
	\]
	existe.\\
	Alors cette limite s'appelle la dérivée de $f$ en $x_0$, notée $f'( x_0) $.
\end{defn}
\begin{rmq}
Si $f$ est dérivable partout, alors on obtient une fonction $f'$.\\
On définit de même la dérivée gauche et droite.
\end{rmq}
\begin{propo}
Si $f$ et $g$ sont dérivables en $x_0$, alors $f+g$ aussi et
\[ 
	( f+g) ' = f'+g'
\]
\end{propo}
\begin{proof}
\[ 
	\lim_{x \to x_0} \frac{f(x) + g( x) - f( x_0) + g( x_0) }{x-x_0} = f'( x_0) + g'( x_0) 
\]

\end{proof}
\begin{propo}
	Soit $f$ définie au voisinage de $x_0$. Alors 
	\[ 
	f \text{ dérivable en  } x_0 \iff \exists a \in \mathbb{R} \exists \text{ fonction  } r \text{ au voisinage de $x_0$ tel que } 
	\]
	\begin{enumerate}
		\item $f( x) = f( x_0) + a( x-x_0) + r( x) $ 
		\item $\lim_{x \to x_0} \frac{r( x) }{x-x_0}=0$
	\end{enumerate}
	Dans ce cas, $a= f'( x_0) $
	
	
\end{propo}











\end{document}	
