\documentclass[../main.tex]{subfiles}
\begin{document}
\lecture{3}{Wed 06 Oct}{Reparametrage}
\subsection{Reparametrage d'une courbe}
On veut formaliser la notion que deux courbes $ \alpha, \beta$ de $ \mathbb{R}^{n}$ representent la "meme" courbe geometrique.\\
On veut $ \alpha( u) = \beta( t) $ avec $u = h( t)  ( =u(t) ) $.\\
Plus precisement, si $\alpha: I \to \mathbb{R}^n$ et $ \beta : J \to \mathbb{R}^n$ ( $u \in I$ , $t \in J$ ), alors $\alpha$ est un reparametrage si il existe un diffeomorphisme $ h : J \to I$ , $t \mapsto u = u( t) =h( t) $, tel que $\alpha= \beta \circ h$ .\\
\begin{rmq}
La condition que deux courbes sont un reparametrage l'une de l'autre est une relation d'equivalence et une classe d'equivalence est une courbe geometrique
\end{rmq}
\begin{defn}[Quantite]
	Une quantite ou une propriete d'une courbe est geometrique si elle est invariante par reparametrage.\\
	Sinon la quantite est dite cinematique.
\end{defn}
\begin{exemple}
\begin{enumerate}
\item La trace d'une courbe est une propriete geometrique
\item La notion de regularite, biregularite sont geometriques
\item Le plan osculateur est une notion geometrique.
\item La longueur d'une courbe est geometrique.
\end{enumerate}
\end{exemple}
\begin{proof}
	On suppose $\alpha: I \to \mathbb{R}^n, \beta: J \to \mathbb{R}^n$, $\alpha( u) = \beta( t) , t=h( u) $ .\\
	On a 
	\[ 
		l_\alpha = \int_I V_\alpha ( u)  du, l_{\beta} = \int_J V_{\beta} ( t) dt
	\]
	avec $ V_{\alpha} = \N { \frac{d\alpha}{du}} , V_{\beta} = \N { \frac{d \beta}{du} \frac{du}{dt}} = | \frac{du}{dt}| V_{\alpha} ( u)  $.\\
	Donc $V_{\beta} ( t) dt = \pm V_{\alpha} ( u) du$ et donc $l_{\beta} = l_{\alpha} $ .
\end{proof}
En general, si $S_{\beta} ( t) $ est une quantite geometrique, alors $ \frac{d}{dt} S_{\beta} $ n'est en general pas geometriuqe, mais $ \frac{1}{V_{\beta} ( t) } \frac{d}{dt}S_{\beta} $ 
\begin{proof}
	On a $S_{\beta} ( t) = S_{\alpha} ( u) $ et $ \frac{1}{V_{\beta} ( t) } \frac{d}{dt}= \frac{1}{V_{\alpha} ( u) } \frac{d}{du}$ 
\end{proof}
\begin{exemple}
	Le vecteur unitaire tangent $ \vec{T}_{\beta} ( t) $ est une quantite geometrique.	
\end{exemple}
\begin{proof}
	On a $T_{\beta} ( t) = \frac{ \dot { \beta} ( t) }{\N { \dot{ \beta }( t) } } =\frac{1}{V_{\beta} ( t) } \frac{d\beta}{dt}$ .\\
	Ainsi, $T_{\alpha} ( u) = \frac{1}{V_{\alpha} ( u) } \frac{d \alpha}{du}$ 
\end{proof}
\begin{defn}[Derivation naturelle]
	On definit 
	\[ 
		\frac{1}{V_{\beta} ( t) } \frac{d}{dt}
	\]
	comme etant la derivation naturelle le long de la courbe.
\end{defn}
\subsection*{Contreexemples}
La vitesse, le vecteur vitesse et l'acceleration sont des quantites cinematiques.
\begin{defn}
	On dit que $ V_{\alpha} ( u) du$ est la differentielle naturelle le long de la courbe
\end{defn}
\begin{exemple}
\begin{enumerate}
\item Masse d'un fil metalique inhomogene.\\
	La quantite utile est la densite lineaire de masse $\rho : I \to \mathbb{R}_+ $.\\
	La masse sera alors $M = \int_I \rho( u) V_{\alpha} ( u) du$ 
\item Centre de gravite\\
	\[ 
		G=  \frac{1}{M}\int_I \alpha( u) \rho( u)  V_{\alpha} ( u) du
	\]
	
\end{enumerate}

\end{exemple}
\begin{defn}[Abscisse Curviligne]
	Soit $\alpha: I \to \mathbb{R}^n$ une courbe reguliere et $u_0 \in I$.\\
	L'abscisse curviligne ou parametre naturel de $\alpha$ par rapport au point initial $\alpha( u_0) $ est la fonction
	\[ 
	S = S_{\alpha} : I \to \mathbb{R} 	
	\]
	definie par 
	\[ 
		S= \int_{ u_0 }^{ u } V_{\alpha} ( \zeta) d\zeta
	\]
	
\end{defn}
On dit que $\alpha$ est parametree naturellement si $S_{\alpha} ( u) =u \iff V_{\alpha} ( u) = 1$ 

\begin{propo}
Toute courbe $C^{1}$ reguliere peut se reparmetriser natruellement.
\end{propo}
\begin{proof}
Soit $\alpha: I \to \mathbb{R}^n$, $C^{1}$ reguliere et $u_0 \in I$.\\
On pose
\[ 
	s= s( u) = \int_{ u_0 }^{ u } V_{\alpha} ( u) du
\]
Alors la fonction $s$ definit un diffeomorphisme
\[ 
s: I \to J
\]


\end{proof}
\section{Courbure d'une courbe}
Soit $\gamma: I \to \mathbb{R}^n$ une courbe parametree reguliere de classe $C^{2}$.\\
Le vecteur de courbure est le champ le long de $\gamma$ 
\[ 
	\vec{K}_{\gamma} ( u)  = \frac{1}{V_{\gamma} ( u) } \dot{ \vec{T} }_{\gamma} ( u) 
\]
La courbure de $\gamma$ est alors la fonction
\[ 
	k_{\gamma} = \N {  \vec{K}_{\gamma} ( u) } 
\]
\begin{rmq}
Si $\gamma$ est parametree naturellement, alors
\[ 
	k_\gamma( u) = \N { \frac{d^{2}\gamma}{du^{2}}} 
\]

\end{rmq}
\begin{rmq}
Le vecteur de courbure et la courbure sont des quantites geometriques.
\end{rmq}
\begin{propo} [ Formule de l'acceleration] 
Soit $\gamma: I \to \mathbb{R}^n$ une courbe de classe $C^{2}$, alors son acceleration est
\[ 
	\ddot { \gamma} = \dot { V} _\gamma( t) + V^{2}_\gamma( t) \vec{K}_{\gamma} ( t) 
\]

\end{propo}
\begin{proof}
On a 
\[ 
	\dot { \gamma} ( t) = V_{\gamma} ( t) \vec{T}_\gamma( t) 
\]
Donc
\[ 
	\ddot { \gamma} = \dot { V_{\gamma} } ( t) \vec{T}_\gamma( t) + V_\gamma( t) \dot { \vec{T}} ( t) = V' K + V^{2}K
\]


\end{proof}
\begin{rmq}
On a toujours $\vec{k}\perp \vec{T}$ 
\end{rmq}
\begin{defn}
	Soit $\gamma: I \to \mathbb{R}^3$ bireguliere de classe $C^{3}$.\\
	On definit le repere mobile de Frenet de $\gamma$ est le repere  $ \left\{ \gamma( t) , T, N_\gamma( t) , B_{\gamma} ( t)  \right\} $ ou 
	\[ 
		T_\gamma( t) = \frac{\dot \gamma}{V_{\gamma} ( t) }, \quad N_\gamma( t) = \frac{\ddot { \gamma} - \eng { \dot \gamma, T} T}{ \N { \ddot { \gamma} - \eng { \dot \gamma, T} T} }\quad B = T\times N
	\]
		
\end{defn}
\begin{defn}[Torsion]
	La torsion de $\gamma$ est
	\[ 
		\tau_{\gamma} ( t) = \frac{1}{V_{\gamma} ( t) }\eng { \dot{B}, N} 
	\]
	
\end{defn}
\begin{thm}[Formules de Serret-Frenet]
	\begin{align*}
		\begin{cases}
		\frac{1}{V_\gamma} \dot{ T }_\gamma = \kappa_\gamma N \\
			 \frac{1}{V_{\gamma} }\dot N = - \kappa_\gamma T_\gamma + \tau_\gamma B_\gamma\\
			 \frac{1}{V_\gamma} \dot{B} = -\tau_\gamma N_\gamma
		\end{cases}
	\end{align*}
	
\end{thm}
\begin{proof}
\begin{enumerate}
\item Par definition, du vecteur de courbure.
\item 
\begin{align*}
	\frac{1}{V} N' = \frac{1}{V} \left(  \eng { N',T} T + \eng { N', B} B \right) 
	\intertext{or}
	\eng { N', T} = - \eng { N, T'} \\
	\eng { N', B} =0\\
	\eng { N',B} = V_\gamma
	\intertext{ De meme}
	B' = \eng { B', T} T + \eng { B',N} N + \eng { B',B} B
\end{align*}


\end{enumerate}

\end{proof}

		



	

	






\end{document}	
