\documentclass[../main.tex]{subfiles}
\begin{document}
\lecture{2}{Wed 29 Sep}{Courbes}
\section{Geometrie des courbes}
Une courbe peut etre concue comme:
\begin{itemize}
\item Le lieu des points geometriques qui satisfont a une certaine contrainte/condition
\item La trajectoire d'un point qui se deplace dans le plan ou l'espace.
\item Une courbe peut etre engendree par un mechanisme
\item Une courbe peut correspondre a un phenome optique.
\end{itemize}
Le premier point de vue va conduire a une description implicite de la courbe par une equation dans $\mathbb{R}^{2}$ ou deux equations dans l'espace.
\begin{defn}
	Une courbe algebrique dans le plan est un ensemble du type $ \Gamma: \left\{ ( x,y) \in \mathbb{R}^{2}| f( x,y) =0 \right\} $.\\
	La courbe est algebrique si $f\in \mathbb{R}[x,y]$ 
	
\end{defn}
\begin{defn}[Courbe parametrique]
	Une courbe parametrique dans $ \mathbb{R}^n$ est une application continue:
	\[ 
	\gamma: I \to \mathbb{R}^{n}
	\]
	avec $I \subset \mathbb{R}$ est un intervalle, $u\in I$ est le parametre.\\
\end{defn}	
	L'image de $\gamma$ est la trace de $\gamma$ 
\begin{defn}
	\begin{itemize}
		\item La courbe $\alpha$ est de classe $C^{k} ( k \geq 0) $ si $\alpha: I \to \mathbb{R}^n$ est de classe $C^{k}$ par $\alpha( u) = ( \alpha_1( u) , \ldots, \alpha_n( u) ) $ et $\alpha_j:I\to \mathbb{R}$ est de classe $C^{k}$ .
		\item Si $\alpha$ est $C^{1}$ et $u_0\in I$ , le vecteur vitesse est 
			\[ 
				\dot { \alpha} ( u_0) = \frac{d \alpha}{du}( u_0) 
			\]
			L'acceleration sera $\ddot { \alpha} ( u_0) = \frac{d^{2}\alpha}{d u}$ 	

		\item La droite tangente a $\gamma$ en $u_0$ est la droite par $\alpha( u_0) $ et de direction $\dot{ \alpha }( u_0) $ 
			\[ 
				T_{\alpha,u_0} : \lambda \mapsto \alpha( u_0) + \lambda \dot{\alpha} ( u_0) 
			\]
		
		\item La vitesse de $\alpha$ en $u_0$ est $V_\alpha( u_0) $ ( en supposant $\alpha$ differentiable en $u_0$ ) 
		\item Le point $\alpha( u_0) $ est regulier si $\dot{\alpha}( u_0) $ et singulier si $\dot{\alpha}( u_0) $ 
		\item Le point $\alpha( u_2) $  est biregulier si $\alpha\in C^{2}$ et $\dot{\alpha}( u_0), \ddot { \alpha} ( u_0) $	sont lineairement independants .
		\item Si $\alpha$ est bireguliere en $u_0$ , le plan par $\alpha( u_0) $ en direction $\dot { \alpha} ( u_0) , \ddot { \alpha} ( u_0) $ est le plan osculateur de $\alpha$ en $u_0$.
	\end{itemize}
	
\end{defn}
\subsection{Exemples de courbes parametrees}
\begin{itemize}
\item La cubique 
	\[ 
		\alpha( u) = ( au,b u^{2}, c u^{3}) 
	\]

\item 
	\[ 
		\beta( u) = ( u^{2}, \ldots, u^{n+1}) 
	\]
	
\item La droite en parametrage affine, par $p$ et $q$ est
	\[ 
		\gamma( t) = p + t( q-p) 
	\]
	
\item Le cercle $C$ de centre $p \in \mathbb{R}^n$ dans un plan ( affine) $\Pi \subset \mathbb{R}^n$ de rayon $r$  se parametrise
	\[ 
		C( t) = p + r \left( \cos( \omega t)b_1 + \sin ( \omega t) b_2\right) 
	\]
	ou $ \left\{ p,b_1,b_2 \right\} $ est une repere affine orthonorme de $\Pi$.
\item L'helice circulaire droite est 
	\[ 
		\gamma( u) = ( a \cos( u) , a \sin( u) , bu) 
	\]
	
\end{itemize}
\begin{defn}[Longueur d'une courbe]
La longueur d'une courbe $\gamma: [ a,b] \to \mathbb{R}^n$ de classe $C^{1}$ est
\[ 
	l( \gamma) = \int_{ a }^{ b } V_\gamma( u) du	
\]
 							
\end{defn}
\begin{propo}
La longueur verifie les proprietes suivantes:
\begin{itemize}
	\item Additivite: Si $\gamma: [ a,b] \to \mathbb{R}^n$ est une courbe $C^{1}$, alors $ l( \gamma|_{[a,c]}  l( \gamma|_{[ c,b]} ) l( \gamma_{ [ a,b] } )   $ 
\item La longueur est invariante par isometrie.\\

\item Pour $f$ une similitude de rapport $\lambda >0$, alors
	\[ 
		l( f\circ \gamma) = \lambda l( \gamma) 
	\]

\item 
	\[ 
		l( \gamma_{[a,b]} ) \geq d( \gamma( a) , \gamma( b) ) 	
	\]
	avec egalite si et seulement si $\gamma $est le segment $ [ \gamma( a) ,\gamma( b) ] $.
\end{itemize}

\end{propo}
\begin{proof}
\begin{itemize}
\item Suit de 
	\[ 
		l( \gamma_{[a,b]} ) = \int_{ a }^{ b } V_\gamma( u) du = \int_{ a }^{ c }V_\gamma( u) du + \int_{ c }^{ b } V_\gamma( u)  du
	\]

\item On sait que $f( x) = \lambda Ax +b$ , $A$ orthogonal, donc pour $\tilde { \gamma} ( u) = f( \gamma( u) ) $ 
	\begin{align*}
		\tilde { \gamma'} = \lambda A \gamma' ( u) \\
		l( \tilde { \gamma} ) = \int_{ a }^{ b }V_{\tilde { \gamma} } ( u) du = \lambda l( \gamma) 
	\end{align*}

\item Soit $p= \gamma( a) , q = \gamma( b) $.\\
	On note $w = \frac{q-p}{\N { q-p} }$ et on definit
	\[ 
		g: [ a,b] \to \mathbb{R}\quad g( u) = \eng { \gamma( u) - p, w} 
	\]
Alors
\[ 
	\frac{dg}{du} = \eng { \dot { \gamma} ( u) , w} \leq \N { \dot { \gamma} ( u) } \N { w} = V_{\gamma} ( u) 
\]
Ainsi,
\[ 
	\int_{ a }^{ b } \frac{dg}{du} du = g( b) - g( a) = \eng { q-p,w} = q-p
\]

\end{itemize}
\end{proof}
\subsection{Champs de vecteurs le long d'une courbe}
\begin{defn}[Champ vectoriel]
	Un champ de vecteurs le long d'une courbe $\gamma: I \to \mathbb{R}^{n}$ est la donnee $\forall u \in I$ d'un vecteur $W( u) =\sum_j  w_j( u) e_j $ .\\
	Ce champ est de classe $C^{k}$ si $w_j: I \to \mathbb{R}$ est de classe $C^{k}$.
\end{defn}
\begin{defn}[Le vecteur tangent]
	Si $\gamma$ est reguliere on definit
	\[ 
		T_\gamma ( u) = \frac{\dot { \alpha} ( u) }{V_{\gamma} ( u) }
	\]
Si $\gamma$ est bireguliere, alors le champ normal principal est donne par
\[ 
N_{\gamma} ( u) = \frac{\ddot { \alpha} ( u) - \eng { \ddot { \alpha} ( u) , t} t}{\N { \ddot { \alpha} ( u) - \eng { \ddot { \alpha} ( u) , t} t}}
\]
		
\end{defn}
\begin{propo}[Regle de Leibniz]
\begin{itemize}
\item 
	\[ 
		\frac{d}{du}\eng { Z( u) , W( u) } = \eng { \dot { Z} ( u) ,   W ( u) }  + \eng { Z( u) , \dot { W} ( u) } 
	\]
	
\end{itemize}

\end{propo}
\begin{crly}
\begin{itemize}
	\item Si $\eng { Z( u) , W( u) } = c$ , alors
\[ 
	\eng { \dot { W} ,Z } = - \eng { W, \dot{Z}} 
\]
\item Si $\N { w} =c$ $\Rightarrow$ $\eng { \dot w, w} =0$ 

\end{itemize}

\end{crly}
	
		
		
	
			
\end{document}	
