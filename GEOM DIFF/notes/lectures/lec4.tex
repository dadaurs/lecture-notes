\documentclass[../main.tex]{subfiles}
\begin{document}
\lecture{4}{Wed 13 Oct}{...}
\begin{thm}
La courbure de $\gamma$ est la variation naturelle de la direction de $\gamma$ 
\end{thm}
\begin{proof}
Soit $\gamma: I \to \mathbb{R}^n$ une courbe $C^{2}$ que l'on suppose parametree naturellement.\\
On fixe $p= \gamma( s_0) $ et on note 
\[ 
	\phi( s) = \phi_{s_0} ( s) = \hat { ( T_{\gamma} ( s) ,T_{\gamma} ( s_0) )  }
\]
On a par trigonometrie elementaire que
\[ 
	\N { T( s) -T( s_0) } = 2 \sin ( \frac{\phi(s ) }{2}) 
\]
Donc
\begin{align*}
	\lim_{s \to s_0+} \frac{\phi( s) -\phi( s_0) }{s-s_0}&= \lim_{s \to s_0} \frac{\phi( s) }{2 \sin ( \phi( s) /2) } \frac{2 \sin( \frac{\phi( s) }{2}) }{s-s_0}\\
							     &= \lim_{s \to s_0} \frac{\phi( s) }{2 \sin( \frac{\phi( s) }{2}) } \frac{\N { T( s) -T( s_0) } }{s-s_0}\\
							     &= \lim_{s \to s_0 +} \N { \frac{T( s) -T( s_0) }{s-s_0}} 	= \kappa( s_0) 
\end{align*}
Donc on prouve que $\frac{d}{ds}\vert_{s_0+} \phi( s) =\kappa( s_0) $
\subsection{Contact entre deux courbes}
\begin{defn}[Contact de courbes]
	Soit $\alpha,\beta: I \to \mathbb{R}^n$ deux courbes $C^{k}$.\\
	Ces deux courbes ont un contact d'ordre $k$ en $t_0\in I$ si
	\[ 
		\alpha( t_0) =\beta( t_0) \text{ et } \frac{d^{n}}{dt^{n}} \alpha( t) = \frac{d^{n}}{dt^{n}\beta( t) }
	\]
	
\end{defn}
\begin{thm}
$\alpha$ et $\beta$ ont un contact d'ordre 2 en $t_0$ $\iff$ 
\[ 
	\alpha( t_0) = \beta( t_0), T_{\alpha} ( t) = T_{\beta} ( t) , V_\alpha( t_0) = V_{\beta } ( t_0) 
\]
et
\[ 
	\kappa_{\alpha} ( t_0) = \kappa_\beta( t_0) 
\]
\end{thm}
\begin{defn}[Cercle osculateur]
	Soit $\gamma:I\to \mathbb{R}^n$ bireguliere et $u_0\in I$.\\
	On appelle cercle osculateur de $\gamma$ en $u_0$ le cercle contenu dans le plan osculateur de $\gamma$ et de centre et rayon
	\[ 
		p= \gamma( u_0) + \rho( u_0) \mathcal{N}_\gamma( u_0) 
	\]
	et rayon $\rho( u_0) = \frac{1}{\kappa_{\gamma} ( u_0) }$ 
	
\end{defn}
\begin{propo}
Le cercle osculateur est l'unique cercle qui a un contact d'ordre 2.
\end{propo}
\section{Repere de Frenet}
\begin{defn}[Reguliere au sens de Frenet]
	La courbe $\gamma$ est reguliere au sens de Frenet si $\gamma \in C^{2}$ et $\gamma$ est bireguliere et $u \to N_{\gamma} ( u) $ est $C^{1}$.
\end{defn}
\begin{rmq}
	\begin{itemize}
	\item Si $\gamma$ est bireguliere et $C^{3}$, alors $\gamma$ est Frenet-reguliere.

	\item Si $\gamma$ est frenet reguliere, alors $T_{\gamma} $ et $B_\gamma$ sont de classe $C^{1}$ .		
	\end{itemize}
	
\end{rmq}
Si $\gamma$ est Frenet-reguliere, alors la torsion $\tau_{\gamma} : I \to \mathbb{R}$ est
\[ 
	\tau_{\gamma} = \frac{1}{V_{\gamma} ( u) }\eng { \dot N, B_{\gamma} } 
\]
\begin{propo}
$\gamma$ est une courbe plane $\iff$ $\tau_{\gamma} =0$ 
\end{propo}
\begin{proof}
Si $\gamma$ est plane, alors le plan osculateur est constant $\iff$  $B_{\gamma} $ est constant $\Rightarrow$ $\dot { B_{\gamma} } =0 \Rightarrow \tau_{\gamma} =0$ .\\
Supposons que $\tau_{\gamma} =0$.\\
Soit $p=\gamma( u_0) \in \mathbb{R}^{3}$.\\
On definit
\[ 
	h( u) = \eng { \gamma( u) -p, B_{\gamma} } 
\]
Notons que $\dot { B_{\gamma} }= - \tau_{\gamma} N =0 $ est constant.\\
Donc
\[ 
	\frac{dh}{du}= \eng {\dot { \gamma} ( u) , B} = V_{\gamma} \eng { T_{\gamma} , B} =0
\]
Donc $h$ est constant et donc $\eng { \gamma( u) , B} = \eng { p,B} \forall u \in I$ qui est l'equation d'un plan.	
\end{proof}
\begin{propo}
La torsion mesure la variation angulaire du plan osculateur, c'est a dire que si 
\[ 
	\theta( s) =\hat { ( B_{\gamma} ( s) ,B_{\gamma } ( s_0) ) } 
\]

\end{propo}
\begin{proof}
Alors
\[ 
	\frac{d}{ds}\vert_{s_0} \theta( s) = | \tau_{\gamma} ( s_0) |
\]
On applique la meme preuve que Serret-Frenet.

\end{proof}
\begin{defn}[Courbe a pente constante]
	Une courbe $\gamma:I\to \mathbb{R}^{3}$ de classe $C^{1}$ est dite de pente constante si $\dot{\gamma}( u) $ fait un angle constant avec une direction fixe.
\end{defn}
\begin{propo}
Soit $\gamma:I\to \mathbb{R}^{3}$ Frenet reguliere, alors $\gamma$ est de pente constante $\iff$ 
\[ 
\frac{\tau}{\kappa}= \text{ cste. } 
\]

\end{propo}
\begin{proof}
	On suppose que $\gamma$ est parametree naturellement et $\eng { T_{\gamma} ( s) , A} =a = \text{ constante } $.\\
	\[ 
0 = \frac{d}{ds} \eng { T,A} = \eng { \dot T, A} = \kappa \eng { N,A} 	
	\]
	or $\kappa\neq 0$ donc $\eng { N,A} = 0$ ce qui implique que $b= \eng { B,A} $.\\
	On a donc
	\begin{align*}
	0 = \frac{d}{ds} \eng { N,A} = \eng { \kappa T- \tau B, A} \\
	\iff \kappa \eng { T,A} = \tau \eng { B,A} \\
	\iff \frac{\tau}{\kappa}= \text{ constante } 	
	\end{align*}
Supposons donc que $\frac{\tau}{\kappa}= \text{ constante } $.\\
On pose $\lambda = \frac{\tau}{\kappa}$ et $A= \lambda T + B$, alors

\begin{align*}
	\eng { T,A} = \lambda \eng { T,T} + \eng { B,T} = \lambda = \text{ constant } 
\end{align*}
Verifions que $A$ est constant, car
\[ 
\frac{dA}{ds}= \lambda \dot T + \dot B = \lambda \kappa N - \tau N = 0
\]
\end{proof}
\subsection{Theoreme fondamental des courbes de $ \mathbb{R}^{3}$ }
Etant donne deux fonctions continues sur l'intervalle $I$ $\kappa,\tau: I \to \mathbb{R}$ avec $\kappa( s) >0$.\\
Alors il existe une courbe $\gamma: I \to \mathbb{R}^{3}$ Frenet reguliere telle que sa courbure et sa torsion sont donnees par $\kappa$ et $\tau$ ie. $\kappa( s) = \kappa_\gamma( s) , \tau( s) = \tau_\gamma( s) \forall s \in I$.\\
Cette courbe est unique a un deplacement pres.		
\begin{proof}
On prouve d'abord l'unicite.\\
On suppose que $\gamma_1,\gamma_{2}: I \to \mathbb{R}^{3}$ 
sont deux courbes Frenet regulieres, de vitesse $1$ tel que $\delta_{\gamma_1} = \delta_{\gamma_2} =\delta$ et $\kappa_{\gamma_1} =\kappa_{\gamma_2} = \kappa$.\\
\hr
Quitte a appliquer une translation et une rotation a $\gamma_2$, on peut supposer que $\gamma_1( 0) =\gamma_2( 0) $.\\
\[ 
	T_1( 0) =T_2( 0) , N_1( 0) = N_2( 0) , B_1( 0) =B_2( 0) 
\]
On note $F_i( s) \in SO( 3) $ la matrice dont les colonnes sont $T_i, N_i, B_i$.\\
Alors on calcule
\[ 
	\frac{dF_i}{ds}=F_i( s) \Omega( s) 
\]
avec
\[ 
	\Omega( s) = 
	\begin{pmatrix}
		0 & - \kappa & 0\\
		\kappa & 0 & -\tau\\
		0 & \tau & 0
	\end{pmatrix} 
\]
Cette equation matricielle est equivalente aux equations de Serret-Frenet.\\
On calcule
\begin{align*}
	\frac{d}{ds}( F_i( s) F_2( s)^{-1}) &= \frac{d}{ds}( F_1 F_2^{T}) \\
					    &= \dot F_1 F_2^{T} +F_1\dot F_2^{T}\\
					    &= ( F_1\Omega) F_2^{T}+ F_1 ( F_2\Omega) ^{T}\\
					    &= F_1 \Omega F_2^{T}+ F_1 \Omega^{T}F_2\\
					    &= 0
\end{align*}
Or $F_1( 0) \cdot F_2( 0) ^{-1}=\id$ et donc $F_1( s) = F_2( s) \forall s \in I$.\\
Donc $T_1( s) = T_2( s) \forall s \in I$.
Donc $\gamma_1'= \gamma_2'$ et donc $\gamma_1=\gamma_2$.
\subsubsection*{Existence}
Sont donnes $\kappa, \tau: I \to \mathbb{R}$, on veut construire $\gamma$.\\
Le theoreme de Cauchy-Lipschitz sur les edo donne l'existence d'une solution au probleme de Cauchy
\[ 
	\frac{dF}{ds}= F( s) \Omega( s) , F( 0) =\id
\]
On affirme que $F( s) \in SO( 3) \forall s$.\\
En effet, on a $F( 0) F( 0) ^{T}= \id$.\\
\begin{align*}
	\frac{d}{ds}F( s) F( s) ^{T}&= \dot F F^{T}+ F \dot F^{T}\\
				    &= F\Omega F^{T}- F\Omega^{T}F^{T}=0
\end{align*}

Et donc $F( s)\in O( 3)  $ et $F( s)\in SO( 3)  $ car l'application $\det$ est continue.\\
On pose donc $\gamma( s) = \int_{ 0 }^{ s }T( u) du$ 


\end{proof}

	
	




\end{proof}
\end{document}	
