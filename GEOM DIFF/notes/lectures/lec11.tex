\documentclass[../main.tex]{subfiles}
\begin{document}
\lecture{11}{Wed 01 Dec}{Courbure }
\begin{thm}
	Toute surface reguliere de classe $C^{2}, S \subset \mathbb{R}^{3}$ admet un parametrage conforme ( local) au voisinage de tout point.
\end{thm}
\begin{defn}[Pseudosphere]
	La pseudosphere est la surface de revolution d'une tractrice autour de son asymptote
\end{defn}
Mise en equation:\\
On suppose que l'asymptote est l'axe $Oz$, que $c=1$ et que $\N { \dot\gamma} =1$ et $\gamma( 0) = ( 1,0) $.\\
On trouve l'equation $\dot x + x = 0 \implies x= e^{-t} $, et
\[ 
\dot x ^{2} + \dot x ^{2} =1\\
z( t) = \int_{ 0 }^{t } \sqrt{1- e^{-z} } dz
\]

donc la pseudosphere admet la parametrisation
\[ 
\psi: [ 0,2\pi] \times [ 0, \infty ) \to P \subset \mathbb{R}^{3} 
\]
avec
\[ 
\psi( \theta,t) = ( \cos\theta e^{-t}, \sin\theta e^{-t} ,f( t) ) 
\]
Le tenseur metrique va valoir
\[ 
G= \begin{pmatrix}
	e^{-2t} & 0\\
	0 & 1
\end{pmatrix} 
\]
On peut maintenant poser $v= e^{t} , u=\theta$, alors
\[ 
ds^{2} = e^{-2t} ( du^{2}+ dv^{2}) = \frac{du^{2}+ dv^{2}}{v^{2}}
\]

\subsection{Geodesiques et courbure des courbes sur une surface}
\begin{defn}[Geodesique]
	Une geodesique d'une surface reguliere $S \subset \mathbb{R}^{3}$ est une courbe $\gamma:I\to S$ de classe $C^{2}$ verifiant 
	\[ 
	\ddot\gamma \perp T_{\gamma} S
	\]

\end{defn}
\begin{rmq}
Si $S$ est une surface de revolution, $\gamma$ est geodesique si et seulement si $\ddot\gamma \times\nabla f$ 
\end{rmq}
\begin{propo}
Toute geodesique est parcourue a vitesse constante.
\end{propo}
\begin{thm}
	Soit $\gamma: [ a,b] \to S$ une courbe de classe $C^{2}$ parametree a vitesse constante telle que $\gamma$ minimise la distance entre $p= \gamma( a) $ et $q= \gamma( b) $.\\
C'est a dire
\[ 
l( \gamma) = d_s( p,q) 
\]
Alors $\gamma$ est geodesique.
\end{thm}
\begin{defn}
	Soit $\gamma:I\to S$ une courbe de classe $C^{2}$ reguliere.\\
	On note $\nu( t) = \nu( \gamma( t) ) $ le vecteur unitaire normal a $T_{\gamma( t) } S$.\\
	\[ 
	T_{\gamma} ( t) = \frac{1}{V}\dot\gamma
	\]
	et
	\[ 
	\mu( t) =\nu( t) \times T_{\gamma} ( t) 
	\]
	On dit que $ \left\{ \nu( t) ,T_\gamma( t),\mu( t)  \right\} $ est le repere de Darboux de la courbe $\gamma$ relatif a $S$.
\end{defn}
\begin{rmq}
\begin{itemize}
\item Le repere est orthonorme et direct
\item $T,\mu$ sont tangents a $S$ et forment une base orthonormee du plan tangent.
\end{itemize}

\end{rmq}
\begin{defn}[Courbure normale]
	On appelle courbure normale de $\gamma$ en $t$ 
	\[ 
k_n( t) = \langle K,\nu\rangle	
	\]
et on appelle la courbure geodesique
\[ 
k_\gamma = \langle K, \mu\rangle
\]

\end{defn}
\begin{thm}[de Meusnier]
	La courbure normale d'une courbe $C^{2}$ tracee sur une surface $S$, la courbure normale ne depend que de la direction de $\dot\gamma$
\end{thm}
\begin{proof}
On a $\gamma:I\to S, $ $C^{2}$, reguliere.\\
On pose $f:I\to \mathbb{R}, f( t) = \langle \nu,\dot\gamma\rangle$ On derive $f:$ 
\[ 
0 = \frac{d}{dt}f = \langle \dot\nu,t\rangle + \langle \nu, \dot T\rangle
\]
On a donc
\[ 
0 = \frac{1}{V} \langle \dot\nu,\dot\gamma\rangle + V \langle \nu, K\rangle
\]
Et donc
\[ 
k_{\gamma} = - \frac{1}{V^{2}}\langle \dot\nu,\dot\gamma\rangle
\]
On se rappelle que $\nu$ est la restriction a $\gamma$ de l'application de Gauss
\[ 
S\to S^{2}
\]
et $\dot\nu= \frac{d}{dt}\nu( \gamma( t) )= d\nu_{\gamma( t) } ( \dot\gamma)  $ 

Donc
\[ 
k_\gamma = - \frac{\langle d\nu_{\gamma( t) ( \gamma) , \dot\gamma\rangle} }{\N { \dot\gamma} }
\]


\end{proof}
On peut donc definir la courbure normale de $S$ en un point $p$
\[ 
k_n: T_p S \to \mathbb{R}: v \mapsto \frac{- \langle d\nu_p( v),v}{\N { V} ^{2}}
\]
\begin{defn}
	L'application de Weingarten d'une surface $S$ reguliere de classe $C^{2}$ est la differentielle de l'application de Gauss.
On la note $L_p: T_p S\to T_p S: L_p( v) = d\nu_p( v) $.\\
$L_p$ est un endomorphisme de $T_pS$ car de facon generale 
\[ 
	\nu: S \to S^{2}
\]
Alors 
\[ 
d\nu_p : T_pS \to T_{\nu( p) } S^{2}
\]
Or $\forall q \in S^{2}$ on a $T_qS = q^{\perp}= \left\{ v \in \mathbb{R}^{3}| \langle v,q\rangle = 0  \right\} $ 
Donc
\[ 
T_pS = \left\{ v| \langle \nu( p) , v\rangle = 0 \right\} = T_{\nu( p) S^{2}} 
\]

\end{defn}
\begin{defn}[Deuxieme forme fondamentale]
	La deuxieme forme fondamentale de $S$ en $p$ est la forme bilineaire
	\[ 
	h( v,w) = - \langle L_pv,w\rangle, v,w \in T_pS
	\]
		
\end{defn}
\begin{propo}
$h$ est une forme bilineaire symmetrique et de facon equivalente
$L_p$ est autoadjointe.
\end{propo}
\begin{proof}
On va montrer que 
\[ 
h( b_i, b_j) = h( b_j,b_i) 
\]
pour $b_i, b_j$ une base de $T_pS$.\\
On se donne une parametrisation au voisinage de $p\in S$.\\
On a donc
\[ 
\psi:\Omega\to S
\]
et on pose $b_1= \frac{\del\psi}{\del u}, b_2= \frac{\del\psi}{\del v}$.\\
$h_p( b_i, b_j) = - \langle d\nu_p( b_i) , b_j\rangle$.\\
Or $d\nu_p( b_i) = \frac{\del}{\del u_i}( \nu\circ\psi( u)  ) $.\\
Donc
\[ 
h_p( b_i, b_j) = - \langle \frac{\del}{\del u_i}kl( \nu\circ\psi( u) ), \frac{\del\psi}{\del u_j}
\]
Or $\langle\nu\psi, \frac{\del\psi}{\del u_j}\rangle =0$ 

Donc
\[ 
0 = \frac{\del}{\del u_i}\langle \nu\circ\psi, \frac{\del\psi}{\del u_j}= \langle\nu\circ\psi, \frac{\del^{2}\psi}{\del u_j\del u_i}\rangle + \del \frac{\del}{\del u_i}( \nu\circ\psi) , \frac{\del\psi}{\del u_j}
\]
Et donc
\[ 
h( b_i, b_j) = \langle \nu\circ\psi, \frac{\del ^{2}\psi}{\del u_i\del u_j} = h( b_j, b_i) 
\]
\end{proof}
Donc par le theoreme spectral:
\begin{crly}
$L_p$ est orthogonalement diagonalisable et on notes les valeurs propres $k_1, k_2$ 
\end{crly}
De plus $k_1, k_2$ sont les valeurs maximales et minimales de la courbure normale.


	



\end{document}	
