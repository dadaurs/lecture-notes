\documentclass[../main.tex]{subfiles}
\begin{document}
\lecture{3}{Wed 23 Sep}{Oscillateurs Harmoniques}
\section{Oscillateurs Harmoniques}
Considerer des systemes ayant des mouvements oscillatoires.\\
Exemples:
\begin{itemize}
	\item masse pendue a un ressort.
	\item pendule simple, pendule de torsion.
	\item vibrations.
	\item Resonateurs quartz ( montres)
	\item oscillations du champ
	\item etc...
\end{itemize}
\begin{rmq}
	Un mouvement oscillatoire permet de mesurer un intervalle de temps.
\end{rmq}
\subsection{Modelisation de la force d'un ressort}
La force exercee par un ressort est proportionelle a son deplaccement par rapport a sa position de repos.

Force de rappel:
\[ 
\vec{F}= - k \vec{\Delta x}
\]
$k=$ constante elastique du ressort  [N/m]
\begin{figure}[H]
    \centering
    \incfig{ressort}
    \caption{ressort}
    \label{fig:ressort}
\end{figure}
\begin{rmq}
Ce modele n'est que valable pour des petits allongements
\end{rmq}

\subsection{Oscillateurs harmoniques a une dimension}
\begin{figure}[ht]
    \centering
    \incfig{ressort-plan-horizontal}
    \caption{Ressort plan horizontal}
    \label{fig:ressort-plan-horizontal}
\end{figure}
\begin{align*}
	\text{ Loi de Hooke } F_x &= -kx\\
	\text{ 2eme loi de Newton } F &=ma
\end{align*}
On arrive a 
\[ 
	m\ddot{x}= -kx
\]
But: connaissant $k,m$ et les conditions initiales, determiner $x(t)$ pour tout temps $t$.
\begin{exemple}
Posons $m=1kg, k=1 \frac{N}{m}=1 \frac{kg}{s^{2}}$\\
Conditions initiales: $x(0)=1m, v(0)=0 \frac{m}{s}$ 
\[ 
	\Rightarrow a(0) = \frac{F(0)}{m} = k \frac{x(0)}{m} = -1 \frac{m}{s^{2}}
\]

Accroissement de $v$ entre $t$ et $t+\Delta t$ :
$\Delta v = a(t) \Delta t$ car $a(t)=dv(t) / dt$ \\
\[ 
	\Rightarrow v(t+\Delta t)= v(t)+a(t) \Delta t
\]
Accroissement de $x$ entre $t$ et $t+\Delta t$ :
\[ 
	\Rightarrow x(t+\Delta t)=x(t) +v(t) \Delta t
\]
Verification analytique:\\
On pose $x(t)= \cos(\omega_0 t) \Rightarrow x(0)=1$\\
$v(t)= \frac{dx}{dt}=-\omega_0\sin(\omega_0 t) \Rightarrow  v(0)=0$.\\
$a(t)= \frac{dv}{dt} - \omega_0^{2} \cos(\omega_0 t) = - \omega_0^{2}x(t)$

Comme $a(t)=-\frac{k}{m}x(t)$, on doit avoir:
\[ 
	\omega_0= \sqrt{\frac{k}{m}}
\]
C'est la pulsation propre de l'oscillateur libre.

\textbf{Solution generale et dependance par rapport aux conditions initiales}\\

Periode:
\[ 
T = \frac{2\pi}{\omega_0}
\]
Frequence
\[ 
\nu = \frac{1}{T} = \frac{\omega_0}{2\pi}
\]

\end{exemple}
Solution generale de $\ddot{x}= \omega_0^{2}x =0$:
\[ 
	x(t) = A\cos(\omega_0 t) + B \sin(\omega_0t )
\]
ou 
\[ 
	x(t) = C \sin(\omega_0t + D)
\]
Deux constantes d'integration a determiner en utilisant les conditions initiales

\[ 
A=x_0
\]
et 
\[ 
B= \frac{v_0}{\omega_0}
\]
ou bien
$ x_0^{2}=x_0^{2}+ (\frac{v_0}{\omega_0})^{2}$ et $\tan(D)= \omega_0\frac{x_0}{v_0}$

Resolution de l'equation differentielle:

\begin{align*}
	x(t) &= A\cos(\omega_0 t) + B \sin(\omega_0t)\\
x(0) &= A \cdot 1 + B \cdot 0 = A = x_0\\
\dot{x}(0) &= -A \omega_0 \sin(0) + B \omega_0 \cos(0)\\
	   &= B \omega_0 = v_0 \Rightarrow B = \frac{v_0}{\omega_0}
\end{align*}
\hr\\
\begin{align*}
	x(t) &= C \sin(\omega_0t+ D), x_0,v_0\\
	x(0)&= C \sin(D) = x_0\\
	\dot{x}(0) &= C\omega_0\cos(D)=v_0\\
	\frac{1}{\omega_0} \tan(D) &= \frac{x_0}{v_0}\\
				   &\Rightarrow \tan(D) = \omega_0 \frac{x_0}{v_0}\\
				   C^{2}(\sin^{2}(D) + \cos^{2}(D)) = x_0^{2} + \frac{v_0^{2}}{\omega_0^{2}}\\
				   &\Rightarrow C= \sqrt{x_0^{2} + \frac{v_0^{2}}{\omega_0^{2}}}
\end{align*}
\subsection{Oscillateur harmonique amorti}
\begin{figure}[H]
    \centering
    \incfig{oscillateur-amorti}
    \caption{oscillateur amorti}
    \label{fig:oscillateur-amorti}
\end{figure}
Par $b$ on definira la force de frottement.\\
Deuxieme loi de Newton: $F+F_{frot}=ma$, alors
\[ 
	m\ddot{x} = -kx - b\dot{x}
\]

\textbf{Resolution}\\

\[ 
	m\ddot{x} = -kx - b\dot{x} \Rightarrow \ddot{x} + 2\gamma \dot{x} + \omega_0^{2}=0 \text{ avec } \gamma=\frac{b}{2m} \text{ et } \omega_0=\sqrt{\frac{k}{m}}
\]
\begin{figure}[ht]
    \centering
    \incfig{types-d'ammortissement}
    \caption{types d'ammortissement}
    \label{fig:types-d'ammortissement}
\end{figure}



\end{document}	
