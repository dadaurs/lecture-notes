\documentclass[../main.tex]{subfiles}
\begin{document}
\lecture{4}{Wed 30 Sep}{Dynamique du point materiel}
\section{Dynamique du point materiel}

Notions abordees:
\begin{itemize}
\item reperes, rappels d'analyse vectorielle
\item referentiel, position, vitesse, acceleration normale et tangentielle
\item rotations, repere en rotation, mouvement circulaire uniforme
\item vitesse et acceleration en coordonnees cylindriques et spheriques
\item contraintes et forces de liaison
\end{itemize}
\begin{defn}[Referentiel]
	Un ensemble de $N$ points $(N \geq 4)$, non coplanaires, immobiles les uns par rapport aux autres.
\end{defn}
\begin{itemize}
\item La description du mouvement d'un systeme se fait toujours par rapport a un certain referentiel.
\item L'observateur et les appareils de mesure sont immobiles par rapport au referentiel ( ils ``font partie'' du referentiel)
\item Le choice du referentie du referentiel est arbitraire
\end{itemize}

\begin{defn}[Repere]
Origine $O$ et trois axes orthogonaux definis par des vecteurs de longueur unite
\end{defn}
\begin{figure}[ht]
    \centering
    \incfig{reperes}
    \caption{reperes}
    \label{fig:reperes}
\end{figure}
Vecteurs unitaires
\[ 
	\abs{\hat{x}_1}= \abs{\hat{x}_2}= \abs{\hat{x}_3} = 1
\]
Vecteurs orthonormaux
\[ 
\hat{x}_1 \cdot \hat{x}_2 = \hat{x}_2 \cdot \hat{x}_3 = \hat{x}_3 \cdot \hat{x}_1 = 0
\]

Base orthonormee
\begin{align*}
\hat{x}_i \cdot \hat{x}_j = \delta_{ij} = 
\begin{cases}
1 \text{ si } i =j\\
0 \text{ si } i \neq j
\end{cases}
\end{align*}
\subsection{Produit scaleaire}
\begin{defn}[Produit scalaire]\index{Produit scalaire}\label{defn:produit_scalaire}
	\begin{align*}
		\vec{a} \cdot \vec{b} &= \abs{\vec{a}} \abs{\vec{b}} \cos \theta\\
	\end{align*}
\begin{figure}[ht]
    \centering
    \incfig{produit-scalaire}
    \caption{produit scalaire}
    \label{fig:produit-scalaire}
\end{figure}
	
En composantes
\[ 
	\vec{a} \cdot \vec{b} = ( a_1 \hat{x}_1 + a_2 \hat{x}_2 + a_3 \hat{x}_3) \cdot \ldots = a_1b_1 + a_2b_2 +a_3 b_3
\]
Proprietes
\begin{itemize}
\item Commutativite
\item Distributivite
\item ...
\end{itemize}


\end{defn}



\subsection{Projections et composantes d'un vecteur}
Projection de $\vec{OP}$ sur l'axe u:
\[ 
	\vec{OP} \cdot \hat{u} = \abs{\vec{OP}} \abs{\vec{u}} \cos \theta = OP \cos \theta = OP'
\]
\begin{align*}
\vec{OP} = \vec{OP'} \hat{u} + \vec{OP''} \vec{v} = \vec{OP}\cdot \hat{u} \hat{u} + \vec{OP} \hat{v} \hat{v}
\end{align*}
Coordonnees cartesiennes du point $P$ ou composantes du vecteur $\vec{OP}$
\begin{align*}
\begin{cases}
x = \vec{OP} \cdot \hat{x}\\
y = \vec{OP} \cdot \hat{y}\\
z = \vec{OP} \cdot \hat{z}
\end{cases}
\end{align*}
Donc
\[ 
\vec{OP} = x \hat{x} + y \hat{y} + z \hat{z}
\]
\subsection{Repere direct}
Par convention, on n'utilise que des reperes dont la chiralite est definie par la ``regle du tire bouchon'' ou la ``la regle de la main droite'' .\\
\begin{figure}[H]
    \centering
    \incfig{repere-droit}
    \caption{repere droit}
    \label{fig:repere-droit}
\end{figure}
\subsection{Produit vectoriel}
\begin{defn}[Produit vectoriel]\index{Produit vectoriel}\label{defn:produit_vectoriel}
\begin{figure}[ht]
    \centering
    \incfig{produit-vectoriel}
    \caption{produit vectoriel}
    \label{fig:produit-vectoriel}
\end{figure}
En composantes
\[ 
\vec{a} \land \vec{b} = 
\begin{pmatrix}
a_2b_3 = a_3b_2\\
a_3 b_1 - a_1b_3\\
a_1b_2 - a_2b_1
\end{pmatrix}
\]
\end{defn}
Proprieteres
\begin{itemize}
\item $\vec{a} \land \vec{b} = - \vec{b} \land \vec{a}$ 
\item $\vec{a} \land ( \vec{b} + \vec{c})  = \vec{a} \land \vec{b} + \vec{a} \land \vec{c}$
\item $\vec{a} \land ( \lambda \vec{b}) = \lambda(\vec{a} \land \vec{b}$ 
\item $\vec{a} \land \vec{b} = \vec{0}$ si $\vec{a}, \vec{b}$ paralleles
\end{itemize}
%\subsection{Produit mixte et double produit vectoriel}
\begin{defn}[Produit Mixte]\label{defn:produit_mixte}
	\[ 
		( \vec{a} \land \vec{b}) \cdot \vec{c} = \det (\vec{c}, \vec{a}, \vec{b})
	\]
		
\end{defn}
Proprietes:
\begin{align*}
	( \vec{a} \land \vec{b}) \cdot \vec{c} = ( \vec{c} \land \vec{a}) \cdot \vec{b} = ( \vec{b}\land \vec{c}) \cdot \vec{a}\\
	( \vec{a} \land \vec{b}) \cdot \vec{c} = 0 \iff \vec{a}, \vec{b} \text{ et } \vec{c} \text{ coplanaires ( dans le meme plan) } 
\end{align*}
\begin{defn}[Double produit vectoriel]\index{Double produit vectoriel}\label{defn:double_produit_vectoriel}
	\begin{align*}
		\vec{a} \land ( \vec{b} \land \vec{c}) = ( \vec{a} \cdot \vec{c}) \vec{b} = ( \vec{a} \cdot \vec{b}) \vec{c}\\
	\end{align*}
\end{defn}
%\subsection{Cinematique}
\begin{figure}[H]
    \centering
    \incfig{trajectoire}
    \caption{trajectoire}
    \label{fig:trajectoire}
\end{figure}

\begin{figure}[H]
    \centering
    \incfig{curviligne}
    \caption{curviligne}
    \label{fig:curviligne}
\end{figure}

\begin{figure}[H]
    \centering
    \incfig{curviligne2}
    \caption{curviligne2}
    \label{fig:curviligne2}
\end{figure}

Donc 
\[ 
	\hat{\tau} \cdot \frac{d \hat{\tau}}{dt} = \frac{1}{2} \frac{d}{dt}( \hat{\tau}^{2}) = 0 \text{ car  } \tau ^{2} = 1 \forall t
\]
On peut approximer le mouvement localement par un cercle
\begin{figure}[H]
    \centering
    \incfig{mouvement-approxime-par-cercle}
    \caption{mouvement approxime par cercle}
    \label{fig:mouvement-approxime-par-cercle}
\end{figure}
\[ 
	\vec{a}_n(t) = v(t) \frac{d \tau}{dt} = v(t) \frac{d \tau}{ds} \cdot \frac{ds}{dt} = v(t)^{2} \cdot \frac{d \tau}{ds}
\]
On peut calculer la norme de $a_n$ 
\begin{align*}
	\abs{\vec{a}_n ( t)} = a_n(t) &= v^{2} \abs{ \frac{d \tau}{ds}} )\\
				     &= v^{2} \frac{d \theta}{R(t) d \theta}\\
				     &= \frac{v^{2}(t)}{R(t)}
\end{align*}

\subsection{Mouvement avec vitesse scalaire constante}
Considerons un point materiel avec
une vitesse scalaire $v = \frac{ds}{dt}$ constante non nulle\\
un vecteur vitesse qui change de direction au cours du temps\\
Acceleration
\[ 
	\vec{a} \cdot \vec{v} = \frac{d \vec{v}}{dt} \cdot \vec{v} = \frac{1}{2} \frac{d}{dt}(\vec{v}^{2}) = 0
\]
pas de composante tangentielle: $a_t = \frac{dv}{dt} = 0$
Donc $a$ est toujours perpendiculaire a $v$.\\
Force $\vec{F} = m \vec{a}$\\
Donc 
\[ 
F = ma = m \frac{v^{2}}{R}
\]
force centripete
\subsection{Systeme de coordonnees}
\begin{defn}[Systeme de coordonnees]\index{Systeme de coordonnees}\label{defn:systeme_de_coordonnees}
	Parametrisation , a un certain temps $t$, de la position des points du referentiel au mouen de trois nombres reels.
\end{defn}
Pour un referentiel donne il e xiste une infinite de systemes de coordonnes
Exemples
\begin{itemize}
	\item Coordonnes cartesiennes $(x,y,z)$ 
	\item Coordonnees cylindriques $(\rho, \phi, z)$ 
	\item Coordonnes spheriques $(r, \Theta, \phi)$
\end{itemize}
Chaque vecteur du repere est parallele a la variation de la position due a une modification de la variable correspondante





\end{document}
