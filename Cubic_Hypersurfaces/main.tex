\documentclass[a4paper,hidelinks]{article}
\usepackage[utf8]{inputenc}
\usepackage[T1]{fontenc}
\usepackage{textcomp}
\usepackage{hyperref}
\usepackage{amsmath}
\usepackage{bookmark}
\usepackage{float}
\usepackage{graphicx}
\usepackage{mdframed}
\usepackage[french]{babel}
\usepackage{bbm} %RCQ
\usepackage{verbatim}
\usepackage{varioref}
\usepackage{theoremref}
\usepackage{tikz}
\usepackage{listings}
\usepackage{faktor}
\usepackage{python}
\usepackage{xurl}
\usepackage{xcolor}
\usepackage[amsmath,hyperref,thmmarks]{ntheorem}
\usepackage[fixed]{fontawesome5}
\usepackage{imakeidx}
\makeindex
% figure support
\usepackage{import}
\usepackage{xifthen}
\usepackage{pdfpages}
\usepackage{transparent}
\usepackage{thmtools}
\usepackage{amssymb}
\usepackage{aligned-overset}
%\usepackage{fancyhdr}
\usepackage{stmaryrd} % for \lightning
%\pagestyle{fancy}
\definecolor{background}{HTML}{FFFFFF}
\definecolor{myred}{HTML}{A54242}
\definecolor{mygreen}{HTML}{8C9440}
\definecolor{myyellow}{HTML}{F0C674}
\definecolor{myblue}{HTML}{5f819d}
\definecolor{mymagenta}{HTML}{85678f}
\definecolor{mycyan}{HTML}{5e8d87}
\definecolor{mygray}{HTML}{373B41}
\renewcommand{\vec}[1]{\overrightarrow{#1}}
\newcommand{\del}{\partial}
\DeclareMathOperator*{\sgn}{sgn}
%\DeclareMathOperator*{\om}{Hom}
%\DeclareMathOperator*{\end}{End}
\DeclareMathOperator*{\id}{Id}
\DeclareMathOperator*{\im}{Im}
\DeclareMathOperator*{\re}{Re}

\setlength\theorempreskipamount{2ex}
\setlength\theorempostskipamount{3ex}
\allowdisplaybreaks

	
\newcommand\norm[1]{\left\vert#1\right\vert}
\newcommand\Norm[1]{\left\lVert#1\right\rVert}
\newcommand\abs[1]{\left\vert#1\right\vert}
\newcommand\inj{\hookrightarrow}
\newcommand\surj{\twoheadrightarrow}

\lstdefinestyle{mystyle}{
    backgroundcolor=\color{background},   
    commentstyle=\color{mygreen},
    keywordstyle=\color{mymagenta},
    numberstyle=\tiny\color{mygray},
    stringstyle=\color{mycyan},
    basicstyle=\ttfamily\footnotesize,
    breakatwhitespace=false,         
    breaklines=true,                 
    captionpos=b,                    
    keepspaces=true,                 
    numbers=left,                    
    numbersep=5pt,                  
    showspaces=false,                
    showstringspaces=false,
    showtabs=false,                  
    tabsize=2
}

\lstset{style=mystyle}

\newcommand\sidenote[1]{\footnote{#1}}

%\renewtheoremstyle{break}%
  %{\item{\theorem@headerfont
          %##1\ ##2\theorem@separator}\hskip\labelsep\relax\nobreakitem}%
  %{\item{\theorem@headerfont
          %##1\ ##2\ (##3)\theorem@separator}\hskip\labelsep\relax\nobreakitem}
%% make \th@nonumberbreak the same as \th@break, but remove "\ ##2"
%\let\th@nonumberbreak\th@break
%\xpatchcmd*{\th@nonumberbreak}{\ ##2}{}{}{}
%\makeatother

\theorempreskip{10pt}
\theorempostskip{5pt}
\theoremstyle{break}



    % theorem envs

\newmdtheoremenv{thm}{Theorème}
\newtheorem{defn}{Definition}

\newmdtheoremenv{propo}[thm]{Proposition}

\newmdtheoremenv{crly}[thm]{Corollaire}

\newmdtheoremenv{lemma}[thm]{Lemme}

\newmdtheoremenv{propr}[thm]{Propriete}

\newtheorem*{rmq}[thm]{Remarque}

\newtheorem{axiom}[thm]{Axiom}

\newtheorem*{exemple}[thm]{Exemple}

\theoremsymbol{\ensuremath{ \square }}
\newtheorem*{proof}{Preuve}
\theoremsymbol{}

\newtheorem{exo}[thm]{Exercice}

\newcommand\eng[1]{\left\langle#1\right\rangle}

%\hypersetup{
    %unicode=true,          % non-Latin characters in Acrobat’s bookmarks
    %pdftoolbar=false,        % show Acrobat’s toolbar?
    %pdfmenubar=false,        % show Acrobat’s menu?
    %pdffitwindow=true,     % window fit to page when opened
    %colorlinks=true,
    %allcolors=magenta,
%}

% tikz


% horizontal rule
\newcommand\hr{
    \noindent\rule[0.5ex]{\linewidth}{0.5pt}
}

\newcommand{\incfig}[1]{%
    \def\svgwidth{\columnwidth}
    \import{./figures}{#1.pdf_tex}
}
\newcommand{\filler}[1][10]%
{   \foreach \x in {1,...,#1}
    {   test 
    }
}

\def\mathnote#1{%
  \tag*{\rlap{\hspace\marginparsep\smash{\parbox[t]{\marginparwidth}{%
  \footnotesize#1}}}}
}
\pdfsuppresswarningpagegroup=1
\newcommand\contra{\scalebox{1.5}{$\lightning$}}
\makeatother
\def\@lecture{}%
\newcommand{\lecture}[3]{
    \ifthenelse{\isempty{#3}}{%
        \def\@lecture{Lecture #1}%
    }{%
        \def\@lecture{Lecture #1: #3}%
    }%
    \subsection*{\@lecture}
    \marginpar{\small\textsf{\mbox{#2}}}
}
\makeatletter
\renewcommand{\baselinestretch}{1.2}
\setcounter{secnumdepth}{3}
\setcounter{tocdepth}{3}
\makeatletter
\patchcmd{\chapter}{\if@openright\cleardoublepage\else\clearpage\fi}{}{}{}
\makeatother
%\titleformat{\chapter}%
  %{\huge\rmfamily\itshape\color{mymagenta}}% format applied to label+text
  %{\llap{\colorbox{mymagenta}{\parbox[c][1cm]{3cm}{\hfill\itshape\Huge\textcolor{background}{\thechapter}}}}}% label
  %{5pt}% horizontal separation between label and title body
  %{\faLeaf}% before the title body
  %[]% after the title body

%\titleformat{\section}%
  %{\normalfont\Large\rmfamily\itshape\color{myblue}}% format applied to label+text
  %{\llap{\colorbox{myblue}{\parbox{3cm}{\hfill\itshape\textcolor{background}{\thesection}}}}}% label
  %{5pt}% horizontal separation between label and title body
  %{}% before the title body
  %[]% after the title body

%% subsection format
%\titleformat{\subsection}%
  %{\normalfont\large\itshape\color{mygreen}}% format applied to label+text
  %{\llap{\colorbox{mygreen}{\parbox{3cm}{\hfill\textcolor{background}{\thesubsection}}}}}% label
  %{1em}% horizontal separation between label and title body
  %{}% before the title body
  %[]% after the title body

%% subsubsection format
%\titleclass{\subsubsection}{straight}
%\titleformat{\subsubsection}%
  %{\normalfont\large\itshape\color{myyellow}}% format applied to label+text
  %{\llap{\colorbox{myyellow}{\parbox{3cm}{\hfill\textcolor{background}{\thesubsubsection}}}}}% label
  %{1em}% horizontal separation between label and title body
  %{}% before the title body
  %[]% after the title body

\usepackage{subfiles}

\renewcommand{\vec}[1]{\overrightarrow{#1}}
\newcommand{\del}{\partial}
\DeclareMathOperator*{\sgn}{sgn}
\DeclareMathOperator*{\id}{Id}
\DeclareMathOperator*{\im}{Im}
\DeclareMathOperator*{\re}{Re}
\DeclareMathOperator*{\vol}{Vol}
\newcommand\norm[1]{\left\vert#1\right\vert}
\newcommand\ns[1]{\left\vert\left\vert\left\vert#1\right\vert\right\vert\right\vert}
\newcommand\Norm[1]{\left\lVert#1\right\rVert}
\newcommand\N[1]{\left\lVert#1\right\rVert}
\newcommand\abs[1]{\left\vert#1\right\vert}
\newcommand\inj{\hookrightarrow}
\newcommand\surj{\twoheadrightarrow}
\newcommand\ded[1]{\overset{\circ}{#1}}
\newcommand\sidenote[1]{\footnote{#1}}
\newcommand\eng[1]{\left\langle#1\right\rangle}
\newcommand\hr{
    \noindent\rule[0.5ex]{\linewidth}{0.5pt}
}

\newcommand{\incfig}[1]{%
    \def\svgwidth{\columnwidth}
    \import{./figures}{#1.pdf_tex}
}
\newcommand{\filler}[1][10]%
{   \foreach \x in {1,...,#1}
    {   test 
    }
}

\newcommand\contra{\scalebox{1.5}{$\lightning$}}
\makeatother
\def\@lecture{}%
\newcommand{\lecture}[3]{
    \ifthenelse{\isempty{#3}}{%
        \def\@lecture{Lecture #1}%
    }{%
        \def\@lecture{Lecture #1: #3}%
    }%
    \subsection*{\@lecture}
    \marginpar{\small\textsf{\mbox{#2}}}
}

\newcommand{\gdim}{\mathrm{gdim}}
\newcommand{\griff}{\mathrm{Griff}}
\title{Derived Categories of Special Cubic Fourfolds}
\author{David Wiedemann}
\date{}
\begin{document}
\maketitle
\begin{abstract}
These are notes prepated for my talk on derived categories of cubic hypersurfaces in the corresponding seminar in Bonn during the wintersemester 2023/24.\\
The main goal is to illustrate the Kuznetsov conjectures, and more generally rationality problems for cubic fourfolds via two key examples, mainly following the book \cite{Geometry_Cubic_Huybrechts}.
\end{abstract}

\tableofcontents

\section{Cubics and Rationality Problems}
Throughout, let $k= \mathbb{C}$ and let $X \subset \mathbb{P}^{n+1}$ be a hypersurface of degree $d$.\\
In the last talk, we defined the Kuznetsov component of a cubic hypersurface, this is a specified \textbf{admissible subcategory} of the bounded derived category of $X$ which can be thought of as the non-trivial part of $D^{\flat}( X) $.
Given a hypersurface $X \subset \mathbb{P}^{n+1}$ of degree $d$, a version of Bott vanishing shows that the longest exceptional sequence of twisting sheaves on $X$ one can get is $\left\langle \O_X, \O_X( 1), \ldots, \O_X( n+1-d)\right\rangle  $. 
We define the Kuznetsov component to be the right orthogonal to this:
\[ 
\mathcal{A}_X \coloneq \langle \O_X,\ldots, \O_X( n+1-d) \rangle^{\perp}.
\]
As the subcategories $ \langle \O_X( j) \rangle \subset D^{\flat}( X) $ are all equivalent\footnote{equivalent as $k$-linear triangulated categories} to $D^{\flat}( \spec k) $, $\mathcal{A}_X$ can be fruitfully thought of as the non-trivial part.\\
Today, we will use derived categories to study the rationality of cubic fourfolds, as a refresher, here is what we know about rationality of (smooth) cubics in lower dimension:
\begin{itemize}
\item If $C$ is a smooth cubic curve, it is an elliptic curve and hence never rational.
\item If $X$ is a smooth cubic surface, then it is the blowup of $\mathbb{P}^{2}$ in 6 points. In particular, it is always rational.
\item If $X$ is a smooth cubic threefold, then, by a result of Clemens and Griffiths, it is never rational.
\end{itemize}
The situation for cubic fourfolds is still unclear and a largely open problem, in fact, not a single cubic fourfold is known to be non-rational.\\
Our goal today is to shed light on a conjecture due to Kuznetsov
\begin{center}
	\textit{A smooth cubic fourfold $X$  is rational if and only if there is a K3 surface $S$ such that there is an equivalence $\mathcal{A}_X\simeq D^{\flat}( S) $.}
\end{center}
From now on, $X$ is a smooth cubic fourfold and $S$ a K3 surface.\\
We first make the observation that $\mathcal{A}_X$ shares some striking similarities with $D^{\flat}( S) $. Recall the general result from last time
\begin{propo}[Kuznetsov]
The Kuznetsov component $\mathcal{A}_X \subset D^{\flat}( X)$ has a Serre functor $\mathcal{S}$  given by $\mathcal{S}= [ 2]$.
Furthermore $\mathcal{A}_X$ is indecomposable, ie. there do not exist subcategories $A,B \subset \mathcal{A}_X$ such that $\langle A,B\rangle = \langle B,A\rangle$ are semi-orthogonal decompositions.
\end{propo}
Much like the case of K3 surfaces!
Moreover, certain numerical invariants of $\mathcal{A}_X$ coincide with those of a K3 surface, indeed, we find isomorphisms in the \textit{Hochschild homology} $HH^{\bullet}( \mathcal{A}_X) \simeq HH^{\bullet}( D^{\flat}( S) ) $.
%Let $\iota_\ast\colon\langle \O_X,\ldots, \O_X( n+1-d) \rangle \to D^{\flat}( X) $ be the inclusion with $\iota^{\ast}\dashv \iota_\ast \dashv \iota^{!}$.\\
%The Kuznetsov component being a right complement, we know that the inclusion $j_\ast\colon \mathcal{A}_X \hookrightarrow D^{\flat}( X) $  has a left-adjoint $j^{\ast}$ given by the completion 
%\[ 
%\iota_\ast\iota^{!}E \to E \to j^{\ast}E.
%\]
%Using \cite[lemma 7.1.14]{Geometry_Cubic_Huybrechts}, we see that $\mathcal{A}_X$ also has a Serre functor and hence it's inclusion also admits a right-adjoint $j^{!}$.\\
%In fact, this Serre functor can be computed explicitly, it is given by
%\[ 
%S^{-1} = j^{\ast}\circ ( \O_X( n+2-d) \otimes -) \circ [ -n] 
%\]
\section{Pfaffian Cubic Fourfolds}
We illustrate the conjecture in a case where we know $X$ to be rational. 
Remember the following theorem from a previous talk.
\begin{thm}[Pfaffian cubics are rational]
	A Pfaffian cubic fourfold is rational.
\end{thm}
We start by recalling what Pfaffian cubic fourfolds are and defining there associated K3 surfaces.\\
Let $W$ be a six dimensional vector space, then $\Lambda^{2}W$ is 15 dimensional and we consider it's projectivization $\mathbb{P}( \Lambda^{2}W) \simeq \mathbb{P}^{14}$.
The Pfaffian is the subvariety $Pf( W) \subset \mathbb{P}( \Lambda^{2}W) $ given by $Pf( W) = \left\{ \omega \in \mathbb{P}( \Lambda ^{2}W) | \omega\wedge\omega\wedge\omega = 0 \right\} $.
\begin{defn}[Pfaffian Cubic Fourfold]
	A smooth cubic fourfold is a \textit{Pfaffian Cubic fourfold} if it is isomorphic to $X_V \coloneq Pf( W^{\ast}) \cap \mathbb{P}( V) $ where $V \subset \Lambda^{2}W^{\ast}$ is a 6 dimensional sub vector space.
\end{defn}
\begin{defn}[Associated K3 surface ]
	Let $V \subset \Lambda^{2}W^{\ast}$ be a subspace as above and consider the Grassmanian of lines $\mathbb{G}( 1, \mathbb{P}( W) ) $  as a closed subscheme of $\mathbb{P}( \Lambda^{2}W) $ via the Pl\"ucker embedding.
	The associated K3 surface to $V$  is defined as $S_V \coloneq \left\{ p \in \mathbb{G}( 1, \mathbb{P}( W) ) | \omega|_p = 0 \text{ for all } \omega \in V \right\}  $.
\end{defn}
It was proven in a previous talk that $S_V$ is indeed a K3 surface that does not contain any lines and that $X_V$ is a smooth cubic fourfold.
%In that same talk, we proved the following result
%\begin{thm}[Pfaffians are Rational]
	%A Pfaffian Cubic Fourfold is rational.
%\end{thm}
%The birational map to $\mathbb{P}^{4}$ was given by $\omega \in X_V \mapsto \ker\omega\cap \mathbb{P}( W') $ 
Today, we will see that the K3 surface $S_V$ is the "associated K3" to $X_V$.
Our main first theorem is
\begin{thm}
If $X$ is a Pfaffian cubic fourfold, then there is an equivalence of categories	
\[ 
D^{\flat}( S_V) \simeq \mathcal{A}_X.
\]
\end{thm}
In the first part of this talk, we will sketch the proof of this theorem. 
An important ingredient is the correspondence
% https://q.uiver.app/#q=WzAsMyxbMCwwLCJcXFNpZ21hX1YgXFxzdWJzZXQgU19WXFx0aW1lcyBYX1YiXSxbMCwxLCJTX1YiXSxbMSwwLCJYX1YiXSxbMCwyLCJwX1giXSxbMCwxLCJwX1MiLDJdXQ==
\[\begin{tikzcd}
	{\Sigma_V \subset S_V\times X_V} & {X_V} \\
	{S_V}
	\arrow["{p_X}", from=1-1, to=1-2]
	\arrow["{p_S}"', from=1-1, to=2-1]
\end{tikzcd}\]
where $\Sigma_V= \left\{ ( p, \omega) \in S_V\times X_V | p \cap \ker \omega \neq 0 \right\} $. We give $\Sigma_V$ the reduced induced scheme structure.\\
Going forward, we will drop the subscript $V$ from our notation.\\

Let us sketch the structure of the proof:
\begin{enumerate}
\item Using $\Sigma$, we construct a (exact, $k$-linear) functor $\Psi\colon D^{\flat}( S) \to D^{\flat}( X)$
\item We show that the image of $\Psi$ is contained in $\mathcal{A}_X$.
\item We show that the restriction $\Psi\colon D^{\flat}( S) \to \mathcal{A}_X$ is fully faithful
\item We show that $\Psi$ is an equivalence
\end{enumerate}
\subsection*{Constructing $\Psi$ }
Define
\[ 
\mathcal{I}( -1) \coloneq \mathcal{I}_{\Sigma} \otimes p^{\ast}_X \O_X( -1),
\]
and let $$\Phi_{\mathcal{I}( -1) } \colon D^{\flat}( X) \to D^{\flat}( S)$$ be the associated Fourier-Mukai functor.
As seen in the previous talk, Fourier-Mukai functors have right-adjoints which are themselves Fourier-Mukai. 
Explicitly, the right adjoint $\Psi$ to $\Phi_{\mathcal{I}( -1) }$ has kernel $\mathcal{I}( -1)^{\vee}\otimes p_X^{\ast}\O_X( -3) [ 4] $.
\subsection*{$\Psi$ lands in the Kuznetsov component}
To prove this, we need a result describing the geometry of the correspondence $\Sigma \subset S\times X$.
\begin{propo}
\begin{enumerate}
	\item The fibers of $\Sigma \to S$ are either isomorphic to $\mathbb{P}^{1}\times \mathbb{P}^{1}$ or $\mathbb{F}_2$ and the embedding $\Sigma_p \to X \to \mathbb{P}^{5}$ describes $\Sigma_p$ as a quartic normal scroll. \footnote{See \cite[Sec. 6.2.6]{Geometry_Cubic_Huybrechts} for a more in depth discussion on these}\\
		In particular, the restriction maps $H^{0}( \mathbb{P}^{5}, \O( n) ) \to H^{0}( \Sigma_p, \O_{\Sigma_p} ( n) ) $ are bijections.
\item Let $P\in S$ and let $\Sigma_P$ be the corresponding fiber, then 
	\[ 
	\chi( RHom_{D^{\flat}( X) } ( \O_{\Sigma_P} , \O_{\Sigma_P} ) ) =10
	\]
\end{enumerate}
\end{propo}
Recall that $\mathcal{A}_X$ is defined as the right orthogonal of the category spanned by $ \left\{ \O_X, \O_X( 1) ,\O_X( 2)  \right\} $, hence it suffices to show that $\hom_{D^{\flat}( X) } ( \O_X( j) , \Psi( E) ) =0 $ for all $j=0,1,2$ and for all $E \in D^{b}( S) $.
As $\Psi$ is right-adjoint, this is equivalent (by Yoneda) to showing that $\Phi_{\mathcal{I}( -1) } ( \O_X( j) ) =0$ for all $j=0,1,2$.\\
%We will only show this for $j=0$, referring to the book for the other cases.
Consider the short exact sequence of sheaves on $S\times X$ 
\[ 
0 \to \mathcal{I}_{\Sigma}  \to \O_{X\times S} \to \O_{\Sigma} \to 0
\]
Twisting by $\O_X( -1) $, we see that to show $\Phi_{\mathcal{I}( -1) } ( \O_X) =0$, it suffices to show the vanishing $p_{S\ast} ( \O_{X\times S}( -1)  ) = p_{S\ast} ( \O_{\Sigma}(-1)  ) =0$.
Using cohomology and base change, it suffices to show that the cohomology of the fibers vanishes
\begin{itemize}
	\item By \cite[lemma 1.1.7]{Geometry_Cubic_Huybrechts} and Serre duality, $H^{\bullet}( X,\O_X( -1) ) =0$.
	\item By \cite[lemma 6.2.20]{Geometry_Cubic_Huybrechts}, the fibers of the projection $\Sigma \to S$ are quartic normal scrolls so either isomorphic to $\mathbb{P}^{1}\times \mathbb{P}^{1}$ or $\mathbb{F}_2$.\\
		If the fiber over $p$  is $\Sigma_p =\mathbb{P}^{1}\times \mathbb{P}^{1}$, we can use K\"unneth and the fact that $p_X^{\ast}\O_X( -1) |_{\Sigma_p} = \O_{\mathbb{P}^{1}\times \mathbb{P}^{1}} ( -2,-1)  $ to obtain the vanishing
		\[ 
		H^{\bullet}( \Sigma_p, \O( -2,-1) ) =0.
		\]
\end{itemize}
The cases $j=1,2$ are proven similarly using normality of $\Sigma_p \subset \mathbb{P}^{5}$  and we omit them here.\\
The case where the fibers are $\mathbb{F}_2$ is the exercise for today's talk.
\subsection*{$\Psi$ is fully faithful}
To prove this, we will use the following general result due to Bondal and Orlov.
\begin{thm}
Let $X, Y$ be smooth projective varieties and let $\Phi_P\colon D^{\flat}( X) \to D^{\flat}( Y)  $ be the Fourier-Mukai Functor associated to $P \in D^{\flat}( X\times Y) $.
Then $\Phi_{P} $ is fully faithful if an only if for any two points $x, y \in X$, we have
\[ 
\hom( \Phi_P( k( x) ) , \Phi_P( k( y) ) [ i] ) =
\begin{cases}
k \text{ if } x=y \text{ and } i=0\\
0 \text{ if } x\neq y \text{ or } i<0 \text{ or } i>\dim X
\end{cases}.
\]
\end{thm}
We know that $\Psi$ is the Fourier-Mukai functor whose associated kernel is $\mathcal{I}( -1)^{\vee}\otimes p_X^{\ast} ( \O_X( -3) ) [ 4] $, hence for $P \in S$, we find
\[ 
\Psi( k( P) ) = \mathcal{I}_{\Sigma_P} ( -1) ( -3) [ 4] 
\]
\textcolor{red}{Why is $ \mathcal{I}_{\Sigma_P} $ simple?}

Let $P_1,P_2\in S$, we compute
\[ 
\ext^{i}_{\mathcal{A}_X} ( \Psi( k( P_1) ) , \Psi( k( P_2) ) ) = \ext^{i}_{\mathcal{A}_X} ( \mathcal{I}_{\Sigma_{P_2} }  , \mathcal{I}_{\Sigma_{P_1} }  )
\]
Now clearly, $\ext^{\bullet <0}_{\mathcal{A}_X}( \mathcal{I}_{\Sigma_{P_2} }, \mathcal{I}_{\Sigma_{P_1} } ) =0$ and hence, because $\mathcal{A}_X$ is 2-Calabi-Yau, $\ext^{\bullet>0}_{\mathcal{A}_X} ( \mathcal{I}_{\Sigma_{P_2} } , \mathcal{I}_{\Sigma_{P_1} } ) =0$.\\
Since $\Sigma_{P_1}, \Sigma_{P_2}  $ are disjoint, $\hom( \mathcal{I}_{\Sigma_{P_1} } , \mathcal{I}_{\Sigma_{P_2} } ) =0$ \textcolor{red}{Why? Maybe write out ses for $P_1$ and take les in ext}
Hence, by Serre duality $\ext^{2}_{\mathcal{A}_X} ( \mathcal{I}_{\Sigma_{P_1} } , \mathcal{I}_{\Sigma_{P_2} } ) =0$.\\
Hence, it suffices to show that $\dim \ext^{1}( \mathcal{I}_{\Sigma_{P_2} }, \mathcal{I}_{\Sigma_{P_1} }  )=0$.\\
To prove this, we consider the Euler characteristic of the complex $R\hom( \mathcal{I}_{\Sigma_{P_2} } , \mathcal{I}_{\Sigma_{P_1} } ) $, by constancy of the Euler characteristic in flat families
\begin{align*}
	\chi( \mathcal{I}_{\Sigma_{P_1} } , \mathcal{I}_{\Sigma_{P_2} } ) &= \chi( \mathcal{I}_{\Sigma_{P} } , \mathcal{I}_{\Sigma_P} ) \\
									  &= \chi( \O_X,\O_X) -\chi( \O_X, \O_{\Sigma_P} ) - \chi( \O_{\Sigma_P} , \O_X) + \chi( \O_{\Sigma_P} ,\O_{\Sigma_P} ) \\
									  &= 0
\end{align*}
\textcolor{red}{Should I present this?}

\subsection{$\Psi$ is essentially surjective}
Recall that $\mathcal{A}_X$ is indecomposable and suppose that $\Psi$ is not essentially surjective, then the image is a full admissible subcategory of $\mathcal{A}_X$ and there is a natural semi-orthogonal decomposition
\[ 
	\mathcal{A}_X = \left\langle { }^{\perp}\Psi( D^{\flat}( S) ) ,\Psi( D^{\flat}( S) )\right\rangle =  \left\langle \Psi( D^{\flat}( S) ) ,{ }^{\perp}\Psi( D^{\flat}( S) )[2]\right\rangle.
\]
This contradicts the indecomposability of $\mathcal{A}_X$, since $\Psi$ is not trivial, it must be essentially surjective.


\section{Twisted Derived Categories}
We will now introduce a new triangulated category called the \textit{twisted derived category} of sheaves.
This category: $D^{\flat}( S, \alpha) $ will depend on the choice of a projective variety $S$ and on a cohomology class $\alpha \in H^{2}_{\mathrm{\'et}}( S, \mathbb{G}_m) $.\\
$D^{\flat}( S,\alpha) $ will be the derived bounded category of an abelian category $\coh( S,\alpha) $  which we now construct.\\
There are two different but equivalent ways of defining $\coh( S,\alpha) $.
\begin{defn}[Twisted coherent sheaves]
Let $\alpha\in H^{2}_{\'et}( S, \mathbb{G}_m)$, let $U \to S$ be an \'etale cover such that $\alpha$ is represented by a cocycle $a \in \Gamma ( U \times_S U \times_S U, \mathbb{G}_m) $.\\
A $\alpha$-twisted sheaf $( \mathcal{F},\phi )$ on $S$ is a sheaf $\mathcal{F}$ on $U$ together with an isomorphism
\[ 
\phi\colon \pr_1^{\ast}\mathcal{F} \to \pr_2^{\ast}\mathcal{F}
\]
satisfying the cocycle conditions "up to" $\alpha$, ie. such that
\[ 
\pr_{1,2} ^{\ast}\phi \circ \pr_{2,3}^{\ast}\phi = a \cdot \pr_{1,3}^{\ast}\phi.
\]
\end{defn}
One would now have to check that this construction does not depend on the choice of cocycle/\'etale cover.
\begin{defn}[Morphisms of twisted sheaves]
	Let $( \mathcal{F}, \phi_{\F}  ), ( \mathcal{G}, \phi_{\G}  )$ be $\alpha$-twisted sheaves on $S$ and suppose that they are defined on the same \'etale cover $U \to S$, a morphism from $( \F ,\phi_{\F} )\to( \G,\phi_\G )$ is a morphism of sheaves $\F \to \G$ on $U$ such that
	\[ 
	\phi_\G \circ \pr_1^{\ast} \F = \pr_2^{\ast}\circ \phi_{\F}.
	\]
\end{defn}

Let $\mathcal{E}$ be a locally-free $\alpha$-twisted and let $\mathcal{A}=\mathcal{E} \otimes \mathcal{E}^{\vee}= \Endo\left( \mathcal{E}\right) $, this has the natural structure of a sheaf of algebras.\\
Then $\mathcal{A}$ is no longer twisted and descends to a sheaf of algebras on $S$ which we also call $\mathcal{A}$. 
There is an equivalence of categories
\[ 
\coh( S, \alpha) \simeq \coh( S, \mathcal{A}),
\]
where the left hand side is the category of right $\mathcal{A}$-modules.\\
Finally, we can associate to the twisted locally free sheaf $\mathcal{E}$, it's projectivisation $\mathbb{P}( \mathcal{E}) $.\\
The isomorphism $\phi$ then gives \'etale descent data for $\mathbb{P}( \mathcal{E}) $ and we call $\mathbb{P}( \mathcal{E}) $ the corresponding scheme over $S$.
In fact, there are bijections
\[ 
	H^{2}_{\'et} ( S, \mathbb{G}_m) \simeq \left\{\text{ equivalence classes of Azumaya algebras } \right\} \simeq \left\{\text{ equivalence classes of $\mathbb{P}^{n}$-fibrations } \right\}
\]
\section{Cubic Fourfolds containing a Plane}
Let $X$ be a general cubic fourfold containing a plane $P \subset X$ , in this last part we will sketch a proof of the following theorem.
\begin{thm}\label{Kuznetsov_con_containing_plane}
	There exists a K3-surface $S$ and a Brauer class $\alpha \in Br( S) $ such that
\[ 
D^{\flat}( S, \alpha) \simeq \mathcal{A}_X.
\]
\end{thm}
Let us start with the geometric setup that was already covered in a previous talk.
Let $p \colon X \dashrightarrow \mathbb{P}^{2}$ be the projection from $P$, this induces a map $\tilde X \coloneq \bl_P X \to \mathbb{P}^{2}$ which is a fibration in 2-dimensional quartics.\\
We make the simplifying assumption here that the discriminant divisor of $p$ is smooth\footnote{Ie. the locus of non-smooth fibers }.
Let $\tilde F$ be the relative Fano variety of $\tilde X \to \mathbb{P}^{2}$, ie. $\tilde F$ is a closed subscheme of $F( \tilde X) $ containing only the points corresponding to lines contained in fibers of $p$.\\
There is a natural map $q\colon \tilde F \to \mathbb{P}^{2}$ sending the class of a line $ [ L] \in \tilde F \to p( [ L] )\in \mathbb{P}^{2} $.
\begin{propo}
The $\O$-connected part of the Stein factorization of $q$ corresponds to a $\mathbb{P}^{1}$-fibration $\tilde F \to S$ with $S$ a K3-surface.
\end{propo}
We can now infer the general strategy of proof of theorem \ref{Kuznetsov_con_containing_plane}, the $\mathbb{P}^{1}$-fibration determines a Brauer class $\alpha\in Br( S) $ and we should use the two maps $S \to \mathbb{P}^{2}$ and $\tilde X\to \mathbb{P}^{2}$ to relate $D^{\flat}( S, \alpha) $ to $\mathcal{A}_X$.
\begin{proof}[(Sketch) of theorem \ref{Kuznetsov_con_containing_plane}]
The Orlov formula for blow-ups gives a semi-orthogonal decomposition of $D^{\flat}( \tilde X) $ 
\[ 
D^{\flat}( \tilde X) = \left\langle D^{\flat}( P) , \mathcal{A}_X, \O( -2 E) , \O( -E),\O\right\rangle
\]
where $E$ is the exceptional divisor.\\
A general formula for quadric fibrations gives a second semi-orthogonal decomposition 
\[ 
D^{\flat}( \tilde X) = \left\langle D^{\flat}( \mathbb{P}^{2}, \mathcal{B}) , D^{\flat}( \mathbb{P}^{2}) \otimes \O( -E) , D^{\flat }( \mathbb{P}^{2})\right\rangle.
\]
Where $\mathcal{B}$ is the pushforward to $\mathbb{P}^{2}$  of the Azumaya algebra on $S$ corresponding to $\alpha$.
Since the map $S\to \mathbb{P}^{2}$ is finite, it induces an equivalence of categories $D^{\flat}( S, \alpha) \simeq D^{\flat}( \mathbb{P}^{2}, \mathcal{B}) $, concluding the proof.
\end{proof}







%\section{Griffith's components}
%In the second part of this talk, we will explore a more conjectural that makes the Kuznetsov conjectures believable.\\
%Throughout, let $X$ be a variety.
%\begin{defn}[Geometric Dimension]
	%Let $\mathcal{A}$ be an admissible subcategory of $D^{\flat}( X) $, we define the geometric dimension $\gdim( \mathcal{A}) $  of $\mathcal{A}$ as the least integer $k$  such that $\mathcal{A}$ can be realized as a full admissible subcategory of an $n$-dimensional variety $X$.
%\end{defn}
%\begin{defn}[Griffith's component]
%Let $D^{\flat}( X) = \left\langle \mathcal{A}_1,\ldots, \mathcal{A}_n\right\rangle$ be a maximal semi-orthogonal decomposition, ie. a semi-orthogonal decomposition such that all $\mathcal{A}_i$ are indecomposable.
%The \textit{Griffith's component} is the subcategory
%\[ 
	%\griff( X) = \left\langle \left\{    \mathcal{A}_i \big| \gdim{\mathcal{A}_i} \geq \dim X-1\right\} \right\rangle.
%\]
%\end{defn}
%Note that this Griffith's component depends on the choice of maximal semi-orthogonal decomposition. 
%Indeed, there exist $k$-linear triangulated categories admitting maximal sod's which yield non-equivalent Griffith's components.\\
%However, it is easy to see the following
%\begin{thm}
%The Griffith's component of a variety $X$, if well-defined, is a birational invariant.
%\end{thm}
%\begin{proof}
%By the weak-factorization theorem, it suffices to show that if $\pi\colon X \to Y$ is a blow-up, then the Griffith's components of $X$ and $Y$ coincide.
%This follows immediatly from Orlov's blowup formula.
%\end{proof}
%A hint that $\griff( X) $ can be fruitfully used to study rationality questions is to notice that $\griff( \mathbb{P}^{n})$ is empty for all $n \geq 2$.\\
%In particular, if $\mathcal{A}_X \simeq D^{\flat}( S) $, then $\gdim( \mathcal{A}_X) =2$ and hence $\griff( X) $ is empty, further supporting the Kuznetsov conjectures.\\
%If however, $\mathcal{A}_X\not\simeq D^{\flat}( S) $, then heuristic considerations show that $\gdim( \mathcal{A}_{X} ) >2$ should hold.\\









\end{document}
