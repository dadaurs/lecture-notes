\documentclass[a4paper,hidelinks]{article}
\usepackage[utf8]{inputenc}
\usepackage[T1]{fontenc}
\usepackage{textcomp}
\usepackage{hyperref}
\usepackage{amsmath}
\usepackage{bookmark}
\usepackage{float}
\usepackage{graphicx}
\usepackage{mdframed}
\usepackage[french]{babel}
\usepackage{bbm} %RCQ
\usepackage{verbatim}
\usepackage{varioref}
\usepackage{theoremref}
\usepackage{tikz}
\usepackage{listings}
\usepackage{faktor}
\usepackage{python}
\usepackage{xurl}
\usepackage{xcolor}
\usepackage[amsmath,hyperref,thmmarks]{ntheorem}
\usepackage[fixed]{fontawesome5}
\usepackage{imakeidx}
\makeindex
% figure support
\usepackage{import}
\usepackage{xifthen}
\usepackage{pdfpages}
\usepackage{transparent}
\usepackage{thmtools}
\usepackage{amssymb}
\usepackage{aligned-overset}
%\usepackage{fancyhdr}
\usepackage{stmaryrd} % for \lightning
%\pagestyle{fancy}
\definecolor{background}{HTML}{FFFFFF}
\definecolor{myred}{HTML}{A54242}
\definecolor{mygreen}{HTML}{8C9440}
\definecolor{myyellow}{HTML}{F0C674}
\definecolor{myblue}{HTML}{5f819d}
\definecolor{mymagenta}{HTML}{85678f}
\definecolor{mycyan}{HTML}{5e8d87}
\definecolor{mygray}{HTML}{373B41}
\renewcommand{\vec}[1]{\overrightarrow{#1}}
\newcommand{\del}{\partial}
\DeclareMathOperator*{\sgn}{sgn}
%\DeclareMathOperator*{\om}{Hom}
%\DeclareMathOperator*{\end}{End}
\DeclareMathOperator*{\id}{Id}
\DeclareMathOperator*{\im}{Im}
\DeclareMathOperator*{\re}{Re}

\setlength\theorempreskipamount{2ex}
\setlength\theorempostskipamount{3ex}
\allowdisplaybreaks

	
\newcommand\norm[1]{\left\vert#1\right\vert}
\newcommand\Norm[1]{\left\lVert#1\right\rVert}
\newcommand\abs[1]{\left\vert#1\right\vert}
\newcommand\inj{\hookrightarrow}
\newcommand\surj{\twoheadrightarrow}

\lstdefinestyle{mystyle}{
    backgroundcolor=\color{background},   
    commentstyle=\color{mygreen},
    keywordstyle=\color{mymagenta},
    numberstyle=\tiny\color{mygray},
    stringstyle=\color{mycyan},
    basicstyle=\ttfamily\footnotesize,
    breakatwhitespace=false,         
    breaklines=true,                 
    captionpos=b,                    
    keepspaces=true,                 
    numbers=left,                    
    numbersep=5pt,                  
    showspaces=false,                
    showstringspaces=false,
    showtabs=false,                  
    tabsize=2
}

\lstset{style=mystyle}

\newcommand\sidenote[1]{\footnote{#1}}

%\renewtheoremstyle{break}%
  %{\item{\theorem@headerfont
          %##1\ ##2\theorem@separator}\hskip\labelsep\relax\nobreakitem}%
  %{\item{\theorem@headerfont
          %##1\ ##2\ (##3)\theorem@separator}\hskip\labelsep\relax\nobreakitem}
%% make \th@nonumberbreak the same as \th@break, but remove "\ ##2"
%\let\th@nonumberbreak\th@break
%\xpatchcmd*{\th@nonumberbreak}{\ ##2}{}{}{}
%\makeatother

\theorempreskip{10pt}
\theorempostskip{5pt}
\theoremstyle{break}



    % theorem envs

\newmdtheoremenv{thm}{Theorème}
\newtheorem{defn}{Definition}

\newmdtheoremenv{propo}[thm]{Proposition}

\newmdtheoremenv{crly}[thm]{Corollaire}

\newmdtheoremenv{lemma}[thm]{Lemme}

\newmdtheoremenv{propr}[thm]{Propriete}

\newtheorem*{rmq}[thm]{Remarque}

\newtheorem{axiom}[thm]{Axiom}

\newtheorem*{exemple}[thm]{Exemple}

\theoremsymbol{\ensuremath{ \square }}
\newtheorem*{proof}{Preuve}
\theoremsymbol{}

\newtheorem{exo}[thm]{Exercice}

\newcommand\eng[1]{\left\langle#1\right\rangle}

%\hypersetup{
    %unicode=true,          % non-Latin characters in Acrobat’s bookmarks
    %pdftoolbar=false,        % show Acrobat’s toolbar?
    %pdfmenubar=false,        % show Acrobat’s menu?
    %pdffitwindow=true,     % window fit to page when opened
    %colorlinks=true,
    %allcolors=magenta,
%}

% tikz


% horizontal rule
\newcommand\hr{
    \noindent\rule[0.5ex]{\linewidth}{0.5pt}
}

\newcommand{\incfig}[1]{%
    \def\svgwidth{\columnwidth}
    \import{./figures}{#1.pdf_tex}
}
\newcommand{\filler}[1][10]%
{   \foreach \x in {1,...,#1}
    {   test 
    }
}

\def\mathnote#1{%
  \tag*{\rlap{\hspace\marginparsep\smash{\parbox[t]{\marginparwidth}{%
  \footnotesize#1}}}}
}
\pdfsuppresswarningpagegroup=1
\newcommand\contra{\scalebox{1.5}{$\lightning$}}
\makeatother
\def\@lecture{}%
\newcommand{\lecture}[3]{
    \ifthenelse{\isempty{#3}}{%
        \def\@lecture{Lecture #1}%
    }{%
        \def\@lecture{Lecture #1: #3}%
    }%
    \subsection*{\@lecture}
    \marginpar{\small\textsf{\mbox{#2}}}
}
\makeatletter
\renewcommand{\baselinestretch}{1.2}
\setcounter{secnumdepth}{3}
\setcounter{tocdepth}{3}
\makeatletter
\patchcmd{\chapter}{\if@openright\cleardoublepage\else\clearpage\fi}{}{}{}
\makeatother
%\titleformat{\chapter}%
  %{\huge\rmfamily\itshape\color{mymagenta}}% format applied to label+text
  %{\llap{\colorbox{mymagenta}{\parbox[c][1cm]{3cm}{\hfill\itshape\Huge\textcolor{background}{\thechapter}}}}}% label
  %{5pt}% horizontal separation between label and title body
  %{\faLeaf}% before the title body
  %[]% after the title body

%\titleformat{\section}%
  %{\normalfont\Large\rmfamily\itshape\color{myblue}}% format applied to label+text
  %{\llap{\colorbox{myblue}{\parbox{3cm}{\hfill\itshape\textcolor{background}{\thesection}}}}}% label
  %{5pt}% horizontal separation between label and title body
  %{}% before the title body
  %[]% after the title body

%% subsection format
%\titleformat{\subsection}%
  %{\normalfont\large\itshape\color{mygreen}}% format applied to label+text
  %{\llap{\colorbox{mygreen}{\parbox{3cm}{\hfill\textcolor{background}{\thesubsection}}}}}% label
  %{1em}% horizontal separation between label and title body
  %{}% before the title body
  %[]% after the title body

%% subsubsection format
%\titleclass{\subsubsection}{straight}
%\titleformat{\subsubsection}%
  %{\normalfont\large\itshape\color{myyellow}}% format applied to label+text
  %{\llap{\colorbox{myyellow}{\parbox{3cm}{\hfill\textcolor{background}{\thesubsubsection}}}}}% label
  %{1em}% horizontal separation between label and title body
  %{}% before the title body
  %[]% after the title body

\usepackage{subfiles}

\renewcommand{\vec}[1]{\overrightarrow{#1}}
\newcommand{\del}{\partial}
\DeclareMathOperator*{\sgn}{sgn}
\DeclareMathOperator*{\id}{Id}
\DeclareMathOperator*{\im}{Im}
\DeclareMathOperator*{\re}{Re}
\DeclareMathOperator*{\vol}{Vol}
\newcommand\norm[1]{\left\vert#1\right\vert}
\newcommand\ns[1]{\left\vert\left\vert\left\vert#1\right\vert\right\vert\right\vert}
\newcommand\Norm[1]{\left\lVert#1\right\rVert}
\newcommand\N[1]{\left\lVert#1\right\rVert}
\newcommand\abs[1]{\left\vert#1\right\vert}
\newcommand\inj{\hookrightarrow}
\newcommand\surj{\twoheadrightarrow}
\newcommand\ded[1]{\overset{\circ}{#1}}
\newcommand\sidenote[1]{\footnote{#1}}
\newcommand\eng[1]{\left\langle#1\right\rangle}
\newcommand\hr{
    \noindent\rule[0.5ex]{\linewidth}{0.5pt}
}

\newcommand{\incfig}[1]{%
    \def\svgwidth{\columnwidth}
    \import{./figures}{#1.pdf_tex}
}
\newcommand{\filler}[1][10]%
{   \foreach \x in {1,...,#1}
    {   test 
    }
}

\newcommand\contra{\scalebox{1.5}{$\lightning$}}
\makeatother
\def\@lecture{}%
\newcommand{\lecture}[3]{
    \ifthenelse{\isempty{#3}}{%
        \def\@lecture{Lecture #1}%
    }{%
        \def\@lecture{Lecture #1: #3}%
    }%
    \subsection*{\@lecture}
    \marginpar{\small\textsf{\mbox{#2}}}
}

\title{Questions}
\author{David Wiedemann}
\date{}
\begin{document}
\maketitle
I have decided to first present the proof of the Kuznetsov conjectures for Pfaffian cubics following Huybrecht's book as it would best tie in with Saverio's talk.
Here are a few questions relating to the proofs of these.
\begin{enumerate}
\item The geometric input we need for this proof is lemma 6.2.20, Saverio told me he sketched a proof of this, but do you think it makes sense to sketch a proof of this again? Especially the fact that the fibers are $\mathbb{P}^{1}\times \mathbb{P}^{1}$ or $\mathbb{F}_2$?
\item In the proof of lemma 3.10, I am clear on how the proof works when the fiber is $\mathbb{P}^{1}\times \mathbb{P}^{1}$, but I would like to at least have some idea of how the $\mathbb{F}_2$ case works, potentially using it as an exercise. (or is this bad practice?)
In fact, I am not even sure what $\O_X( 1) $ restricts to on a fiber isomorphic to $\mathbb{F}_2$.
\item Turning to lemma 3.11, there are certain claims made in the proof that I am not confident about.
	Why is $\mathcal{I}_{\Sigma_p} $ simple? This is simply stated and I tried to use the LES in ext to show this but didn't get far. 

\item I have the same question for $Hom( \mathcal{I}_{\Sigma_{p_1} } , \mathcal{I}_{\Sigma_{p_2} } ) = 0$.

\item Is $\mathcal{I}$ the ideal sheaf associated to the reduced induced scheme structure on $\Sigma$ 
\item Finally, do you think it makes sense to present the computation of $\chi( \mathcal{I}_{\Sigma_{p_1} } , \mathcal{I}_{\Sigma_{p_2} } ) $? Of course, the proof that $\chi( \O_{\Sigma_p} , \O_{\Sigma_p} ) = 10$ is not within the scope of the talk, but I can put that into one "geometric input" lemma.
\end{enumerate}

I believe the introduction together with the proof of this result will occupy a bit more than half of the talk and I have been debating what to present in the second part.\\
As my main goal should be to motivate the Kuznetsov conjectures in general, two things come to mind.\\
First off, I am thinking of defining what the twisted derived category of sheaves on a variety is using Azumaya algebras and mentioning proposition 3.3 as well as its corollary 3.5.
This would imply presenting the main geometry of the situation (in particular, proposition 6.1.10).
Unfortunately I would not go into the details of the proof of any lemma as it is very similar to the first proof I gave and I don't have the time.\\
Another option I have been considering is to follow Kuznetsov's expository paper \url{https://arxiv.org/abs/1509.09115} to present a more theoretical reason that motivates the formulation of the Kuznetsov conjectures.
In this paper, he defines the Griffith's component of a $k$-linear triangulated category and shows that, if it is well-defined, it is a birational invariant of $D^{\flat}( X)$.
He then uses this Griffith's component and another conjecture to argue that, if $\mathcal{A}_X\not\simeq S$, then the geometric dimension of $\mathcal{A}_X$ is not 2 and hence $X$ could not be rational.
The main argument is presented on page 21 of the paper while a few definitions and results are defined before.
I believe this second option has a few distinct advantages which I would like your feedback on.
\begin{itemize}
\item It has a very different flavour than the proof of proposition 6.3.9, while the proof of 6.3.3 is very similar. For time reasons, I don't think I could do the proof of 6.3.3 justice.
\item It doesn't rely on extra geometric input that has not been covered in the seminar up until now
\item The whole argument is reasonably easy to follow and manages to omit the use of Hochschild homology which is out of the scope of my talk.
\end{itemize}




\end{document}
