\documentclass[../main.tex]{subfiles}
\begin{document}
\lecture{4}{Thu 15 Sep}{distribution of ideals in number rings}
The basic result we want to show is that ideals are approximatively evenly distributed among class group, more precisely:
Let $R$ be a number ring corresponding to $K / \mathbb{Q}$ ( an extension of degree n) and let $C$ be a class in it's ideal class group, let $i_C( t) $ denote the number of ideals $I$  in $C$ satisfying $ \N I \leq t$, then
\[ 
i_C( t) = \kappa t + \epsilon( t) , \quad \epsilon ( t) \text{ is } O( t^{1-\frac{1}{n}}) 
\]
To show this, we resort to a few tricks.\\
We first start off by noticing a bijection between the set of ideals we want to count and another "nice" set, namely, there is a one-to-one correspondence
\[ 
A \coloneqq \left\{ \text{ Ideals $I$ in $C$ such that } \N I \leq t \right\} \leftrightarrow \left\{ \text{ Principal ideals $( \alpha) \subset J$ such that } \N \alpha \leq t \N J \right\} = :B
\]
Where $J$ is a fixed element of the class $C^{-1}$.\\
The correspondence is given as follows, to an ideal  $I \in C$, we associate $ IJ$, by multiplicativity of the norm, it is clear that $IJ$ is in the right hand set.\\
As an inverse, for a principal ideal $( \alpha) \subset J$, let $I$ be an ideal such that $IJ = ( \alpha) $ ( exists by some previous theorem), then $I$ is the corresponding ideal in the left set.\\
If the group of units of $R$ ( call it $U$ ) is finite, and we are able to count the elements whose norm is $ \leq t \N J$ ( call it $n$ ) , then we would have  $i_C( t) |U| =n $.\\

To count elements of $B$, we now resort to the second main trick in the proof: we construct a subset of $R$ which has a representative for every coset of $U$.\\
Then intersecting this subset with the set of all elements of norm smaller than $t$, we will be able to count elements.\\
Recall from the previous chapter that $U = W\times V$ with $V$ free abelian and $W$ cyclic.\\
To construct this subset, we recall the log maps defined a chapter earlier, namely, we consider the compositions
\[ 
V \subset U \subset R- 0 \to \wedge_R - 0 \to \mathbb{R}^{r+s}
\]
Here, as usual, $r$ is the number of real embeddings of $K$ and $2s$ is the number of complex ones, recall also that $\wedge_R$ is a $r+2s$ dimensional lattice.\\
We recall that units get mapped onto the hyperplane $H: \sum_i y_i =0$, because for $u$  a unit, it's norm is $1$. We also remember that the kernel of this composition is $W$ and thus the composition $V \to \mathbb{R}^{r+s}$ is injective. We denote it's image by $\wedge_U$.\\
We replace $\wedge_R \subset \mathbb{R}^{n} \simeq \mathbb{R}^{r}\times \mathbb{C}^{s}$ and extend the log mapping by defining 
\[ 
\log ( x_1,\ldots,x_r, z_1,\ldots,z_s) \coloneqq ( \log |x_1|, \ldots, 2 \log |z_1| ,\ldots) 
\]
It is less an extension and more so a redefinition made to agree with the usual log mapping.\\
Note that with this definition, counting elements of $R$ satisfying $ \N \alpha \leq t \N J$ is equivalent to counting elements of $ \mathbb{R}^{r}\times \mathbb{C}^{s}$ ( in the image of the inclusion) satisfying $N( x) \leq t \N J$ 

We get onto the third trick in this proof, namely, instead of finding representatives of cosets directly in $R$, we will search for them in $\mathbb{R}^{r+s}$ and then pull them back to representatives in $R$. To this end we use the following proposition:
\begin{propo}
If $f:G \to G'$ is a morphism of abelian groups and $S \subset G $ gets mapped isomorphically onto it's image in $G'$, then a set of representatives of cosets of $f( S) \subset G'$ pull back to representatives of cosets in $G$ 
\end{propo}
\begin{proof}
Let $D'$ be a set of coset representatives of $f( S) $, let $aS$ be a coset in $G$, let $d$ be a representative for $f( aS) $, then any element of $f^{-1}( d) $ is a representative for $aS$.
\end{proof}
We apply this to $V \subset R$, and so we have to find a set of coset representatives in $ \mathbb{R}^{r+s}$ for $V$.\\
Here's a general procedure to find such a set, start of with a fundamental parallelotope $F$ for $\wedge_U$ and let $D'= F \oplus \mathbb{R}v$, where we choose $v= ( 1,\ldots, 2) $ for technical reasons ( though any $v$ not in $H$ would work.), then $D = \log ^{-1}( D') $ is a set of coset representatives in $ \mathbb{R}^{r} \times \mathbb{C}^{s}$.\\
We can check that this indeed is a set of coset representatives by using the proposition above.
Indeed, let $x \in \mathbb{R}^{r+s}$, let $x_v$ be it's projection onto $v$, then $x-x_v$ is an element of the subvector space spanned by the image of $V$ and thus has a representative in $F$. ( this is because the group of units has rank $r+s -1$ ).\\


Now stuff gets technical, define $D_a = \left\{ x : N( x) \leq a \right\} $, then $D_a = a^{\frac{1}{n}} D_1$, which means we want to estimate the number of points $ ( t \N J)^{\frac{1}{n}} D_1$ and this is not something im about to latex.



\end{document}	
