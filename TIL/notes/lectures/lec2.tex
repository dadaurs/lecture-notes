\documentclass[../main.tex]{subfiles}
\begin{document}
\lecture{2}{Mon 05 Sep}{Crossed Homomorphisms}
For a finite group $G$ together with a $G$-module $A$, we can nicely describe the first cohomology group $H^{1}( G,A) $.\\
Indeed, letting $d_n: C_n( G,A) \to C_{n+1} ( G,A) $ be the boundary maps, we get that $d_1( f) ( g,h) = g f( h) - f( gh) +f( g) $ so an element of the kernel $d_1$ is precisely an element satisfying $f( gh) = gf( h) - f(g) $.
We can also describe coboundaries as elements of the form $f( g) = ga-a$ which are called principal crossed homomorphisms.

From this we can deduce Hilbert's theorem 90 which says that for a finite galois extension $K/L$ with galois group $G$, we have $H^{1}( G, K^{\times})=0 $, here $K^{\times}$ is made into a $G$-module in the obvious way.\\
To prove this, we use that the Galois automorphisms are linearly independent in $End( K,K) $ thus for a cocycle $ \left\{ a_\sigma \right\} $ ( ie a map $G\to A$ satisfying $f( \sigma\tau) = \sigma f( \tau) f( \sigma)^{-1}= \sigma a_\tau a_\sigma^{-1}$) hence there is some $\gamma\in K$ such that
\[ 
\beta = \sum_{\sigma \in G} a_\sigma \sigma( \gamma) 	
\]
is nonzero. We can now easily check that
\[ 
\sigma( \beta) = a_\sigma^{-1}( \beta) \beta \implies a_\sigma = \frac{\beta}{\sigma( \beta) }
\]
Which in turn means that 1-cocycles satisfy the $1$-coboundary condition and thus $H^{1}( G, K^{\times}) = 0$ 



\end{document}	
