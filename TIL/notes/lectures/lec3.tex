\documentclass[../main.tex]{subfiles}
\begin{document}
\lecture{3}{Tue 06 Sep}{Exterior Derivatives}
We start with a smooth manifold $M$, on this, we have differential k-forms, these are maps which, to each $p \in M$ associate an alternating $k$-form $ \omega_p$.\\
We think of 1-forms as projections, ie. differential forms act on vector fields in the obvious way, by defining $\left( \omega( X)  \right)_p = \omega_p( X_p) $, as such, at least locally, differential forms can be thought of as projections of a vector field onto a subvector.\\
Somehow, $k$-forms then represent higher dimensional projections ( as they act on $k$ vector fields at the same time and will somehow locally describe a $k$-dimensional volume) 
To describe smoothness of these $k$-forms, we turn to vector bundles, first, we can define the cotangent bundle ( usually written $T^{\ast}M$), this is just the disjoint union  $\bigcup_{p \in M} T_p^{\ast}M$.\\
We topologise this in a way analoguous to the tangent bundle.\\
For a coordinate chart $( U, y_1,\ldots, y_n)\ni p $, we have a canonical choice of basis for $T_p^{*}M$ which is just the projections onto the coordinates (equivalently, the dual basis to $ \frac{\del }{\del y_i}$ ).\\
Now we have trivialization maps $\phi: \bigcup_{p \in U} T_p^{* }M \to U \times \mathbb{R}^{n}$ and we can endow $T^{*}M$ with the topology induced from the trivialization maps.\\

For $k$-forms, we have to refine the construction a bit.\\
First of, we recall that an alternating $k$-form on a v.s. $V$ is a multilinear map $ V^{k}\to \mathbb{R}$ which is alternating (respects sign of signature when we permute the coefficients).\\
These alternating forms form a vector space called the $k$th exterior power of $V$ which we denote by $\bigwedge^{k}( V^{*}) $, we can now, again, form a vector bundle over $M$, written $\bigwedge^{k}( T^{*}M) $ and called the kth exterior power and defined, as a set, to be $ \bigcup_{p \in M} \bigwedge^{k}( T^{*}_pM) $ and topologized as above( with corresponding basis explained below) \\
Indeed, a natural choice for a basis of $\bigwedge^{k}( T^{*}_p M) $ is the set $dx^{I}= dx_{i_1} \wedge \ldots \wedge dx_{i_k} $ where $I$ runs over all growing multiindices.\\

We can now get to the first important property of $k$-forms, namely a pullback of a $k$-form.\\
Recall that every smooth map $F: M\to N$ induces a pushforward on tangent spaces by sending, for $f \in C^{ \infty }_{F( p) }  ( N)$,  $X$ a tangent vecotr at $p$  (ie. a derivation $C^{ \infty }_{p} ( M) \to \mathbb{R}$ ) 
\[ 
	( F_{*,p} X) ( f) = X ( f\circ F) 
\]
One important thing to note is that this pushforward is not (in general) defined globally as a point $p' \in N$ may have non-trivial fibers, while the differentials at the different points of the fiber may disagree.\\
This means we can \underline{NOT} pushforward vector fields.\\

In this respect differential forms are better behaved because we can pull them back 
With the same setup as above, let $\omega$ be a differential $k$-form on $N$, we can define the pullback of $\omega$, written $F^{*}\omega$.\\
This pullback is just precomposition with $ F_\ast^{\times n}$, namely, for $v_1,\ldots,v_k\in T_pM$ , we have
\[ 
( F^{*}\omega )_p( v_1,\ldots, v_k) = \omega_{F( p) } ( F_{*,p} v_1, \ldots) 
\]
This pullback is linear and in fact, pullback of $C^{ \infty }$ forms are smooth, though this is thougher to prove.
The definition of wedge product of differential forms is simply their wedge product evaluated pointwise and pullbacks commute with wedge products.\\

We now finally get to the exterior derivative, we define an antiderivation on a grader algebra $A = \bigoplus A^{k}$ as an $ \mathbb{R}$-linear endomorphism of $A$ which satisfies an equivalent of the chain rule, namely an antiderivation $D$ needs to satisfy for two homogeneous elements $\omega$ and $\tau$ of degree $k,l$ respectively
\[ 
D( \omega \tau) = D( \omega) \cdot \tau + ( -1)^{k} \omega \cdot D\tau
\]
We say that $D$ is of degree $n$ if for a homogeneous piece $\omega$ of degree $k$, the degree of $D\omega$ is $k+n$.\\

We now define the exterior algebra on $M$ as a direct sum $\Omega^{*}( M) = \bigoplus \Omega^{k}( M) $, here $\Omega^{k}( M) $ represents the set of all smooth sections of $\bigwedge^{k}( T^{*}M) \to M$.\\
We define the exterior derivative to be the unique antiderivation satisfying
\begin{itemize}
\item $D$ is of degree 1
\item $D\circ D =0$ 
\item For a function $f$ (which is just an element of $\Omega^{0}( M) $ ) and for a vector field $X$, we have $Df( X) = Xf$ 
\end{itemize}

From these properties we can see how $D$ can be expressed in local charts, if $( U, y_1,\ldots, y_n) $ is such a chart on $M$, and if $\omega = \sum a_I dx^{I}$ is $k$-form on $U$, we have
\begin{align*}
	D\omega &= \sum D( a_I dx^{I}) \\
	&= \sum d a_I\wedge dx^{I} + \sum a_I D ( dx^{I}) \\
	&= \sum d a_I \wedge dx^{I} \\
	&= \sum_I \sum_{j \in [ n] } \frac{\del a_I}{\del y_j} dy_j \wedge dx^{I}
\end{align*}
Using the fact that $D$ is a local operator and partitions of unity, we can show that $D$ is uniquely  determined and independent of the choice of coordinate charts.\\
Using this, we can define the unique exterior derivative of a manifold $M$, we denote it by $d$.






\end{document}	
