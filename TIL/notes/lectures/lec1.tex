\documentclass[../main.tex]{subfiles}
\begin{document}
\lecture{1}{Wed 31 Aug}{Tensor Products preserve Flatness}
Let $S$  be flat over $R$ and $\phi: R\to T$ a morphism of rings, then $S\otimes_R T$ is a flat $T$-module.\\
Recall that a module $S$ is flat iff $\tor _n^{R}( S,M) =0$ for any $R$ module $M$.\\
Let $0 \to A \to B \to C \to 0 $ be an SES of $T-$modules, then tensoring with $S\otimes_R T$ gives
\[ 
0 \to A\otimes_T ( S\otimes_R T) \to B \otimes_T ( S\otimes_R T) \to C \otimes_T ( S\otimes_R T) \to 0
\]
But recall that $A\otimes_T ( S\otimes_R T) = A \otimes_R S$ where $A$ is given the structure of an $R-$module through $\phi$, from this it is immediate that $S\otimes_R T$ is flat.\\

This is useful for a version of going down for flat extensions, if $\phi:R \to S$ is a morphism which makes $S$ into a flat $R-$module, if $P' \subset P$ are two prime ideals of R and $Q$ is an ideal of $S$ whose contraction is $P$ then there exists a prime ideal $Q' \subset Q$ whose contraction is $P'$.\\

To show this ( 10.11 in Eisenbud), we first reduce to the case $P'=0$ by modding out by $P'$, now $R$ is a domain and we can replace $S$ with $S/P'S$ \\
Now we use the above to show that $S /P'S = S \otimes R /P' $ is flat over $R /P'$.\\
Now every nonzero divisor of $R$ ( which is just $R\setminus 0$ since $R$  is a domain) is a nonzerodivisor on $S$ by some lemma.\\
We now choose $Q'$ a minimal prime of $S$, hence, it's elements  consist of zerodivisors (since $Ass S$ contains all minimal primes and is equal to 0 and the set of zerodivisors ) and thus $Q' \cap R = 0$ 

\subsection*{$\dim R[x] = 1 + \dim R$ }

The inequality $\dim R[x] \leq 1 + \dim R$ is obvious, since for a chain of ideals $P_i$ in $R$, we can form $P_1R[x] \subset \ldots \subset P_n R[x] \subset P_n R[x] + ( x)  $, for the other inequality, Eisenbud claims that the inequality holds if we assume the result that:\\
For $P$ prime in $R$, an ideal $Q$ maximal wrt to the property of contracting to $P$
$\dim R[x]_Q = 1 + \dim R_P$.\\

Indeed, once we have this result, take $Q$ to be maximal in $R[x]$ and $P= R\cap Q$.\\
The other inequality then follows from
\[ 
1+ \dim R \geq 1 + \dim R_p = \dim R[x]_Q
\]
Taking $Q$ to be a maximal ideal in a maximal chain for $R[x]$, we get $\dim R[x]_Q = \dim R[x]$ 

\end{document}	
